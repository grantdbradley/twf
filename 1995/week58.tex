
% </A>
% </A>
% </A>
\week{July 12, 1995}

A few weeks ago I went to the IVth Porto Meeting on Knot Theory and
Physics, to which I had been kindly invited by Jose Mourao.  Quite 
a few of the (rather few) believers in the relevance of n-categories to physics
were there.  I spoke on higher-dimensional algebra and topological quantum
field theory, and also a bit on spin networks.  Louis Crane spoke on his ideas, 
especially the idea of getting 4-dimensional TQFTs out of state-sum
models.  And John Barrett spoke on 

1)  John Barrett, Quantum gravity as topological quantum field theory, to 
appear in the November 1995 special issue of Jour. Math. Physics, also 
available as <A HREF = "http://xxx.lanl.gov/abs/gr-qc/9506070">gr-qc/9506070</A>.

This is a nice introduction to the concepts of topological quantum field
theory (TQFT) that doesn't get bogged down in the (still substantial) 
technicalities.  In particular, it pays more emphasis than usual to the physical
interpretation of TQFTs, and how this meshes with more traditional issues
in the interpretation of quantum mechanics.  One of the main things
I got out of the conference, in fact, was a sense that there is a budding field
along these lines, just crying out to be developed.  As Barrett notes, Atiyah's
axioms for a TQFT can really be seen as coming from combining 

a) The rules of quantum mechanics for composing amplitudes

and 

b) Functoriality, or the correct behavior under diffeomorphisms of manifolds.

Indeed, he convincingly recovers the TQFT axioms from these two principles.
And of course these two principles could be roughly called "basic quantum
mechanics" and "general covariance"... lending credence to the idea that 
whatever the theory of quantum gravity turns out to be, it should be 
something closely related to a TQFT.  (I should emphasize, though, 
that this question is one of the big puzzles in the subject.) 

The richness inherent in b) makes the business of erecting a formalism
to interpret topological quantum field theory much more interesting than the
(by now) rather stale discussions that only treat a), or "basic quantum 
mechanics".  In particular, in a TQFT, every way of combining 
manifolds - spaces or spacetimes - yields a corresponding rule for 
composing amplitudes.  For example, if we have two spacetimes that look 
like


\begin{verbatim}

 O
| |
| |
| |
| |
 O
\end{verbatim}
    

(that's supposed to look like a pipe!) and 


\begin{verbatim}

 O       O
 \ \   / /
  \ \ / /
   \   /
    | |
     O
\end{verbatim}
    

- that is, a cylinder and a "trinion" (or upside-down pair of pants) - 
we can combine them either "horizontally" like this:


\begin{verbatim}

 O       O      O
 \ \   / /     | |
  \ \ / /      | |
   \   /       | |
    | |        | | 
     O          O
\end{verbatim}
    

or "vertically" like this:


\begin{verbatim}

 O       O
 \ \   / /
  \ \ / /
   \   /
    | |
    | |
    | |
    | |
    | |
     O
\end{verbatim}
    

Corresponding to each spacetime we have a "time evolution operator" -
a linear operator that describes how states going in one end pop out the
other, "evolved in time".  And corresponding to horizontal and vertical 
composition of spacetimes we have two ways to compose operators: 
horizontal composition usually being called "tensor product", and 
vertical composition being called simply "composition".  These two ways 
satisfy some compatibility conditions, as well.

Now if one has read a bit about n-categories and/or "extended" topological
quantum field theories, one already knows that this is just the 
tip of the iceberg.  If we allow ourselves to cut spacetimes into 
smaller bits - e.g., pieces with "corners", such as tetrahedra or their 
higher-dimensional kin - one gets more
possible ways of composing operators, and more compatibility conditions.  These
become algebraically rather sophisticated, but luckily, there is a huge 
amount of evidence that existing TQFTs extend to more 
sophisticated structures of this sort, through a miraculous 
harmony between algebra and topology.  

This leads to some interesting new concepts when it comes to the physical
interpretation of extended TQFTs.   As Crane described in his talk
(see also his papers listed in "<A HREF = "week2.html">week2</A>", "<A HREF = "week23.html">week23</A>" and "<A HREF = "week56.html">week56</A>"), in a 
4-dimensional extended TQFT one expects the following sort of thing.  
If we think of an "observer" as a 3-manifold with boundary - 
imagine a person being the 3-manifold and his skin being the 
boundary, if one likes - the extended TQFT should assign to his boundary a 
"Hilbert category" or "2-Hilbert space".   This is the categorical analog of a
Hilbert space.  In other words, just as a Hilbert space is a \emph{set} 
in which 
you can \emph{sum} things and \emph{multiply} them 
by \emph{complex numbers}, and 
get \emph{complex numbers} by taking \emph{inner products} of things, a 
2-Hilbert space is an analogous structure in which every term 
surrounded by asterisks is replaced by its analog
one step up the categorical ladder.  This means:


\begin{verbatim}

set             \to  category
sum             \to  direct sum
multiply        \to  tensor
complex numbers \to  vector spaces
inner products  \to  homs
\end{verbatim}
    

There's a good chance that you know the analogy between numbers and vector
spaces: just as you can add numbers and multiply them, you can take direct sums
and tensor products of vector spaces, and many of the same rules still 
apply (in a somewhat more sophisticated form, because laws that were 
equations are now isomorphisms).  A little less familiar is the 
analogy between inner products and "homs".  Given two vectors v and w in a 
Hilbert space you can take the inner product <v,w> and get a number; 
similarly, given two (finite-dimensional) Hilbert 
spaces V and W you can form hom(V,W) - that is, the set of all linear 
maps from V to W - and get a Hilbert space.  The same thing works in any 
"2-Hilbert space".  

The most basic example of a 2-Hilbert space would be Hilb, the category of 
finite-dimensional Hilbert spaces, but also Reps(G), the category of finite-
dimensional unitary representations of a finite group.  (Similar remarks hold 
for quantum groups at root of unity.)  Just as the inner product is linear 
in one argument and conjugate-linear in the other, "hom" behaves nicely 
under direct sums in each argument, but each argument behaves a bit 
differently under tensor product, so one can say it's "linear" in 
one and "conjugate-linear" in the other.  
      
So anyway, just as in a 4d TQFT a 3-manifold M determines a Hilbert space Z(M),
and a 4-manifold N with boundary equal to M determines a vector Z(N) in 
Z(M), something similar happens in an extended TQFT.  (For experts, here
I'm really talking about "unitary" TQFTs and extended TQFTs - these are
the physically sensible ones.)   Namely, a "skin of observation" or 2-manifold
S determines a 2-Hilbert space Z(S), and an "observer" or 3-manifold M with
boundary equal to S determines an object in Z(S).  Now, given two observers
M and M' with the same "skin" - for example, the observer "you" and the 
observer "everything in the world except you" - one gets two 
objects Z(M) and Z(M') in Z(S), so one can form the "inner product" 
hom(Z(M),Z(M')), which is a Hilbert space.  This is \emph{your} Hilbert 
space for describing states of \emph{everything in the world except you}.
Note that we are using the term "observer" here in a 
somewhat whimsical sense; in particular, every region of space counts as an 
observer in this game, so we can flip things around and form the inner product 
hom(Z(M'),Z(M)), which is the Hilbert space that \emph{everything in the world except you} can use to describe states of \emph{you}.  These two 
Hilbert spaces, with roles reversed, are conjugate to each 
other (using an obvious but perhaps slightly unfamiliar definition 
of "conjugate" Hilbert space), so they're pretty much the same.

Now this may at first seem weird, but if you think about it, it 
becomes a bit less so.   Of course, all of this stuff simply follows from the 
notion of a unitary extended TQFT, and whether the actual laws of physics 
are given by such a structure is a separate issue.  But there is 
clearly a lot of relevance to the "holographic hypothesis" and 
Lee Smolin's more mathematical version of that hypothesis,
as sketched in "<A HREF = "week57.html">week57</A>".  The basic idea, there as here, is that we are 
concentrating our attention on the things about a system that can be 
measured at its boundary, and what we measure there can be either 
thought of describing the state of the "inside" or dually the "outside".  

I think if I go out on a limb here, and rhapsodize a bit, the point might be 
clearer: but don't take this too seriously.  Namely: all of the stuff you 
see, hear, and otherwise observe about the world - which seems to 
be "information about the outside" - is also stuff going on in 
your brain, hence "information about the inside".  What this stuff 
really is, of course, is \emph{correlations} between the inside and the 
outside.  This is the reason for the duality between observer and 
observed mentioned above.  Note: we need not worry here whether or not 
there's "really" a lot more going on outside than what you observe.  
The point is simply that everything \emph{you} observe about what's going 
on in the world outside is correlated to stuff that the world 
could observe about what is going on in you.  (Maybe.)

I should perhaps also add that the mathematicians are getting a bit 
behind on the job of developing the "higher linear algebra" 
needed to support this sort of physics.  So it's a bit hard to point 
to a good reference for all this 2-Hilbert space stuff.  I'm 
slowly writing a paper on it, but for now the best sources seem to be 
Kapranov and Voevodsky's work on 2-vector spaces:

2) M. Kapranov and V. Voevodsky, 2-Categories and
Zamolodchikov tetrahedra equations, in Proc. Symp. Pure
Math. 56, Part 2 (1994), AMS, Providence, pp. 177-260.

(see also "<A HREF = "week4.html">week4</A>") Dan Freed's work on higher algebraic structures in 
gauge theory ("<A HREF = "week12.html">week12</A>," "<A HREF = "week48.html">week48</A>"), and David Yetter's new paper:

3) David Yetter, Categorical linear algebra: a setting for questions
from physics and low-dimensional topology, Kansas U. preprint, available
as <A HREF = "http://math.ucr.edu/home/baez/yetter.pdf">http://math.ucr.edu/home/baez/yetter.pdf</A> and <A HREF = "http://math.ucr.edu/home/baez/yetter.ps">http://math.ucr.edu/home/baez/yetter.ps</A>

This treats 2-vector spaces in a very beautiful way, but not 2-Hilbert spaces.
Definitely worth reading for anyone interested in this sort of thing!

While visiting Porto, I managed somehow to miss talking to Eugenia Cesar
de Sa, which was really a pity because she was the one who developed the
way of describing 4-manifolds that Broda (see "<A HREF = "week9.html">week9</A>","<A HREF = "week10.html">week10</A>") used to 
construct a 4-dimensional TQFT.  This TQFT was later shown by Roberts 
(see "<A HREF = "week14.html">week14</A>") to be isomorphic to that described by Crane and Yetter 
using a state sum model -- i.e., by a discrete analog of a path integral in 
which one chops spacetime up into 4-dimensional "hypertetrahedra" 
(better known as 4-simplices!), labels their 2d and 3d faces by spins, and 
sums over labellings.  Her work is cited in the Broda reference
in "<A HREF = "week17.html">week17</A>", but I managed luckily to get a copy of her thesis:

4) Eugenia Cesar de Sa, Automorphisms of 3-manifolds and representations of 
4-manifolds, Ph.D. thesis, University of Warwick, 1977.  

This should let me learn more about 4-dimensional topology, a fascinating
subject on which I'm woefully ignorant.

One reason why Broda's work, and thus de Sa's, is interesting to me, is
that people have suspected for a while that the Crane-Yetter-Broda theory, 
which is constructed purely combinatorially, is isomorphic to BF theory with
cosmological term.  BF theory (see "<A HREF = "week36.html">week36</A>") is a 4-dimensional field theory 
that can be described starting from a Lagrangian in the traditional 
manner of physics.  The theory "with cosmological term" can be 
regarded as a baby version of quantum gravity with nonzero 
cosmological constant, a baby version having only one state, the
"Chern-Simons state". As I discussed in "<A HREF = "week56.html">week56</A>", it's this Chern-Simons state
that plays a key role in Smolin's attempt to "exactly solve" quantum gravity.  
Thus I suspect that BF theory is a good thing to understand really well.   
Recently I showed in  

5) John Baez, 4-dimensional BF theory with cosmological term as a topological
quantum field theory, available as 
<A HREF = "http://xxx.lanl.gov/abs/q-alg/9507006">q-alg/9507006</A>.

that the Crane-Yetter-Broda theory is indeed isomorphic as a TQFT to a certain 
BF theory. With a bit more work, this should give us a state sum model for 
the BF theory that's a baby version of quantum gravity in 4 dimensions.  
This should come in handy for studying Smolin's hypothesis and its 
ramifications.

6) Timothy Porter, TQFTs from homotopy n-types, University of Wales, 
Bangor preprint, available at
<A HREF = "http://www.bangor.ac.uk/~mas013/preprint.html">http://www.bangor.ac.uk/~mas013/preprint.html</A>

The Dijkgraaf-Witten model is an n-dimensional TQFT one gets from a finite 
group G.  It's given by a really simple state sum model.  Chop your manifold 
into simplices; then the allowed "states" are just labellings of  the 
edges with elements of G subject to the constraint that the product around 
any triangle is 1.  You can think of a labelling as a kind of "connection" 
that tells you how to parallel transport along the edges, and the 
constraint says the connection is flat.  Expectation values of
physical observables are then computed as sums over these states.
In fact, this TQFT is a baby version of BF theory \emph{without}
cosmological constant.  A toy model of a toy model of quantum gravity, 
in other words: the classical solutions of BF theory without 
cosmological constant are just flat connections on some G-bundle where 
G is a Lie group, while the Dijkgraaf-Witten model does something 
similar for a finite group.  

In a previous paper (see "<A HREF = "week54.html">week54</A>") Porter studied the Dijkgraaf-Witten 
model and a generalization of it due to Yetter that allows one to label 
faces with things too... one can think of this generalization as 
allowing "curvature", because the product of elements of
G around a triangle need no longer be 1; instead, it's something 
determined by the labelling of the face.  

7) David Yetter, TQFTs from homotopy 2-types, Journal of Knot Theory and its 
Ramifications 2 (1993), 113-123.

In his new paper Porter takes this idea to its logical conclusion and 
constructs analogous theories that allow labellings of simplices in any 
dimension.  Technically, the input data is no longer just a finite group, 
but a finite simplicial group G.  

What's a simplicial group?  It's a wonderful thing; 
using the "internalization" trick I've referred to in some previous Finds, 
all I need to say is that it's a simplicial object in the category of 
groups.  A simplicial set is a bunch of sets, one for each natural 
number, together with a bunch of "face" and "degeneracy" maps satisfying
the same laws that the face and degeneracy maps do for a simplex.  
(Students of singular or simplicial homology will know what I'm 
talking about.)  Similarly, a simplicial group is a bunch of 
\emph{groups}, 
together with a bunch of of "face" and "degeneracy" \emph{homomorphisms}
satisfying the same laws.  

A triangulated manifold gives a simplicial set in an obvious way, and 
from any simplicial set one can obtain a simplicial groupoid (like a 
simplicial group, but with groupoids instead!) called its "loop 
groupoid".  The sort of labellings Porter considers are homomorphisms
from this simplicial groupoid to the given simplicial group G.

I will refrain from trying to say what all this has to do with 
homotopy n-types.  Nonetheless, from a pure mathematics point of view, 
that's the most exciting aspect of the whole business!
Part of the puzzle about TQFTs is their relation to traditional 
algebraic topology (and not-so-traditional algebraic topology like 
nonabelian cohomology, n-stacks, etc.), and this work 
serves as a big clue about that relationship.  


\par\noindent\rule{\textwidth}{0.4pt}
% </A>
% </A>
% </A>
