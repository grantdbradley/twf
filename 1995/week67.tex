
% </A>
% </A>
% </A>
\week{October 23, 1995}

I'm pretty darn busy now, so the forthcoming Weeks will probably be
pretty hastily written.  This time I'll mainly write about quantum
gravity.  

1) Margaret Wertheim, Pythagoras' Trousers: God, Physics, and the Gender
Wars, Times Books/Random House, New York, 1995.

I enjoyed this book, despite or perhaps because of the fact that it may
annoy lots of physicists.  It notes that, starting with Pythagoras,
theoretical physics has always had a crypto-religious aspect.  With
Pythagoras it was obvious; he seems to have been the leader of a special
sort of religious cult.  With people like Kepler, Newton and Einstein it
is only slightly less obvious.  The operative mythology in every case is
that of the mage.  Think of Einstein, stereotypically with long white
hair (though most of best work was actually done before his hair got
white), a powerful but benign figure devoted to finding out "the
thoughts of God".  The mage, of course, is typically male, and Wertheim
argues that this makes it harder for women to become physicists.  (A lot
of the same comments would apply to mathematics.)  It is not a very
scholarly book, but I wouldn't dismiss it.

2) Stephen W. Hawking, Virtual black holes, available as
<A HREF = "http://xxx.lanl.gov/abs/hep-th/9510029">hep-th/9510029</A>.

Hawking likes the "Euclidean path-integral approach" to quantum
gravity.  The word "Euclidean" is a horrible misnomer here, but it
seems to have stuck.  It should really read "Riemannian", the idea
being to replace the Lorentzian metric on spacetime by one in which
time is on the same footing as space.  One thus attempts to compute
answers to quantum gravity problems by integrating over all Riemannian
metrics on some 4-manifold, possibly with some boundary conditions.
Of course, this is tough - impossible so far - to make rigorous.  But
Hawking isn't scared; he also wants to sum over all 4-manifolds
(possibly having a fixed boundary).  Of course, to do this one needs
to have some idea of what "all 4-manifolds" are.  Lots of people like
to consider wormholes, which means considering 4-manifolds that aren't
simply connected.  Here, however, Hawking argues against wormholes,
and concentrates on simply-connected 4-manifolds.  He writes: "Barring
some pure mathematical details, it seems that the topology of simply
connected four-manifolds can be essentially represented by gluing
together three elementary units, which I call bubbles.  The three
elementary units are S^{2} \times  S^{2},
CP^{2}, and K3.  The latter two have orientation reversed
versions, -CP^{2} and -K3."  S^{2} \times 
S^{2} is just the product of the 2-dimensional sphere with
itself, and he argues that this sort of bubble corresponds to a
virtual black hole pair.  He considers the effect on the Euclidean
path integral when you have lots of these around (i.e., when you take
the connected sum of S^{4} with lots of these).  He argues that
particles scattering off these lose quantum coherence, i.e., pure
states turn to mixed states.  And he argues that this effect is very
small at low energies \emph{except} for scalar fields, leading him
to predict that we may never observe the Higgs particle!  Yes, a real
honest particle physics prediction from quantum gravity!  As he notes,
"unless quantum gravity can make contact with observation, it will
become as academic as arguments about how many angels can dance on the
head of a pin".  I suspect he also realizes that he'll never get a
Nobel prize unless he goes out on a limb like this.  <img src =
"emoticons/rolleyes.gif" alt = ""/> He also gives an argument for why
the "\theta  angle" measuring CP violation by the strong force may be
zero.  This parameter sits in front of a term in the Standard Model
Lagrangian; there seems to be no good reason for it to be zero, but
measurements of the neutron electric dipole moment show that it has to
be less than 10^{-9}, according to the following book...

3) Kerson Huang, Quarks, Leptons, and Gauge Fields, World Scientific
Publishing Co., Singapore, 1982.  

Perhaps there are better bounds now, but this book is certainly one of
the nicest introductions to the Standard Model, and if you want to 
learn about this "\theta  angle" stuff, it's a good place to start.

Anyway, rather than going further into the physics, let me say a bit
about the "pure mathematical details".  Here I got some help from
Greg Kuperberg, Misha Verbitsky, and Zhenghan Wang (via Xiao-Song Lin, a
topologist who is now here at Riverside).  Needless to say, the mistakes
are mine alone, and corrections and comments are welcome!

First of all, Hawking must be talking about homeomorphism classes of
compact oriented simply-connected smooth 4-manifolds, rather than
diffeomorphism classes, because if we take the connected sum of 9
copies of CP^{2} and one of -CP^{2}, that has
infinitely many different smooth structures.  Why the physics depends
only on the homeomorphism class is beyond me... maybe he is being
rather optimistic.  But let's follow suit and talk about homeomorphism
classes.  Folks consider two cases, depending on whether the
intersection form on the second cohomology is even or odd.  If our
4-manifold has an odd intersection form, Donaldson showed that it is
an connected sum of copies of CP^{2} and -CP^{2}.  If
its intersection form is even, we don't know yet, but if the "11/8
conjecture" is true, it must be a connected sum of K3's and
S^{2} \times  S^{2}'s.  Here I cannot resist adding
that K3 is a 4-manifold whose intersection is E_{8} \oplus 
E_{8} \oplus  H \oplus  H \oplus  H, where H is the 2\times 2 matrix


\begin{verbatim}

 0 1
 1 0
\end{verbatim}
    

and E_{8} is the nondegenerate symmetric 8\times 8 matrix describing
the inner products of vectors in the wonderful lattice, also called
E_{8}, which I discussed in "<A HREF =
"week65.html">week65</A>"!  So E_{8} raises its ugly head yet
again....  By the way, H is just the intersection form of
S^{2} \times  S^{2}, while the intersection form of
CP^{2} is just the 1\times 1 matrix


\begin{verbatim}

1.
\end{verbatim}
    

Even if the 11/8 conjecture is not true, we could if necessary resort
to Wall's theorem, which implies that any 4-manifold becomes
homeomorphic - even diffeomorphic - to a connected sum of the 5
basic types of "bubbles" after one takes its connected sum with
sufficiently many copies of S^{2} \times  S^{2}.  This
suggests that if Euclidean path integral is dominated by the case
where there are lots of virtual black holes around, Hawking's
arguments could be correct at the level of diffeomorphism, rather than
merely homeomorphism.  Indeed, he says that "in the wormhole picture,
one considered metrics that were multiply connected by wormholes.
Thus one concentrated on metrics [I'd say topologies!] with large
values of the first Betti number[....]  However, in the quantum
bubbles picture, one concentrates on spaces with large values of the
second Betti number."

4) Ted Jacobson, Thermodynamics of spacetime: the Einstein equation of
state, available as <A HREF =
"http://xxx.lanl.gov/abs/gr-qc/9504004">gr-qc/9504004</A>.   

Well, here's another paper on quantum gravity, also very good, which
seems at first to directly contradict Hawking's paper.  Actually,
however, I think it's another piece in the puzzle.  The idea here is to
derive Einstein's equation from thermodynamics!  More precisely, "The
key idea is to demand that this relation hold for all the local Rindler
causal horizons through each spacetime point, with [the change in heat]
and [the temperature] interpreted as the energy flux and Unruh
temperature seen by an accelerated observer just inside the horizon.  
This requires that gravitational lensing by matter energy distorts the
causal structure of spacetime in just such a way that the Einstein
equation holds".  It's a very clever mix of classical and quantum (or
semiclassical) arguments.  It suggests that all sorts of quantum
theories on the microscale could wind up yielding Einstein's equation on
the macroscale.  

5) Lee Smolin, The Bekenstein bound, topological quantum field theory
and pluralistic quantum field theory, available as
<A HREF = "http://xxx.lanl.gov/abs/gr-qc/9508064">gr-qc/9508064</A>.

This is a continued exploration of the themes of Smolin's earlier paper,
reviewed in "<A HREF = "week56.html">week56</A>" and "<A HREF = "week57.html">week57</A>".  Particularly interesting is the
general notion of "pluralistic quantum field theory", in which different
observers have different Hilbert spaces.  This falls out naturally in
the n-categorical approach pursued by Crane (see "<A HREF = "week56.html">week56</A>"), which I am
also busily studying.  

6) Rodolfo Gambini, Octavio Obregon and Jorge Pullin, Towards a loop
representation for quantum canonical supergravity, available as
<A HREF = "http://xxx.lanl.gov/abs/hep-th/9508036">hep-th/9508036</A>. 

Some knot theorists and quantum group theorists had better take a look at
this!  This paper considers the analog of SU(2) Chern-Simons theory
where you use the supergroup GSU(2), and perturbatively work out the
skein relations of the associated link invariant (up to a certain low
order in \hbar ).  If someone understood the quantum supergroup "quantum
GSU(2)", they could do this stuff nonperturbatively, and maybe get an
interesting invariant of links and 3-manifolds, and make some physicists
happy in the process.

7) Roh Suan Tung and Ted Jacobson, Spinor one-forms as gravitational
potentials, available as <A HREF = "http://xxx.lanl.gov/abs/gr-qc/9502037">gr-qc/9502037</A>.

This paper writes out a new Lagrangian for general relativity, closely
related to the action that gives general relativity in the Ashtekar
variables.  It's incredibly simple and beautiful!  I am hoping that if I
work on it someday, it will explain to me the mysterious relation
between quantum gravity and spinor fields (see the end of "<A HREF = "week60.html">week60</A>").  

8) Joseph Polchinski and Edward Witten, Evidence for heterotic - type I
string duality, available as <A HREF = "http://xxx.lanl.gov/abs/hep-th/9510169">hep-th/9510169</A>.

I'm no string theorist, so I've been lagging vastly behind the new work
on "dualities" that has revived interest in the subject.  Roughly
speaking, though, it seems folks have discovered a host of secret
symmetries relating superficially different string theories... making
them, in some deeper sense, the same.  The heterotic and type I strings
are two of the most famous string theories, so if they were really
equivalent as this paper suggests, it would be very interesting.


\par\noindent\rule{\textwidth}{0.4pt}
% </A>
% </A>
% </A>
