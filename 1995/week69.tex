
% </A>
% </A>
% </A>
\week{November 11, 1995}



One of the great things about starting to work on quantum gravity was
getting to know some of the people in the field.  Ever since the
development of string theory and the loop representation of quantum
gravity, there has been a fair amount of interest in understanding how
quantum theory and gravity fit together.  Indeed, now that the Standard
Model seems to be giving a spectacularly accurate description of all the
forces \emph{except} gravity, quantum gravity is one of the few really big
mysteries left when it comes to working out the basic laws of physics
--- or at least, one of the few \emph{obvious} big mysteries.  (As soon as
one mystery starts becoming less mysterious, new mysteries tend to
become more visible.)  But back when particle physics was big business,
only a few rather special sorts of people were seriously devoted to
quantum gravity.  These people seem to be often more than averagely
interested in philosophy, often more interested in mathematics (which is
one of the few solid handholds in this slippery subject), and always
more resigned to the fact that Nature does not reveal all her secrets
very readily.

One of these folks is Chris Isham, whom I first saw at a conference in
Seattle in 1991.  The conference was on classical field theory but
somehow he, Abhay Ashtekar, and Renate Loll sneaked in and gave some
talks on the loop representation of quantum gravity.  This is when I
first became really interested in this subject, which I was later to
work on quite a bit.  I remember Isham saying how he had been working on
quantum gravity for many years, and that he'd gotten used to the fact that
nothing ever worked, but that \emph{this} approach \emph{seemed} to be working so
far.  He went on to talk about work he'd done with Ashtekar on making
the loop representation rigorous, which was based on Gelfand-Naimark
spectral theory.  He said that as a student, when he'd learned about
this theory, he was really excited, because it completely depends on the
fact that if we have a space X, we can think of any point x in X as a
functional on the space of functions on X, basically defining by
defining

x(f) 

to be

f(x).  

He said this with a laugh, but I knew what he meant, because I too had
found this idea tremendously exciting when I first learned the
Gelfand-Naimark theory.  I guess it's something about how what seems
at first like some sort of bizarre joke can turn out to be very useful....

Anyway, later, when I decided to work on this sort of thing and
was trying to learn more about quantum gravity, I found his review
article on the problem of time (see "<A HREF = "week9.html">week9</A>")
tremendously helpful,  
and I constantly recommend it to everyone who is trying to get their
teeth into this somewhat elusive issue.  So it's not surprising that
Isham figures prominently in the following nice popular article on the
problem of time: 

1) Marcia Bartusiak, When the universe began, what time was it?,
Technology Review (edited at the Massachusetts Institute of Technology),
November/December 1995, pp. 54-63.

If you can find this, read it: it also features Karel Kuchar and Carlo
Rovelli. 

This spring, I visited Isham at Imperial College in London and found him
to be just as interesting in person as in print, and not at all scary...
a bit of an cynic about all existing approaches to quantum gravity
(probably because he sees so clearly how flawed they all are), but
thoroughly good-humored about it and perfectly open to all sorts of
ideas, even my own nutty ideas about n-categories and physics.

Anyway, Isham has recently written a review article on quantum gravity
that gives a nice overview of the basic issues of the field:

2) C. J. Isham, Structural issues in quantum gravity, plenary session
lecture given at the GR14 conference, Florence, August 1995, preprint
available as <A HREF =
"http://xxx.lanl.gov/abs/gr-qc/9510063">gr-qc/9510063</A>. 

One interesting thing about it is the emphasis on the question of
whether spacetime is really a manifold the way we all usually think, or
perhaps something that just looks like a manifold at sufficiently large
distance scales.   This is one of those fundamental issues that is
rather hard to make direct progress on; one has to sort of sneak up on
it, but it's nice to see someone boldly holding the problem up for
examination.  Often the most important issues are the ones everyone is
scared to talk about, because they are so intractable.

Much of Abhay Ashtekar's early work dealt with asymptotically flat
solutions of Einstein's equation, but in about 1986 he somehow invented
a new formulation of general relativity, which everyone now calls the
"new variables" or "Ashtekar variables".  In terms of these new
variables general relativity looks a whole lot more like Yang-Mills
theory (the theory of all the forces \emph{except} gravity), and this let
Rovelli and Smolin formulate a radical new approach to quantum gravity,
the "loop representation".  (For a fun, nontechnical introduction to
this, try the article by Bartusiak reviewed in "<A HREF =
"week10.html">week10</A>".) 

Nowadays, Ashtekar is the main person behind the drive to make the loop
representation of quantum gravity into a mathematically rigorous theory.
Thus it's natural that after that first time in Seattle I would wind up
seeing him pretty often... first at Syracuse University and then at the
Center for Gravitational Physics and Geometry which he started at Penn
State.  It's really impressive how he has organized people into an
effective team there... and how he is systematically converting people's
hopes and dreams concerning the loop representation into a beautiful set
of rigorous \emph{theorems}.  For a good mathematical introduction to his
program, see his paper reviewed in "<A HREF = "week7.html">week7</A>".
A less mathematical introduction is:

3) Abhay Ashtekar, Polymer geometry at Planck scale and quantum Einstein
equations.  

This will probably appear on gr-qc in a while.  

I have also seen Renate Loll fairly often in the years since that
Seattle conference.  She is younger than Ashtekar and Isham (in fact,
she was a postdoc with Isham at one point), hence less intimidating to
me, which meant that I really enjoyed pestering her with stupid
questions when I was just starting to learn about this loop
representation stuff.  One of her specialties is lattice gauge theory,
and recently she has developed a lattice version of quantum gravity that
is eminently suitable for computer calculations.  The last time I saw
her was at a conference in Warsaw this spring (as reported in "<A HREF =
"week55.html">week55</A>" 
and "<A HREF = "week56.html">week56</A>").  In the process of working on
her lattice approach, she gave Rovelli and Smolin a big shock by turning
up an error in their 
computation of the volume operator in quantum gravity.  A state of
quantum gravity can be visualized roughly as a graph embedded in space,
with edges labelled by spins.  Rovelli and Smolin had thought there were
states of nonzero volume corresponding to graphs with only trivalent
vertices (3 edges meeting a vertex, that is).  As it turns out, they'd
made a sign error, and these states have zero volume; you need a
quadrivalent vertex to get some volume.  She has just written a paper on
this topic:

4) Renate Loll, Spectrum of the volume operator in quantum gravity, 
14 pages in plain tex, with 4 figures (postscript, compressed and
uu-encoded), available as <A HREF =
"http://xxx.lanl.gov/abs/gr-qc/9511030">gr-qc/9511030</A>.   

The abstract reads as follows:

The volume operator is an important kinematical quantity in the
non-perturbative approach to four-dimensional quantum gravity in the
connection formulation. We give a general algorithm for computing its
spectrum when acting on four-valent spin network states, evaluate some
of the eigenvalue formulae explicitly, and discuss the role played by
the Mandelstam constraints.

<H4> 

Quote of the week: "Nothing is too wonderful to be true, if it be
consistent with the laws of nature, and in such things as these,
experiment is the best test of such consistency."  Faraday, laboratory
diaries, entry 10,040, March 19, 1849.
</H4>
\par\noindent\rule{\textwidth}{0.4pt}

% </A>
% </A>
% </A>
