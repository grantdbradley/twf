
% </A>
% </A>
% </A>
\week{November 30, 1998}

If you like \pi , take a look at this book:

1) Lennart Berggren, Jonathan Borwein and Peter Borwein, \Pi : A Source
Book, Springer-Verlag, New York, 1997.

It's full of reprints of original papers about \pi , from the Rhind
Papyrus right on up to the 1996 paper by Bailey, Borwein and Plouffe in
which they figured out how to compute a given hexadecimal digit of \pi 
without computing the previous digits - see "<A HREF = "week66.html">week66</A>" for more about
that.  By the way, Colin Percival has recently used this technique to
compute the 5 trillionth binary digit of \pi !  (It's either zero or one,
I forget which.)  Percival is a 17-year old math major at Simon Fraser
University, and now he's leading a distributed computation project to
calculate the quadrillionth binary digit of \pi .  Anyone with a Pentium
or faster computer using Windows 95, 98, or NT can join.   For more
information, see:

2) PiHex project, <A HREF = "http://www.cecm.sfu.ca/projects/pihex/pihex.html">http://www.cecm.sfu.ca/projects/pihex/pihex.html</A>

Anyway, the above book is \emph{full} of fun stuff, like a one-page proof 
of the irrationality of \pi  which uses only elementary calculus, due 
to Niven, and the following weirdly beautiful formula due to Euler, 
which unfortunately is not explained:


\begin{verbatim}

              3   5   7   11   13   17   19
       \pi /2 = - \times  - \times  - \times  -- \times  -- \times  -- \times  -- \times  ...
              2   6   6   10   14   18   18
\end{verbatim}
    
Here the numerators are the odd primes, and the denominators are the
closest numbers of the form 4n+2.  

Since I've been learning about elliptic curves and the like lately, I
was also interested to see a lot of relations between \pi  and modular
functions.  For example, the book has a reprint of Ramanujan's  paper
"Modular equations and approximations to \pi ", in which he derives  a
bunch of bizarre formulas for \pi , some exact but others approximate,
like this:


\begin{verbatim}

              12
      \pi  ~ ---------  ln[ (2 \sqrt 2 + \sqrt 10) (3 + \sqrt 10) ]
           sqrt(190)
\end{verbatim}
    
which is good to 18 decimal places.  The strange uses to which genius
can be put know no bounds!  

Okay, now I'd like to wrap up my little story about why bosonic string
theory works best in 26 dimensions.  This time I want to explain how
you do the path integral in string theory.  Most of what I'm about to
say comes from some papers that my friend Minhyong Kim recommended 
to me when I started pestering him about this stuff:

3) Jean-Benoit Bost, Fibres determinants, determinants regularises, et
mesures sur les espaces de modules des courbes complexes, Asterisque
152-153 (1987), 113-149.

4) A. A. Beilinson and Y. I. Manin, The Mumford form and the Polyakov 
measure in string theory, Comm. Math. Phys. 107 (1986), 359-376.

For a more elementary approach, try chapters IX and X.4 in this book:

5) Charles Nash, Differential Topology and Quantum Field Theory, 
Academic Press, New York, 1991.


As I explained in "<A HREF = "week126.html">week126</A>", a
string traces out a surface in spacetime called the "string
worldsheet".  Let's keep life simple and assume the string
worldsheet is a torus and that spacetime is Euclidean R^{n}.
Then to figure out the expectation value of any physical observable, we
need to calculate its integral over the space of all maps from a torus
to R^{n}.


To make this precise, let's use X to denote a map from the torus to
R^{n}.  Then a physical observable will be some function f(X),
and its expectation value will be

                   (1/Z) \int  f(X) exp(-S(X)) dX
Here S(X) is the action for string theory, which is just the \emph{area} of 
the string worldsheet.  The number Z is a normalizing factor called the
partition function: 

                     Z = \int  exp(-S(X)) dX
But there's a big problem here!  As usual in quantum field thoery, the
space we're trying to integrate over is infinite-dimensional, so the
above integrals have no obvious meaning.  Technically speaking, the
problem is that there's no Lebesgue measure "dX" on an
infinite-dimensional manifold. 

Mathematicians might throw up our hands in despair and give up at this
point.  But physicists take a more pragmatic attitude: they just keep
massaging the problem, breaking rules here and there if necessary, until
they get something manageable.   Physicists call this "calculating the
path integral", but from a certain viewpoint what they're really doing
is \emph{defining} the path integral, since it only has a precise meaning
after they're done.  

In the case at hand, it was Polyakov who figured out the right massage:

6) A. M. Polyakov, Quantum geometry of bosonic strings, Phys. Lett. B103
(1981), 207.

He rewrote the above integral as a double integral: first an integral
over the space of metrics g on the torus, and then inside, for each
metric, an integral over maps X from the torus into spacetime.   Of 
course, any such map gives a metric on the torus, so this double
integral is sort of redundant.  However, introducing this redundancy
turns out not to hurt.  In fact, it helps!   

To keep life simple, let's just talk about the partition function

                     Z = \int  exp(-S(X)) dX
If we can handle this, surely we can handle the integral with the
observable f(X) in it.   Polyakov's trick turns the partition function
into a double integral:

            Z = \int  ( \int  exp(-<X, \Delta X>) dX) dg


where \Delta  is the Laplacian on the torus and the angle
brackets stand for the usual inner product of R^{n}-valued functions, both
defined using the metric g.

At first glance Polyakov's trick may seem like a step backwards: now we
have two ill-defined integrals instead of one!   However, it's actually
a step forward.  Now we can do the inside integral by copying the
formula for the integral of a Gaussian of finitely many variables - a
standard trick in quantum field theory.  Ignoring an infinite constant
that would cancel later anyway, the inside integral works out to be:

                        (det \Delta )^{-1/2}

But wait!  The Laplacian here is a linear operator on the vector space
of R^{n}-valued functions on the torus.  This is an infinite-dimensional
vector space, so we can't blithely talk about determinants the way we
can in finite dimensions.   In finite dimensions, the determinant of a
self-adjoint matrix is the product of its eigenvalues.   But the Laplacian 
has infinitely many eigenvalues, which keep getting bigger and bigger.
How do we define the product of all its eigenvalues?  

Of course the lowest eigenvalue of the Laplacian is zero, and you might
think that would settle it: the product of them all must be zero!  But
that would make the above expression meaningless, and we are not going to
give up so easily.  Instead, we will simply ignore the zero eigenvalue!
That way, we only have to face the product of all the \emph{rest}.   

(Actually there's something one can do which is slightly more careful
than simply ignoring the zero eigenvalue, but I'll talk about that later.)

Okay, so now we have a divergent product to deal with.  Well, in
"<A HREF = "week126.html">week126</A>" I used a trick called
zeta function regularization to make sense of a divergent sum, and we
can use that trick here to make sense of our divergent product.  Suppose
we have a self-adjoint operator A with a discrete spectrum consisting of
positive eigenvalues.  Then the "zeta function" of A is
defined by:

                      \zeta (s) = tr(A^{-s})

To compute \zeta (s) we just take all the eigenvalues of A, raise them to
the -s power, and add them up.   For example, if A has eigenvalues
1,2,3,..., then \zeta (s) is just the usual Riemann zeta function, which
we already talked about in "<A HREF = "week126.html">week126</A>".

This stuff doesn't quite apply if A is the Laplacian on a compact
manifold, since one of its eigenvalues is zero.  But we have already
agreed to throw out the zero eigenvalue, so let's do that when defining
\zeta (s) as a sum over eigenvalues.  Then it turns out that the sum
converges when the real part of s is positive and large.  Even better,
there's a theorem saying that \Zeta (s) can be analytically continued to 
s = 0.  Thus we can use the following trick to define the determinant of
the Laplacian.

Suppose that A is a self-adjoint matrix with positive eigenvalues.  Then

                    \zeta (s) = tr(exp(-s ln A))
Differentiating gives

                    \zeta '(s) = -tr(ln A exp(-s ln A))
and setting s to zero we get

                    \zeta '(0) = -tr(ln A).
But there's a nice little formula saying that

                    det(A) = exp(tr(ln A))
so we get

                    det(A) = exp(-\zeta '(0)).
Now we can use this formula to \emph{define} the determinant of the Laplacian
on a compact manifold!  Sometimes this is called a "regularized 
determinant".  

So - where are we?   We used Polyakov's trick to write the partition
function of our torus-shaped string as

            Z = \int  ( \int  exp(-<X, \Delta X>) dX) dg, 
then we did the inside integral and got this:

            Z = \int  (det \Delta )^{-1/2} dg
and then we figured out a meaning for the determinant here.

What next?  Well, since the Laplacian on R^{n}-valued functions is the 
direct sum of n copies of the Laplacian on real-valued functions, we 
expect that 

         (det \Delta )^{-1/2} = (det laplacian)^{-n/2}
where "laplacian" stands for the Laplacian on ordinary real-valued
functions on the torus.  One can actually check this rigorously using
the definition in terms of zeta functions.  That's reassuring: at 
least \emph{one} step of our calculation is rigorous!  So we get

                Z = \int  (det laplacian)^{-n/2} dg
Great.  But we are not out of the woods yet.  We still have an integral
over the space of metrics to do - another nasty infinite-dimensional
integral.

Time for another massage!

Look at this formula again:

            Z = \int  ( \int  exp(-<X, \Delta X>) dX) dg

The Laplacian depends on the metric g, and so does the inner product.
However, the combination <X, \Delta X> depends only on the
"conformal structure" - i.e., it doesn't change if we multiply
the metric by a position-dependent scale factor.  It also doesn't change
under diffeomorphisms.

Now the space of conformal structures on a torus modulo diffeomorphisms
is something familiar: it's just the moduli space of elliptic curves!
We figured out what this space looks like in "<A HREF =
"week125.html">week125</A>".  It's finite-dimensional and there's a
nice way to integrate over it, called the Weil-Petersson measure.  So we
can hope to replace the outside integral - the integral over metrics -
by an integral over this moduli space.

Indeed, we could hope that

      Z = \int  (\int  exp(-<X, \Delta X>) dX) d[g]   
      [hope!]
where now the outside integral is over moduli space and d[g] is 
the Weil-Petersson measure.  The hope, of course, is that the 
stuff on the inside is well-defined as a function on moduli space.

Actually this hope is a bit naive.  Even though <X, \Delta X> 
doesn't change if we recale the metric, the whole inside integral

            \int  exp(-<X, \Delta X>) dX 
\emph{does} change.  This may seem odd, but remember, we did a lot of
hair-raising manipulations before we even got this integral to mean
anything!  We basically wound up \emph{defining} it to be 

                  (det \Delta )^{-1/2},
and one can check that this \emph{does} change when we rescale the metric. 
This problem is called the "conformal anomaly".  

Are we stuck?  No!  Luckily, there is \emph{another} problem, which cancels
this one when n = 26.   They say two wrongs don't make a right, but 
with anomalies that's often the only way to get things to work....

So what's this other problem?  It's that we shouldn't just replace the
measure dg by the measure d[g] as I did in my naive formula for the
partition function.  We need to actually figure out the relation between
them.  Of course this is hard to do, because the measure dg doesn't
exist, rigorously speaking!  Still, if we do a bunch more hair-raising
heuristic manipulations, which I will spare you, we can get a formula
relating dg and and d[g], and using this we get
 
      Z = \int  ( \int  exp(-<X, \Delta X>) dX) f(g) d[g]   
where f(g) is some function of the metric.  There's a perfectly explicit
formula for this function, but your eyeballs would fall out if I showed
it to you.  Anyway, the real point is that IN 26 DIMENSIONS AND ONLY IN 26
DIMENSIONS, the integrand

               ( \int  exp(-<X, \Delta X>) dX) f(g) 
is invariant under rescalings of the metric (as well as diffeomorphisms).
In other words, the conformal anomaly in

                 \int  exp(-<X, \Delta X>) dX
is precisely canceled by a similar conformal anomaly in f(g), so their
product is a well-defined function on moduli space, so it makes sense
to integrate it against d[g].  We can then go ahead and figure out the
partition function quite explicitly.

By now, if you're a rigorous sort of pure mathematician, you must be
suffering from grave doubts about the sanity of this whole procedure.
But physicists regard this chain of reasoning, especially the miraculous
cancellation of anomalies at the end, as a real triumph.  And indeed,
it's far \emph{better} than \emph{most} 
of what happens in quantum field theory!

I've heard publishers of science popularizations say that each equation
in a book diminishes its readership by a factor of 2.  I don't know if
this applies to This Week's Finds, but normally I try very hard to keep
the equations to a minimum.  This week, however, I've been very bad, and
if my calculations are correct, by this point I am the only one reading
this.  So I might as well drop all pretenses of expository prose and
write in a way that only I can follow!  The real reason I'm writing
this, after all, is to see if I understand this stuff.

Okay, so now I'd like to see if I understand how one explicitly
calculates this integral:

                 \int  exp(-<X, laplacian X>) dX
Since we're eventually going to integrate this (times some stuff) over
moduli space, we might as well assume the metric on our torus is gotten
by curling up the following parallelogram in the complex plane:



\begin{verbatim}

                         \tau            \tau  + 1 
                         *             *




                      *              *
                      0              1
\end{verbatim}
    

There are at least two ways to do the calculation.  One is to actually
work out the eigenvalues of the Laplacian on this torus and then do the
zeta function regularization to compute its determinant.  Di Francesco,
Mathieu, and Senechal do this in the textbook I talked about in "<A
HREF = "week124.html">week124</A>".  They get

      \int  exp(-<X, laplacian X>) dX = 1 / (\sqrt Im(\tau )  |\eta (\tau )|^{2})

where "\eta " is the Dedekind eta function, regarded as function
of \tau .  But the calculation is pretty brutal, and it seems to me that
there should be a much easier way to get the answer.  The left-hand side
is just the partition function for an massless scalar field on the
torus, and we basically did that back in "<A HREF =
"week126.html">week126</A>".  More precisely, we considered just
the right-moving modes and we got the following partition function:

                           1/\eta (\tau )
How about the left-moving modes?  Well, I'd guess that their partition 
function is just the complex conjugate,

                           1/\eta (\tau )*
since right-movers correspond to holomorphic functions and left-movers
correspond to antiholomorphic functions in this Euclidean picture.  It's
just a guess!  And finally, what about the zero-frequency mode?  I have
no idea.  But we should presumably multiply all three partition
functions together to get the partition function of the whole system -
that's how it usually works.  And as you can see, we \emph{almost} get the
answer that Di Francesco, Mathieu, and Senechal got.  It would work out
\emph{perfectly} if the partition function of the zero-frequency mode were
1/\sqrt Im(\tau ).  By the way, Im(\tau ) is just the \emph{area} of the
torus.

As evidence that something like this might work, consider this: the
zero-frequency mode is presumably related to the zero eigenvalue of the
Laplacian.  We threw that out when we defined the regularized
determinant of the Laplacian, but as I hinted, more careful calculations
of

                 \int  exp(-<X, laplacian X>) dX
don't just ignore the zero eigenvalue.  Instead, they somehow use it to
get an extra factor of 1/\sqrt Im(\tau ).  Admittedly, the calculations are not
particularly convincing: a more obvious guess would be that it gives a
factor of infinity.  Di Francesco, Mathieu, and Senechal practically
admit that they \emph{need} this factor just to get modular invariance, 
and that they'll do whatever it
takes to get it.  Nash just sticks in the factor of 1/\sqrt Im(\tau ), mutters 
something vague, and hurriedly moves on.

Clearly the reason people want this factor is because of how the eta
function transforms under modular transformations.  In "<A HREF =
"week125.html">week125</A>" I said that the group PSL(2,Z) is
generated by two elements S and T, and if you look at the formulas there
you'll see they act in the following way on \tau :

                     S: \tau  |\to  -1/\tau   
                     T: \tau  |\to  \tau  + 1
The Dedekind eta function satisfies 

                \eta (-1/\tau )  =      (\tau  / i)^{1/2} \eta (\tau )
                \eta (\tau  + 1) = exp(2 \pi  i / 24) \eta (\tau )
The second one is really easy to see from the definition; the first
one is harder.  Anyway, using these facts it's easy to see that 

                     1 / (\sqrt Im(\tau ) |\eta (\tau )|^{2})
is invariant under PSL(2,Z), so it's really a function on moduli
space - but only if that factor of 1/sqrt(Im(\tau )) is in there!

Finally, I'd like to say something about why the conformal anomalies
cancel in 26 dimensions.  When I began thinking about this stuff I
was hoping it'd be obvious from the transformation properties of the
eta function - since they have that promising number "24" in them - 
but right now I do \emph{not} see anything like this going on.  Instead,
it seems to be something like this: in the partition function

      Z = \int  (\int  exp(-<X, \Delta X>) dX) f(g) d[g]   
the mysterious function f is basically just the determinant of the
Laplacian on \emph{vector fields} on the torus.  So ignoring those 
darn zero eigenvalues the whole integrand here is

                det(laplacian)^{n/2} det(laplacian')

where "laplacian" is the Laplacian on real-valued functions
and " laplacian' " is the Laplacian on vector fields.  Now these
determinants aren't well-defined functions on the space of conformal
structures; they're really sections of certain "determinant
bundles".  But in this situation, the determinant bundle for the
Laplacian on vector fields \emph{just so happens} to be the 13th tensor power
of the determinant bundle for the Laplacian on functions - so the whole
expression above is a well-defined function on the space of conformal
structures, and thence on moduli space, precisely when n = 26!!!


Now this "just so happens" cannot really be a coincidence.
There \emph{are} no coincidences in mathematics.  That's why it pays to be
paranoid when you're a mathematician: nothing is random, everything fits
into a grand pattern, it's all just staring you in the face if you'd
only notice it.  (Chaitin has convincingly argued otherwise using
Goedel's theorem, and certainly some patterns in mathematics seem
"purely accidental", but right now I'm just waxing rhapsodic,
expressing a feeling one sometimes gets....)

Indeed, look at the proof in Nash's book that one of these determinant
bundles is the 13th tensor power of the other - I think this result is
due to Mumford, but Nash's proof is easy to read.  What does he do?  He
works out the first Chern class of both bundles using the index theorem
for families, and he gets something involving the Todd genus - and the
Todd genus, as we all know, is defined using the same function

                x / (1 - e^{x}) = -1 + x/2 - x^{2}/12 + ...
that we talked about in "<A HREF = "week126.html">week126</A>"
when computing the zero-point energy of the bosonic string!  And yet
again, it's that darn -1/12 in the power series expansion that makes
everything tick.  That's where the 13 comes from!  It's all an elaborate
conspiracy!

But of course the conspiracy is far grander than I've even begun
to let on.  If we keep digging away at it, we're eventually led to 
nothing other than....

<DIV ALIGN = CENTER>
                       MONSTROUS MOONSHINE!!!
</DIV>
But I don't have the energy to talk about \emph{that} now.  For more, try:

7) Richard E. Borcherds, What is moonshine?, talk given upon winning
the Fields medal, preprint available as <A HREF = "http://xxx.lanl.gov/abs/math.QA/9809110">math.QA/9809110</A>.

8) Peter Goddard, The work of R. E. Borcherds, preprint available as
<A HREF = "http://xxx.lanl.gov/abs/math.QA/9808136">math.QA/9808136</A>.

Okay, if you've actually read this far, you deserve a treat.  First, 
try this cartoon, which you'll see is quite relevant:

9) Cartoon by J. F. Cartier, <A HREF = "http://www.physik.uni-frankfurt.de/~jr/gif/cartoon/cart0785.gif">http://www.physik.uni-frankfurt.de/~jr/gif/cartoon/cart0785.gif</A>

Second, let's
calculate the determinant of an operator A whose eigenvalues are the
numbers 1, 2, 3, ....  You can think of this operator as the Hamiltonian
for the wave equation on the circle, where we only keep the right-moving
modes.  As I already said, the zeta function of this operator is the
Riemann zeta function.  This function has \zeta '(0) = -ln(2 \pi )/2, so
using our cute formula relating determinants and zeta functions, we get

                 det(A) = exp(-\zeta '(0)) = (2 \pi )^{1/2} . 
Just for laughs, if we pretend that the determinant of A is the product
of its eigenvalues as in the finite-dimensional case, we get:

                    1 \times  2 \times  3 \times  ... = (2 \pi )^{1/2} 
or if you really want to ham it up,

                      \infty ! = (2 \pi )^{1/2}.
Cute, eh?  Dan Piponi told me this, as well as some of the other things
I've been talking about.  You can also find it in Bost's paper.  

\par\noindent\rule{\textwidth}{0.4pt}
Notes and digressions:

<UL>
<LI>
In all of the above, I put a minus sign into my Laplacian, so that it
has nonnegative eigenvalues.  This is common among erudite mathematical
physics types, who like "positive elliptic operators".

<LI>
The zeta function trick for defining the determinant of the Laplacian
works for any positive elliptic operator on a compact manifold.  A huge
amount has been written about this trick.  It's all based on the fact
that the zeta function of a positive elliptic operator analytically 
continues to s = 0.  This fact was proved by Seeley:

10) R. T. Seeley, Complex powers of an elliptic operator, Proc. Symp.
Pure Math. 10 (1967), 288-307.

<LI>

Why is the Polyakov action <X, \Delta X> conformally invariant?
Because the Laplacian has dimensions of 1/length^{2}, 
while the integral
used to define the inner product has dimensions of length^{2}, 
since the
torus is 2-dimensional.  This is the magic of 2 dimensions!  The path
integral for higher-dimensional "branes" has not yet been made
precise, because this magic doesn't happen there.

<LI>
About Euler's weirdly beautiful formula for \pi , Robin Chapman writes:

\begin{verbatim}


               3   5   7   11   13   17   19
        \pi /2 = - x - x - x -- x -- x -- x -- x ...               (1)
               2   6   6   10   14   18   18

Using the Euler product for \zeta (2) gives

\pi ^2/6 = \zeta (2) = 1 + 1/2^2 + 1/3^2 + ...
                 = (4/3) (9/8) (25/24) ... (p^2/(p^2-1)) ...

and dropping the p = 2 term and dividing by (1) we see that (1) is equivalent
to

\pi /4 = (3/4)(5/4)(7/8)  .... (p/(p-\chi (p))) ....                  (2)

where the numerators are odd primes and the denominators are their adjacent
multiples of 4. Also \chi  is the modulo 4 Dirichlet character. Now

p/(p - \chi (p)) = 1 + \chi (p)/p + \chi (p^2)/p^2 + .... 

and if we multiply these formally the RHS of (2) becomes

1 - 1/3 + 1/5 - 1/7 + 1/9 - .....

i.e., Gregory's series for \pi /4. Admittedly it's not apparent that this
formal manipulation is valid. However for Dirichlet L-functions the
Euler product is valid at s = 1. This requires some delicate analysis: for
details see Landau's book on prime numbers or Davenport's Multiplicative
Number Theory.

\end{verbatim}
    
</UL>


 \par\noindent\rule{\textwidth}{0.4pt}

% </A>
% </A>
% </A>
