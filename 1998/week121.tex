
% </A>
% </A>
% </A>
\week{May 15, 1998}

This time I want to talk about higher-dimensional algebra and its
applications to topology.  Marco Mackaay has just come out with a
fascinating paper that gives a construction of 4-dimensional TQFTs
from certain "monoidal 2-categories".  

1) Marco Mackaay, Spherical 2-categories and 4-manifold invariants,
preprint available as <A HREF = "http://xxx.lanl.gov/abs/math.QA/9805030">
math.QA/9805030</A>

Beautifully, this construction is just a categorified version of
Barrett and Westbury's construction of 3-dimensional topological
quantum field theories from "monoidal categories".
Categorification - the process of replacing equations by
isomorphisms - is supposed to take you up the ladder of dimensions.
Here we are seeing it in action!

To prepare you understand Mackaay's paper, maybe I should explain the
idea of categorification.  Since I recently wrote something about
this, I think I'll just paraphrase a bit of that.  Some of this is
already familiar to long-time customers, so if you know it all
already, just skip it.

2) John Baez and James Dolan, Categorification, to appear in the
Proceedings of the Workshop on Higher Category Theory and Mathematical
Physics at Northwestern University, Evanston, Illinois, March 1997,
eds. Ezra Getzler and Mikhail Kapranov.  Preprint available as
<A HREF = "http://xxx.lanl.gov/abs/math.QA/9802029">math.QA/9802029</A>
or at <A HREF = "http://math.ucr.edu/home/baez/cat.ps">
http://math.ucr.edu/home/baez/cat.ps</A>.

So, what's categorification?  This tongue-twisting term, invented by
Louis Crane, refers to the process of finding category-theoretic
analogs of ideas phrased in the language of set theory, using the
following analogy between set theory and category theory:


\begin{verbatim}


elements                            objects                       
equations between elements          isomorphisms between objects        
sets                                categories                    
functions                           functors                      
equations between functions         natural isomorphisms between functors  

\end{verbatim}
    
Just as sets have elements, categories have objects.  Just as there
are functions between sets, there are functors between categories.
Interestingly, the proper analog of an equation between elements is
not an equation between objects, but an isomorphism.  More generally,
the analog of an equation between functions is a natural isomorphism
between functors.

For example, the category FinSet, whose objects are finite sets and
whose morphisms are functions, is a categorification of the set N of
natural numbers.   The disjoint union and Cartesian product of finite
sets correspond to the sum and product in N, respectively.  Note that
while addition and multiplication in N satisfy various equational
laws such as commutativity, associativity and distributivity, disjoint
union and Cartesian product satisfy such laws <em>only up to natural
isomorphism</em>.  This is a good example of how equations between functions
get replaced by natural isomorphisms when we categorify.

If one studies categorification one soon discovers an amazing fact: many
deep-sounding results in mathematics are just categorifications of facts
we learned in high school!  There is a good reason for this.  All along,
we have been unwittingly "decategorifying" mathematics by pretending
that categories are just sets.  We "decategorify" a category by
forgetting about the morphisms and pretending that isomorphic objects
are equal.  We are left with a mere set: the set of isomorphism classes
of objects. 

To understand this, the following parable may be useful.  Long ago, when
shepherds wanted to see if two herds of sheep were isomorphic, they
would look for an explicit isomorphism.  In other words, they would line
up both herds and try to match each sheep in one herd with a sheep in
the other.  But one day, along came a shepherd who invented
decategorification.  She realized one could take each herd and
"count"
it, setting up an isomorphism between it and some set of "numbers",
which were nonsense words like "one, two, three,..." specially
designed for this purpose.  By comparing the resulting numbers, she
could show that two herds were isomorphic without explicitly
establishing an isomorphism!  In short, by decategorifying the category
of finite sets, the set of natural numbers was invented.   

According to this parable, decategorification started out as a stroke
of mathematical genius.  Only later did it become a matter of dumb
habit, which we are now struggling to overcome by means of
categorification.  While the historical reality is far more
complicated, categorification really has led to tremendous progress in
mathematics during the 20th century.  For example, Noether
revolutionized algebraic topology by emphasizing the importance of
homology groups.  Previous work had focused on Betti numbers, which
are just the dimensions of the rational homology groups.  As with
taking the cardinality of a set, taking the dimension of a vector
space is a process of decategorification, since two vector spaces are
isomorphic if and only if they have the same dimension.  Noether noted
that if we work with homology groups rather than Betti numbers, we can
solve more problems, because we obtain invariants not only of spaces,
but also of maps.  

In modern lingo, the nth rational homology is a \emph{functor} defined
on the \emph{category} of topological spaces, while the nth Betti number is
a mere \emph{function}, defined on the \emph{set} of isomorphism classes of
topological spaces.  Of course, this way of stating Noether's insight
is anachronistic, since it came before category theory.  Indeed, it
was in Eilenberg and Mac Lane's subsequent work on homology that
category theory was born!

Decategorification is a straightforward process which typically
destroys information about the situation at hand.  Categorification,
being an attempt to recover this lost information, is inevitably
fraught with difficulties.  One reason is that when categorifying, one
does not merely replace equations by isomorphisms.  One also demands
that these isomorphisms satisfy some new equations of their own,
called "coherence laws".  
Finding the right coherence laws for a given
situation is perhaps the trickiest aspect of categorification.

For example, a monoid is a set with a product satisfying the associative
law and a unit element satisfying the left and right unit laws.   The
categorified version of a monoid is a "monoidal category".   
This is a category C with a product 

\otimes : C \times  C \to  C 

and unit object 1.  If we naively impose associativity and the left
and right unit laws as equational laws, we obtain the definition of a
"strict" monoidal category.  However, the philosophy of
categorification suggests instead that we impose them only up to
natural isomorphism.  Thus, as part of the structure of a "weak"
monoidal category, we specify a natural isomorphism

a_{x,y,z}: (x \otimes  y) \otimes  z \to  x \otimes  (y \otimes  z)  

called the "associator", together with natural isomorphisms

            l_{x}: 1 \otimes  x \to  x, 
            r_{x}: x \otimes  1 \to  x. 

Using the associator one can construct isomorphisms between any two
parenthesized versions of the tensor product of several objects.
However, we really want a \emph{unique} isomorphism.  For example, there
are 5 ways to parenthesize the tensor product of 4 objects, which are
related by the associator as follows:


\begin{verbatim}

((x \otimes  y) \otimes  z) \otimes  w ----------------> (x \otimes  (y \otimes  z)) \otimes  w
         |                                   |
         |                                   |
         |                                   |
         V                                   |
(x \otimes  y) \otimes  (z \otimes  w)   _{  }                       |
         |                                   |
         |                                   |
         |                                   |
         V                                   V
{x \otimes  (y \otimes  (z \otimes  w)) <---------------- x \otimes  ((y \otimes  z) \otimes  w)

\end{verbatim}
    
In the definition of a weak monoidal category we impose a coherence
law, called the "pentagon identity", saying that this diagram
commutes.  Similarly, we impose a coherence law saying that the
following diagram built using a, l and r commutes:


\begin{verbatim}


(1 \otimes  x) \otimes  1  ----------> 1 \otimes  (x \otimes  1)
      |                       |
      |                       | 
      V                       V
   x \otimes  1 ------> x <------ 1 \otimes  x

\end{verbatim}
    
This definition raises an obvious question: how do we know we have
found all the right coherence laws?  Indeed, what does "right" even
\emph{mean} in this context?  Mac Lane's coherence theorem gives one answer
to this question: the above coherence laws imply that any two
isomorphisms built using a, l and r and having the same source and
target must be equal.

Further work along these lines allow us to make more precise the sense
in which N is a decategorification of FinSet.  For example, just as N
forms a monoid under either addition or multiplication, FinSet becomes
a monoidal category under either disjoint union or Cartesian product
if we choose the isomorphisms a, l, and r sensibly.  In fact, just as
N is a "rig", satisfying all the ring axioms except those involving
additive inverses, FinSet is what one might call a "rig category".  
In other words, it satisfies the rig axioms up to natural isomorphisms
satisfying the coherence laws discovered by Kelly and Laplaza, who
proved a coherence theorem in this context.  

Just as the decategorification of a monoidal category is a monoid, the
decategorification of any rig category is a rig.  In particular,
decategorifying the rig category FinSet gives the rig N.  This
idea is especially important in combinatorics, where the best proof of
an identity involving natural numbers is often a "bijective proof":
one that actually establishes an isomorphism between finite sets.

While coherence laws can sometimes be justified retrospectively by
coherence theorems, certain puzzles point to the need for a deeper
understanding of the \emph{origin} of coherence laws.   For example,
suppose we want to categorify the notion of "commutative monoid".   
The strictest possible approach, where we take a strict monoidal category
and impose an equational law of the form x \otimes  y = y \otimes  x, is
almost completely uninteresting.  It is much better to start with a weak
monoidal category equipped with a natural isomorphism 

B_{x,y}: x \otimes  y \to  y \otimes  x

called the "braiding"
and then impose coherence laws called "hexagon
identities" saying that the following two diagrams built from the
braiding and the associator commute:


\begin{verbatim}


x \otimes  (y \otimes  z)  ------------>  (y \otimes  z) \otimes  x 
     |                            ^
     |                            |
     V                            |

(x \otimes  y) \otimes  z                 y \otimes  (z \otimes  x)
     |                            ^
     |                            |
     V                            |

(y \otimes  x) \otimes  z  ------------>  y \otimes  (x \otimes  z)



(x \otimes  y) \otimes  z  ------------>  z \otimes  (x \otimes  y)
     |                            ^
     |                            |
     V                            |

x \otimes  (y \otimes  z)                 (z \otimes  z) \otimes  y
     |                            ^
     |                            |
     V                            |

x \otimes  (z \otimes  y)  ------------>  (x \otimes  z) \otimes  y

\end{verbatim}
    
This gives the definition of a weak "braided monoidal category".  
If we impose an additional coherence law saying that B_{x,y} is the
inverse of B_{y,x}, we obtain the definition of a 
"symmetric monoidal category".  
Both of these concepts are very important; which one is
"right" depends on the context.  However, neither implies that every
pair of parallel morphisms built using the braiding are equal.  A good
theory of coherence laws must naturally account for these facts.

The deepest insights into such puzzles have traditionally come from
topology.  In homotopy theory it causes problems to work with spaces
equipped with algebraic structures satisfying equational laws, because
one cannot transport such structures along homotopy equivalences.  It
is better to impose laws \emph{only up to homotopy}, with these homotopies
satisfying certain coherence laws, but again only up to homotopy, with
these higher homotopies satisfying their own higher coherence laws,
and so on.  Coherence laws thus arise naturally in infinite sequences.
For example, Stasheff discovered the pentagon identity and a sequence
of higher coherence laws for associativity when studying the algebraic
structure possessed by a space that is homotopy equivalent to a loop
space.  Similarly, the hexagon identities arise as part of a sequence
of coherence laws for spaces homotopy equivalent to double loop
spaces, while the extra coherence law for symmetric monoidal
categories arises as part of a sequence for spaces homotopy equivalent
to triple loop spaces.  The higher coherence laws in these sequences
turn out to be crucial when we try to \emph{iterate} the process of
categorification.

To \emph{iterate} the process of categorification, we need a concept of
"n-category" - roughly, an algebraic structure consisting of a
collection of objects (or "0-morphisms"), morphisms between objects
(or "1-morphisms"), 2-morphisms between morphisms, and so on up to
n-morphisms.  There are various ways of making this precise, and right
now there is a lot of work going on devoted to relating these
different approaches.  But the basic thing to keep in mind is that
the concept of "(n+1)-category" is a categorification of the concept
of "n-category".  
What were equational laws between n-morphisms in
an n-category are replaced by natural (n+1)-isomorphisms, which need
to satisfy certain coherence laws of their own.

To get a feeling for how these coherence laws are related to homotopy
theory, it's good to think about certain special kinds of n-category.
If we have an (n+k)-category that's trivial up to but not including
the k-morphism level, we can turn it into an n-category by a simple
reindexing trick: just think of its j-morphisms as (j-k)-morphisms!
We call the n-categories we get this way "k-tuply monoidal 
n-categories".  Here is a little chart of what they amount to for
various low values of n and k:


\begin{verbatim}

                   k-tuply monoidal n-categories 

              n = 0           n = 1             n = 2

k = 0         sets          categories         2-categories
     

k = 1        monoids         monoidal           monoidal
                            categories        2-categories

k = 2       commutative      braided            braided
             monoids         monoidal           monoidal
                            categories        2-categories 

k = 3         " "           symmetric            weakly
                             monoidal          involutory
                            categories          monoidal
                                              2-categories

k = 4         " "             " "               strongly 
                                               involutory
                                                monoidal
                                              2-categories

k = 5         " "             " "                "  "

\end{verbatim}
    
One reason James Dolan and I got so interested in this chart is the
"tangle hypothesis".  Roughly speaking, this says that n-dimensional
surfaces embedded in (n+k)-dimensional space can be described purely
algebraically using the a certain special "k-tuply monoidal n-category
with duals".  If true, this reduces lots of differential
topology to pure algebra!  It also helps you understand the parameters
n and k: you should think of n as "dimension" 
and k as "codimension".

For example, take n = 1 and k = 2.  Knots, links and tangles in
3-dimensional space can be described algebraically using a certain
"braided monoidal categories with duals".  This was the first
interesting piece of evidence for the tangle hypothesis.  It has
spawned a whole branch of math called "quantum topology", which
people are trying to generalize to higher dimensions.

More recently, Laurel Langford tackled the case n = 2, k = 2.  She
proved that 2-dimensional knotted surfaces in 4-dimensional space can
be described algebraically using a certain "braided monoidal
2-category with duals".  These so-called "2-tangles" 
are particularly
interesting to me because of their relation to spin foam models of
quantum gravity, which are also all about surfaces in 4-space.  For
references, see "<A HREF = "week103.html">week103</A>".  But if you want to learn about more about
this, you couldn't do better than to start with:

3) J. S. Carter and M. Saito, Knotted Surfaces and Their Diagrams,
American Mathematical Society, Providence, 1998.

This is a magnificently illustrated book which will really get you
able to \emph{see} 2-dimensional surfaces knotted in 4d space.  At the end
it sketches the statement of Langford's result.

Another interesting thing about the above chart is that k-tuply
monoidal n-categories keep getting "more commutative" as k increases,
until one reaches k = n+2, at which point things stabilize.  There is
a lot of evidence suggesting that this "stabilization hypothesis" is
true for all n.  Assuming it's true, it makes sense to call a k-tuply
monoidal n-category with k \ge  n+2 a "stable n-category".

Now, where does homotopy theory come in?  Well, here you need to look
at n-categories where all the j-morphisms are invertible for all j.
These are called "n-groupoids".  Using these, one can develop a
translation dictionary between n-category theory and homotopy theory,
which looks like this:


\begin{verbatim}


\omega -groupoids                    homotopy types 
n-groupoids                        homotopy n-types
k-tuply groupal \omega -groupoids    homotopy types of k-fold loop spaces
k-tuply groupal n-groupoids        homotopy n-types of k-fold loop spaces 
k-tuply monoidal \omega -groupoids   homotopy types of E_{k} spaces 
k-tuply monoidal n-groupoids       homotopy n-types of E_{k} spaces
stable \omega -groupoids             homotopy types of infinite loop spaces
stable n-groupoids                 homotopy n-types of infinite loop spaces 
Z-groupoids                        homotopy types of spectra  

\end{verbatim}
    
The entries on the left-hand side are very natural from an algebraic
viewpoint; the entries on the right-hand side are things topologists
already study.  We explain what all these terms mean in the paper, but
maybe I should say something about the first two rows, which are the
most basic in a way.  A homotopy type is roughly a topological space
"up to homotopy equivalence", and an \omega -groupoid is a kind of
limiting case of an n-groupoid as n goes to infinity.  If infinity is
too scary, you can work with homotopy n-types, which are basically
homotopy types with no interesting topology above dimension n.  These
should correspond to n-groupoids.

Using these basic correspondences we can then relate various special
kinds of homotopy types to various special kinds of \omega -groupoids,
giving the rest of the rows of the chart.  Homotopy theorists know a
lot about the right-hand column, so we can use this to get a lot of
information about the left-hand column.  In particular, we can work
out the coherence laws for n-groupoids, and - this is the best part,
but the least understood - we can then \emph{guess} a lot of stuff about
the coherence laws for \emph{general} n-categories.  In short, we are using
homotopy theory to get our foot in the door of n-category theory.

I should emphasize, though, that this translation dictionary is
partially conjectural.  It gets pretty technical to say what exactly
is and is not known, especially since there's pretty rapid progress
going on.  Even in the last few months there have been some
interesting developments.  For example, Breen has come out with a
paper relating k-tuply monoidal n-categories to Postnikov towers and
various far-out kinds of homological algebra:

4) Lawrence Breen, Braided n-categories and \Sigma -structures,
Prepublications Matematiques de l'Universite Paris 13, 98-06,
January 1998, to appear in the Proceedings of the Workshop on
Higher Category Theory and Mathematical Physics at Northwestern
University, Evanston, Illinois, March 1997, eds. Ezra Getzler
and Mikhail Kapranov.

Also, the following folks have also developed a notion of "iterated
monoidal category" whose nerve gives the homotopy type of a k-fold
loop space, just as the nerve of a category gives an arbitrary
homotopy type:

5) C. Balteanu, Z. Fiedorowicz, R. Schwaenzl, and R. Vogt, Iterated
monoidal categories, available at <A HREF = 
"http://arxiv.org/abs/math.AT/9808082">math.AT/9808082</A>.

Anyway, in addition to explaining the relationship between n-category
theory and homotopy theory, Dolan's and my paper discusses iterated
categorifications of the very simplest algebraic structures: the
natural numbers and the integers.  The natural numbers are the free
monoid on one generator; the integers are the free group on one
generator.  We believe this is just the tip of the following iceberg:


\begin{verbatim}


  algebraic strutures and the free such structure on one generator

    sets                               the one-element set
---------------------------------------------------------------------
   monoids                             the natural numbers
----------------------------------------------------------------------
    groups                                the integers
----------------------------------------------------------------------
 k-tuply monoidal                      the braid n-groupoid 
   n-categories                           in codimension k
----------------------------------------------------------------------
  k-tuply monoidal                    the braid \omega -groupoid 
  \omega -categories                        in codimension k
----------------------------------------------------------------------
 stable n-categories                  the braid n-groupoid 
                                     in infinite codimension
----------------------------------------------------------------------
stable \omega -categories               the braid \omega -groupoid 
                                      in infinite codimension
----------------------------------------------------------------------
  k-tuply monoidal                 the n-category of framed n-tangles
n-categories with duals                  in n+k dimensions
----------------------------------------------------------------------
 stable n-categories                the framed cobordism n-category
    with duals
----------------------------------------------------------------------
 k-tuply groupal                         the homotopy n-type 
  n-groupoids                        of the kth loop space of S^{k}
----------------------------------------------------------------------
 k-tuply groupal                          the homotopy type  
 \omega -groupoids                    of the kth loop space of S^{k}
---------------------------------------------------------------------
stable \omega -groupoids                   the homotopy type 
                                   of the infinite loop space of S^{\infty }
-------------------------------------------------------------------------
   Z-groupoids                          the sphere spectrum       

\end{verbatim}
    
You may or may not know the guys on the right-hand side, but some of
them are very interesting and complicated, so it's really exciting that
they are all in some sense categorified and/or stabilized versions of 
the integers and natural numbers.  

Whew!  There is more to say, but I'll just mention a few related
papers and then quit.  If you're interested in n-categories you
could also check out "the tale of n-categories", starting in 
<A HREF = "week73.html">week73</A>.

6) Representation theory of Hopf categories, Martin Neuchl, Ph.D.
dissertation, Department of Mathematics, University of Munich,
1997.   Available at <A HREF = "http://math.ucr.edu/home/baez/neuchl.ps">http://math.ucr.edu/home/baez/neuchl.ps</A>

Just as the category of representations of a Hopf algebra gives a nice
monoidal category, the 2-category of representations of a Hopf category
gives a nice monoidal 2-category!  Categorification strikes again - and
this is perhaps our best hopes for getting our hands on the data needed
to stick into Mackaay's machine and get concrete examples of a 4d topological
quantum field theories!

7) Jim Stasheff, Grafting Boardman's cherry trees to quantum field theory,
preprint available as <A HREF = "http://xxx.lanl.gov/abs/math.AT/9803156">
math.AT/9803156</A>.


Starting with Boardman and Vogt's work, and shortly thereafter that of
May, operads have become really important in homotopy theory, string
theory, and now n-category theory; this review article sketches some 
of the connections.

8) Masoud Khalkhali, On cyclic homology of A_{\infty } 
algebras, preprint 
available 
as <A HREF = "http://xxx.lanl.gov/abs/math.QA/9805051">math.QA/9805051</A>.
 
Masoud Khalkhali, Homology of L_{\infty } algebras and cyclic homology,
preprint available 
as <A HREF = "http://xxx.lanl.gov/abs/math.QA/9805052">math.QA/9805052</A>.

An A_{\infty } algebra is an algebra that is associative <em>up to an
associator</em> which satisfies the pentagon identity <em>up to a
pentagonator</em> which satisfies it's own coherence law up to something,
ad infinitum.  The concept goes back to Stasheff's work on A_{\infty }
spaces - spaces with a homotopy equivalence to a space equipped with
an associative product.  (These are the same thing as what I called
E_{1} spaces in the translation dictionary between n-groupoid theory and
homotopy theory.)   But here it's been transported from Top over
to Vect.  Similarly, an L_{\infty } 
algebra is a Lie algebra "up to an infinity
of higher coherence laws".  Loday-Quillen and Tsygan showed that that the
Lie algebra homology of the algebra of stable matrices over an
associative algebra is isomorphic, as a Hopf algebra, to the exterior
algebra of the cyclic homology of the algebra.  In the second paper
above, Khalkali gets the tools set up to extend this result to the
category of L_{\infty } algebras.   







\par\noindent\rule{\textwidth}{0.4pt}
% </A>
% </A>
% </A>
