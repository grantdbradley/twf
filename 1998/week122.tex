
% </A>
% </A>
% </A>
\week{June 24, 1998}

In summertime, academics leave the roost and fly hither and thither,
seeking conferences and conversations in far-flung corners of the 
world.  At the end of May, everyone started leaving the Center
for Gravitational Physics and Geometry: Lee Smolin for the Santa Fe 
Institute, Abhay Ashtekar for Uruguay and Argentina, Kirill Krasnov for
his native Ukraine, and so on.  It got so quiet that I could actually 
get some work done, were it not for the fact that I, too, flew the coop: 
first for Chicago, then Portugal, and then to one of the most isolated, 
technologically backwards areas on earth: my parents' house.  Connected
to cyberspace by only the thinnest of threads, writing new issues of This 
Week's Finds became almost impossible....

I did, however, read some newsgroups, and by this means Jim Carr 
informed me that an article on spin foam models of quantum gravity had 
appeared in Science News.  I can't resist mentioning it, since it 
quotes me:

1) Ivars Peterson, Loops of gravity: calculating a foamy quantum
space-time, Science News, June 13, 1998, Vol. 153, No. 24, 376-377. 

It gives a little history of loop quantum gravity, spin networks,
and the new burst of interest in spin foams.  Nothing very technical -
but good if you're just getting started.  If you want something more 
detailed, but still user-friendly, try Rovelli's new paper: 

2) Carlo Rovelli and Peush Upadhya, Loop quantum gravity and quanta of
space: a primer, available as <A HREF =
"http://xxx.lanl.gov/abs/gr-qc/9806079">gr-qc/9806079</A>.

I haven't read it yet, since I'm still in a rather low-tech portion of
the globe, but it gives simplified derivations of some of the basic
results of loop quantum gravity, like the formula for the eigenvalues of
the area operator.  As explained in "<A HREF =
"week110.html">week110</A>", one of the main predictions of loop
quantum gravity is that geometrical observables such as the area of any
surface take on a discrete spectrum of values, much like the energy
levels of a hydrogen atom.  At first the calculation of the eigenvalues
of the area operator seemed rather complicated, but by now it's
well-understood, so Rovelli and Upadhya are able to give a simpler
treatment.

While I'm talking about the area operator, I should mention another
paper by Rovelli, in which he shows that its spectrum is not affected
by the presence of matter (or more precisely, fermions):

3) Carlo Rovelli and Merced Montesinos, The fermionic contribution
to the spectrum of the area operator in nonperturbative quantum 
gravity, available as
<A HREF = "http://xxx.lanl.gov/abs/gr-qc/9806120">gr-qc/9806120</A>.

This is especially interesting because it fits in with other pieces
of evidence that fermions could simply be the ends of wormholes - an
old idea of John Wheeler (see "<A HREF = "week109.html">week109</A>").

I should also mention some other good review articles that have turned 
up recently.  Rovelli has written a survey comparing string theory, 
the loop representation, and other approaches to quantum gravity,
which is very good because it points out the flaws in all these approaches,
which their proponents are usually all too willing to keep quiet about:

4) Carlo Rovelli, Strings, loops and others: a critical survey of the
present approaches to quantum gravity.  Plenary lecture on quantum
gravity at the GR15 conference, Pune, India, available as
<A HREF = "http://xxx.lanl.gov/abs/gr-qc/9803024">gr-qc/9803024</A>.

Also, Loll has written a review of approaches to quantum gravity that
assume spacetime is discrete.  It does \emph{not} discuss the spin foam
approach, which is too new; instead it mainly talks about lattice
quantum gravity, the Regge calculus, and the dynamical triangulations
approach.  In lattice quantum gravity you treat spacetime as a fixed
lattice, usually a hypercubical one, and work with discrete versions of
the usual fields appearing in general relativity.  In the Regge calculus
you triangulate your 4-dimensional spacetime - i.e., chop it into a
bunch of 4-dimensional simplices - and use the lengths of the edges of
these simplices as your basic variables. (For more details see "<A
HREF = "week120.html">week120</A>".)  In the dynamical
triangulations approach you also triangulate spacetime, but not in a
fixed way - you consider all possible triangulations.  However, you
assume all the edges of all the simplices have the same length - the
Planck length, say.  Thus all the information about the geometry of
spacetime is in the triangulation itself - hence the name
"dynamical triangulations".  Everything becomes purely
combinatorial - there are no real numbers in our description of
spacetime geometry anymore.  This makes the dynamical triangulations
approach great for computer simulations.  Computer simulations of
quantum gravity!  Loll reports on the results of a lot of these:

5) Renate Loll, Discrete approaches to quantum gravity in four dimensions,
available as 
<A HREF = "http://xxx.lanl.gov/abs/gr-qc/9805049">gr-qc/9805049</A>,
also available as a webpage on Living Reviews in Relativity at
<A HREF = "http://www.livingreviews.org/Articles/Volume1/1998-13loll/">
http://www.livingreviews.org/Articles/Volume1/1998-13loll/</A>

By the way, "Living Reviews in Relativity" is a cool website run by
the AEI, the Albert Einstein Institute for gravitational physics,
located in Potsdam, Germany.  The idea is that experts will write
review articles on various subjects and \emph{keep them up to date} as new
developments occur.  You can find this as follows:

6) Living Reviews in Relativity, <A HREF = "http://www.livingreviews.org">
http://www.livingreviews.org</A>

Here are some other good places to learn about the dynamical triangulations
approach to quantum gravity:

7) J. Ambjorn, Quantum gravity represented as dynamical triangulations,
Class. Quant. Grav. 12 (1995) 2079-2134.

8) J. Ambjorn, M. Carfora, and A. Marzuoli, The Geometry of Dynamical
Triangulations, Springer-Verlag, Berlin, 1998.  Also available
electronically as <A HREF =
"http://xxx.lanl.gov/abs/hep-th/9612069">hep-th/9612069</A> - watch out,
this is 166 pages long!

I can't resist pointing out an amusing relationship between dynamical
triangulations and mathematical logic, which Ambjorn mentions in his
review article.  In computer simulations using the dynamical triangulations
approach, one wants to compute the average of certain quantities over all
triangulations of a fixed compact manifold - e.g., the 4-dimensional sphere, 
S^{4}.  The typical way to do this is to start with a particular triangulation
and then keep changing it using various operations - "Pachner moves" - 
that are guaranteed to eventually take you from any triangulation of a
compact 4-dimensional manifold to any other.  

Now here's where the mathematical logic comes in.  Markov's theorem says
there is no algorithm that can decide whether or not two triangulations
are triangulations of the same compact 4-dimensional manifold.
(Technically, by "the same" I mean "piecewise linearly
homeomorphic", but don't worry about that!)  If they \emph{are}
triangulations of the same manifold, blundering about using the Pachner
moves will eventually get you from one to the other, but if they are
\emph{not}, you may never know for sure.

On the other hand, S^{4} may be special.  It's an open question whether or
not S^{4} is "algorithmically detectable".  In other words, it's an open
question whether or not there's an algorithm that can decide whether or
not a triangulation is a triangulation of the 4-dimensional sphere.  

Now, suppose S^{4} is \emph{not} algorithmically detectable.  Then the maximum
number of Pachner moves it takes to get between two triangulations of 
the 4-sphere must grow really fast: faster than any computable function!
After all, if it didn't, we could use this upper bound to know when to
give up when using Pachner moves to try to reduce our triangulation
to a known triangulation of S^{4}.  So there must be "bottlenecks" that 
make it hard to efficiently explore the set of all triangulations of S^{4} 
using Pachner moves.  For example, there must be pairs of triangulations 
such that getting from one to other via Pachner moves requires going through 
triangulations with a \emph{lot} more 4-simplices.  

However, computer simulations using triangulations with up to 65,536
4-simplices have not yet detected such "bottlenecks".  What's going on?  
Well, maybe S^{4} actually \emph{is} algorithmically detectable.  Or perhaps
it's not, but the bottlenecks only occur for triangulations that have 
more than 65,536 4-simplices to begin with.  Interestingly, one dimension 
up, it's known that the 5-dimensional sphere is \emph{not} algorithmically 
detectable, so in this case bottlenecks \emph{must} exist - but computer 
simulations still haven't seen them.  

I should emphasize that in addition to this funny computability stuff,
there is also a whole lot of interesting \emph{physics} coming out of the
dynamical triangulations approach to quantum gravity.  Unfortunately
I don't have the energy to explain this now - so read those review
articles, and check out that nice book by Ambjorn, Carfora and Marzuoli!

On another front... Ambjorn and Loll, who are both hanging out at the 
AEI these days, have recently teamed up to study causality in a lattice 
model of 2-dimensional Lorentzian quantum gravity:

9) J. Ambjorn and R. Loll, Non-perturbative Lorentzian quantum
gravity, causality and topology change, available as
<A HREF = "http://xxx.lanl.gov/abs/hep-th/9805108">hep-th/9805108</A>.

I'll just quote the abstract:

\begin{quote}
	We formulate a non-perturbative lattice model of
   	two-dimensional Lorentzian quantum gravity by performing the
	path integral over geometries with a causal structure. The
	model can be solved exactly at the discretized level. Its
	continuum limit coincides with the theory obtained by
	quantizing 2d continuum gravity in proper-time gauge, but it
	disagrees with 2d gravity defined via matrix models or
	Liouville theory. By allowing topology change of the compact
	spatial slices (i.e. baby universe creation), one obtains
	agreement with the matrix models and Liouville theory.
\end{quote}
And now for something completely different...

I've been hearing rumbles off in the distance about some interesting
work by Kreimer relating renormalization, Feynman diagrams, and Hopf
algebras.  A friendly student of Kreimer named Mathias Mertens
handed me a couple of the basic papers when I was in Portugal:

10) Dirk Kreimer, Renormalization and knot theory, Journal of Knot
Theory and its Ramifications, 6 (1997), 479-581.  Preprint available as
<A HREF = "http://xxx.lanl.gov/abs/q-alg/9607022">q-alg/9607022</A> -
beware, this is 103 pages long!

Dirk Kreimer, On the Hopf algebra structure of perturbative quantum
field theories, available as <A HREF =
"http://xxx.lanl.gov/abs/q-alg/9707029">q-alg/9707029</A>.

I'm looking through them but I don't really understand them yet.  
The basic idea seems to be something like this.  In quantum field
theory you compute the probability for some reaction among particles
by doing integrals which correspond in a certain way to pictures
called Feynman diagrams.  Often these integrals give infinite answers,
which forces you to do a trick called renormalization to cancel the
infinities and get finite answers.  Part of why this trick works is
that while your integrals diverge, they usually diverge at a well-defined 
rate.  For example, you might get something asymptotic to a constant 
times 1/d^k, where d is the spatial cutoff you put in to get a finite 
answer.  And the constant you get here can be explicitly computed.  
For example, it often involves numbers like \zeta (n), where \zeta  is the 
Riemann zeta function, much beloved by number theorists:

           \zeta (n) = 1/1^{n} + 1/2^{n} + 1/3^{n} + ....

Kreimer noticed that if you take the Feynman diagram and do some
tricks to turn it into a drawing of a knot or link, the constant
you get is related in interesting ways to the topology of this knot
or link!  More complicated knots or links give fancier constants, 
and there are all sorts of suggestive patterns.  He worked out a 
bunch of examples in the first paper cited above, and since then 
people have worked out lots more, which you can find in the 
references.  

Apparently the secret underlying reason for these patterns comes
from the combinatorics of renormalization, which Kreimer was able
to summarize in a certain algebraic structure called a Hopf algebra.
Hopf algebras are important in both combinatorics and physics, so 
perhaps this shouldn't be surprising.  But there is still a lot of 
mysterious stuff going on, at least as far as I can tell.  

What's really intriguing about all this is \emph{which} quantum field 
theories Kreimer was studying when he discovered this stuff: \emph{not}
topological quantum field theories like Chern-Simons theory, which 
already have well-understood relationship to knot theory, but instead, 
field theories that ordinary particle physicists have been thinking 
about for decades, like quantum electrodynamics, \phi ^{4} theory in 4 
dimensions, and \phi ^{3} theory in 6 dimensions - field theories where 
renormalization is a deadly serious business, thanks to nasty problems 
like "overlapping divergences".

The idea that knot theory is relevant to \emph{these} field theories 
is exciting but also somewhat puzzling, since they don't live in 
3-dimensional spacetime the way Chern-Simons theory does.  People 
familiar with Chern-Simons theory have already been seeing fascinating 
patterns relating knot theory, quantum field theory and number theory.  
Is this new stuff related?  Or is it something completely different?  
Kreimer seems to think it's related.  

According to Kirill Krasnov, the famous mathematician Alain Connes is 
going around telling people to learn about this stuff.  Apparently 
Connes is now writing a paper on it with Kreimer, and it was Connes 
who got the authors of this paper interested in the subject:

11) Thomas Krajewski and Raimar Wulkenhaar, On Kreimer's Hopf algebra
structure of Feynman graphs, available as <A HREF =
"http://xxx.lanl.gov/abs/hep-th/9805098">hep-th/9805098</A>.

Since I haven't plunged in yet, I'll just quote the abstract:

\begin{quote}
        We reinvestigate Kreimer's Hopf algebra structure of perturbative
	quantum field theories. In Kreimer's original work, overlapping
	divergences were first disentangled into a linear combination
	of disjoint and nested ones using the Schwinger-Dyson
	equation. The linear combination then was tackled by the Hopf
	algebra operations. We present a formulation where the
	coproduct itself produces the linear combination, without
	reference to external input.
\end{quote}
With any luck, mathematicians will study this stuff and finally understand
renormalization!










 \par\noindent\rule{\textwidth}{0.4pt}

% </A>
% </A>
% </A>
