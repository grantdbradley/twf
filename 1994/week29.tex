
% </A>
% </A>
% </A>
\week{January 14, 1994}

I'm awfully busy this week, but feel like attempting to keep up with the
pile of literature that is accumulating on my desk, so this will be a
rather terse description of papers.  All of these papers are related to
my current obsession with "higher-dimensional algebra" and its
applications to physics.  

1) On algebras and triangle relations, by Ruth J. Lawrence, to appear in
Proc. Top. \text{\&}  Geom. Methods in Field Theory (1992), eds. J. Mickelsson and O.
Pekonen, World Scientific, Singapore.

A presentation for Manin and Schechtman's higher braid groups, by R. J.
Lawrence, available as MSRI preprint 04129-91.

Triangulations, categories and extended topological field theories, by R. J.
Lawrence, to appear in Quantum Topology, eds L. Kauffman and R.
Baadtrio, World Scientific, Singapore, 1993.

Algebras and triangle relations, by R. J. Lawrence, Harvard U. preprint.

Many people are busily trying to extend the remarkable relationship
between knot theory and physics, which is essentially a feature of 3
dimensions, to higher dimensions.  Since the 3-dimensional case required
the development of new branches of algebra (namely, quantum groups and
braided tensor categories), it seems that the higher-dimensional cases
will require still further "higher-dimensional algebra."  One approach,
which is still being born, involves the use of "n-categories," which are
generalizations of braided tensor categories suited for higher-
dimensional physics.  (See for example the papers by Crane in "<A HREF = "week2.html">week2</A>,"
by Kapranov and Voevodsky in "<A HREF = "week4.html">week4</A>," by Fischer and Freed (separately)
in <A HREF = "week12.html">week12</A>, and the one by Gordon, Power, and Street below.)  Lawrence
has instead chosen to invent "n-algebras," which are vector spaces
equipped with operations corresponding to the ways one can subdivide
(n-1)-dimensional simplices into more such simplices.  (See the paper by
Chung, Fukuma and Shapere in "<A HREF = "week16.html">week16</A>" for some of the physics motivation
here.)  

These alternative approaches should someday be seen as different aspects
of the same thing, but there as yet I know of no theorems to this
effect, so there is a lot of work to be done.  Even more importantly,
there is a lot of work left to be done about inventing \emph{examples} of
these higher-dimensional structures.  For example, there may eventually
be general results on "boosting" n-algebras to (n+1)-algebras, or
n-categories to (n+1)-categories, which will explain how generally
covariant physics in n-dimensional spacetime relates to the same thing
in one higher dimension.  So far, however, all we have is a few
examples, which are not even clearly related to each other.  For
example, Crane calls this boosting process "categorification" and has
done it starting with the braided tensor category of representations of
a quantum group.  Lawrence, on the other hand, shows how to construct
some 3-algebras from quantum groups.  And Freed has given a general
procedure for "boosting" using path integral methods that are not yet
rigorous in the most interesting cases.

2) Coherence for tricategories, by R. Gordon, A. J. Power, and R.
Street, preprint, 81 pages.

An "n-category" is a kind of algebraic structure that has "objects,"
"morphisms" between objects, "2-morphisms" between morphisms, and so on
up to "n-morphisms."  However, the \emph{correct} definition of an n-category
for the purposes of physics is still unclear!  I gave a rough
explanation of the importance of 2-categories in physics in <A HREF = "week4.html">week4</A>, where
I discussed Kapranov and Voevodsky's nice definition of braided tensor
2-categories.  However, it seems likely that we will need to understand
the situation for larger n as well.  This paper makes a big step in this
direction, by defining "tricategories" (what I might call "weak
2-categories") and proving a "strictification" or "coherence" result
analogous to the result that every braided tensor category is equivalent
to a "strict" one.  The result is, however, considerably more subtle, as
it involves a special way of defining the tensor product of 2-categories
due to Gray:

3) Formal Category Theory: Adjointness for 2-categories, by John W.
Gray, Lecture Notes in Mathematics 391, Springer-Verlag, New York, 1974.   

Coherence for the tensor product of 2-categories, and braid groups, in
Algebras, Topology, and Category Theory, eds. A. Heller and M. Tierney,
Academic Press, New York, 1976, pp. 63-76.  

Briefly speaking, Gordon-Power-Street use a category they call
"Gray," the category of all 2-categories, made into a symmetric monoidal
closed category using a modified version of Gray's tensor product.  Then
they show that every tricategory (as defined by them) is "triequivalent"
to a category enriched over Gray.  

4) On pentagon and tetrahedron equations, by J. M. Maillet, preprint
available in LaTeX form as <A HREF = "http://xxx.lanl.gov/abs/hep-th/9312037">hep-th/9312037</A>.
 
Maillet shows how to obtain solutions of the tetrahedron equations from 
solutions of pentagon equations; all these geometrical equations are
part of the theory of 2-categories, and this is yet another example of a
"boosting" construction as alluded to above.  

5) Homologically twisted invariants related to (2+1)- and (3+1)-dimensional
state-sum topological quantum field theories, by David N. Yetter,
preprint, 6 pages, available in LaTeX form as <A HREF = "http://xxx.lanl.gov/abs/hep-th/9311082">hep-th/9311082</A>.

Let me simply quote the abstract: "Motivated by suggestions of Paolo
Cotta-Ramusino's work at the physical level of rigor relating BF theory
to the Donaldson polynomials, we provide a construction applicable to
the Turaev/Viro and Crane/Yetter invariants of \emph{a priori} finer
invariants dependent on a choice of (co)homology class on the manifold."
The dream is that this would give a state-sum formula for the Donaldson
polynomials, but Yetter is careful to avoid claiming this!  A while
back, Crane and Yetter showed how to get 4-dimensional TQFTs from certain 3d
TQFTs by another kind of "boosting" procedure related to those mentioned
above, but the resulting TQFT in 4-dimensions did not by itself yield
interesting new invariants of 4-manifolds.  The procedure Yetter
describes here generalizes the earlier work by allowing the inclusion of
an embedded 2-manifold.  
\par\noindent\rule{\textwidth}{0.4pt}

% </A>
% </A>
% </A>
