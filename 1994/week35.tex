
% </A>
% </A>
% </A>
\week{June 5, 1994}

1) Pursuing stacks (A la poursuite des champs), 1983 letter from Alexandre
Grothendieck to Daniel Quillen, 593 pages.  Scanned version available
from the Grothendieck Circle at <A HREF = "http://www.grothendieckcircle.org/">http://www.grothendieckcircle.org/</A>

I owe somebody enormous thanks for sending this to me, but I won't
mention his name, since I don't want people pestering him for copies.
(This no longer matters, now that it's available online.)
Grothendieck is mainly famous for his work on algebraic geometry, in
which he introduced the concept of "schemes" to provide a modern
framework for the subject.  He was also interested in reformulating the
foundations of topology, which is reflected in "Pursuing Stacks".  
This
is a long letter to Quillen, inspired by Quillen's 1967 book
"Homotopical Algebra".  It's a fascinating mixture of visionary
mathematics, general philosophy and a bit of personal chat.  Let me
quote a bit:

\begin{quote}
I write you under the assumption that you have not entirely lost
interest for those foundational questions you were looking at more than
fifteen years ago.  One thing which strikes me, is that (as far as I
know) there has not been any substantial progress since - it looks to me
that understanding of the basic structures underlying homotopy theory,
or even homological algebra only, is still lacking - probably because
the few people who have a wide enough background and perspective
enabling them to feel the main questions, are devoting their energies to
things which seem more directly rewarding. Maybe even a wind of
disrepute for any foundational matters whatever is blowing nowadays!  In
this respect, what seems to me even more striking than the lack of
proper foundations for homological and homotopical algebra, is the
absence I daresay of proper foundations for topology itself!  I am
thinking here mainly of the development of a context of "tame" topology,
which (I am convinced) would have on the everyday technique of geometric
topology (I use this expression in contrast to the topology of use for
analysts) a comparable impact or even a greater one, than the
introduction of the point of view of schemes had on algebraic geometry.
The psychological drawback here I believe is not anything like
messyness, as for homological and homotopical algebra (or for schemes),
but merely the inrooted inertia which prevents us so stubbornly from
looking innocently, with fresh eyes, upon things, without being dulled
and emprisoned by standing habits of thought, going with a familiar
context - \emph{too} familiar a context!\end{quote}

One reason why I'm interested in this letter is that Grothendieck seems
to have understood the importance of "higher algebraic structures"
before most people.  Recently, interest in these has been heating up,
largely because of the recent work on "extended topological quantum
field theories."  The basic idea is that, just as a traditional quantum
field theory is (among other things) a representation of the symmetry
group of spacetime, a topological quantum field theory is a
representation of a more sophisticated algebraic structure, a "cobordism
n-category."  An n-category is a wonderfully recursive sort of thing in
which there are objects, 1-morphisms between objects, 2-morphisms
between morphisms, and so on up to n-morphisms.  In a "cobordism
n-category" the objects are 0-manifolds, the 1-morphisms are
1-dimensional manifolds that go between 0-manifolds (as the unit
interval goes from one endpoint to another), the 2-morphisms are
2-dimensional manifolds that go between 1-manifolds (as a cylinder goes
from on circle to another), etc.  In practice one must work with
manifolds admitting certain types of "corners", and equipped with extra
structures that topologists and physicist like, such as orientations,
framings, or spin structures.  The idea is that all the
cutting-and-pasting constructions in n-dimensional topology can be
described algebraically in the cobordism n-category.  To wax rhapsodic
for a moment, we can think of an n-category as exemplifying the notion
of "ways to go between ways to go between ways to go between..... ways
to go between things," and cobordism n-categories are the particular
n-categories that algebraically encode the possibilities along these
lines that are implicit in the notion of n-dimensional spacetime.

Now, the problem is that the correct \emph{definition} of an n-category is a
highly nontrivial affair!  And it gets more complicated as n increases! 
A 0-category is nothing but a bunch of objects.  In other words, it's
basically just a \emph{set}, if we allow ourselves to ignore certain problems
about classes that are too big to qualify as sets.  A 1-category is
nothing but a category.  Recall the definition of a <A HREF = "categories.html">category</A>: 

A category consists of a set of \textbf{objects}
and a set of \textbf{morphisms}.  Every
morphism has a \textbf{source} object and a \textbf{target} 
object.  (A good example to think
of is the category in which the objects are sets and the morphisms are
functions.  If f:X \to  Y, we call X the source and Y the target.)  Given
objects X and Y, we write Hom(X,Y) for the set of morphisms from X to Y
(i.e., having X as source and Y as target).

The axioms for a category are that it consist of a set of objects and
for any 2 objects X and Y a set Hom(X,Y) of morphisms from X to Y, and

<OL>
<LI> Given a morphism g in Hom(X,Y) and a morphism f in Hom(Y,Z), there
is morphism which we call f o g in Hom(X,Z).  (This binary operation  o  is
called \textbf{composition}.)

<LI> Composition is associative:   (f o g) o h = f o (g o h).

<LI>  For each object X there is a morphism id|X from X to X, called the
\textbf{identity on} X.

<LI>  Given any f in Hom(X,Y), f o id|X = f and id|Y o f = f.
</OL>

Now, a 2-category is more complicated.  There are objects, 1-morphisms,
and 2-morphisms, and one can compose morphisms and also compose 
2-morphisms.  There is, however, a choice: one can make ones 2-category
"strict" and require that the rules 2) and 4) above hold for the
1-morphisms and 2-morphisms, or one can require them "literally" only
for the 2-morphisms, and allow the 1-morphisms some slack.  Technically,
one can choose between "strict" 2-categories, usually just called
2-categories, or "weak" ones, which are usually called "bicategories."  

What do I mean by giving the 1-morphisms some "slack"?  This is a very
important aspect of the n-categorical philosophy... I mean that in a
2-category one has the option of replacing \emph{equations} between
1-morphisms by \emph{isomorphisms} --- that is, by 2-morphisms that have
inverses!  The basic idea here is that in many situations when we like
to pretend things are equal, they are really just \emph{isomorphic}, and we
should openly admit this when it occurs.  So, for example, in a "weak"
2-category one doesn't have associativity of 1-morphisms.  Instead, one
has "associators", which are 2-morphisms like this:

a_{f,g,h}: (f o g) o h \to  f o (g o h) 

In other words, the associator is the \emph{process of rebracketing} made
concrete.  Now, when one replaces equations between 1-morphisms by
isomorphisms, one needs these isomorphisms to satisfy "coherence
relations" if we're going to expect to be able to manipulate them more
or less as if they \emph{were} equations.  For example, in the case of the
associators above, one can use associators to go from

f o (g o (h o k)) 

to 

((f o g) o h) o k

in two different ways: either

f o (g o (h o k)) \to  (f o g) o (h o k) \to  ((f o g) o h) o k

or

f o (g o (h o k)) \to  f o ((g o h) o k) \to  (f o (g o h)) o k \to  ((f o g) o h) o k

Actually there are other ways, but in an important sense these are the
basic two.   In a "weak" 2-category one requires that these two ways are
equal... i.e., this is an identity that the associator must satisfy,
known as the pentagon identity.  This is one of the first examples of a
coherence relation.  It turns out that if this holds, \emph{all} ways of
rebracketing that get from one expression to another are equal.  (Here
I'm being rather sloppy, but the precise result is known as Mac Lane's
theorem.) 

To learn about weak 2-categories, which as I said people usually call
bicategories, try:

2) J. Benabou, Introduction to bicategories, Lect. Notes in Math., vol.
47, Berlin, Springer-Verlag, 1968, pp. 1-71.

Now, one can continue this game, but it gets increasingly complex if one
goes the "weak" route.  In a "weak n-category" the idea is to replace
all basic identities that one might expect between j-morphisms, such as
the associative law, by (j+1)-isomorphisms.  These, in turn, satisfy
certain "coherence relations" that are really not equations, but
(j+2)-morphisms, and so on... up to level n.  This becomes so
complicated that only recently have "weak 3-categories" been properly
defined, by Gordon, Power and Street, who call them tricategories (see
"<A HREF = "week29.html">week29</A>").  

A bit earlier, Kapranov and Voevodsky succeeded in defining a
certain class of weak 4-categories, which happen to be called "braided
monoidal 2-categories" (see "<A HREF = "week4.html">week4</A>").  The
interesting thing, you see, which justifies getting involved in this
business, is that a lot of topology \emph{automatically pops out} of the
definition of an n-category.  In particular, n-categories have a lot
to do with n-dimensional space.  A weak 3-category with only one
object and one 1-morphism is usually known as a "braided monoidal
category," and the theory of these turns out to be roughly the same as
the study of knots, links and tangles!  (See "tangles".)  The "braided
monoidal 2-categories" of Kapranov and Voevodsky are really just weak
4-categories with only one object and one 1-morphism.  (The reason for
the term "2-category" here is that since all one has is 2-morphisms,
3-morphisms, and 4-morphisms, one can pretend one is in a 2-category
in which those are the objects, morphisms, and 2-morphisms.)

In any event, these marvelous algebraic structures have been cropping up
more and more in physics (see especially Crane's stuff listed in
"<A HREF = "week2.html">week2</A>" and Freed's paper listed in "<A HREF = "week12.html">week12</A>"), so I got ahold of a copy
of Grothendieck's letter and have begun trying to understand it.  

Actually, it's worth noting that these n-categorical ideas have been
lurking around homotopy theory for quite some time now.  As Grothendieck wrote:

\begin{quote}
At first sight it had seemed to me that the Bangor group had indeed
come to work out (quite independently) one basic intuition of the
program I had envisioned in those letters to Larry Breen - namely,
that the study of n-truncated homotopy types (of semisimplicial sets,
or of topological spaces) was essentially equivalent to the study of
so-called n-groupoids (where n is any natural integer).  This is
expected to be achieved by associating to any space (say) X its
"fundamental n-groupoid" \Pi _{n}(X), generalizing the
familiar Poincare fundamental groupoid for n = 1.  The obvious idea is
that 0-objects of \Pi _{n}(X) should be the points of X,
1-objects should be "homotopies" or paths between points, 2-objects
should be homotopies between 1-objects, etc.  This \Pi _{n}(X)
should embody the n-truncated homotopy type of X, in much the same way
as for n = 1 the usual fundamental groupoid embodies the 1-truncated
homotopy type.  For two spaces X, Y, the set of homotopy-classes of
maps X \to  Y (more correctly, for general X, Y, the maps of X into
Y in the homotopy category) should correspond to n-equivalence classes
of n-functors from \Pi _{n}(X) to \Pi _{n}(Y) - etc.
There are some very strong suggestions for a nice formalism including
a notion of geometric realization of an n-groupoid, which should imply
that any n-groupoid is n-equivalent to a \Pi _{n}(X).
Moreover when the notion of an n-groupoid (or more generally of an
n-category) is relativized over an arbitrary topos to the notion of an
n-gerbe (or more generally, an n-stack), these become the natural
"coefficients" for a formalism of non commutative
cohomological algebra, in the spirit of Giraud's thesis.\end{quote}

The "Bangor group" referred to includes Ronnie Brown, who
has done a lot of work on "\omega -groupoids".  A while back
he sent me a nice long list of references on this subject; here are
some that seemed particularly relevant to me (though I haven't looked
at all of them).

3) G. Abramson, J.-P.Meyer, J.Smith, A higher  dimensional
analogue  of  the fundamental groupoid, in Recent developments
of  algebraic  topology,  RIMS Kokyuroku 781, Kyoto, 38-45,
1992. 

F.Al-Agl, Aspects of multiple categories, University of Wales
PhD  Thesis, 1989.

F.Al-Agl and R.J.Steiner, Nerves of  multiple  categories,
Proc.  London Math. Soc., 66, 92-128, 1992. 

N.Ashley, Simplicial T-complexes, University of Wales PhD
Thesis, 1976, published as Simplicial T-complexes: a non-abelian
version of a theorem of Dold-Kan, Diss. Math. 165, 11-58 (1988).

H.J.Baues, Algebraic homotopy, Cambridge University Press, 1989.

H.J.Baues, Combinatorial homotopy and 4-dimensional complexes,
De Gruyter,  1991. 

L.Breen, Bitorseurs et cohomologie non-Ab&eacute;lienne, The
Grothendieck Festschrift: a collection of articles written in
honour of the 60th birthday of Alexander Grothendieck, Vol. I,
edited P.Cartier, et  al., Birkhauser, Boston, Basel, Berlin,
401-476, 1990. 

R.Brown, Higher dimensional group theory, in Low-dimensional
topology, ed. R.Brown and T.L.Thickstun, London Math. Soc. Lect.
Notes 46, Cambridge University Press, 215-238, 1982. 

R.Brown, From groups to groupoids: a brief survey,  Bull.
London  Math.  Soc., 19, 113-134, 1987. 

R.Brown, Elements  of  Modern  Topology,  McGraw  Hill,
Maidenhead,  1968;    Topology: a geometric account of general
topology, homotopy types  and  the fundamental groupoid, Ellis
Horwood, Chichester, 1988. 

R.Brown, Some problems in non-Abelian homological and
homotopical  algebra, Homotopy theory and related topics:
Proceedings Kinosaki, 1988, Edited M.Mimura, Springer Lecture
Notes in Math. 1418,  105-129,  1990. 

R.Brown,  P.J.Higgins,  The  equivalence  of
\omega -groupoids   and   cubical T-complexes, Cah. Top.
G\eom. Diff. 22, 349-370, 1981. 

R.Brown,  P.J.Higgins,  The  equivalence  of 
\infty -groupoids   and   crossed complexes, Cah. Top. G\eom.
Diff. 22, 371-386, 1981. 

R.Brown, P.J.Higgins, The algebra of cubes, J. Pure  Appl.
Algebra,  21,  233-260, 1981. 

R.Brown, P.J.Higgins, Tensor products and homotopies  for
\omega -groupoids  and  crossed complexes, J. Pure Appl.
Algebra, 47, 1-33, 1987. 

R.Brown, J.Huebschmann, Identities among relations,  in
Low-dimensional  topology, ed. R.Brown and T.L.Thickstun, London
Math. Soc. Lect. Notes   46, Cambridge University Press,
153-202, 1982. 

R.A.Brown, Generalised group presentations, Trans. Amer.
Math. Soc., 334, 519-549, 1992.  

M.Bullejos, A.M.Cegarra, J.Duskin, On cat^{n}-groups and
homotopy types, J. Pure Appl. Algebra 86 (1993) 135-154. 

M.Bullejos, P. Carrasco, A.Cegarra, Cohomology with
coefficients in symmetric cat^{n}-groups. An extension of
Eilenberg-Mac Lanes classification theorem. Granada Preprint,
1992.  

P.J.Ehlers and T. Porter, From simplicial groupoids to crossed
complexes,  UCNW Maths Preprint 92.19, 35pp, 1992.  

D.W.Jones, Polyhedral T-complexes, University of  Wales  PhD
Thesis,  1984;  published  as  A general theory of
polyhedral sets and their corresponding T-complexes,
Diss. Math. 266, 1988. 

M.M.Kapranov, V.Voevodsky, Combinatorial-geometric aspects of
polycategory theory: pasting schemes and higher Bruhat orders
(list of  results), Cah. Top. Geom. Diff. Cat. 32, 11-27,
1991. 

M.M.Kapranov, V. Voevodsky, \infty -groupoids  and  homotopy
types  Cah.  Top. G\eom. Diff. Cat. 32, 29-46, 1991. 

M.M.Kapranov, V.  Voevodsky,  2-categories  and  Zamolodchikov
tetrahedra  equations, preprint, 102pp, 1992. 

J.-L.Loday, Spaces with finitely many  non-trivial  homotopy
groups,  J. Pure Appl. Algebra, 24, 179-202, 1982. 

G.Nan Tie, Iterated  W and  T-groupoids,  J.  Pure  Appl.
Algebra,  56, 195-209, 1989.

T.Porter, A combinatorial definition of \infty -types, Topology 22
(1993) 5-24. 

S.J.Pride, Identities among relations of group  presentations,
in  E.Ghys, A.Haefliger, A. Verjodsky, eds. Proc. Workshop on
Group Theory from  a  Geometrical Viewpoint, International
Centre  of  Theoretical  Physics, Trieste, 1990, World
Scientific, (1991) 687-716.  

R.Steiner, The algebra of directed complexes, University of
Glasgow  Math Preprint, 29pp, 1992. 

A.Tonks, Cubical groups which are Kan, J. Pure Appl. Algebra
81,  83-87, 1992.  

A.Tonks and R.Brown, Calculation with simplicial and cubical
groups in Axiom, UCNW Math Preprint 93.04. 

A.R.Wolf, Inherited asphericity, links and identities among
relations, J. Pure Appl. Algebra 71 (1991) 99-107. 


\par\noindent\rule{\textwidth}{0.4pt}
<em>Since the month of March last year, so nearly a year ago, the
greater part of my energy has been devoted to a work of reflection on
the \textbf{foundations of non-commutative (co)homological algebra},
or what is the same, after all, of homotopic algebra.  These reflections
have taken the concrete form of a voluminous stack of typed notes, 
destined to for then first volume (now being finished) of a work in
two volumes to be published by Hermann, under the overall title


% parser failed at source line 446
