
% </A>
% </A>
% </A>
\week{August 19, 1994}


I've been busy, and papers have been piling up; there are lots of
interesting ones that I really should describe in detail, but I
had better be terse and list them now, rather than waiting for the
mythical day when I will have time to do them justice.  

So:

1) B. Durhuus, H. P. Jakobsen and R. Nest, 
Topological quantum field theories from generalized
6j-symbols, Reviews in Math.
Physics 5 (1993), 1-67. 

In "<A HREF = "week16.html">week16</A>" I explained a paper by Fukuma, Hosono and Kawai in which
they obtained topological quantum field theories in 2 dimensions
starting with a triangulation of a 2d surface.  The theories were
"topological" in the sense that the final answers one computed didn't
depend on the triangulation.  One can get between any two triangulations
of a surface by using a sequence of the following two moves (and their
inverses), called the (2,2) move:


\begin{verbatim}

    O                 O
   /|\               / \
  / | \             /   \
 /  |  \           /     \
O   |   O <---->  O-------O
 \  |  /           \     /
  \ | /             \   /
   \|/               \ /
    O                 O
\end{verbatim}
    

and the (3,1) move:



\begin{verbatim}

          O                      O
         /|\                    / \
        / | \                  /   \
       /  |  \                /     \
      /   |   \              /       \
     /   _O_   \   <---->   /         \
    /  _/   \_  \          /           \
   / _/       \_ \        /             \
  /_/           \_\      /               \
 O-----------------O    O-----------------O
\end{verbatim}
    

Note that in either case these moves amount to replacing one part of the
surface of a tetrahedron with the other part!  In fact, similar moves
work in any dimension, and they are often called the Pachner moves.

The really \emph{wonderful} 
thing is that these moves are also very significant
from the point of view of algebra... and especially what I call
"higher-dimensional algebra" (following Ronnie Brown), in which the
distinction between algebra and topology is largely erased, or, one
might say, revealed for the sham it always was.

For example, as explained more carefully 
in "<A HREF = "week16.html">week16</A>", the (2,2) move is
really just the same as the \emph{associative} law for multiplication.  The
idea is that we are in a 2-dimensional spacetime, and a triangle
represents multiplication: two "incoming states" go in two sides and
their product, the "outgoing state", pops out the third side:


\begin{verbatim}

                      O
                     / \
                    /   \
                   /     \
                  A       B
                 /         \
                /           \
               /             \
              /               \
             O--------AB-------O
\end{verbatim}
    

Then the (2,2) move represents associativity:



\begin{verbatim}

    O                   O
   /|\                 / \
  A | (AB)C           A   A(BC)
 /  |  \             /     \
O   AB  O   <---->  O--BC---O
 \  |  /             \     /
  B | C               B   C
   \|/                 \ /
    O                   O
\end{verbatim}
    

Of course, the distinction between "incoming" and "outgoing" sides of
the triangle is conventional, and the more detailed explanation in
"<A HREF = "week16.html">week16</A>" shows how that fits into the formalism.  
Roughly speaking, what
we have is not just any old algebra, but an algebra that, thought of as
a vector space, is equipped with an isomorphism between it and its
dual.   This isomorphism allows us to forget whether we are coming or
going, so to speak.

Hmm, and here I was planning on being terse!  Anyway, the still \emph{more}
interesting point is that when we think about 3-dimensional topology and
"3-dimensional algebra," we should no longer think of 



\begin{verbatim}

    O                 O
   /|\               / \
  / | \             /   \
 /  |  \           /     \
O   |   O   and   O-------O
 \  |  /           \     /
  \ | /             \   /
   \|/               \ /
    O                 O
\end{verbatim}
    

as representing \emph{equal} operations (the 3-fold multiplication of A, B,
and C); instead, we should think of them as merely \emph{isomorphic}, with
the tetrahedron of which they are the front and back being the
isomorphism.  The basic philosophy is that in higher-dimensional
algebra, as one ascends the ladder of dimensions, certain things which
had been regarded as \emph{equal} are revealed to be merely isomorphic.  This
gets tricky, since certain \emph{isomorphisms} that were regarded as equal 
at one level are revealed to be merely isomorphic at the next level...
leading us into a subtle world of isomorphisms between isomorphisms
between isomorphisms... which the theory of n-categories attempts to
systematize.  (I should note, however, that in the particular case of
associativity this business was worked out by Jim Stasheff quite a
while back: it's the homotopy theorists who were the ones with the guts
to deal with such issues first.)  

Now, it turns out that in 3-dimensional algebra, the isomorphism
corresponding to the (2,2) move is not something marvelously obscure.
It is in fact precisely what physicists call the "6j symbol", a gadget
they've been using to study angular momentum in quantum mechanics for a
long time!  In quantum mechanics, the study of angular momentum is just
the study of representations of the group SU(2), and if one has
representations A, B, and C of this group (or any other), the tensor
products (A tensor B) tensor C and A tensor (B tensor C) are not
\emph{equal}, but merely \emph{isomorphic}.  
It should come as no surprise that
this isomorphism is represented by physicists as a big gadget with 6
indices dangling on it, the "6j symbol".

Quite a while back, Regge and Ponzano tried to cook up a theory of
quantum gravity in 3 dimensions using the 6j symbols for SU(2).  More
recently, Turaev and Viro built a 3-dimensional topological quantum
field theory using the 6j-symbols of the \emph{quantum group} 
SU_{q}(2), and
this led to lots of work, which the above article explains in a
distilled sort of way.

The original Ponzano-Regge and Turaev-Viro papers, and various other
ones clarifying the relation of the Turaev/Viro theory to quantum
gravity in spacetimes of dimension 3, are listed in "<A HREF = "week16.html">week16</A>".  It's also
worth checking out the paper by Barrett and Foxon listed in "<A HREF = "week24.html">week24</A>", as
well as the following paper, for which I'll just quote the abstract:

2) Timothy J. Foxon, Spin networks, Turaev-Viro theory and the loop 
representation, available as 
<A HREF = "http://xxx.lanl.gov/abs/gr-qc/9408013">gr-qc/9408013</A>.

\begin{quote}

We investigate the Ponzano-Regge and Turaev-Viro topological field theories
using spin networks and their q-deformed analogues. I propose a new
description of the state space for the Turaev-Viro theory in terms of skein
space, to which q-spin networks belong, and give a similar description of
the Ponzano-Regge state space using spin networks.
I give a definition of the inner product on the skein space and show
that this corresponds to the topological inner product, defined as the
manifold invariant for the union of two 3-manifolds.
Finally, we look at the relation with the loop representation of quantum
general relativity, due to Rovelli and Smolin, and suggest that the above
inner product may define an inner product on the loop state space.
\end{quote}

(Concerning the last point I cannot resist mentioning my own paper on 
knot theory and the inner product in quantum gravity, available as
<A HREF = "http://math.ucr.edu/home/baez/tang.tex">tang.tex</A>.)


In addition to the papers by Turaev-Viro and Fukuma-Shapere listed in
"<A HREF = "week16.html">week16</A>", there are some other papers on Hopf algebras and 3d
topological quantum field theories that I should list:

3) Greg Kuperberg, Involutory Hopf algebras and three-manifold invariants, 
Internat. Jour. Math 2 (1991), 41-66.

A definition of #(M,H) in the non-involutory case, by Greg Kuperberg,
unpublished.  

Greg Kuperberg is one of the few experts on this subject who is often
found on the net; he is frequently known to counteract my rhetorical
excesses with a dose of precise information.  The above papers, one of
which is sadly still unpublished, make it beautifully clear how "algebra
knows more about topology than we do", since various basic structures
on Hopf algebras have a pleasant tendency to interact just as needed
to give 3d topological quantum field theories.  

4) John W. Barrett and Bruce W. Westbury,
Spherical categories, 
available as 
<A HREF = "http://xxx.lanl.gov/abs/hep-th/9310164">hep-th/9310164</A>.

John W. Barrett and Bruce W. Westbury, 
Invariants of piecewise-linear 3-manifolds, 
Trans. Amer. Math. Soc. 348 (1996), 3997-4022.  Also 
available
as <A HREF = "http://xxx.lanl.gov/abs/hep-th/9311155">hep-th/9311155</A>.

John W. Barrett and Bruce W. Westbury, 
The equality of 3-manifold invariants, 
available as
<A HREF = "http://xxx.lanl.gov/abs/hep-th/9406019">hep-th/9406019</A>.

Let me quote the abstract for the first one; the second one gives 
a construction of 3-manifold invariants, and the third shows that 
the authors' 3-manifold invariants agree with Kuperberg's when both
are defined.

\begin{quote}

This paper is a study of monoidal categories with duals where the
tensor product need not be commutative. The motivating examples are
categories of representations of Hopf algebras and the motivating
application is the definition of 6j-symbols as used in topological field
theories.

We introduce the new notion of a spherical category.  In the first
section we prove a coherence theorem for a monoidal category with duals
following MacLane (1963).  In the second section we give the definition
of a spherical category, and construct a natural quotient which is also
spherical.

In the third section we define spherical Hopf algebras so that the
category of representations is spherical. Examples of spherical Hopf
algebras are involutory Hopf algebras and ribbon Hopf algebras. Finally
we study the natural quotient in these cases and show it is semisimple.

\end{quote}

5) Louis H. Kauffman and David E. Radford, 
Invariants of 3-Manifolds derived from finite dimensional Hopf algebras, by
available as
<A HREF = "http://xxx.lanl.gov/abs/hep-th/9406065">hep-th/9406065</A>. 

This is paper also relates 3d topology and certain finite-dimensional
Hopf algebras, and it shows they give 3-manifold invariants distinct
from the more famous ones due to Witten (and a horde of mathematicians).
I have not had time to think about how they relate to the above ones,
but I have a hunch that they are the same, since all of them make heavy
use of special grouplike elements associated to the antipode.  

6) Louis Crane and Igor Frenkel, 
Four dimensional topological quantum field theory, Hopf categories,
and the canonical bases, 
available as
<A HREF = "http://xxx.lanl.gov/abs/hep-th/9405183">hep-th/9405183</A>.   

Work in 4 dimensions is, as one expect, still more subtle than in 3,
since again various things that were equalities becomes isomorphisms.
In particular, this means that various things one thought were vector
spaces - which are \emph{sets} 
that have \emph{elements} that you can \emph{add} and
\emph{multiply by numbers}, and which satisfy \emph{equations} 
like 

A + B = B + A

are now reinterpreted as "2-vector spaces", which are \emph{categories} that have \emph{objects} that you can \emph{direct sum} and \emph{tensor with vector spaces}, and which have certain \emph{natural isomorphisms} like the isomorphism 

A \oplus  B \cong  B \oplus  A.  

In particular, using
Lusztig's canonical basis, Crane and Frenkel start with quantum groups
(which are Hopf algebras of a certain sort) and build marvelous "Hopf
categories" out of them.  While they do not construct a 4d TQFT in this
paper, they indicate the game plan in terms clear enough that they will
probably now have to race other workers in the field to see who can get
the first interesting 4d TQFT... or perhaps something a bit subtler than
a 4d TQFT (e.g. Donaldson theory).

Finally, let me turn to a subject that is closely related (though
unfortunately this has not yet been made sufficiently clear), namely,
holonomy algebras and the loop representation of quantum gravity.  Let
me simply list the references now; many of these papers were discussed
at my session on knots and quantum gravity at the Marcel Grossman
conference, so I promise to explain at some later time (and in some
papers I'm writing) a bit more about how the loop representation of a
gauge theory is interesting from the viewpoint of higher-dimensional
algebra!

7)  
A. Ashtekar, J. Lewandowski, D. Marolf, J. Mourao and T. Thiemann,
A manifestly gauge-invariant approach to quantum theories of gauge
fields, 
contribution to the Cambridge meeting proceedings, 
available as 
<A HREF = "http://xxx.lanl.gov/abs/hep-th/9408108">hep-th/9408108</A>.  

Jerzy Lewandowski, 
Topological measure and graph-differential geometry on the quotient
space of connections, Proceedings of
"Journees Relativistes 1993", 
available as 
<A HREF = "http://xxx.lanl.gov/abs/gr-qc/9406025">gr-qc/9406025</A>.  

Abhay Ashtekar, Donald Marolf and Jose Mourao, 
Integration on the space of connections modulo gauge transformations,
available
as <A HREF = "http://xxx.lanl.gov/abs/gr-qc/9403042">gr-qc/9403042</A>. 

A. Ashtekar and R. Loll,
New loop representations for 2+1 gravity, 
available as 
<A HREF = "http://xxx.lanl.gov/abs/gr-qc/9405031">gr-qc/9405031</A>.  

R. Loll, 
Independent loop invariants for 2+1 gravity, available as
<A HREF = "http://xxx.lanl.gov/abs/gr-qc/9408007">gr-qc/9408007</A>.  

R. Loll, J.M. Mour\~{a}o and J.N. Tavares, 
Generalized coordinates on the phase space of Yang-Mills theory, 
available as <A HREF = "http://xxx.lanl.gov/abs/gr-qc/9404060">gr-qc/9404060</A>.  

C. Di Bartolo, R. Gambini and J. Griego, 
The extended loop representation of quantum gravity, 
available as <A HREF = "http://xxx.lanl.gov/abs/gr-qc/9406039">gr-qc/9406039</A>.  

Rodolfo Gambini, Alcides Garat and Jorge Pullin, 
The constraint algebra of quantum gravity in the loop representation, 
available as <A HREF = "http://xxx.lanl.gov/abs/gr-qc/9404059">gr-qc/9404059</A>.  


\par\noindent\rule{\textwidth}{0.4pt}
% </A>
% </A>
% </A>
