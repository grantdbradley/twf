
% </A>
% </A>
% </A>
\week{January 14, 1994}

For the most part, this is a terse description of some papers dealing
with quantum gravity.  Some look to be quite important, but as I have
not had time to read them as thoroughly as I would like, I won't say
much.

First, however, let me note some books:

1) QED and the Men Who Made It: Dyson, Feynman, Schwinger and Tomonaga,
by Silvan S. Schweber, Princeton Series in Physics, Princeton U. Press,
784 pages, available May 1994.

Back in the 1930s there was a crisis in physics: nobody knew how to
reconcile quantum theory with special relativity.  This book describes
the history of how people struggled with this problem and achieved a
marvelous, but flawed, solution: quantum electrodynamics (QED).
Marvelous, because it made verified predictions of unparalleled
accuracy, involves striking new concepts, and gave birth to beautiful
new mathematics.  Flawed, only because nobody yet knows for sure whether
the theory is mathematically well-defined -- for reasons profoundly
related to physics at ultra-short distance scales.  This story should
give some inspiration to those currently attempting to reconcile quantum
theory with general relativity!  Feynman, Schwinger, and Tomonaga won
Nobel prizes for QED, but Dyson was also instrumental in inventing the
theory, and the book is mainly a story of these 4 men.

2) The Music of the Heavens: Kepler's Harmonic Astronomy, by Bruce
Stephenson, Princeton U. Press, 296 pages, available July 1994. 

Kepler's Physical Astronomy, by Bruce Stephenson, Princeton U. Press,
218 pages, paperback available June 1994.

Kepler's work on astronomy was in part based on the notion of the
"music of the spheres," and in his Harmonice Mundi (1619) he sought to
relate planetary velocities to the notes of a chord.  He was also
fascinated with geometry, and sought to relate the radii of the
planetary orbits to the Platonic solids.  While this may seem a bit
silly nowadays, it's clear that this faith that mathematical
patterns pervade the heavens was a crucial part of how Kepler found his
famous laws of planetary motion.  Also important, of course, was his
use of what we would now call "physical" reasoning to understand the
heavens -- that is, the use of analogies between the motions of heavenly
bodies and that of ordinary terrestial matter.  But even this is not as
straightforward as one might hope, since (Stephenson argues in the
second book) this physical reasoning was what we would now consider
incorrect, even though it led to valid laws.  More inspiration for those
now struggling amid error to understand what the universe is really
like!

3) Temperley-Lieb Recoupling Theory and Invariants of 3-Manifolds, by
Louis Kauffman and Sostenes Lins, Annals of Mathematics Studies No. 133,
Princeton U. Press, 304 pages, available July 1994. 

I described this briefly in "<A HREF = "week17.html">week17</A>," before I had spent much time on
it.  Let me recall the main point: in the late 80's Jones invented a new
invariant of knots and links in ordinary 3d space, but then Witten
recognized that this invariant came from a quantum field theory, and
thus could be extended to obtain an invariant of links in arbitrary 3d
manifolds.  (In particular, taking the link to be empty, one obtains a
3-manifold invariant.)  In fact, there is a whole family of such
invariants, essentially one for each semisimple Lie algebra, and Jones
original example corresponded to the case su(2).  In this case the
combinatorics of the invariants are so simple that one can write a nice
exposition in which one forgets the underlying, fairly sophisticated,
mathematical physics (quantum groups, conformal field theory and the
like) and simply presents the "how-to" using a kind of diagrammatic
calculus known as "skein relations," or what Kauffman calls
"Temperley-Lieb recoupling theory."  That is the approach the authors
take here.  The curious reader will naturally want to know more!  For
example, anyone familiar with quantum theory and "6j symbols" will sense
that this kind of thing is lurking in the background, and indeed, it is.

Now for the papers:

4) The physical hamiltonian in quantum gravity, by C. Rovelli and L.
Smolin, 11 pages, preprint available in LaTeX form as <A HREF = "http://xxx.lanl.gov/abs/gr-qc/9308002">gr-qc/9308002</A>.

Fermions in quantum gravity, by H. A. Marales-Tecotl and C. Rovelli,
37 pages, preprint available in LaTeX form as <A HREF = "http://xxx.lanl.gov/abs/gr-qc/9401011">gr-qc/9401011</A>. 
 
The Rovelli-Smolin loop variables program proceeds apace!  In the former
paper, Rovelli and Smolin consider quantum gravity coupled to a scalar
matter field which plays the role of a clock.  (Using part of the system
described to play the role of a clock is a standard idea for dealing
with the "problem of time," which arises in quantum theories on
spacetimes having no preferred coordinates, like quantum gravity.
However, getting this idea to actually work is not at all easy.  For a
bit on this issue see "<A HREF = "week27.html">week27</A>".)  Only after choosing this "clock field"
can one work out a Hamiltonian for the theory, write down the analog of
Schrodinger's equation, and examine the dynamics.  Before, there is only
a "Hamiltonian constraint" equation, also known as the Wheeler-DeWitt
equation.  

In the latter paper, Rovelli and Marales-Tecotl discuss how to include
fermions.  The beautiful thing here is that fermions are described
in the loop representation by the \emph{ends} of arcs, while pure gravity is
described by loops.  This is completely analogous to the old string
theory of mesons, in which mesons were represented as arcs of "string"
-- the gluon field -- connecting two fermionic "ends" -- the quarks.  

5) Extended loops: a new arena for nonperturbative quantum gravity, 
by C. Di Bartolo, R. Gambini, J. Griego and J. Pullin, 12 pages,
preprint available in Revtex form as <A HREF = "http://xxx.lanl.gov/abs/gr-qc/9312029">gr-qc/9312029</A>.

For a while now, Gambini and collaborators have been developing a
modified version of the loop representation that appears to be
especially handy for doing perturbative calculations (perturbing in the
coupling constant, that is, Newton's gravitational constant -- \emph{not}
perturbing about a fixed flat "background" spacetime, which is regarded
as a "no-no" in this philosophy).  The mathematical basis for this
"extended" loop representation is something quite charming in itself: it
amounts to embedding the loop group into an (infinite-dimensional) Lie
group.  The "perturbative" calculations described above are thus
analogous to how one uses Lie algebras to study Lie groups.  In fact,
this analogy is a deep one, since the extended loop representation also
permits perturbative calculations in Chern-Simons theory, allowing one
to calculate "Vassiliev invariants" starting just from Lie-algebraic
data.  In fact this was done by Bar-Natan (cf "<A HREF = "week3.html">week3</A>"), who was using
the extended loop representation without particularly knowing about that
fact!

This paper puts the extended loop representation to practical use by
finding some new solutions to the quantum version of Einstein's
equations.   These solutions are essentially Vassiliev invariants!
(See also the paper by Gambini and Pullin listed in "<A HREF = "week23.html">week23</A>").  

6) Ashtekar variables in classical general relativity, by Domenico
Giulini, 43 pages, preprint available in TeX form as <A HREF = "http://xxx.lanl.gov/abs/gr-qc/9312032">gr-qc/9312032</A>.

This was a lecture given at the 117th WE-Heraeus Seminar: ``The Canonical
Formalism in Classical and Quantum General Relativity'', 13-17
September 1993, Bad-Honnef, Germany, the goal of which was to give an
introduction to Ashtekar's "new variables" for general relativity.
\par\noindent\rule{\textwidth}{0.4pt}

% </A>
% </A>
% </A>
