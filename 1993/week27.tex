
% </A>
% </A>
% </A>
\week{December 16, 1993}

This week I would like to describe some of the essays from the
following volume:

1) Conceptual Problems of Quantum Gravity, edited by Abhay Ashtekar and
John Stachel, based on the proceedings of the 1988 Osgood Hill
Conference, 15-19 May 1988, Birhhaueser, Boston, 1991.

As the title indicates, this conference concentrated not on
technical, mathematical aspects of quantum gravity but on issues
with a more philosophical flavor.  The proceedings make it clear
how many problems we still have in understanding how to fit
quantum theory and gravity together.  Indeed, the book might be a
bit depressing to those who thought we were close to the "theory
of everything" which some optimists once assured us would be
ready by the end of the millenium.  But to those like myself who
enjoy the fact that there is so much left to understand about the
universe, this volume should be exciting (if perhaps a bit
daunting).

The talks have been divided into a number of groups:


\begin{verbatim}

Quantum mechanics, measurement, and the universe
The issue of time in quantum gravity
Strings and gravity
Approaches to the quantization of gravity
Role of topology and black holes in quantum gravity
\end{verbatim}
    

Let me describe a few of the talks, or at least their background, in
some detail rather than remaining general and vague.


2) Quantum measurements and the environment-induced transition from
quantum to classical, by Wojciech H. Zurek, the volume above.

Loss of quantum coherence for a damped oscillator, by W. G.
Unruh, the volume above.


These talks by Zurek and Unruh fit into what one might call the
``post-Everett school'' of research on the foundations of quantum
theory.  To understand what Everett did, and what the
post-Everett work is about, you will need to be comfortable with
the notions of pure versus mixed states, and superpositions of
states versus mixtures of states (which are very different
things).  So, rather than discussing the talks above, it probably
makes more sense for me to talk about these basic notions.   A
brief mathematical discussion  appears below; one really needs
the clarity of mathematics to get anywhere with this sort of
issue.  First, though, let me describe them vaguely in English.  

In quantum theory, associated to any physical system there are
states and observables.  An observable is a real-valued quantity
we might conceivably measure about the system.  A state
represents what we might conceivably  know about the system.  The
previous sentence is quite vague; all it really means is this:
given a state and an observable there is a mathematical recipe
that lets us calculate a probability distribution on the real
number line, which represents the probability of measuring the
observable to have a value lying in any subset of the real line. 
We call this the probability distribution of the observable in
the state.  Using this we can, if we want, calculate the mean of this
probability distribution (let us assume it exists!), which we
call the expectation value of the observable in the state.  

Given two states \Psi  and \Phi , and a number c between 0 and 1
there is a recipe for getting a new state, called c \Psi  +
(1-c)\Phi .  This can be described roughly in words as follows:
"with probability c, the system is in state \Psi ; with probability
1-c it is in state \Phi ."   This is called a \textbf{mixture} of the
states \Psi  and \Phi .  If a state is a mixture of two different
states, with c not equal to 0 or 1, we call that state a \textbf{mixed}
state.  If a state is not mixed it is \textbf{pure}.  Roughly speaking,
a pure state is a state with as little randomness as possible. 
(More precisely, it has as little entropy as possible.)  

All the remarks so far apply to classical mechanics as well as
quantum mechanics.  A simple example from classical mechanics is
a 6-sided die.  If we ignore everything about the die except
which side is up, we can say there are six pure states: the state
in which the side of the die showing one dot is up, the state in
which the side showing two dots is up, etc..  Call these states
1,2,3,4,5, and 6. If it's a fair die, and we roll it and don't
look at it, the best state we can use to describe what we know
about the die is a mixed state which is a mixture: 1/6 of state 1
plus 1/6 of state 2, etc..  Note that if you peek at the die and
see that side 4 is actually up, you will be inclined to use a
different state to describe your knowledge: a pure state, state
4.   Your honest friend, who didn't peek, will still want to use
a mixed state.  There is no contradiction here; the state simply
is a way of keeping track of what you know about the system, or
more precisely, a device for calculating expectation values of
observables; which state you use reflects your knowledge, and
some people may know more than others.

Things get trickier in quantum mechanics.  They also get trickier
when the system being described includes the person doing the
describing.  They get even trickier when the system being
described is the whole universe -- for example, some people rebel
at the thought that the universe has "many different states" --
after all, it is how it is, isn't it?  (Gell-Mann gave a talk at
this conference, which unfortunately does not appear in this
volume, entitled "Quantum mechanics of this specific universe." 
I have a hunch it deals with this issue, which falls under the
heading of "quantum cosmology.")  

The first way things get trickier in quantum mechanics is that
something we are used to in classical mechanics fails.  In
classical mechanics, pure states are always dispersion-free --
that is, for \emph{every} observable, the probability measure assigned
by the state to that observable is a Dirac delta measure, that
is, the observable has a 100% chance of being some specific value
and a 0% chance of having any other value.  (Consider the example
of the dice, with the observable being the number of dots on the
face pointing up.)   In quantum mechanics, pure states need NOT
be dispersion-free.  In fact, they usually aren't.  

A second, subtler way things get trickier in quantum mechanics
concerns systems made of parts, or subsystems.  Every observable
of a subsystem is automatically an observable for the whole
system (but not all observables of the whole system are of that
form; some involve, say, adding observables of two different
subsystems).  So every state of the whole system gives rise to,
or as we say, "restricts to," a state of each of its subsystems.
In classical mechanics, pure states restrict to pure states.  For
example, if our system consisted of 2 dice, a pure state of the
whole system would be something like "the first die is in state 2
and the second one is in state 5;" this restricts to a pure state
for the first die (state 2) and a pure state for the second die
(state 5).  In quantum mechanics, it is \emph{not} true that a pure
state of a system must restrict to a pure state of each
subsystem.  

It is this latter fact that gave rise to a whole bunch of quantum
puzzles such as the Einstein-Podolsky-Rosen puzzle and Bell's
inequality.  And it is this last fact that makes things a bit
tricky when one of the two subsystems happens to be \emph{you}.
It is possible, and indeed very common, for the following thing
to happen when two subsystems interact as time passes.  Say the
whole system starts out in a pure state which restricts to a pure
state of each subsystem.  After a while, this need no longer be
the case!  Namely, if we solve Schroedinger's equation to
calculate the state of the system a while later, it will
necessarily still be a pure state (pure states of the whole
system evolve to pure states), but it need no longer restrict to
pure states of the two subsystems.  If this happens, we say that
the two subsystems have become "entangled."  

In fact, this is the sort of thing that often happens when one of
the systems is a measuring apparatus and the other is something
measured.  Studying this issue, by the way, does \emph{not} require a
general definition of what counts as a "measuring apparatus" or a
"measurement" -- on the contrary, this is exactly what is not
needed, and is probably impossible to attain.  What is needed is a
description in quantum theory of a \emph{particular} kind of measuring
apparatus, possibly quite idealized, but hopefully reasonably
realistic, so that we can study what goes on using quantum
mechanics and see what it actually predicts will occur. 
For example:, taking a very idealized case for simplicity:

Our system consists of two subsystems, the "detector" and an
"electron."  The systems starts out, let's suppose, in a pure
state which restricts to a pure state of each subsystem: the
detector is "ready to measure the electron's spin in the z
direction" and the electron is in a state with its spin pointing
along the x axis.  After a bit of time passes, if we restrict the
state of the whole system to the first subsystem, the detector,
we get a mixed state like "with 50% probability it has measured
the spin to be up, and with 50% probability it has measure the
spin to be down."  Meanwhile, the if we restrict the state to the
second subsystem, the electron it is in the mixed state "with 50%
change it has spin up, and with 50% chance it has spin down."  In
fact these two mixed states are \emph{correlated} in an obvious sense.
Namely, the observable of the \emph{whole} system that equals 1 if the
reading on the detector agrees with the spin of the electron, and
0 otherwise, will have expectation value 1 (if the detector is
accurate).  The catchy term "entangled," which is a little silly,
really just refers to this correlation.  I don't want to delve
into the math of correlations, but it is perhaps not surprising
that, in classical or quantum mechanics, interesting correlations
can only occur between subsystems if both of them are in mixed
states.  What's sneaky about quantum mechanics is that the whole
system can be in a pure state which when restricted to each
subsystem gives a mixed state, and that these mixed states are
then correlated (necessarily, as it turns out).  That's what
"entanglement" is all about.

It was through analyses like this, but more detailed, that
Everett realized what was going on in a quantum system composed
of two subsystems, one of which was a measuring apparatus (or
person, for that matter), the other of which was something
measured.   The post-Everett work amounts to refining Everett's
analysis by looking at more realistic examples, and more varied
examples.  In particular, it is interesting to study situations
where nothing very controlled like a scientific "measurement" is
going on.  For example, one subsystem might be an atom in outer
space, and the other subsystem might be its environment (a bunch
of other atoms or radiation).  If one started out in a state
which restricted to a pure state of each subsystem, how fast
would the subsystems become entangled?  And exactly \emph{how} would
they become entangled? -- this is very interesting.  When we are
doing a scientific measurement, it's pretty clear what sort of
correlation is involved in the entanglement.  In the above
example, say, the detector reading is becoming correlated to the
electron's spin about the z axis.  If all we have is an atom
floating about in space, it's not so clear. Can we think of the
environment as doing something analogous "measuring" something
about the atom, which establishes correlations of a particular
kind?  This is the kind of thing Zurek and Unruh are studying.

In my description above I have tried to be very matter-of-fact,
but probably you all know that this subject is shrouded in
mystery, largely because of the misty and dramatic rhetoric
people like to use, which presumably makes it seem more profound. 
At least "entangled" has a precise technical meaning.  But anyone
studying this subject will soon run into "collapse of the
wavefunction," "branches," "the many-worlds interpretation," the
"observer," and so on.  These things mean many things to many
people, and nothing in particular to many more, so one must
always be on the alert.  

Now for a little math to ground the above discussion. To keep
life simple suppose we have a quantum system described by a
n-dimensional Hilbert space H which we'' just think of as C^n,
n-dimensional complex space.  The main thing to get straight is
the difference between superpositions and mixtures of quantum
states.  An observable in quantum theory is described by a
self-adjoint operator A, which for us is just an nxn self-adjoint
matrix.  A state is something that assigns to each observable a
number called its expectation value, in a manner that is 1)
linear, 2) positive, and 3) normalized.  To explain this let us
call our state \Psi .  Linearity means \Psi (A + B) = \Psi (A) + \Psi (B)
and \Psi (cA) = c \Psi (A) for all observables A,B and real numbers
c.  Positivity means \Psi (A) > 0 when A is a nonzero matrix that
has non-negative eigenvalues (a so-called non-negative matrix).  
And the normalization condition is that \Psi (1) = 1.

This may seem unfamiliar, and that is because elementary quantum
mechanics only considers states of the form 

\Psi (A) = <v, Av>

where  v  is a unit vector in H.  Not all states are of this
form, but they are an extremely important special class of
states.  It is also important to consider states that are
represented as ``density matrices,'' which are non-negative
matrices D with trace 1:

tr(D) = Sum_i D_{ii} = 1

Such a density matrix defines a state \Psi  by

\Psi (A) = tr(AD).

It's worth checking that this really meets the definition of a
``state'' given above!

The states corresponding to unit vectors in H are in fact a
special case of the density matrices.  Namely, if v is a unit
vector in H we can let D be the self-adjoint matrix corresponding
to projection onto v.  I.e., the matrix D acts on any other
vector, say w, by

Dw = <v,w> v.

It's not to hard to check that the matrix D really is a density
matrix (do it!) and that this density matrix defines the same
state as does the vector v, that is,

tr(AD) = <v, Av>

for any observable A.

The entropy of a state \Psi  corresponding to the density matrix D is
defined to be 

S(\Psi ) = -tr(D lnD)

where one calculates D lnD by working in a basis where D is diagonal and
replacing each eigenvalue x of D by the number x lnx, which we
decree to be 0 if x = 0.  Check that if D corresponds to
a \emph{pure} state as above then D lnD = 0 so the entropy is zero.

Now about superpositions versus mixtures.  They teach you how to take
superpositions in basic quantum mechanics.  They usually don't tell you
about density matrices; all they teach you about is the states
that correspond to unit vectors in Hilbert space.  Given two unit vectors in
H, one can take any linear combination of them and, if it's not zero,
normalize it to be a unit vector again, which we call a superposition.

Mixtures are an utterly different sort of linear combination.  Given
two states \Psi  and \Phi  -- which recall are things that assign numbers
to observables in a linear way -- and given any number c between 0 and
1, we can form a new state by taking

c \Psi  + (1-c) \Phi 

This is called a mixture of \Psi  and \Phi .  Finally, some
nontrivial exercises:

Exercise: Recall that a pure state is defined to be a state which
is not a mixture of two different states with 0 < c < 1.  Show
that the states corresponding to unit vectors in Hilbert space
are pure.

Exercise: Conversely, show (in the finite-dimensional case we are
considering) that all the pure states correspond to unit vectors
in Hilbert space.

Exercise: Show that every density matrix is a mixture of states
corresponding to unit vectors in Hilbert space.  

Exercise: Show (in the finite-dimensional case we are
considering) that all states correspond to density matrices.
Show that such a state is pure if and only if its entropy is
zero.

\par\noindent\rule{\textwidth}{0.4pt}

Well, this took longer than expected, so let me quickly say a bit
more about a few other papers in the conference proceedings....


3) Is there incompatibility between the ways time is treated in
general relativity and in standard quantum mechanics?, by Carlo
Rovelli, the volume above.

The problem of time in canonical quantization of relativistic
systems, by Karel V. Kuchar, the volume above.

Time and prediction in quantum cosmology, by James B. Hartle, the
volume above.

Space and time in the quantum universe, by Lee Smolin, the volume
above.  


In the section on the problem of time in quantum gravity, these
papers in particular show a lively contrast between points of
view.  One nice thing is that discussions after the papers were
presented have been transcribed; these make the disagreements
even more clear.  Let me simply give some quotes that highlight
the issues:

Rovelli: A \textbf{partial observable} is an operation on the system
that produces a number.  But this number may be totally
unpredictable even if the sates is perfectly known. 
Equivalently, this number by itself may give no information on
the state of the system [in the Heisenberg picture - jb].  For
example, the reading of a clock, or the vluae of a field, not
knowing where and when it has been measured, are partial
observables.  

A \textbf{true observable} or simply an \textbf{observable} is an operation on
the system that produces a number than can be predicted (or whose
probability distribution may be predicted) if the (Heisenberg)
state is known.  Equivalently, it is an observable that gives
information about the state of the system.  

....

Time is an experimental fact of nature, a very basic and general
experimental fact, but just an experimental fact.  The formal
development of mechanics, and in particular Heisenberg quantum
mechanics and the presymplectic formulation of classical
mechanics, suggests that it is possible to give a coherent
description of the world that is independent of the presence of
time.

....

From the mathematical point of view, \textbf{time} is a structure on the
set of observables (the foliation that I called a time
structure).  

From the physical point of view, time is the \emph{experimental fact}
that, in the nature as we see it, meaningful observables are
always constructed out of two partial observables.  That is, it
is the experimental fact (not a priori required), that knowing
the position of a paritlce is meaningless unless we also know "at
what time" a particle was at that position.

In the formulation of the theory, this experimental fact is coded
inthe time structure of the set of observables.  If true
observables are composed of correspondences of partial
observables, one of which is the reading of a clock, then the set
of true observables can be foliated into one-parameter families
that are given by the same partial observables at different clock
readings.  

From an operational point of view, mechanics is perfectly well
defined in the absence of this time structure.  It will describe
a world (maybe one slightly unfamiliar to us) in which
observables are not arranged along one-parameter lines, in which
they have no such time structure (a kind of fixed-time world), or
have more complicated structures.  We must not confuse the
phsychological difficulties in visualizing such worlds with their
logical impossibility.

... Heisenberg states, observables, measurement theory -- none of
these require time.

The notion of probability does not require time....

What I am proposing is that there may exist a coherent
description of a system in the framework of standard quantum
mechanics even if it does not have a standard "time evolution." 

Why should we be interested in mechanics with no time structure? 
Because general relativity \emph{is} a system (a classical system)
with no time structure.  At least, it has no clearly defined time
structure.  

... What we have to do is simple: "forget time."

Kuchar: For myself, I want to see observables changing along my
world line and therefor associated with individual leaves of a
foliation.  In that sense, the problem of time is shifted to the
problem of constucting an appropriate class of quantities one
would like to call observables.  Now, what I would like to call
observables probably differs from what Carlo Rovelli would like
to call observables.  Carlo may like to restrict that term to
constants of motion, while I would like to use variables that
depend on a time hypersurface.  Of course, both of us know that
there is a technical way of translating my observables into his
observables.  However, it is difficult to subject such a
translated observable to an actual observation.  In principle, of
course, it does not matter at what instant of time one measures a
constant of motion.  But the constants of motion that are
translations of my observables are much to complicated when
expressed in terms of the coordinates and the momenta at the time
of measurement.  You thus have a hard time to design an apparatus
that would measure such a constant of motion at a time different
from the moment for which it was originally designed. 

Smolin: Now, as I discussed above, and as Jim Hartle argues at
length, there can be no strict implementation of the principle of
conservation of probability for a time that this the value of a
dynamical variable of a quantum system.  Therefore, a sensible
measurement theory for quantum cosmology can only be constructed
if there is a time variable that is not a dynamical variable of
the quantum system that describes the universe.

Does this mean that quantum cosmology is impossible, since there
is no possibility of a clock outside of the system?

There is, as far as I know, exactly one loophole in this
argument, which sit eh one exploited by the program ofintrinsic
time.  This is that one coordinate on the phase space of general
relativity might be singled out and called time in such a way
that the states, represented by functions on the configuration
space, could be read as time-dependent functions over a reduced
configuration space from which the privileged time coordinate is
excluded.  

\par\noindent\rule{\textwidth}{0.4pt}

Jokes:

Kuchar: Because Leibniz didn't believe in the ontological
significance of time, he dropped the letter "t" from his name.

Smolin: Is that true?

Kuchar: Yes!  He spelled his name with a "z".

DeWitt: It's a good thing that he did believe in space because
the "z" would've gone too.

[Of course, time is "Zeit" in German, which complicates things. - jb]


\par\noindent\rule{\textwidth}{0.4pt}

4) Old problems in the light of new variables, by Abhay Ashtekar,
the volume above.

Loop representation in quantum gravity, by Carlo Rovelli, the
volume above.

Nonperturbative quantum gravity via the loop representation, by
Lee Smolin, the volume above.


These are a bit more technical papers that give nice
introductions to various aspects of the loop representation.
\par\noindent\rule{\textwidth}{0.4pt}

% </A>
% </A>
% </A>
