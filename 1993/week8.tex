\week{March 5, 1993}
I was delighted to find that Louis Kauffman wants to speak at the workshop at UCR on knots and quantum gravity; he'll be talking on "Temperley Lieb recoupling theory and quantum invariants of links and manifolds". His books

On knots, by Louis H. Kauffman, Princeton, N.J., Princeton University Press, 1987 (Annals of Mathematics Studies, no. 115)

and more recently

Knots and physics, by Louis H. Kauffman, Teaneck, NJ, World Scientific Press, 1991 (K & E Series on Knots and Everything, vol. 1)

are a lot of fun to read, and convinced me to turn my energies towards the intersection of knot theory and mathematical physics. As you can see by the title of the series he's editing, he is a true believer the deep significance of knot theory. This was true even before the Jones polynomial hit the mathematical physics scene, so he was well-placed to discover the relationship between the Jones polynomial (and other new knot invariants) and statistical mechanics, which seems to be what won him his fame. He is now the editor of a journal, "Journal of knot theory and its ramifications."

He sent me a packet of articles and preprints which I will briefly discuss. If you read any of the stuff below, please read the delightful reformulation of the 4-color theorem in terms of cross products that he discovered! I am strongly tempted to assign it to my linear algebra class for homework....

\find{\paper{Map coloring and the vector cross product, by Louis Kauffman, J. Comb. Theory B, 48 (1990) 45.}}

Map coloring, 1-deformed spin networks, and Turaev-Viro invariants for 3-manifolds, by Louis Kauffman, Int. Jour. of Mod. Phys. B, 6 (1992) 1765-1794.

An algebraic approach to the planar colouring problem, by Louis Kauffman and H. Saleur, in Comm. Math. Phys. 152 (1993), 565-590.

As we all know, the usual cross product of vectors in $R^3$ is not associative, so the following theorem is slightly interesting:

Theorem: Consider any two bracketings of a product of any finite number of vectors, e.g.: 
\[L = a \times (b \times ((c \times d) \times e)  \quad and \quad  R = ((a \times b) \times c) \times (d \times e)\]

Let $i,j,k$ be the usual canonical basis for $R^3$:
\[i = (1,0,0) \quad j = (0,1,0) \quad k = (0,0,1)\].

Then we may assign $a,b,c,...$ values taken from ${i,j,k}$ in such a way that $L = R$ and both are nonzero.

But what's really interesting is:

Meta-Theorem: The above proposition is equivalent to the 4-color theorem. Recall that this theorem says that any map on the plane may be colored with 4 colors in such a way that no two regions with the same color share a border (an edge).

What I mean here is that the only way known to prove this Theorem is to deduce it from the 4-color theorem, and conversely, any proof of this Theorem would easily give a proof of the 4-color theorem! As you all probably know, the 4-color theorem was a difficult conjecture that resisted proof for about a century before succumbing to a computer-based proof require the consideration of many, many special cases:

Every planar map is four colorable, by K. I. Appel and W. Haken, Bull. Amer. Math. Soc. 82 (1976) 711.

So the Theorem above may be regarded as a profoundly subtle result about the "associativity" of the cross product!

Of course, I hope you all rush out now and find out how this Theorem is equivalent to the 4-color theorem. For starters, let me note that it uses a result of Tait: first, to prove the 4-color theorem it's enough to prove it for maps where only 3 countries meet at each vertex (since one can stick in a little new country at each vertex), and second, 4-coloring such a map is equivalent to coloring the edges with 3 colors in such a way that each vertex has edges of all 3 colors adjoining it. The 3 colors correspond to $i, j$, and $k$!

Kauffman and Saleur (the latter a physicist) come up with another algebraic formulation of the 4-color theorem in terms of the Temperley-Lieb algebra. The Temperley-Lieb algebra $TL_n$ is a cute algebra with generators $e_1, ..., e_{n-1}$ and relations that depend on a constant d called the "loop value":\\
$e_i^2 = de_i$\\
$e_i e_{i+1} e_i = e_i$\\
$e_i e_{i-1} e_i = e_i$\\
$e_i e_j = e_j e_i$     for $|i -j| > 1$.
The point of it becomes clear if we draw the $e_i$ as tangles on $n$ strands. Let's take $n=3$ to keep life simple. Then $e_1$ is\\

\begin{verbatim}
\  /   |
 \/    |
       |
 /\    |
/  \   |

\end{verbatim}
while $e_2$ is
\begin{verbatim}
|   \  /  
|    \/   
|      
|    /\   
|   /  \  

\end{verbatim}
In general, $e_i$ "folds over" the ith and (i+1)st strands. Note that if we square $e_i$ we get a loop - e.g., $e_1$ squared is
\begin{verbatim}
\  /   |
 \/    |
       |
 /\    |
/  \   |
\  /   |
 \/    |
       |
 /\    |
/  \   |
\end{verbatim}
Here we are using the usual product of tangles (see the article "tangles" in the collection of my expository posts, which can be obtained in a manner described at the end of this post). Now the rule in Temperley-Lieb land is that we can get rid of a loop if we multiply by the loop value d; that is, the loop "equals" $d$. So $e_1$ squared is just $d$ times 
\begin{verbatim}
\  /   |
 \/    |
       |
       |
       |
       |
       |
       |
 /\    |
/  \   |

\end{verbatim}

which - since we are doing topology - is the same as $e_1$. That's why $e_i^2$ = $de_i$.

The other relations are even more obvious. For example, $e_1$ $e_2$ $e_1$ is just \\
\\
\\
\\

\begin{verbatim}
\  /   |
 \/    |
       |
 /\    |
/  \   |
|   \  /  
|    \/   
|      
|    /\   
|   /  \  
\  /   |
 \/    |
       |
 /\    |
/  \   |
\end{verbatim}
which, since we are doing topology, is just $e_1$! Similarly, $e_2 e_1 e_2 = e_1$, and $e_i$ and $e_j$ commute if they are far enough away to keep from running into each other.

As an exercise for combinatorists: figure out the dimension of $TL_n$.

Okay, very cute, one might say, but so what? Well, this algebra was actually first discovered in statistical mechanics, when Temperley and Lieb were solving a 2-dimensional problem:

Relations between the `percolation' and `coloring' problem and other graph-theoretical problems associated with tregular planar lattices: some exact results on the `percolation' problem, by H. N. V. Temperley and E. H. Lieb, Proc. Roy. Soc. Lond. A 322 (1971), 251 - 280.

It gained a lot more fame when it appeared as the explanation for the Jones polynomial invariant of knots - although Jones had been using it not for knot theory, but in the study of von Neumann algebras, and the Jones polynomial was just an unexpected spinoff. Its importance in knot theory comes from the fact that it is a quotient of the group algebra of the braid group (as explained in "Knots and Physics"). It has also found a lot of other applications; for example, I've used it in my paper on quantum gravity and the algebra of tangles. So it is nice to see that there is also a formulation of the 4-color theorem in terms of the Temperley-Lieb algebra (which I won't present here).

\find{\paper{Knots and physics, by Louis Kauffman, Proc. Symp. Appl. Math. 45 (1992), 131-246.}}

Spin networks, topology and discrete physics, by Louis Kauffman, University of Illinois at Chicago preprint.

Vassiliev invariants and the Jones polynomial, by Louis Kauffman, University of Illinois at Chicago preprint.

Gauss codes and quantum groups, by Louis Kauffman, University of Illinois at Chicago preprint.

Fermions and link invariants, by Louis Kauffman and H. Saleur, Yale University preprint YCTP-P21-91, July 5, 1991.

State models for link polynomials, by Louis Kauffman, L'Enseignement Mathematique, 36 (1990), 1 - 37.

The Conway polynomial in $R^3$ and in thickened surfaces: a new determinant formulation, by F. Jaeger, Louis Kauffman and H. Saleur, preprint.

These are a variety of papers on knots, physics and everything.... The more free-wheeling among you might enjoy the comments at the end of the first paper on "knot epistemology."

I am going to a conference on gravity at UC Santa Barbara on Friday and Saturday, which I why I am posting this early, and why I have no time to describe the above papers. I'll talk about my usual obsessions, and hear what other people are up to, perhaps bringing back some words of wisdom for next week's "This Week's Finds". 