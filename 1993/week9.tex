\week{March 12, 1993}

\find{\paper{Surgical invariants of four-manifolds, by Boguslaw Broda, preprint available as {\href{https://arxiv.org/abs/hep-th/9302092}{arxiv:hep-th/9302092}.}}}

There a number of attempts underway to get invariants of four-dimensional manifolds (and 4d topological quantum field theories) by techniques analogous to those that worked in three dimensions. The 3-manifold invariants and 3d topological quantum field theories got going with the work of Witten on Chern-Simons theory, but since this was not rigorous a number of ways were devised to make it so. These seem different at first glance but all give the same answer. Two approaches that use a lot of category theory are the Heegard splitting approach (due to Crane, Kohno and Kontsevich, in which one writes a 3-manifold as two solid n-holed tori glued together by a diffeomorphism of their boundaries), and the surgery on links approach (due to Reshetikhin and Turaev, in which one builds up 3-manifolds by starting with the 3-sphere, cutting out thickened links and gluing them back in a different way, allowing one to define invariants of 3-manifolds from link invariants). In the case of 3 dimensions a nice paper relating the Heegard splitting and the surgery on links approaches is

Reshetikhin-Turaev and Crane-Kohno-Kontsevich 3-manifold invariants coincide, by Sergey Piunikhin, preprint, 1992. (Piunikhin is at serguei@math.harvard.edu.)

People are now trying to generalize all these ideas to 4-manifolds. There is already an interesting bunch of 4-manifold invariants out there, the Donaldson invariants, which are hard to compute, but were shown (heuristically) by Witten to be related to a quantum field theory. Lately people have been trying to define invariants using category theory; these may or may not turn out to be the same.

I've already been trying to keep you all updated on the story about Crane and Yetter's 4d TQFT. This week I'll discuss another approach, with a vast amount of help from Daniel Ruberman, a topologist at Brandeis. Any errors in what I write on this are likely to be due to my misunderstandings of what he said - caveat emptor! Broda's paper is quite terse - probably due to the race that is going on - and is based on:

A link calculus for 4-manifolds, by E. Cesar de Sa, in Topology of Low-Dimensional Manifolds, Proc. Second Sussex Conf., Lecture Notes in Math., vol. 722, Springer, Berlin, 1979, pp. 16-30.

so I should start by describing what little I understand of de Sa's work.

One can describe (compact, smooth) 4-manifolds in terms of handlebody decomposition. This allows one to actually draw pictures representing 4-manifolds. A lot of times when people first hear about topology they get they impression that it's all about rubber doughnuts, Mobius strips, and other Dali-esque wiggly objects in hyperspace. Then, when they take courses in it, they are confronted with nasty separation axioms and cohomology theories! This is just to scare away outsiders! Handlebody theory really is about wiggly objects in hyperspace, and it's lots of fun - though to be good in it you need to know your point set topology and your algebraic topology, I'm afraid - and much better than I do!
\\
Recall: 
\\
$D^n$ = unit ball in $R^n$\\
$S^n$ = unit sphere in $R^{n+1}$\\
\\
In particular note that $S^0$ is just two points. Note that:
\\
the boundary of $D^4$ is $S^3$\\
the boundary of $D^3 \times D^1$ is $D^3 \times S^0 \bigcap S^2 \times D^1$\\
the boundary of $D^2 \times D^2$ is $D^2 \times S^1 \bigcap S^1 \times D^2$\\
the boundary of $D^1 \times D^3$ is $D^1 \times S^2 \bigcap S^0 \times D^3$\\
the boundary of $D^4$ is $S^3$\\

I have written this rather redundant chart in a way that makes the pattern very clear and will come in handy below for those who aren't used to this stuff.

To build up a 4-manifold we can start with a "0-handle," $D^4$, which has as boundary $S^3$.

Then we glue on "1-handles," that is, copies of $D^3 \times D^1$. Note that part of the boundary of $D^1 \times D^3$ is $D^3 \times S^0$, which is two $D^3$'s; when we glue on a 1-handle we simply attach these two $D^3$'s to the $S^3$ by a diffeomorphism. The resulting space is not really a smooth manifold, but it can be smoothed. It then becomes a smooth 4-manifold with boundary.

Then we glue on "2-handles" by attaching copies of $D^2 \times D^2$ along the part of their boundary that is $D^2 \times S^1$. Then we smooth things out.

Then we glue on "3-handles" by attaching copies of $D^1 \times D^3$ along the part of their boundary that is $D^1 \times S^2$. Then we smooth things out.

Then we glue on "4-handles" by attaching copies of $D^4$ along their boundary, i.e. $S^3$.

We can get any compact oriented 4-manifold this way using attaching maps that are compatible with the orientations. The reader who is new to this may enjoy constructing 2-manifolds in an analogous way. Compact oriented 2-manifolds with boundary are just n-holed tori.

What's cool is that with some tricks one can still draw what's going in the case of 3-manifolds and 4-manifolds. Here I'll just describe how it goes for 4-manifolds, since that's what Cesar de Sa and Broda are thinking about. By the way, a good introduction to this stuff is

The Topology of 4-manifolds, by Robion C. Kirby, Springer-Verlag Lecture Notes in Mathematics (1989), vol. 1374.

So - here is how we draw what's going on. I apologize for being somewhat sketchy here (sorry for the pun, too). I am a bit rushed since I'm heading off somewhere else next weekend... and I am not as familiar with this stuff as I should be.

So, when we start with our 0-handle, or $D^4$, we "draw" its boundary, $S^3$. Think of $S^3$ as $R^3$ and a point at infinity. Since we use perspective when drawing pictures of 3-d objects, this boils down to pretending that our blackboard is a picture of $S^3$!

As we add handles we continue to "draw" what's happening at the boundary of the 4-manifold we have at each stage of the game. 1-handles are attached by gluing a $D^3 \times D^1$ onto the boundary along two $D^3$'s - or balls - so we can just draw the two balls.

2-handles are attached by gluing a $D^2 \times D^2$ onto the boundary of the 4-manifold we have so far along a $D^2 \times S^1$ - or solid torus, so we just need to figure out how to draw an embedded solid torus. Well, for this we just need to draw a knot (that is, an embedded circle), and write an integer next to it saying how many times the embedded solid torus "twists" - plus or minus depending on clockwise or counterclockwise - as we go around the circle. In other words, an embedded solid torus is (up to diffeomorphism) essentially the same as a framed knot. If we are attaching a bunch of 2-handles we need to draw a framed link.

Things get a bit hairy in the case when one of the framed links goes through one of the 1-handles that we've already added. It's easier to draw this situation if we resort to another method of drawing the 1-handles. It's a bit more subtle, and took me quite a while to be able to visualize (unfortunately I seem to have to visualize this stuff to believe it). So let's go back to the situation where we have $D^4$ $S^3$ as its boundary, and we are adding 1-handles. Instead of drawing two balls, we draw an unknotted circle with a dot on it! The dot is just to distinguish this kind of circle from the framed links we already have. But what the circle means is this. The circle is the boundary of an obvious $D^2$, and we can push the interior of this $D^2$ (which is sitting in the $S^3$) into the interior of $D^4$. If we then remove a neighborhood of the $D^2$, what we have left is $S^1 \times D^3$, which is just the result of adding a 1-handle to $D^4$.

This is probably easier to visualize one dimension down: if we have a good old unit ball, $D^3$, and slap an interval, or $D^1$, onto its boundary, and then push the interior of the interval into the interior of the ball, and remove a neighborhood of the interval, what we have left is just an $S^1 \times D^2$.

So in short, we can draw all the 1-handles by drawing unlinked, unknotted circles with dots on them, and then draw all the 2-handles by drawing framed links that don't intersect these circles.

At this point, if you have never seen this before, you are probably dreading the 3-handles and 4-handles. Luckily a theorem comes to our rescue! If we start at the other end of our handlebody decomposition, as it were, we start with 4-handles and glue on 3-handles. If you ponder the chart and see what the pattern of what we're doing is, you'll see that a single 4-handle with some 3-handles stuck on is just the same as a 0-handle with some 1-handles stuck on. So when we now glue this thing (or things) onto the stuff we've built out of 0-, 1-, and 2-handles, we are doing so using a diffeomorphism of its boundary. But a theorem of Laudenbach and Poenaru,

A note on 4-dimensional handlebodies, by F. Laudenbach and V. Poenaru, Bull. Math. Soc. France 100 (1972), pp. 337-344,

says that any such diffeomorphism extends to one of the interior. This means that it doesn't make a darn bit of difference which diffeomorphism we use to glue it on. In short, all the information is contained in the 1- and 2-handles, so we can draw 4-manifolds by first drawing a batch of unknotted unlinked circles with dots on them and then drawing a framed link in the complement.

[A question for the experts, since I'm just learning this stuff: in the above we seem to be assuming that there's only one 0-handle. Is this an okay assumption or do we need something fancier if there's more?]

Now a given 4-manifold may have lots of different handlebody decompositions. So, as usual, we would like to have a finite set of "moves" that allow us to get between any pair of handlebody decompositions of the same 4-manifold. Then we can construct a 4-manifold invariant by cooking up a number from a handlebody decomposition - presented as a picture as above, if we want - and showing that it doesn't change under these "moves".

So, what de Sa did was precisely to find such a set of moves. (There, that's what I understand of his work!)

And what Broda did was precisely to use the Kauffman bracket invariant of framed links to cook up an invariant of 4-manifolds from the handlebody decomposition - which, note, involves lots of links. Recall that the Kauffman bracket assigns to each link a polynomial in one variable, $q$. Here "$q$" is just the same $q$ that appears in the quantum group $SU_q(2)$. As I mentioned in {\hyperref[week5]{week5.tex}} this acts quite differently when $q$ is a root of unity, and the 3d topological quantum field theories coming from quantum groups, as well as Crane and Yetter's 4d topological quantum field theory, come from considering this root-of-unity case. So it's no surprise that Broda requires $q$ to be a root of unity.

Ruberman had some other remarks about Broda's invariant, but I think I would prefer to wait until I understand them....

\find{\paper{Minisuperspaces: symmetries and quantization, by Abhay Ashtekar, Ranjeet S. Tate and Claes Uggla Syracuse University preprint SU-GP-92/2-5, 14 pages, available in latex form as \href{https://arxiv.org/abs/gr-qc/9302026}{gr-qc/9302026}.}}

Minisuperspaces: observables and quantization, Abhay Ashtekar, Ranjeet S. Tate and Claes Uggla Syracuse University preprint SU-GP-92/2-6, 34 pages, available in latex form as \href{https://arxiv.org/abs/gr-qc/9302027}{gr-qc/9302026}

I was just at the Pacific Coast Gravity Meeting last weekend and heard Ranjeet Tate talk on this work. Recall first of all that minisuperspaces are finite-dimensional approximations to the phase space of general relativity, and are used to get some insight into quantum gravity. I went through an example in {\hyperref[week6]{week6.tex}}. In these papers, the authors quantize various "Bianchi type" minisuperspace models. The "Bianchi type" business comes from a standard classification of homogeneous (but not necessarily isotropic) cosmologies and having a lot of symmetry. It is based in part on Bianchi's classification of 3-dimensional Lie algebras into nine types. The second paper gives a pretty good review of this stuff before diving into the quantization, and I should learn it!

The most exciting aspect of these papers, at least to the dilettante such as myself, is that one can quantize these models and show that quantization does NOT typically remove the singularities ("big bang" and/or "big crunch"). Of course, these models have only finitely many degrees of freedom, and are only a caricature of full-fledged quantum gravity, so one can still argue that real quantum gravity will get rid of the singularities. But a number of general relativists are arguing that this is not the case, and we simply have to learn to live with singularities. So it's good to look at models, however simple, where one can work things out in detail, and not just argue about generalities.

\find{\paper{Unique determination of an inner product by adjointness relations in the algebra of quantum observables, by Alan D. Rendall, Max-Planck-Institut fuer Astrophysik preprint.}}

I had known Rendall from his work on the perturbative expansion of the time evolution operators in classical general relativity. He became interested in quantum gravity a while ago and visited Ashtekar and Smolin at Syracuse University, since (as he said) the best way to learn is by doing. There he wrote this paper on Ashtekar's approach to finding the right inner product for the space of states of quantum gravity. I had heard about this paper, but hadn't seen it until I met Rendall at the gravity meeting last weekend. He gave me a copy and explained it. It is a simple and beautiful paper - such nice mathematical results that I am afraid someone else may have found them earlier somewhere.

Ashtekar's idea is to fix the inner product by requiring that the physical observables, which are operators on the space of states, be self-adjoint. Rendall shows the following. Let $A$ be a *-algebra acting on a vector space $V$. Let us say that an inner product on $V$ is "strongly admissable" if 1) the representation is a *-representation with respect to this inner product, 2) for each element of $A$, the corresponding linear transformation on $V$ is bounded relative to the norm given by this inner product, and 3) the completion of $V$ in the inner product is a topologically irreducible representation of $A$. Rendall shows the uniqueness of a strongly admissible inner product on any representation $V$ of $A$ (up to a constant multiple). Of course, such an inner product need not exist, but when it does, it is unique. This is as nice a result along these lines as one could hope for. He also has a more complicated result that applies to unbounded operators. A good piece of work on the foundations of quantum theory!

\find{\paper{Thawing the frozen formalism: the difference between observables and what we observe, by Arlen Anderson, preprint available as}\href{https://arxiv.org/abs/gr-qc/9211028}{gr-qc/9211028}.}


There were a number of youngish folks giving talks at the gravity meeting who have clearly been keeping up with the recent work on the problem of time and other conceptual problems in quantum gravity. In very brief terms, the problem of time is that in general relativity, we have not a Hamiltonian in the traditional sense, but a "Hamiltonian constraint" H = 0, so when we quantize it superficially appears that there are no dynamics whatsoever (as it seems like we have a zero Hamiltonian!). That's the reason for the term "frozen formalism" - and the desire to "thaw" it, or find the dynamics lurking in it. In fact, the Hamiltonian constraint is just a reflection of the fact that general relativity has no preferred time coordinate, and we are just learning how to deal with the quantum theory of such systems. For a good survey of the problem and some new proposed solutions, I again refer everyone to Isham's paper:

\find{\paper{Canonical Quantum Gravity and the Problem of Time, Chris J. Isham, 125 pages of LaTeX output, preprint available as \href{https://arxiv.org/abs/gr-qc/9210011} {gr-qc/9210011}.}}

In particular, one interesting approach is due to Rovelli, and is called "evolving constants of motion" (a deliberate and very accurate oxymoron). While there are serious technical problems with this approach, it's very natural from a physical point of view - at least once you get used to it. I have the feeling that the younger physicists are, as usual, getting used to it a lot more quickly than the older folks who have been pondering the problem of time for many years. Anderson is one of these younger folks, and his paper develops Rovelli's approach in terms of in a toy model, namely the case of two free particles satisfying the Schrodinger equation.

\find{\paper{The extended loop group: an infinite dimensional manifold associated with the loop space, by Cayetano Di Bartolo, Rodolfo Gambini and Jorge Griego, 42 pages, preprint available as \href{https://arxiv.org/abs/gr-qc/9303010}{gr-qc/9303010}.}}

Unfortunately I don't have the time now to give this paper the discussion it deserves. Gambini is one of the original inventors of the loop representation of gauge theories, so his work is especially worth paying attention to. He explained the idea of this paper to me a while back. Its aim is to provide a workable "calculus" for the loop representation by enlarging the ordinary loop group to a larger group which is actually an infinite-dimensional Lie group - the point being that the usual loop group doesn't have a Lie algebra, but this one does. As one might expect, the Lie algebra of this group is closely related to the theory of Vassiliev invariants. The paper considers some applications to quantum gravity and knot theory. 



