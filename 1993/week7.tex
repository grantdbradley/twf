\week{March 1, 1993}
\find{\paper{Mathematical problems of non-perturbative quantum general relativity (lectures delivered at the 1992 Les Houches summer school on Gravitation and Quantization), by Abhay Ashtekar, 87 pp, Plain TeX, available as {\href{https://arxiv.org/abs/gr-qc/9302024}{arXiv:gr-qc/9302024}.}}}

I described this paper in {\hyperref[week3]{week3.tex}}, but now it's available from gr-qc. It's a good quick introduction to the loop representation of quantum gravity.

\find{\paper{Lectures on Non-perturbative Canonical Gravity, by Abhay Ashtekar, World Scientific Press, 1991.}}

This book, which I finally obtained, is the introduction to the loop representation of quantum gravity. What's the loop representation? Well, this is a long story, so you really should read the book. But just to get you going, let me describe Ashtekar's "new variables," which form the basis for Rovelli and Smolin's construction of the loop representation.

First, recall that general relativity is usually thought of as a theory about a metric on spacetime - more precisely, a Lorentzian metric. Here spacetime is a 4-dimensional manifold, and a Lorentzian metric allows you to calculate the "dot product" of any two tangent vectors at a point. This is in quotes because, while a normal dot product might look like.

\[(v_0,v_1,v_2,v_3) \cdot (w_0,w_1,w_2,w_3) =  v_0w_0 + v_1w_1 + v_2w_2 + v_3w_3\]
relative to some basis, for a Lorentzian metric we can always find a basis of the tangent space such that
\[(v_0,v_1,v_2,v_3)\cdot (w_0,w_1,w_2,w_3) = v_0w_0 - v_1w_1 - v_2w_2 - v_3w_3\]

Now the metric in general relativity defines a "connection," which tells you a tangent vector might "twist around" as you parallel translate it, that is, move it along while trying to keep it from rotating unnecessarily. Here "twist around" is in quotes because, since you are parallel translating the vector, it's not really "twisting around" in the usual sense, but it might seem that way relative to some coordinate system. For example, if you used polar coordinates to describe parallel translation on the plane, it might seem that the unit vector in the r direction "twisted around" towards the $\theta$ direction as you dragged it along. But in another coordinate system - say the usual $x-y$ system - it would not appear to be "twisting around". This fact is expressed by saying "the connection is not a tensor".

But from the connection we can cook up a big fat tensor, the "Riemann tensor" $R^i_{jkl}$, which says how much the vector in the lth direction (here the indices range from 0 to 3) twists towards the ith direction when you move it around a teeny little square in the j-k plane. The Lagrangian in ordinary GR is just the integral of the "Ricci scalar curvature," R, which is gotten from the Riemann tensor by "contraction", i.e. summing over the indices in a certain way:

\[R = R^i_{ji}^j\]

Where we are raising indices using the metric in a manner beloved by physicists and feared by many mathematicians. If you integrate the Lagrangian over a region of spacetime you get the "action", and in classical general relativity (in a vacuum, for simplicity) one can formulate the laws of motion simply by saying: any teeny change in the metric that vanishes on the boundary of the region should leave the action constant to first order. In other words, the solutions of the equations of general relativity are the *stationary points* of the action. If you know how to do variational calculus you can derive Einstein's equations from this variational principle, as it's called. Mathematicians will be pleased to know that Hilbert beat Einstein to the punch here, so the integral of R is called the "Einstein-Hilbert" action for general relativity.

But there's another formulation of general relativity in terms of an action principle. This is called the "Palatini" action - and actually I'm going to describe a slight variation on it, that is conceptually simpler, and apparently appears for the first time in Ashtekar's book. The Palatini approach turns out to be more elegant and is a nice stepping-stone to the Ashtekar approach. In the Palatini approach one thinks of general relativity not as being a theory of a metric, but of a "tetrad" and an "$\mathfrak{so(3,1)}$ connection". To explain what these are, I will cut corners and assume all the fiber bundles lurking around are trivial; the experts will easily be able to figure out the general case. So: an (orthonormal) tetrad, or "vierbein," is a just a kind of field on spacetime which at each point consists of an (ordered) orthonormal basis of the tangent space. If we express the metric in terms of a tetrad, it looks just like the formula for the standard "inner product"
\[(v_0,v_1,v_2,v_3)\cdot(w_0,w_1,w_2,w_3) = v_0w_0 - v_1w_1 - v_2w_2 - v_3w_3\]
This allows us to identify the group of linear transformations of the tangent space that preserve the metric with the group of linear transfomations preserving the standard "inner product," which is called $SO(1,3)$ since there's one plus sign and three minuses. And from the connection mentioned above one gets an $SO(1,3)$ connection, or, what's more or less the same thing, an so (1,3)-valued 1-form, that is, a kind of field that can eat a tangent vector at any point and spits out element of the Lie algebra $\mathfrak{so}(1,3)$.

What's $\mathfrak{so}(1,3)$? Well, elements of $\mathfrak{so(1,3)}$ include "infinitesimal" rotations and Lorentz transformations, since $SO(1,3)$ is generated by rotations and Lorentz transformations. More precisely, $\mathfrak{so(1,3)}$ is a 6-dimensional Lie algebra having as a basis the three infinitesimal rotations $J_1, J_2$, and $J_3$ around the three axes, and the three infinitesimal Lorentz transformations or "boosts" $K_1, K_2, K_3$. The bracket in this most important Lie algebra is given by\\
$[J_i,J_j] = J_k$\\
$[K_i,K_j] = -J_k$\\
$[J_i,K_j] = K_k$\\
where $(i,j,k)$ is a cyclic permutation (1,2,3). (I hope I haven't screwed up the signs.) Note that the $J$'s by themselves form a Lie subalgebra called $\mathfrak{so}(3)$, the Lie algebra of the rotation group $SO(3)$. Note that $\mathfrak{so}(3)$ is isomorphic to the the cute little Lie algebra $\mathfrak{su}(3)$ I described in my post {\hyperref[week5]{week5.tex}}; $J_1, J_2$, and $J_3$ correspond to the guys $I, J$, and $K$ divided by two.

The $\mathfrak{so(1,3)}$ connection has a curvature, and using the tetrads again we can identify this with the Riemann curvature tensor. So the Palatini trick is to rewrite the Einstein-Hilbert action in terms of the curvature of the $\mathfrak{so(1,3)}$ connection and the tetrad field. This is called the Palatini action. Charmingly, even though the tetrad field is utterly unphysical, we can treat it and the $\mathfrak{so(1,3)}$ connection as independent fields and, doing calculus of variations to find stationary points of the action, we get equations equivalent to Einstein's equations.

Ashtekar's "new variables" - from this point of view - rely on a curious and profound fact about $\mathfrak{so(1,3)}$. Note that $\mathfrak{so(1,3)}$ is a Lie algebra over the real numbers. But if we allow ourselves to form complex linear combinations of the $J$'s and $K$'s, thus:
\\
$M_i = \frac{(J_i + iK_i)}{2}$\\
$N_i = \frac{(J_i - iK_i)}{2}$\\
(please don't mix up the subscript $i = 1,2,3$ with the other $i$, the square root of minus one) we get the following brackets:

\\
$[M_i,M_j] = M_k$\\
$[N_i,N_j] = N_k$\\
$[M_i,N_j] = 0$\\
I think the signs all work but I wouldn't trust me if I were you. The wonderful thing here is that the $M$'s and $N$'s commute with each other, and each set has commutation relations just like the $J$'s! The $J$'s, recall, are infinitesimal rotations, and the Lie algebra they span is $\mathfrak{so}(3)$. So in a sense the Lie algebra of the Lorentz group can be "split" into "left-handed" and "right-handed" copies of $\mathfrak{so}(3)$, also known as "self-dual" and "anti-self-dual" copies. This is, in fact, what lies behind the handedness of neutrinos, and many other wonderful things.

But let me phrase this result more precisely. Since we allowed ourselves complex linear combinations of the J's and K's, we are now working in the "complexification" of the Lie algebra $\mathfrak{so}(3,1)$, and we have shown that this Lie algebra over the complex numbers splits into two copies of $\mathfrak{so}(3,\C)$, the complexification of $\mathfrak{so}(3)$.

Ashtekar came up with some "new variables" for general relativity in the context of the Hamiltonian approach. Here we are working in the Lagrangian approach, where things are simpler because they are "generally covariant," not requiring a split of spacetime into space and time. The Lagrangian approach to the new variables is due to Samuel, Jacobson and Smolin, and in this approach all they amount to is this: $\mathfrak{so}(1,3)$ connection of the Palatini approach, think of the$\mathfrak{so}(1,3)$ as sitting inside the complexification thereof, and consider only the "right-handed" part! Thus, from an $\mathfrak{so}(1,3)$ connection, we get a $\mathfrak{so}(3,\C)$ connection. The "new variables" are just the tetrad field and this $\mathfrak{so}(3,\C)$ connection.

I have tried to keep down the indices but I think I will write down the Palatini Lagrangian and then the "new variables" Lagrangian, without explaining exactly what they mean, just to show how amazingly similar-looking they are. In the Palatini approach we have a tetrad field, which now we write in its full glory as $e_I^i$, and the curvature of the $\mathfrak{so}(1,3)$ connection, which now we write as $\Omega_{ij}^{IJ}$. The Lagrangian is then
$e_I^i e_J^j \Omega_{ij}^{IJ}$
(which we integrate against the usual volume form to get the action). In the new variables approach we have a tetrad field again, and we write the curvature of the $\mathfrak{so}(3,\C)$  connection as $F_{ij}^{IJ}$. (This turns out to be just the "right-handed" part of  $\Omega_{ij}^{IJ}$.) The Lagrangian is
$e_I^i e_J^j F_{ij}^{IJ}$ !

Miraculously, this also gives Einstein's equations.

What's utterly unclear from what I've said so far is why this helps so much in trying to quantize gravity. I may eventually get around to writing about that, but for now, read the book!

\find{\paper{We are not stuck with gluing, by David Yetter and Louis Crane, preprint available as {\href{https://arxiv.org/abs/hep-th/9302118}{arXiv:hep-th/9302118}.}}}

Well, in {\hyperref[week2]{week2.tex}} I mentioned Crane and Yetter's marvelous construction of a 4d topological quantum field theory using the representations of the quantum group $SU_q(2)$ - and in {\hyperref[week5]{week5.tex}} I mentioned Ocneanu's "proof" that the resulting 4-manifold invariants were utterly trivial (equal to 1 for all 4-manifolds). Now Crane and Yetter have replied, saying that their 4-manifold invariants are not trivial and that Ocneanu interpreted their paper incorrectly. I look forward to the conference on quantum topology in Kansas at the end of May, where the full story will doubtless come out.
\\
\find{\paper{The initial value problem in light of Ashtekar's variables, by R. Capovilla, J. Dell and T. Jacobson, preprint available as {\href{https://arxiv.org/abs/gr-qc/9302020}{{arxiv:gr-qc/9302020}},15 pages.}}}
The advantage of Ashtekar's new variables is that the simplify the form of the constraint equations one gets in the initial-value problem for general relativity. This is true both of the classical and quantum theories. Rovelli and Smolin used this to find, for the first time, lots of states of quantum gravity defined by link invariants. Here the above authors are trying to apply the new variables to the \textit{classical} theory.

\find{\paper{Combinatorial expression for universal Vassiliev link invariant, by Sergey Piunikhin, preprint available as {\href{https://arxiv.org/abs/hep-th/9302084}{{arxiv:hep-th/9302084}}.}}}
Somebody ought to teach those Russians how to use the word "the" now that the cold war is over. Anyway, this paper defines a kind of universal object for Vassiliev invariants, which is sort of similar to what I was trying to do in

\find{\paper{Link invariants of finite type and perturbation theory, by John Baez, Lett. Math. Phys. 26 (1992) 43-51.}}

but more concrete, and (supposedly) simpler than Kontsevich's approach. My parenthesis simply indicates that I haven't had time to figure out what's going on here.


