
% </A>
% </A>
% </A>
\week{August 8, 2001}


I've been travelling around a lot lately.  For a couple of weeks I was
in Turkey, resisting the lure of the many internet cafes.  I urge you
all to visit Istanbul when you get a chance!  Fascinating music fills
the streets.  There are a lot of nice bookstore-cafes on Istiklal
Caddesi near Taksim Square, and a huge number of musical instrument
shops at the other end of this street, down near Tunel Square.  I bought
a nice doumbek at one of these shops, and looked at lots of ouzes, sazes
and neys, none of which I can play.  It's also imperative to check out
the Grand Bazaar, the mosques, and the Topkapi Palace - the harem there
has most beautiful geometric tiling patterns I've ever seen.  I'm not
sure why that's true; perhaps this is where the sultans spent most of
their time.

The mathematics of tilings is a fascinating subject, but that's not what
I'm going to talk about.  After my trip to Turkey, I went to a conference 
at Stanford:

1) Conference on Algebraic Topological Methods in Computer Science,
Stanford University, 
<A HREF = "http://math.stanford.edu/atmcs/index.htm">http://math.stanford.edu/atmcs/index.htm</A>

There were lots of fun talks, but I'll just mention two.

The talk most related to physics was the one by my friend Dan Christensen, 
who spoke on "Spin Networks, Spin Foams and Quantum Gravity", describing 
a paper he is writing with Greg Egan on efficient algorithms for computing 
Riemannian 10j symbols.  Dan is a homotopy theorist at the University of 
Western Ontario, and Greg is my favorite science fiction writer.   They're 
both interested in quantum gravity, and they're both good at programming.  
Together with some undergraduate students of Dan's, the three of us are 
starting to study the Riemannian and Lorentzian Barrett-Crane models of 
quantum gravity with the help of computer simulations.  But to get anywhere 
with this, we need to get good at computing "10j symbols".  

Huh?  "10j symbols"??

Well, as with any quantum field theory, the key to the Barrett-Crane model 
is the partition function.  In the Riemannian version of this theory, you 
compute the partition function as follows.  First you take your 4-dimensional 
manifold representing spacetime and triangulate it.   Then you label all the 
triangles by spins j = 0, 1/2, 1, 3/2, etcetera.  Following certain specific 
formulas you then calculate a number for each 4-simplex, a number for each 
tetrahedron, and a number for each triangle, using the spin labellings.  Then 
you multiply all these together.   Finally you sum over all labellings to 
get the partition function.   The only tricky part is the convergence of 
this sum, which was proved by Perez:

2) Alejandro Perez, Finiteness of a spin foam model for euclidean 
quantum general relativity, Nucl. Phys. B599 (2001) 427-434. 
Also available as
<A HREF = "http://xxx.lanl.gov/abs/gr-qc/0011058">gr-qc/0011058</A>.


The most interesting aspect of all this is the formula giving numbers for 
4-simplices.  A 4-simplex has 10 triangular faces all of which get labelled 
by spins, and the formula says how to compute a number from these 10 spins - 
the so-called "10j symbol".

How do you compute 10j symbols?  One approach involves representation theory,
or in lowbrow terms, multiplying a bunch of matrices.  Unfortunately, if you 
go about this in the most simple-minded obvious fashion, when the spins 
labelling your triangles are all about equal to j, you wind up needing to 
work with matrices that are as big as N x N, where

N = (2j+1)^{12}.

If you do this, already for j = 1/2 you are dealing with square matrices 
that are 2^{12} by 2^{12}.  This is too big to be practical!

In computer science lingo, this algorithm sucks because it uses
O(j^{12}) time and also O(j^{12}) space.  You might
think it was O(j^{24}), but it's not that bad... however, it's
still very bad!  


Luckily, Dan and Greg have figured out a much more efficient
algorithm, which uses only O(j^{6}) time and O(j) space.
Alternatively, with more caching of data, they can get
O(j^{5}) time and O(j^{3}) space, or maybe even
better.  Using an algorithm of this sort, Dan can compute the 10j
symbol for spins up to 55.  For all spins equal to 55, the calculation
took about 10 hours on a normal desktop computer.  However, for
computing partition functions it appears that small spins are much
more important, and then the computation takes milliseconds.

(Actually, for computing partition functions, Dan is not using a desktop: 
he is using a Beowulf cluster, which is a kind of supercomputer built out 
of lots of PCs.  This works well for partition functions because the 
computation is highly parallelizable.)

John Barrett has also figured out a very different approach to computing 
10j symbols:

3) John W. Barrett, The classical evaluation of relativistic spin networks, 
Adv. Theor. Math. Phys. 2 (1998), 593-600.  Also available as 
<A HREF = "http://xxx.lanl.gov/abs/math.QA/9803063">math.QA/9803063</A>.

In this approach one computes the 10j symbols by doing an integral over the 
space of geometries of a 4-simplex - or more precisely, over a product of 
5 copies of the 3-sphere, where a point on one of these 3-spheres describes 
the normal vector to one of the 5 tetrahedral faces of the 4-simplex.  

Dan and Greg have also written programs that calculate the 10j symbols by 
doing these integrals.  The answers agree with their other approach. 

We've already been getting some new physical insights from these 
calculations.  If you write down the integral formula for the Riemannian 
10j symbols, a stationary phase argument due to John Barrett and Ruth 
Williams suggests that, at least in the limit of large spins, the dominant 
contribution to the integral for the 10j symbol comes from 4-simplices whose 
face areas are the 10 spins in your 10j symbols:

4) John W. Barrett and Ruth M. Williams, The asymptotics of an amplitude 
for the 4-simplex, Adv. Theor. Math. Phys. 3 (1999), 209-215.  Also available 
as <A HREF = "http://xxx.lanl.gov/abs/gr-qc/9809032">gr-qc/9809032</A>.

However, Dan and Greg's calculations suggest instead that the dominant 
contribution comes from certain "degenerate" configurations.  Some of 
these correspond to points on the product of 5 copies of the 3-sphere that 
are close to points of the form (v,v,v,v,v) - or roughly speaking, 
4-simplices whose 5 normal vectors are all pointing the same way.  Others 
come from sprinkling minus signs in this list of vectors.  Heuristically,
we can think of these degenerate configurations as extremely flattened-out 
4-simplices.

For simplicity, we have concentrated so far on studying the 10j
symbols in the case when all 10 spins are equal.  In this case we can
show that the only nondegenerate 4-simplex with these spins as face
areas is the regular 4-simplex (all of whose faces are congruent
equilateral triangles).  Greg used stationary phase to compute the
contribution of this regular 4-simplex to Barrett's integral formula
for the 10j symbols, and it turned out that asymptotically, for large
j, this contribution decays like j^{-9/2}.  On the other hand,
Dan's numerical computations of the 10j symbol suggests that it goes
like j^{-2}.  This suggests that for large j, the contribution
of the regular 4-simplex is dwarfed by that of the degenerate
4-simplices.

Greg has gotten more evidence for this by studying the integral formula for 
the 10j symbols and estimating the contribution due to degenerate 4-simplices.
This estimate indeed goes like j^{-2} for large j.  

There is a lot more to be understood here, but plunging ahead recklessly, we 
can ask what all this means for the physics of the Barrett-Crane model.  For 
example: is the dominant contribution to the partition function going to come 
from spacetime geometries with lots of degenerate 4-simplices?

I think that's a premature conclusion, because we already have evidence that 4
-simplices with large face areas are not contributing that much compared to 
those with small face areas when we compute the partition function as a sum 
over spin foams.  In other words, it seems that in the Riemannian Barrett-Crane
model, spacetime is mostly made of lots of small 4-simplices, rather than a 
few giant ones.   If so, the tendency for the giant ones to flatten out may 
not be so bad. 

Of course the really important thing will be to study these questions for 
the Lorentzian theory, but it's good to look at the Riemannian theory too.

Another talk on a subject close to my heart was given by Noson Yanofsky.   
It was based on these papers of his, especially the last:

5) Noson S. Yanofsky, Obstructions to coherence: natural noncoherent 
associativity, Jour. Pure Appl.  Alg. 147 (2000), 175-213.  Also 
available at 
<A HREF = "http://xxx.lanl.gov/abs/math.QA/9804106">math.QA/9804106</A>.

The syntax of coherence.  To appear in Cahiers Top. Geom. Diff..
Also available at 
<A HREF = "http://xxx.lanl.gov/abs/math.CT/9910006">math.CT/9910006</A>.

Coherence, homotopy and 2-theories.  To appear in K-Theory.
Also available at 
<A HREF = "http://xxx.lanl.gov/abs/math.CT/0007033">math.CT/0007033</A>.

One of the cool things Yanofsky has done is to study what happens when 
we categorify Lawvere's concept of an "algebraic theory".  I've already 
explained this idea of "algebraic theory" in "week53" and "week136", so 
I'll just quickly recap it here:
 
The notion of "algebraic theory" is just a slick way to study sets
equipped with extra algebraic structure.  We call a category C with
finite products an "algebraic theory" if its objects are all of the
form 1, X, X^{2}, X^{3}, ...  for some particular
object X.  We call a product-preserving functor F: C \to  Set a "model"
of the theory.  And we call a natural transformation between such
functors a "homomorphism" between models.  This gives us a category
Mod(C) consisting of models and homomorphisms between them, and it
turns out that many categories of algebraic gadgets are of this form:
the category of monoids, the category of groups, the category of
abelian groups, and so on.


Since algebraic theories are good for studying sets with extra
algebraic structure, we might hope that by categorifying, we could
obtain a concept of "algebraic 2-theories" which is good for studying
\emph{categories} with extra algebraic structure.  And it's true!
In 1974, John Gray defined an "algebraic 2-theory" to be a 2-category
C with finite products, all of whose objects are of the form 1, X,
X^{2}, X^{3}, ...  for some particular object X.
Define a "model" of this 2-theory to be a product-preserving 2-functor
F: C \to  Cat.  And define a "homomorphism" between models to be a
pseudonatural transformation between such 2-functors.

Huh?  "Pseudonatural"??  

Sorry, now things are getting a bit technical: the right thing going between 
2-functors is not a natural transformation but something a bit weaker called 
a "pseudonatural transformation", where the usual commuting squares in the 
definition of a natural transformation are required to commute only up to 
certain specified 2-isomorphisms, which in turn satisfy some coherence laws 
described here:

6) G. Maxwell Kelly and Ross Street, Review of the elements of 2-categories, 
Springer Lecture Notes in Mathematics 420, Berlin, 1974, pp. 75-103.

However, you don't need to understand the details right now.  There is also 
something going between pseudonatural transformations called a "modification",
and this gives us "2-homomorphisms" between homomorphisms between models of 
our algebraic theory.  Thanks to these there is a 2-category Mod(C) consisting 
of models of our 2-theory homomorphisms between those, and 2-homomorphisms 
between those.  

Some examples might help!  For example, there's a 2-theory C called the 
"theory of weak monoidal categories".  Models of C are weak monoidal 
categories, homomorphisms are monoidal functors, and 2-homomorphisms are 
natural transformations, so Mod(C) is the usual 2-category of monoidal 
2-categories.   There's a similar 2-theory C' called "the theory of strict 
monoidal categories", for which Mod(C') is the usual 2-category of strict 
monoidal categories.  

(Hyper-technical note for n-category mavens only: in both examples here, 
monoidal functors are required to preserve unit and tensor product only 
\emph{up to coherent natural isomorphism}.  This nuance is what we get from 
working with pseudonatural rather than natural transformations.  Without 
this nuance, some of the stuff I'm about to say would be false.)

Now, whenever we have a product-preserving 2-functor between 2-theories, 
say F: C \to  C', we get an induced 2-functor going the other way,

F*: Mod(C') \to  Mod(C).

For example, there's a product-preserving 2-functor from the theory of 
weak monoidal categories to the theory of strict monoidal categories, 
and this lets us turn any strict monoidal category into a weak one.  

Now in this particular example, F* is a biequivalence, which is the nice 
way to say that the 2-categories Mod(C) and Mod(C') are "the same" for all 
practical purposes.  And in fact, saying that this particular F* is a 
biequivalence is really just an ultra-slick version of Mac Lane's theorem - 
the theorem we use to turn weak monoidal categories into strict ones.  

Now, Mac Lane's theorem is the primordial example of a "strictification
theorem" - a theorem that lets us turn "weak" algebraic structures on 
categories into "strict" ones, where lots of isomorphisms, like the 
associators in the monoidal category example, are assumed to be equations.  
This suggests that lots of coherence theorems can be stated by saying that 
2-functors of the form F* are biequivalences.  

So: is there a super-general strictification theorem where we can start
from any 2-theory C and get a "strictified" version C' together with an
F: C \to  C' such that F* is a biequivalence?

As a step in this direction, Yanofsky has cooked up a model category of 
algebraic 2-theories, in which F: C \to  C' is a weak equivalence precisely
when F* is a biequivalence.  

Huh?  "Model category"?? 

Well, if you don't know what a "model category" is, you're in serious 
trouble now!  They're a concept invented by Quillen for generalizing
the heck out of homotopy theory.  Try reading his book:

7) Daniel G. Quillen, Homotopical Algebra, Springer Lecture Notes in 
Mathematics, vol. 43, Springer, Berlin, 1967.

or for something newer:

8) Mark Hovey, Model Categories, American Mathematical Society Mathematical 
Surveys and Monographs, vol 63., Providence, Rhode Island, 1999.

or else:

9) Paul G. Goerss and John F. Jardine, Simplicial Homotopy Theory, 
Birkhauser, Boston, 1999.

(By the way, Jardine was one of the organizers of this Stanford conference, 
along with Gunnar Carlsson.  He told me he had created a hypertext version 
of this book, but has not been able to get the publisher interested in it.  
Sad!)

Anyway, in the framework of model categories, the problem of "strictifying" 
an algebraic structure on categories then amounts to finding a "minimal 
model" of a given 2-theory C - roughly speaking, a weakly equivalent 
2-theory with as little flab as possible.  The concept of "minimal model" 
is important in homotopy theory, but apparently Yanofsky is the first to 
have given a general definition of this concept applicable to any model 
category.  Yanofsky has not shown that every algebraic 2-theory admits a 
minimal model, but this seems like a fun and interesting question.



 \par\noindent\rule{\textwidth}{0.4pt}
<em>all ignorance toboggans into know<br>
and trudges up to ignorance again. - </em> e.e.cummings, 1959
\par\noindent\rule{\textwidth}{0.4pt}

% </A>
% </A>
% </A>
