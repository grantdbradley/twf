
% </A>
% </A>
% </A>
\week{June 5, 2005 }



Classes are over!  Summer is here!  Now I can finally get some work done!
I'll be travelling to Sydney, Canberra, Beijing, Chengdu and Calgary, but 
mainly I want to finish writing some papers.

First, though, I need to recover from a hard quarter.  I need to goof off a 
bit!  I spent most of yesterday lying in bed reading.  Now I want to talk 
some more about number theory.

Let's see, where were we?   I had just begun to introduce the theme of 
L-functions and their corresponding automorphic forms.   My ultimate goal
is to understand the Langlands Conjectures well enough to give a decent 
explanation of what they say.  Instead of simply stating them, I'd like
to really make them plausible, and this will take quite an elaborate warmup.
So, this Week I want to talk about some background.


Actually, this reminds me of something Feynman wrote: whenever he
worked on a problem, he needed the feeling he had some "inside
track" - some insight or trick up his sleeve that nobody else
had.  Most of us will never be as good as Feynman at choosing an
"inside track".  But I think we all need one to convert what
would otherwise be a dutiful and doomed struggle to catch up with the
experts into something more hopeful and exciting: a personal quest!

For anyone with a physics background, a good "inside track" on 
almost any math problem is to convert it into some kind of crazy physics 
problem.  It doesn't 
need to be realistic physics, just anything you can apply physics intuition 
to!  This is part of why string theorists have been so successful in cracking 
math problems.  It also underlies Alain Connes' attempt to prove the Riemann
Hypothesis:

1) Alain Connes, Noncommutative Geometry, Trace Formulas, and the
Zeros of the Riemann Zeta Function.  Ohio State course notes and
videos at
<A HREF = "http://www.math.ohio-state.edu/lectures/connes/Connes.html">http://www.math.ohio-state.edu/lectures/connes/Connes.html</A>

Alain Connes, Trace formula in noncommutative geometry and the zeros of
the Riemann zeta function, available as 
<A HREF = "http://www.arxiv.org/abs/math.NT/9811068">math.NT/9811068</A>.

2) Mathilde Marcolli, Noncommutative geometry and number theory,
available at 
<A HREF = "http://www.math.fsu.edu/~marcolli/ncgntE.pdf">http://www.math.fsu.edu/~marcolli/ncgntE.pdf</A>

Of course, Connes also has another "inside track", namely
his theory of noncommutative geometry.

By the way: a number theorist I know says he thinks Connes has
essentially proved the Riemann Hypothesis, in the same way that
Riemann "essentially" proved the Prime Number Theorem.
Namely, he has reduced it to some facts that seem obviously true!  Of
course, it took about 40 years, from 1859 to 1896, for Riemann's plan
to be fulfilled by Hadamard and De La Vallee Poussin.  So, even if
Connes' insights are correct, it may be a while before the Riemann
Hypothesis is actually proved.

For anyone with a background in geometry, a good "inside track" 
on almost any math problem is to convert it into a geometry problem.  
In the case of number theory this trick is old news, but still very much 
worth knowing.  It's based on an analogy which I began discussing in 
"<A HREF = "week198.html">week198</A>".  

The analogy starts out like this:


\begin{verbatim}


   NUMBER THEORY           COMPLEX GEOMETRY                               

   Integers, Z             Polynomial functions on the complex plane, C[z]
   Rational numbers, Q     Rational functions on the complex plane, C(z)
   Prime numbers, P        Points in the complex plane, C           

\end{verbatim}
    

Why is this analogy good?   Well, for starters:

\begin{quote}
   Every rational number is a ratio of integers.<br>

   Every rational function is a ratio of polynomials.  
\end{quote}
Better yet:

\begin{quote}
   Every integer can be uniquely factored into primes <br>
   (modulo invertible integers, namely +1 and -1). <br>

   Every complex polynomial can be uniquely factored into linear polynomials <br>
   (modulo invertible polynomials, namely nonzero constants).
\end{quote}
There's one linear polynomial z-a for each point a in the complex plane, 
so \emph{primes} are like \emph{points} in the complex plane. 


We can make this precise using the concept of "spectrum",
which I defined in "<A HREF = "week199.html">week199</A>".
Ignoring a certain little sublety which is discussed there:

\begin{quote}
   The spectrum of Z is the set of prime numbers.<br>
   The spectrum of C[z] is the complex plane.
\end{quote}

This way of thinking lets us treat the spectrum of any algebraic
extension of the integers, like the Gaussian integers, as a
"covering space" of the set of prime numbers.  I've already
drawn this picture:



\begin{verbatim}

                    2+i                  3+2i
  --- 1+i --- 3 ---     --- 7 --- 11 ---      ---   GAUSSIAN INTEGERS 
                    2-i                  3-2i           

  -----2------3------5------7-----11------13-----      INTEGERS

\end{verbatim}
    

But, now I'm saying that the "line" down below really acts
like the complex \emph{plane}.  Taking this strange idea seriously leads to
all sorts of amazing insights.

For example, if you poke a hole in this "plane" at some
prime, there's something like a little \emph{loop} that goes around this
hole!  In other words, there's a sense in which the spectrum of Z has
a nontrivial "fundamental group", which contains an element
for each prime.  Technically this group is called the Galois group
Gal(Q^{-}/Q), and we get an element in it for each prime, called the
"Frobenius automorphism" for that prime.

Another cool thing is that we can study integers "locally", one prime at a 
time, just like we study complex functions locally.  We can analyze functions
at a point using Taylor series and Laurent series.   And, we can stretch our 
analogy to include these concepts:


$$

   NUMBER THEORY              COMPLEX GEOMETRY     

   Integers, Z                Polynomial functions on the complex plane, C[z]
   Rational numbers, Q        Rational functions on the complex plane, C(z)
   Prime numbers, P           Points a in the complex plane, C    
   Integers mod p^{n}, Z/p^{n}      (n-1)st-order Taylor series, C[z]/(z-a)^{n}
   p-adic integers, Z_{p}        Taylor series, C[[z-a]]
   p-adic numbers, Q_{p}         Laurent series, C((z-a))
$$
    

All the weird symbols are just the standard notations for these gadgets.
The analogy goes as follows:

\begin{quote}
   To study a polynomial "at a point" a in the complex plane, <br>
   we can look at its value modulo (z-a),  
   or more generally mod (z-a)^{n}. <br>
   
   To study an integer "at a prime" p, <br>
   we can look at its value modulo p, 
  or more generally mod p^{n}.
\end{quote}

This is nice because the value of a polynomial modulo
(z-a)^{n} is just its Taylor series at the point a, where we
keep terms up to order n-1.

We can also also take the limit as n \to  \infty .  If we do this to
the integers mod p^{n} we get a ring called the "p-adic
integers".  For example, a typical 3-adic integer, written in
base 3, looks like this:


\begin{verbatim}

        ......21001102020110102012102201
\end{verbatim}
    
They're just like natural numbers in base 3, except they go on forever to the
left!  We add and multiply them in the obvious way, for example:


\begin{verbatim}

        ......21001102020110102012102201
      + ......10201101012201201122010012
     -----------------------------------
        ......01202210110012010211112220
\end{verbatim}
    
If we take the same sort of limit for Taylor series, we get Taylor series 
that go on forever - in other words, formal power series.  

We can also ratios of p-adic integers, which are called p-adic numbers,
and ratios of Taylor series, which are called Laurent series.   A typical
3-adic number, written in base 3, looks like this:


\begin{verbatim}

       .......121010010012121201201201011.21021
\end{verbatim}
    
They have to stop at some finite stage at the right, just as Laurent
series have to stop at some finite stage: they can't have arbitrarily
large negative powers of z-a.  
                     
Laurent series can be used to describe functions that have a pole at some
point, like rational functions.  Similarly, p-adic numbers can be used 
to describe rational numbers.  Using more math jargon:

\begin{quote}
   For any point a in C, there's a homomorphism  <br>
   from the field of rational functions <br>
   to the field of Laurent series,<br>
   which sends polynomials to Taylor series.<br>

   For any prime p, there's a homomorphism <br>
   from the field of rational numbers<br>
   to the field of p-adic numbers,<br>
   which sends integers to p-adic integers.
\end{quote}

This lets us study rational numbers "locally" at the prime p
using p-adic numbers, just as we can study a rational function locally
at a point using its Laurent series.  This technique can be quite
useful.  For example, a polynomial equation can have rational
solutions only if it has p-adic solution for all primes p.

We might hope for the converse, but then we would be ignoring a funny
extra "prime" besides the usual ones... something called the
"real prime"!

The point is, besides being able to embed the rational numbers in the
p-adics for any prime p, we can also embed them in the real numbers!
This embedding is a bit different than the rest: it's based on a weird
thing called an "Archimedean valuation", while the usual
primes correspond to non-Archimedean valuations.

I'm sort of joking here, since if you're more used to real numbers
than p-adics, you'll probably find Archimedean valuations to be \emph{less}
weird than non-Archimedean ones.  The Archimedean valuation on the
rational numbers is just the usual absolute value, while the
non-Archimedean ones are other concepts of "absolute value",
one for each prime p.  If we take limits of rational numbers that
converge using the usual distance function |x-y|, we get real numbers;
if we take limits that converge using one of the non-Archimedean
versions of this distance function, we get p-adic numbers.  But from
the viewpoint of number theory, it's the Archimedean valuation that's
the odd man out!  It indeed does act very weird and different than all
the rest.  That's why someone wrote this book:

3) M. J. Shai Haran, The Mysteries of the Real Prime, Oxford
U. Press, Oxford, 2001. 

... which you will see is deeply connected to mathematical physics.

If we take this weird "real prime" into account, things work better.  
We sometimes get results saying that some kind of polynomial equations 
have a rational solution if they have p-adic solutions for all primes p 
and also a real solution.  For example, Hasse proved this was true for 
systems of quadratic equations in many variables.

Results like this are called "local-to-global" results,
since they're analogous to constructing a function from local
information, like its Laurent series at all different points.

In 1950, in his famous PhD thesis, John Tate came up with a clever way to 
formalize this "Laurent series at all different points" idea in the context 
of number theory.  To do this, he formed a ring called the "adeles".  

Indeed, this is what my whole discussion so far has been leading up
to!  Adeles are a really nice formalism, and you pretty much need to
understand them to follow what people are doing in work on the
Langlands Conjectures, or even simpler things, like class field
theory.  But, adeles seem like an arbitrary construction until you see
them as an inevitable outgrowth of our desire to study integers
"locally" at all different primes, including the real prime.

The definition is simple.   An adele consists of a p-adic number for each 
prime p, together with a real number... but where all but finitely many 
of the p-adic numbers are p-adic integers!

This is the number-theoretic analogue of a Laurent series for each point in 
the complex plane, including the point at infinity... but with poles at only 
finitely many points!  We could call such a thing an "adele for the rational
functions".  

Any rational function gives such a thing, just as any rational number gives 
an adele.  And, we don't lose any information this way:

\begin{quote}
   There's a one-to-one (but not onto) homomorphism <br>
   from the rational functions to the adeles for the rational functions.

   There's a one-to-one (but not onto) homomorphism <br>
   from the rational numbers to the adeles for the rational numbers.
\end{quote}

So, our table now looks like this.  For good measure, I'll combine it with
the related table in "<A HREF = "week205.html">week205</A>":


$$

   NUMBER THEORY                 COMPLEX GEOMETRY    

   Integers                      Polynomial functions on the complex plane
   Rational numbers              Rational functions on the complex plane
   Prime numbers                 Points in the complex plane     
   Integers mod p^{n}               (n-1)st-order Taylor series
   p-adic integers               Taylor series
   p-adic numbers                Laurent series
   Adeles for the rationals      Adeles for the rational functions
   Fields                        One-point spaces
   Homomorphisms to fields       Maps from one-point spaces
   Algebraic number fields       Branched covering spaces of the complex plane
$$
    
There's a \emph{lot} 
more to say about this analogy, but I think this is enough 
for now.  Again, one of my secret goals was to start getting you comfy with 
adeles and the idea of studying number theory "locally".  

For more on the geometrical side of number theory, I again recommend these:

4) Juergen Neukirch, Algebraic Number Theory, trans. Norbert Schappacher, 
Springer, Berlin, 1986.

5) Dino Lorenzini, An Invitation to Arithmetic Geometry, American 
Mathematical Society, Providence, Rhode Island, 1996.

But now, back to the subject of "inside tracks" - sneaky
ways to get the beneficial feeling that you have secret insights into
some problem.

For anyone with a background in categories, a good "inside track" on 
almost any math problem is to categorify it: to see that people are using 
sets where they could, and therefore \emph{should}, be using categories or 
n-categories.

I've already hinted that zeta functions are an example of 
"decategorification".  Now I'd like to make this more precise.  

Let's think about the zeta function of a set X equipped with a one-to-one 
and onto function

f: X \to  X

If you're a physicist, you might call this a "discrete dynamical
system", with f describing one step of "time
evolution".  If you're a mathematician, you might call this a
"Z-set".  After all, for any group G, a "G-set" is
a set equipped with an action of G.  If G = Z (the additive group of
integers), this amounts to a one-to-one and onto function from some
set to itself.

No matter what you call them, these are fundamental things.  So, let's 
look at the \emph{category} of Z-sets!  Here the objects are Z-sets and the 
morphisms are functions that commute with time evolution.  

As explained near the end of "<A HREF =
"week216.html">week216</A>", we can define a kind of zeta
function for a Z-set as follows:

Z(x) = exp(\sum_{n>0} |fix(f^{ n})| x^{n} / n)

where |fix(f^{ n})| is the number of fixed points of 
f^{ n}.  Of course, this
only makes sense if all these numbers are finite; henceforth I'll assume 
my Z-sets are "finite" in this special sense.  

It turns out that you know a finite Z-set up to isomorphism if you know its 
zeta function.  So, a zeta function is just a sneaky way of talking about 
an \emph{isomorphism class} of finite Z-sets.  


This is a fancy example of something we all learn as kids: counting!
When we "count" a finite set, assigning a natural number to
it, we are really determining its isomorphism class.  Two finite sets
are isomorphic if and only if they have the same number of elements.
Operations on finite sets, like disjoint union and Cartesian product,
are what give rise to operations on natural numbers, like addition and
multiplication.

Summarizing this, we have the following motto, suitable for making into
a bumper sticker:
<div align = "center">
  THE SET OF NATURAL NUMBERS IS THE DECATEGORIFICATION OF<br>
                THE CATEGORY OF FINITE SETS
</div>
Similarly, this is what we're seeing now:
<div align = "center">
  THE SET OF ZETA FUNCTIONS IS THE DECATEGORIFICATION OF  <br>
               THE CATEGORY OF FINITE Z-SETS
</div>
Beware: here I'm only talking about zeta functions of the above form.  There 
are lots of other things people call zeta functions.   So, don't read too 
much into this statement.  But don't read too little into it, either!   With 
an extra twist we can get most of the zeta functions showing up in number 
theory.  In number theory, we typically get a Z-set for each prime p, 
coming from the "Frobenius" for that prime.  We thus get a bunch of "local" 
zeta functions Z_{p}(x), one for each prime.  We then multiply these to get 
one big fat "global" zeta function:

\zeta (s) = &prod;_{p} Z(p^{-s})

Each local zeta function is a formal power series, while this global
zeta function is a Dirichlet series.  As I mentioned in "<A HREF
= "week217.html">week217</A>", formal power series live in the
monoid algebra of (N,+,0), while Dirichlet series live in the monoid
algebra of (N,\times ,1).  (N,+,0) is the free commutative monoid on one
generator, while (N,\times ,1) is the free commutative monoid on countably
many generators - the primes!  Everything fits together sweetly.

So, it's a good first step to think about the zeta function of a single 
Z-set.

Now, there's another motto along the lines of the above two, which I've
talked about before:
<div align = "center">
   THE SET OF GENERATING FUNCTIONS IS A DECATEGORIFICATION OF <br>
                 THE CATEGORY OF FINITE STRUCTURE TYPES
</div>
I explained this in "<A HREF = "week185.html">week185</A>", "<A HREF = "week190.html">week190</A>", and "<A HREF = "week202.html">week202</A>".  I've even taught a
whole course on structure types (also known as "species") and the 
combinatorics of Feynman diagrams.  The course notes by Derek Wise are 
available online:

6) John Baez and Derek Wise, Quantization and Categorification, available at:<br>
<A HREF = "http://math.ucr.edu/home/baez/qg-fall2003/">http://math.ucr.edu/home/baez/qg-fall2003/</A> <br>
<A HREF = "http://math.ucr.edu/home/baez/qg-winter2004/">http://math.ucr.edu/home/baez/qg-winter2004/</A> <br>
<A HREF = "http://math.ucr.edu/home/baez/qg-spring2004/">http://math.ucr.edu/home/baez/qg-spring2004/</A> 

So, I think this third example of decategorification is great.  But, I'm not 
going to explain it in much detail here - just enough to say how it's related
to zeta functions!

A stucture type F is a gadget that gives a set F_{n}
for each n = 0,1,2,....
We think of the elements of F_{n} 
as "structures of type F" on an n-element 
set - for example, orderings, or cyclic orderings, or n-colorings, or 
whatever type of structure you like.  We only require that permutations of 
the n-element set act on this set of structures.  

Let's say a structure type is "finite" if all the
sets F_{n} are finite.
Any finite structure type has a "generating function", 
which is a formal power series |F| given by


$$

              |F_{n}|
|F|(x) =  \sum  ------  x^{n}
               n! 
$$
    

Isomorphic structure types have the same generating function.
However, structure types with the same generating function can fail to
be isomorphic.  This is why I said generating functions are
"a" decategorification of finite structure types, instead of
"the" decategorification.

Despite this defect, generating functions are still very useful in 
combinatorics.  So, when we see a zeta function like

Z(x) = exp(\sum_{n>0} |fix(f^{ n})| x^{n} / n)

as a trick for decategorifying Z-sets, we should instantly wonder if it's 
a generating function in disguise.  And of course, it is!

Actually it's easiest to leave out the exponential at first.  This power
series:

\sum_{n>0} |fix(f^{ n})| x^{n} / n

is the generating function for the structure type "being cyclically 
ordered and equipped with a morphism to the Z-set X".   

Huh?

We "cyclically order" a finite set by drawing it as a little circle of dots 
with arrows pointing clockwise from each dot to the next.  A cyclically 
ordered set is automatically a Z-set in an obvious way.   So, here's a type 
of structure you can put on a finite set: cyclically ordering it and 
equipping the resulting Z-set with a morphism to the Z-set X.  

And, if you work out the generating function of this structure type, you get 

\sum_{n>0} |fix(f^{ n})| x^{n} / n

Check it and see!  

What about the exponential?  Luckily, there's a standard way to take the 
exponential of a structure type: to put an exp(F)-structure on a finite set 
S, we chop S into disjoint parts and put an F-structure on each part.  So, 
the zeta function

Z(x) = exp(\sum_{n>0} |fix(f^{ n})| x^{n} / n)

is the generating function for "being chopped up into cyclically ordered 
parts, each equipped with a morphism to the Z-set X". 

But this is just a long way of saying: "being made into a Z-set
and equipped with a morphism to the Z-set X".

Or, in category theory jargon, "being a Z-set over X".

So: 
<div align = center>
     THE ZETA FUNCTION OF THE Z-SET X IS THE GENERATING FUNCTION OF <br>
                      "BEING A Z-SET OVER X"
</div>

By the way, this is the kind of thing you could do with \emph{any}
structure type F.  Given an F-structured set X, we get a new structure
type "being an F-structured set equipped with a morphism to
X".  Or, in category theory jargon, "being an F-structured
set over X".  The generating function of this could be called the
"zeta function" of our F-structured set X.  I have no idea
how important this is...

... but I want to keep gnawing away on the connection between zeta functions
and the generating functions of combinatorics, because to understand number 
theory, I need all the "inside tracks" I can get!
\par\noindent\rule{\textwidth}{0.4pt}
<B>Addendum</B>: 
After reading this Week's Finds, Kevin Buzzard emailed me the
following remarks.  He begins by talking about adeles for any
algebraic number field K - a fairly straightforward generalization of
the case I discussed above, where K is the rational numbers:

\begin{quote}
 The adeles were used in a really powerful way in the theorems and proofs
 of global class field theory (you don't want to read the proofs. I did
 this precisely once in my life and they are very unilluminating).  But the
 theorem---if K is a number field then the abelianisation of 
  Gal(K^{-}/K)
  is canonically isomorphic to 
K*\Adeles_{K}*/(K_{\infty }*)^{0}
 (the last term being the connected component of the product of the
 infinite completions of K)---is incredibly important.

  Much easier going is Tate's thesis (in the book by Cassels and Froehlich).
 Tate observes that Fourier analysis works on any locally compact abelian 
 group (Haar measure is the replacement for "usual" measure), and then gives 
 a very short proof of the analytic continuation and functional equation of
 all Hecke's L-functions by simply pushing through an analogue of the proof
 you know of the functional equation of the zeta function in this much more
 general context.  I think this is an amazingly powerful use of the adeles.
 Tate's approach explains the fudge factors, the factors at infinity,
 everything.

 A word on analogies.  If you want to say that the p-adic integers are 
 analogous to the formal power series ring C[[z-a]] (call it C[[z]] 
 for simplicity) then in fact some people would say that this was \emph{not} 
 an analogy---this was simply two instances of the same thing, namely a 
 complete discrete valuation ring. Similarly, you might say that Z is 
 analogous to C[z], but again some people would just tell you to go and 
 get a book on commutative algebra and look up the word "Dedekind 
 domain"---both of these are examples. A geometer might even go and 
 tell you to go and find out what a regular 1-dimensional scheme was!

 One thing I didn't realise when I was learning all this stuff however, was
 that there is some stuff that just goes through for all Dedekind domains
 (e.g. construction of adeles, existence of class group etc), and there is
 some stuff that actually requires more. Tate's thesis for example requires
 more---it doesn't just work for all Dedekind domains because Tate needs a
 Haar measure and so he needs completions to be locally compact,
 which is basically the same as demanding that <em>all residue fields are
 finite</em>.  Here's something you can do for Z_{p} 
 which you can't do for
 C[[z]]: let's define the measure of a+p^{n}Z_{p} 
 to be p^{-n}. Then this is
 finitely additive, because Z_{p} 
 is the disjoint union pZ_{p} \cup  1+pZ_{p} \cup 
 ... \cup  (p-1)+pZ_{p}, and 
 1/p+1/p+...+1/p (p times) is 1.
 But you can't do this for C[[z]] because the cardinality of 
 C is infinite.  This naive
 measure on Z_{p} is exactly what you need to define p-adic 
 L-functions, 
 by the way! But they are another (related) story.

 When you move from Discrete Valuation Rings to Dedekind Domains the same
 care needs to be applied: it's a famous theorem that the ideal class group 
 of (the integers of) a number field is finite.  But it's not true that the
 class group of a Dedekind domain is finite: the class group of C[z] is
 finite as C[z] is a principal ideal domain, but the class group of
 C[x,y]/(y^{2}-x^{3}-1) is infinite (the class group is essentially the
 underlying elliptic curve, which is an infinite group).  Again you have to
 demand that residue fields are finite.  So this stops you thinking about
 C[z] and its finite extensions, it forces you to start thinking about k[z]
 where k is a \emph{finite} field.  
  Of course algebraic geometers aren't scared
 of finite fields (well, at least, the ones I talk to the most aren't), so
 after a while your analogy is going to break because C is infinite.
 Langlands' philosophy is (or at least, was---it has been generalised in
 various directions now) about global fields, which means either number
 fields or finite extensions of k(z) where k is a finite field.  Of course
 Lafforgue recently proved everything in the function field case, hence the
 Fields Medal.
 
 Kevin
\end{quote}

I replied:

\begin{quote}


\begin{verbatim}

 > The adeles were used in a really powerful way in the theorems and 
 > proofs of global class field theory (you don't want to read the 
 > proofs. I did this precisely once in my life and they are very 
 > unilluminating).
\end{verbatim}
    

 Then I think there must exist nicer proofs!  There can't possibly 
 be such important and beautiful results where the best possible proof
 is unilluminating.  So, someone needs to work on this more... perhaps
 me, if everyone else is too busy.  :-) 


$$

 > But the theorem---if K is a number field then the abelianisation 
 > of Gal(K^{-}/K) is canonically isomorphic to 
 > K*\Adeles_{K}*/(K_{\infty }*)^{0}
 > (the last term being the connected component of the product of 
 > the infinite completions of K)---is incredibly important.
$$
    

 It's beautiful, too!


$$

 > Much easier going is Tate's thesis (in the book by Cassels and 
 > Froehlich).  Tate observes that Fourier analysis works on any 
 > locally compact group (Haar measure is the replacement for 
 > "usual" measure), and then gives a very short proof 
 > of the analytic continuation and functional equation of all 
 > Hecke's L-functions by simply pushing through an analogue of 
 > the proof you know of the functional equation of the zeta 
 > function in this much more general context.  I think this is 
 > an amazingly powerful use of the adeles.  Tate's approach 
 > explains the fudge factors, the factors at infinity, everything.
$$
    

 This sounds great.  I've always heard people rave about Tate's thesis,
 and now it's time for me to read it... or at least the book you mention -
 but I get the feeling the actual thesis is good.


\begin{verbatim}

 > A word on analogies.If you want to say that the p-adic integers 
 > are analogous to the formal power series ring C[[z-a]] (call it 
 > C[[z]] for simplicity) then in fact some people would say that 
 > this was <em>not</em> an analogy---this was simply two instances 
 > of the same thing, namely a complete discrete valuation ring.
\end{verbatim}
    

 Yes, but I don't want to intimidate my readers with concepts
 like "complete discrete valuation ring" - I'd rather lure them 
 in with the charm of a mysterious analogy!  I think think this 
 is how things went historically, too... judging from Weil's
 remarks: 
7) Martin H. Krieger, A 1940 letter of Andre Weil on analogy in 
 mathematics, AMS Notices 52 (March 2005), 334-341. Available 
 at <A HREF = "http://www.ams.org/notices/200503/200503-toc.html"> http://www.ams.org/notices/200503/200503-toc.html</A>

 He even talks about how the charm of an analogy evaporates when
 you find a generalization that encompasses both terms:

\begin{quote}
  "The 
  day dawns when the illusion vanishes; intuition turns to certitude; 
  the twin theories reveal their common source before disappearing; as 
  the Gita teaches us, knowledge and indifference are attained at the 
  same moment. Metaphysics has become mathematics, ready to form the 
  material for a treatise whose icy beauty no longer has the power to 
  move us."
\end{quote}

 Wouldn't want that!


$$

 > Similarly, you might say that Z is analogous to C[z], but again 
 > some people would just tell you to go and get a book on commutative 
 > algebra and look up the word "Dedekind domain"---both of these 
 > are examples. A geometer might even go and tell you to go 
 > and find out what a regular 1-dimensional scheme was!
$$
    

 I've read lots of theorems about Dedekind domains, and on good
 days I can even remember the definition... 

 But, I really want to keep things a bit vague and misty for my 
 readers - most importantly because This Week's Finds is supposed
 to be fun, but also because a lot of the coolest stuff happens 
 when you extend vague analogies in shocking ways.  

 For example, thinking of Spec(Z) as a plane that gets a fundamental 
 group when you poke a hole in it and remove a prime is nice
 for visualizing an individual Frobenius generator, but 
 deeper results suggest that it's good to think of Spec(Z) 
 as 3-dimensional!  This leads to the extensive analogy between 
 Spec(Z) and knot theory discussed here...

8) Adam 
Sikora, Analogies between group actions on 3-manifolds and number fields,
available as <A HREF = "http://www.arXiv.org/abs/math.NT/0107210">math.GT/0107210</A>.

9) Christopher Deninger, 
A note on arithmetic topology and dynamical systems,
available as <A HREF = "http://www.arXiv.org/abs/math.NT/0204274">math.NT/0204274</A>.


\begin{verbatim}

 > (Actually I think there is even a kind of Langlands philosophy 
 > for C(z) and its finite extensions nowadays worked out recently 
 > by Beilinson and Drinfeld.  I saw Beilinson give several lectures 
 > on it, more than once, and still didn't really get it, I am too 
 > number-theoretic.)
\end{verbatim}
    

 Is this the "geometric Langlands program" stuff?  Physicists are
 getting interested in that... 
 Best,<br>
 jb
\end{quote}

To understand Kevin's reply, recall that
any algebraic number field K has a "maximal abelian extension"
K^{ab}.  This is the
biggest algebraic extension of K whose Galois group is
\emph{abelian}.  When K = Q, the Kronecker-Weber theorem says
this is obtained by throwing in all the roots of unity.
Since a field obtained from Q by throwing in a root of unity is called
a "cyclotomic field", people sometimes call this Q^{cyc}.

In "<A HREF = "week201.html">week201</A>" I described
the Galois group Gal(Q^{cyc}/Q).  Unsurprisingly,
this is the abelianization
of the big bad Galois group Gal(Q^{-}/Q): the Galois group
of the algebraic closure of Q.    In what follows,
Kevin more generally discusses Gal(K^{ab}/K), which is
the abelianization of Gal(K^{-}/K).  

Understanding groups like Gal(K^{-}/K)
is one of the great unfulfilled dreams of number theory.  Understanding
its abelianization is one of the great triumphs of late nineteenth
to mid twentieth century mathematics.  This is called <em>class field
theory</em>.

\begin{quote}

\begin{verbatim}

 > > The adeles were used in a really powerful way in the theorems and proofs
 > > of global class field theory (you don't want to read the proofs. I did
 > > this precisely once in my life and they are very unilluminating).

 > Then I think there must exist nicer proofs!
\end{verbatim}
    

 This is related to one of Hilbert's problems! (the 12th one). So you must
 be thinking along the right lines :-)

 Abstract theorem: if K is a number field then the abelianisation of
 Gal(K^{-}/K) is isomorphic to K*\Adeles_{K}*/(K_{\infty }*)^{0}.

 Remark: the right hand group is very "concrete", 
 in the sense that one can
 write down explicit finite quotients of it. (Why quotients? Because quotients
 of Galois groups are again Galois groups.) For example, I can
 just write down a big subgroup e.g. K_{\infty }* times the product of
 O_{K<sub> v}</sub>*, where v runs through all the finite places of K, and the
 quotient of the big group by the big subgroup can be checked to be compact
 and discrete, so it's finite. We have hence "explicitly" written down a
 finite quotient of Gal(K^{-}/K), corresponding to a finite extension H of K.
 The objects on the right hand side are rather abstract, but this is as
 explicit a quotient group of the right hand side as you could possibly ask for---we
 understand exactly what's going on at every place, for example. Hence this
 is as explicit a finite extension of K as you could possibly ask for, if
 you admit the isomorphism of class field theory. Indeed, if you know a bit
 more about the isomorphism, you will know that this quotient H is
 unramified at all the primes of K, and is indeed the largest
 abelian extension of K with this property. H is (by definition) the
 Hilbert Class Field of K. In your analogy, given a curve, a natural thing
 to think of would be the universal covering space of the curve.
 Unfortunately number theorists aren't good enough to understand all of
 Gal(K^{-}/K), they have to abelianise first, so H corresponds to the
 maximal unramified cover of the curve with abelian covering group.
 
 Great! So we have all this machinery, this beautiful isomorphism, this
 completely canonical description of the Galois group, and we make a very
 explicit and natural construction on the right hand side, so now let me
 give you a number field like Q(\sqrt 10) and ask you what H is!

 Now suddenly you see a big disadvantage of the glorious proof of the
 isomorphism which goes via all this cohomological chasing around---it
 shows the existence of H but doesn't tell us what it is. At all. And at
 the end of the day, there are a lot of number theorists out there that are
 actually interested \emph{in numbers}, rather than in abstract results which
 hold for all number fields or whatever.

 Hilbert's question was: "well, this is all well and good, but can anyone
 actually \emph{write down} the isomorphism, rather than actually prove its
 existence? Can anyone write it down sufficiently concretely so that people
 can just read off the Hilbert Class Group of a number field, given the
 field?" And, to be honest, although a lot is known, Hilbert would probably
 say that the answer is still "no". If you were to find a 
 "better" proof of
 global class field theory then perhaps the answer would change. In fact
 the Hilbert Class Field is just the tip of the iceberg---global class field
 theory gives
 us a description of the abelianisation of Gal(K^{-}/K), and this
 abelianisation corresponds to a field K^{ab}, 
 of infinite degree over K,
 but Galois, with infinite abelian Galois group. If I give you K, can
 you tell me K^{ab}? Hilbert even wanted to know this (his questions
 are maximally greedy, I guess).

 Hilbert's question was not totally out of the blue. It can be done for
 K=Q, indeed it had been done 50 years before Hilbert's question. Kronecker
 and Weber knew not just the Hilbert class field of Q, they even knew
 Q^{ab}, it's just the union of Q(1^{1/n}), 
 where 1^{1/n} is exp(2\pi i/n).
 Let me labour a point which experts in the theory feel is highly
 important: the exponential function is transcendental---it doesn't
 belong in algebraic geometry because it's not in C(z). On the other
 hand, this transcendental function, when evaluated at certain places,
 gives algebraic numbers out, and it is these algebraic numbers which
 explain all the class field theory of the rationals.
 
 Now a \textbf{TOTALLY AMAZING GENERALISATION}: let K be an imaginary quadratic
 field, so Q(\sqrt d) for some integer d<0.  Let L be the lattice in the
 complex numbers with basis 1 and \sqrt d (this is a lattice as d<0).
 Quotient out the complex numbers by this lattice.  You get a 1-dimensional 
 complex torus, so an elliptic curve. The curve has a j-invariant, which is 
 going to be a "random" complex number.  One can compute this number to as 
 many decimal places as one likes nowadays (in practice).  For example, if 
 d=-5 then my computer instantly tells me that the j-invariant of the 
 corresponding elliptic curve is

 1264538.90947514050932022704741070342148144212156690839688175141278172815944442224994634954784218993...
 Equally quickly, my computer spots that this (to 100 decimal places,
 at least) looks awfully like one of the roots of
 
 x^{2} - 1264000x - 681472000
 
 (it agrees with it to 100 decimal places, despite the fact that the j-function
 is again "transcendental"---we have put an algebraic number in and appear
 to have got an algebraic number out).

 The awesome fact is that the splitting field of this polynomial
 over Q(\sqrt -5) (i.e. the field you get by adjoining all of the roots
 of this polynomial to Q(\sqrt -5) ) is the Hilbert Class Field of 
 Q(\sqrt 5)!
 \emph{Even better}: 
 I can even tell you K^{ab}, if K is Q(\sqrt -5): write
 down a model for the elliptic curve in the form y^{2}=f(x) 
 with f a cubic
 with coefficients in K (use the Weierstrass P-function and its
 derivative), and now look at all the points of finite order
 on this elliptic curve. The x and y coordinates of all these points
 are algebraic numbers, and they generate K^{ab}.
 
 I am proud now to give you a genuine analogy :-)
 

$$

 RATIONAL FIELD                IMAGINARY QUADRATIC FIELD

 Group C/Z                     Elliptic Curve C/(integers of the field)

 Element of finite order       Element of finite order 
 in the group                  in the group

 function z \to  exp(2\pi iz)        Weierstrass &weierp;-function 
                               (and its derivative)
$$
    

 In both cases, the function maps the group to an algebraic variety
 (C* in the first case, y^{2}=cubic in the second), and evaluating
 the function at complex numbers which give torsion points of the group
 (rational numbers in the first case, elements of the imaginary quadratic
 field in the
 second) gives numbers which by all rights should be random complex numbers,
 but turn out to be not only algebraic, but to generate the maximal abelian
 extension of the number field.


 This really is an analogy because no-one has (as far as I know) a clue how
 to do this more generally. Note that the rationals and the imaginary quadratic
 fields are the only fields with exactly one infinite place. Is this why
 they are the only fields we can "do"? This technique, of 
 "explicitly"
 generating abelian extensions of a number field, is called "explicit 
 class field theory" and, other than the (non-trivial) contribution by 
 Shimura and Taniyama where they used higher-dimensional abelian varieties 
 to push the analogy slightly further, it's still a big mystery.


\begin{verbatim}

 > There can't possibly be such important and beautiful results where 
 > the best possible proof is unilluminating.
\end{verbatim}
    

 In the case of local class field theory, there are now some really
 neat proofs, where you in some sense really do write down the maximal
 abelian extension of an arbitrary finite extension of Q_{p}, 
 again using torsion
 points in groups (formal groups). But people have spent a century looking
 for more illuminating proofs, motivated by Hilbert's question.
 Until then, we just have to rely on known algorithms for computing
 Hilbert Class Fields (there are algorithms that work in lots of cases,
 and they rely on known abstract theorems, but one might argue that
 none of them are really "explicit", they just go, I think, by essentially
 looking at lots of fields until one finds the one that works, rather than
 working out which one is the right one by pure thought).

 You should talk about special values of L-functions! Do you know the
 analytic class number formula? The degree of H over K is called the
 class number of K and, totally amazingly, it is related to the special
 value of an L-function. The Birch-Swinnerton-Dyer conjecture is just
 a natural generalisation of this formula to elliptic curves over the
 rationals, but again, what used to be an analogy has turned into two
 instances of a more general piece of mathematics (ranks of K-groups,
 Beilinson's conjectures etc.).

 Sorry to go on! I just get quite enthusiastic about this stuff.


\begin{verbatim}

 > This sounds great.  I've always heard people rave about Tate's 
 > thesis, and now it's time for me to read it... or at least the 
 > book you mention - but I get the feeling the actual thesis is good.
\end{verbatim}
    

 The thesis was never published, Tate I guess wasn't happy that he just
 reproved a known theorem or something? Cassels-Froehlich is the canonical
 reference. Serre's article in there talks about the links to elliptic
 curves in the im quad case too. A nice book! 

 Thanks for the link to the letter of Weil---interesting stuff! I have
 pity on his sister. I have heard that (Andre) Weil's house in Paris had a
 plaque on it saying "Simone Weil used to live here" (because she did).
 Funny that a genius had to live in the shadows of his sister (who by
 all accounts might also have been a genius).

 Another funny piece of gossip. I think it was Chevalley who originally
 started thinking about ideles (the ideles are the group of units
 of the adeles). I am no historian so might have this wrong. Chevalley(?)
 wrote a book on algebraic number theory where he talked about ideals
 and also about these ideles, which he referred to as "ideal elements"
 and which he abbreviated as "id.eles". Later on the period was dropped,
 so they became ideles. I think it was Serre who saw that the ideles
 were the units of a ring, and christened the ring with the name of "adeles".
 If you get Serre's collected works and look at his CV at the beginning,
 you will see that his mother was called Adele. Coincidence? :-)


\begin{verbatim}

 > > (Actually I think there is even a kind of Langlands 
 > > philosophy for C(z) and its finite extensions nowadays 
 > > worked out recently by Beilinson and Drinfeld.  I saw 
 > > Beilinson give several lectures on it, more than once, and
 > > still didn't really get it, I am too number-theoretic.)

 > Is this the "geometric Langlands program" stuff?  Physicists are
 > getting interested in that...
\end{verbatim}
    
 Yes.
 Kevin
\end{quote}


\par\noindent\rule{\textwidth}{0.4pt}
<em>
 The scientific life of mathematicians can be pictured as a trip
 inside the geography of the "mathematical reality" which
 they unveil gradually in their own private mental frame.

  It often begins by an act of rebellion with respect to the existing
 dogmatic description of that reality that one will find in existing
 books. The young "to be mathematician" realize in their own
 mind that their perception of the mathematical world captures some
 features which do not fit with the existing dogma.  This first act is
 often due in most cases to ignorance but it allows one to free
 oneself from the reverence to authority by relying on one's intuition
 provided it is backed by actual proofs. Once mathematicians get to
 really know, in an original and "personal" manner, a small
 part of the mathematical world, as esoteric as it can look at first,
 their trip can really start. It is of course vital not to break the
 "fil d'arianne" which allows one to constantly keep a fresh
 eye on whatever one will encounter along the way, and also to go back
 to the source if one feels lost at times...

 It is also vital to always keep moving. The risk otherwise is to confine 
 oneself in a relatively small area of extreme technical specialization, thus 
 shrinking one's perception of the mathematical world and its bewildering 
 diversity.

 The really fundamental point in that respect is that while so many
 mathematicians have been spending their entire life exploring that
 world they all agree on its contours and on its connexity: whatever
 the origin of one's itinerary, one day or another if one walks long
 enough, one is bound to reach a well known town i.e. for instance to
 meet elliptic functions, modular forms, zeta functions. "All
 roads lead to Rome" and the mathematical world is
 "connected".

 In other words there is just "one" mathematical world, whose exploration is 
 the task of all mathematicians, and they are all in the same boat somehow. 
</em>  - Alain Connes


\par\noindent\rule{\textwidth}{0.4pt}

% </A>
% </A>
% </A>
