
% </A>
% </A>
% </A>
\week{June 14, 2009 }


Long time no see!  I've been really busy, but now classes are over,
and like last summer, I'm visiting Paul-Andr&eacute; Melli&egrave;s,
who works on logic, computer science and categories.

I should be finishing up some more papers, but let me take a little 
break, and tell you about an old dream that's starting to come true.
People are finally starting to understand extended topological
quantum field theories using n-categories!   

Back in 1995, Jim Dolan and I argued that n-dimensional extended 
TQFTS were representations of a certain n-category called nCob in 
which:

<ul>
<li>
 objects are 0-dimensional manifolds: that is, collections of points,
</li>
<li>
 morphisms are 1-dimensional manifolds with boundary going between
 collections of points,
</li>
<li>
 2-morphisms are 2-dimensional manifolds with corners going between
 1-dimensional manifolds with boundary going between collections of
 points,
</li>
</ul>

... and so on, up to dimension n.  In particular, any n-dimensional
manifolds is an n-morphism in nCob.  

And, we thought we could glimpse a purely algebraic description of
nCob.  We called this the "cobordism hypothesis", and we explained 
it here:

1) John Baez and James Dolan, Higher-dimensional algebra and
topological quantum field theory, J. Math. Phys. 36 (1995), 6073-6105.
Also available as <a href =
"http://arxiv.org/abs/q-alg/9503002">q-alg/9503002</a>.

I talked about this back in "<a href =
"week49.html">week49</A>".  For more, try these talks:

2) Eugenia Cheng, n-Categories with duals and TQFT, 4 lectures at the
Fields Institute, January 2007.  Audio available at
<a href = "http://www.fields.utoronto.ca/audio/06-07/#crs-ncategories">http://www.fields.utoronto.ca/audio/06-07/#crs-ncategories</a>
and lecture notes by Chris Brav at
<a href = "http://math.ucr.edu/home/baez/fields/eugenia.pdf">http://math.ucr.edu/home/baez/fields/eugenia.pdf</a>

Now Jacob Lurie has come out with a draft of an expository paper 
that outlines a massive program, developed with help from Mike Hopkins,
to reformulate the cobordism hypothesis using more ideas from homotopy theory, and prove it:

3) Jacob Lurie, On the classification of topological field theories,
available as <a href = "http://arxiv.org/abs/0905.0465">arXiv:0905.0465</a>.

He's running around giving talks about this work, and you can see some
here:

4) Jacob Lurie, TQFT and the cobordism hypothesis, four lectures at
the Geometry Research Group of the University of Texas at Austin,
January 2009.  Videos available at 
<a href = "http://lab54.ma.utexas.edu:8080/video/lurie.html">http://lab54.ma.utexas.edu:8080/video/lurie.html</a>
and lecture notes by Braxton Collier, Parker Lowrey and Michael 
Williams at
<a href = "http://www.ma.utexas.edu/users/plowrey/dev/rtg/notes/perspectives_TQFT_notes.html">http://www.ma.utexas.edu/users/plowrey/dev/rtg/notes/perspectives_TQFT_notes.html</a>

Excited by this new progress, I decided to run around giving some 
talks about it myself - just to explain the basic intuitions to 
people who'd never thought about this stuff before.  You can see
my slides here:

5) John Baez, Categorification and topology, available at
<a href = "http://math.ucr.edu/home/baez/cat/">http://math.ucr.edu/home/baez/cat/</a>

A key feature of Lurie's approach is that instead of using
n-categories he uses (\infty ,n)-categories, which are
\infty -categories where everything is invertible above dimension n.
This is what gets ideas from homotopy theory into the game.
I should talk about this more someday.

Meanwhile, Chris Schommer-Pries has written a thesis on 2d extended
TQFTs which follows an approach much closer to what Jim and I had originally
imagined.  You could say he gives more of an individually hand-crafted
treatment of the n = 2 case, as compared with Lurie's high-tech industrial 
approach that clobbers all n at once:

6) Chris Schommer-Pries, The Classification of Two-Dimensional 
Extended Topological Field Theories, Ph.D. theis, U.C. Berkeley, 2009.
Available at <a href = "http://sites.google.com/site/chrisschommerpriesmath/">http://sites.google.com/site/chrisschommerpriesmath/</a>

Instead of (\infty ,n)-categories, Schommer-Pries just uses n-categories - 
and since he's doing 2d TQFTs, that means 2-categories.
Or more precisely, "weak" 2-categories, where all the laws hold
only up to equivalence.  Like most people, he calls these 
"bicategories".  And one of the charms of his thesis is that he 
gives a detailed treatment of the n = 2 column of the periodic
table of n-categories - which in his language looks like this:


\begin{verbatim}

                   k-tuply monoidal n-categories 

              n = 0           n = 1              n = 2

k = 0         sets          categories        bicategories

k = 1        monoids         monoidal           monoidal
                            categories        bicategories

k = 2       commutative      braided            braided
             monoids         monoidal           monoidal
                            categories        bicategories 

k = 3         " "           symmetric          sylleptic
                             monoidal          monoidal 
                            categories        bicategories

k = 4         " "             " "              symmetric
                                               monoidal
                                              bicategories

k = 5         " "             " "                "  "
\end{verbatim}
    

A k-tuply monoidal n-category is an (n+k)-category that's boring
at the bottom k levels.  For example, a category with just one
object is a monoid.  As we increase k, we get more and more 
commutative flavors of n-category.  But after k hits n+2, we
expect that increasing k further has no effect.  At this point we 
say our n-category is "stable".

If the cobordism hypothesis is true, nCob is a stable n-category.  
For n = 2, such a gadget is often called a "symmetric monoidal 
bicategory".  Schommer-Pries shows that 2Cob is indeed a symmetric
monoidal bicategory.  Even better, he gives a "generators and 
relations" description of this gadget, which is just the sort of thing 
we need for the 2d version of the cobordism hypothesis:

<div align = "center">
<img src = "schommer_pries.jpg">
</div>

At this point, any n-category theorist could finish off the job.  

(Well, the really nice statement of the cobordism hypothesis involves
\emph{framed} oriented cobordisms, and we may need a topologist to tell us
how those work - but there's also a version of the hypothesis for plain 
old oriented cobordisms, and that's what Schommer-Pries' thesis will
give.)

For more on nCob as an n-category, try this:

7) Eugenia Cheng and Nick Gurski, Toward an n-category of cobordisms,
Theory and Applications of Categories 18 (2007), 274-302.  Available at
<a href = "http://www.tac.mta.ca/tac/volumes/18/10/18-10abs.html">http://www.tac.mta.ca/tac/volumes/18/10/18-10abs.html</a>

I should add that a lot of the 2-category theory in Schommer-Pries'
thesis relies on a thesis by a student of Ross Street:

8) Paddy McCrudden, Balanced coalgebroids, Theory and Applications of
Categories 7 (2000), 71-147.  Available at
<a href = "http://www.tac.mta.ca/tac/volumes/7/n6/7-06abs.html">
http://www.tac.mta.ca/tac/volumes/7/n6/7-06abs.html</a>

Two students of mine should read the stuff about symmetric monoidal
bicategories in this thesis!  One is Alex Hoffnung, whose work on
Hecke algebras uses the symmetric monoidal bicategory where:

<ul>
<li>
objects are groupoids,
</li>
<li>
morphisms are spans of groupoids where the legs are fibrations,
</li>
<li>
2-morphisms are maps of spans of groupoids.
</li>
</ul>

The other is Mike Stay, whose work on computer science uses
the symmetric monoidal bicategory where:

<ul>
<li>
objects are categories,
</li>
<li> 
morphisms are profunctors,
</li>
<li>
2-morphisms are natural transformations between profunctors.
</li>
</ul>

A profunctor is a categorified version of a matrix.  More precisely,
a profunctor from C to D is a functor

F: C \times  D^{op} \to  Set

so it's like a matrix of sets.  A span of groupoids where the
legs are fibrations is also a categorified version of a matrix,
since by a theorem of Grothendieck we can reinterpret it as a
weak 2-functor

F: C \times  D^{op} \to  Gpd

where now C and D are groupoids.  So, both these students are
studying aspects of "categorified matrix mechanics"... and we
need symmetric monoidal bicategories to provide the proper context
for such work.  This should connect up to the 2d version of the
cobordism hypothesis in some interesting ways.

\par\noindent\rule{\textwidth}{0.4pt}

<em>
As for your problems... I am so tired of mathematics and hold it
in such low regard, that I could no longer take the trouble to solve
them myself.</em> - Descartes to Mersenne

\par\noindent\rule{\textwidth}{0.4pt}

% </A>
% </A>
% </A>
