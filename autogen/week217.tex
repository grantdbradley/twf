
% </A>
% </A>
% </A>
\week{May 30, 2005 }


Last week I described lots of different zeta functions, but didn't say much 
about what they're good for.   This week I'd like to get started on fixing 
that problem.  

People have made lots of big conjectures related to zeta functions.  
So far they've just proved just a few... but it's still a big deal.

For example, Andrew Wiles' proof of Fermat's Last Theorem was just 
a tiny spin-off of his work on something much bigger called the 
Taniyama-Shimura conjecture.  Now, personally, I think Fermat's Last 
Theorem is a ridiculous thing.  The last thing I'd ever want to know is 
whether this equation:

x^{n} + y^{n} = z^{n}

has nontrivial integer solutions for n > 2.  But the Taniyama-Shimura
Conjecture is really interesting!   It's all about the connection between 
geometry, complex analysis and arithmetic, and it ties together some big 
ideas in an unexpected way.  This is how it usually works in number theory: 
cute but goofy puzzles get solved as a side-effect of deep and interesting 
results related to zeta functions and L-functions - sort of like how the 
powdered drink "Tang" was invented as a spinoff of going to the moon.

For a good popular book on Fermat's Last Theorem and the Taniyama-Shimura
Conjecture, try:

1) Simon Singh, Fermat's Enigma: The Epic Quest to Solve the World's 
Greatest Mathematical Problem, Walker, New York, 1997.

Despite the "world's greatest mathematical problem" baloney, this 
book does 
a great job of telling the story without drowning the reader in math.

But you read This Week's Finds because you \emph{want} to be drowned in math, 
and I wouldn't want to disappoint you.  So, let me list a few of the big 
conjectures and theorems related to zeta functions.

Here goes:


<b>A) The Riemann Hypothesis - the zeros of the Riemann zeta function in 
the critical strip
0 \le  Re(s) \le  1
actually lie on the line Re(s) = 1/2.  </b>

First stated in 1859 by Bernhard Riemann; still open.

This implies a good estimate on the number of primes less than a given 
number, as described in "<A HREF = "week216.html">week216</A>".

<b>B) The Generalized Riemann Hypothesis - the zeros of any Dirichlet 
L-function 
that lie in the critical strip actually lie on the line Re(s) = 1/2.
</B>
Still open, since the Riemann Hypothesis is a special case.

A "Dirichlet L-function" is a function like this:

L(&chi;,s) = &sum;_{n > 0} &chi;(n)/n^{s}

where &chi; 
is any "Dirichlet character", meaning a periodic complex function 
on the positive integers such that

&chi;(nm) = &chi;(n) &chi;(m)

If we take &chi; = 1 we get back the Riemann zeta function.

Dirichlet used these L-functions to prove that there are infinitely many 
primes equal to k mod n as long as k is relatively prime to n.  The 
Generalized Riemann Hypothesis would give a good estimate on the number 
of such primes less than a given number, just as the Riemann Hypothesis 
does for plain old primes.

Erich Hecke established the basic properties of Dirichlet L-functions 
in 1936, including a special symmetry called the "functional equation" 
which Riemann had already shown for his zeta function.   So I bet Hecke 
must have dreamt of the Generalized Riemann Hypothesis, even if he didn't 
dare state it.

<b>C) The Extended Riemann Hypothesis - for any number field, the zeros of its 
zeta function in the critical strip actually lie on the line Re(s) = 1/2.</b>  

Still open, since the Riemann Hypothesis is a special case.

I described the zeta functions of number fields in "<A HREF = "week216.html">week216</A>".
These are usually called "Dedekind zeta functions".  Hecke also
proved a functional equation for these back in 1936.

<b>D) The Grand Riemann Hypothesis - for any automorphic L-function,
the zeros in the critical strip actually lie on the line Re(s) = 1/2.</b>

This is still open too, since it includes A)-C) as special cases!

I don't want to tell you what "automorphic L-functions" are yet.  
For now, you can just think of them as grand generalizations of both
Dirichlet L-functions and zeta functions of number fields.

<b>E) The Weil Conjectures - The zeros and poles of the zeta function of any 
smooth algebraic variety over a finite field lie on the lines Re(s) = 1/2,
1, 3/2, ... d/2 where d is the dimension of the variety.</b>  The zeros lie
on the lines 1/2, 3/2, ... while the poles lie on the lines 1, 2, .... 
Also: such zeta functions are quotients of polynomials, they satisfy a 
functional equation, and a lot of information about their zeroes and poles 
can be recovered from the topology of the corresponding \emph{complex} 
algebraic varieties.  

First stated in 1949 by Andre Weil; proof completed by Pierre Deligne 
in 1974 based on much work by Michael Artin, J.-L. Verdier, and especially 
Alexander Grothendieck.  Grothendieck invented topos theory as part of 
the attack on this problem!

<b>F) The Taniyama-Shimura Conjecture - every elliptic curve over the rational 
numbers is a modular curve.</b>  Or, equivalently: every L-function of an 
elliptic curve is the L-function of a modular curve.  

This was first conjectured in 1955 by Yukata Taniyama, who worked on it 
with Goro Shimura until committing suicide in 1958.  Around 1982 Gerhard 
Frey suggested that this conjecture would imply Fermat's Last Theorem; this 
was proved in 1986 by Ken Ribet.  In 1995 Andrew Wiles and Richard Taylor 
proved a big enough special case of the Taniyama-Shimura Conjecture to get 
Fermat's Last Theorem.  The full conjecture was shown in 1999 by Breuil, 
Conrad, Diamond, and Taylor.

I don't want to say what L-functions of curves are yet... but they
are a lot like Dirichlet L-functions.

<b>G) The Langlands Conjectures - any automorphic representation \pi  of a 
connected reductive group G, together with a finite-dimensional representation 
of its L-group, gives an automorphic L-function L(s,\pi ).</b>  Also:
these L-functions 
all satisfy functional equations.  Furthermore, they depend functorially on
the group G, its L-group, and their representations. 

Zounds!  Don't worry if this sounds like complete gobbledygook!  I only 
mention it to show how scary math can get.  As Stephen Gelbart once wrote:

\begin{quote}
     The conjectures of Langlands just alluded to amount (roughly)
     to the assertion that the other zeta-functions arising in 
     number theory are but special realizations of these L(s,\pi ).

     Herein lies the agony as well as the ecstacy of Langlands'
     program.  To merely state the conjectures correctly requires
     much of the machinery of class field theory, the structure
     theory of algebraic groups, the representation theory of real
     and p-adic groups, and (at least) the language of algebraic
     geometry.  In other words, though the promised rewards are 
     great, the initiation process is forbidding.
\end{quote}
I hope someday I'll understand this stuff well enough to say something more 
helpful!  Lately I've been catching little glimpses of what it's about....

But, right now I think it's best if I talk about the "functional equation" 
satisfied by the Riemann zeta function, since this gives the quickest way 
to see some of the strange things that are going on.

The Riemann zeta function starts out life as a sum:

&zeta;(s) = 1^{-s} + 2^{-s} + 3^{-s} + 4^{-s} + ....

This only converges for Re(s) > 1.  It blows up as we approach s = 1, 
since then we get the series

1/1 + 1/2 + 1/3 + 1/4 + ....

which diverges.  However, in 1859 Riemann showed that we can analytically 
continue the zeta function to the whole complex plane except for this pole 
at s = 1.  

He also showed that the zeta function has an unexpected symmetry: 
its value at any complex number s is closely related to its value at 1-s.  
It's not true that &zeta;(s) = &zeta;(1-s), but something similar is true, 
where we multiply the zeta function by an extra fudge factor.  

To be precise: if we form the function

\pi ^{-s/2} \Gamma (s/2) &zeta;(s)

then this function is unchanged by the transformation

s |\to  1 - s

This symmetry maps the line 

Re(s) = 1/2

to itself, and the Riemann Hypothesis says all the &zeta; zeros in 
the critical strip actually lie on this magic line.  

This symmetry is called the "functional equation".  It's the tiny tip of a 
peninsula of a vast and mysterious continent which mathematicians are still 
struggling to explore.  Riemann gave two proofs of this equation.  You can 
find a precise statement and a version of Riemann's second proof here:

2) Daniel Bump, Zeta Function, lecture notes on "the functional
equation" available at <A HREF = "http://math.stanford.edu/~bump/zeta.html">http://math.stanford.edu/~bump/zeta.html</A>
and <A HREF = "http://www.maths.ex.ac.uk/~mwatkins/zeta/fnleqn.htm">http://www.maths.ex.ac.uk/~mwatkins/zeta/fnleqn.htm</A>

This proof is a beautiful application of Fourier analysis.  Everyone
should learn it, but let me try to sketch the essential idea here.

I will deliberately be VERY rough, and use some simplified nonstandard
definitions, since the precise details have a way of distracting your 
eye just as the magician pulls the rabbit out of the hat.

We start with the function &zeta;(2s):

1^{-s} + 4^{-s} + 9^{-s} + 25^{-s} + ....

Then we apply a curious thing called the "inverse
Mellin transform", which turns 
this function into 

z^{1}  + z^{4}  + z^{9}  + z^{25} + ....

Weird, huh?  This is almost the "theta function"

\theta (t) = &sum;_{n} e^{\pi  i n<sup>2} t</sup>

where we sum over all integers n.  Indeed, it's easy to see that

(\theta (t) - 1)/2 = z^{1} + z^{4} + z^{9} + 
z^{25} + .... 

when 

z = e^{\pi  i t}

The theta function transforms in a very simple way when we replace
t by -1/t, as one can show using Fourier analysis.

Unravelling the consequences, this implies that the zeta function 
transforms in a simple way when we replace s by 1-s.  You have to
go through the calculation to see precisely how this works... but 
the basic idea is: a symmetry in the theta function yields a symmetry
in the zeta function.

Hmm, I'm not sure that explained anything!  But I hope at least the 
mystery is more evident now.  A bunch of weird tricks, and then 
\emph{presto} - 
the functional equation!  To make progress on understanding the Riemann 
Hypothesis and its descendants, we need to get what's going on here.  

I feel I \emph{do} get the inverse
Mellin transform; I'll say more about that later.  
But for now, note that the theta function transforms in a simple way, not 
just when we do this:
 
t |\to  -1/t

but also when we do this:

t |\to  t + 2

Indeed, it doesn't change at all when we add 2 to t, since e^{2\pi  i} = 1.   

Now, the maps

t |\to  -1/t

and 

t |\to  t + 1

generate the group of all maps


$$

       at + b
t |\to   --------
       ct + d
$$
    
where a,b,c,d form a 2\times 2 matrix of integers with determinant 1.  
These maps form a group called PSL(2,Z), or the "modular group".  

A function that transforms simply under this group and doesn't blow up 
in nasty ways is called a "modular form".  In "<A HREF = "week197.html">week197</A>" I gave the precise 
definition of what counts as transforming simply and not blowing up in 
nasty ways.  I also explained how modular forms are related to elliptic 
curves and string theory.  So, please either reread "<A HREF = "week197.html">week197</A>" or take my 
word for it: modular forms are cool!  

The theta function is almost a modular form, but not quite.  It doesn't 
blow up in nasty ways.  However, it only transforms simply under a subgroup 
of PSL(2,Z), namely that generated by 

t |\to  -1/t

and

t |\to  t + 2

So, the theta function isn't a full-fledged modular form.  
But since it comes close, we call it an "automorphic form".

Indeed, for any discrete subgroup G of PSL(2,Z), functions that transform 
nicely under G and don't blow up in nasty ways are called "automorphic forms" 
for G.  They act a lot like modular forms, and people know vast amounts 
about them.  It's the power of automorphic forms that makes number theory 
what it is today!

We can summarize everything so far in this slogan:

<DIV ALIGN = CENTER>
    THE FUNCTIONAL EQUATION FOR THE RIEMANN ZETA FUNCTION SAYS: <br> 
           "THE THETA FUNCTION IS AN AUTOMORPHIC FORM"

</DIV>

But before you start printing out bumper stickers, I should explain....


The point of this slogan is this.  We \emph{thought} we were interested in 
the Riemann zeta function for its own sake, or what it could tell us 
about prime numbers.  But with the wisdom of hindsight, the first thing we 
should do is hit this function with the inverse 
Mellin transform and repackage all 
its information into an automorphic form - the theta function.  

&zeta; is dead, long live \theta !

The Riemann zeta function is just like all the fancier zeta functions and 
L-functions in this respect.  The fact that they satisfy a "functional 
equation" is just another way of saying their inverse
Mellin transforms are 
automorphic forms... and it's these automorphic forms that exhibit the 
deeper aspects of what's going on.

Now let me say a little bit about the inverse Mellin transform.

Ignoring various fudge factors, the inverse 
Mellin transform is basically just 
the linear map that sends any function of s like this:

n^{-s}

to this function of z:

z^{n}

In other words, it basically just turns things upside down, replacing the 
base by the exponent and vice versa.  The minus sign is just a matter of 
convention; don't worry about that too much.

So, the inverse
Mellin transform basically sends any function like this, called a 
"Dirichlet series":

a_{1} 1^{-s} + a_{2} 2^{-s} + a_{3} 3^{-s} + a_{4} 4^{-s} + ....

to this function, called a "Taylor series":

a_{1} z^{1} + a_{2} z^{2} + a_{3} z^{3} + a_{4} z^{4} + ....

Now, why would we want to do this? 

The reason is that multiplying Taylor series is closely related 
to \emph{addition} 
of natural numbers:

z^{n} z^{m} = z^{n+m}

while multiplying Dirichlet series is closely related to 
\emph{multiplication}
of natural numbers:

n^{-s} m^{-s} = (nm)^{-s}

The Mellin transform (and its inverse) are how we switch between these two 
pleasant setups!

Indeed, it's all about algebra - at least at first.  We can add natural 
numbers and multiply them, so N becomes a monoid in two ways.  A "monoid", 
recall, is a set with a binary associative product and unit.  So, we have 
two closely related monoids:

(N,+,0)

and 

(N,\times ,1)

Given a monoid, we can form something called its "monoid algebra" by taking 
formal complex linear combinations of monoid elements.  We multiply these 
in the obvious way, using the product in our monoid.

If we take the monoid algebra of (N,+,0), we get the algebra of Taylor 
series!  If we take the monoid algebra of (N,\times ,1), we get the algebra of 
Dirichlet series!  

(Actually, this is only true if we allow ourselves to use \emph{infinite} 
linear 
combinations of monoid elements in our monoid algebra.  So, let's do that.  
If we used finite linear combinations, as people often do, (N,+,0) would give 
us the algebra of polynomials, while (N,\times ,0) would give us the algebra of 
"Dirichlet polynomials".)

Of course, algebraically we can combine these structures.  (N,+,\times ,0,1) is 
a rig, and by taking formal complex linear combinations of natural numbers 
we get a "rig algebra" with two products: the usual product of Taylor series, 
and the usual product of Dirichlet series.  They're compatible, too, since 
one distributes over the other.  They both distribute over addition.

However, if we're trying to get an algebra of functions on the complex plane, 
with pointwise multiplication as the product, we need to make up our mind: 
either Taylor series or Dirichlet series!  We then need the Mellin transform 
to translate between the two.

So, what seems to be going on is that people take a puzzle, like 

\begin{quote}
"what is the sum of the squares of the divisors of n?"
\end{quote}

or 
\begin{quote}
"how many ideals of order n are there in this number field?"
\end{quote}

and they call the answer a_{n}.  

Then they encode this sequence as either a Dirichlet series:

a_{1} 1^{-s} + a_{2} 2^{-s} + a_{3} 3^{-s} + a_{4} 4^{-s} + ....

or a Taylor series:

a_{1} z^{1} + a_{2} z^{2} + a_{3} z^{3} + a_{4} z^{4} + ....

The first format is nice because it gets along well with multiplication of 
natural numbers.  For example, in the puzzle about ideals,
every ideal is a product of prime ideals, and its norm is 
the product of the norms of those prime ideals... so our Dirichlet series 
will have an Euler product formula.

The second format is nice \emph{if} our Taylor series is an automorphic form.
This will happen precisely when our Dirichlet series satisfies a functional 
equation.

(For experts: I'm ignoring some fudge factors involving the gamma function.)

I still need to say more about \emph{which} puzzles give automorphic forms,
what it really means when they \emph{do}.  But, not this week!  I'm tired, 
and I bet you are too.

For now, let me just give some references.  There's a vast amount of material 
on all these subjects, and I've already referred to lots of it.  But right now 
I want to focus on stuff that's free online, especially stuff that's readable 
by anyone with a solid math background - not journal articles for experts, but 
not fluff, either.

Here's some information on the Riemann Hypothesis provided by the Clay 
Mathematics Institute, which is offering a million dollars for its solution:

3) Clay Mathematics Institute, Problems of the Millennium: 
the Riemann Hypothesis, <A HREF = "http://www.claymath.org/millennium/">http://www.claymath.org/millennium/</A>

The official problem description by Enrico Bombieri talks about evidence 
for the Riemann Hypothesis, including the Weil Conjectures.  The article by 
Peter Sarnak describes generalizations leading up to the Grand Riemann 
Hypothesis.  In particular, he gives a super-rapid introduction to 
automorphic L-functions.

Here's a nice webpage that sketches Wiles and Taylor's proof of Fermat's last 
theorem:

4) Charles Daney, The Mathematics of Fermat's Last Theorem,
<A HREF = "http://www.mbay.net/~cgd/flt/fltmain.htm">http://www.mbay.net/~cgd/flt/fltmain.htm</A>

I like the quick introductions to "Elliptic curves and elliptic functions", 
"Elliptic curves and modular functions", "Zeta and L-functions", and "Galois 
Representations" - they're neither too detailed nor too vague, at least for 
me.  

Here's a nice little intro to the Weil Conjectures:

5) Runar Ile, Introduction to the Weil Conjectures,
<A HREF = "http://folk.uio.no/~ile/WeilA4.pdf">http://folk.uio.no/~ile/WeilA4.pdf</A>

James Milne goes a lot deeper - his course notes on etale cohomology include 
a proof of the Weil Conjectures:

6) James Milne, Lectures on Etale Cohomology,
<A HREF = "http://www.jmilne.org/math/CourseNotes/math732.html">http://www.jmilne.org/math/CourseNotes/math732.html</A>

while his course notes on elliptic curves sketch the proof of Fermat's Last 
Theorem:

7) James Milne, Elliptic Curves, 
<A HREF = "http://www.jmilne.org/math/CourseNotes/math679.html">http://www.jmilne.org/math/CourseNotes/math679.html</A>

Here's a nice history of what I've been calling the Taniyama-Shimura
Conjecture, which explains why some people call it the Taniyama-Shimura-Weil 
conjecture, or other things:

8) Serge Lang, Some history of the Shimura-Taniyama Conjecture,
AMS Notices 42 (November 1995), 1301-1307.  Available at
<A HREF = "http://www.ams.org/notices/199511/forum.pdf">http://www.ams.org/notices/199511/forum.pdf</A>

Here's a quick introduction to the proof of this conjecture, whatever 
it's called:

9) Henri Darmon, A proof of the full Shimura-Taniyama-Weil Conjecture
is announced, AMS Notices 46 (December 1999), 1397-1401.  Available
at <A HREF = "http://www.ams.org/notices/199911/comm-darmon.pdf">http://www.ams.org/notices/199911/comm-darmon.pdf</A>

I won't give any references to the Langlands Conjectures, since
I hope to talk a lot more about those some other time.

And, I hope to keep on understanding this stuff better and better!  

I thank James Borger and Kevin Buzzard for help with this issue of
This Week's Finds.

\par\noindent\rule{\textwidth}{0.4pt}
\textbf{Addendum:} Here's part of an email exchange I had with Kevin Buzzard
of Imperial College after he read this Week's Finds:

I wrote:


$$

 > What I *REALLY* want to know is: what combinatorial properties of
 > an integer sequence a_{n} are we being told when we're told that
 > the Dirichlet series
 >
 > sum_n a^n n^{-s}
 >
 > comes from an automorphic form?
$$
    
He replied:

\begin{quote}
 Yeah, that's a really key question. I'm not sure that there is an elementary
 answer.  Here is another question: given a sequence of complex numbers
 a_{1}, a_{2},...,a_{n},..., with 

a_{n} =
 O(n^{r&nbsp;}), 

 what is a neat easy-to-understand
 property of this sequence which implies (or is implied by, or is equivalent 
 to) the statement that 

&sum; a_{n}/n^{s} 
has analytic (or meromorphic) continuation
 beyond Re(s)>r?  Maybe even this is hard---or maybe there is no such
 elementary criterion.


$$

 > I'll be happy to assume for starters that a_{n} is multiplicative.
$$
    

 This might not "logically speaking" be necessary, but on the other hand
 probably the most interesting cases have this property. Here is an
 example. Take a sequence of complex numbers a_{1}, 
 a_{2},... which
 is periodic with prime period p (primality probably isn't necessary but it
 simplifies the combinatorics). Then the associated L-function

 &sum; a_{n}/n^{s}
 has
 meromorphic continuation with at worst a simple pole at s=1 and no
 other poles, and one could even argue that the a_{i} "came from an
 automorphic form".

 On the other hand, this is not the kind of automorphic form that
 people usually think about because it's not an eigenform. What is
 happening is that "there are enough Dirichlet characters": consider
 the trivial character &chi;(n)=1 for all n, and then the p-1 Dirichlet
 characters of level p, those defined by group homomorphisms 

&chi;:(Z/pZ)* \to  C*
 and extended to functions on Z by &chi;(n)=0 for n a multiple of p.
 These p functions on Z form a basis of the vector space of periodic functions
 on Z with period p.

 The Dirichlet characters give automorphic forms, but automorphic
 forms are a vector space so you can add them together and get
 an automorphic form for any periodic function. However the Dirichlet
 characters are the most interesting such forms because these are the ones
 which are eigenforms. The eigenforms give multiplicative a_{n}, but
 it's certainly not true in general that a periodic function is 
 multiplicative.
 
 But I can't really enlighten you much further. I know that the
 L-function of an automorphic form has meromorphic continuation
 and that we understand the poles (but we only conjecturally understand
 the zeroes). However if someone put some a_{n} in front of me I would
 demand that they told me where the a_{n} had come from before I put
 my money on whether there was an automorphic form.
 
 The example where I actually proved something was in the case
 where the a_{n} came from a finite-dimensional complex representation 
 of Gal(K/Q) with K a number field, Galois over Q.  (In fact my only 
 contribution was in the 2-dimensional case, the 1-dimensional case 
 having been understood for some time.)  The a_{n} are then related to
 traces of the representation.  Artin conjectured in the 1930s that
 &sum; a_{n}/n^{s} 
 should have analytic continuation to all of the
 complex plane if the representation was irreducible and not the trivial
 1-dimensional representation.  Langlands conjectured much later that 
 the a_{n} should come from an automorphic form, and we knew by then 
 that Langlands' conjecture implied Artin's.  But none of the analytic 
 guys know how to prove Artin's conjecture without essentially deducing 
 it from Langlands'!  I did something with some other Brits in the 
 2-dimensional case. Ironically, to deduce our analytic continuation 
 results, we proved some p-adic analytic continuation results first :-) 
 We constructed a modular form using p-adic techniques (and all of 
 Wiles' machinery).


$$

 > I get the feeling that nobody knows the answer, except perhaps
 > for specific cases like modular forms, where we know they're all
 > linear combinations of products of Eisenstein series, so that a_{n}
 > is built out of sequences like \sigma _{k}(n) - sums of kth powers of divisors.
$$
    

 But unfortunately this is only true for "level 1" modular forms: you
 can build all modular forms of level 1 from the Eisenstein series
 E_{4} and E_{6}. There is no neat analogous result for modular forms
 for the group \Gamma _{0}(N) for general N. In particular you will never
 see the a_{n} for an elliptic curve built up from Eisenstein series in this
 way :-(


\begin{verbatim}

 > What I'd like is a really "conceptual" answer... or else for someone
 > to admit that nobody knows such an answer yet.
\end{verbatim}
    

 I think I will freely admit that, although that's just a personal opinion.

 ***

 Here is why Dirichlet characters are the same as 1-dimensional complex
 representations of Gal(Q^{-}/Q). It's called the Kronecker-Weber
 theorem, it pre-dates class field theory (although it is now a special
 case), and you can \emph{just about} prove it at the end of an introductory
 undergraduate course on number fields, as the last starred question
 on the example sheet, and only for people that have done the Galois
 theory course too.  Let K be a number field and assume K is Galois
 over Q (equivalently, that there is a polynomial f with rational
 coefficients such that K is the smallest subfield of the complex
 numbers containing all the roots of f; K is called the "splitting 
 field" of f).

 Then K has a Galois group Gal(K/Q), which is the field automorphisms 
 of K that fix Q.  Say f has degree n and n roots z_{1}, z_{2}, ... , z_{n}. 
 Then any automorphism of K fixes f, so it permutes the roots of f. 
 So we get an embedding 

 Gal(K/Q) \to  S_{n}

 where S_{n} is the symmetric group on the set  z_{1}, z_{2}, ... , z_{n}. 
 Then any automorphism of K fixes f, so it permutes the roots of f. 
 So we get an embedding 
 "Generically" this map is an isomorphism. But certainly not 
 always---if there are subtle relations amongst the z_{i} with 
 rational coefficients then these subtle relations must also be 
 preserved by the Galois group.  One of my favourite examples is the 
 equation x^{4}-2.  This is an irreducible polynomial of degree 4, 
 the four roots are z, iz, -z and -iz where z is the real 4th root 
 of 2 and i is \sqrt -1.  But z+(-z) = 0, so if K is the field 
 generated by these roots and \sigma  is an automorphism of K then 
 \sigma (z)+\sigma (-z) will also be zero, and the Galois group cannot 
 possibly be all of S_{4}.  In fact one can check that the Galois group 
 is D_{8}, the dihedral group with 8 elements (the four roots form a 
 square in the complex plane and it's the symmetries of this square).

 So there's Galois theory. Now here's a question: can we classify
 all the number fields K, Galois over Q, with Gal(K/Q) an abelian group?

 Here are some examples: K=Q(\sqrt d), the splitting field of x^{2}-d.
 If d is the square of a rational then K=Q and if not then it's
 an extension of degree 2, with Galois group S_{2} which is abelian.

 The splitting field of x^{n}-1, called the nth cyclotomic field,
 also turns out to have abelian Galois group; if z = exp(2\pi in)
 then any automorphism of K must send z to another nth root of 
 unity and furthermore the nth root of unity must have "exact 
 order n", 
 i.e. its nth power must be 1 but none of its mth powers can be 1 
 for 0 < m < n.  So z must get sent to z^{a} with 0 < a < n coprime to n. 
 This gives us an injection 

 Gal(K/Q) \to  (Z/nZ)*,

 where (Z/nZ)* is the units in the ring Z/nZ (send \sigma  to a),
 and it's tricky but true that this is in fact an isomorphism 
 (use the fundamental theorem of Galois theory, and cyclotomic 
 polynomials, or try and get a "trick" proof that uses as little 
 of this as possible, but it's still some work). In any case Gal(K/Q) 
 is certainly abelian.

 Next example: any subfield of any previous example, because this
 is how Galois theory works: if K contains L contains Q, and
 both K and L are Galois over Q, then the obvious restriction
 map Gal(K/Q) \to  Gal(L/Q) is a surjection.

 So we now have quite a general example of a number field K with
 Gal(K/Q) abelian: any subfield of a cyclotomic field. The hard
 theorem (not really too hard, but quite messy) is that the
 converse is true: Gal(K/Q) abelian implies that K is contained
 within a cyclotomic field. For example Q(\sqrt 5) is in the 5th 
 cyclotomic field because cos(72&deg;) = (\sqrt 5-1)/4.
 
 Any Dirichlet character gives a group homomorphism

 (Z/nZ)* \to  C*, 
 so a map 
Gal(K_{n}/Q) \to  C*,
 with K_{n} the nth cyclotomic field,
 so a continuous group homomorphism 
Gal(Q^{-}/Q) \to  C*.
 Conversely any
 continuous group homomorphism
 Gal(Q^{-}/Q) \to  C*
 factors through a compact
 discrete quotient of Gal(Q^{-}/Q), 
 which is just Gal(K/Q) for some number
 field K, and we get an injection 
Gal(K/Q) \to  C*, 
so Gal(K/Q) is abelian,
 so K is contained in a K^{n} 
for some n, so we get a map 
Gal(K_{n}/Q) \to  C*
 so it's a Dirichlet character.
 
 Amazingly it might have been Langlands who really sold this "duality"
 point, 100 years after it was understood: people always used to
 state Kronecker-Weber as "any abelian number field is contained
 in a cyclo field" rather than the dual "any 1-dimensional rep
 of Gal(Q^{-}/Q) comes from a Dirichlet character".  It was perhaps
 Langlands who realised that the correct generalisation of this
 statement was "any n-dimensional rep of Gal(Q^{-}/Q) comes from
 an automorphic form", rather than a statement about non-abelian
 extensions of Gal(Q^{-}/Q).  Perhaps you will like the reason that
 people find the representation-theoretic approach appealing: I have
 been talking about number fields as subfields of the complexes,
 but really a number field is an abstract object which is a field
 and happens to have finite dimension over Q, but it does not have
 a preferred embedding into the complexes.  As a result of this
 sort of thinking, one realises that Q^{-} is unique, but only up
 to highly non-unique isomorphism, and hence Gal(Q^{-}/Q) is
 a "group only defined up to inner automorphism"!  Hence it almost
 makes no sense to study this group---we cannot make any serious
 sense of an element of this group, because it's only the
 conjugacy classes that are well-defined.  Hence we should study
 the representations of the group on abstract vector spaces (i.e.
 ones without preferred bases), because these are well-defined up
 to isomorphism.  The reason there was so much mileage in the abelian
 case was that this subtlety goes away: an abelian group up to inner
 automorphism is an abelian group.

 ***

 I know some facts about the sequence a_{n} coming from, say, an elliptic
 curve over the rationals, but the killer, and one that is really
 hard to "read off" from the a_{n}, is that the a_{n} are related to the
 traces of Frobenius elements on a 2-dimensional p-adic Galois representation
 (the Tate module of the curve).  The moment I see a Galois representation
 I think that this must be something to do with automorphic forms,
 so that's why I believe that the L-function of an elliptic
 curve should come from a modular form.  And it does!  On the other
 hand, if you give me any finite set of primes p, and any integers
 a_{p} with p running through the set, such that 
 |a_{p}|^{2}\le 4p, then
 I can concoct an elliptic curve with these a_{p}, so at the very
 least one has to look at infinitely many of the a_{n} before one
 can begin to guess whether the a_{n} come from an automorphic form.

 Kevin
\end{quote}

\par\noindent\rule{\textwidth}{0.4pt}
<em>If I were to awaken after having slept for a thousand years, my 
first question would be: "Has the Riemann hypothesis 
been proven?"</em> -
David Hilbert

\par\noindent\rule{\textwidth}{0.4pt}

% </A>
% </A>
% </A>
