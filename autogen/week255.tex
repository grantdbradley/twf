
% </A>
% </A>
% </A>


I've been roaming around Europe this summer - first Paris, then
Delphi and Olympia, then Greenwich, then Oslo, and now back to
Greenwich.  I'm dying to tell you about the Abel Symposium in 
Oslo.  There were lots of cool talks about topological quantum 
field theory, homotopy theory, and motivic cohomology.  

I especially want to describe
Jacob Lurie and Ulrike Tillman's talks on cobordism n-categories,
Dennis Sullivan and Ralph Cohen's talks on string topology,  
Stephan Stolz's
talk on cohomology and quantum field theory, and Fabien Morel's
talk on A^{1}-homotopy theory.
But this stuff is sort of technical, and
I usually try to start each issue of This Week's Finds with something
you don't need a PhD to enjoy.

So, here's a tour of the Paris Observatory:

1) John Baez, Astronomical Paris, <a href =
"http://golem.ph.utexas.edu/category/2007/07/astronomical_paris.html">
http://golem.ph.utexas.edu/category/2007/07/astronomical_paris.html</a>

Back when England and France were battling to rule the world, 
each had a team of astronomers, physicists and mathematicians
devoted to precise measurement of latitudes, longitudes, and times.
The British team was centered at the Royal Observatory here in 
Greenwich.  The French team was centered at the Paris Observatory,
and it featured luminaries such as Cassini, Le Verrier and Laplace.

In "<A HREF = "week175.html">week175</A>", written during an
earlier visit to Greenwich, I mentioned a book on this battle:

2) Dava Sobel, Longitude, Fourth Estate Ltd., London, 1996.

It's a lot of fun, and I recommend it highly.

There's a lot more to say, though.  The speed of light was first
measured by Ole Romer at the Paris Observatory in 1676.  


<div align = "center">
<img border = "1" src = "diary/paris_2007/paris_observatory_romer_speed_of_light.jpg">
</div>

Later,
Henri Poincar&eacute; worked for the French Bureau of Longitude.  Among
other things, he was the scientific secretary for its mission to 
Ecuador.  

To keep track of time precisely all over the world, you need to 
think about the finite speed of light.  This may have spurred 
Poincar&eacute;'s work on relativity!  Here's a good book that argues
this case:

3) Peter Galison, Einstein's Clocks, Poincar&eacute;'s Maps: Empires 
of Time, W. W. Norton, New York, 2003.   Reviewed by Robert Wald
in Physics Today at <a href = "http://www.physicstoday.org/vol-57/iss-9/p57.html">http://www.physicstoday.org/vol-57/iss-9/p57.html</a>

I met Galison in Delphi, and it's clear he like to think about
the impact of practical stuff on math and physics.

I was in Delphi for a meeting of "Thales and Friends":

4) Thales and Friends, <a href =
"http://www.thalesandfriends.org">http://www.thalesandfriends.org</a>

This is an organization that's trying to bridge the gap between
mathematics and the humanities.  It's led by Apostolos Doxiadis, 
who is famous for this novel:

5) Apostolos Doxiadis, Uncle Petros and Goldbach's Conjecture,
Bloomsbury, New York, 2000.  Review by Keith Devlin at <a href =
"http://www.maa.org/reviews/petros.html">http://www.maa.org/reviews/petros.html</a>

There's a lot I could say about this meeting, but I just want
to advertise a forthcoming book by Doxiadis and a computer
scientist friend of his.  It's a comic book - sorry, I mean 
"graphic novel"! - about the history of mathematical logic
from Russell to Goedel:

6) Apostolos Doxiadis and Christos Papadimitriou, Logicomix,
to appear.

I saw a partially finished draft.  I think it does a good job 
of explaining to nonmathematicians what the big deal was with 
mathematical logic around the turn of the last century... and 
how these ideas eventually led to computers.  It's also a fun 
story.  

If you're eager for summer reading and can't wait for Logicomix,
you might try this other novel by Papadimitrou:

7) Christos Papadimitriou, Turing (a Novel about Computation),
MIT Press, Boston, 2003. 

It's a history of mathematics from the viewpoint of computer 
science, as told by a computer program named Turing to a 
lovelorn archaeologist.  I haven't seen it yet.

Okay - enough fun stuff.  On to the Abel Symposium!

8) Abel Symposium 2007, <a href =
"http://abelsymposium.no/2007">http://abelsymposium.no/2007</a>

Actually this was a lot of fun too.  A bunch of bigshots were 
there, including a bunch who didn't even give talks, like Eric
Friedlander, Ib Madsen, Jack Morava, and Graeme Segal.  

(My apologies to all the bigshots I didn't list.)

Speaking of bigshots, Vladimir Voevodsky gave a special surprise 
lecture on symmetric powers of motives.  He wowed the audience not 
only with his mathematical powers but also his ability to solve a 
technical problem that had stumped all the previous speakers!  The 
blackboards in the lecture hall were controlled electronically,
by a switch.  But, the blackboards only moved a few inches before 
stalling out.  So, people had to keep hitting the switch over and 
over.  It was really annoying, and it became the subject of running
jokes.  People would ask the speakers: "Can't you talk and press
buttons at the same time?"

So, what did Voevodsky do?  He lifted the blackboard by hand!  He
laughed and said "Russian solution".  But, I think it's a
great example of how he gets around problems by creative new
approaches.

It really pleased me how many talks mentioned n-categories, and 
even used them to do exciting things.  This seems quite new.  In
the old days, bigshots might think about n-categories, but they'd
be embarrassed to actually mention them, since they had a 
reputation for being "too abstract".

In fact, Dan Freed alluded to this in his talk on topological
quantum field theory.  He said that every mathematician has
an "n-category number".  Your n-category number is the largest n 
such that you can think about n-categories for a half hour without 
getting a splitting headache. 

When Freed first invented this concept, he felt pretty 
self-satisfied, since his n-category number was 1, while for 
most mathematicians it was 0.  But lately, he says, other 
people's n-category numbers have been increasing, while his has 
stayed the same.

He said this makes him suspicious.  In light of the scandals 
plaguing the Tour de France and American baseball, he suspects 
mathematicians are taking "category-enhancing substances"!

Freed shouldn't feel bad: he was among the first to introduce
n-categories in the subject of topological quantum field theory! 
He gave a nice talk on this, clear and unpretentious, leading 
up to a conjecture for the 3-vector space that Chern-Simons 
theory assigns to a point.  

That would make a great followup to these papers on the 2-vector 
space that Chern-Simons theory assigns to a circle:

9) Daniel S. Freed, The Verlinde algebra is twisted equivariant
K-theory, available as <a href =
"http://arxiv.org/abs/math/0101038">arXiv:math/0101038</a>.

Daniel S. Freed, Twisted K-theory and loop groups, available 
as 
<a href =
"http://arxiv.org/abs/math/0206237">arXiv:math/0206237</a>.

Daniel S. Freed, Michael J. Hopkins and Constantin Teleman,
Loop groups and twisted K-theory II, available as 
<a href = "http://arxiv.org/abs/math/0511232">arXiv:math/0511232</a>.

Daniel S. Freed, Michael J. Hopkins and Constantin Teleman,
Twisted K-theory and loop group representations, available as 
<a href = "http://arxiv/abs/math/0312155">arXiv:math/0312155</a>.
	
In a similar vein, Jacob Lurie talked about his work with Mike 
Hopkins in which they proved a version of the "Baez-Dolan cobordism 
hypothesis" in dimensions 1 and 2.   I'm calling it this because
that's what Lurie called it in his title, and it makes me feel good.  

You can read about this hypothesis here:

10) John Baez and James Dolan, Higher-dimensional algebra and
topological quantum field theory, J.Math.Phys. 36 (1995) 6073-6105
Also available as arXiv:<A HREF = "http://xxx.lanl.gov/abs/q-alg/9503002">q-alg/9503002</A>.

It was an attempt to completely describe the algebraic structure of 
the n-category nCob, where:

<ul>
<li>
 objects are 0d manifolds,
</li><li>
 1-morphisms are 1d manifolds with boundary, 
</li><li>
 2-morphisms are 2d manifolds with corners,
</li><li>
 3-morphisms are 3d manifolds with corners,
</li><li>
 ...
</li>
</ul>

and so on up to dimension n.  Unfortunately, at the time we
proposed it, little was known about n-categories above n = 3.
For a more recent take on these ideas, see:

11) Eugenia Cheng and Nick Gurski, Towards an n-category of 
cobordisms, Theory and Applications of Categories 18 (2007),
274-302.  Available at 
<a href = "http://www.tac.mta.ca/tac/volumes/18/10/18-10abs.html">http://www.tac.mta.ca/tac/volumes/18/10/18-10abs.html</a>

Lurie and Hopkins use a new trick: they redefine nCob to be a 
special sort of \infty -category.  The idea is to use 
diffeomorphisms and homotopies between these as morphisms above 
dimension n.  This gives an \infty -category version of nCob, 
where:

<ul>
<li>
 objects are 0-dimensional manifolds,
</li><li>
 1-morphisms are 1-dimensional manifolds with boundary,
</li><li>
 2-morphisms are 2-dimensional manifolds with corners,
</li><li>
 3-morphisms are 3-dimensional manifolds with corners,
</li><li>
...
</li><li>
 n-morphisms are n-dimensional manifolds with corners,
</li><li>
 (n+1)-morphisms are diffeomorphisms,
</li><li>
 (n+2)-morphisms are homotopies between diffeomorphisms,
</li><li>
 (n+3)-morphisms are homotopies between homotopies,
</li><li>
...
</li>
</ul>

and so on for ever!

Since everything here is invertible above dimension n, this is 
called an "(\infty ,n)-category".  

This sounds worse than an n-category, but it's okay for small n.  In
particular, (\infty ,1)-categories are pretty well understood by now.
There are a bunch of different approaches, with scary names like
"topological categories", "simplicial categories",
"A_{\infty } categories", "Segal categories",
"complete Segal spaces", and "quasicategories".
Luckily, all these approaches are known to be equivalent - see
"<A HREF = "week245.html">week245</A>" for some good
introductory material by Julie Bergner and Andre Joyal.  Joyal is now
writing a book on this stuff.

Lurie is a real expert on (\infty ,1)-categories.  In fact, 
starting as a grad student, he wrote a mammoth tome generalizing 
topos theory from categories to (\infty ,1)-categories:

12) Jacob Lurie, Higher topos theory, available as 
<a href = "http://arxiv.org/abs/math/0608040">arXiv:math/0608040</a>.

I'm sure Freed would suspect him of taking category-enhancing
substances: his category number is infinite, and this book is
619 pages long!  Then he went on to apply this stuff to algebraic
geometry... and the world is still reeling.  I was happy to 
discover that he's a nice guy, enthusiastic and friendly - not
the terrifying fiend I expected.  

Anyway, Lurie and Hopkins have worked out the precise structure 
of the (\infty ,1)-category version of 1Cob, and also the 
(\infty ,2)-category version of 2Cob.  Unfortunately this work
is not yet written up.  But, they use results from this paper:

13) Soren Galatius, Ib Madsen, Ulrike Tillmann, Michael Weiss,
The homotopy type of the cobordism category, available as 
<a href = "http://arxiv.org/abs/math/0605249">arXiv:math/0605249</a>.

And, Ulrike Tillmann gave a talk about this paper!  It computes
the "nerve" of the (\infty ,1)-category where:

<ul>
<li>
 objects are (n-1)-dimensional manifolds,
</li><li>
 1-morphisms are n-dimensional manifolds with boundary,
</li><li>
 2-morphisms are diffeomorphisms,
</li><li>
 3-morphisms are homotopies between diffeomorphisms,
</li><li>
 4-morphisms are homotopies between homotopies,
</li><li>
...
</li>
</ul>

The "nerve" is a trick for turning any sort of
\infty -category into a space, or simplicial set.  (See item J of
"<A HREF = "week117.html">week117</A>" for the nerve of a
plain old category.  This should give you the general idea.)

In her talk, she went further and computed the nerve of 
the (\infty ,k)-category where:

<ul>
<li>
 objects are (n-k)-dimensional manifolds,
</li><li>
 1-morphisms are (n-k+1)-dimensional manifolds with boundary,
</li><li>
 2-morphisms are (n-k+2)-dimensional manifolds with corners,
</li><li>
...
</li><li>
 k-morphisms are n-dimensional manifolds with corners,
</li><li>
 (k+1)-morphisms are diffeomorphisms,
</li><li>
 (k+2)-morphisms are homotopies between diffeomorphisms,
</li><li>
 (k+3)-morphisms are homotopies between homotopies,
</li><li>
...
</li>
</ul>

This is also joint work with the same coauthors, but it
seems not to be written up yet, except for k = 1, where it's
proved in the above paper.  The cool thing about the new work
is that it uses an idea familiar from higher category theory - a 
k-simplicial space - to give a rigorous description of the 
nerve of the above (\infty ,k)-category!  Indeed, Tillmann
told me she thinks of k-simplicial spaces as just a convenient
way of dealing with higher categories.

Stephan Stolz's talk also involved cobordism n-categories,
but I'll say more about that later.

Ralph Cohen and Dennis Sullivan both gave talks on string 
topology - a trick for studying a space by studying collections
of loops in that space, and relating this to ideas from string 
theory.  

String topology started when Chas and Sullivan took the ideas
of string theory and applied them in a somewhat ethereal form to
strings propagating in any manifold.  

In full-fledged string theory, one of the main tools is "conformal
field theory".  In a CFT, if you have a state of n strings, and a 
Riemann surface going from n strings to m strings, you get a state 
of m strings.   

A good way to get CFTs is to consider strings propagating on some
manifold or other.  Of course the manifold needs some sort of 
geometry, like a Riemannian metric, for your strings to know how 
to propagate.

But Chas and Sullivan figured out what you can do if the spacetime 
is a bare manifold, without any metric.  Basically, you just 
need to stick the word "homology" in front of everything!  This
makes everything sufficiently floppy.

So, instead of considering actual loops in a manifold M, 
which form a space LM, they took the homology of LM and got
a vector space or abelian group H(LM).  Then, for each homology 
class C on the moduli space of Riemann surfaces that go from n 
circles to m circles, they got an operation with n inputs and m
outputs:

Z(C): H(LM)^{\otimes n} \to  H(LM)^{\otimes m}

All these operations fit together into a slight generalization of 
an operad, called a "PROP".

If you don't remember what an "operad" is, give yourself
twenty lashes with a wet noodle and review "<A HREF =
"week220.html">week220</A>".  Suitably punished, you can then
enjoy this:

14) Ralph Cohen and Alexander Voronov, Notes on string topology,
available as <a href =
"http://arxiv.org/abs/math/0503625">arXiv:math/0503625</a>.

Both PROPs and operads are defined near the beginning here.
PROPs and operads are gadgets for describing operations with 
any number of inputs.  Operads can only handle operations with 
one output.  PROPS can handle operations with any number of outputs.

To see a more geometrical treatment of string topology, the way
it looked before the operadchiks got ahold of it, try the original 
paper by Chas and Sullivan:

15) Moira Chas and Dennis Sullivan, String topology, available as <a
href = "http://arxiv.org/abs/math/9911159">arXiv:math/9911159</a>.

Sullivan talked about some recent refinements of string topology
which deal with the fact that the moduli space of Riemann surfaces 
has a "boundary", so it doesn't have a closed "top-dimensional
homology class".  

Cohen's talk described some cool relations between string topology and
symplectic geometry!  In physics we use symplectic manifolds to
describe the space of states - the so-called "phase space" -
of a classical system.  So, if you have a loop in a symplectic
manifold, it can describe a periodic orbit of some classical system.
In particular, if we pick a periodic time-dependent Hamiltonian for
this system, a loop will be a solution of Hamilton's equations iff
it's a critical point for the "action".

But, we can also imagine letting loops move in the direction of 
decreasing action, following the "gradient flow".  They'll trace 
out 2d surfaces which we can think of as string world-sheets!
This is just what string topology studies, but now we can get 
"Morse theory" into the game: this studies a space (here LM)
by looking at critical points of a function on this space, and
its gradient flow.

So, we get a nice interaction between periodic orbits in phase
space, and the string topology of that space, and Morse theory!
For more, try this:

16) Ralph Cohen, The Floer homotopy type of the cotangent bundle,
available as <a href =
"http://arxiv.org/abs/math/0702852">arXiv:math/0702852</a>.

Next, let me say a bit about Stephan Stolz's talk.  He spoke
on his work with Peter Teichner, which is a very ambitious 
attempt to bring quantum field theory right into the heart of
algebraic topology.

I discussed this in "<A HREF = "week197.html">week197</A>".
I said they were working on a wonderful analogy between quantum field
theories and different flavors of cohomology.  It's been published
since then:

17) Stephan Stolz and Peter Teichner, What is an elliptic object? 
Available at <a href = "http://math.berkeley.edu/~teichner/papers.html">http://math.berkeley.edu/~teichner/papers.html</a>

Back then, the analogy looked like this:

\begin{verbatim}
1-dimensional supersymmetric QFTs            complex K-theory
2-dimensional supersymmetric conformal QFTs  elliptic cohomology
\end{verbatim}
    

When I saw this, I tried to guess a generalization to higher
dimensions.

There's an obvious guess for the right-hand column, since there's 
something called the "chromatic filtration", which is - very 
roughly - a list of cohomology theories.   Complex K-theory 
is the 1st entry on this list, and elliptic cohomology is the 2nd!  
(For a lot more details, see "<A HREF = "week149.html">week149</A>" and "<A HREF = "week150.html">week150</A>".)

There's also an obvious guess for the left-hand column:
n-dimensional supersymmetric QFTs of some sort!

The problem is the word "conformal" in the second row.  In
2 dimensions, a conformal structure is a way of making 
spacetime look locally like the complex plane.  This is great, 
because elliptic cohomology has a lot to do with complex 
analysis - or more precisely, elliptic curves and modular forms.  
But, it's not clear how one should generalize this to higher 
dimensions!

Luckily, thanks to a subsequent conversation with Witten, 
Stolz and Teichner realized that the partition function of a 
2d supersymmetric QFT gives a modular form even if the QFT is
not invariant under conformal transformations.  This means we 
can remove the word "conformal" from the second row!  For more
details, try this:

18) Stephan Stolz and Peter Teichner, Super symmetric field 
theories and integral modular forms, preliminary version
available at <a href = "http://math.berkeley.edu/~teichner/papers.html">http://math.berkeley.edu/~teichner/papers.html</a>

They've also gone back and added a 0th row to their chart. 
It's always wise to start counting at zero!  Now the chart 
looks much nicer:

\begin{verbatim}
0-dimensional supersymmetric QFTs            deRham cohomology
1-dimensional supersymmetric QFTs            complex K-theory
2-dimensional supersymmetric QFTs            elliptic cohomology
\end{verbatim}
    

Yes, good old deRham cohomology is the 0th entry in the 
"chromatic filtration"!  It's the least scary sort of cohomology
theory, at least for physicists.  They get scarier as we move
down the chart.  

Quantum field theory also gets scarier as we move down the chart -
the infinities that plague quantum field theory tend to get worse  
in higher dimensions of spacetime.  So, while we can dream about 
extensions of this chart, there's already plenty to handle here.

The most audacious idea in Stolz and Teichner's work is to take 
a manifold X and study the set of all n-dimensional QFT's 
"parametrized by X".  

For X a point, such a thing is just an ordinary n-dimensional 
QFT.  Roughly speaking, this is a gadget Z that assigns:

<ul>
<li>
 a Hilbert space Z(S) to any (n-1)-dimensional Riemannian 
 manifold S; 
</li><li>
 a linear operator Z(M): Z(S) \to  Z(S') to any n-dimensional 
 Riemannian manifold M going from S to S'. 
</li>
</ul>
If you're a mathematician, you may know that M is really a
"cobordism" from S to S', written M: S \to  S'.  And if you're 
really cool, you'll know that cobordisms form a symmetric monoidal 
category nCob, and that Z should be a symmetric monoidal 
functor.  

If you're a physicist, you'll know that S stands for "space"
and "M" stands for "spacetime".  All the stuff I'm
describing should remind you of the definition of a "TQFT",
except now our spaces and spacetimes have Riemannian metrics, because
we're doing honest QFTs, not topological ones.

Given a spacetime M, we try to compute the operator Z(M) as 
a path integral; for example, an integral over all maps

f: M \to  T

where f is a "field" taking values in a "target
space" T.

If this seems too scary, take n = 1.  Then we've got a 
1-dimensional quantum field theory, so we can take our 
spacetime M to be an interval.   Then f is just a path in some 
space T.  In this case the path integral is really an integral 
over all paths a particle could trace out in T.  So, 1-dimensional 
quantum field theory is just ordinary quantum mechanics!

There are a lot of subtleties I'm skipping over here, both on
the math and physics sides.   But never mind - the really cool 
part is this generalization:

Roughly speaking, an n-dimensional QFT "parametrized by X" 
assigns:

<ul>
<li>
 a Hilbert space Z(S) to any (n-1)-dimensional Riemannian 
 manifold S \emph{equipped with a map} g: S \to  X; 

</li><li>
 a linear operator Z(M): Z(S) \to  Z(S') to any n-dimensional 
 Riemannian cobordism M: S \to  S' \emph{equipped with a map} 
g: M \to  X.
</li>
</ul>

If you're a mathematician, you may see we've switched to using 
cobordisms "over X".  It's a straightforward generalization.

But what does it mean physically?  Here the path integral picture
is helpful.  Now we're doing a path integral over all fields

f: M \to  T \times  X

where we demand that the second component of this function is

g: M \to  X

For example, if we've got a 1d QFT, we're letting a particle 
roam over T \times  X, but demanding that its X coordinates follow 
a specific path g.  

So, we're doing a \emph{constrained} path integral!

In heaven, everything physicists do can be made mathematically 
rigorous.  Up there, knowing how to do these constrained path 
integrals would tell us how to do unconstrained path integrals: 
we'd just integrate over all choices of the path g.  So, a QFT 
parametrized by X would automatically give us an ordinary QFT.

Now, an ordinary QFT is just a QFT parametrized by a point!  
So, if we use QFT(X) to mean the set of n-dimensional QFTs 
parametrized by X, we'd have a map 

QFT(X) \to  QFT(point)

This is called "pushing forward to a point".  

More generally, we could hope that any map 

F: X \to  X'

gives a "pushforward" map

F_{<sub>*}</sub>: QFT(X) \to  QFT(X')

Let's see if this makes any sense.   In fact, I've been 
overlooking some important issues.  An example will shed light 
on this.  

Consider a 0-dimensional QFT parametrized by some manifold X.  
Let's call it Z.  What is Z like, concretely?

For starters, notice that the only (-1)-dimensional manifold
is the empty set.  A 0-dimensional manifold "going from the 
empty set to the empty set" is just a set of points.  Also, 
while I didn't mention it earlier, all manifolds in this game
must be \emph{compact}.  So, this set of points must be finite.

If you now take the definition I wrote down and use that
"symmetric monoidal functor" baloney, you'll see Z assigns a 
\emph{number} to any finite set of points mapped into X.  Furthermore, 
this assignment must be multiplicative.  So, it's enough to know 
a number for each point in X.   In short, our QFT is just a 
function:

Z: X \to  C

Now suppose we map X to a point:

F: X \to  point

What should the pushforward 

F_{<sub>*}</sub>: QFT(X) \to  QFT(point)

do to the function Z?

There's an obvious guess: we should \emph{integrate} this function 
on X to get a number - that is, a function on a point.  Indeed, 
that's what "path integration" should reduce to in this 
pathetically simple case: plain old integration!

Alas, there's no good way to integrate a function over X unless 
this manifold comes equipped with a measure.  But, if X is 
compact, oriented and p-dimensional, we can integrate a \emph{p-form}
over X. 

More generally, if we have a bundle

F: X \to  X' 

with compact d-dimensional fibers, we can take a p-form on X
and integrate it over the fibers to get a (p-d)-form on X'.
This is how you "push forward" differential forms.  

So, pushing forward is a bit subtler than I led you to believe at
first.  We should really talk about n-dimensional QFTs "of degree
p" parametrized by X.  Let's call the set of these

QFT^{p}(X)

I won't define them, but for n = 0 they're just p-forms on X.   
Anyway: if we have a bundle

F: X \to  X'

with compact d-dimensional fibers, we can hope there's a
pushforward map

F_{<sub>*}</sub>: QFT^{p}(X) \to  QFT^{p-d}(X')

There should also be a pullback map

F^{*}: QFT^{p}(X') \to  QFT^{p}(X)

This is a lot less tricky, and I'll let you figure out how
it works.

I should warn you, I've been glossing over lots of important
aspects of this work - like the role played by n-categories,
and the role played by supersymmetry.  Supersymmetry doesn't 
matter much for the broad conceptual picture I've been sketching.  
But, we need it for this analogy to work:

\begin{verbatim}
0-dimensional supersymmetric QFTs            deRham cohomology
1-dimensional supersymmetric QFTs            complex K-theory
2-dimensional supersymmetric QFTs            elliptic cohomology
\end{verbatim}
    
The idea is to impose an equivalence relation on supersymmetric 
QFTs, called "concordance", and try to show:

<ul>
<li>
 The set of concordance classes of degree-p 0d supersymmetric 
 QFTs parametrized by X is the pth de Rham cohomology group of X.
</li><li>
 The set of concordance classes of degree-p 1d supersymmetric 
 QFTs parametrized by X is the pth K-theory group of X.
 
</li><li>
 The set of concordance classes of degree-p 2d supersymmetric 
 QFTs parametrized by X is the pth elliptic cohomology group of X.
</ul>

So far people have done this in the 0d and 1d cases.  The 2d 
case is a major project, because it pushes the limits of what 
people can do with quantum field theory. 

Why did I spend so much time talking about pushforwards of
QFTs?   Well, it's very important for defining invariants like 
the "fundamental class" of an oriented manifold, or the "A-hat 
genus" of a spin manifold, or the "Witten index" of a string 
manifold.

Here's how it goes, very roughly.  Suppose X is a compact 
Riemannian manifold.  Then the simplest n-dimensional QFT 
parametrized by X is the one where we take the target space T 
(mentioned a while back) to be just a point!  

This parametrized QFT is called the "nonlinear \sigma  model",
for stupid historic reasons.  All the fun happens when we push 
this QFT forwards to a point.  Then we integrate over all the 
maps g: M \to  X.  The result - usually called the "partition 
function" of the nonlinear \sigma  model - should be an interesting
invariant of X.

In the case n = 1, this trick gives the "A-hat genus" of X, 
but it only works when X is a spin manifold: we need this to
define the 1d supersymmetric nonlinear \sigma -model.

In the case n = 2, this trick gives the "Witten genus" of X,
but it only works when X is a string manifold: we need this to
define the 2d supersymmetric nonlinear \sigma -model.  

For more on the n = 1 case, see:

19) Henning Hohnhold, Peter Teichner and Stephan Stolz, From 
minimal geodesics to super symmetric field theories.  In
memory of Raoul Bott.  Available at 
<a href = "http://math.berkeley.edu/~teichner/papers.html">http://math.berkeley.edu/~teichner/papers.html</a>

For the n = 2 case, see the papers I already listed.

(I'm confused about the case n = 0, for reasons having to do
with the "degree" I mentioned earlier.)  

Finally: the cool part, which I haven't even mentioned, is 
that we really need to describe n-dimensional QFTs using an 
\emph{n-category} of cobordisms - not just a mere 1-category, as
I sloppily said above.  

This first gets exciting when we hit n = 2: you'll see a bunch 
of stuff about 2-categories (or technically, "bicategories") 
in the old Stolz-Teichner paper "What is an elliptic object",
listed above.

In short: we're starting to see a unified picture where we
study spaces by letting particles, strings, and their
n-dimensional cousins roam around in these spaces.
There are lots of slight variants: string topology, the 
Stolz-Teichner picture, and of course good old-fashioned
topological quantum field theory.  All of them have a lot
to do with n-categories.

There's a lot more to say about all this... but luckily, there 
should be a proceedings of this conference, where you can read 
more.  My own talk is here:

20) John Baez, Higher gauge theory and elliptic cohomology,
<a href = "http://math.ucr.edu/home/baez/abel/">http://math.ucr.edu/home/baez/abel/</a>

It, too, is about studying spaces by letting strings roam
around inside them! 

But instead of summarizing my own talk, I want to say a
bit about the other side of the symposium - the motivic 
cohomology side!

I'll only summarize a few basic definitions.  I got these from
the talks by Fabian Morel and Vladimir Voevodsky, and I want
to write them down before I forget!  For more, try these:

21) Fabian Morel and Vladimir Voevodsky, A^{1}-homotopy theory
of schemes, September 1998.  Available at 
<a href = "http://citeseer.ist.psu.edu/morel98suphomotopy.html">http://citeseer.ist.psu.edu/morel98suphomotopy.html</a>

22) Vladimir Voevodsky (notes by Charles Weibel), 
Voevodsky's Seattle Lectures: K-theory and motivic 
cohomology.  Available at <a href = "http://citeseer.ist.psu.edu/249068.html">http://citeseer.ist.psu.edu/249068.html</a>

Okay:

A^{1}-homotopy theory is an attempt to do homotopy theory for 
algebraic geometry.  In algebraic geometry we often work over
a fixed field k, and the goal here is to create a category 
which contains smooth algebraic varieties over k as objects, 
but also other more general spaces, providing a sufficiently 
flexible category in which to do homotopy theory.

One of the simplest smooth algebraic varieties over k is the
"affine line" A^{1}.  The algebraic functions on
this line are just polynomials in one variable with coefficients in k.
In A^{1}-homotopy theory, we want to set up a context where we
can use the affine line A^{1} to parametrize homotopies, much
as we use the unit interval [0,1] in ordinary homotopy theory.

For this, people start by looking at Sm(k), the category of
smooth algebraic varieties over k.  Then, they consider 
the category of "simplicial presheaves" on Sm(k).  

A simplicial presheaf on Sm(k) is just a functor

F: Sm(k)^{op} \to  SimpSet

where SimpSet is the category of simplicial sets (see item C of
"<A HREF = "week115.html">week115</A>") We think of F as
specifying some sort of space by telling us for each smooth algebraic
variety X the simplicial set F(X) of all maps into this space.

To make this kind of abstract space work nicely, F(X) should
depend "locally" on X.  For this, we insist that given a cover 
of a variety X by varieties U_{i}, guys in F(X) are the same as
guys in F(U_{i}) that agree on the intersections 

U_{i} &cap; U_{j}.

Here "cover" means "cover in the Nisnevich
topology" - that is, an &eacute;tale cover such that every point being
covered is the image of a point in the cover for which the covering
map induces an isomorphism of residue fields.

If you've come this far, you may not be scared to hear that the
Nisnevich topology is really a "Grothendieck topology" on Sm(k),
and I'm really demanding that F be a "sheaf" with respect to this
topology.

So, the kind of "space" we're studying is a simplicial sheaf on 
the category of smooth varieties over k with its Nisnevich 
topology.  We call these category of these guys Space(k)

Just saying this already makes me feel smart.  Just think how
smart I'd feel if I knew why the Nisnevich topology was better
than the good old etale topology!

Anyway, to do homotopy theory with these simplicial sheaves, we 
need to make Space(k) into a "model category".  I should have 
explained model categories in some previous Week, but I've never 
gotten around to it, and right now is not the time.  So, I'll 
just say one key thing.

The \emph{most} important thing about a model category is that 
it's equipped with a collection of morphisms that act 
like homotopy equivalences.  They're called "weak equivalences".

Already in ordinary topology, these weak equivalences are a
slight generalization of homotopy equivalences.  They're
actually the same as homotopy equivalences when the spaces
involved are nice; they're designed to work better for nasty 
spaces.

In A^{1} homotopy theory, the weak equivalences are generated
by two kinds of morphisms:

<ul>
<li>
the projection maps X \times  A^{1} \to  X
</li><li>
the maps C(U) \to  X coming from covers U of X.
</ul>

Here X is any space in Space(k), and C(U) is the "Cech nerve"
of the cover U.

This framework seems like a really cool blend of algebraic 
geometry and homotopy theory.  But, to do homology theory
in a good way we need to go a bit further, and introduce "motives".

However, I'm tired, and I bet you are too!  Motives are a big
idea, and it doesn't make sense to start talking about them now.
So, some other day....

\par\noindent\rule{\textwidth}{0.4pt}
\textbf{Addendum:} 
For more discussion, go to the
<a href = "http://golem.ph.utexas.edu/category/2007/08/this_weeks_finds_in_mathematic_16.html">\emph{n}-Category Caf&eacute;</a>.


\par\noindent\rule{\textwidth}{0.4pt}
<em>And if a bird can speak, who once was a dinosaur, and a dog
can dream, should it be implausible that a man might supervise the
construction of light?</em> - King Crimson

</p>\par\noindent\rule{\textwidth}{0.4pt}
% </A>
% </A>
% </A>
