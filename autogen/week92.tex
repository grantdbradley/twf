
% </A>
% </A>
% </A>
\week{October 17, 1996}



% <A NAME = "tale">
I'm sure most of you have lost interest in my "tale of n-categories",
because it takes a fair amount of work to keep up with all the abstract
concepts involved.  However, we are now at a point where we can have
some fun with what we've got, even if you haven't really followed all
the previous stuff.  So what follows is a rambling tour through monads,
adjunctions, the 4-color theorem and the large-N limit of SU(N) gauge
theory....

Okay, so in "<A HREF = "week89.html">week89</A>" we defined a gadget called a "monad".  Using
the string diagrams we talked about, you can think of a monad as
involving a process like this:



\begin{verbatim}

                     \               /
                      \             /
                      s\          s/
                        \         /
                         \       /
                          \     /
                           \   /
                            \ /
                             |M               
                             |
                             |
                             |
                             |
                             |
                            s|
                             |


\end{verbatim}
    
which we read downwards as describing the "fusion" of two copies of something 
called s into one copy of the same thing s.  The fusion process itself is 
called M.

I can hear you wonder, what exactly \emph{is} this thing s?  What 
\emph{is} this 
process M?  Well, I gave the technical answer in "<A HREF = "week89.html">week89</A>" - but the 
point is that n-category theory is deliberately designed to be so 
general that it covers pretty much anything you could want!  For example, 
s could be the set of real numbers and M could be multiplication of real 
numbers, which is a function from s \times  s to s.  Or we could be doing 
topology in the plane, in which case the picture above stands for exactly 
what it looks like: two lines merging to form one line!  These and many 
other situations are analogous, and the formalism allows us to treat them 
all at once.  Here I will not review all the rules of the game.  If you 
just play along and trust me everything will be all right.  If you
don't trust me, go back and check the definitions.  

Let me turn to the axioms for a monad.  In addition to the multiplication
M we want to have a "multiplicative identity", I, looking like this:
           

\begin{verbatim}

                           I
                           |
                           |
                           |
                           |
                           |s

\end{verbatim}
    
Here nothing is coming in, and a copy of s is going out.  Because ordinary 
multiplication has 1x = x and x1 = x for all x, we want the following axioms 
to hold:



\begin{verbatim}

                                /              |
                               /               |
                             s/                |s
                   I         /                 |  
                    \       /                  | 
                     \     /                   |
                      \   /                    | 
                       \ /                     |
                        |M               =     | 
                        |                      |
                        |                      |
                        |                      |
                        |                      |
                        |                      | 
                       s|                      |
                        |                      | 

   
\end{verbatim}
    
and



\begin{verbatim}

                \                              |
                 \                             |
                  \                            |s
                  s\         I                 |  
                    \       /                  | 
                     \     /                   |
                      \   /                    | 
                       \ /                     |
                        |M               =     | 
                        |                      |
                        |                      |
                        |                      |
                        |                      |
                        |                      | 
                       s|                      |
                        |                      | 


\end{verbatim}
    
Also, since ordinary multiplication has (xy)z = x(yz), we want
the following associativity law to hold, too:



\begin{verbatim}

             \      /        /        \        \      /
              \    /        /          \        \    /
              s\  /s      s/           s\       s\  /s
                \/        /              \        \/
                M\       /                \       /M 
                  \     /                  \     /
                  s\   /                    \   /s
                    \ /                      \ /
                     |M                       |M
                     |                        |
                     |            =           |
                     |                        |
                     |                        |
                     |                        |
                    s|                       s|
                     |                        |


\end{verbatim}
    
These rules are a translation of the rules given in "<A HREF = "week89.html">week89</A>" into
string diagram form.  

If you are a physicist, you can think of these diagrams as being funny
Feynman diagrams where you've got some kind of particle s and two
processes M and I.  M is a bit like what you'd call a "cubic
self-interaction", where two particles combine to form a third.  These
interactions show up in simple textbook theories like the "\phi ^{3} theory"
and, more importantly, in nonabelian gauge field theories like quantum
chromodynamics, where the gauge bosons have cubic self-interactions.  On 
the other hand, I is a bit like what you'd usually call a "source" or an 
"external potential", some sort of field imposed from outside that can 
create particles of type s.  You shouldn't take the analogy with Feynman
diagrams too seriously yet, because the context we're working in is so
general, and the most interesting physics theories don't correspond to
monads but to more elaborate setups.  However, we could flesh out the
analogy to make it very precise and accurate if we wanted, and this is
especially important in topological quantum field theory.  More later
about that.

Now in "<A HREF = "week83.html">week83</A>" I discussed a different sort of gadget, called an
"adjunction".   Here you have two guys x and x*, and two
processes  U and C called the "unit" and "counit", which look like
this:


\begin{verbatim}

                    U
                   / \
                  /   \
                x/     \x*
                /       \

\end{verbatim}
    
and


\begin{verbatim}

                \       /
               x*\     /x
                  \   / 
                   \ /
                    C


\end{verbatim}
    
They satisfy the following axioms:


\begin{verbatim}

                        |       |
             U         x|      x|
            / \         |       |
           /   \        |       |
          /     \       |       |
         |     x*\     /   =    | 
         |        \   /         |
         |         \ /          |
        x|          C           |
         |                      |



         |                      |
       x*|          U         x*|
         |         / \          |
         |        /   \         |
         |       /     \        |
          \    x/       |  =    | 
           \   /        |       |
            \ /         |       |
             C        x*|       |
                        |       |


\end{verbatim}
    
Physically, we can think of x* as the antiparticle of x, and then U is
the process of creation of a particle-antiparticle pair, while C is the
process of annihilation.  The axioms just say that for a particle or
antiparticle to "double back in time" by means of these processes isn't
really different than for it to march obediently along forwards. 
Mathematically, one nice example of an adjunction involves a vector
space x and its dual vector space x*.  This is really the same example,
since if the behavior of a particle under symmetry transformations is
described by some group representation, its antiparticle is described by
the dual representation.  For more details on the math, see "<A HREF = "week83.html">week83</A>".

Now, let's see how to get a monad from an adjunction!  We need to get
s, M, and I from x, x*, U, and C.  To do this, we first define s to be xx*.
Then define M to be

                     

\begin{verbatim}

                      \ \       / /
                      x\ \x*  x/ /x*
                        \ \   / /
                         \ \ / /
                          \ C /
                           \ /
                           | |
                           | |
                          x| |x*
                           | |


\end{verbatim}
    
Again, to really understand the rules of the game you need to learn
a bit about string diagrams and 2-categories, but the basic idea is
supposed to be simple: we can get two xx*'s to turn into one
xx* by letting an x* and x annihilate each other!  

Finally, we define I to be
                     

\begin{verbatim}

                            U
                           / \
                           | |
                           | |
                           | |
                          x| |x*
                           | |


\end{verbatim}
    
In other words, an xx* can be created out of nothing since it's
a "particle/antiparticle pair".  

Now one can check that all the axioms for a monad hold.  You
really need to know a bit about 2-categories to do it carefully,
but basically you just let yourself deform the pictures, in part
with the help of the axioms for an adjunction, which let you
straighten out curves that "double back in time."  So for example, 
we can prove the identity law


\begin{verbatim}

                                 / /              | |
                                / /               | |
                     U        x/ /x*             x| |x*
                    /\        / /                 | |
                   x\ \x*    / /                  | |
                     \ \    / /                   | |
                      \ \  / /                    | |
                       \ \/ /                     | |
                        |C |               =      | |
                        |  |                      | |
                        |  |                      | |
                        |  |                      | |
                        |  |                      | |
                        |  |                      | |
                       x|  |x*                    | |
                        |  |                      | |


\end{verbatim}
    
by canceling the U and the C on the left using one of the
axioms for an adjunction.   Similarly, associativity holds
because the following two pictures are topologically the same:



\begin{verbatim}

           x\ \x* x/ /x*  / /      \ \   x\ \x* x/ /x*
             \ \  / /    / /        \ \    \ \  / /
              \ \/ /    / /          \ \    \ \/ /
               \ C/   x/ /x*         x\ \x*  \ C/
                \ \   / /              \ \   / /
                 \ \ / /                \ \ / /
                  \ C /                  \ C /
                   | |                    | |
                   | |                    | |
                   | |            =       | |
                   | |                    | |
                   | |                    | |
                   | |                    | |
                  x| |x*                 x| |x*
                   | |                    | |

\end{verbatim}
    
Whew!  Drawing these is tough work.  

Now, as I said, an example of an adjunction is a vector space x and
its dual x*.  What monad do we get in this case?  Well, the vector
space x tensored with x* is just the vector space of linear transformations
of x, so that's our monad in this case.  In less high-brow terms,
we've proven that matrices form an algebra when you define matrix
multiplication in the usual way!  In particular, the above picture
serves as a diagrammatic proof that matrix multiplication is associative.

Of course, people didn't invent all this fancy-looking (but actually
very basic) stuff just to deal with matrix multiplication!  Or did
they?  Well, actually, Penrose \emph{did} invent a diagrammatic notation
for tensors which is just a slightly souped-up version of the above
stuff.  You can find it in:

1) Applications of negative dimensional tensors, by Roger Penrose, in 
Combinatorial Mathematics and its Applications, ed. D. J. A. Welsh, 
Academic Press, 1971. 

But most of the work on this sort of thing has been aimed at 
applications of other sorts.  

Now let me drift over to a related subject, the large-N limit of
SU(N) gauge theory.  Quantum chromodynamics, or QCD, is an SU(N) gauge theory
with N = 3, but it turns out that things simplify a lot in the 
limit as N \to  \infty , and one gets some nice qualitative insight into
the strong force by considering this simplified theory.  One can
even treat the number 3 as a small perturbation around the number
\infty  and get some decent answers!   A good introduction to this 
appears in Coleman's delightful book, essential reading for anyone learning
particle physics:

2) Sidney Coleman, Aspects of Symmetry, Cambridge University Press,
Cambrdige, 1989.  

Check out section 8.3.1, entitled "the double line representation
and the dominance of planar graphs".  Coleman considers Yang-Mills
theories, like QCD, but many of the same ideas apply to other gauge
theories.   

The idea is that if we start out studying the Feynman diagrams for a gauge 
field theory with gauge group SU(N), and see how much various diagrams
contribute to any process for large N, the diagrams that contribute the
most are those that can be drawn on a plane without any lines crossing.
Technically, the reason is that diagrams that can only be drawn on a 
surface of genus g grow like N^{2 - 2g} as N increases.  This number
2 - 2g is called the Euler characteristic and it's biggest when
your surface has no handles.  

Even better, in the N \to  \infty  limit we can think of the Feynman
diagrams using diagrams like the ones above.  For example, we can 
think of the cubic self-interaction in Yang-Mills theory as simply
matrix multiplication:


\begin{verbatim}

                      \ \       / /
                      x\ \x*  x/ /x*
                        \ \   / /
                         \ \ / /
                          \ C /
                           \ /
                           | |
                           | |
                          x| |x*
                           | |

\end{verbatim}
    
and the quartic self-interaction as something a wee bit fancier:


\begin{verbatim}

                      \ \       / /
                      x\ \x*  x/ /x*
                        \ \   / /
                         \ \ / /
                          \ C /
                           \ /
                           / \
                          / U \
                         / / \ \
                        / /   \ \
                      x/ /x*  x\ \x*
                      / /       \ \ 


\end{verbatim}
    
Apparently these ideas have spawned a whole field of physics called
"matrix models".

These ideas work not only for Yang-Mills theory but also for Chern-Simons 
theory, which is a topological quantum field theory: a theory that doesn't 
require any metric on spacetime to make sense.  Here they have been exploited
by Dror Bar-Natan to come up with a new formulation of the famous 4-color 
theorem:

3) Dror Bar-Natan, Lie algebras and the four color theorem, preprint
available as <A HREF = "http://xxx.lanl.gov/ps/q-alg/9606016">q-alg/9606016</A>.

As I explained in "<A HREF = "week8.html">week8</A>" and "<A HREF = "week22.html">week22</A>", there is a way to formulate
about the 4-color theorem as a statement about trivalent graphs.
In particular, Penrose invented a little recipe that lets us calculate an 
invariant of trivalent graphs, which is zero for some \emph{planar} graph
only if some corresponding map can't be 4-colored.  This recipe involves 
the vector cross product, or equivalently, the Lie algebra of the group SU(2).  
You can generalize it to work for SU(N).  And if you then consider the 
N \to  \infty  limit, you get the above stuff!  (The point is that
the above stuff also gives a rule for computing a number from any 
trivalent graph.)

Now as I said, in the N \to  \infty  limit all the nonplanar Feynman diagrams 
give negligible results compared to the planar ones.  So another way to state 
the 4-color theorem is this: if the SU(2) invariant of a trivalent graph
is zero, the SU(N) invariant is negligible in the N \to  \infty  limit.

This doesn't yet give a new proof of the 4-color theorem.  But it makes
it into sort of a \emph{physics} problem: a problem about the relation of
SU(2) Chern-Simons theory and the N \to  \infty  limit of Chern-Simons
theory.  

Now, the 4-color theorem is one of the two deep mysteries of 2-dimensional
topology - a subject too often considered trivial.  The other mystery is
the Andrews-Curtis conjecture, discussed in "<A HREF = "week23.html">week23</A>".  Often a problem
is hard or unsolvable until you get the right tools.  Topological quantum
field theory is a new tool in topology, so one could hope it'll shed
some light on these problems.  Bar-Natan's paper is a tantalizing piece
of evidence that maybe, just maybe, it will.

One can't really tell yet.   

Anyway, I don't really care much about the 4-color theorem per se.  
If I ever need to color a map I'll hire a cartographer.   It's the
connections between seemingly disparate subjects that I find interesting.
2-categories are a very abstract formalism developed to describe 
2-dimensional ways of glomming things together.  Starting from the study
of 2-categories, we very naturally get the notions of "monad" and "adjunction".
And before we know it, this leads us to some interesting questions
about 2-dimensional quantum field theory: for really, the dominance of
planar diagrams in the N \to  \infty  limit of gauge theory is saying that in 
this limit the theory becomes essentially a 2-dimensional field theory, 
in some funny sense.  And then, lo and behold, this turns out to be related to 
the 4-color theorem!  

By the way, I guess you all know that the 4-color theorem was proved 
using a computer, by breaking things down into lots of separate cases.  
(See "<A HREF = "week22.html">week22</A>" for references.)  Well, there's a new proof out, which also 
uses a computer, but is supposed to be simpler:

4) Neil Robertson, Daniel P. Sanders, Paul Seymour, and Robin Thomas,
A new proof of the four-colour theorem, Electronic Research Announcements
of the American Mathematical Society 2 (1996), 17-25.  Available at
<A HREF = "http://www.ams.org/journals/era/1996-02-01/">http://www.ams.org/journals/era/1996-02-01/</A>

I'm still hoping for the 2-page "physicist's proof" using 
path integrals.  <img src = "emoticons/tongue.gif" alt = ":-)"/>

<A HREF = "week99.html#tale">To continue reading the "Tale of
n-Categories", click here.</A>
For more on adjunctions and monoid objects, try "<A HREF =
"week173.html">week173</A>" and especially
"<A HREF = "week174.html">week174</A>".
\par\noindent\rule{\textwidth}{0.4pt}

% </A>
% </A>
% </A>
