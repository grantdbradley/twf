
% </A>
% </A>
% </A>
\week{February 11, 2007 }

The University of Toronto is an urban campus, rather grey and chilly 
at this time of year.  Nestled amid other buildings at the southern
edge of campus, the Fields Institute doesn't stand out.  

But inside, you'll find a spacious and peaceful atrium, with a
fireplace to keep you cozy.  A spiral staircase winds up three or four
stories.  Hanging from the ceiling far above is a 3d model of the
"<a href = "fields/pictures/Picture%20044.jpg">120-cell</a>": a beautiful 4-dimensional solid with 120 regular
dodecahedra as faces.

This is a tribute to the great geometer H. S. M. Coxeter, master of
polyhedra, who worked for 60 years at the University of Toronto after 
studying philosophy at Cambridge under Wittgenstein.  You'll also
find Coxeter's piano sitting at the base of the spiral staircase.  

<div align = center>
<img src = "fields/pictures/fields_piano.jpg">
</div>

It's out of tune, but resting on it there's a wonderful strange 
portrait of him playing the very same piano - at the age of three.
He looks a bit like the child Mozart.  And indeed, at the age of 12 
Coxeter composed an opera!

The Fields Institute specializes in having conferences, and it's
a great place for that.  A friendly and efficient staff, public 
workstations, wireless internet everywhere, a nice little cafe in 
the back, and the centerpiece: a large lecture room with 3 double 
blackboards.  Unfortunately the middle blackboard doesn't stay up - 
it's needed that repair for years, old-timers say.  But apart from 
that, everything is as close to mathematician's heaven as could be 
expected.

Eugenia Cheng, Peter May and I ran a workshop at the Fields
Institute from January 9th to 13th:

1) Higher Categories and Their Applications, 
<a href = "http://math.ucr.edu/home/baez/fields/">http://math.ucr.edu/home/baez/fields/</a>

You can see photos of people and abstracts of their talks
at this site.  You can also see PDF files of many of their 
talks - and even listen to talks!

The first day, Tuesday, was all about 2-categories and 3-categories - 
"lower category theory", you might say.   While some are 
eagerly sailing into the stratosphere of n-categories for 
general n, or even n = \infty , there's still a lot to 
understand for n = 2 and 3.   

For starters, Tom Leinster spoke about strict 2-categories versus weak
ones (also known as bicategories).  It's a famous fact - a
generalization of Mac Lane's coherence theorem - that every weak
2-category C is equivalent to a strict one st(C).  However, this is
true \emph{if} your notion of equivalence is suitably weak!  In
short, what we've got is an inclusion of weak 3-categories:

i: Strict2Cat \to  Weak2Cat

where 


\begin{verbatim}

Strict2Cat = [strict 2-categories, 
              strict 2-functors, 
              strict natural transformations,
              modifications]
\end{verbatim}
    
and 


\begin{verbatim}

Weak2Cat = [weak 2-categories,
            weak 2-functors,
            weak natural transformations,
            modifications]
\end{verbatim}
    
Every object in Weak2Cat is equivalent to one in the image of
this inclusion.  But, the inclusion is not itself an equivalence!

Steve Lack spoke about Gray-categories, also known as
"semistrict" 3-categories - a convenient middle ground
between the strict 3-categories and the weak ones (also known as
tricategories).

The idea here goes back to John Gray.  In the usual Cartesian 
product of categories, whenever we have a morphism

f: A \to  B

in the first category and a morphism

f ': A' \to  B'

in the second, we get a commuting square:
           

$$

           (f,1)
   (A,A') -------> (B,A')
     |               |
(1,g)|               |(1,g)
     |               |
     v               v
   (A,B') -------> (B,B')
           (f,1)
$$
    

in their Cartesian product.  The same is true for the Cartesian
product of 2-categories.  But in the "Gray" tensor product
of 2-categories, these squares commute only up to 2-isomorphism.  And,
we can use this weakening of the Cartesian product to weaken the
concept of strict 3-category, and obtain the concept of
"semistrict" 3-category, or "Gray-category".

Here's how.  A strict 3-category is a gizmo with:

<ul>
<li>
a bunch of objects,
</li>
<li>
for any pair of objects x,y, a 2-category hom(x,y), 
</li>
</ul>

and

<ul>
<li>
for any triple of objects x,y,z, a 2-functor 

o: hom(x,y) \times  hom(y,z) \to  hom(x,z)
</li>
</ul>

such that

<ul>
<li>
associativity and the unit laws hold.
</li>
</ul>

A semistrict 3-category is a gizmo with:

<ul>
<li>
a bunch of objects,
</li>
<li>
for any pair of objects x,y, a 2-category hom(x,y), 
</li>
</ul>

and

<ul>
<li>
for any triple of objects x,y,z, a 2-functor 

o: hom(x,y) \otimes  hom(y,z) \to  hom(x,z)
</li>
</ul>

where \otimes  is the Gray tensor product, such that

<ul>
<li>
associativity and the unit laws hold.
</li>
</ul>

The slight difference is very important.  Not every weak 3-category
is equivalent to a strict one.  But, they're all equivalent to 
semistrict ones!   

There are, alas, some deficiencies in the semistrict world, which
Steve Lack has recently noted:

2) Steve Lack, Bicat is not triequivalent to Gray, available as <A
HREF = "http://xxx.lanl.gov/abs/math.CT/0612299">math.CT/0612299</A>.

To understand this, you may need a little warmup.  Given strict 
2-categories B and C there's a strict 2-category hom(B,C) such that 
strict 2-functors

A \times  B \to  C

are in natural 1-1 correspondence with strict 2-functors

A \to  hom(B,C)

Here's what hom(B,C) is like:


\begin{verbatim}

hom(B,C) has strict 2-functors from B to C as objects,
             strict natural transformations between these as morphisms,
             modifications between these as 2-morphisms.
\end{verbatim}
    
We can pose the same question with the Gray tensor product replacing
the Cartesian product.  Given 2-categories B and C there's a 
2-category [B,C] such that strict 2-functors

A \otimes  B \to  C

are in natural 1-1 correspondence with strict 2-functors

A \to  [B,C]

Here's what [B,C] is like:


\begin{verbatim}

[B,C] has strict 2-functors from B to C as objects,
          weak natural transformations between these as morphisms, 
          modifications between these as 2-morphisms.
\end{verbatim}
    
This suggests that we consider a 3-category intermediate between
Strict2Cat and Weak2Cat.   It's called Gray, and it goes like this:


\begin{verbatim}

Gray = [strict 2-categories, 
        strict 2-functors, 
        weak natural transformations,
        modifications]
\end{verbatim}
    
We have inclusions of weak 3-categories:

Strict2Cat \to  Gray \to  Weak2Cat

and Lack shows, not only that the second inclusion fails to be an
equivalence, but that there's \emph{no} equivalence between Gray and
Weak2Cat.

All this suggests that for some purposes we really need to face up
to weak 2-categories: the strict and semistrict setups aren't flexible
enough for every job.  The same is undoubtedly true at the 3-category
level - and that's where the next talk comes in!

In the next talk, Nick Gurski spoke about weak 3-categories.  He wrote
his thesis about these, and I'm starting to really wish he'd put his
thesis on the arXiv, so everyone can see how cool it is and learn more
about 3-categories.  But, I guess he wants to perfect it.  

In his talk, Nick not only explained the definition of weak
3-category, which is famously complicated - he did his best to
convince us that we could reinvent this definition ourselves if we
tried!  Then he went ahead and discussed various proofs that every
weak 3-category is equivalent to a semistrict one.

An interesting theme of all three talks was the idea of treating 
the "strictification" functor implicit in Mac Lane's coherence 
theorem:

st: Weak2Cat \to  Strict2Cat

as the left adjoint of the inclusion

i: Strict2Cat \to  Weak2Cat

where now we think of both Strict2Cat and Weak2Cat as mere 
1-categories.  You can read more about this idea here:

3) Miles Gould, Coherence for categorified operadic theories, 
available as <A HREF = "http://xxx.lanl.gov/abs/math.CT/0607423">math.CT/0607423</A>.

On Tuesday night, Mike Shulman gave an introduction to model 
categories, which are a tool developed by Quillen in the late
1960s to unify homotopy theory and homological algebra.
If you want to understand the basics of model categories, you 
should probably start by listening to his talk, and then read
this:

4) W. G. Dwyer and J. Spalinski, Homotopy theories and model 
categories, available at 
http://hopf.math.purdue.edu/Dwyer-Spalinski/theories.pdf

For more references, try "<A HREF =
"week170.html">week170</A>".

Here's the rough idea:

In homotopy theory we study topological spaces; in homological
algebra we study chain complexes.  But, in both cases we study
them in a funny way.  There's a category of topological spaces
and continuous maps, and there's a category of chain complexes 
and chain maps, but these categories are not everything that 
counts.  Normally, we say two objects in a category are "the
same" if they're isomorphic.  But in this case we often use a
weaker concept of equivalence!

In homotopy theory, we say a map between spaces

f: X \to  Y

is a "weak homotopy equivalence" if it induces isomorphisms on
homotopy groups:

\pi _{n}(f): \pi _{n}(X) \to  \pi _{n}(Y)

In homological algebra, we say a map between chain complexes

f: X \to  Y 

is a "quasi-isomorphism" if it induces isomorphisms on
homology groups:

H_{n}(f): H_{n}(X) \to  H_{n}(Y)

Model category theory formalizes this by speaking of a category C
equipped with a classes of morphisms called "weak
equivalences".  We can formally invert these and get a new
category Ho(C) where the weak equivalences are isomorphisms: this is
called the "homotopy category" or "derived
category" of our model category.  But this loses information, so
it's often good \emph{not} to do this.

In a model category, we also have a class of morphisms called
"fibrations", which you should imagine as being like fiber
bundles.  Dually, we have a class of morphisms called
"cofibrations", which you should imagine as well-behaved
inclusions, like the inclusion of the closed unit interval in the real
line - not the inclusion of the rationals into the real line.

Finally, the weak equivalences, fibrations and cofibrations 
satisfy some axioms that make them interlock in a powerful way.
These axioms are a bit mind-numbing at first glance, so I won't
list them.  But, they encapsulate a lot of wisdom about homotopy
theory and homological algebra!  

On Wednesday the talks were about n-categories and homotopy theory.  I
kicked them off with a general introduction to the "Homotopy
Hypothesis": Grothendieck's idea that homotopy theory was
secretly about \infty -groupoids - that is, \infty -categories where
all the j-morphisms have weak inverses.

5) John Baez, The homotopy hypothesis, 
<a href = "http://math.ucr.edu/home/baez/homotopy/">http://math.ucr.edu/home/baez/homotopy/</a>

Part of the idea is that if you hand me a space X, I can cook up 
an \infty -groupoid which has:


\begin{verbatim}

 points of X as objects,
 paths in X as morphisms,
 homotopies between paths in X as 2-morphisms,
 homotopies between homotopies between paths in X as 3-morphisms,
 etc....
\end{verbatim}
    

This is called the "fundamental \infty -groupoid of X".

But another part of the idea is that if you hand me a model 
category C, I can cook up an \infty -category which has:


\begin{verbatim}

 nice objects of C as objects,
 morphisms in C as morphisms,
 homotopies between morphisms in C as 2-morphisms,
 homotopies between homotopies between morphisms in C as 3-morphisms,
 etc....
\end{verbatim}
    
The basic idea here is simple: we're studying homotopies between
homotopies between... and so on.

(But, there's a little technicality - this "nice object"
business.  An object of C is "fibrant" if its unique
morphism from the initial object is a fibration, and
"cofibrant" if its unique morphism to terminal object is a
cofibration.  Objects with both properties are what I'm calling
"nice".  For example, in the category of topological spaces,
the "cell complexes" (made by gluing balls together) are
nice.  In the category of chain complexes, the "projective"
chain complexes are nice.  Only for these nice objects do homotopies
work as well as you'd hope.  Luckily, every object in C is weakly
equivalent to one of these nice ones.)

The interesting thing about the above \infty -category is that it's
an "(\infty ,1)-category", meaning that all its j-morphisms
are weakly invertible for j > 1.  For example, maps between spaces
aren't necessarily invertible, even up to homotopy - but homotopies
are always invertible.

We can define "(\infty ,k)-categories" for any k in the same
way, and we see that (\infty ,0)-categories are just
\infty -groupoids.  So, the Homotopy Hypothesis reveals the beginning
of what might be a very nice pattern.  Roughly:

$$
 Topological spaces, as studied in homotopy theory, are secretly 
 (\infty ,0)-categories.

 Model categories, as studied in homotopy theory, are secretly
 (\infty ,1)-categories.

 ????, as studied in homotopy theory (not yet?), are secretly
 (\infty ,2)-categories.

 Etcetera....
$$
    

Presumably the ???? should be filled in with something like 
"model 2-categories", with the primordial example being the 
2-category of model categories, just as the primordial example
of a model category is the category of spaces.  

But, there's only been a little study of this sort of
"meta-homotopy theory" so far.  For example:

6) Julie Bergner, Three models for the homotopy theory of homotopy
theories, available as <a href =
"http://arxiv.org/abs/math.AT/0504334">math.AT/0504334</a>.

After my talk, Simona Paoli spoke about her work on turning the
homotopy hypothesis from a dream into a reality:

7) Simona Paoli, Semistrict models of connected 3-types and 
Tamsamani's weak 3-groupoids, available as 
<a href = "http://arxiv.org/abs/math.AT/0607330">math.AT/0607330</a>.

8) Simona Paoli, Semistrict Tamsamani n-groupoids and connected 
n-types, available as 
<a href = "http://arxiv.org/abs/math.AT/0701655">math.AT/0701655</a>.

Eugenia Cheng then spent the afternoon leading us through another
approach:

9) Clemens Berger, A cellular nerve for higher categories, 
available at <a href = "http://citeseer.ist.psu.edu/383423.html">http://citeseer.ist.psu.edu/383423.html</a>

10) Denis-Charles Cisinski, Batanin higher groupoids and homotopy 
types, available as <a href = "http://arxiv.org/abs/math.AT/0604442">math.AT/0604442</a>.

I would love to explain this stuff, mainly as an excuse for 
learning it better!  But alas, I'm getting a bit tired and we're
only on the second day of the workshop... I must hurry on.

On Wednesday evening, Peter May spoke about some applications of
weak 2-categories that appear in his new book:

11) Peter May and J. Sigurdsson, Parametrized Homotopy Theory,
American Mathematical Society, 2006.

The rough idea is that we have a weak 2-category with:


\begin{verbatim}

 spaces as objects,
 spectra over X x Y as morphisms from X to Y,
 maps between spectra over X x Y as 2-morphisms.
\end{verbatim}
    

Lots of ideas from "parametrized" stable homotopy theory
are neatly encoded as results about this 2-category.

Thursday was all about (\infty ,1)-categories.  The first talk was by
Mike Shulman, who gave a nice intuitive treatment of Andr&eacute;
Joyal's approach to (\infty ,1)-categories.

In 1957, Daniel Kan figured out a nice way to describe
\infty -groupoids as simplicial sets with a certain property: now
they're called "Kan complexes".  They're very popular among
homotopy theorists.  You can read about them here:

12) Paul G. Goerss and J. F. Jardine, Simplicial Homotopy Theory, 
Birkh&auml;user, Basel, 1999.

Given this, it's not so surprising that we can describe
(\infty ,1)-categories as simplicial sets with some more general
property.  In fact this was done by Boardmann and Vogt back in 1973.
In the last decade or so, Joyal has developed an enormous body of
results about these (\infty ,1)-categories, which he calls
"quasicategories".  He is writing a book on the subject,
which is not quite done yet - but it's already enormously influenced
the state of higher category theory, and I expect it will continue to
do so.

Next Julie Bergner compared different approaches to (\infty ,1)-
categories.  I mentioned a while back that she's one of the few people
who has worked hard on "meta-homotopy theory".  That was
very much in evidence in her talk.

She began by describing a bunch of different definitions of 
(\infty ,1)-category.  But then she showed these definitions
weren't really so different!  For each definition, she constructed 
a model category of all (\infty ,1)-categories of that type.   
And then, she sketched the proof that all these model categories 
were "Quillen equivalent".  

For details, listen to her talk or try this paper:

13) Julie Bergner, A survey of (\infty , 1)-categories,
available as <a href = "http://arxiv.org/abs/math.AT/0610239">math.AT/0610239</a>.

In the afternoon, Andr&eacute; Joyal spoke about quasicategories.
I urge you to listen to his talk and also the minicourse he
later gave on this subject:

14) Andr&eacute; Joyal, Graduate course on basic aspects of 
quasicategories, <a href = "http://www.fields.utoronto.ca/audio/#crs-quasibasic">http://www.fields.utoronto.ca/audio/#crs-quasibasic</a>

I can't possibly summarize this stuff!  It basically amounts to
taking the whole of category theory and extending it to 
quasicategories.  

(Well, I guess I just summarized it, but....)

After Joyal's talk, Joshua Nichols-Barrer spoke about using
quasicategories as an approach to understanding "stacks",
which are like sheaves, only categorified.

In the evening, Kathryn Hess spoke about some work she's doing
with Steve Lack, on parallel transport in bundles of bicategories.
Sounds like physics, but they came to the subject from a completely
different motivation!

Finally, Dorette Pronk spoke about weak 2-categories and weak
3-categories of fractions.  The notion of a "calculus of
fractions" goes back at least to the work of Gabriel and Zisman
in 1967:

15) P. Gabriel and M. Zisman, Categories of Fractions and Homotopy 
Theory, Springer-Verlag, Berlin, 1967.

Say you have a category and you want to throw in formal inverses to 
some morphisms.  Well, you can do it!  But in general, the morphisms 
in the resulting category will be arbitrarily long "zig-zag" 
diagrams in your original category, like this:


$$

X_{1} ---> X_{2} <--- X_{3} ---> X_{4} <--- X_{5} ---> X_{6}
$$
    
The arrows pointing backwards are the ones you threw in formal 
inverses for.  

This is a nuisance!  But luckily, in nice cases, you only need to use
zig-zags of length two.  This is what a "calculus of
fractions" achieves.  A classic example is when you start with a
model category C, and you throw in formal inverses for the weak
equivalences to get the "homotopy category" Ho(C).

Dorette Pronk has been looking at how all this generalizes when
you have a weak 2-category or weak 3-category and you throw in 
\emph{weak} inverses to some morphisms.  This has some interesting 
applications to stacks:

16) Dorette A. Pronk, Etendues and stacks as bicategories of 
fractions, Compositio Mathematica, 102 (1996), 243-303.  Also
available at <a href = "http://www.numdam.org/numdam-bin/recherche?h=nc\text{\&} id=CM_1996__102_3_243_0">http://www.numdam.org/numdam-bin/recherche?h=nc\text{\&} id=CM_1996__102_3_243_0</a>

Dorette's talk ended at 9 pm, and everyone went home and collapsed
after a hard day's work.  Actually not: a bunch of us went out and
partied!  One of the great things about working on n-categories is
the sense of camaraderie among the small crowd that does this.

Friday's talks were about higher gauge theory.  Since I've discussed
this many times here, I'll be terse.  Alissa Crans explained Lie 
2-groups and Lie 2-algebras, and then Danny Stevenson explained his 
work on connections, 2-connections and Schreier theory (see 
"<A HREF = "week223.html">week223</A>").  In the afternoon, Urs Schreiber described his ideas 
on higher-dimensional parallel transport and local trivializations,
with a little help from Toby Bartels.

Friday evening, we heard talks from Simon Willerton (on Hopf monads)
and Igor Bakovic (on 2-bundles).  Quite an evening!  Bakovic is an
impressive young Croatian fellow who seems to have taught himself
n-categories.  We were all horrified when it became clear he had over
30 pages of transparencies, but his talk was actually quite nice.  

And if you like higher-dimensional diagrams anywhere near as much 
as I do, you've got to take a look at Willerton's slides:

17) Simon Willerton, 
The diagrammatics of Hopf monads, 
<a href = "http://math.ucr.edu/home/baez/fields/willerton/">
http://math.ucr.edu/home/baez/fields/willerton/</a>

Again the talks ended at 9 pm.

Finally, on Saturday morning, spoke about Frobenius algebras and
their relation to Khovanov homology:

18) Aaron Lauda, Frobenius algebras, quantum topology and higher 
categories, available at 
<a href = "http://www.math.columbia.edu/~lauda/talks/Fields/">http://www.math.columbia.edu/~lauda/talks/Fields/</a>

Urs Schreiber then wrapped things up with a talk about the
quantization of strings from a higher category viewpoint.  You can get
a good feeling for this from his blog entries at the
<a href = "http://golem.ph.utexas.edu/category/">\emph{n}-Category Caf&eacute;</a>, which are all listed on <a href = "http://math.ucr.edu/home/baez/fields/index.html#urs">my webpage
for this workshop</a>.

Speaking of the \emph{n}-Category Caf&eacute; - after the workshop
ended, Bruce Bartlett interviewed Urs and me about this blog, which
we run together with David Corfield.  You can see the interview here:

19) John Baez and Urs Schreiber, Interview by Bruce Bartlett, 
<a href = "http://math.ucr.edu/home/baez/interview2.html">
http://math.ucr.edu/home/baez/interview2.html</a>  


\par\noindent\rule{\textwidth}{0.4pt}
\textbf{Addendum:} For more discussion, go to the <a href = "http://golem.ph.utexas.edu/category/2007/02/this_weeks_finds_in_mathematic_6.html">\emph{n}-Category
Caf&eacute;</a>.



\par\noindent\rule{\textwidth}{0.4pt}
% </A>
% </A>
% </A>
