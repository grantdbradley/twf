
% </A>
% </A>
% </A>
\week{January 1, 2010 }


Happy New Decade!  I hope you're doing well.  This week I'll say more
about rational homotopy theory, and why the difference between
equality and isomorphism is important for understanding the weather 
in space.  But first: electrical circuits!

But even before that... guess what this is a picture of:

<div align = "center">
<img border = "2" src = "http://math.ucr.edu/home/baez/linear_dunes.jpg">
</div>

Now, what about electrical circuits?

I've been thinking This Week's Finds has become a bit too far removed
from its roots in physics.  This problem started when I quit working
on quantum gravity and started focusing on n-categories.  Overall it's
been a big boost to my sanity.  But I don't want This Week's Finds to
be comprehensible only to an elite coterie of effete mathematicians -
the sort who eat simplicial presheaves for breakfast and burp up
monoidal bicategories.

So, in an effort to prevent This Week's Finds from drifting off into
the stratosphere of abstraction, I've decided to talk a bit about
electrical circuits.  Admittedly, these are less glamorous than
theories of quantum gravity.  But: they actually work!  And there is 
a lot of nice math involved.

I rarely dare predict what \emph{future} Week's Finds will discuss, because
I know from bitter experience that I change my mind.  But lately I've
started writing a new way: long stories with lots of episodes, which I
can dole out a bit at a time.  So I know that for at least a few Weeks
I'll talk about electrical circuits - and various related things.

<div align = "center">
<a href = "http://www.free-circuits.com/circuits/audio/135/10w-audio-amplifier-with-bass-boost">
<img border = "none" src = "electronics_circuit_diagram_10W_amplifier_with_bass_boost.gif">
% </a>
<font size = "-1">
<br/> 10 watt amplifier with bass boost </font>
</div>

I've been trying to understand electrical circuits using category
theory for a long time.  Indeed, Peter Selinger and I are very slowly
writing a paper on this subject.  The basic inspiration is that
electrical circuit diagrams look sort of like Feynman diagrams, flow
charts, and various other diagrams that have "inputs" and
"outputs".  I love diagrams like this!  All the kinds I've
met so far can be nicely formalized using category theory.  For an
explanation, try this:

1) John Baez and Mike Stay, Physics, topology, logic and computation:
a Rosetta Stone, to appear in New Structures in Physics, ed. Bob
Coecke.  Available at <a href = "http://arxiv.org/abs/arXiv:0903.0340">arXiv:0903.0340</a>.

And after I spent a while thinking about electrical circuits using
category theory, I realized that this perspective might shed light on
analogies between circuits and other systems.

For example: mechanical systems made from masses and springs!  

Indeed, whenever I teach linear differential equations, I like to
explain the basic equation describing a "damped harmonic
oscillator": for example, a rock hanging on a spring.

<div align = "center">
<img src = "damped_spring.gif">
</div>

Then I explain how the same equation describes the simplest circuit
made of a resistor, an inductor, and a capacitor - the so-called
"RLC circuit".  It's a nice easy example of how the same
math applies to superficially different but secretly isomorphic
problems!

Let me explain.  I hope this is a chance to help mathematicians review
their physics and ask questions about it over on the <i>n</i>-Category 
Caf&eacute;.

Let the height of a rock hanging on a spring be q(t) at time t, where
q(t) is negative when the end of the spring is down below its
equilibrium position.  Then making all sort of simplifying assumptions
and approximations, we have:

m q"(t) = - c q'(t) - k q(t) + F(t)

where:

<ul>
<li>
   m is the <a href = "http://en.wikipedia.org/wiki/Mass">mass</a> of the rock.
</li>
<li>
   c is the <a href = "http://en.wikipedia.org/wiki/Damping">damping 
   coefficient</a>, which describes the force due to
   friction: we're assuming this force is proportional to the rock's
   velocity, but points the other way.
</li>
<li>
   k is the <a href = "http://en.wikipedia.org/wiki/Hooke%27s_law">spring 
   constant</a>, which describes the force due to the
   spring: we're assuming this force is proportional to how much the
   spring is stretched from its equilibrium position, but points the
   other way.
</li>
<li>
   F(t) is the externally applied <a href =
   "http://en.wikipedia.org/wiki/Force">force</a>, e.g. if you push on
   the rock.  
</li> 
</ul>

This equation is just <a href =
"http://en.wikipedia.org/wiki/Newton%27s_laws_of_motion#Newton.27s_second_law">Newton's
second law</a>, force equals mass times acceleration.  The left side
of the equation is mass times acceleration; the right side is the
total force.

Now for the analogy.  Everything here is analogous to something in an
RLC circuit!  An <a href =
"http://en.wikipedia.org/wiki/RLC_circuit#Series_RLC_with_Th.C3.A9venin_power_source">RLC
circuit</a> has current flowing around a loop of wire with 4 gizmos on
it: a resistor, an inductor, a capacitor, and a voltage source - for
example, a battery.

<div align = "center">
<img src = "RLC_series_circuit.png">
</div>



I won't say much about these gizmos.  I just want to outline the
analogy.  The amount of current is analogous to the velocity of the
rock, so let's call it q'(t).  The resistor acts to slow the current
down, just as friction acts to slow down the rock.  The inductor is
analogous to the mass of the rock.  The capacitor is analogous to the
spring - but according to the usual conventions, a capacitor with a
big "capacitance" acts like a weak spring.  Finally, the
voltage source is analogous to the external force.

So, here's the equation that governs the RLC circuit:

L q"(t) = - R q'(t) - (1/C) q(t) + V(t)
where

<ul>
<li>
L is the <a href =
"http://en.wikipedia.org/wiki/Inductance">inductance</a> of the
<a href = "http://en.wikipedia.org/wiki/Inductor">inductor</a>.  

<div align = "center">
<img src = "electronics_inductor_symbol.png">
<br/>
<font size = "-1">
inductor
</font>
</div>

</li>
<li>
R is the <a href =
"http://en.wikipedia.org/wiki/Electrical_resistance">resistance</a> of the
<a href = "http://en.wikipedia.org/wiki/Resistor">resistor</a>.
<div align = "center">
<img src = "electronics_resistor_symbol.png">
<br/>
<font size = "-1">
resistor
</font>
</div>

</li>
<li>
C is the <a href =
"http://en.wikipedia.org/wiki/Capacitance">capacitance</a> of the
<a href = "http://en.wikipedia.org/wiki/Capacitor">capacitor</a>.
<div align = "center">
<img src = "electronics_capacitor_symbol.png">
<br/>
<font size = "-1">
capacitor
</font>
</div>

</li>
<li>
V is the <a href =
"http://en.wikipedia.org/wiki/Voltage">voltage</a> of the <a href = "http://en.wikipedia.org/wiki/Voltage_source">voltage source</a>.
<div align = "center">
<img src = "electronics_voltage_source_symbol.jpg">
<br/>
<font size = "-1">
voltage source
</font>
</div>
</li> </ul>

As you can see, the equation governing the RLC circuit is the same as
the one that governs a rock on a spring!  True, 1/C plays the role of
k, since a capacitor with a big capacitance acts like a spring with a
small spring constant.  But this is just a difference in conventions.
The systems are isomorphic!

We could have fun solving the above equation and pondering what the 
solutions mean, but that would be the class I teach.  Instead,
I want to explain how this famous analogy between mechanics and 
electronics is just one of many analogies.

When I started thinking seriously about electrical circuits and
category theory, I mentioned them my student Mike Stay, and he
reminded me of the "hydraulic analogy" where you think of an
electrical current flowing like water through pipes.  There's a 
decent introduction to this here:

2) Wikipedia, Hydraulic analogy, 
<a href = "http://en.wikipedia.org/wiki/Hydraulic_analogy">http://en.wikipedia.org/wiki/Hydraulic_analogy</a>
 
Apparently this analogy goes back to the early days when people were
struggling to understand electricity, before electrons had been
observed.  The famous electrical engineer Oliver Heaviside pooh-poohed
this analogy, calling it the "drain-pipe theory".  I think
he was making fun of William Henry Preece.  Preece was another
electrical engineer, who liked the hydraulic analogy and disliked
Heaviside's fancy math.  In his inaugural speech as president of the
Institution of Electrical Engineers in 1893, Preece proclaimed:

\begin{quote}
  True theory does not require the abstruse language of mathematics to 
  make it clear and to render it acceptable.   All that is solid and 
  substantial in science and usefully applied in practice, have been 
  made clear by relegating mathematic symbols to their proper store 
  place - the study.
\end{quote}

According to the judgement of history, Heaviside made more progress in
understanding electromagnetism than Preece.  But there's still a nice
analogy between electronics and hydraulics.

In this analogy, a pipe is like a wire.  Water is like electrical
charge.  The flow of water plays the role of "current".
Water pressure plays the role of "voltage".

A resistor is like a narrowed pipe:

<div align = "center">
<img src = "electronics_analogy_reduced_pipe_resistor.png">
</div>

An inductor is like a heavy turbine placed inside a pipe: this makes
the water tend to keep flowing at the same rate it's already flowing!
In other words, it provides a kind of "inertia", analogous
to the mass of our rock.  Finally, a capacitor is like a tank with
pipes coming in from both ends, and a rubber sheet dividing it in two
lengthwise:

<div align = "center">
<img src = "electronics_analogy_flexible_tank_capacitor.png">
</div>

When studying electrical circuits as a kid, I was shocked when I first
learned that capacitors <i>don't let the electrons through</i>.
Similarly, this gizmo doesn't let the water through.

Okay... by now you're probably wanting to have the analogies laid out
more precisely.  So that's what I'll do.  But I'll throw in one more!
I've been talking about the mechanics of a rock on a spring, where the
motion of the rock up and down is called \emph{translation}.  But we can
also study \emph{rotation} in mechanics.  And then we get these analogies:




% parser failed at source line 421
