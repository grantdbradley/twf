
% </A>
% </A>
% </A>
\week{December 29, 1999 }

Since this is the last Week of the millennium, I'll make sure to
pack it full of retrospectives and prognostications.  But I'd like 
to start with an update on something I discussed a while back.

By the way, please don't give me flak about how the millennium starts in
2001.  I use the CE or Common Era system, which starts counting at the
year zero, not the AD system, which starts at the year one because it
was invented in 526 CE by Dennis the Diminutive, long before the number 
zero caught on.  In my opinion, the real "millennium bug" is 
that anyone is still using the antiquated AD system!  

Anyway....

In "<A HREF = "week73.html">week73</A>", I mentioned a theory about why molecules important in
biology tend to come in a consistent chirality, or handedness.   For
example, there's lots of dextrose in nature - this being the right-handed 
form of the sugar sucrose - but not much of its left-handed counterpart,
levulose.  It's no surprise that \emph{one or the other} would dominate, but
you might guess that \emph{which one} was just an accident of history.  After
all, there's no fundamental difference between right and left, right?

Or is there?  Actually there is: the weak nuclear force distinguishes
between the two!  So some people theorized that very slight differences
in energy levels, due to the weak force, favor the formation of
left-handed amino acids and right-handed sugars - which is what we see
in nature.  

Recently people have found evidence for a somewhat different version of
this theory:

1) Robert F. Service, Does life's handedness come from within?, 
Science 286 (November 12, 1999), 1282-1283.

When radioactive atoms decay via the weak force, the electrons they
shoot off tend to have a left-handed spin.  Could this affect the
handedness of molecules or crystals that happen to be forming in the
vicinity?  Sodium chlorate is a chemical that can form both left-handed
and right-handed crystals, so researchers took some solutions of the 
stuff and let them crystallize while blasting them with electrons formed
by the decay of radioactive strontium.  Sure enough, this biased the
handedness of the crystals!  Blasting the stuff with right-handed
positrons favored formation of crystals of the opposite handedness.  
The strangest part was that the effect was even bigger than expected.

I still think the whole business is pretty iffy - after all, the flux of
radiation in this experiment was a lot bigger than what we normally see
on earth.  But it would sure be neat if the origin of chirality in
biology was related to the deeper mystery of chirality in particle physics.

Okay, now for a little retrospective.  Don't worry - I won't list the
top ten developments in mathematical physics of the last millennium!
Instead, I just want to recommend two papers.  First, an old paper by 
Poincare: 

2) Henri Poincare, The present and future of mathematical physics,
Bull. Amer. Math. Soc. 12 (1906), 240-260.  Reprinted as part of a
retrospective issue of the Bull. of the Amer. Math. Soc., 37 (2000), 
25-38, available 
at <A HREF = "http://www.ams.org/bull/">http://www.ams.org/bull/</A>

This article is based on a speech he gave in 1904.   After a
fascinating review of the development of mathematical physics, 
he makes some accurate predictions about quantum mechanics and 
special relativity - but closes on a conservative note: 

\begin{quote}
     In what direction we are going to expand we are unable to foresee.  
     Perhaps it is the kinetic theory of gases that will forge ahead 
     and serve as a model for the others.  In that case, the facts  
     that appeared simple to us at first will be nothing more than
     the resultants of a very large number of elementary facts which 
     the laws of probability alone would induce to work toward the 
     same end.  A physical law would then assume an entirely new aspect; 
     it would no longer be merely a differential equation, it would 
     assume the character of a statistical law. 

     Perhaps too we shall have to construct an entirely new mechanics, 
     which we can only just get a glimpse of, where, the inertia 
     increasing with the velocity, the velocity of light would be a 
     limit beyond which it would be impossible to go. The ordinary, 
     simpler mechanics would remain a first approximation since it 
     would be valid for velocities that are not too great, so that 
     the old dynamics would be found in the new.  We should have no 
     reason to regret that we believed in the older principles, and 
     indeed since the velocities that are too great for the old 
     formulas will always be exceptional, the safest thing to do in 
     practice would be to act as though we continued to believe in 
     them.  They are so useful that a place should be saved for them. 
     To wish to banish them altogether would be to deprive oneself of 
     a valuable weapon. I hasten to say, in closing, that we are not 
     yet at that pass, and that nothing proves as yet that they will 
     not come out of the fray victorious and intact.

\end{quote}
    
The same issue of the AMS Bulletin also has a lot of other interesting
papers and book reviews from the last century, by folks like Birkhoff, 
Einstein and Weyl.

The second paper I recommend is a new one by Rovelli:

3) Carlo Rovelli, The century of the incomplete revolution: searching 
for general relativistic quantum field theory, to appear in the
Journal of Mathematical Physics 2000 Special Issue, preprint available
as <A HREF = "http://xxx.lanl.gov/abs/hep-th/9910131">hep-th/9910131</A>.  

Let me just quote the abstract:

\begin{quote}
     In fundamental physics, this has been the century of quantum 
     mechanics and general relativity.  It has also been the century 
     of the long search for a conceptual framework capable of embracing 
     the astonishing features of the world that have been revealed by 
     these two ``first pieces of a conceptual revolution''.  I discuss 
     the general requirements on the mathematics and some specific 
     developments towards the construction of such a framework.  Examples 
     of covariant constructions of (simple) generally relativistic 
     quantum field theories have been obtained as topological quantum 
     field theories, in nonperturbative zero-dimensional string theory 
     and its higher dimensional generalizations, and as spin foam models. 
     A canonical construction of a general relativistic quantum field 
     theory is provided by loop quantum gravity.  Remarkably, all these 
     diverse approaches have turn out to be related, suggesting an 
     intriguing general picture of general relativistic quantum physics.

\end{quote}
    
Now for the prognostications.  Since we should never forget that the towering
abstractions of mathematical physics are ultimately tested by experiment, 
I'd like to talk about some interesting physics \emph{experiments} that are 
coming up in the next millennium.  These days more and more interesting
information about physics is coming from astronomy, so I'll concentrate
on work that lies on this interface.

In "<A HREF = "week80.html">week80</A>" I talked about how Gravity Probe B will try to detect an
effect of general relativity called "frame-dragging" caused by the
earth's rotation.   I also talked about how LIGO - the Laser 
Interferometric Gravitational Wave Observatory - will try to detect 
gravitational waves:

4) LIGO homepage, <A HREF = "http://www.ligo.caltech.edu/">http://www.ligo.caltech.edu/</A>  

If all works as planned, LIGO should be great for studying the final 
death spirals of binary black holes and/or neutron stars.  When it 
starts taking data sometime around 2002, it should be able to detect 
the final "chirp" of gravitational radiation produced a pair of 
inspiralling neutron stars in the Virgo Cluster, a cluster of galaxies 
about 15 megaparsecs away.  Such an event would distort the spacetime 
metric \emph{here} by only about 1 part in 10^{21}.   This is why LIGO needs 
to compare oscillations in the lengths of two arms of an interferometer, 
each 4 kilometers long, with an accuracy of 10^{-16} centimeters: about 
one hundred-millionth of the diameter of a hydrogen atom.  To do this 
will require some \emph{very} clever tricks to reduce noise.  

As the experiment continues, they intend to improve the sensitivity 
until it can detect distortions in the metric of only 1 part in 10^{22},
and second-generation detectors should get to 1 part in 10^{23}.  At 
that point, we should be able to detect neutron star "chirps" from a
distance of 200 megaparsecs.  Events of this sort should happen once or 
twice a year.  

Since it's crucial to rule out spurious signals, LIGO will have two 
detectors, one in Livingston, Louisiana and one in Hanford, Washington.
This should also allow us to tell where the gravitational waves are
coming from.  And there are other gravitational wave detection projects 
underway too!  France and Germany are collaborating on a laser 
interferometer called VIRGO, with arms 3 kilometers long, to be built 
in Cascina, Italy:

5) VIRGO homepage, <A HREF = "http://www.pi.infn.it/virgo/">http://www.pi.infn.it/virgo/</A>

Germany and Great Britain are collaborating on a 600-meter-long
one called GEO 600, to be built south of Hannover:

6) GEO 600 homepage, <A HREF = "http://www.geo600.uni-hannover.de/">http://www.geo600.uni-hannover.de/</A>

The Japanese are working on one called TAMA 300, which is a 300-
meter-long warmup for a planned kilometer-long interferometer: 

7) TAMA 300 homepage, <A HREF = "http://tamago.mtk.nao.ac.jp/">http://tamago.mtk.nao.ac.jp/</A>

In addition, the Brazilian GRAVITON project is building something 
called the Einstein Antenna, which uses mechanical resonance rather
than interferometry.  The basic principle goes back to Joseph Weber's 
original bar detectors, which tried to sense the vibrations of a 
2-meter-long aluminum cylinder induced by gravitational waves.  But 
the design involves lots of hot new technology: SQUIDS, buckyballs, 
and the like:

8) GRAVITON homepage, <A HREF = "http://www.das.inpe.br/graviton/project.html">http://www.das.inpe.br/graviton/project.html</A>

There are also other gravitational wave detectors being built...
but ultimately, the really best ones will probably be built in 
outer space.  There are two good reasons for this.  First, outer 
space is big: when you're trying to detect very small distortions 
of the geometry of spacetime, it helps to measure the distance
between quite distant points.  Second, outer space is free of 
seismic noise and most other sources of vibration.  This is why 
people are working on the LISA project - the Laser Interferometric 
Space Antenna:

9) European Space Agency's homepage on the LISA project, 
<A HREF = "http://www.estec.esa.nl/spdwww/future/html/lisa.htm">http://www.estec.esa.nl/spdwww/future/html/lisa.htm</A>

NASA's homepage on the LISA project: <A HREF ="http://lisa.jpl.nasa.gov/">http://lisa.jpl.nasa.gov/</A>

The idea is to orbit 3 satellites in an equilateral triangle with
sides 5 million kilometers long, and constantly measure the distance 
between them to an accuracy of a tenth of an angstrom - 10^{-11} 
meters - using laser interferometry.  (A modified version of the plan 
would use 6 satellites.)  The big distances would make it possible to 
detect gravitational waves with frequencies of .0001 to .1 hertz, much
lower than the frequencies for which the ground-based detectors are 
optimized.  The plan involves a really cool technical trick to keep 
the satellites from being pushed around by solar wind and the like: 
each satellite will have a free-falling metal cube floating inside it, 
and if the satellite gets pushed to one side relative to this mass, 
sensors will detect this and thrusters will push the satellite back 
on course.

I don't think LISA has been funded yet, but if all goes well, it
may fly within 10 years or so.  Eventually, a project called LISA 2 
might be sensitive enough to detect gravitational waves left over 
from the early universe - the gravitational analogue of the cosmic
microwave background radiation!   

The microwave background radiation tells us about the universe 
when it was roughly 10^5 years old, since that's when things 
cooled down enough for most of the hydrogen to stop being ionized, 
making it transparent to electromagnetic radiation.   In physics
jargon, that's when electromagnetic radiation "decoupled".  But 
the gravitational background radiation would tell us about the 
universe when it was roughly 10^{-38} seconds old, since that's
when gravitational radiation decoupled.  This figure could be way
off due to physics we don't understand yet, but anyway, we're 
talking about a window into the \emph{really} early universe.

Actually, Mark Kamionkowski of Caltech has theorized that the 
European Space Agency's "Planck" satellite may detect subtle  
hints of the gravitational background radiation through its tendency 
to polarize the microwave background radiation.  You probably heard 
how COBE, the Cosmic Background Explorer, detected slight anisotropies 
in the microwave background radiation.  Now people are going to redo 
this with much more precision: while COBE had an angular resolution 
of 7 degrees, Planck will have a resolution of 4 arcminutes.  They
hope to launch it in 2007:

10) Planck homepage, 
<A HREF = "http://astro.estec.esa.nl/SA-general/Projects/Planck/planck.html">http://astro.estec.esa.nl/SA-general/Projects/Planck/planck.html</A>

What else is coming up?  Well, gravity people should be happy about 
the new satellite-based X-ray telescopes, since these should be great 
for looking at black holes.  In July 1999, NASA launched one called 
"Chandra".  (This is the nickname of Subrahmanyan Chandrasekhar, who 
won the Nobel prize in 1983 for his work on stellar evolution, neutron 
stars, black holes, and closed-form solutions of general relativity.)
The first pictures from Chandra are already coming out - check out
this website:

11) Chandra homepage, <A HREF = "http://chandra.harvard.edu/">http://chandra.harvard.edu/</A>

On December 10th, the Europeans launched XMM, the "X-ray Multi-Mirror
Mission":

12) XMM homepage, <A HREF = "http://sci.esa.int/xmm/">http://sci.esa.int/xmm/</A>

This is a set of three X-ray telescopes that will have lower angular
resolution than Chandra, but 5-15 times more sensitivity.  It'll also
be able to study X-ray spectra, thanks to a diffraction grating that
spreads the X-rays out by wavelength.   And in January, the Japanese
plan to launch Astro-E, designed to look at shorter wavelength X-rays:

13) MIT's Astro-E homepage, <A HREF = "http://acis.mit.edu/syseng/astroe/xis_home.html">http://acis.mit.edu/syseng/astroe/xis_home.html</A>
 
Taken together, this new generation of X-ray telescopes should tell us
a lot about the dynamics of the rapidly changing accretion disks of
black holes, where infalling gas and dust spirals in and heats up to
the point of emitting X-rays.  They may also help us better understand 
the X-ray afterglow of \gamma  ray bursters.  As you probably have heard,
these rascals make ordinary supernovae look like wet firecrackers!
Some folks think they're caused when a supernova creates a black hole.
But nobody is sure.

Peering further into the future, here's a nice article about new
projects people are dreaming up to study physics using astronomy:

14) Robert Irion, Space becomes a physics lab, Science 286 (1999),
2060-2062.

In 2005 folks plan to launch GLAST, the Gamma-Ray Large Area Space 
Telescope, designed to study \gamma -ray bursters and the like, and
also the Alpha Magnetic Spectrometer, designed to search for 
antimatter in space.   But there are also a bunch of interesting
projects that are still basically just a twinkle in someone's eye....

For example: OWL, the Orbiting Wide-Angle Light Collector, a
pair of satellites that would trace the paths of super-high-energy 
cosmic rays through the earth's atmosphere.  As I explained in
"<A HREF = "week81.html">week81</A>", people have seen 
cosmic rays with ridiculously high
energies, like 320 Eev - the energy of a 1-kilogram rock moving
at 10 meters per second, all packed into one particle.  OWL would
orbit the earth, watch these things, and figure out where the
heck they're coming from.    

Or how about this: The Dark Matter Telescope!  This would use 
gravitational lensing to chart the "dark matter" which seems 
to account
for a good percentage of the mass in the universe - if, of course, dark
matter really exists.

15) Dark Matter Telescope homepage, <A HREF = "http://dmtelescope.org">
http://dmtelescope.org</A>

Anyway, there should be a lot of exciting experiments coming up.  But
as usual, the really exciting stuff will be the stuff we can't predict.




<p> <hr>

% </A>
% </A>
% </A>


% parser failed at source line 421
