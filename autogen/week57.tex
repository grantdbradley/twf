
% </A>
% </A>
% </A>
\week{July 3, 1995 }

This week I'll start by finishing up my introduction to
the following paper:

1) Lee Smolin, Linking topological quantum field theory and 
nonperturbative quantum gravity, available as <A HREF = "http://xxx.lanl.gov/abs/gr-qc/9505028">gr-qc/9505028</A>.

So: recall where we were.  Let me not repeat the details, but
simply note that we were playing around with quantum gravity
on a 4-dimensional spacetime, using the Ashtekar `new variables'
formalism, and we'd noticed that in the theory with nonzero cosmological 
constant \Lambda , there is an explicit solution of the theory, 
the `Chern-Simons' state.  Actually I hadn't really shown that this
state satisfies the key equation, the Wheeler-DeWitt equation, but if you
know the formulas it's easy to check. 

Now one might think that one solution isn't all that much, apart from
it being a whole lot better than none, which was the situation before
these discoveries.  However, as I began to explain last time, one can get
a whole slew of states if one takes as ones space S, not a closed 3-dimensional 
manifold (as we were doing at first) but a 3-manifold with boundary.  
This is where Lee Smolin starts.  He considers quantum gravity with
certain `self-dual boundary conditions' that are specially compatible with 
Chern-Simons theory.   

There is presumably an enormous space of states of quantum gravity 
satisfying these boundary conditions, although we don't really know what 
they look like.  Say we want to understand these states as much as possible. 
What do they look like?  Well, first of all, the loop representation gives us a 
nice picture of the `kinematical states' --- i.e., states not 
necessarily satisfying the diffeomorphism constraint or 
the Wheeler-DeWitt equation.  (This picture may be wrong, of course, 
but let me throw caution to the winds and just explain the picture.)  
Every kinematical state is a linear combination of `spin network states'.  
For more on spin networks, check out "<A HREF = "week55.html">week55</A>" and the references in there,
but let me remind you what spin networks look like in this case.

For simplicity and ease of visualization, you can pretend S is a ball, 
so its boundary is a sphere.  Think of a spin network state as a 
graph embedded in this ball, possibly with some edges ending on the 
the boundary, with all the edges labelled by spins j = 0,1/2,1,3/2, etc., and 
with the vertices labelled by some extra numbers that we won't worry 
about here.  Let's call the points where edges end on the 
boundary `punctures', because the idea is that they really poke through 
the boundary and keep on going.    Physically, these edges 
graph represent `flux tubes of area': if we measure the area of some surface
in this state (or at least a surface that doesn't intersect the vertices), 
the area is just the quantity


\begin{verbatim}

                      L^2  sqrt(j(j+1))
\end{verbatim}
    

summed over all edges that poke through the surface, where
L is the Planck length and j is the spin labelling that edge.
Gauge theories often have "flux tube" solutions when you quantize
them: for example, type II superconductors admit flux tubes of
the magnetic field, while superfluids admit flux tubes of
angular momentum (vortices).  The idea behind spin networks in quantum
gravity, physically speaking, is that gravity is a gauge field
which at the Planck scale is organized into branching flux tubes
of area.

But we want to understand, not the kinematical states in general, but
the actual physical states, which satisfy the diffeomorphism constraint
and the Wheeler-DeWitt equation. We can start by measuring everything we
measure by doing experiments right at the boundary of S.  More precisely, we 
can try to find a maximal set of commuting observables that `live on the 
boundary' in this sense.  For example, the area of any patch of S counts 
as one of these observables, and all these `surface patch area' 
observables commute.  If we measure all of them, we know everything 
there is to know about the area of all regions on the 
boundary of S.  Thanks to spin network technology, as described above,
specifying all their eigenvalues amounts to specifying the location of 
a bunch of punctures on the boundary of S, together with the spins 
labelling the edges ending there.

Now Chern-Simons theory gives an obvious candidate for the space of
physical states of quantum gravity for which these `surface patch area' 
observables have specified eigenvalues.  In fact, if you hand Chern-Simons 
theory a surface like the boundary of S, together with a bunch of punctures
labelled by spins, it gives you a FINITE-DIMENSIONAL state space. 
Let's not explain just now how it gives you this state space; let's simply
mumble that it gives you this space by virtue of being an `extended topological
quantum field theory.'  If you want, you can think of these states as being
the `relative states' I discussed in last week's Finds, but not all of them: 
only those for which the `surface patch area' observables have specified 
eigenvalues.   There is a wonderfully simple combinatorial recipe for 
describing all these states in terms of spin networks living in S, having
edges that end at the punctures, with the right spins at these ends.
 
Smolin's hypothesis is that this finite-dimensional space of states coming
from Chern-Simons theory \emph{is} the space of all physical states of quantum 
gravity on S that 1) satisfy the self-dual boundary conditions and 2) have 
the specified values of the surface patch area observables.  Now if this
hypothesis is true, it means we have a wonderfully simple description of
all the physical states on S satisfying the self-dual boundary conditions!

So why should such a wonderful thing be true?  I wish I knew!  In fact,
I'm busily trying to figure it out.  Smolin doesn't give any direct evidence
that it \emph{is} true, so it might not be.  But he does give some very interesting
indirect evidence, coming from thermodynamics.  

Thanks to work by Hawking, Bekenstein and others, there is a lot of evidence
that if one takes quantum gravity into account, the maximal entropy of any 
system contained in a region with surface area A should be proportional to A.
The basic idea is this.  For various reasons, one expects that the entropy 
of a black hole is proportional to the area of its event horizon.  For example, 
when you smash some black holes together it turns out that the total area 
of the event horizons goes up --- this is called the `second law of black 
hole thermodynamics'.  This and many more fancy thought experiments 
suggest that when you have some black holes around the right notion of 
entropy should include a term proportional to the total area of their 
event horizons.  Now suppose you had some other system which 
had even MORE entropy than this, but the same surface area.  Then you could 
dump in extra matter until it became a black hole, which would therefore have 
less entropy, violating the second law.  

This is a hand-waving argument, all right!  It's not the sort of thing
that would convince a mathematician.  But it does suggest an intriguing
connection between the vast literature on black hole thermodynamics and
the more mathematical problem of relating quantum gravity and Chern-Simons
theory.  

Now the maximum entropy of a system is proportional to the logarithm 
of the total number of states it can assume.  So if the `Bekenstein bound'
holds, the dimension of the space of states of a system contained in 
a region with surface area A is proportional to exp(A/c) for some constant
c (which should be about the Planck length squared).   Now the remarkable
thing about Smolin's hypothesis is that if it's true, this is what one
gets, because the dimension of the space given by Chern-Simons theory
does grow like this.  

There is another approach leading to this conclusion that the space
of states of a bounded region should have dimensional proportional to 
exp(A/c), called the 't Hooft-Susskind holographic hypothesis.  I was
going to bone up on this for This Week's Finds, but I have been too
busy!  It's getting late and I'm getting bleary-eyed, so I'll stop
here.  I will simply give the references to this `holographic hypothesis'; 
if anyone wants to explain it, please post to sci.physics.research --- 
I'd be immensely grateful.

2) G 't Hooft, Dimensional reduction in quantum gravity, preprint
available as <A HREF = "http://xxx.lanl.gov/abs/gr-qc/9310006">gr-qc/9310006</A>.

3) L. Susskind, The world as a hologram, to appear in the November
1995 special issue of Jour. Math. Phys. on diffeomorphism-invariant physics, 
preprint available as <A HREF = "http://xxx.lanl.gov/abs/hep-th/9409089">hep-th/9409089</A>.

L. Susskind, Strings, black holes and Lorentz contractions, preprint
available as <A HREF = "http://xxx.lanl.gov/abs/hep-th/9308139">hep-th/9308139</A>.

Note: in earlier Finds I referred to the October 1995 special issue of 
Jour. Math. Phys., but now I've heard it's coming out in November.
\par\noindent\rule{\textwidth}{0.4pt}

% </A>
% </A>
% </A>
