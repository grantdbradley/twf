
<TITLE> week90 </TITLE>

% </A>
% </A>
% </A>
\week{September 30, 1996 }

If you've been following This Week's Finds, you know that I'm in love
with symmetry.  Lately I've been making up for my misspent youth by
trying to learn more about simple Lie groups.  They are, roughly 
speaking, the basic building blocks of the symmetry groups of physics.    
In trying to learn about them, certain puzzles come up.  In July 
I asked Bertram Kostant about one that's been bugging me for years:
"Why does E8 exist?"  In a word, his answer was: "Triality!"  This was
incredibly exciting to me; it completely blew my mind.  But I should
start at the beginning....
In my youth, I found the classification of simple Lie groups to be
unintuitive and annoying.  I still do, but over the years I've
realized that suffering through this classification theorem is the
necessary entrance fee to a whole world of symmetry.  I gave a tour of 
this world in "<A HREF = "week62.html">week62</A>" - "<A HREF = 
"week63.html">week65</A>", but here I want to make everything as
simple as possible, so I won't assume you've read that stuff.  Experts
should jump directly to the end of this article and read backwards
until it becomes boring.
A Lie group is a group that can be given coordinates for which all the group
operations are infinitely differentiable.  A good example is the group
SO(n) of rotations in n-dimensional Euclidean space.  You can multiply
rotations by doing first one and then the other, or mathematically by
doing matrix multiplication.  Every rotation has an inverse, given
mathematically by the inverse matrix.  Since matrices are just bunches
of numbers, you can coordinatize SO(n), at least locally, and in terms
of these coordinates the operations of multiplication and taking inverses 
are infinitely differentiable, or "smooth", so SO(n) is a Lie group.
Using the magic of calculus, we can think of tangent vectors at the
identity element of SO(n) as "infinitesimal rotations".  So for
example, taking n = 3, let's start with the rotation by the angle t
about the z axis, given by the matrix:
\begin{verbatim}
cost  -sint  0
sint   cost  0
0      0     1
\end{verbatim}
    
Then we can differentiate this and set t = 0 to get an "infinitesimal
rotation about the z axis":
\begin{verbatim}
0    -1     0
1     0     0
0     0     0
\end{verbatim}
    
Let's call this J_z, since it's very related to angular momentum about
the z axis.  (Folks often throw in a factor of -i when they define
J_z in quantum mechanics, but let's not bother with that here.)
Similarly we have J_x and J_y.  Now rotations about different axes
don't commute, so these infinitesimal rotations don't either.  In
fact, we have
J_x J_y - J_y J_x = J_z,
J_y J_z - J_z J_y = J_x,
and
J_z J_x - J_x J_z = J_y.
If you have never done it, there are few things in life as rewarding
at this point as computing J_x and J_y for yourself and checking the
above "commutation relations".   
Folks usually write the "commutators" on the left hand side using
brackets, like this:
     [J_x,J_y] = J_z, 

  [J_y,J_z] = J_x, 

  [J_z,J_x] = J_y
These relations are lurking in the definition of quaternions
and also the vector cross product.  Quaternions and cross products are good for
understanding rotations in 3-dimensional space; they let us
describe infinitesimal
rotations and their failure to commute.  Here we are calling a spade a
spade and working directly with the algebra of infinitesimal
rotations, which folks call so(3).  (For related stuff, see "<A
HREF = "week5.html">week5</A>".)
Okay.  The point is, we can do this trick for any Lie group!  The
space of "infinitesimal group elements", or more precisely tangent
vectors at the identity element of a Lie group, is called the "Lie
algebra of the group".  It's a vector space whose dimension is the
dimension of the group, and it always has a bracket operation on it
satisfying certain axioms (listed in "<A HREF = "week3.html">week3</A>").
The classification of Lie groups can be reduced to the classification
of Lie algebras, because the Lie algebra almost determines the Lie
group.  More precisely, every Lie algebra is the Lie algebra of a
unique Lie group that is "simply connected" - i.e., one for which 
every loop in it can be continuously shrunk to a point.  People 
understand how to get from any Lie group to a simply connected one 
(called its "universal cover"), so if we understand simply connected 
Lie groups, we pretty much understand all Lie groups.   See "<A HREF
= "week61.html">week61</A>" 
for an instance of this philosophy.
Now classifying Lie algebras is just a matter of heavy-duty linear
algebra.  Let me explain what the "simple" Lie algebras are; you'll
have to take my word for it that understanding these is a big step
towards understanding all Lie algebras.
At one extreme in the world of Lie groups are the commutative, or
"abelian" Lie groups.  Here multiplication is commutative, so [x,y] =
0 for all x and y in the Lie algebra of the group.  At the other
extreme are the "semisimple" Lie groups.  Here every element in the
Lie algebra is of the form [x,y] for some x and y: roughly, if we
bracket the whole Lie algebra with itself, we get itself back again.
The semisimple Lie algebras turn out to be incredibly important in
physics, where they are the typical "gauge groups" of field theories.
The "simple" Lie algebras are the building blocks of the semisimple ones:
every semisimple Lie algebra can be broken down into pieces that are
simple.  (Technically, we say it's a "direct sum" of simple Lie algebras).
We say a Lie group is simple if its Lie algebra is simple.    
So: what are the simple Lie algebras?  They were classified, thanks to
some heroic work by Killing and Cartan, in the early part of the 20th
century.  To keep life simple (ahem) I'll only give the classification
of those simple Lie algebras whose corresponding Lie groups are
\emph{compact} - meaning roughly that they are finite in size.  (For
example, SO(n) is compact.)  It turns out that if we understand the
compact ones, we can understand the noncompact ones too.
So, here are the Lie algebras of the compact simple Lie groups!  There are
4 straightforward infinite families and 5 delightful and puzzling
exceptions.  The 4 infinite families are easy to understand and are
called "classical groups".  They are the workhorses of mathematics and
physics.  The other 5 are called "exceptional groups".  They have
always seemed very mysterious to me.  
The 4 infinite families are:
A_n: This is the Lie algebra of SU(n), the group of n x n complex
matrices that preserve lengths (i.e., are unitary) and have
determinant 1.   This Lie algebra is also called su(n).  
B_n: This is the Lie algebra of SO(2n+1), the group of (2n+1) x (2n+1)
real matrices that preserve lengths (i.e., are orthogonal) and have
determinant 1.  This Lie algebra is also called so(2n+1).
C_n: This is the Lie algebra of Sp(n), the group of n x n quaternionic
matrices that preserve lengths.  This Lie algebra is also called sp(n).
D_n: This is the Lie algebra of SO(2n), the group of 2n x 2n real matrices
that preserve lengths and have determinant 1.  This Lie algebra is also
called so(2n).
You may justly wonder why the heck they are called A_n, B_n, C_n, and
D_n, and why we separated out the even and odd cases of SO(n) as we
did!  This is explained in "<A HREF = "week64.html">week64</A>", and I don't 
want to worry about
it here.  Anyway, glossing over some nuances, we see that these guys
are all pretty much just groups of rotations in real, complex, and
quaternionic vector spaces.  
The 5 exceptions are as follows:
F4: A 52-dimensional Lie algebra.  
G2: A 14-dimensional Lie algebra. 
E6: A 78-dimensional Lie algebra.  
E7: A 133-dimensional Lie algebra.  
E8: A 248-dimensional Lie algebra.
Here I am being rather reticent about what these Lie algebras - or
the corresponding Lie groups, which go by the same names - actually
ARE!  The reason is that it's not so easy to explain.  One can
certainly describe the exceptional Lie groups as groups of matrices
with certain complicated properties, but often this is done in a way
that leaves one utterly puzzled as to the real reason why these
simple Lie groups exist.
Of course, the answer to "why" a mathematical object exists is a
matter of taste.  You may feel satisfied if you can easily construct
it from other objects you know and love, or you may feel satisfied
once it is so tightly woven into your overall scheme of things that
you can't imagine life without it.  
In any case, I have long been asking people why the exceptional Lie
groups exist, but without much luck.  Until recently I only felt happy
about one of them, the one called G2: it's the group of rotations of
the octonions!  The real numbers, complex numbers, quaternions and
octonions are the only "normed division algebras" - a property which
makes it easy to define rotation groups - but the octonions are
weirder than the other three because, unlike the others, they are not
associative.  (See "<A HREF = "week59.html">week59</A>" and "<A HREF
= "week61.html">week61</A>" for details.)  One might
expect a series of simple Lie groups coming from rotations in
octonionic vector spaces, like the other classical series... but there
isn't one!  The only simple Lie group like this is the group of
rotations of a ONE-dimensional octonionic vector space, G2.
(More precisely, we say that G2 is the group of automorphisms
of the octonions, that is, the linear transformations that preserve
the octonion product.  These all preserve lengths.)
The idea that the exceptional groups are all related to octonions
is sort of pleasing, because one might easily \emph{expect} that the
reals, complexes and quaternions give nice infinite series of
"classical" Lie groups, while the octonions, being much more bizarre,
give only 5 bizarre "exceptional" Lie groups.  Indeed, in "<A HREF = 
"week64.html">week64</A>"
I described how F4 and E6 are related to the octonions... but in a 
pretty complicated way!  As for E7 and E8, here until recently I had 
always been completely in the dark.  This is all the more irksome 
because the biggest, most mysterious exceptional Lie group of
all, E8, plays an important role in string theory!
Luckily, on Thursday July 11th I ran into Bertram Kostant, who had
been attending the previous workshop here at the Erwin Schroedinger
Institute.  As I described in "<A HREF = "week79.html">week79</A>", Kostant is one of the expert's
experts on group theory.  So I got up my nerve and asked him, "Why
does E8 exist?"  And he told me!  Best of all, he explained both E8
and F4 in terms of a principle that I knew was crucial for
understanding G2 and the octonions ... the principle of triality!
I sketched a description of triality in "<A HREF = "week61.html">week61</A>".  
Let me just
summarize the idea here.  One of the main way to understand Lie
algebras is to understand their "representations".  A representation
of a Lie algebra is simply a function from it to the space of nxn
matrices that preserves the bracket operation.  (The nxn matrices form
a Lie algebra with the commutator as the bracket operation.)  For
example, so(n) has a representation where we map each element to an
nxn matrix in the most utterly obvious way: each element IS an nxn
matrix, so don't do anything to it!  This is called the "vector"
representation, because this is how we do infinitesimal rotations to
vectors.  But so(n) also has representations called "spinor"
representations.  In physics, the vector representation describes
spin-1 particles, while the spinor representations describe spin-1/2 
particles.
Spinor representations work differently depending on whether the
dimension n is even or odd.  (This is one reason why people distinguish
the even and odd n case of so(n) in that classification of simple
Lie algebras above!)  When n is odd there is one spinor representation.  
That's why in ordinary 3-dimensional space there is just one kind of 
spinor to worry about, as you learn when you learn about spin-1/2 particles
in undergraduate quantum mechanics.  When n is even there are two different 
spinor representations, called the "left-handed" and "right-handed" 
spinor representations.  This shows up when you do quantum mechanics taking
special relativity - and 4-dimensional spacetime - into account.  For
example, the way neutrinos transform under rotations is described by
the left-handed spinor representation, while anti-neutrinos are described
by right-handed spinors. 
When n is even, both the spinor representations of so(n) are of 
dimension 2^{n/2 - 1}.  That is, they are functions from so(n) to the space of 
2^{n/2 - 1} x 2^{n/2 - 1} matrices.  Now something marvelous happens when 
n = 8.  Namely, 2^{n/2 - 1} = n, so the spinor representations are 
just as big as the vector representation.  This might lead one to hope 
that in some sense they are "the same" as the vector representation.  
This is actually true, but in a subtle way.... they are not "equivalent" 
representations in the standard sense of Lie algebra theory, but something 
sneakier is true.  
The Lie algebra so(8) has interesting symmetries!  It has a little 
symmetry group with 6 elements, the same as the symmetries of a
equilateral triangle, and using these 6 symmetries we can permute the 
vector, left-handed spinor, and right-handed spinor representations 
into each other however we please!  
For example, one of these symmetries switches the left-handed
and right-handed spinor representations, but leaves the vector 
representation alone.  Actually, this symmetry works in any even
dimension, not just dimension 8.  Its analogue in 4-dimensional spacetime
is called "parity", a symmetry that turns left-handed particles
into right-handed ones and vice versa.  The fact that there are
no right-handed neutrinos means that the laws of nature do not actually have
this symmetry... but it's still very important in math and physics.
What's special about dimension 8 is that there are symmetries switching
the vector representation and the spinor representations.  For example:
if we take an element x of so(8), apply the right symmetry of so(8) to turn it 
into another element of so(8), and then use the right-handed spinor
representation to it to turn it into a matrix, we get the same thing as if 
we just used the vector representation to turn x into a matrix.
Now so(8) is the Lie algebra of the Lie group SO(8), but SO(8) is
not "simply connected" in the sense defined above.  The simply connected
group whose Lie algebra is SO(n) is called Spin(n).  I gave an introduction
to these "spin groups" in "<A HREF = "week61.html">week61</A>", and I don't want to say much about
them here, except for this: the triality symmetries of so(8) do not
give symmetries of SO(8), but they do give symmetries of Spin(8).  
Experts say the group of outer automorphisms modulo inner automorphisms 
of SO(8) is S_3 (the group of permutations of 3 things).
Pretty sneaky, how a group of symmetries can have its own group of 
symmetries, no?  As we'll now see, this is what gives birth to 
G2, F4, E8, and the octonions.
To get G2 is pretty simple; we look at those elements of Spin(8) that
are fixed (i.e., unaffected) by all the triality symmetries, and these
form a subgroup, which is G2.   
For the rest, we need one more fact: there is a way to "multiply" a 
left-handed spinor and a right-handed spinor and get a vector.  This is true 
in all even dimensions, not just n = 8, so in particular it is familiar to
particle theorists who live in 4-dimensional spacetime.   As I noted, 
what happens to a neutrino when you rotate (or Lorentz transform) it
is described using left-handed spinors, while anti-neutrinos are described
by right-handed spinors.  Similarly, photons are described by vectors.  
So as far as \emph{rotational} properties go, one could think of a photon as
a bound state of a neutrino and an antineutrino.  This led Schroedinger 
(or someone) to propose at one point that photons were actually neutrino-
antineutrino pairs.  Subsequent experiments showed this theory has lots of 
problems, and nobody sane believes it any more.   Still, it's sort of
cute.
Now, in 8 dimensions, it shouldn't be surprising that we can also 
multiply a left-handed spinor and a vector to get a right-handed 
spinor, and so on.  The point is, you can just use triality to permute the
three representations whichever way you please... they are not really
all that different.  
So in particular, you can multiply two 8-dimensional vectors and get 
another vector.  And this gives us the octonions!  
Now how about F4 and E8?  This is the cool stuff Kostant told me about.
Here I will describe the Lie algebras, not the Lie groups.  
Let's call the right-handed and left-handed spinor representations
S+ and S-, respectively.  (Us left-handers are always getting shafted,
being "sinister" rather than "dextrous" and all that, so we get S-
rather than S+.)  And let's call the vector representation V.   And let's
be sloppy, the way people usually are, and also use these letters to stand 
for the 8-dimensional vector spaces on which so(8) acts as transformations.
Now let's form the direct sum of vector spaces
\begin{verbatim}
                  so(8)  +  S+  +  S-  +  V
\end{verbatim}
    
A vector in this vector space is just a list consisting of a 
guy in so(8), a guy in S+, a guy in S-, and a guy in V.  The
dimension of this vector space is therefore
\begin{verbatim}
                    28   +  8   +  8   +  8  =  52
\end{verbatim}
    
since it takes n(n-1)/2 numbers to describe a rotation in n
dimensions.  Hey!  Look!  52 is the dimension of F4!  So maybe
this thing is F4.  
Yes, it is!   Here's how it works.  To make this gadget
into a Lie algebra - which turns out to be F4 - we need a way
to take the "bracket" of any two elements in it.  We already
know how to take the bracket of two guys in so(8), so that's
no problem.  Since so(8) acts as transformations of S+ and
S- and V, we also know how to multiply a guy in so(8) by
one of these other guys.  We also know how to multiply a
guy in S+ by a guy in S- to get a guy in V, and so on.  Finally,
we can multiply two guys in V to get a guy in so(8) as follows:
two vectors determine an infinitesimal rotation which starts
rotating the first vector in the direction of the second.  
(More technically, we say that so(8) is isomorphic to the 
second exterior power of V, so we can multiply two guys in
V to get a guy in so(8) using the wedge product.)  Using
triality, we can equally well multiply two guys in S+ to
get a guy in so(8), or multiply two guys in S- to get a guy
in so(8).  
So taking all these multiplication operations together, we
can cook up a way to take the bracket of any two guys in 
so(8) + S+ + S- + V and get another such guy.  If you do it
right - I've been pretty vague, so I leave it to you to fill 
in the details - you can get this bracket to satisfy the Lie 
algebra axioms, and you get F4!
Emboldened with our success, we now look at the vector space
\begin{verbatim}
           so(8) + so(8) + end(S+) + end(S-) + end(V)
\end{verbatim}
    
Here end(S+) is the space of all linear transformations of
the vector space S+, so if you like, it's just the space
of 8x8 matrices.  Similarly for end(S-) and end(V).  Now the
dimension of this space is
\begin{verbatim}
            28   +  28    +   64   +   64    +   64   =  248
\end{verbatim}
    
Hey!  This is just the dimension of E8!  Maybe this space is E8!
Yes indeed.  Again, you can cook up a bracket operation on 
this space using all the stuff we've got.  Here's the basic
idea.  end(S+), end(S-), and end(V) are already Lie algebras,
where the bracket of two guys x and y is just the commutator
[x,y] = xy - yx, where we multiply using matrix multiplication.
Since so(8) has a representation as linear transformations of
V, it has two representations on end(V), corresponding to left
and right matrix multiplication; glomming these two together
we get a representation of so(8) + so(8) on end(V).  Similarly
we have representations of so(8) + so(8) on end(S+) and end(S-).
Putting all this stuff together we get a Lie algebra, if we
do it right - and it's E8.  At least that's what Kostant said;
I haven't checked it.
So now we see, at least roughly, how triality gives birth
to the octonions, G2, F4, and E8.  That leaves E8's "little
brothers" E6 and E7.  These are contained in E8 as Lie subalgebras,
but apart from that I don't know any especially beautiful way 
to get ahold of them, except for the way to get E6 from 3x3
matrices of octonions, which I described in "<A HREF = 
"week64.html">week64</A>".  
For some references to this stuff, try:
1) Claude C. Chevalley, The algebraic theory of spinors, 
Columbia University Press, New York, 1954.
2) F. Reese Harvey, Spinors and calibrations, Perspectives in 
Mathematics, 9, Academic Press, Inc., Boston, MA, 1990.
3) Ian R. Porteous, Topological geometry, 2nd ed., Cambridge University 
Press, Cambridge, 1981.
4) Ian R. Porteous, Clifford algebras and the classical groups,
Cambridge University Press, Cambridge, 1995.  
5) Hans Freudenthal and H. de Vries, Linear Lie groups, Academic Press, 
New York, 1969.
6) Alex J. Feingold, Igor B. Frenkel, and John F. X. Rees,
Spinor construction of vertex operator algebras, triality, and
E_8^{(1)}, Contemp. Math. 121, AMS, Providence Rhode Island.
I haven't looked at all these books lately, and the only 
source I \emph{know} contains the above construction of E8 from
triality is the last one, by Feingold, Frenkel, and Rees.
Now let me allow myself to get a bit more technical.
I am still not entirely happy, by any means, because what I'd
really like would be a simple explanation of why these exceptional
simple Lie algebras arise from triality, \emph{and no others}.  In other
words, I'd like a classification of the simple Lie algebras
that proceeded not by the usual exhaustive (and exhausting) case-by-case
study of Dynkin diagrams, but by some less combinatorial and more
"synthetic" approach.  For example, it would be nice to really see
a good explanation of how the reals, the complexes, the quaternions 
and octonions each give rise to a family of simple Lie algebras, and 
one gets \emph{all} of them this way.
On the other hand, don't think I'm knocking the Dynkin diagram stuff.
As I explained in "<A HREF = "week62.html">week62</A>" - 
"<A HREF = "week64.html">week64</A>", what's really fundamental
to the Dynkin diagram approach seems to be the not the Lie
algebras themselves but their root lattices.  Taking lattices
as fundamental to the study of symmetry \emph{does} seem to be a good
idea, since it gets you to not just the simple Lie algebras
described above, but also the "Kac-Moody algebras" so important
in string theory and other forms of 2-dimensional physics, as well
as marvelous things like the Leech lattice and the Monster group.
The Dynkin diagram approach also makes it clear \emph{why} triality 
exists: symmetries of Dynkin diagrams always give outer automorphisms
of the corresponding Lie algebras, and as you examine the Dynkin
diagrams of D_n, you get
\begin{verbatim}
 
              o  
                
                     D_2 or so(4) 
                
              o  
 
 
              o  
             /   
            o        D_3 or so(6)
             \   
              o  
 
 
              o  
             /   
        o---o        D_4 or so(8)
             \   
              o  
 
 
              o  
             /   
    o---o---o        D_6 or so(10)
             \   
              o  
 
\end{verbatim}
    
and you can just \emph{see} how when you get to so(8) there is that
amazing triality symmetry, flashing briefly into being before reverting
to the boring old duality symmetry which only interchanges the 
left-handed and right-handed spinor representations, corresponding
to the two dots on the far right of the Dynkin diagram.  (The dot
on the far left corresponds to the vector representation.)
Of course, people don't usually talk about D_2 or D_3, because 
D_2 is two copies of A_1, and D_3 is the same as A_3.  However, 
there is no shame in doing so, and indeed a lot of insight to 
be gained: the fact that D_2 consists of two copies of A_1 
corresponds to the isomorphism
\begin{verbatim}
                   so(4) = su(2) + su(2),
\end{verbatim}
    
while the fact that D_3 is the same as A_3 corresponds to the 
isomorphism
\begin{verbatim}
                   so(6) = su(4).
\end{verbatim}
    
Each of these could easily serve as the springboard for a very
long and interesting discussion.  However, I will refrain.  Here
let me simply note that you can always "fold" a Dynkin diagram using
one of its symmetries, and if you do this to D_4 using triality
you go from 
\begin{verbatim}
              o  
             /   
        o---o        D_4
             \   
              o  

\end{verbatim}
    
down to
\begin{verbatim}
           6
         o->-o        G_2
\end{verbatim}
    
(Here the number 6 means that the two roots are at an angle of
\pi /6 from each other.  People usually just draw a triple line
to indicate this.  The arrow points from the long root to the shorter
root.)  This corresponds to how G_2 is the subgroup of Spin(8) consisting 
of elements that are invariant under triality.  You can also go from
\begin{verbatim}
               o  
               |          
       o---o---o---o---o    E_6

\end{verbatim}
    
down to
\begin{verbatim}

             4
       o---o->-o---o    F_4

\end{verbatim}
    
by folding along the reflection symmetry.  And Friedrich Knop
told me a neat way to get triality symmetry \emph{from} F_4, if you
happen to have F_4 around: the long roots of F_4 form a root system
of type D_4, which defines an embedding of Spin(8) into the Lie
group F_4 (more precisely, the compact real form).  On the other
hand, the two short simple roots define an embedding of SU(3)
in F_4.  The Weyl group of SU(3) is S_3 and can be lifted to
SU(3), so we have an S_3 subgroup of F_4.  This acts by conjutation
on the Spin(8) subgroup, implementing the triality symmetries!
But I digress.  My main point is, the Dynkin diagram symmetries
do give a nice way to understand outer automorphisms of simple
Lie groups, and these provide some important ties between simple
Lie algebras, including triality, which links the "classical" 
world to the "exceptional" world.  But it is also nice to try
to understand these in a somewhat more "conceptual" way.  This is one of the
reasons I'm interested in 2-Hilbert spaces... they seem to help
one understand this stuff from a new angle.  But more on those,
later.  They tie into the n-category stuff I'm always talking
about.  I will return to that tale soon, and I'll keep building
up some of the tools we need, until we are ready to launch
into a description of 2-Hilbert spaces. 
In writing this Week's Finds, I benefitted greatly from email
correspondence with Robt Bryant, Christopher Henrich, 
Geoffrey Mess, Friedrich Knop, and others.
<HR>

% </A>
% </A>
% </A>


% parser failed at source line 629
