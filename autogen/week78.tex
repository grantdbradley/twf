
% </A>
% </A>
% </A>
\week{March 28, 1996}


% <A NAME = "tale">
Last Week I began explaining the concept of "adjoint functor".  
This Week I want to finish explaining it - or at least finish
one round of explanation!   Then we'll begin to be able to 
see the unity of category theory, topology, and quantum theory.  
These may seem rather distinct subjects, but they have
an interesting tendency to blur together when one is doing topological
quantum field theory.  Part of the point of higher-dimensional
algebra is to explain this.  

So, remember the idea of adjoint functors.  Say we have categories C and D 
and functors L: C \to  D and R: D \to  C.  Then we say L is the "left adjoint" of 
R, or that R is the "right adjoint" of L, if for any object c of C and object
d of D, there is a natural one-to-one correspondence between hom(Lc,d) and 
hom(c,Rd).  An example to keep in mind is when C is the category of sets
and D is the category of groups.   Then L turns any set into the free group
on that set, while R turns any group into the underlying set of that group.
All sorts of other "free" and "underlying" constructions are also left and
right adjoints, respectively.   

Now the only thing I didn't make very precise is what I mean by
"natural" in the above paragraph.  Informally, the idea of a
"natural" one-to-one correspondence is that doesn't depend
on any arbitrary choices.  The famous example is that if we have a
finite-dimensional vector space V, it's always isomorphic to its dual
V*, but not naturally so: to set up an isomorphism we need to pick a
basis e_{i} of V, and this gives a dual basis f^{i} of V*, and
then we get an isomorphism sending e_{i} to f^{i}, but this
isomorphism depends on our choice of basis.  But V is
\emph{naturally} isomorphic to its double dual V**.

Now, it's hard to formalize the idea of "not depending on any arbitrary
choices" directly, so one needs to reflect on why it's bad for an 
isomorphism to depend on arbitrary choices.  The main reason 
is that the arbitrariness may break a useful symmetry.  
In fact, Eilenberg and Mac Lane invented category theory in order to 
formalize this idea of "naturality as absence of symmetry-breaking".  
Of course, they needed the notion of category to get a sufficiently general
concept of "symmetry".  They realized that a nice way 
to turn things in the category C into things in the category D is typically
a functor F: C \to  D.  And then, if we have two functors F,G: C \to  D, they
defined a "natural transformation" from F to G to be a bunch of morphisms

T_{c} : F(c) \to  G(c),

one for each object c of C, such that the following diagram 
commutes for every morphism f: c \to  c' in C: 


$$

                              F(f)
                       F(c) -------> F(c') 
 _{ }                      |             |
                     T_{c} |             | T_{c'}
 _{ }                      |             |
 _{ }                      V             V
                       G(c) -------> G(c')
                              G(f)

$$
    
This condition says that the transformation T gets along with all the
"symmetries", or more precisely morphisms f, in the category C.   We
can do it either before or after applying one of these symmetries, and
we get the same result.  For example, a vector space construction which 
depends crucially on a choice of basis will fail this condition if we take 
f to be a linear transformation corresponding to a change of basis. 

A "natural isomorphism" is then just a natural transformation that's
invertible, or in other words, one for which all the morphisms T_{c}
are isomorphisms.

Okay.  Hopefully that explains the idea of "naturality" a bit better.
But right now we are trying to figure out what we mean by 
saying that hom(Lc,d) and hom(c,Rd) are naturally isomorphic.  To do this, 
we need to introduce a couple more ideas: the product of categories,
and the opposite of a category.

First, just as you can take the Cartesian product of two sets, you can take
the Cartesian product of two categories, say C and D.
It's not hard.   An object of C \times  D is just a pair of objects,
one from C and one from D.  A morphism in C \times  D is just a pair of morphisms,
one from C and one from D.  And everything works sort of the way you'd expect.

Second, if you have a category C, you can form a new category
C^{op}, the opposite of C, which has the same objects as C, and
has the arrows in C turned around backwards.  In other words, a morphism
f: \times  \to  y in C^{op} is defined to be a morphism f: y \to  x
in C, and the composite fg of morphisms in C^{op} is defined to
be their composite gf in C.  C^{op} is like a
through-the-looking-glass version of C where they do everything
backwards.  A functor F: C^{op} \to  D is also called a
"contravariant" functor from C to D, since we can think of it
as a screwy functor from C to D with F(fg) = F(g)F(f) instead of the
usual F(fg) = F(f)F(g).  Whenever you see this perverse contravariant
behavior going on, you should suspect an opposite category is to blame.

Now, it turns out that we can think of the "hom" in a category C as a
functor  

                      hom(-,-): C^{op} \times  C \to  Set

Here the -'s denote blanks to be filled in.  Obviously, for any object
(x,x') in C^{op} \times  C, there is a nice juicy set hom(x,x'), the set of
all morphisms from \times  to x'.  But what if we have a morphism 

(f,f'): (x,x') \to  (y,y') 

in C^{op} \times  C?   For hom(-,-) to be a functor, we should get a
nice juicy function

hom(f,f'): hom(x,x') \to  hom(y,y').

How does this work?  Well, remember that a morphism (f,f') as above
is really just a pair consisting of a morphism f: \times  \to  y in C^{op}
and a morphism f': x' \to  y' in C.  A morphism f: \times  \to  y in C^{op} is
just a morphism f: y \to  \times  in C.  Now say we have an unsuspecting
element g of hom(x,x') and we want to hit it with hom(f,f') to get
something in hom(y,y').  Here's how we do it:

hom(f,f'): g |\to  f'gf

We compose it with f' on the left and f on the right!  Composing on
the left is a nice covariant thing to do, but composing on the right
is contravariant, which is why we needed the opposite category C^{op}.

Okay, now back to our adjoint functors L: C \to  D and R: D \to  C.
Now we are ready to say what we mean by hom(Lc,d) and hom(c,Rd) being
naturally isomorphic.  Using the stuff we have set up, we can define
two functors

hom(L-,-): C^{op} \times  D \to  Sets

and

hom(-,R-): C^{op} \times  D \to  Sets

and we are simply saying that for L and R to be adjoints, we demand
the existence of a natural isomorphism between these functors!

Of course, this seems abstract, but if you work it out in some of
the examples of adjoint functors given in "<A HREF = "week76.html">week76</A>" you'll see it all
makes good sense.

Now let me start explaining what this all has to do with quantum theory.
(I'll put off the topology until next Week.)  First of all, the "hom functor"
we introduced,

hom(-,-): C^{op} \times  C \to  Set

should remind you a whole lot of the inner product on a Hilbert space
H.  The inner product is linear in one slot and conjugate-linear in the
other, just like hom is covariant in one slot and contravariant in the other.
In fact, the inner product can be thought of as a bilinear map

<-,->: H^{op} \times  H \to  C

where H^{op}, the "opposite" Hilbert space, is like H 
but with a complex
conjugate thrown into the definition of scalar multiplication, and here
C denotes the complex numbers!

Second of all, the definition of adjoint functor, with hom(Lc,d) and
hom(c,Rd) being naturally isomorphic, should remind you of adjoint linear
operators on Hilbert spaces.  If we have a linear operator L: H \to  K from
a Hilbert space H to a Hilbert space K, its adjoint R: K \to  H is given
by

<Lh,k> = <h,Rk>

for all h in H and k in K.  

In fact, the whole situation with adjoint functors is a kind of
"categorified" version of the situation with adjoint linear
operators.  Everything has been boosted up one notch on the
n-categorical ladder.  What I mean is this: the Hilbert spaces H and K
above are \emph{sets}, with \emph{elements} h and k, while the
categories C and D are \emph{categories}, with \emph{objects} c and
d.  The inner product of two elements of a Hilbert space is a
\emph{number}, while the hom of two objects in a category is a
\emph{set}.  Most interesting, the definition of adjoint operators
requires that <Lh,k> and <h,Rk> be \emph{equal}, while the
definition of adjoint functors requires only that <Lc,d> and
<c,Rd> be \emph{naturally isomorphic}.

So we can think of adjoints in category theory as a boosted-up version
of the adjoints in quantum theory.  But these days, I prefer to think
of the adjoints in quantum theory as a watered-down or
"decategorified" version of the adjoints in category theory.
The reason is that categorification - as noted by Louis Crane, who I
believe invented the term - is a risky, hit-or-miss business, while
decategorification is much more systematic.  Decategorification is the
simply the process of neglecting the difference between isomorphism
and equality.  If we start with an n-category and then get lazy and
decide to think of invertible n-morphisms as \emph{equations}
between the (n-1)-morphisms, we get an (n-1)-category.  If we keep
slacking off like this, before you know it we're doing set theory!
The final stage of decategorification is when we get sloppy and
instead of keeping track of \emph{set}, we merely record the
\emph{number} of its elements.

It's amusing to imagine this process of decategorification as 
one of those elaborate Gnostic myths about the Fall.  We start in
the paradise of \omega -categories (or perhaps even higher up), but
by the repeated sin of confusing equality with isomorphism we fall all
the way down the n-categorical ladder to the crude world of sets, or
worse, simply numbers.  But all this happened a long time ago: now
we need to work our butt off to climb back up!  In other words,
historically our early ancestors dealt with finite sets by replacing
them with something cruder: their numbers of elements.  Counting is
actually very handy, of course, but it can only tell if the
cardinalities of two sets are \emph{equal}; it doesn't address the problem of
specific \emph{isomorphisms} between sets.  To climb back up the
n-categorical ladder, we needed to start with the set N of natural numbers 

0, 1, 2, 3, ...

and by dint of strenous mental effort realize that this is just the
decategorification of the category FinSet of finite sets.  (In fact,
category-theorists routinely use 2 to stand for the 2-element set in
the skeletal category equivalent to FinSet, and so on - see "<A
HREF = "week76.html">week76</A>".)

Now, you are certainly entitled to wonder if this elaborate
mathematical-theological fantasy is just a joke or if it has some
practical spinoffs.  For example, is there anything we can \emph{do}
with the analogy between adjoint operators and adjoint functors?  As
it turns out, there is.  The point is that the analogy is not quite
precise.  For example, every linear operator has an adjoint, but not
every functor has an adjoint - nor need it be "linear" in
any sense.  If we endeavor to make the analogy precise, we will invent
a special sort of category called a "2-Hilbert space" which
is the precise categorified analog of a Hilbert space.  And we will
invent a nice sort of "linear" functor between these, and
all such functors will have adjoints.  Furthermore, in this situation
all left adjoints will also be right adjoints... fixing another funny
discrepancy.  And these 2-Hilbert spaces turn out to be closely
related to 2-dimensional topological quantum field theories (in
general, n-Hilbert spaces appear to be related to n-dimensional
TQFTs), as well as some interesting aspects of group representation
theory.

I'm busily writing a paper on exactly this stuff, but I have not
explained enough category theory here to describe it in detail yet.  For
now, let me just make the connection between the hom(-,-) of category
theory and the <-,-> of quantum theory more clear, and hopefully
more plausible.  If we have states h and h' in a Hilbert space,
<h,h'> keeps track of the \emph{amplitude} of getting from h to
h'.  (Often people will say "from h' to h", but here I think I
really want to go the other way.)  This is a mere \emph{number}.  If
we have objects c and c' in a category, hom(c,c') is the actual \emph{set} of
ways to get from c to c', that is, the set of morphisms from c to c'.

When one computes transition amplitudes by summing over paths, as in
Feynman path integrals, one is in a sense decategorifying, that is,
turning a set of ways to get from here to there into a number, the
transition amplitude.  However, one is not just counting the ways, one
is counting them "with phase".... and I must admit that the
role of the \emph{complex numbers} in quantum theory is still puzzling
from this viewpoint.  For some food for thought, you might want to check
out Dan Freed's work on torsors, which are a categorified version of
phases:

1) Higher algebraic structures and quantization, by Daniel Freed,
Commun. Math. Phys. 159 (1994), 343-398, also available as
<A HREF = "http://xxx.lanl.gov/abs/hep-th/9212115">hep-th/9212115</A>. 





<A HREF = "week79.html#tale">To continue reading the "Tale of
n-Categories", click here.</A>


\par\noindent\rule{\textwidth}{0.4pt}
% </A>
% </A>
% </A>
