
% </A>
% </A>
% </A>
\week{July 14, 1996 }



I'm spending this month at the Erwin Schroedinger Institute in Vienna,
where Abhay Ashtekar and Peter Aichelburg are running a workshop called 
Mathematical Problems of Quantum Gravity.  

Ashtekar is one of the founders of an approach to quantizing gravity
called the loop representation.  I've explained this approach in
"<A HREF = "week7.html">week7</A>", "<A HREF = "week43.html">week43</A>", and other places, but let me just remind you of the
basic idea.  In the traditional approach to reconciling general
relativity with quantum theory, excitations of the gravitational field
were described by small ripples in the geometry of flat spacetime, or
"gravitons".  In the loop representation, they are instead described
by collections of loops, which we can think of as "flux tubes of area"
floating in an otherwise utterly featureless void.  More recently, the
loop approach has been supplemented by a technical device known as
"spin networks": roughly speaking, a spin network is a graph whose
edges are labelled by spins 0,1/2,1,3/2,... with an edge of spin j
corresponding to a flux tube carrying area equal to sqrt(j(j+1)) times
the square of the Planck length --- the fundamental length scale in
quantum gravity, about 10^{-35} meters.  For more on spin networks,
try "<A HREF = "week55.html">week55</A>".

Quantum gravity has always been a tough subject.  After a lot of work,
a lot of people concluded that the traditional approach to quantum
gravity didn't make sense, mathematically.  This led to string theory,
an attempt to quantize gravity together with all the other forces and
particles.  But in the late 1980s, Rovelli and Smolin revived hopes of
quantizing gravity alone by introducing the loop representation.  

One doesn't expect the loop representation to describe much real physics
until one introduces other forces and particles.  Pure gravity is just
a warm-up exercise --- but it's not at all easy!  When the loop
representation was born, it was rather sketchy at many points.  A lot
of mathematical problems had to be overcome to make it as precise as
it is now.... and there are a lot of formidable difficulties left, any
one of which could spell doom for the theory.  Luckily, progress has
been rapid.  Many of the problems which seemed hopelessly beyond our
reach a few years ago can now be formulated precisely, and maybe even
solved.  The idea of this workshop is to start tackling these
problems.

A lot has been going on!  People give talks at 11 in the morning,
while afternoons are devoted to more informal discussions in small
groups.  There are general introductory talks on Tuesdays, more
technical talks on Thursdays, and short talks on research in progress
on some other days.

To give a bit of the flavor of the workshop, let me describe things
day by day.  I'll need to describe some days very sketchily, though,
or I'll never finish writing this!  

Wednesday, July 3 - Rodolfo Gambini spoke on gauge-invariance in the
extended loop representation.  The idea of the loop representation is
to study the gravitational vector potential by studying certain
integrals of it around loops.  Mathematicians call this the trace of
the holonomy, and physicists call it a Wilson loop or the trace of a
path-ordered exponential.   In the loop representation, states of
quantum gravity are described by certain functions that eat Wilson 
loops and spit out complex numbers... i.e., that assign an "amplitude"
to each Wilson loop.

In quantum field theory you often need to average a quantum field over
some 3-dimensional region of space or 4-dimensional region of
spacetime to get a well-defined operator.  Wilson loops are rather
singular because a loop is a one-dimensional object.  On the other
hand, they are nice because they are gauge-invariant: they don't
change when we do a gauge transformation to the vector potential.

In the "extended" loop representation one tries to make the integral
less singular by not dealing with actual loops, but certain analogous
integrals over all 3-dimensional space.  Heuristic calculations
suggest that they are gauge-invariant, but Troy Schilling noticed a
while ago that they aren't always \emph{really} gauge-invariant ---
basically because the the path-ordered exponential is given by a
certain Taylor series, and nasty things can happen when you manipulate
infinite series without checking if your manipulations are legitimate!
See:

1) Troy Schilling, Non-covariance of the generalized holonomies: Examples, 
preprint available as <A HREF = "http://xxx.lanl.gov/abs/gr-qc/9503064">gr-qc/9503064</A>.  

There has been a certain amount of competition between the extended
loop representation, developed by Gambini and various coauthors, and
Ashtekar's approach.  Thus Schilling's result was seen as a blow
against the extended loop representation.  In Gambini's talk, he
argued that gauge-invariance is rigorously maintained by certain
extended loops, e.g. those for which the power series has finitely
many terms.  The most famous examples of functions of extended loops
with only finitely many terms are the Vassiliev invariants, which come
up in knot theory (see "<A HREF = "week3.html">week3</A>").  Gambini and Pullin have claimed that
certain Vassiliev invariants are states of quantum gravity, so these
are of special interest.

The feeling was that we needed to compare these different loop
representations more carefully because they both have advantages.

Also, Renate Loll spoke about "Lattice Gravity".  See "<A HREF = "week55.html">week55</A>" for a
bit more on this.  Her talk led to an interesting discussion of the
meaning of the limit, as the lattice spacing goes to zero, of quantum
gravity as done on a lattice.  Does it make sense?  One needs,
apparently, to look at ones formula for the Hamiltonian constraint on
the lattice, and see if it depends on the Planck length in a manner
\emph{other than} having the Planck length as an overall prefactor.
Various people tried to do the calculation on the spot, and got mixed
up.

Thursday, July 4 - Thomas Thiemann spoke on "The Hamiltonian
Constraint for Lorentzian Canonical Quantum Gravity".  This was a big
bombshell.  The Hamiltonian constraint in quantum gravity is one of
the biggest, baddest problems we are facing.  It's the analog of
Schrodinger's equation in quantum mechanics, but it's a constraint:

                      H \Psi  = 0

All the dynamics of the theory is contained in this equation, yet we
only roughly understand how to define it in a rigorous way.  Thiemann,
a student of Ashtekar who is now a postdoc at Harvard, had put the
following 5 papers on the general relativity preprint server right
before the workshop.  The first one gives a rigorous definition of the
Hamiltonian constraint!

2) Thomas Thiemann, Quantum Spin Dynamics (QSD), preprint available
as <A HREF = "http://xxx.lanl.gov/abs/gr-qc/9606089">gr-qc/9606089</A>.  

Thomas Thiemann, Quantum Spin Dynamics (QSD) II, preprint available
as <A HREF = "http://xxx.lanl.gov/abs/gr-qc/9606090">gr-qc/9606090</A>.

Thomas Thiemann, Anomaly-free formulation of non-perturbative,
four-dimensional Lorentzian quantum gravity, 
Phys. Lett. B 380 (1996) 257-264,
preprint available as <A HREF = 
"http://xxx.lanl.gov/abs/gr-qc/9606088">gr-qc/9606088</A>.

Thomas Thiemann, Closed formula for the matrix elements of the volume
operator in canonical quantum gravity, preprint available as
<A HREF = "http://xxx.lanl.gov/abs/gr-qc/9606091">gr-qc/9606091</A>.

Thomas Thiemann, A length operator for canonical quantum gravity,
preprint available as <A HREF = "http://xxx.lanl.gov/abs/gr-qc/9606092">gr-qc/9606092</A>.

It is interesting to compare "Quantum Spin Dynamics" to the
paper by Ashtekar and Lewandowksi, so far available only in draft form to a
select few, in which they gave a rigorous definition of the square
root of the Hamiltonian constraint.  The advantage of "QSD" is that it
deals directly with the Hamiltonian constraint, rather than its square
root, and that it does this using some ingenious formulas for the
Hamiltonian constraint of Lorentzian gravity in terms of the
Hamiltonian constraint for Riemannian gravity and the total volume and
total extrinsic curvature of the universe (which we assume is compact).

You see, quantum gravity comes in two flavors, Lorentzian and
Riemannian, depending on whether we work with real time ---- the
obviously sensible thing to do --- or imaginary time --- not at all
obviously sensible, but with a curious mathematical charm to it, which
has won many hearts.  The interplay between these two has long been a
bugaboo of the loop representation.  The Lorentzian theory is harder
to work with, so lots of people cheat and study the Riemannian theory.
Sometimes they do this covertly, with a guilty conscience, so in some
papers it's left unclear which theory the author is actually working
with!  Thiemann's work, however, seems to exploit the interplay
between the theories in a benign way --- related to earlier ideas of
Ashtekar, but different.  I would like to understand this interplay
more deeply.

Due to jetlag I woke up at 4 am on the morning of this talk, and I
couldn't get back to sleep, so I read his paper.  When I came to the
Institute at 9 am --- a shockingly early hour for people working on
quantum gravity --- I was sure nobody would be there yet.  But as I
entered I bumped into Carlo Rovelli.  It turned out he had stayed up
all night reading Thiemann's paper, too excited to sleep!  

After this talk everyone was busily trying to learn Thiemann's stuff,
trying to figure out if it is physically correct, and trying to figure
out what to do next.

Tuesday, July 9 - Abhay Ashtekar gave a general talk on the "Quantum
Theory of Geometry".  Most of it was well-known stuff to fans of the
loop representation, but one new tidbit concerned the noncommutativity
of area operators.  Since the area of surfaces in space depends only
on the metric on space, not on its first time derivative, one might
expect their quantum analogs to commute, since the metric and its
first time derivative are analogous to position and momentum in
quantum mechanics.  But they don't commute!  In a later talk, Ashtekar
explained that this is not really a strange new feature of quantum
gravity, but one which has its classical analog.

Wednesday, July 10 - Kirill Krasnov gave a talk on a paper we started
working on together just recently, "Quantization of diffeomorphism
invariant theories with fermions".  Kirill is a young Ukrainian
physicist whom I first met last summer in Warsaw; he had written a
nice paper on the loop representation of quantum gravity coupled to
electromagnetism and fermions:

3) Kirill Krasnov, Quantum loop representation for fermions coupled
to Einstein-Maxwell field, Phys. Rev. D53 (1996), 1874; preprint
available as <A HREF = "http://xxx.lanl.gov/abs/gr-qc/9506029">gr-qc/9506029</A>.

When I met him again here, it turned out he was continuing this work,
and also making it more rigorous.  Now, I had for some time been
meaning to write something with Hugo Morales-Tecotl showing that a
slight generalization of spin network states form a basis of states
for such theories.  These states had already appeared, for example, in
his work with Rovelli:

4) Carlo Rovelli and Hugo Morales-Tecotl, Fermions in quantum 
gravity, Phys. Rev. Lett. 72 (1994), 3642-3645.

Carlo Rovelli and Hugo Morales-Tecotl, Nucl. Phys. B451 (1995), 325,
preprint available as <A HREF = "http://xxx.lanl.gov/abs/gr-qc/9401011">gr-qc/9401011</A>. 

But we had never gotten around to it.  So, I decided to team up with Kirill
and write a paper on this stuff.

<HR>

% </A>
% </A>
% </A>


% parser failed at source line 298
