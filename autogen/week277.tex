
% </A>
% </A>
% </A>
\week{August 10, 2009 }

I'm very happy to have finished writing a huge paper with Aaron
Lauda... a paper we've been working on for 5 years!  It's a chronology
of 20th century math and physics, focused on why people are
starting to use n-categories in physics:

1) John Baez and Aaron Lauda, A prehistory of n-categorical physics,
to appear in Deep Beauty: Mathematical Innovation and the Search for 
an Underlying Intelligibility of the Quantum World, edited by Hans 
Halvorson.  Available at <a href = 
"http://math.ucr.edu/home/baez/history.pdf">
http://math.ucr.edu/home/baez/history.pdf</a>

This is a companion to a paper I talked about in "<a href =
"week261.html">week261</A>":

2) John Baez and Mike Stay, Physics, topology, logic and computation:
a Rosetta Stone, to appear in New Structures in Physics, edited by Bob
Coecke.  Available as <a href =
"http://arxiv.org/abs/arXiv:0903.0340">arXiv:0903.0340</a>.

Both of them are supposed to be gentle explanations of how 
"diagrammatic thinking" unifies and clarifies our perspective
on many subjects.  I can imagine merging them and expanding
them to form a book someday... but not today.  Today I want to 
take a break from n-categories, and talk about some basic physics.

Let's start with a riddle. 

What's a million times thinner than paper, stronger than diamond,
a better conductor than copper, and absorbs exactly 

\pi  \alpha  ~ 3.14159 / 137.035 ~ 2.29254%
 
of the light you shine through it?   

Hint: \alpha  is the "fine structure constant" - a fundamental 
dimensionless constant that specifies the strength of the 
electromagnetic force.

Can't guess?  Okay, here's the answer: graphene!  

<div align = "center">
<img src = "graphene.jpg">
</div>

Graphene is what you get when you take one slice of a crystal of
graphite.  It's a hexagonal array of carbon atoms, each connected to
three neighbors.  I mentioned this substance back in "<a href =
"week262.html">week262</A>", when I was visiting the National
University of Singapore, because researchers there were working on
graphene as a possible substitute for silicon chips, which might
operate 1000 times faster.  Now a group there led by Barbaros &Ouml;zyilmaz
is trying to use graphene for storing information:

3) Prachi Patel, A step towards superfast carbon memory, 
Technology Review, Wednesday April 1, 2009, available at  
<a href = "http://www.technologyreview.com/computing/22377/">http://www.technologyreview.com/computing/22377/</a>.

No, this is not an April Fool's joke.

Graphene was only discovered in 2004, but the easiest way to 
make some is surprisingly lowtech.  You press a chunk of 
graphite onto some scotch tape, and hope a thin layer sticks:

4) Making graphene 101, &Ouml;zyilmaz' group, YouTube video, 
available at <a href = "http://www.youtube.com/watch?v=rphiCdR68TE">http://www.youtube.com/watch?v=rphiCdR68TE</a>

Graphene has some amazing properties.  For starters, it acts 
like a toy universe containing massless spin-1/2 particles, 
which zip around at the speed of light.  But in this toy
universe, light moves 300 times slower than in the real universe!

To understand this, you have to start by realizing that any sort of
wave also acts like a particle when you take quantum mechanics into
account.  The idea is that when something can wiggle back and forth at
some frequency, it can wiggle a little or a lot - but it can't wiggle
just \emph{any} amount.  The amount of wiggling is
"quantized" - it comes in discrete steps.  These steps are
too small to see in normal life, where for example you might see a
rubber band seem to vibrate with any amplitude it wants.  But these
quantized steps are there nonetheless, no matter what is doing the
wiggling - it appears to be a completely general principle.

When the wiggles are waves that are moving along, these quantized
steps are called "particles".  So, for example, when we have
a wave of light of some particular frequency that contains 500 times
more energy than the bare minimum, we say it consists of 500
"particles" called photons.  At first this may sound weird,
but it turns out that many things we normally consider particles -
like electrons and protons and neutrons - really are just quantized
wiggles of some sort or another.

In cases like these, the stuff that's doing the wiggling is rather
absract: for photons, we say it's the "electromagnetic
field", and for other particles we say it's various other fields.

But since the principle is completely general, there are also cases
where the stuff doing the wiggling is quite mundane.  For example, if
you have a crystal, the crystal's atoms can wiggle.  This is of course
how \emph{sound} propagates.  But this should mean that sound comes in
quantized packets called "phonons".  And indeed, experiments
with crystals show that this is true!

5) Wikipedia, Phonon, <a href = "http://en.wikipedia.org/wiki/Phonon">http://en.wikipedia.org/wiki/Phonon</a>

<div align = "center">
<a href = "http://en.wikipedia.org/wiki/Phonon">
<img style="border:none;" src = "normal_modes.gif">
% </a>
</div>

Different vibrational modes of the crystal have different numbers of
wiggles per \emph{distance}, and different numbers of wiggles per
\emph{time}; these give phonons with different \emph{momentum} and
\emph{energy}.  The relation is simple:

momentum = h (wiggles per distance)

energy = h (wiggles per time)

where h is Planck's constant, a tiny little constant. 

(Why am I talking about sound in crystals, rather than air?   It's
because our story applies in its simplest form to vibrational 
modes that don't interact much with other modes or get damped out
quickly by friction.  These give rise to particles that don't 
interact much with other particles, and don't decay very fast -
so they're easy to see.) 

There are other things about a crystal that can vibrate besides
the atoms.  For example, the atoms may have an angular momentum,
or "spin", that can change directions.  Often each atom's spin
affects its neighbors through magnetic forces.  This allows waves
of changed spin - "spin waves" - to propagate through the crystal.  
And again, these waves come in quantized packets called "magnons":

6) Wikipedia, Spin wave, <a href = "http://en.wikipedia.org/wiki/Spin_wave">http://en.wikipedia.org/wiki/Spin_wave</a>

A crystal can also have a "hole": an atom with a missing
electron These holes can move around, so they too act like particles.
This example is perhaps a bit more obvious than the previous ones...
so here I should probably emphasize that just as sound waves or spin
waves come in discrete particle-like units, these holes act like
waves!

Similarly, a crystal can have atom with an \emph{extra} electron.  
Here it's even \emph{more} obvious that these electrons act like 
particles - you probably want to just say they \emph{are} particles.
But at this point it's crucial to emphasize that this example
is really like all the rest.  In particular (pardon the pun),
we need to understand these extra electrons in terms of waves,
and we need to compute their energy and momentum using these
formulas:

momentum = h (wiggles per distance)

energy = h (wiggles per time)

so the relation between the energy and momentum depends completely 
on the details of the crystal: you can't just use some formula
you may happen to know about a "free" electron in empty space.
So, electrons in crystals can do all sorts of crazy things that
you'd never expect, and indeed this is what makes things like
conductors and transistors possible.

Let's talk about relations between energy and momentum.  For a 
particle of mass m in empty space, special relativity says that 

energy^{2} = momentum^{2} + mass^{2}

in units where the speed of light is 1.  If we work in units
where the speed of light is some number c, the formula looks messier:

energy^{2} = momentum^{2} c^{2} + mass^{2} c^{4}

If you look at a particle at rest, its momentum is zero so a little
calculation reveals that:

energy = mass c^{2}

which is a pretty famous formula.  

But if we go to the other extreme, and look at "massless" particle, 
meaning one with mass = 0, we get

energy = momentum c

which is also famous, but less so.  If we apply the formulas

momentum = h (wiggles per distance)

energy = h (wiggles per time)

we get

wiggles per time = c (wiggles per distance)

This fact is easy to believe if we imagine light as a wave moving
along, and c as the speed of this wave.  Indeed, this fact was 
important in helping guess the relation between momentum and
"wiggles per distance", and energy and "wiggles per
time", back when they were first inventing quantum mechanics.

Now, the above formulas apply to photons in empty space.  But my 
real point is this: if we have any sort of substance and any sort of
wave that can move around in this substance, obeying these formulas,
it \emph{acts like a massless particle}.  So, we can use a lot of the same 
math and intuition that we use for photons.  But, quite likely the
value of c will be different than the real-world speed of light!

And this in fact is what happens for graphene.  The calculation is
pathetically simple by ordinary physics standards, yet complicated
enough that I don't feel like presenting the details here.  You 
can see it near the beginning of this wonderful paper:

7) Jiannis K. Pachos, Manifestations of topological effects in
graphene, to appear in Contemp. Phys..  Also available as <a href =
"http://arxiv.org/abs/arXiv:0812.1116">arXiv:0812.1116</a>.

You start with a really simplified model: a planar hexagonal honeycomb
that can either have or lack an electron at each vertex.  You ignore 
the spin of the electron for some reason.  It's a bit surprising you 
can get away with that!  You assume the energy is a bit less whenever 
you've got electrons at both ends of any given edge of your hexagonal 
honeycomb.  

It helps to color the vertices of the honeycomb alternately black and
white.  Then there are two kinds of electrons: "black" and
"white" ones.  The formula for the energy can be expressed
nicely in terms of these, because it's a sum over pairs of neighboring
vertices, one black and one white.  Note that whenever an electron
moves, it changes from black to white or vice versa.

Then you do a little calculating: you do a Fourier transform here,
you diagonalize a 2\times 2 matrix there, and... presto!

There turn out to be two different kinds of waves, each consisting of
a certain linear combination of black and white electrons.  The energy of
these waves (or if you prefer, their number of wiggles per time) is
related to their momentum (or wiggles per distance) by a complicated
formula.

To write down this formula, we have to treat momentum as a
\emph{vector}, since the energy of a wave depends on which direction
it's going: the hexagonal honeycomb means that not all direction
behave the same.  If you plot the energy E as a function of the
momentum vector p = (p_{x},p_{y}), you get a graph
like this:

<div align = "center">
<img src = "graphene_dispersion_relation.png">
</div>

It's pretty fancy!  But you can see it has hexagonal symmetry, as you'd
expect.  And it's also periodic.  This comes from the periodicity 
of the crystal lattice - but in sneaky way: the crystal lattice in
actual space gives rise to a "dual lattice" in the space of momentum
vectors, and the energy stays the same if you take the momentum and
add any vector in this dual lattice.  This is no big deal, it happens 
for any crystal:

8) Wikipedia, Reciprocal lattice, 
<a href = "http://en.wikipedia.org/wiki/Reciprocal_lattice">http://en.wikipedia.org/wiki/Reciprocal_lattice</a>

In crystal jargon, the original crystal lattice is called the "Bravais
lattice" and its dual is called the "reciprocal lattice".

The big deal is that if you look carefully at certain points in the
graph, it looks almost like a cone.  

<div align = "center">
<img src = "graphene_dispersion_relation_2.jpg">
</div>

If we move the tip of this cone
to the origin, the formula for it is:

E^{2} = (p_{x}^{2} + 
p_{y}^{2}) c^{2} 

or in other words:

energy^{2} = momentum^{2} c^{2}

for some constant c.  And this is just the relation we had for a
massless particle!

A digression: when I wrote this relation before, I took a square root 
and got 

energy = momentum c


But the above graph includes negative 
energies as well.  What do the negative energies mean, physically?
That's an interesting puzzle, which people had to confront when they
were first trying to combine special relativity and quantum mechanics.
I don't want to talk about it here, but if you're curious, see this:

9) M. I. Katnelson, K. S. Novoselov, A. K. Geim, Chiral tunnelling and
the Klein paradox in graphene, Nature Physics 2 (2006), 620.

Anyway, the upshot is that near one of these special points, graphene
acts like it has two kinds of massless particles in it:
"black" and "white".  They are approximately
described by the usual equation for massless spin-1/2 particles, the
Dirac equation - but in a universe with just 2 dimensions of space!
The speed c is about 1/300th of the real-world speed of light.  Also,
a spin-1/2 particle in 2d space usually comes in two states: it can
rotate clockwise or counterclockwise.  But here those two states are
certain linear combinations of "black" and "white".

Furthermore, because the formula 

energy^{2} = momentum^{2} c^{2}

remains unchanged when you rescale both distances and times by the
same factor, graphene reacts exactly the same way to light of 
different wavelengths - at least within some range where all the
approximations we've made are good.  And, it turns out to absorb
\pi \alpha  of the light you shine through it:

10) A. B. Kuzmenko, E. van Heumen, F. Carbone, and D. van der Marel,
Universal infrared conductance of graphite, Phys. Rev. Lett. 100
(2008), 117401. 

11) R. R. Nair, P. Blake, A. N. Grigorenko, K. S. Novoselov, T.
J. Booth, T. Stauber, N. M. R. Peres, and A. K. Geim, A. K.,
Fine structure constant defines visual transparency of graphene,
Science 320 (2008), 1308.  Also available at
<a href = "http://onnes.ph.man.ac.uk/nano/Publications/Science_2008fsc.pdf">http://onnes.ph.man.ac.uk/nano/Publications/Science_2008fsc.pdf</a>

So, graphene is an amazing physical system, but so far I've just
scratched the surface.  Graphene also has "topological
defects", and "anyons", and there are wonderful
applications of Euler's theorem and its big brother, the Atiyah-Singer
index theorem.  All this in a physical system you can make by
pressing graphite against a piece of tape!

I should say more about all this, but I need to go to bed and then
catch an early morning plane.  So, I'll quit here - but I highly recomend
Pachos' paper "Manifestations of topological effects in graphene".

\par\noindent\rule{\textwidth}{0.4pt}
<p
\textbf{Addenda:} For more discussion visit the <a href =
"http://golem.ph.utexas.edu/category/2009/08/this_weeks_finds_in_mathematic_38.html.html">\emph{n}-Category
Caf&eacute;</a>.

\par\noindent\rule{\textwidth}{0.4pt}
<em>
Error and doubt no longer encumber us with mist;<br/>
We are now admitted to the banquets of the Gods;<br/>
We may deal with laws of heaven above; and we now have<br/>
The secret keys to unlock the obscure earth.</em> - Halley

\par\noindent\rule{\textwidth}{0.4pt}

% </A>
% </A>
% </A>
