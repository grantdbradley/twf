
% </A>
% </A>
% </A>
\week{August 26, 1996 }

This issue concludes my report of what happened at the
Mathematical Problems of Quantum Gravity workshop in Vienna.  I
left the workshop at the end of July, so my reportage ends there,
but the workshop went on for a few more weeks after that.  I'll
be really bummed out if I find out that they solved all the
problems with quantum gravity after I left.
Before I launch into my day-by-day account of what happened,
let me note that I've written a little introduction to Thiemann's
work on the Hamiltonian constraint, which he presented at the
workshop (see "<A HREF = "week85.html">week85</A>"):
1) John Baez, The Hamiltonian constraint in the loop
representation of quantum gravity, available 
at <A HREF =
"http://math.ucr.edu/home/baez/hamiltonian/">http://math.ucr.edu/home/baez/hamiltonian/</A>
 A less technical version of this appears in Jorge Pullin's 
newsletter Matters of Gravity, issue 8, at
<A HREF = "http://www.phys.lsu.edu//mog/mog8/node7.html">http://www.phys.lsu.edu//mog/mog8/node7.html</A>

Okay... I'll start out simple today since there is something nice
and simple to ponder:
Tuesday, July 23rd - Ted Jacobson spoke on the "Geometry and
Evolution of Degenerate Metrics".  One of the interesting things
about Ashtekar's reformulation of general relativity is that it
extends general relativity to the case of degenerate metrics,
that is, metrics where there are vectors whose dot product
with all other vectors is zero.   However, one needs to be very
careful because different versions of Ashtekar's formulation give
\emph{different} ways of handling degenerate metrics.
To see why in a simple example, remember that the usual metric on
Minkowski spacetime is nondegenerate and in nice coordinates
looks like
-dt^2 + dx^2 + dy^2 + dz^2
Here we are setting the speed of light equal to 1.  In general
relativity, one way people describe the metric is using a tensor
g_{ab}, where the indices a and b go from 0 to 3.  In
Minkowski space this tensor equals
\begin{verbatim}
-1  0  0  0 
 0  1  0  0
 0  0  1  0
 0  0  0  1
\end{verbatim}
    
What this tensor means is that if you have two vectors v and w,
their dot product is g_{ab} v^a w^b, where as usual we multiply
the entries of the metric tensor and the vectors v and w as
indicated, and then sum over repeated indices.  So, for example,
the dot product of the vector 
v = (1, 1, 0, 0)
with itself is 0, though its dot product with other vectors
needn't be zero.   There is a bunch of vectors whose dot products
with themselves are zero, and these are called lightlike vectors,
because light travels in these directions, moving one unit in
space for each unit in time.   There is actually a cone of
lightlike vectors, called the lightcone.
One can imagine a world where the metric g_{ab} is
\begin{verbatim}
-1  0  0  0
 0  1  0  0
 0  0  k  0
 0  0  0  k
\end{verbatim}
    
</P>
for some k > 0.  This world isn't really so different from
Minkowski space, because you can also think of it as Minkowski
space described in screwy coordinates where you are measuring
distances in the y and z directions in different units than the x
direction.   When k gets small, you can check that the lightcone
gets stretched out in the y and z directions.  Alternatively,
when k gets big, the lightcone gets squashed in the y and z
directions.  
Another way to formulate general relativity uses the inverse
metric g^{ab}.  This is just the inverse of the matrix g_{ab},
which is indeed invertible when the metric is nondegenerate.
So for example in the above case the inverse metric g^{ab} is
\begin{verbatim}
-1  0  0  0
 0  1  0  0
 0  0  K  0
 0  0  0  K
\end{verbatim}
    
where K = 1/k.  You can think of K as the speed of light in the
y and z directions, which is different from the speed of light in
the x direction.  
Now there are two different limiting cases we can consider,
depending on whether we work with the metric or the inverse
metric.  If we work with the metric, we can let k = 0.  This
corresponds to making the speed of light in the y and z
directions \emph{infinite}, so that information can go as fast as it
likes in those directions and the lightcone gets completely
stretched out in those directions.   Note that now the metric
g_{ab} is
\begin{verbatim}
-1  0  0  0
 0  1  0  0
 0  0  0  0
 0  0  0  0
\end{verbatim}
    
so the inverse metric doesn't even make sense --- you can't
invert this matrix.  If we extend general relativity to
degenerate metrics, we are allowing ourselves to study weird
worlds like this.  Why we'd want to --- well, that's another
matter.
If we work with the inverse metric, we can't let k = 0, but we
can let K = 0.  This corresponds to making the speed of light in
the y and z directions \emph{zero}, so that information can't go at
all in those directions: the lightcone is squashed down onto the
t-x plane.   Now it's the inverse metric that equals
\begin{verbatim}
-1  0  0  0
 0  1  0  0
 0  0  0  0
 0  0  0  0
\end{verbatim}
    
and the metric doesn't even make sense.
Ted Jacobson's talk was about doing general relativity in weird
worlds like this, where the inverse metric is degenerate.  Here
information flows only along surfaces, like the x-t plane in the
example above, and these different surfaces don't really talk to
each other very much.  It's as if the world was split up (or in
math jargon, foliated) into lots of different 2-dimensional
worlds, which didn't know about each other.  Jacobson showed that
in this case, the equations of general relativity (extended in a
certain way to degenerate inverse metrics) boil down to saying
that there are two kinds of massless spin-1/2 particle living on
all these 2-dimensional worlds.  
This got me quite excited because it reminded me of string
theory, which is all about massless particles (or in physics
jargon, conformal fields) living on the 2-dimensional string
worldsheet.   I am always hunting around for relationships
between string theory and the loop representation of quantum
gravity, and I think this is an important clue.  The reason is
that I think the loop representation can be thought of as a
quantum version of the theory of degenerate solutions of general
relativity where the metric is \emph{zero} most places and less
degenerate (but still degenerate) on certain surfaces.   When you
slice one of these surfaces with the hyperplane t = 0 you get a
bunch of loops (or more generally a graph), and these are the
loops of the loop representation.  Jacobson's talk may give a way
to understand the conformal field theory living on these
surfaces, which one needs if one wants to think of these surfaces
as the "string worldsheets" of string theory fame.  Anyway, I am
busily thrashing this stuff out and trying to write a paper on
it, but it may or may not hang together.
Jacobson's talk is based on a short paper he'd just been editing
the galley proofs for; so it should come out soon: 
2) Ted Jacobson, 1+1 sector of 3+1 gravity, Class. Quant. Grav.
13 (1996), L1-L6.
Now around this time the Erwin Schroedinger Institute, where the
workshop was being held, moved from its comfortable old spot on
Pasteurgasse to a more spacious location on Boltzmanngasse, near
the physics department.  (In Germany the word "Gasse" means
"alley", and one might find it disrespectful that Pasteur and
Boltzmann have mere alleys named after them, but in Vienna even
lots of large streets are called "Gasse", when in Germany they'd
be called "Strasse".  But then even the word for potato is
different in Austria; it's all part of the charm of the place.)  
The move disrupted the schedule of the talks a bit, and it also
seems to have disrupted my note-taking, which gets more sketchy
from here on out.  Some of the dates below might be a bit off.
Thursday, July 25th - I spoke on "Topological Quantum Field
Theory".  I am always talking about this on This Week's Finds so
I won't bore you with the details.  Basically I summarized what
is known about BF theory (a particular topological quantum field
theory) in dimensions 2, 3, and 4, and the discrete formulation
of BF theory where you chop spacetime into simplices and label
the edges and so on with spins and the like --- so-called "state
sum models".  You can read more about this in 
"<A HREF = "week38.html">week38</A HREF>".
Later that day, Jerzy Lewandowski spoke on "Degenerate Metrics".
Being somewhat less degenerate than Ted Jacobson, he spoke about
extending general relativity to cases where the inverse metric
looks like
\begin{verbatim}
-1  0  0  0
 0  1  0  0
 0  0  1  0 
 0  0  0  0
\end{verbatim}
    
In other words, where the speed of light is zero only in the z
direction.  Basically what happens is that spacetime gets
foliated with a lot of 3-dimensional slices, and on each one you
get the equations of 3-dimensional general relativity.
Friday, July 26th - Thomas Strobl spoke on 2-dimensional
gravity. I don't understand his work well enough yet to have
anything much to say, but the most interesting thing about it to
\emph{me} is that it allows one to see how quantum groups emerge from
the G/G gauged Wess-Zumino-Witten model (a certain 2-dimensional
topological quantum field theory), by describing this theory as
the quantization of a Poisson \sigma -model --- a field theory
where the fields take values in a Poisson manifold.  For more,
try:
3) Peter Schaller and Thomas Strobl, A brief introduction to
Poisson \sigma -models, preprint available as 
<A HREF = "http://xxx.lanl.gov/abs/hep-th/9507020">hep-th/9507020</A>.
Peter Schaller and Thomas Strobl, Poisson \sigma -models: a
generalization of 2d gravity-Yang-Mills systems, preprint
available as <A HREF =
"http://xxx.lanl.gov/abs/hep-th/9411163">hep-th/9411163</A>.

Later, I had a great conversation with Mike Reisenberger and
Carlo Rovelli on reformulating the loop representation of quantum
gravity in terms of surfaces embedded in spacetime.  This again
touched upon my interest in relating string theory and the loop
representation.  They are writing a paper on this which should be
on the preprint servers pretty soon, so I'll wait until then to
talk about it.
Saturday, July 27th - Carlo Rovelli explained some things about
the problem of time to me.
Monday, July 30th - I spoke about relative states and
entanglement entropy in two-part quantum systems (see "<A HREF =
"week27.html">week27</A>" 
and the applications of these ideas to topological quantum field
theory and quantum gravity.  A lot of this came from my attempts
to understand the relation between quantum gravity and
Chern-Simons theory, and Lee Smolin's work where he tries to use
this relation to derive the Bekenstein bound on the entropy of a
system in terms of its surface area (see "<A HREF =
"week56.html">week56</A>").   
An interesting little fact that I needed to use is that if you
have a two-part quantum system in a pure state --- a state of
zero entropy --- the two parts, regarded individually, can
themselves  have entropy, but the entropies of the two parts are
equal.  I worked this out using the symmetry of the situation but
Walter Thirring, who attended the talk, pointed out that it can
also be derived from a wonderful general fact: the triangle
inequality!  Namely, if your two-part system has entropy S, and
the two parts individually have entropies S1 and S2, then S can
never be less than |S1 - S2| or greater than S1 + S2.  (In
classical mechanics it's also true that S can never be less than
\emph{either} S1 \emph{or} S2, but this fails in quantum mechanics, where
for example you can have S be zero but S1 = S2 > 0.)  
Wednesday, August 1st - Full of excitement and new ideas, I
somewhat regretfully left the workshop and flew to London.  Then
I spent most of August working at Imperial College, thanks to a
kind offer of office space from Chris Isham. I had some nice
talks with Isham and his students on quantum gravity and the
decoherent histories approach to quantum mechanics.  I'll say a
bit about this in a while, but next Week I am going to talk about
triality and the secret inner meaning of E8.
<HR>

% </A>
% </A>
% </A>


% parser failed at source line 325
