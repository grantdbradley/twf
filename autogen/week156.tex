
% </A>
% </A>
% </A>
\week{September 17, 2000}


This week I want to catch you up on some of the experiments that have
been going on lately.  Mathematical physics is no fun without some 
experiments to think about now and then.  So here's some news about 
black holes, superfluid hydrogen, T violation, the \tau  neutrino, and 
the Higgs boson.  

I like black holes because they are a nice example of what general
relativity can do.  Once upon a time they seemed very exotic, but now 
it seems they're common.  In particular, there appear to be black holes
with masses between a million and several billion times that of the Sun
at the centers of all galaxies with a "bulge".  This includes galaxies
like the Milky Way, which has a central bulge in addition to a flat
spinning disk, and also elliptical galaxies, which consist solely of a
bulge.  Many of these supermassive black holes emit lots of X-rays
as they swallow hapless stars.  As I mentioned in <A HREF = "week144.html">week144</a>, the X-ray 
telescope Chandra has seen evidence for about 70 million of these
black holes! 
Recently, two teams of researchers have found that the mass of these
central black holes is correlated very closely to the dispersion of 
stellar velocities in the galaxy:
1) John Kormendy, Monsters at the heart of galaxy formation, 
Science 289 (2000), 1484-1485.  Available online at 
<A HREF = "http://www.sciencemag.org/cgi/content/full/289/5484/1484">http://www.sciencemag.org/cgi/content/full/289/5484/1484</A>

2) Laura Ferrarese and David Merritt, A fundamental relation between 
supermassive black holes and their host galaxies, 
Astrophys. J. Lett., 539, (2000) L9, preprint available as
<A HREF = "http://xxx.lanl.gov/abs/astro-ph/0006053">astro-ph/0006053</A>.

3) Karl Gebhardt et al, A relationship between nuclear black hole 
mass and galaxy velocity dispersion, Astrophys. J. Lett. 539, (2000) L13, 
preprint available as
<A HREF = "http://xxx.lanl.gov/abs/astro-ph/0006289">astro-ph/0006289</A>.

Tight correlations are a bit rare in astrophysics, so they tend 
to be important when they exist.  If you look at a graph you'll see 
how nice this one is:
4) Supermassive Black Hole Group, Theory of black holes
and galaxies, <A HREF = "http://www.physics.rutgers.edu/~merritt/theory.htm">http://www.physics.rutgers.edu/~merritt/theory.htm</A>

Ferrarese and Merrit estimate that the black hole mass grows as 
roughly the 4.8th power of the stellar velocity dispersion, which
they define as the standard deviation of the radial component of 
the velocities of stars in the galaxy.  

But what does this correlation \emph{mean}?  Astrophysicists are still
arguing about that.  But at the very least, it suggests an intimate
relation between supermassive black holes and the process of galaxy
formation.  

Part of the puzzle is that nobody knows how these supermassive
black holes formed.  You see, until very recently, all we've ever 
seen are small black holes formed by the collapse of a single star
(between 3 and 20 solar masses), and these supermassive ones at the
centers of galaxies.  But last year, people started seeing middle-
sized ones!  Colbert and Mushotzky found black holes between 100 
and 10,000 solar masses in about half of 30 nearby spiral and

elliptical galaxies that they examined:
5) Ed Colbert's homepage, <A HREF = "http://www.pha.jhu.edu/~colbert/">http://www.pha.jhu.edu/~colbert/</A>
E. J. M. Colbert and R. F. Mushotzky, The nature of accreting black 
holes in nearby galaxy nuclei, preprint available as 
<A HREF = "http://xxx.lanl.gov/abs/astro-ph/9901023">astro-ph/9901023</A>.

Ptak and Griffiths found a black hole of over 460 solar masses in
an irregular galaxy called M82:
6) A. Ptak, R. Griffiths, Hard X-ray variability in M82: evidence 
for a nascent AGN?, preprint available as <A HREF = "http://xxx.lanl.gov/abs/astro-ph/9903372">astro-ph/9903372</A>.

This is a "starburst galaxy", meaning that it's full of supernovae going 
off like a big firework display.  When a star dies in a supernova explosion, 
that's when a neutron star or black hole is formed - so it seems likely 
that this black hole in M82 was formed by the merger of several such 
black holes.  Could we be seeing the gradual formation of a supermassive
black hole?  

Maybe someday we'll understand the complete ecology of black holes.
I can't help but feel there's some important role they play which we
don't understand yet.  (For one theory about this, see the end of <A HREF = "week33.html">week33</A>.)

Now: you've all heard how helium-4 becomes a superfluid below 2.18 
kelvin and helium-3 does it below 2.4 millikelvin.   But what about
superfluid hydrogen?  Unlike helium, hydrogen is not a snobbish loner: 
it's a friendly, sticky molecule.  So usually it solidifies before it 
gets cold enough to go superfluid!   But in 1997, some folks at the 
University of Illinois noticed a possible loophole: films of liquid 
hydrogen about one molecule thick on a silver substrate should form 
a 2d superfluid at a temperature of 1.2 kelvin.   Here's a picture 
of a computer simulation:

7) David Ceperley et al, Prospective superfluid molecular hydrogen, 
<A HREF = "http://www.aip.org/physnews/graphics/html/h2.htm">http://www.aip.org/physnews/graphics/html/h2.htm</A>

Since then, other people have cooked up other schemes.  

Now it seems people have actually made the stuff.  Tiny amounts of it!  
The way they do it is to take superfluid helium and put in a bit of 
carbonyl sulfide (OCS) and hydrogen.  About 14 to 16 hydrogen molecules 
stick to the carbonyl sulfide molecule, and when the temperature drops 
to .15 kelvin, these molecules form a superfluid.  The hard part is 
checking experimentally that this really happens - and even \emph{defining}
what it means for a cluster of so few molecules to be a superfluid.   
I can't explain the details; for that you'll have to read the paper:

8) Slava Grebenev, Boris Sartakov, J. Peter Toennies, and 
Andrei F. Vilesov, Evidence for superfluidity in para-hydrogen 
clusters inside helium-4 droplets at 0.15 Kelvin, Science 5484 
(2000), 1532-1535, available online at 
<A HREF = "http://www.sciencemag.org/cgi/content/abstract/289/5484/1532">http://www.sciencemag.org/cgi/content/abstract/289/5484/1532</A>

Here "para-hydrogen" refers to a molecule of hydrogen where the spins 
on the two nuclei are anti-parallel - as opposed to "ortho-hydrogen",
where they're lined up.  The two states have different properties and
this matters a lot in delicate situations like these.

Next: T violation.  Once people thought the laws of physics
were symmetrical under exchanging either particles with their
antiparticles, left with right, or future with past.  These three 
symmetries are called C (for "charge conjugation"), P (for "parity") 
and T (for "time reversal").  The weak interaction is now believed
to violate all of these.   

Very briefly, the story goes like this: Yang and Lee won the Nobel 
prize for helping discover P violation in the \beta  decay of radioactive
cobalt back in 1956, though in retrospect it was only the sexism 
of the Nobel committee that prevented Wu from sharing this prize - 
she did the actual experiment.   In \beta  decay, a neutron turns into
a proton, an electron and an electron anti-neutrino via the weak
interaction.  Since the electron anti-neutrino only comes in a
right-handed form, this process violates P symmetry.

Cronin and Fitch won the Nobel prize for discovering in 1964 that 
neutral kaons decay in a way that violates CP symmetry - i.e., the 
symmetry where you switch particles with their antiparticles \emph{and} 
switch left with right.  I believe that neutral kaons are still the 
only system where CP violation has been seen.  

Now there's something called the CPT theorem which says that various
reasonable axioms for a quantum field theory imply symmetry under the
\emph{combination} of C, P and T.  For the math of this, the obvious place
to go is this classic text on axiomatic quantum field theory:

9) R. F. Streater and A. S. Wightman, PCT, Spin and Statistics, and
All That, Addison-Wesley, Reading, Massachusetts, 1989.

In case you're worried, PCT is the same thing as CPT.  I like this
book a lot.  The only thing I dislike is how it unleashed a flood of 
physics papers whose titles end with "and all that".  For example:

"CFT, BCFT, ADE and all that"

"Quantum cohomology and all that"

"String theory, supersymmetry, unification, and all that"

"Anti-de Sitter space, branes, singletons, superconformal field theories 
and all that" 

"The modified Bargmann-Wigner formalism: longitudinal fields, parity and 
all that"

"The Zamolodchikov C-Function, classical closed string field theory, the 
Duistermaat-Heckman theorem, the renormalization group, and all that"

Enough!  Listen, guys: it was funny once, but now it's just lame.  Stop it!

But I digress.  Where was I?   Oh yeah: given the CPT theorem, from 
CP violation we can conclude T violation.  The future and the past 
are slightly different - but of all the known forces, only the weak 
force notices the difference!  This is bizarre and fascinating.  But 
the way we reached this conclusion was not completely satisfying, since 
we needed to assume the usual axioms of quantum field theory to get the 
CPT theorem.  What if the axioms are wrong?  It would be better to have 
more \emph{direct} evidence of T violation, given how important this issue is.  

So in the late 1990s, people in the CPLEAR collaboration at CERN did 
some precision experiments on neutral kaon decay, and found more direct
evidence of T violation!

10) CPLEAR homepage, <A HREF = "http://cplear.web.cern.ch/cplear/Welcome.html">http://cplear.web.cern.ch/cplear/Welcome.html</A>

11) CPLEAR collaboration, First direct observation of time-reversal 
non-invariance in the neutral kaon system, Phys. Lett. B 444 (1998) 43,
available online with all other papers by this collaboration at
<A HREF = "http://cplear.web.cern.ch/cplear/cplear_pub.html">http://cplear.web.cern.ch/cplear/cplear_pub.html</A>

Now we can all sit back and rack our brains even harder about what
T violation really \emph{means}.  So far, all we know is that it arises
from the darkest corner of the Standard Model: the Kobayashi-Maskawa 
matrix.  This is a matrix describing quarks' couplings to the Higgs.
The fact that it's not diagonal means that the "flavor eigenstates"
of the quarks - up and down, strange and charmed, bottom and top - 
are not the "mass eigenstates".   Why does the Kobayashi-Maskawa 
matrix equal what it equals?   Why is it of a form that violates 
T symmetry?  Nobody knows.

Another nice confirmation of what we already believed was the recent
discovery of direct evidence for the \tau  neutrino.  If you don't
remember the particles in the Standard Model, try <A HREF = "week119.html">week119</A>: you'll
see that it has 3 generations of quarks (listed above) and 3 generations 
of leptons: the electron, muon and \tau  and their corresponding neutrinos.
Of the leptons, the \tau  is the heaviest and thus hardest to produce.  Tau 
neutrinos are produced by the decay of \tau  particles, but since it's hard
to make these particles and hard to catch neutrinos, until recently nobody 
had ever done the clinching experiment: creating a beam of a \tau  neutrinos 
and letting it collide with some stuff to form \tau  particles again.   

On July 21st, 2000, the DONUT collaboration at Fermilab announced that 
they had successfully done this experiment:

12) Christina Hebert, Phyisicists find first direct evidence for \tau 
neutrino at Fermilab, 
<A HREF = "http://www.fnal.gov/directorate/public_affairs/story_neutrino/p1.html">http://www.fnal.gov/directorate/public_affairs/story_neutrino/p1.html</A>

In case you're wondering, "DONUT" stands for "Direct Observation of 
the Nu Tau", where \nu _{\tau } is the standard abbrevation for \tau  neutrino.

In short, the final details of the Standard Model are all falling
into place just as expected - except for the fact that neutrinos are
doing lots of weird stuff they shouldn't be doing!  As I explained in 
<A HREF = "week130.html">week130</A>, neutrino physics is the big place for surprises in particle
physics these days.  This is yet another reason why it was good to 
directly observe the \tau  neutrino.

And then, of course, there's the Higgs - the final particle in the
Standard Model.  As you've probably heard, we're getting awfully close
to seeing it - or at least definitively \emph{not} seeing it.   Right now
they're looking for it at LEP - the big particle accelerator at
CERN, in Geneva.  They're just about to shut LEP down, since it's 
done pretty much all it can do, and they need to deactivate it to 
build an even more powerful accelerator - LHC, the Large Hadron  
Collider.  But at the last minute they decided to extend its life 
to November 2nd, 2000:

13) LEP shutdown postponed by one month, 
<A HREF = "http://press.web.cern.ch/Press/Releases00/PR08.00ELEPRundelay.html">http://press.web.cern.ch/Press/Releases00/PR08.00ELEPRundelay.html</A>

They're going for broke, boosting its power to the utter max, so 
that they can see hints of the Higgs as long as its mass is 114
GeV or so.  In fact they have already seen a couple of events that
suggest a Higgs of about this mass.  

Whether or not LEP sees the Higgs the folks at the Tevatron at Fermilab
should see it when they start Run II in a while, as long as its mass
below 130 GeV.  And if \emph{they} don't see it, folks at CERN should see it
with the LHC accelerator by around 2005, as long as its mass is below
180 GeV.  A Higgs more massive than that would mean the Standard Model
is seriously screwed up, so at that point, even \emph{not} seeing the Higgs
would be an important discovery.

The folks getting ready to analyze the Run II data at the Tevatron are
doing so with a few theories in mind: the Standard Model, the minimal
supersymmetric extension of the Standard Model, and a "next-to-minimal"
supersymmetric extension.  This is a major project; you can find lots
of details here:

14) Higgs Working Group webpage, <A HREF = "http://fnth37.fnal.gov/higgs/higgs.html">http://fnth37.fnal.gov/higgs/higgs.html</A>


That's basically it for this week.  I just have a couple of
questions about CPT.  A while back on sci.physics.research I emphasized
a little theorem that says: any self-dual irreducible unitary group
representation H must admit an antiunitary intertwiner J: H \to  H with
either J^{2} = 1 or J^{2} = -1.  In the first case H
comes from a real representation; in the second case it comes from a
quaternionic representation.  For more details, try this:

15) John Baez, Symplectic, quaternionic, fermionic, 
<A HREF = "http://math.ucr.edu/home/baez/symplectic.html">http://math.ucr.edu/home/baez/symplectic.html</A>

Now, after I mentioned this, someone who goes by the name of "squark"
suggested that the CPT operator for massive spin-1/2 particles was a
an antiunitary intertwiner with (CPT)^{2} = -1.  I'm not sure this is
true, but it's definitely antiunitary, so we have an intesting question:
which unitary irreducible representations of the Poincare group are
self-dual?  Of these, which come from real representations and which
come from quaternionic ones?  My hunch is that the bosonic (i.e. 
integral-spin) reps are real and the fermionic (i.e. half-integral-spin) 
reps are quaternionic.  And then the question is: is the operator J
just the the CPT operator?  This would certainly shed some nice mathematical
light on the meaning of CPT symmetry.  

By the way, This Week's Finds has a nice new feature, courtesy of
Laurent Bartholdi: now you can search all the old issues for a keyword
or phrase!  This is very useful, at least for me.  Check it out on 
my website.

\par\noindent\rule{\textwidth}{0.4pt}


Footnotes: 
Squark found in Volume 1 of Weinberg's "Quantum Field
Theory" that the CPT operator on the Hilbert space of a spin-j
representation of the Poincare group is an antiunitary operator
with (CPT)^{2} = -1^{2j}.  So indeed we do have 
(CPT)^{2} = 1 in the bosonic case, making these representations
real, and (CPT)^{2} = -1 in the fermionic case, making these
representations quaternionic.


 Allen Knutson points out that Streater and Wightman's title "PCT,
Spin and Statistics, and All That" was itself modelled after that of
Sellar and Yeatman's humorous history: "1066 and all that; a memorable
history of England, comprising all the parts you can remember including
one hundred and three good things, five bad kings and two genuine dates."

Martin Hardcastle wonders if Streater and Wightman were inspired by the
similarity of their names to those of Sellar and Yeatman!







 \par\noindent\rule{\textwidth}{0.4pt}

% </A>
% </A>
% </A>
