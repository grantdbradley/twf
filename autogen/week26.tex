
% </A>
% </A>
% </A>
\week{ovember 21, 1993}

I have been struggling to learn the rudiments of Teichmueller 
theory, and it's almost time for me to face up to my ignorance of it by
posting a "This Week's Finds" attempting to explain the stuff, but I am
going to put off the inevitable and instead describe a variety of papers
on different subjects... 

\par\noindent\rule{\textwidth}{0.4pt}

1) Cosmology, time's arrow, and that old double standard, by Huw Price,
26 pages, in LaTeX with 3 figures appended as postscript file, available
as <A HREF = "http://xxx.lanl.gov/abs/gr-qc/9310022">gr-qc/9310022</A>.  (Written for Time's Arrows Today Conference, UBC,
Vancouver, June 1992; forthcoming in Savitt, S., ed., "Time's Arrows
Today," Cambridge University Press, 1994.)

Why is the future different from the past?  Because it hasn't
happened yet?  Well, sure, but that's not especially enlightening, in
fact, it's downright circular.  Unfortunately, a lot of work on the
"arrow of time" is just as circular, only so erudite that it is hard to
spot it!  That's what this article takes some pains to clarify.  I think
I will be lazy and quote the beginning of the paper:

\par\noindent\rule{\textwidth}{0.4pt}
A century or so ago, Ludwig Boltzmann and others attempted to explain
the temporal asymmetry of the second law of thermodynamics.  The
hard-won lesson of that endeavour - a lesson still commonly
misunderstood - was that the real puzzle of thermodynamics lies not in
the question why entropy increases with time, but in that as to why it
was ever so low in the first place.  To the extent that Boltzmann
himself appreciated that this was the real issue, the best suggestion he
had to offer was that the world as we know it is simply a product of a
chance fluctuation into a state of very low entropy.  (His statistical
treatment of thermodynamics implied that although such states are
extremely improbable, they are bound to occur occasionally, if the
universe lasts a sufficiently long time.) This is a rather desperate
solution to the problem of temporal asymmetry, howevet, and one of the
great achievements of modern cosmology has been to offer us an
alternative. It now appears that temporal asymmetry is cosmological in
origin, a consequence of the fact that entropy is much lower than its
theoretical maximum in the region of the Big Bang - i.e., in what we
regard as the \emph{early} stages of the universe.  

The task of explaining temporal asymmetry thus becomes the task of
explaining this condition of the early universe. In this paper I want to
discuss some philosophical constraints on the search for such an
explanation. In particular, I want to show that cosmologists who discuss
these issues often make mistakes which are strikingly reminiscent of
those which plagued the nineteenth century discussions of the
statistical foundations of thermodynamics. The most common mistake is to
fail to recognise that certain crucial arguments are blind to temporal
direction, so that any conclusion they yield with respect to one
temporal direction must apply with equal force with respect to the
other. Thus writers on thermodynamics often failed to notice that the
statistical arguments concerned are inherently insensitive to temporal
direction, and hence unable to account for temporal asymmetry. And
writers who did notice this mistake commonly fell for another:
recognising the need to justify the double standard - the application of
the arguments in question `towards the future' but not `towards the
past' - they appealed to additional premisses, without noticing that in
order to do the job, these additions must effectively embody the very
temporal asymmetry which was problematic in the first place. To assume
the uncorrelated nature of initial particle motions (or incoming
`external influences'), for example, is simply to move the problem from
one place to another. (It may \emph{look} less mysterious as a result, but
this is no real indication of progress. The fundamental lesson of these
endeavours is that much of what needs to be explained about temporal
asymmetry is so commonplace as to go almost unnoticed. In this area more
than most, folk intuition is a very poor guide to explanatory priority.)

One of the main tasks of this paper is to show that mistakes of these
kinds are widespread in modern cosmology, even in the work of some of
the contemporary physicists who have been most concerned with the
problem of the cosmological basis of temporal asymmetry - in the course
of the paper we shall encounter illicit applications of a temporal
double standard by Paul Davies, Stephen Hawking and Roger Penrose, among
others. Interdisciplinary point- scoring is not the primary aim, of
course: by drawing attention to these mistakes I hope to clarify the
issue as to what would count as adequate cosmological explanation of
temporal asymmetry.
 
I want to pay particular attention to the question as to whether it is
possible to explain why entropy is low near the Big Bang without thereby
demonstrating that it must be low near a Big Crunch, in the event that
the universe recollapses. The suggestion that entropy might be low at
both ends of the universe was made by Thomas Gold in the early 1960s.
With a few notable exceptions, cosmologists do not appear to have taken
Gold's hypothesis very seriously. Most appear to believe that it leads
to absurdities or inconsistencies of some kind.  However, I want to show
that cosmologists interested in time asymmetry continue to fail to
appreciate how little scope there is for an explanation of the low
entropy Big Bang which does not commit us to the Gold universe.  I also
want criticise some of the objections that are raised to the Gold view,
for these too often depend on a temporal double standard. And I want to
discuss, briefly and rather speculatively, some issues that arise if we
take the view seriously. (Could we observe a time-reversing future, for
example?)

[And now let me jump forward to a very interesting issue, Hawking's
attempt to derive the arrow of time from his "no-boundary boundary
conditions" choice of the wavefunction of the universe.  (See "<A HREF = "week3.html">week3</A>.")
I found this rather unsatisfying when I read it, and had a sneaking
suspicion that he was falling into the fallacy Gold describes above.
Let's hear what Price has to say about it. - jb]

Our second example is better known, having been described in Stephen
Hawking's best seller, A Brief History of Time.  It is Hawking's
proposal to account for temporal asymmetry in terms of what he calls the
No Boundary Condition (NBC) - a proposal concerning the quantum wave
function of the universe. To see what is puzzling about Hawking's claim,
let us keep in mind the basic dilemma.  It seemed that provided we avoid
double standard fallacies, any argument for the smoothness of the
universe would apply at both ends or at neither. So our choices seemed
to be to accept the globally symmetric Gold universe, or to resign
ourselves to the fact that temporal asymmetry is not explicable (without
additional assumptions or boundary conditions) by a time-symmetric
physics.  The dilemma is particularly acute for Hawking, because he has
a more reason than most to avoid resorting to additional boundary
conditions. They conflict with the spirit of his NBC, namely that one
restrict possible histories for the universe to those that `are finite
in extent but have no boundaries, edges, or singularities.'

Hawking tells us how initially he thought that this proposal favoured
the former horn of the above dilemma: `I thought at first that the no
boundary condition did indeed imply that disorder would decrease in the
contracting phase.'  He changed his mind, however, in response to
objections from two colleagues: `I realized that I had made a mistake:
the no boundary condition implied that disorder would in fact continue
to increase during the contraction. The thermodynamic and psychological
arrows of time would not reverse when the universe begins to contract or
inside black holes.'

This change of mind enables Hawking to avoid the apparent difficulties
associated with reversing the thermodynamic arrow of time. What is not
clear is how he avoids the alternative difficulties associated with 
\emph{not} reversing the thermodynamic arrow of time. That is, Hawking
does not explain how his proposal can imply that entropy is low near the
Big Bang, without equally implying that it is low near the Big Crunch.
The problem is to get a temporally asymmetric consequence from a
symmetric physical theory. Hawking suggests that he has done it, but
doesn't explain how. Readers are entitled to feel a little dissatisfied.
As it stands, Hawking's account reads a bit like a suicide verdict on a
man who has been stabbed in the back: not an impossible feat, perhaps,
but we'd like to know how it was done!
 
It seems to me that there are three possible resolutions of this
mystery. The first, obviously, is that Hawking has found a way round the
difficulty. The easiest way to get an idea of what he would have to have
established is to think of three classes of possible universes: those
which are smooth and ordered at both temporal extremities, those which
are ordered at one extremity but disordered at the other, and those
which are disordered at both extremities. If Hawking is right, then he
has found a way to exclude the last class, without thereby excluding the
second class. In other words, he has found a way to exclude disorder at
one temporal extremity of the universe, without excluding disorder at
both extremities. Why is this combination the important one? Because if
we can't exclude universes with disorder at both extremities, then we
haven't explained why our universe doesn't have disorder at both
extremities - we know that it has order at least one temporal extremity,
namely the extremity we think of as at the beginning of time. And if we
do exclude disorder at both extremities, we are back to the answer that
Hawking gave up, namely that order will increase when the universe
contracts.
 
Has Hawking shown that the second class of universal histories, the
order-disorder universes, are overwhelmingly probable? It is important
to appreciate that this would not be incompatible with the underlying
temporal symmetry of the physical theories concerned. A symmetric
physical theory might be such that all or most of its possible
realisations were asymmetric. Thus Hawking might have succeeded in
showing that the NBC implies that any (or almost any) possible history
for the universe is of this globally asymmetric kind. If so, however,
then he hasn't yet explained to his lay readers how he managed it. In a
moment I'll describe my attempts to find a solution in Hawking's
technical papers. What seems clear is that it can't be done by
reflecting on the consequences of the NBC for the state of one temporal
extremity of the universe, considered in isolation. For if that worked
for the `initial' state it would also work for the `final' state; unless
of course the argument had illicitly assumed an objective distinction
between initial state and final state, and hence applied some constraint
to the former that it didn't apply to the latter. What Hawking needs is
a more general argument, to the effect that disorder-disorder universes
are impossible (or at least overwhelmingly improbable). It needs to be
shown that almost all possible universes have at at least one ordered
temporal extremity - or equivalently, at most one disordered extremity.
(As Hawking points out, it will then be quite legitimate to invoke a
weak anthropic argument to explain why we regard the ordered extremity
thus guaranteed as an \emph{initial} extremity. In virtue of its consequences
for temporal asymmetry elsewhere in the universe, conscious observers
are bound to regard this state of order as lying in their past.)
 
That's the first possibility: Hawking has such an argument, but hasn't
told us what it is (probably because he doesn't see why it is so
important).  As I see it, the other possibilities are that Hawking has
made one of two mistakes (neither of them the mistake he claims to have
made). Either his NBC does exclude disorder at both temporal extremities
of the universe, in which case his mistake was to change his mind about
contraction leading to decreasing entropy; or the proposal doesn't
exclude disorder at either temporal extremity of the universe, in which
case his mistake is to think that the NBC accounts for the low entropy
Big Bang.

[And, eventually, Price concludes that there is indeed something lacking
in Hawking's attempts to derive an arrow of time. - jb]


\par\noindent\rule{\textwidth}{0.4pt}
I have said this many times here and there, but I'll say it again.  For
a good introduction to these issues, read:

2) The Physical Basis of the Direction of Time, by H. D. Zeh, Second
Edition, Springer-Verlag, 1992.

Zeh is one of the most clear-headed writers I know on this vexing
problem.  Interestingly, Price acknowledges Zeh in his paper.

\par\noindent\rule{\textwidth}{0.4pt}

3)  Chromodynamics and gravity as theories on loop space, by R.
Loll, 56 pages, 10 figures (postscript, compressed and uuencoded), TeX,
available as <A HREF = "http://xxx.lanl.gov/abs/hep-th/9309056">hep-th/9309056</A>.

This is an especially thorough review of work on the loop representation
of gauge theories, especially the theories of the strong force and
gravity.  A lot of work has been done on this subject but there are
still very many basic mathematical problems when it comes to making any
of this work rigorous, and one nice thing about Loll's work is that she
is just as eager to point out the problems as the accomplishments.  It
can be dangerous when people become complacent and simply shrug off
various problems just because they are difficult and can be temporarily
ignored.  

\par\noindent\rule{\textwidth}{0.4pt}

4) Intersecting braids and intersecting knot theory, by Daniel Armand-Ugon,
Rodolfo Gambini and Pablo Mora, Latex 14 pages (6 figures included),
available as <A HREF = "http://xxx.lanl.gov/abs/hep-th/9309136">hep-th/9309136</A>. 

There are a lot of hints that classical knot theory, which only
considers a circle smoothly embedded in space, is only the tip of a very
interesting iceberg.  Namely, if one looks at the space of all loops,
this has the knots as an open dense subset, but then it has loops with a
single ``transverse double point'' like

\begin{verbatim}
\    /
 \  /
  \/
  /\
 /  \
/    \
\end{verbatim}
    

as a codimension 1 subset (like a hypersurface in the space of all
loops), and more fancy singularities appear as still smaller subsets, or
as the jargon has it, strata of higher codimension.  The recent flurry
of work on Vassiliev invariants points out the importance of these other
strata - or at least a few - to knot theory.  Namely, knot invariants
that extend nicely to knots with arbitrarily many transverse double
points include the famous quantum group knot invariants like the Jones
polynomial, and there is a close relationship between these "Vassiliev
invariants" and Lie algebra theory.

Meanwhile, the physicists have been forging ahead into more complicated
strata, motivated mainly by the loop representation of quantum gravity.
Gambini is one of the originators of the loop representation of gauge
theory, so it is not surprising that he is ahead of the game on this
business.  Together with Pullin and Bruegmann he has been working on
extending the Jones polynomial, for example, to loops with various sorts
of self-intersections, calculating these extensions directly from the
path-integral formula for the Jones polynomial as a Wilson loop
expectation value in Chern-Simons theory.  The relationship of this
extension to the theory of Vassiliev invariants was recently clarified
by Kauffman (see "<A HREF = "week23.html">week23</A>"), but there is much more to do.  Here Gambini
and collaborators look at loops with transverse triple points.  I guess
I'll just quote the abstract:

"An extension of the Artin Braid Group with new operators that generate
double and triple intersections is considered. The extended Alexander
theorem, relating intersecting closed braids and intersecting knots is
proved for double and triple intersections, and a counter example is
given for the case of quadruple intersections. Intersecting knot
invariants are constructed via Markov traces defined on intersecting
braid algebra representations, and the extended Turaev representation is
discussed as an example. Possible applications of the formalism to
quantum gravity are discussed."
<HR>

% </A>
% </A>
% </A>


% parser failed at source line 364
