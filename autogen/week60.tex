
% </A>
% </A>
% </A>
\week{August 8, 1995 }

The end of a sabbatical is a somewhat sad affair... so many plans
one had, and so few accomplished!  As I pack my bags to return from 
Cambridge England to Cambridge Massachusetts, and then
wing my way back to Riverside, I find I have quite a stack of preprints 
that I meant to include in This Week's Finds, but haven't gotten around 
to yet.  

1) N. P. Landsman, Rieffel induction as generalized quantum Marsden-
Weinstein reduction, Journal of Geometry and Physics 15 (1995), 285-319.

Marsden-Weinstein reduction, also called symplectic reduction,
is the modern way to deal with constraints in classical mechanics
problems.  It's a two-step procedure where first one looks at the
subspace of your phase space on which the constraints vanish, and
then a quotient of this by a certain equivalence relation.  For example,
if you have a particle in a plane, its phase space is R^4, with coordinates
(x,y,p_x,p_y) representing the x and y components of the position and
the x and y components of the momentum.  If we have a constraint
x = 0, Marsden-Weinstein reduction tells us first to form the subspace
of our phase space on which x = 0, and then quotient by the equivalence
relation where two points are equivalent if they have the same value of
p_x.  We get down to R^2, with coordinates (y,p_y).  But Marsden-
Weinstein reduction works in great generality and has become a basic
part of the toolkit of mathematical physics.  

What's the quantum analog of Marsden-Weinstein reduction?  That's
what this paper is about.  Quantum mechanics in the presence of
constraints is a tricky and important business, and there are lots of theories
about how to do it.   Gauge theories all have constraints, and so does
general relativity... and in quantizing general relativity, the presence of
constraints is what gives rise to the "problem of time".  (See "<A HREF = "week43.html">week43</A>"
for more on this.)  What attracted my attention to this paper is a two-stage 
procedure for dealing with contraints, quite analogous to Marsden-Weinstein 
reduction.  This should shed some interesting light on the problem of time.


2) T. Ohtsuki, Finite type invariants of integral homology 3-spheres, preprint,
1994.

L. Rozansky, The trivial connection contribution to Witten's invariant
and finite type invariants of rational homology spheres, preprint
available as
<A HREF = "http://xxx.lanl.gov/abs/q-alg/9504015">q-alg/9505015</A>.

Stavros Garoufalidis, On finite type 3-manifold invariants I, MIT preprint,
1995.

Stavros Garoufalidis and Jerome Levine, On finite type 3-manifold
invariants II, MIT preprint, June 1995.   (Garoufalidis is at 
stavros@math.mit.edu, and Levine is at levine@max.math.brandeis.edu.)

Ruth J. Lawrence, Asymptotic expansions of Witten-Reshetikhin-Turaev 
invariants for some simple 3-manifolds, to appear in Jour. Math. Physics.

Chern-Simons theory gives invariant of links in R^3, which are functions
of Planck's constant h, and if one expands them as power series in h,
the coefficients are link invariants with special properties, which one 
summarizes by calling them "Vassiliev invariants" or "invariants of 
finite type".  (See "<A HREF = "week3.html">week3</A>" for more.)   But the partition function of 
Chern-Simons theory on a compact oriented 3-manifold is also interesting; 
it's an invariant of the 3-manifold defined for certain values of h. 
(Often instead one thinks of it instead as a function of a quantity q, 
the limit q \to  1 corresponding to the limit h \to  0.)  

Recently people have studied the partition function of special 3-manifolds 
called homology spheres, which have the same homology as S^3.  (People have 
looked at both integral and rational homology spheres.)   After a bit of 
subtle fiddling, one can extract from the partition function of a homology 
sphere a power series in 

\begin{verbatim}
                          h' = q - 1,
\end{verbatim}
    

and the coefficients of the powers of h' have been conjectured by
Rozansky to have nice properties which one may summarize by calling them
"finite type" invariants, in analogy to the link invariant case.
(Namely, that they transform in nice ways under Dehn surgery.)  For
example, the coefficient of h' itself is 6 times the Casson invariant of
the (integral) homology 3-sphere.  So there appears to be a budding
branch of "perturbative 3-manifold invariant theory".  I just wish I
understood better what's really going on behind all this!

3) Thomas Friedrich, Neue Invarianten der 4-dimensionalen Mannigfaltigkeiten,
Berlin preprint.  

This is a nice introduction to the new Seiberg-Witten approach to Donaldson
theory, which does not assume you already know the old stuff by heart.  Very
pretty mathematics!


4) Andre Joyal, Ross Street, and Dominic Verity, Traced monoidal categories,
to appear in Math. Proc. Camb. Phil. Soc..

This is an abstract characterization of monoidal categories (categories
with tensor products) which have a good notion of the "trace" of a morphism.
Many abstract treatments of traces assume that your category is "rigid 
symmetric" or "balanced", meaning that your objects have duals and you can 
switch around objects in order to define the trace of a 
morphism f: V \to  V in a manner analogous to how one usually does it in 
linear algebra, as a certain composite:

$$
                                  f tensor 1
               I \to  V tensor V* ---------------> V tensor V* \to  I
$$
    

where I is the "unit object" for the tensor product (e.g. the 
complex numbers, when we're working in the category of vector
spaces.)  But one does not really need all this extra structure 
if all one wants is a good notion of "trace".  This paper isolates
the bare minimum.  As one might expect if one knows the relation
between knot theory and category theory, there are lots of
nice pictures of tangles in this paper!


5)  Michael Reisenberger, Worldsheet formulations of gauge theories
and gravity, University of Utrecht preprint, 1994, available as 
<A HREF = "http://xxx.lanl.gov/abs/gr-qc/9412035">gr-qc/9412035</A>.

The loop representation of a gauge theory describes states as linear
combinations of loops in space, or more generally, "spin networks".  
What's the spacetime picture of which this is a spacelike slice?  The
obvious thing that comes to mind is a two-dimensional surface of some
sort.  I've advocated this point of view myself in an attempt to relate
the loop representation of gravity to string theory (see "<A HREF = "week18.html">week18</A>").  Here
Reisenberger makes some progress in making this precise for some
simpler theories analogous to gravity --- for example, BF theory.  

And now for some things I \emph{did} manage to finish up on my sabbatical:


6)  John Baez and Stephen Sawin, Functional integration on spaces
of connections, available as <A HREF = "http://xxx.lanl.gov/abs/q-alg/9507023">q-alg/9507023</A>.

As I described in "<A HREF = "week55.html">week55</A>", it's now possible to set up a rigorous
version of the loop representation without assuming (as had earlier
been required) that ones manifold is real-analytic and the loops are 
all analytic.  This means that one can do things in a manner invariant
under all diffeomorphisms, not just analytic ones.  To achieve this,
one needs to ponder rather carefully the complicated ways smooth
paths, even embedded ones, can intersect (for example, they can intersect
in a Cantor set).   


7) John Baez, Javier P. Muniain and Dardo Piriz, Quantum gravity
hamiltonian for manifolds with boundary, available as <A HREF = "http://xxx.lanl.gov/abs/gr-qc/9501016">gr-qc/9501016</A>.

When space is a compact manifold with boundary, there is no Hamiltonian
in quantum gravity, just a Hamiltonian constraint (see "<A HREF = "week43.html">week43</A>").  This
makes it tricky to understand time evolution in the theory --- the "problem
of time".  But with a boundary, there is a Hamiltonian, given by a surface
integral over the boundary.  (The reason is that, at least when the equations
of motion hold, the Hamiltonian is a total divergence, so you can use
Gauss' theorem to express it as an integral over the boundary, which of
course is zero when there is no boundary.)  

Rovelli and Smolin (see "<A HREF = "week42.html">week42</A>") worked out a loop representation
of quantum gravity --- in a heuristic sort of way which various slower
sorts like myself have been struggling to make rigorous in the subsequent
years --- and a key step in this was expressing the Hamiltonian constraint
in terms of loops.  In this paper we do the same sort of thing for the
Hamiltonian, when there is a boundary.  This requires considering not
only loops but also paths that start and end at the boundary.  

Remarkably, the Hamiltonian acts on paths that start and end at the 
boundary in a manner very similar to the Hamiltonian constraint for 
quantum gravity coupled to massless chiral spinors (e.g. neutrinos, if 
neutrinos are really massless and have a "handedness" as they appear to).  
This suggests that on a manifold with boundary, the degrees of freedom 
"living on the boundary" are described by a chiral spinor field.  Steve Carlip 
has already shown something very similar for quantum gravity in 2+1
dimensional spacetime, a more tractable simplified model --- see "<A HREF = "week41.html">week41</A>".
Moreover, he used this to explain why the entropy of a black hole is 
proportional to its area (or length in 2+1 dimensions).  The idea is that 
the entropy is really accounted for by the degrees of freedom of the event 
horizon itself.  It would be nice to do something similar in 3+1-dimensional 
spacetime.
<HR>

% </A>
% </A>
% </A>


% parser failed at source line 257
