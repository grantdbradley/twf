
% </A>
% </A>
% </A>
\week{November 10, 2009 }


We had a great AMS meeting this weekend at UCR, with far too many 
interesting talks going on simultaneously.  For example, there were 
two sessions on math related to knot theory, one on operator algebras, 
one on noncommutative geometry, and one on homotopy theory and higher 
algebraic structures!  If I could clone myself, I'd have gone to all of
them.  

I'd like to discuss some of the talks, and maybe even point you to
some videos.  But the videos aren't available yet, so for now I'll
just summarize my own talk on "Who Discovered the
Icosahedron", and some geometry related to that.  I'll conclude
with a puzzle.

But first - the astronomy pictures of the week!

Galaxies are beautiful things, and there are lots of ways to enjoy 
them.  Here's the Milky Way in visible light - a detailed panorama 
created from over 3000 individual pictures, carefully calibrated to 
show large dust clouds:

<div align = "center">
<a href = "milky_way_axel_mellinger.jpg">
<img width = "700" src = "milky_way_axel_mellinger.jpg">
% </a>
</div>

1) Axel Mellinger, All-sky Milky Way panorama 2.0,
<a href = "http://home.arcor.de/axel.mellinger/">
http://home.arcor.de/axel.mellinger/</a>

You can see even more structure in this infrared panorama of the Milky 
Way, created by the Spitzer Space Telescope:

<div align = "center">
<a href = "milky_way.html">
<img width = "700" src = "milky_way_small.jpg">
% </a>
</div>

2) Astronomy Picture of the Day, GLIMPSE the Milky Way,
<a href = "http://apod.nasa.gov/apod/ap051216.html">http://apod.nasa.gov/apod/ap051216.html</a>

The bright white splotches are star-forming regions.   The greenish 
wisps are hot interstellar gas.  The red clouds are dust and organic 
molecules like polycyclic aromatic hydrocarbons (see "<a href = "week258.html">week258</A>").  The 
darkest patches are regions of cool dust too thick for Spitzer to see 
through.

But here's my favorite: the Andromeda Galaxy in viewed in ultraviolet
light:

<div align = "center">
<a href = "http://apod.nasa.gov/apod/image/0909/UVAndromeda_swiftH600.jpg">
<img width = "700" src = "andromeda_ultraviolet_swift.jpg">
% </a>
</div>

3) Astronomy Picture of the Day, Ultraviolet Andromeda, 
<a href = "http://apod.nasa.gov/apod/ap090917.html">http://apod.nasa.gov/apod/ap090917.html</a>

This was taken by Swift, NASA's ultraviolet satellite telescope.
At this frequency, young hot stars and dense star clusters dominate 
the view.  It's sort of ghostly looking, no?

Now for my talk on the early history of the icosahedron. 

<div align = "center">
<img border = "none" src = "icosahedron/Icosahedron.gif">
% </a>
</div>

 This
continues the tale begun in "<a href =
"week236.html">week236</A>" and "<a href =
"week241.html">week241</A>".  Someday it'll get folded into a
paper on special properties of the number 5, and 5-fold symmetry:

4) John Baez, Who discovered the icosahedron?, talk at the Special 
Session on History and Philosophy of Mathematics, 2009 Fall Western 
Section Meeting of the AMS, November 7, 2009. 
Available at <a href = "http://math.ucr.edu/home/baez/icosahedron/">
http://math.ucr.edu/home/baez/icosahedron/</a>

The dodecahedron and icosahedron are the most exotic of the Platonic
solids, because they have 5-fold rotational symmetry - a possibility
that only exists for regular polytopes in 2, 3 or 4 dimensions.  The
dodecahedron and icosahedron have the same symmetry group, because 
they are Poincar&eacute; duals: the vertices of one correspond to faces of
the other.  But the icosahedron was probably discovered later.  As 
Benno Artmann wrote:

\begin{quote}
  The original knowledge of the dodecahedron may have come from 
  crystals of pyrite, but in contrast the icosahedron is a pure 
  mathematical creation.... It is the first realization of an 
  entity that existed before only in abstract thought.  (Well,  
  apart from the statues of gods!)
\end{quote}

I'm not sure it's really anything close to the first "realization of 
an entity that existed before only in abstract thought".  But 
it may have been the first "exceptional" object in mathematics - 
roughtly speaking, an entity that doesn't fit into any easy pattern,
which is discovered as part of proving a classification theorem!   

Other exceptional objects include the simple Lie group E_{8},
and the finite simple group M_{12}.  Intriguingly, many of
these exceptional objects" are related.  For example, the
icosahedron can be used to construct both E_{8} and
M_{12}.  But the first interesting classification theorem was
the classification of regular polyhedra: convex polyhedra with
equilateral polygons as faces, and the same number of faces meeting at
each vertex.  This theorem appears almost at the end of the last book
of Euclid's Elements - Book XIII.  It shows that the only
possibilities are the Platonic solids: the tetrahedron, the cube, the
octahedron, the dodecahedron and the icosahedron. And according to
traditional wisdom, the results in this book were proved by <a href = "http://en.wikipedia.org/wiki/Theaetetus_%28mathematician%29">Theatetus</a>,
who also discovered the icosahedron!

Indeed, Artmann cites an "an ancient note written in the margins of
the manuscript" of Book XIII, which says:

\begin{quote}
  In this book, the 13th, are constructed the five so-called Platonic 
  figures which, however, do not belong to Plato, three of the five 
  being due to the Pythagoreans, namely the cube, the pyramid, and 
  the dodecahedron, while the octahedron and the icosahedron are due 
  to Theaetetus. 
\end{quote}

You may know Theaetetus through Plato's dialog of the same name, where 
he's described as a mathematical genius.  He's also mentioned in 
Plato's dialogue called the Sophist.  In the Republic, written around 
380 BC, Plato complained that not enough is known about solid geometry:

\begin{quote}
  ... and for two reasons: in the first place, no government places 
  value on it; this leads to a lack of energy in the pursuit of it, 
  and it is difficult.  In the second place, students cannot learn it 
  unless they have a teacher.  But then a teacher can hardly be 
  found....
\end{quote}

Theaetetus seems to have filled the gap: he worked on solid geometry
between 380 and 370 BC, perhaps inspired by Plato's interest in the 
subject.  He died from battle wounds and dysentery in 369 after Athens
fought a battle with Corinth.

But how certain are we that Theatetus discovered - or at least studied -
the icosahedron?   The only hard evidence seems to be this "ancient 
note" in the margins of the Elements.  But who wrote it, and when?

First of all, if you hope to see an ancient manuscript by Euclid with 
a scribbled note in the margin, prepare to be disappointed!  All we 
have are copies of copies of copies.  The oldest remaining fragments 
of the Elements date to centuries after Euclid's death: some from a 
library in Herculaneum roasted by the eruption of Mount Vesuvius in 79 
AD, a couple from the Fayum region near the Nile, and some from a 
garbage dump in the Egyptian town of Oxyrhynchus.  

There are various lines of copies of Euclid's Elements.  Comparing
these to guess the contents of the \emph{original} Elements is a
difficult and fascinating task.  Unfortunately, in the fourth century
AD, the Greek mathematician <a href =
"http://www-history.mcs.st-andrews.ac.uk/Biographies/Theon.html">Theon
of Alexandria</a> - <a href = "http://www-history.mcs.st-andrews.ac.uk/Biographies/Hypatia.html">Hypatia</a>'s dad - made a copy that became extremely
popular.  So popular, in fact, that for many centuries European
scholars knew no line of copies that hadn't passed through Theon!  And
Theon wasn't a faithful copyist: he added extra propositions,
lengthened some proofs, and omitted a few things too.  It seems he
wanted to standardize the language and make it easier to follow.  This
may have helped people trying to learn geometry - but certainly not
scholars trying to understand Euclid.

In 1808, <a href =
"http://www.sabix.org/bulletin/b3/peyrard.html">Francois Peyrard</a> made
a marvelous discovery.  He found that the Vatican library had a copy
of Euclid's Elements that hadn't descended through Theon!  

<div align = "center">
<a href = "http://www.loc.gov/exhibits/vatican/math.html">
<img border = "none" src = "http://www.loc.gov/exhibits/vatican/images/math22.jpg" alt = ""/>
% </a>
</div>

This copy
is now called "P".  It dates back to about 850 AD.  I would
love to know how Peyrard got his hands on it.  One imagines him
rooting around in a dusty basement and opening a trunk... but it seems
that Napoleon somehow took this manuscript from the Vatican to Paris.

In the 1880s, the great Danish scholar <a href =
"http://en.wikipedia.org/wiki/Johan_Ludvig_Heiberg_%28historian%29">Johan
Heiberg</a> used "P" together with various
"Theonine" copies of the Elements to prepare what's still
considered the definitive Greek edition of this book.  The
all-important English translation by Thomas Heath is based on this.
As far as I can tell, "P" is the only known non-Theonine
copy of Euclid except for the fragments I mentioned.  Heath also used
these fragments to prepare his translation.

This is just a quick overview of a complicated detective story. 
As always, the fractal texture of history reveals more complexity 
the more closely you look.

Anyway, Heath thinks that <a href =
"http://www-history.mcs.st-andrews.ac.uk/Biographies/Geminus.html">Geminus
of Rhodes</a> wrote the "ancient note" in the Elements
crediting Theatetus.  I'm not sure why Heath thinks this, but Geminus
of Rhodes was a Greek astronomer and mathematician who worked during
the 1st century BC.

In his charming article "The discovery of the regular
solids", William Waterhouse writes:

\begin{quote}
   Once upon a time there was no problem in the history of the regular
   solids.  According to Proclus, the discoveries of Pythagoras 
   include "the construction of the cosmic solids," and early 
   historians could only assume that the subject sprang full-grown 
   from his head.  But a better-developed picture of the growth
   of Greek geometry made such an early date seem questionable, and 
   evidence was uncovered suggesting a different attribution.  A 
   thorough study of the testimony was made by E. Sachs, and her 
   conclusion is now generally accepted: the attribution to Pythagoras 
   is a later misunderstanding and/or invention.

   The history of the regular solids thus rests almost entirely on a 
   scholium to Euclid which reads as follows: 

   "In this book, the 13th, are constructed the 5 figures called 
   Platonic, which however do not belong to Plato.  Three of these 
   5 figures, the cube, pyramid, and dodecahedron, belong to the 
   Pythagoreans; while the octahedron and icosahedron belong to 
   Theaetetus."

   Theaetetus lived c. 415-369 B.C., so this version gives a 
   moderately late date; and it has the considerable advantage of 
   seeming unlikely.  That is, the details in the scholium are not 
   the sort of history one would naively conjecture, and hence it 
   is probably not one of the stories invented in late antiquity.  
   As van der Waerden says, the scholium is now widely accepted 
   "precisely because [it] directly contradicts the tradition which 
   used to ascribe to Pythagoras anything that came along."  

   But probability arguments can cut both ways, and those scholars 
   who hesitate to accept the scholium do so primarily because it 
   seems too unlikely.  There have been two main sticking places: 
   first, the earliness of the dodecahedron in comparison with the 
   icosahedron; and second, the surprising lateness of the octahedron.
   The first objection, however, has been fairly well disposed of.  
   The mineral pyrite (FeS_{2}) crystallizes most often in cubes and 
   almost-regular dodecahedra; it is quite widespread, being the most 
   common sulphide, and outstanding crystals are found at a number 
   of spots in Italy.  Moreover it regularly occurs mixed with the 
   sulphide ores, and underlying the oxidized ores, of copper; these 
   deposits have been worked since earliest antiquity.  Thus natural 
   dodecahedra were conspicuous, and in fact they did attract 
   attention: artificial dodecahedra have been found in Italy dating 
   from before 500 BC.  Icosahedral crystals, in contrast, are much
   less common.  Hence there is no real difficulty in supposing that 
   early Pythagorean geometers in Italy were familiar with dodecahedra 
   but had not yet thought of the icosahedron.
\end{quote}

Indeed, while I've heard that iron pyrite forms "pseudoicosahedra":

<div align = "center">
<a href = "http://www.uwgb.edu/dutchs/symmetry/isometuc.htm">
<img border = "none" src = "pseudoicosahedron.gif">
% </a>
</div>

I've never seen one, while the "pyritohedra" resembling regular 
dodecahedra are pretty common:

<div align = "center">
<a href = "http://www.uwgb.edu/dutchs/symmetry/isometuc.htm">
<img border = "none" src = "pyritohedron.gif">
% </a>
</div>

<div align = "center">
<a href = "http://www.minerals.net/mineral/sulfides/pyrite/pyrite.htm">
<img src = "pyrite.jpg">
% </a>
</div>



The puzzle of why the octahedron showed up so late seems to have this
answer: it was known earlier, but it was no big deal until the concept of
regular polyhedron was discovered!  As Waterhouse says, the discovery
of the octahedron would be like the discovery of the 4rd perfect number.
Only the surrounding conceptual framework makes the discovery 
meaningful.

So far, so good.  But maybe the Greeks were not the first to discover 
the icosahedron!  In 2003, the mathematicians Michael Atiyah and 
Paul Sutcliffe wrote:

\begin{quote}
   Although they are termed Platonic solids there is convincing 
   evidence that they were known to the Neolithic people of Scotland 
   at least a thousand years before Plato, as demonstrated by the stone
   models pictured in Fig. 1 which date from this period and are kept 
   in the Ashmolean Museum in Oxford.
\end{quote}

<div align = "center">
<img src = "blocks.jpg">
<br/>
<font size = "-1">
Figure 1.  Stone models of the cube, tetrahedron, dodecahedron,
icosahedron and octahedron. <br/> They date from about 2000 BC and are
kept in the Ashmolean Museum in Oxford.
</font>
</div>

Various people including John McKay and myself spread this story without
examining it very critically.  I did read Dorothy Marshall's excellent
paper "Carved stone balls", which catalogues 387 carved stone balls 
found in Scotland, dating from the Late Neolithic to Early Bronze Age.
It has pictures showing a wide variety of interesting geometric 
patterns carved on them, and maps showing where people have found 
balls with various numbers of bumps on them.  But it doesn't say 
anything about Platonic solids.

<div align = "center">
<img src = "carved_stone_balls.jpg">
  
<img src = "carved_stone_balls_2.jpg">
<br/>
Maps by Dorothy Marshall. <br/>
Left: balls with 3 or 4 knobs.
Right: balls with 6 knobs.

</div>


In March of 2009, Lieven le Bruyn posted a skeptical investigation of 
Atiyah and Sutcliffe's claim.  For starters, he looked hard at the
photo in their paper:

<div align = "center">
<img src = "icosahedron/blocks.jpg">
</div>

\begin{quote}
   ... where's the icosahedron?  The fourth ball sure looks like one 
   but only because someone added ribbons, connecting the centers of 
  the different knobs.  If this ribbon-figure is an icosahedron, the 
  ball itself should be another dodecahedron and the ribbons illustrate
  the fact that icosa- and dodecahedron are dual polyhedra.  Similarly 
  for the last ball, if the ribbon-figure is an octahedron, the ball 
  itself should be another cube, having exactly 6 knobs.  Who did adorn
  these artifacts with ribbons, thereby multiplying the number of 
  "found" regular solids by two (the tetrahedron is self-dual)? 
\end{quote}

\emph{Who put on the ribbons?}  Lieven le Bruyn traced back the photo to 
Robert Lawlor's 1982 book Sacred Geometry.  In this book, Lawlor wrote:

\begin{quote}
   The five regular polyhedra or Platonic solids were known and 
   worked with well before Plato's time.  Keith Critchlow in his 
   book Time Stands Still presents convincing evidence that they 
   were known to the Neolithic peoples of Britain at least 1000 
   years before Plato.  This is founded on the existence of a number 
   of spherical stones kept in the Ashmolean Museum at Oxford. 
   Of a size one can carry in the hand, these stones were carved into 
   the precise geometric spherical versions of the cube, tetrahedron, 
   octahedron, icosahedron and dodecahedron, as well as some additional
   compound and semi-regular solids... 
\end{quote}

\emph{But is this really true?}  Le Bruyn discovered that the Ashmolean owns 
only 5 Scottish stone balls - and their webpage shows a photo of them,
which looks quite different than the photo in Lawlor's book!  

<div align = "center">
<a href = ""http://www.ashmolean.org/ash/britarch/highlights/stone-balls.html">
<img src = "icosahedron/scottishballs_ashmolean.jpg">
% </a>
</div>

They have no ribbons on them.  More importantly, they're different
shapes!  The Ashmolean lists their 5 balls as having 7, 6, 6, 4 and 14
knobs, respectively - nothing like an icosahedron.

And here is where I did a little research of my own.  The library 
at UC Riverside has a copy of Keith Critchlow's 1979 book Time 
Stands Still.  In this book, we see the same photo of stones with 
ribbons that appears in Lawlor's book - the photo that Atiyah and 
Suttcliffe use.  In Critchlow's book, these stones are called "a full 
set of Neolithic 'Platonic solids'".  He says they were photographed 
by one Graham Challifour - but he gives no information as to where 
they came from!

And Critchlow explicitly denies that the Ashmolean has an icosahedral
stone!  He writes:

\begin{quote}
   ... the author has, during the day, handled five of these 
   remarkable objects in the Ashmolean museum.... I was rapt 
   in admiration as I turned over these remarkable stone objects 
   when another was handed to me which I took to be an icosahedron.... 
   On careful scrutiny, after establishing apparent fivefold symmetry 
   on a number of the axes, a count-up of the projections revealed 14! 
   So it was not an icosahedron. 
\end{quote}

It seems the myth of Scottish balls shaped like Platonic solids
gradually grew with each telling.  Could there be any truth to it?
Dorothy Marshall records Scottish stone balls with various numbers 
of knobs, from 3 to 135 - but just two with 20, one at the National 
Museum in Edinburgh, and one at the Kelvingrove Art Gallery and Museum 
in Glasgow.  Do these look like icosahedra?  I'd like to know.  But 
even if they do, should we credit Scots with "discovering the 
icosahedron"?  Perhaps not. 

So, it seems the ball is in Theaetetus' court.

Here are some references:

The quote from Benno Artmann appeared in a copy of the AMS Bulletin
where the cover illustrates a construction of the icosahedron: 

5) Benno Artmann, About the cover: the mathematical conquest of 
the third dimension, Bulletin of the AMS, 43 (2006), 231-235.
Also available at 
<a href = "http://www.ams.org/bull/2006-43-02/S0273-0979-06-01111-6/">http://www.ams.org/bull/2006-43-02/S0273-0979-06-01111-6/</a>

For more, try this wonderfully entertaining book:

6) Benno Artmann, Euclid - The Creation of Mathematics, Springer, 
New York, 2nd ed., 2001.  (The material on the icosahedron is not 
in the first edition.)

It's not a scholarly tome: instead, it's a fun and intelligent 
introduction to Euclid's Elements with lots of interesting digressions.
A great book for anyone interested in math!

I should also get ahold of this someday:

7) Benno Artmann, Antike Darstellungen des Ikosaeders, Mitt. DMV 13 
(2005), 45-50.

Heath's translation of and commentary on Euclid's Elements is available
online thanks to the Perseus Project.  The scholium crediting Theatetus
for the octahedron and icosahedron is discussed here:

8) Euclid, Elements, trans. Thomas L. Heath, Book XIII, 
Historical Note, p. 438.  Also available at 
<a href = "http://old.perseus.tufts.edu/cgi-bin/ptext?doc=Perseus%3Atext%3A1999.01.0086\text{\&} query=head%3D%23566">http://old.perseus.tufts.edu/cgi-bin/ptext?doc=Perseus%3Atext%3A1999.01.0086\text{\&} query=head%3D%23566</a>

while the textual history of the Elements is discussed here:

9) Euclid, Elements, trans. Thomas L. Heath, Chapter 5: The Text,
p. 46.  Also available at
<a href = "http://old.perseus.tufts.edu/cgi-bin/ptext?lookup=Euc.+5">http://old.perseus.tufts.edu/cgi-bin/ptext?lookup=Euc.+5</a>

Anyone interested in Greek mathematics also needs these books by
Heath, now available cheap from Dover:

10) Thomas L. Heath, A History of Greek Mathematics. Vol. 1: From
Thales to Euclid.  Vol. 2: From Aristarchus to Diophantus.  
Dover Publications, 1981.  

The long quote by Waterhouse comes from here:

11) William C. Waterhouse, The discovery of the regular solids,
Arch. Hist. Exact Sci. 9 (1972-1973), 212-221. 

I haven't yet gotten my hold on this "thorough study" mentioned by
Waterhouse - but I will soon:

12) Eva Sachs, Die F&uuml;nf Platonischen K&ouml;rper, zur Geschichte der 
Mathematik und der Elementenlehre Platons und der Pythagoreer, 
Berlin, Weidmann, 1917.

I also want to find this discussion of how Peyrard got ahold of the 
non-Theonine copy of Euclid's Elements:

13) N. M. Swerlow, The Recovery of the exact sciences of antiquity: 
mathematics, astronomy, geography, in Rome Reborn: The Vatican 
Library and Renaissance Culture, ed. Grafton, 1993.

Here is Atiyah and Sutcliffe's paper claiming that the Ashmolean
has Scottish stone balls shaped like Platonic solids:

14) Michael Atiyah and Paul Sutcliffe, Polyhedra in physics, 
chemistry and geometry, available as <a href = "http://arxiv.org/abs/math-ph/0303071">arXiv:math-ph/0303071</a>.

Here is le Bruyn's critical examination of that claim:

15) Lieven le Bruyn, The Scottish solids hoax, March 25, 2009, 
<a href = "http://www.neverendingbooks.org/index.php/the-scottish-solids-hoax.html">http://www.neverendingbooks.org/index.php/the-scottish-solids-hoax.html</a>

Here are the books by Critchlow and Lawlor -speculative books from
the "sacred geometry" tradition:

16) Keith Critchlow, Time Stands Still, Gordon Fraser, London, 1979.

17) Robert Lawlor, Sacred Geometry: Philosophy and Practice, 
Thames and Hudson, London, 1982.  Available at
<a href = "http://www.scribd.com/doc/13155707/robert-lawlor-sacred-geometry-philosophy-and-practice-1982">http://www.scribd.com/doc/13155707/robert-lawlor-sacred-geometry-philosophy-and-practice-1982</a>

Here's the Ashmolean website:

18) British Archaeology at the Ashmolean Museum, Highlights of the 
British collections: stone balls, 
<a href = "http://ashweb2.ashmus.ox.ac.uk/ash/britarch/highlights/stone-balls.html">http://ashweb2.ashmus.ox.ac.uk/ash/britarch/highlights/stone-balls.html</a>

and here's Dorothy Marshall's paper on stone balls:

19) Dorothy N. Marshall, Carved stone balls, Proc. Soc. Antiq. Scotland, 
108 (1976/77), 40-72.  Available at 
<a href = "http://www.tarbat-discovery.co.uk/Learning%20Files/Carved%20stone%20balls.pdf">http://www.tarbat-discovery.co.uk/Learning%20Files/Carved%20stone%20balls.pdf</a>

Finally, a bit of math.

In the process of researching my talk, I learned a lot about Euclid's
Elements, where the construction of the icosahedron - supposedly due
to Theaetetus - is described.  This construction is <a href =
"http://aleph0.clarku.edu/~djoyce/java/elements/bookXIII/propXIII16.html">Proposition
XIII.16</a>, in the final book of the Elements, which is largely about
the Platonic solids.  This book also has some fascinating results
about the golden ratio and polygons with 5-fold symmetry!

The coolest one is <a href =
"http://aleph0.clarku.edu/~djoyce/java/elements/bookXIII/propXIII10.html">Proposition
XIII.10</a>.  It goes like this.

Take a circle and inscribe a regular pentagon, a regular hexagon, and 
a regular decagon.   Take the edges of these shapes, and use them as the 
sides of a triangle.  Then this is a right triangle!

In other words, if

P

is the side of the pentagon, 

H 

is the side of the hexagon, and

D 

is the side of the decagon, then

P^{2} = H^{2} + D^{2}

We can prove this using algebra - but Euclid gave a much cooler proof,
which actually find this right triangle hiding inside an icosahedron.

First let's give a completely uninspired algebraic proof.

Start with a unit circle.  If we inscribe a regular hexagon in it, 
then obviously 

H = 1

So we just need to compute P and D.  If we think of the unit circle as 
living in the complex plane, then the solutions of 

z^{5} = 1

are the corners of a regular pentagon.  So let's solve this equation.
We've got

0 = z^{5} - 1 = (z - 1)(z^{4} + z^{3} +
z^{2} + z + 1)

so ignoring the dull solution z = 1, we must solve

z^{4} + z^{3} + z^{2} + z + 1 = 0

This says that the center of mass of the pentagon's corners lies right
in the middle of the pentagon.

Now, quartic equations can always be solved using radicals, but it's 
a lot of work.  Luckily, we can solve this one by repeatedly using 
the quadratic equation!  And that's why the Greeks could construct 
the regular pentagon using a ruler and compass.  

The trick is to rewrite our equation like this:

z^{2} + z + 1 + z^{-1} + z^{-2} = 0

and then like this:

(z + z^{-1})^{2} + (z + z^{-1}) - 1 = 0

Now it's a quadratic equation in a new variable.  So while I said this 
proof would be uninspired, it did require a tiny glimmer of inspiration.
But that's all!  Let's write

z + z^{-1} = x

so our equation becomes

x^{2} + x - 1 = 0

Solving this, we get two solutions.  The one I like is the golden
ratio:

x = \phi  = (-1 + \sqrt 5)/2 ~ 0.6180339...

Next we need to solve

z + z^{-1} = \phi 

This is another quadratic equation:

z^{2} - \phi  z + 1 = 0

with two conjugate solutions, one being

z = (\phi  + (\phi ^{2} - 4)^{\frac{1}{2} })/2

I've sneakily chosen the solution that's my favorite 5th root of unity:

z = exp(2\pi i/5) = cos(2\pi /5) + i sin(2\pi /5)

So, we're getting

cos(2\pi /5) = \phi /2

A fact we should have learned in high school, but probably never did.

Now we're ready to compute P, the length of the side of a pentagon
inscribed in the unit circle:

P^{2} = |1 - z|^{2}   <br/><br/>
      = (1 - cos(2\pi /5))^{2} + (sin(2\pi /5))^{2}   <br/><br/>
        = 2 - 2 cos(2\pi /5)   <br/><br/>
         = 2 - \phi 

Next let's compute D, the length of the side of a decagon inscribed
in the unit circle!   We can mimic the last stage of the above 
calculation, but with an angle half as big:

D^{2} = 2 - 2 cos(\pi /5)

To go further, we can use a half-angle formula:

cos(\pi /5) = ((1 + cos(2\pi /5))/2)^{\frac{1}{2} } <br/><br/>
              = (\frac{1}{2}  + \phi /4)^{\frac{1}{2} }

This gives

D^{2} = 2 - (2 + \phi )^{\frac{1}{2} }

But we can simplify this a bit more.  As any lover of the golden ratio 
should know, 

2 + \phi  = 2.6180339...

is the square of 

1 + \phi  = 1.6180339...

So we really have

D^{2} = 1 - \phi 

Okay.  Your eyes have glazed over by now - unless you've secretly been
waiting all along for This Week's Finds to cover high-school algebra
and trigonometry.  But we're done.  We see that

P^{2} = H^{2} + D^{2}

simply says

2 - \phi  = 1 + (1 - \phi )

That wasn't so bad, but imagine discovering it and proving it using 
axiomatic geometry back around 300 BC!  How did they do it?

For this, let's turn to

20) Ian Mueller, Philosophy of Mathematics and Deductive Structure in 
Euclid's Elements, MIT Press, Cambridge Massachusetts, 1981.

This is reputed to be be the most thorough investigation of the 
logical structure of Euclid's Elements!  And starting on page 257 he
discusses how people could have discovered P^{2} = 
H^{2} + D^{2} by staring at
an icosahedron!  

This should not be too surprising.  After all, there are pentagons, 
hexagons and decagons visible in the icosahedron.  But I was completely
stuck until I cheated and read Mueller's explanation.

If you hold an icosahedron so that one vertex is on top and one is
on bottom, you'll see that its vertices are arranged in 4 horizontal
layers.  From top to bottom, these are:

<ul>
<li>
1 vertex on top
</li><li>
5 vertices forming a pentagon: the "upper pentagon"
</li><li>
5 vertices forming a pentagon: the "lower pentagon"
</li><li>
1 vertex on bottom
</li>
</ul>

Pick a vertex from the upper pentagon: call this A.  Pick a vertex 
as close as possible from the lower pentagon: call this B.  A is not 
directly above B.  Drop a vertical line down from A until it hits the 
horizontal plane on which B lies.  Call the resulting point C.  
If you think about this, you'll see that ABC is a right triangle.  
Greg Egan drew a picture of it:

<div align = "center">
<img width = "400" src = "icosahedron_with_right_triangle.gif">
</div>

And if we apply the Pythagorean theorem to 
this triangle we'll get the equation

P^{2} = H^{2} + D^{2}

To see this, we only need to check that:

<ul>
<li>
the length AB equals the edge of a pentagon inscribed in a circle;
</li>
<li>
the length AC equals the edge of a hexagon inscribed in a circle;
</li>
<li>
the length BC equals the edge of a decagon inscribed in a circle.
</li>
</ul>

Different circles, but of the same radius!  What's this radius?  The 5
vertices of the lower pentagon lie on the circle shown in
blue.  This circle has the right radius.

Using this idea, it's easy to see that the length AB equals the edge
of a pentagon inscribed in a circle.  It's also easy to see that 
BC equals the edge of a decagon inscribed in a circle of the same
radius.  The hard part, at least for me, is seeing that AC equals the 
edge of a hexagon inscribed in a circle of the same radius... or in 
other words, the radius of that circle!  (The hexagon seems to be a
red herring.)

To prove this, it would suffice to show the following marvelous fact:
the distance between the "upper pentagon" and the
"lower pentagon" equals the radius of the circle containing
the vertices of the upper pentagon!

Can you prove this?

In Mueller's book, he suggests various ideas the Greeks could
have had about this.  Here's one:

<div align = "center">
<img width = "400" src = "icosahedron_with_right_triangles.gif">
</div>

The right triangle ABC is shown here.  The trick is to construct
another right triangle AB'C'.  Here B' is the top vertex, and C'
is where a line going straight down from B' hits the plane
containing the upper pentagon.

Remember, we're trying to show the distance 
between the upper pentagon and lower pentagon 
equals the radius of the circle containing the
vertices of the upper pentagon.

But that's equivalent to showing that AC' is congruent to AC.

To do this, it suffices to show that the right triangles ABC
and AB'C' are congruent!  Can you do it?

In the references to Mueller's book, he says the historians
Dijksterhuis (in 1929) and Neuenschwander (in 1975) claimed this
is "intuitively evident".  But I don't know if that means
it's easy to prove!

I thank Toby Bartels and Greg Egan for help with this stuff.
I also thank Jim Stasheff for passing on an email from Joe
Neisendorfer pointing out Mellinger's picture of the Milky Way.

\par\noindent\rule{\textwidth}{0.4pt}

\textbf{Addendum:} Kevin Buzzard explained some of the Galois theory
behind why the pentagon can be constructed with ruler and compass -
or in other words, why the quartic 

z^{4} + z^{3} + z^{2} + z + 1 = 0

can be solved by solving first one quadratic and then another.

He wrote:

\begin{quote}

\begin{quote}
    Now, quartic equations can always be solved using radicals
\end{quote}

That's because S_{4} is a solvable group, and all Galois
groups of quartics will live in S_{4} (and will usually be
S_{4})...

\begin{quote}
    Luckily, we can solve this one by repeatedly using the quadratic equation!
\end{quote}

("this one" being z^{4} + z^{3} + z^{2} + z + 1 = 0.)

...and \emph{that's} because the Galois group of that \emph{specific}
irreducible polynomial is "only" cyclic of order 4. The splitting
field is Q(\zeta _{5}), which is a cyclotomic field, so has
Galois group (Z/5Z)*. No Z/3Z factors so no messing around
with cube roots, for example...

\begin{quote}
    So while I said this
    proof would be uninspired, it did require a tiny glimmer of inspiration.
\end{quote}

With this observation above, I'm trying to convince you that the proof
really \emph{is} completely uninspired <img src =
"emoticons/tongue2.gif"> To solve the quartic by solving two
quadratics, you need to locate the degree 2 subfield of Q(z)
(z=\zeta _{5}) and aim towards it (because it's your route to
the solution). This subfield is clearly the real numbers in Q(z), and
the real numbers in Q(z) contains z+z*=z+z^{-1}. So that's sort
of a completely conceptual explanation of why the trick works and why
it's crucial to introduce z+z^{-1}.

\end{quote}

Here \zeta _{5} is number-theorist's jargon for a "primitive
5th root of unity", which in turn is number-theorist's jargon for any 
5th root of 1 except for 1 itself.

Greg Egan gave a nice modern version of Euclid's original 
proof of Prop. XIII.10, which states that if you take
take a circle and inscribe a regular pentagon, a regular hexagon, and 
a regular decagon, and make a triangle out of their sides, it's
a right triangle!

\begin{quote}

<div align = "center">
<img width = "500" src = "pentagon_hexagon_decagon.gif">
</div>

Here's a version of the proof Euclid gave, adapted from <a
href="http://aleph0.clarku.edu/~djoyce/java/elements/bookXIII/propXIII10.html">the version JB cited</a>.  Rather than proving that various angles here
are identical, I've just written in the (easily established) numerical
values; there's nothing tricky here, so we might as well take them as
given.</p>


Triangle ABF is similar to triangle BFN.  So AB/BF = BF/FN = BF/BN, 
with the last equality true because the triangles are isosceles with 
FN = BN.  Thus BF ^{2} = AB &middot; BN </p>


Triangle BAK is similar to triangle KAN.  
So BA/AK = KA/AN.  Thus AK^{2} = AB &middot; AN.</p>


Adding our two results, we have:
BF^{2} + AK^{2} = AB &middot; (AN + BN) = AB^{2}.</p>


BF is our radius, AK is a decagon side, and AB is a pentagon
side. Well done Euclid.</p>

\end{quote}

On Thanksgiving of 2009, I got ahold of Eva Sachs' 1917 book <i>Die
F&uuml;nf Platonischen K&ouml;rper</i>, mentioned above.  It's
supposed to be the authoritative tome on the early history of the
Platonic solids.

If anyone out there reads German, I'd love a translation of what she says
about the icosahedron and the pentagon-hexagon-decagon identity.  Click on
the pictures for larger views:

<div align = "center">
<a href = "http://math.ucr.edu/home/baez/icosahedron_sachs_1.jpg">
<img width = "500" src = "http://math.ucr.edu/home/baez/icosahedron_sachs_1.jpg" alt = ""/>
</div>
% </a>

<div align = "center">
<a href = "http://math.ucr.edu/home/baez/icosahedron_sachs_2.jpg">
<img width = "500" src = "http://math.ucr.edu/home/baez/icosahedron_sachs_2.jpg" alt =
""/>
</div>
% </a>

I think the first page gets interesting around "Der Satz XIII 10 lautet..."
("Proposition XIII.10 says...") and more interesting after "Dieser Satz,
obwohl planimetrisch, ist durch Betrachtung der ebenen Geometrie nicht
leicht zu finden". ("This proposition, though planar in character, is not
so easy to find through considerations of plane geometry.")

<div align = "center">
<a href = "http://math.ucr.edu/home/baez/icosahedron_sachs_figure.jpg">
<img src = "http://math.ucr.edu/home/baez/icosahedron_sachs_figure.jpg" alt
= ""/>
</div>
% </a>

You'll notice that she focuses our attention on the right triangles ZWQ and
QEP, which are the triangles ABC and AB'C' that I mentioned above:

<div align = "center">
<img width = "300" src =
"http://math.ucr.edu/home/baez/icosahedron_with_right_triangles.gif" alt =
""/>
</div>

Proving that these are congruent is the key to the pentagon-decagon-hexagon
identity!

I'm especially curious about the footnote on page 104, and also the remark
further up this page saying "So fand er den Satz XIII 9..." ("This is how
he found Proposition XII.9").

To see Greg Egan's beautiful proof of the pentagon-decagon-hexagon
identity, which meets my challenge above, see "<a href =
"week284.html">week284</a>".

For more discussion visit the <a href =
"http://golem.ph.utexas.edu/category/2009/11/this_weeks_finds_in_mathematic_44.html">\emph{n}-Category
Caf&eacute;</a>.


\par\noindent\rule{\textwidth}{0.4pt}

<em>Geometry enlightens the intellect and sets one's mind right.  All its 
proofs are very clear and orderly.  It is hardly possible for errors to 
enter into geometric reasoning, because it is well arranged and orderly.
Thus the mind that constantly applies itself to geometry is unlikely to
fall into error.</em> - Ibn Khaldun

\par\noindent\rule{\textwidth}{0.4pt}

% </A>
% </A>
% </A>
