
% </A>
% </A>
% </A>
\week{ January 25, 2005 }

As you've probably heard, the Huygens probe has successfully landed
on Saturn's moon Titan and is sending back pictures:

1) Huygens Probe Descent,
<A HREF = "http://saturn.jpl.nasa.gov/news/events/huygensDescent/index.cfm">http://saturn.jpl.nasa.gov/news/events/huygensDescent/index.cfm</A>

Titan averages a chilly -180 degrees Celsius, and its smoggy orange
atmosphere is thicker than the Earth's, mostly nitrogen but 6 percent
methane, together with substantial traces of all sorts of other
hydrocarbons.  The orange color may come from "tholins":
polymers made by irradiating a mix of nitrogen and methane.  Some
other icy moons in the outer solar system are covered with this goop,
but Titan is the only moon in the Solar System to have a substantial
atmosphere.  It even has clouds.

As the Huygens probe parachuted to the surface, it photographed
what look like twisty riverbeds flowing into a large lake:

<DIV ALIGN="CENTER">
<IMG SRC=huyghens.2.jpg>
</DIV ALIGN>

People 
have long suspected that Titan has lakes of made of methane and/or 
ethane, but now we may be seeing them.  And when Huygens landed, its
sensors reported that it broke through a crusty surface and sunk 
about 20 centimeters into something mushy: probably methane mud! 

The first color photo of the surface looks disappointingly like Mars 
at first sight:

<DIV ALIGN="CENTER">
<IMG SRC=huyghens.1.jpg>
</DIV ALIGN>



But, the surface is pumpkin-colored due to tholins
or something, not rust red.  The sky is orange too!  
The "rocks" could 
be water ice.  And they've detected hints of volcanos that spew molten
water and ammonia!  So, it's a strange new world.

Back here on Earth, there was a conference in December in honor of 
Larry Breen's 60th birthday:

2) Arithmetic, Geometry and Topology: Conference on occasion of Larry
Breen's sixtieth birthday, 
<A HREF = "http://www-math.univ-paris13.fr/~lb2004/">http://www-math.univ-paris13.fr/~lb2004/</A>

It was in Paris.  This was my first visit to that city, but luckily 
I got to stay there an extra week after the conference, so I could focus
on the math while it was going on.  

But I can't resist a digression!  Paris won my heart, despite my suspicions 
that it had somehow been hyped all along.  First of all, it's beautiful. 
Second, it's nice to be someplace where people take simple foods like 
bread, cheese, fruits and vegetables really seriously, and don't settle 
for the tasteless crud we so often eat in the US.  

None of this came as a surprise, of course.  What surprised is that I've 
never seen a city with so many bookstores - and good ones, too!   They're 
clustered thick near the Sorbonne, but the Latin Quarter is dotted with 
them, and there are even lots along the Boulevard St-Germain, which is the 
biggest most fashionable shopping street.  I don't think there's any place 
in the English speaking world with so many bookstores.  Not London, not 
New York... Cambridge Massachusetts used to have lots near Harvard Square, 
back when I was a grad student, but the high rents have long since squeezed
them out, replacing bohemian diversity with clothing shops for boring rich 
people, like Abercrombie and Fitch.  Somehow in Paris fancy clothing and 
books coexist.  

Umm, but what about the conference?

Well, Breen's work is mainly on algebraic geometry a la Grothendieck, with
a strong emphasis on category theory.  Beautiful stuff, and lately it's
it's begun to find applications to string theory - especially his work on
gerbes.  People at his conference spoke on all sorts of topics, most of 
which I didn't understand very well - some heavy-duty number theory, 
for example.  I understood a few well enough to really enjoy them, like
Mike Hopkins' talk on derived algebraic geometry, Clemens Berger's talk on
geometric Reedy categories, and Ieke Moerdijk's talk on the homotopy theory
of operads.   But I won't try to explain these - I want to explain what a 
"gerbe" is, so I have my work cut out for me.

One way to get going on the idea of gauge theory is to start with
electromagnetism, where the concept of "phase" turns out to play a
crucial role.  If you move a charged particle through an electromagnetic
field, its wavefunction gets multiplied by a unit complex number, or
"phase" - and it turns out, rather wonderfully, that all effects of
electricity and magnetism on charged particles is due to this!

However, phases are funny.  You can't actually measure the phase of a
charged particle - at least, there's no such thing as a
"phasometer" where you stick in a particle and the dial on
the meter points to a unit complex number.  Of course a unit complex
number is just a fancy name for a point on the circle, and a dial is
precisely the right shape for that... but you just can't build this
machine.

Instead, you can only measure the \emph{change} in phase of a particle as
it goes around a loop.  Or, equivalently, the \emph{difference} in phases
when a particle takes two different paths from here to there.  See, 
in quantum mechanics you can play tricks like the "double slit 
experiment",
where you coax a particle's wavefunction to smear out and take two routes 
from here to there... and then when it arrives, it interferes with itself, 
and if you're smart you can see by these interference effects what the 
relative phase of the two paths is.

Pretty weird, eh?  I'm so used to this that it seems completely normal
to me, but I should admit that this way of understanding the
electromagnetic field came fairly late.  Weyl had a hint of it in 1918
when he invented the term "gauge theory" in his quest to
unify electromagnetism and gravity, but he was mixed up in some
crucial ways that only got sorted out quite a bit later.  For more
details, try O'Raiferteagh's book "The Dawning of Gauge
Theory", which I discussed in "<A HREF =
"week116.html">week116</A>".

Anyway, the concept of relative phase, or difference in phase, is
nicely captured by the concept of a "torsor".  A unit
complex number is a point on the unit circle in the complex plane.
This circle is a group since we can multiply unit complex numbers and
get unit complex numbers back.  This group is called U(1).  Like a
dial, U(1) has standard names for all the points on it - and it has
one god-given special point, the identity element, namely the number
1.

A "U(1) torsor" is a lot like U(1), but subtly different.
It's a circle where the points aren't given these standard
names... but where you can still tell measure angles, and tell the
difference between clockwise and counterclockwise.

You can't get an element of U(1) from \emph{one} point on a U(1)
torsor.  But, if you have \emph{two} points on a U(1) torsor, you
can say how much rotation it takes to get from one to the other, and
this give an element of U(1).  In other words, you can describe the
"difference in phase" between these two points.

For more on torsors, try this:

3) John Baez, Torsors made easy, <A HREF = "http://math.ucr.edu/home/baez/torsors.html">http://math.ucr.edu/home/baez/torsors.html</A>

Anyway, the real idea behind electromagnetism is that sitting over
each point in spacetime is a U(1) torsor.  If a particle is sitting at
some point in spacetime, its phase is not really a number: it's an
element of the U(1) torsor sitting over that point!  To get a
\emph{number}, we have to carry the particle around a loop!  Its
phase will change when we do this, so we get \emph{two} points in a
U(1) torsor, and their difference is an element of U(1).

So while it sounds far-out, the key mathematical structure in
electromagnetism is a bunch of U(1) torsors, one over each point in
spacetime.  This is called a "principal U(1) bundle" or
sometimes just a "U(1) bundle" for short.

If we wanted to describe some force other than electromagnetism, we could
take this whole setup and replace U(1) with some other group.  In fact, 
this idea works great: it's the main idea behind gauge theories, which do 
an excellent job of describing all the forces in nature.  

To set up a gauge theory, the first thing you need to do is pick a 
group G and pick a "principal G-bundle" over spacetime.  Spacetime 
will be some manifold X.  A principal G-bundle over X is gadget that 
assigns a G-torsor to each point of X.  A G-torsor is a space where if 
you pick two points in it, you get an element of G which describes their 
"difference".

I'm being fairly sloppy here, so don't take these as precise definitions!
I give a precise definition of a G-torsor in the above webpage, and any
decent book on differential geometry will give you a definition of a
principal G-bundle.  However, only rather highbrow books define principal
G-bundles with the help of G-torsors... which is sad, because it's not
that hard, and rather enlightening.

Anyway, in gauge theory the forces of nature are described by
"connections" on principal G-bundles.  Let's say we have a
principal G-bundle P which assigns to each point x of our manifold a
G-torsor P(x).  Then a "connection" on P is a gadget that
says how any path from x to y gives a map from P(x) to P(y).  If G is
U(1), for example, this gadget says how the phase of a charged
particle changes as we move it along any path from x to y.

Now suppose we have a loop that starts and ends at x.  Then our connection 
gives a map from P(x) to itself.  If we start with a point in P(x), and 
apply this map, we get another point in P(x).  Since P(x) is a G-torsor, 
these two points determine an element of G.  This is how we get group 
elements from loops in gauge theory!   

Now let me sketch how gerbes enter the game.  First I'll do the case where 
the group G is abelian, for example U(1).  It's the nonabelian gerbes that 
really interest me... but the abelian case is a lot easier.  The reason
is that when G is abelian, the group element we get in the previous 
paragraph doesn't depend on the choice of a point of P(x).  

Gerbes show up when we try to invent a kind of "higher gauge theory" 
that describes how not just point particles but 1-dimensional objects
transform when you move them around.  For example, the strings in string 
theory, or the loops in loop quantum gravity.  

This leads to a mind-boggling self-referential twist, which is just the
kind of thing I love:

As we've seen, a connection describes how a point particle transforms when 
you carry it along a path:


$$

            f
  x--------->-------y     a path f from the point x to the point y:
                                  we write this as f: x \to  y
$$
    
Now we need a gadget that'll describe how a \emph{path} transforms when you 
carry it along a \emph{path of paths:}


$$

             f
    --------->-------
   /        ||       \
  x         ||F       y   a path-of-paths F from the path f to the path g:
   \        \/       /            we write this as F: f => g
    --------->-------
               g
$$
    

To do this, we need to boost our level of thinking a notch, working
not with "G-torsors" and "principal G-bundles" but
instead with "G-2-torsors" and "G-gerbes".

Here's how it goes:

We start by picking an abelian group G and a manifold X.  

Then we pick a "G-gerbe" over X, say P.   

What's that?  It's a thing that assigns to each point x of X a
"G-2-torsor", say P(x).

What's that?  Well, it's a thing where if you pick two points in it, you 
get a \emph{G-torsor} describing their difference!   

Get it?  This is the beginning of a story that goes on forever:


\begin{verbatim}

 Two points in a G-torsor determine an element of G;
 two points in a G-2-torsor determine a G-torsor;
 two points in a G-3-torsor determine a G-2-torsor;
 .
 .
 .
\end{verbatim}
    
But, you'll probably be relieved to know we won't go beyond G-2-torsors 
today.

Next, we pick a "connection" on P.  

What's that?  Well, it's a gadget that for each path from x to y
gives us a map from the G-2-torsor P(x) to the G-2-torsor P(y).
If we call the path


$$

f: x \to  y
$$
    
then we call this map


$$

P(f): P(x) \to  P(y)
$$
    

Moroever, this sort of connection also gives a "map between
maps" for each path-of-paths!  So, from


$$

F: f => g
$$
    
it gives


$$

P(F): P(f) => P(g)
$$
    
I haven't explained enough stuff to say yet what these "maps between
maps" are, so let's just see what happens if we have a loop 


$$

f: x \to  x
$$
    
and then a loop-of-loops 


$$

F: f => f
$$
    
From the loop f: x \to  x, our connection gives us a map:


$$

P(f): P(x) \to  P(x)
$$
    
If we start with a point in P(x), and apply this map, we get another 
point in P(x).  Since P(x) is a G-2-torsor, these two points determine a 
G-torsor.  This G-torsor doesn't depend on our initial choice of 
point, and it completely determines the map P(f).   So, we can think 
of P(f) as \emph{being} this G-torsor, if we like.

From the loop-of-loops F: f => f, our connection gives us a map:


$$

P(F): P(f) => P(f)
$$
    
If we start with a point in P(f), and apply this map, we get 
another point.  Since P(f) is a G-torsor, these two points determine an 
element of G.  This element of G doesn't depend on our initial choice 
of point, and it completely determines the map P(F).   So, 
we can think of P(F) as \emph{being} this element of G, if we like.

In short, the machinery functions just as you'd hope, giving a group
element that describes how a loop of string "changes phase"
as you carry it around a loop-of-loops!

So far I've been strenously avoiding the language of categories and 
2-categories, but if you're at all familiar with that language, you'll
have guessed that it's precisely what we need to make everything I'm 
saying precise.

It's actually incredibly beautiful... but I'm getting lazy, so I'll 
explain it very tersely now, in a way that only true lovers of abstraction 
will enjoy:

If G is a group, it acts on itself by left translation.  So, it
becomes a left G-set.  Any left G-set isomorphic to this one is called
a "G-torsor".  There's a category G-Tor whose objects are
G-torsors and whose morphisms are maps compatible with the action of
G.  Since all G-torsors are isomorphic, and the automorphism group of
any one is just G, this category G-Tor is equivalent to G (regarded as
a 1-object category).

If G is abelian, every left G-set becomes a right G-set too.  This
allows us to define a "tensor product" of G-sets.  The
tensor product of G-torsors is a G-torsor, so G-Tor becomes a monoidal
category.  In fact, it's a "2-group": a monoidal category
where all the objects and morphisms are invertible.

This allows us to iterate what we've just done:

Since G-Tor is a 2-group, it acts on itself by left translation.  So,
it becomes a "left G-category".  Any left G-category
isomorphic to this one is called a "G-2-torsor".  There's a
2-category G-2-Tor whose objects are G-2-torsors, whose morphisms are
functors compatible the action of G, and whose morphisms are natural
transformations compatible with the action of G.  Since all
G-2-torsors are isomorphic, any the automorphism 2-group of any one is
just G-Tor, this 2-category is equivalent to G-Tor (regarded as a
1-object 2-category).

And so on!  This infinite hierarchy only works when G is abelian;
when G is nonabelian we need a different hierarchy, which uses 
"bitorsors", where G acts on both left and right, instead 
of "torsors".  

To learn more about this stuff, here are some references.  I'll stick
to ones I didn't already list in "<A HREF =
"week71.html">week71</A>" and "<A HREF =
"week151.html">week151</A>".

First, for physicists, some work on the role of gerbes and 2-gerbes in 
string theory and M-theory:

4) Paolo Aschieri, Luigi Cantini and Branislav Jurco, Nonabelian bundle
gerbes, their differential geometry and gauge theory, available as 
<A HREF = "http://xxx.lanl.gov/abs/hep-th/0312154">hep-th/0312154</A>.

5) Paolo Aschieri and Branislav Jurco, Gerbes, M5-brane anomalies and
E8 gauge theory, available as <A HREF =
"http://xxx.lanl.gov/abs/hep-th/0409200">hep-th/0409200</A>.

Second, for mathematicians, some classic works by Breen:

6) Lawrence Breen, Bitorseurs et cohomologie non-abelienne,
in The Grothendieck Festschrift, eds. P. Cartier et al, Progress 
in Mathematics vol. 86, Birkhauser, Boston, 1990, pp. 401-476.

7) Lawrence Breen, Theorie de Schreier superieure, Ann. Sci. Ecole Norm.
Sup. 25 (1992), 465-514.

8) Lawrence Breen, Classification of 2-stacks and 2-gerbes, Asterisque
225, Societe Mathematique de France, 1994.

2-gerbes are what you get if you climb the hierarchy one more step.
They should be good for describing the parallel transport of 
2-dimensional surfaces, or "2-branes" - and indeed they make an 
appearance in Aschieri and Jurco's paper for precisely that reason.

Another key reference is Breen's paper with Messing about connections 
on nonabelian gerbes:

9) Lawrence Breen and William Messing, The differential geometry of gerbes,
available as <A HREF = "http://www.arXiv.org/abs/math.AG/0106083">math.AG/0106083</A>.

and Breen's lecture notes from the IMA workshop on higher categories:

10) Larry Breen, n-Stacks and n-gerbes: homotopy theory.
Notes available at <A HREF = "http://www.ima.umn.edu/categories/#thur">http://www.ima.umn.edu/categories/#thur</A>

I've been working on this stuff myself lately, from a somewhat different
viewpoint.  So far I've written papers with Aaron Lauda and Alissa Crans 
about 2-groups and Lie 2-algebras:

11) John Baez and Aaron Lauda, Higher-dimensional algebra V: 2-groups,
Theory and Applications of Categories 12 (2004), 423-491.  Available
online at <A HREF =
"http://www.tac.mta.ca/tac/volumes/12/14/12-14abs.html">http://www.tac.mta.ca/tac/volumes/12/14/12-14abs.html</A>
or as <A HREF =
"http://www.arXiv.org/abs/math.QA/0307200">math.QA/0307200</A>.

12) John Baez and Alissa Crans, Higher-dimensional algebra VI: Lie
2-algebras, Theory and Applications of Categories 12 (2004), 492-528.
Available online at <A HREF =
"http://www.tac.mta.ca/tac/volumes/12/15/12-15abs.html">http://www.tac.mta.ca/tac/volumes/12/15/12-15abs.html</A>
or as <A HREF =
"http://www.arXiv.org/abs/math.QA/0307263">math.QA/0307263</A>.

Aaron Lauda was getting a masters degree in physics at UCR when we started 
our paper on 2-groups.  Now he's a grad student in math at the University of
Cambridge, working on things related to topological quantum field theory 
with the category theorist Martin Hyland.  Alissa Crans did her PhD in math
at UCR, and our paper on Lie 2-algebras contains a lot of stuff from her 
thesis.  Now she has a job at Loyola Marymount University, in Los Angeles.  

I've had a huge amount of fun working with both of them!  Luckily Alissa 
lives nearby, and I visit Cambridge most summers.  So, we can all keep 
working on other projects together - and we are.

I also have some gerbe-related projects going on with my grad student
Toby Bartels, Danny Stevenson (who is teaching at UCR now) and Urs 
Schreiber, a fellow moderator of sci.physics.research who will soon be 
a postdoc at Hamburg with Christoph Schweigert.  Urs will be visiting 
UCR for two weeks in February, and we plan to figure a lot of stuff out.
So, I've got gerbes on the brain, and I'll probably be saying more about 
them in the future, unless I burn up all my expository energy writing 
papers.

In fact, one of the best places to learn about the differential geometry 
of abelian gerbes and 2-gerbes is Danny's thesis:

13) Danny Stevenson, The geometry of bundle gerbes, Ph.D. thesis, 
University of Adelaide, 2000.  Available as <A HREF = "http://www.arXiv.org/abs/math.DG/0004117">math.DG/0004117</A>.

He's also written lots of other papers on gerbes, which you can find
on the arXiv.  Physicists may find these the most interesting:

14) Michael K. Murray and Danny Stevenson, Higgs fields, bundle gerbes and
string structures, Comm. Math. Phys. 236 (2003), 541-555.  Also available
as <A HREF = "http://www.arXiv.org/abs/math.DG/0106179">math.DG/0106179</A>.

15) Alan L. Carey, Stuart Johnson, Michael K. Murray, Danny Stevenson
and Bai-Ling Wang, Bundle gerbes for Chern-Simons and Wess-Zumino-Witten
models, available as <A HREF = "http://www.arXiv.org/abs/math.DG/0410013">math.DG/0410013</A>.

Toby is doing his thesis on categorified bundles, or "2-bundles", and
you can already get a preview here:

16) Toby Bartels, Categorified gauge theory: 2-bundles, available as
<A HREF = "http://www.arXiv.org/abs/math.CT/0410328">math.CT/0410328</A>.

2-bundles are meant to be an alternative to gerbes: although I've done
my best to hide it above, a gerbe is really more like a categorified
\emph{sheaf} than a bundle.  And, just as a bundle has a sheaf of
sections, we're hoping that a 2-bundle has a stack of sections, which
in certain cases will be a gerbe.  That's one of the things we need to
figure out, though.

And, while I'm listing the papers of the gerbe gang, I should admit that 
Urs and I have written a paper about connections on 2-bundles.  But, I
want to polish this paper a bit before talking about it here.

As for 2-groups, various people have been studying their representations
lately, and this should become an important part of higher gauge theory, 
just as group representations are crucial in gauge theory:

17) Magnus Forrester-Barker, Representations of crossed modules and 
cat^{1}-groups, Ph.D. thesis, Department of Mathematics, University of
Wales, Bangor, 2004.  Available at 
<a href = "http://www.maths.bangor.ac.uk/research/ftp/theses/forrester-barker.pdf">
http://www.maths.bangor.ac.uk/research/ftp/theses/forrester-barker.pdf</a>

18) John Barrett and Marco Mackaay, Categorical representations of 
categorical groups, available as <A HREF = "http://www.arXiv.org/abs/math.CT/0407463">math.CT/0407463</A>.

19) Josep Elgueta, Representation theory of 2-groups on finite dimensional
2-vector spaces, available as <A HREF = "http://www.arXiv.org/abs/math.CT/0408120">math.CT/0408120</A>.
 
20) Louis Crane and David Yetter, Measurable categories and 2-groups,
available as 
<A HREF = "http://www.arXiv.org/abs/math.QA/0305176">math.QA/0305176</A>.

21) David Yetter, Measurable categories, 
available as
<A HREF = "http://www.arXiv.org/abs/math.CT/0309185">math.CT/0309185</A>.

22) Louis Crane and Marnie D. Sheppeard, 2-categorical Poincare
representations and state sum applications, available as
<A HREF = "http://www.arXiv.org/abs/math.QA/0306440">math.QA/0306440</A>.

Hendryk Pfeiffer's papers on higher gauge theory are also very 
interesting.  Since he works on lattice gauge theory and spin foam
models, the first two papers here develop higher gauge theory on a
discrete spacetime, and then compare it to higher gauge theory on a 
manifold:

23) Hendryk Pfeiffer, Higher gauge theory and a non-Abelian generalization of
2-form electrodynamics, Annals Phys. 308 (2003), 447-477.  Also available as
<A HREF = "http://xxx.lanl.gov/abs/hep-th/0304074">hep-th/0304074</A>.

24) Florian Girelli and Hendryk Pfeiffer, Higher gauge theory - differential 
versus integral formulation, Jour. Math. Phys. 45 (2004), 3949-3971.  
Also available as <A HREF = "http://xxx.lanl.gov/abs/hep-th/0309173">hep-th/0309173</A>.

25) Hendryk Pfeiffer, 2-groups, trialgebras and their Hopf
categories of representations, available as <A HREF = "http://www.arXiv.org/abs/math.QA/0411468">math.QA/0411468</A>.

The third one partially fulfills an old dream of Crane and Frenkel - a
dream I vaguely hinted at way back in "<A HREF =
"week50.html">week50</A>".  Their dream was to find a concept of
"trialgebra" such that a trialgebra has a Hopf category of
representations, which in turn can have a monoidal 2-category of
representations of its own.  This is a kind of aggravated version of a
pattern already familiar in algebra, where a Hopf algebra (or
bialgebra) has a monoidal category of representations.

Pfeiffer doesn't define general trialgebras, but only
"cocommutative trialgebras" and "commutative
cotrialgebras".  A cocommutative trialgebra is a category in the
category of cocommutative Hopf algebras, while a commutative
cotrialgebra is a category in the opposite of the category of
commutative Hopf algebras.  Zounds - say that three times fast!

He shows you can get these two gadgets from 2-groups in analogy to how you get 
cocommutative or commutative Hopf algebras from groups, by taking the group 
algebra or the algebra of functions on a group.   He also proves a Tannaka-
Krein theorem that lets you reconstruct commutative cotrialgebras from their
Hopf categories of representations. 

Really cool stuff!

By the way, here are some photos of Larry Breen's conference,
and of Paris:

26) John Baez, Paris, <A HREF = "http://math.ucr.edu/home/baez/paris/">http://math.ucr.edu/home/baez/paris/</A>


\par\noindent\rule{\textwidth}{0.4pt}
% </A>
% </A>
% </A>
