
% </A>
% </A>
% </A>
\week{February 18, 1994}

Well, I'm really busy these days trying to finish up a big project,
hence the low number of "Weeks" per week, but papers are piling up, and
there are some pretty interesting ones, so I thought I'd quickly mention
a few.  This bunch will mainly concern quantum gravity.  

1) Possible implications of the quantum theory of gravity, by Louis
Crane, 5 pages in LaTeX format, available as <A HREF = "http://xxx.lanl.gov/abs/hep-th/9402104">hep-th/9402104</A>.  

This is one paper that everyone can read and enjoy, for although one may
find it too close to science fiction for comfort, it is far more
interesting than most science fiction.  Louis Crane has been doing a lot
of excellent work on topological quantum field theory for the last few
years, strongly advocating the use of category theory as a unifying
principle in physics (essentially as an extension of the concept of
symmetry embodied in \emph{group} theory), but this is quite different in
flavor.  

To begin with, Lee Smolin, one of the originators of the loop representation
of quantum gravity, has been spending the last year or so writing a book
in a popular style, to be entitled "Life and Light," which tours the
cosmos and makes some interesting speculations on "evolutionary
cosmology."  These speculations are based on 2 hypotheses.

A. The formation of a black hole creates "baby universe," the final
singularity of the black hole tunnelling right on through to the initial
"big bang" singularity of the new universe thanks to quantum effects.
While this must undoubtedly seem outre to anyone unfamiliar with the
sort of thing theoretical physicists amuse themselves with these days,
in a recent review article by John Preskill on the information loss
paradox for black holes, he reluctantly concluded that this was the
\emph{most conservative} solution of that famous problem!  Recall the
problem: if a black hole evaporates its mass away via Hawking radiation,
and that radiation is pure blackbody radiation, hence carries none of
the information about the matter that originally formed the black hole,
one does not have conservation of information, or more technically
speaking, the time evolution is not unitary, since a pure state is
evolving into a mixed state.  Hawking's original solution to this
problem was to bite the bullet and accept the nonunitarity, even though
it goes against the basic principles of quantum theory.  This appears in:

2) S. W. Hawking, Phys. Rev. D13, 191 (1976).

The "baby universe" solution simply says that the matter seeds a baby
universe and the information goes \emph{there}.  Many other solutions have
been proposed; two recent review articles are

3) Do Black Holes Destroy Information? by J. Preskill, Caltech
report CALT-68-1819, available as <A HREF = "http://xxx.lanl.gov/abs/hep-th/9209058">hep-th/9209058</A>, Sept. 1992.  

4) Black hole information, by Don Page, review lecture to be published
in Proceedings of the 5th Canadian Conference on General Relativity and
Relativistic Astrophysics, University of Waterloo, 13-15 May, 1993},
edited by R. B.  Mann and R. G. McLenaghan (World Scientific, Singapore,
1994), now available in LaTeX form as <A HREF = "http://xxx.lanl.gov/abs/hep-th/9305040">hep-th/9305040</A>.

Personally, I am a complete agnostic about this problem, since it rests
upon so many phenomena that are hypothesized but not yet observed, and
since any solution would require a theory of quantum gravity.  I am
merely reporting the ideas of respected physicists!  In any event, the
second hypothesis is:

B. Certain parameters of the baby universe are close to but different
than those of the parent universe.  The notion that certain physical
facts that appear as "laws" are actually part of the state of the
univese has in fact been rather respectable since the application of
spontaneous symmetry breaking to the Weinberg-Salam model of electroweak
interactions, part of the standard model.  (Again, being my usual
cautious self, I must note that a crucial piece of evidence for this
model, the Higgs boson, has not yet been seen.)  The notion of
spontaneous symmetry breaking has become quite popular in particle
physics and is a key component of all current theories, such as GUTs or
string theory, that attempt to model the messy heap of observed
particles and interactions by some pristinely symmetrical Lagrangian.
The spontaneous symmetry breaking would be expected to have occured
shortly after the big bang, when it got cool enough, much as a hot piece
of iron will randomly settle upon some direction of magnetization as its
temperature fall below the Curie temperature.  One application of this
notion to cosmology is already widely popular, namely, inflation.  In
fact, pursuing the analogy with magnetic domains, i.e.  small regions
with different directions of magnetization, cosmologists have spend a
fair amount of energy thinking about "domain walls," "cosmic strings,"
monopoles and other defects that might occur as residues of this
cooling-down process.

So again, while the idea must seem wild to anyone who has not
encountered it before, physicists these days are fairly comfortable with
the idea that certain "fundamental constants" could have been other than
they were.  As for the constants of a baby universe being close to, but
different than, those of the parent universe, there is as far as I know
no suggested mechanism for this.  This is perhaps the weakest link in
Smolin's argument (though I haven't seen his book yet).  But it is at
least conceivable.

Now, given these hypotheses a marvelous consequence ensues: Darwinian
evolution!  Those universes whose parameters are such that many black
holes are formed will have many progeny, so the constants of physics can
be expected to be "tuned" for the formation of many black holes.  As
Smolin emphasizes, while the hypotheses A and B may seem impossible to
test directly at present, we do at least have a hope of testing this
consequence.  He has studied the marvelously intricate process of star
formation in the galaxy and attempted to see whether altering the
constants of physics appear "tuned" for maximizing black hole
production, and he argues in his book that they do appear so tuned.  
Of course, this is an extremely delicate business, since our
understanding of galaxy formation, star formation and black hole
formation even in \emph{this} universe is still rather weak - much less for
other conceivable universes in which the fundamental constants take
different values.  

Crane enters the fray at this point, and proposes an additional
conjecture:

<div align = center>
SUCCESSFUL INDUSTRIAL CIVILIZATIONS WILL EVENTUALLY CREATE BLACK
HOLES.
</div>

(The capital letters are his.)  He breaks it up into two parts for us:

<div align = center>
SUBCONJECTURE 1: SUCCESSFUL INDUSTRIAL CIVILIZATIONS WILL EVENTUALLY
WANT TO MAKE BLACK HOLES
</div>

and

<div align = center>
SUBCONJECTURE 2: SUCCESSFUL INDUSTRIAL CIVILIZATIONS WILL EVENTUALLY
BE ABLE TO PRODUCE BLACK HOLES.
</div>

and argues for each.  The result, as any good evolutionist will
recognize, is a kind of feedback loop whereby intelligence and baby
universe formation both affect each other.  Indeed, Crane calls his
hypothesis the "meduso-anthropic hypothesis," after certain jellyfish
with a two-stage life cycle in which medusids produce polyps and vice
versa.  This has the charm of completely destroying the usual approach
(dare I say "paradigm"?) of physics in which the parameters of the
universe are regarded as indifferent to the existence of intelligence.
Of course, the anthropic hypothesis is a previous attempt to breach this
firewall, but a much less dramatic one, since the only role intelligence
plays in that is \emph{noticing} the laws of the universe.

At this point let me leave off with a quote from Crane's paper:

\begin{quote}
"It is not hard to see that if these ideas are true, they will be the
victims of abuse to dwarf quantum healing and even quantum golf. That
is not sufficient reason to ignore them."
\end{quote}

and let me \emph{gradually} turn towards slightly less speculative realms,
eventually finishing with some papers containing rigorous mathematics!
To begin with, some more on black hole entropy:

5)  Some Speculations about Black Hole Entropy in String Theory,
Leonard Susskind, 11 pages in AMSTeX, available as <A HREF = "http://xxx.lanl.gov/abs/hep-th/9309145">hep-th/9309145</A>.

Black hole entropy in canonical quantum gravity and superstring
theory, by L. Susskind and J. Uglum, 29 pages, available as <A HREF = "http://xxx.lanl.gov/abs/hep-th/9401070">hep-th/9401070</A>.

The fact that the entropy of a black hole is (at least under certain
circumstances) proportional to the area of its event horizon is a
curious relationship between general relativity, quantum field theory
and statistical mechanics that many people believe to pointing
somewhere, but unfortunately nobody is sure where.  Part of the reason
is that the standard derivations are somewhat indirect, and the event
horizon is not a physical object, so the sense in which it is the locus
of entropy is difficult to understand.  These authors suggest that in
string theory it can be explained in terms of open strings having both
ends attached to the horizon.  

6) Black hole evaporation without information loss, by C.R. Stephens, G.
't Hooft and B. F. Whiting, 35 pages in TeX format, 3 figures in
postscript, available as <A HREF = "http://xxx.lanl.gov/abs/gr-qc/9310006">gr-qc/9310006</A>.  

This is an attempt to make black holes radiate away and disappear
in a manner that preserves unitarity.  I've been too busy to read it.
And now for some wormholes:

7) Complementarity in Wormhole Chromodynamics, by Hoi-Kwong Lo, Kai-Ming
Lee, and John Preskill, 12 pages and 2 figures, phyzzx macros required,
available as <A HREF = "http://xxx.lanl.gov/abs/hep-th/9308044">hep-th/9308044</A>.

Let me just quote the abstract and note that there is probably some quite
interesting topology to be obtained by applying this sort of idea to
mathematics:

\begin{quote}
The electric charge of a wormhole mouth and the magnetic flux 
"linked" by
the wormhole are non-commuting observables, and so cannot be simultaneously
diagonalized.  We use this observation to resolve some puzzles in wormhole
electrodynamics and chromodynamics.  Specifically, we analyze the color
electric field that results when a colored object traverses a wormhole, and
we discuss the measurement of the wormhole charge and flux using Aharonov-Bohm
interference effects.  We suggest that wormhole mouths may obey conventional
quantum statistics, contrary to a recent proposal by Strominger.
\end{quote}

Finally, lest the mathematicians think I have abandoned ship, some
rigorous results:

8) "No Hair" Theorems - Folklore, Conjectures, Results, by  Piotr T.
Chrusciel, Garching preprint MPA 792, 30 pages available in LaTeX form
as <A HREF = "http://xxx.lanl.gov/abs/gr-qc/9402032">gr-qc/9402032</A>. 

The famous "no hair" theorem says that in general relativity static
black hole solutions are determined by very few parameters - typically
listed as mass, angular momentum and charge in "rest frame" of the black
hole.  There have been many attempts to extend this result, especially
because no \emph{actual} black hole is likely to be utterly static, since it
presumably formed at some time.  I have not read this but Chrusciel is a
very careful person so I expect it will be up to the standards of his
nice review of work on the cosmic censorship hypothesis,

9) On uniqueness in the large of solutions of Einstein's equations
("Strong cosmic censorship"), by Piotr T. Chrusciel, in Mathematical
Aspects of Classical Field Theory, Contemp. Math. 132, eds. Gotay,
Marsden and Moncrief, AMS, Rhode Island, 1992, pp. 235-274.

\par\noindent\rule{\textwidth}{0.4pt}
\textbf{Addendum}: 
See "<A HREF = "week33.html">week33</A>" for a paper 
by Smolin on his evolutionary cosmology theory.
His book came out in 1997 under the title
"The Life of the Cosmos" - see 
"<A HREF = "week101.html">week101</A>" for details.

\par\noindent\rule{\textwidth}{0.4pt}
<em>The thing that makes things
and the thing that makes things fall apart - they're the same
thing.  Entropy maximization!</em> - Chris Lee

\par\noindent\rule{\textwidth}{0.4pt}

% </A>
% </A>
% </A>
