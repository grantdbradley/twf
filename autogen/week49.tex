
% </A>
% </A>
% </A>
\week{ebruary 27, 1995}

For the last year or so I've been really getting interested in
n-categories as a possible tool for unifying a lot of strands
in mathematics and physics.  What's an n-category?  Well, in a
sense that's the big question!  Roughly speaking, it's a structure
where there are a bunch of "objects", and for any pair of objects x,y 
a bunch of "morphisms" from x to y, written f: x \to  y, and for
any pair of morphisms f, g: x \to  y a bunch of "2-morphisms"
from f to g, written F: f => g, and for any pair of 2-morphisms
F, G: f => g a bunch of "3-morphisms" from F to G,... and so on,
up to n-morphisms.  Ordinary categories, or 1-categories, have
been studied since the 1940s or so, when they were invented by 
Eilenberg and Mac Lane.  Anyone wanting to get going on those could try:

1) Categories for the Working Mathematician, by S. Mac Lane,
Springer, Berlin, 1988.

Roughly, categories give a general framework for
dealing with situations where you have "things" and "ways to
go between things", like sets and functions, vector spaces
and linear maps, points in a space and paths between points, etc.
That's a pretty broad territory!  n-categories show up when you
also have "ways to go between ways", "ways to go between ways
to go between ways", etc.  That may seem a little weird at first.
But in fact they show up in a lot of places if you look for them.
Perhaps the most obvious place is topology.  If you think of a
point in a space as an object, and a path between two points as
a morphism:


$$
                            f
         x ----------------->-------------------- y
$$
    
you are easily tempted to think of a "path of paths" as a 2-morphism.
Here a "path of paths" is just a continuous 1-parameter family of
paths from x to y, which you can think of as tracing out a 2-dimensional
surface, as follows:

$$
                          f
             ------------>-----------------
           /                                \
          /              |                   \
         /               | F                  \
        x                |                     y
         \               V                    /
          \                                  /
           \             g                  /
             ----------->------------------  

$$
    

And one can keep on going and look at "paths of paths of paths", etc.
In fact, people in homotopy theory do this all the time.  

There is another example, equally primordial, which is a bit more
"inbred" in flavor.  In other words, if you already know and love
n-categories, there is a wonderful example of an (n+1)-category which
you should know and love too!  Now this isn't so bad, actually,
because a 0-category is basically just a \emph{set}, namely the set of
objects.  Since everyone knows and loves sets, everyone can start
here!  Okay, there is a wonderful 1-category called Set, the category
of all sets.  This has sets as its objects and functions between sets
as its morphisms.  So now that you know an example of a 1-category,
you know and love 1-categories, right?  Well, it turns out there is
this wonderful thing called Cat, the 2-category of all 1-categories.
(Usually people restrict to "small" 1-categories, which have a mere
\emph{set} of objects, so that set theorists don't start freaking out at a
certain point.)  To understand why Cat is a 2-category is a bit of
work, but as objects it has categories, as morphisms it has the usual
sort of morphisms between categories, so-called "functors", and as
2-morphisms it has the usual sort of things that go between functors,
so-called "natural transformations".  These are the bread and butter
of category theory; just take my word for it if you haven't studied
them yet!  Okay, so Cat is a 2-category, so now you know and love
2-categories, right?  (Well, I haven't even told you the definitions,
but just nod your head.)  Guess what: there is this wonderful thing
called 2Cat, the 3-category of all 2-categories!  And so on.

So in short, the detailed theory of n-categories at each level
automatically leads one to get interested in (n+1)-categories.  Now
for the bad news: so far, people have only figured out the right
definition of n-category for n = 0, 1, 2, and 3.  By the "right"
definition I mean the ultimate, most general definition, which should
be the most useful in many ways.  So far people only know about
"strict" n-categories for all n, which one can think of as a special
case of the ultimate ones; the ultimate 1-categories are just
categories, the ultimate 2-categories are often called bicategories
(see the reference to Benabou in <A HREF = "week35.html">week35</A>), and ultimate 3-categories
are usually called tricategories (see the reference to the paper by
Gordon, Power and Street) in <A HREF = "week29.html">week29</A>.  Tricategories were just defined
last year!  They have a whole lot to do with knots, Chern-Simons
theory, and other 3-dimensional phenomena, as one might expect.  If we
could understand the ultimate 4-categories - tetracategories? - it
would probably help us with some of the riddles of topology and
physics in 4 dimensions.  (Indeed, what little we \emph{do} understand
is already helping a bit.)  

So anyway, I have been trying to learn about these things, and
had the good luck to meet James Dolan via the net, who has
helped me immensely, since he eats, lives and breathes category theory,
and he is now at Riverside hard at work figuring out the ultimate 
definition of n-categories for all n.   (Although when he reads this,
he will not be hard at work; he will be goofing off, reading the news.)

He and I have have written one paper so far espousing our philosophy
concerning n-categories, topology, and physics:

2) Higher-dimensional algebra and topological quantum field theory,
by John Baez and James Dolan, Jour. Math. Phys. 36 (1995), 6073-6105. 
Also available as
<a href = "http://arxiv.org/abs/q-alg/9503002">arXiv:q-alg/9503002</a>.

One of the main themes of this paper is what I sometimes jokingly call
the "periodic table".  Say you have an (n+k)-category with only one
object, one morphism, one 2-morphism, ... and only one (k-1)-morphism.
Then all the interest lies in the k-morphisms, the (k+1)-morphisms,
and so on up to the (k+n)-morphisms.  So there are n interesting
levels of morphism, and we can actually think of our (n+k)-category as
an n-category of a special sort.  Let's call this kind a "k-tuply
monoidal n-category".  Now we can make a chart of these:


\begin{verbatim}
                   k-tuply monoidal n-categories 

              n = 0           n = 1             n = 2

k = 0         sets          categories         2-categories
     

k = 1        monoids         monoidal           monoidal
                            categories        2-categories

k = 2       commutative      braided            braided
             monoids         monoidal           monoidal
                            categories        2-categories 

k = 3         " "           symmetric            weakly
                             monoidal          involutory
                            categories          monoidal
                                              2-categories

k = 4         " "             " "               strongly 
                                               involutory
                                                monoidal
                                              2-categories

k = 5         " "             " "                "  "
\end{verbatim}
    

First, I should emphasize that some parts of the chart as I've
drawn it here are a bit conjectural; since we don't know what
the most general 7-categories are like, for example, we don't
really know for sure what 5-tuply monoidal 2-categories are like.
The exact status of all the entries on the table is made more
clear in the paper.  For now, let me just say, first, that these
various flavors of n-categories turn out to be of great interest
in topology - some have already been used a lot in topological
quantum field theory and knot theory, other less, so far, but they
all seem to have lot to do with generalizations of knot theory to
different dimensions.  Second, it seems that the nth column "stabilizes"
by the time you get down to the (n+2)nd row.  This very interesting
pattern turns out also to have a lot to do with knots and their
generalizations, and also to a subject called stable homotopy theory.

Now it also appears that there is a nice recipe for hopping down the 
columns.   (Again, we only understand this \emph{perfectly} in certain
cases, but the pattern seems pretty clear.)  In other words, there's
a nice recipe to get a (k+1)-tuply monoidal n-category from a
k-tuply monoidal one.  It goes like this.  Hang on to your seat.
You start with a k-tuply monoidal n-category C.  It's a special
sort of (n+k)-category, so its an object in (n+k)Cat.  But
(n+k)Cat, remember, is an (n+k+1)-category.  Now look at the
largest sub-(n+k+1)-category of (n+k)Cat which has C as its
only object, 1_{C} (the identity of C) as its only morphism,
1_{1<sub>C}</sub> 
as its only 2-morphism, 1_{1<sub>1<sub>C}</sub></sub> 
as its only 3-morphism,
and so on, up to 1_{1<sub>1<sub>1<sub>&hellip;}</sub></sub></sub>
as its only k-morphism.  Let's call
this C'.  If one keeps count, this should be a (k+1)-tuply monoidal 
n-category.   That's how it goes. 

Now say we do this to an example.  Say we do it to the category C
of all representations of a finite group G.  This is in fact a monoidal
category, so the result C' is a braided monoidal category.  It is,
in fact, just the category of representations of the "quantum double"
of G, which is an example of what one might call a "finite quantum
group".  These play a big role in the study of Chern-Simons theory
with finite gauge group (see the papers by Freed and Quinn in <A HREF = "week48.html">week48</A>).  
One can also get the other quantum groups with the aid of this
"quantum double" trick.  A good description of this case appears
in:

3) Double construction for monoidal categories, by Christian Kassel and
Vladimir Turaev, Publication de l'Institute de Recherche Mathematique
Avancee, 1992.

So this is rather remarkable: starting from a finite group, and
all this n-categorical abstract nonsense, out pops precisely the raw
ingredients for a perfectly respectable 3-dimensional topological
quantum field theory!  Understanding \emph{why} this kind of thing works
is part of the aim of Dolan's and my paper, though there are some
important pieces of the puzzle that we don't get around to mentioning
there.

Right now I'm busily working out the details of how to get braided
monoidal 2-categories from monoidal 2-categories by the same trick,
with the aid of Martin Neuchl and Frank Halanke here.  These should
have a lot to do with 4-dimensional topological quantum field theories
(see e.g. the paper by Crane and Yetter cited in <A HREF = "week46.html">week46</A>).  And here I
can't resist mentioning a very nice paper by Neuchl and Schauenburg,

4) Reconstruction in braided categories and a notion of commutative
bialgebra, Martin Neuchl and Peter Schauenburg, Mathematisches
Institut, Theresienstr. 39, 80333 Muenchen, Feb. 20, 1995.

Let me conclude by describing this.  I always let myself get a bit
more technical at the end of each issue, so I'll do that now.
The relationship between Hopf algebras and monoidal categories is
given by "Tannaka-Krein reconstruction theorems", which give conditions under
which a monoidal category is equivalent to the category of representations
of a Hopf algebra, and actually constructs the Hopf algebra for you.
In physics people use related but fancier "Doplicher-Haag-Roberts"
theorems to reconstruct the gauge group of a quantum field theory.
This paper starts with the beautiful Tannaka-Krein theorem in

5) Tannaka duality for arbitrary Hopf algebras, by Peter Schauenburg,
Algebra-Berichte 66 (1992).

Leaving out a bunch of technical conditions that make the theorem
actually TRUE, it says roughly that when you have a braided monoidal
category B, a category, and a functor f: C \to  B, there is a coalgebra
object a in B, the universal one for which f factors through the forgetful
functor from a-Comod (the category of a-comodule objects in B) to B.
The point is that the ordinary Tannaka-Krein theorem is a special
case of this one where B is the category of vector spaces.  The point
of the new paper is as follows.  Suppose C is actually braided monoidal and
f preserves the braiding and monoidal structure.  Then we expect a to 
actually be something like commutative bialgebra object in B.  The
paper makes this precise.  There are actually some sneaky issues involved
in doing so.  In particular, the "quantum double" trick for categories
makes an appearance here.  I guess I'll leave it at that!
<HR>

% </A>
% </A>
% </A>


% parser failed at source line 318
