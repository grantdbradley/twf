
% </A>
% </A>
% </A>
\week{April 9, 1997 }


Darwinian evolution through natural selection is an incredibly powerful
way to explain the emergence of complex organized structures.  However,
it is not the \emph{only} important process that naturally gives rise to
complex structures.  Maybe when we study biology we should also look for
other ways that order can spontaneously arise.

After all, there are plenty of complex structures in the nonbiological
world.  When it snows, we see lots of beautiful snowflakes with similar
but different hexagonal structures.  Do we conclude that snowflakes
\emph{evolved} to be hexagonal through natural selection?  No.

But wait!  Maybe in some sense a hexagonal snowflake is "more fit" in
certain weather conditions.  Perhaps this shape is more efficient at
getting water molecules to adhere to it than other shapes.  We can
think of different snowflakes as engaged in "competition" for water
molecules, and the ones that grow fastest as the "winners".  In fact,
the exact shapes of snowflakes in a snowstorm depend crucially on 
the temperature, humidity and so on... so who the "winners" are depends
on the environment, just as in Darwinian evolution!

A biologist will reply: fine, but this is still not "Darwinian
evolution".  For Darwinian evolution in the strict sense, we require
that there be a "lineage".  Darwinian evolution applies only to entities
that reproduce and pass some of their traits down to descendants.  The
idea is that over the course of many generations, traits that aid
reproduction will accumulate, while traits that hinder it will be weeded
out.  Snowflakes don't have kids.  A one-shot competition for resources,
followed by melting into oblivion the next day, is not what Darwinian
evolution is about.

Okay, okay, so it's not Darwinian evolution.  But it's still
interesting.  It's showing us that Darwinian evolution is just \emph{one}
of various ways that order can arise.  So we shouldn't study Darwinian
evolution in isolation.  We should study \emph{all} the ways that systems
spontaneously generated complex patterns, and see how they relate.  If
we do that, perhaps we'll see a bunch of interesting relationships
between physics and chemistry and biology.  Also, maybe we'll get a better
handle on how life arose in the first place... that curious transition
from chemistry to biology.  

If I wasn't so hooked on quantum gravity I would love to work on this
stuff.  It's obviously cool, and obviously a lot more \emph{practical} than
quantum gravity.  The origin of complexity a very hot topic these days.
But alas, I am just an old-fashioned guy in love with simplicity.
Whenever I see a new journal come out with a title like "Complex
Systems" or "Journal of Complexity" or "Santa Fe Institute Studies in
the Science of Complexity", I heave a wistful sigh and dream of starting
a journal entitled "Simplicity".

Actually, the fun lies in the interplay between complexity and
simplicity: how complex phenomena can arise from simple laws, and
sometimes obey new simple laws of their own.  I like to hang out on the
simple end of things, but that doesn't stop me from enjoying the new
work on complexity.  At one point I got a big kick out of Manfred
Eigen's work on "hypercycles" --- systems of chemicals that catalyze
each others formation.  (You may remember Eigen as the discoverer of the
"Eigenvalue"... in which case I pity you.)  Presumably life started as
some sort of hypercycle, so the mathematical study of the competition
between hypercycles may shed some light on why there is only one genetic
code.  There is a lot of nice math of this type in:


1) Manfred Eigen, The Hypercycle, a Principle of Natural
Self-Organization, Springer-Verlag, Berlin, 1979.


Another name that comes up in this context is Ilya Prigogine, mainly for
his work on non-equilibrium thermodynamics and the spontaneous formation
of patterns in dissipative systems.  The following are just a few of his
many books:


2) G. Nicolis and I. Prigogine, Self-Organization in Nonequilibrium 
Systems: from Dissipative Structures to Order Through Fluctuations,
Wiley, New York, 1977.

Ilya Prigogine, From Being to Becoming: Time and Complexity in the 
Physical Sciences, W. H. Freeman, San Francisco, 1980.

Ilya Prigogine, Introduction to Thermodynamics of Irreversible
Processes,  3d ed., Interscience Publishers, New York, 1967.


A bit more recently, the work of Stuart Kauffman has dominated the
subject.  It's really him who has pushed for the unified study of the
whole gamut of methods of spontaneous generation of order, particularly
in the context of biological systems.  He's written two books.  The
latter, in particular, includes a lot of math problems just \emph{waiting}
to be tackled by good mathematicians and physicists.


3) Stuart A. Kauffman, At Home in the Universe: the Search for Laws of
Self-Organization and Complexity, Oxford University Press, New York,
1995.

Stuart A. Kauffman, The Origins of Order: Self-Organization and
Selection in Evolution, Oxford University Press, New York, 1993.


If non-Darwinian forms of spontaneous pattern-formation can be important
in biology, can Darwinian evolution be important in non-biological
contexts?  Well, as I mentioned in "<A HREF = "week31.html">week31</A>" and "<A HREF = "week33.html">week33</A>", the physicist
Lee Smolin has an interesting hypothesis about how the laws of nature
may have evolved to their present point by natural selection.  The idea
is that black holes beget new "baby universes" with laws similar but not
necessarily quite the same as their ancestors.  Now this is extremely
speculative, but it has the saving virtue of making a lot of testable
predictions: it predicts that all the constants of nature are tuned so
as to maximize black hole production.  Smolin has just come out with
a book on this, which also happens to be a good place to learn about
his work on quantum gravity:


4) Lee Smolin, The Life of the Cosmos, Crown Press, 1997.


Interestingly, Stuart Kauffman and Lee Smolin have teamed up to
write a paper on the problem of time in quantum gravity:


5) Stuart Kauffman and Lee Smolin, A possible solution to the problem
of time in quantum cosmology, preprint available as <A HREF = "http://xxx.lanl.gov/abs/gr-qc/9703026">gr-qc/9703026</A>.


Right now you can also read this paper on John Brockman's website called
"Edge".  This website features all sorts of fun interviews and
discussions.  For example, if you look now you'll find an intelligent
interview with my favorite living musician, Brian Eno.  More to the point,
a discussion of Kauffman and Smolin's paper is happening there now.
As a long-time fan of USENET newsgroups and other electronic forms of
chitchat, I'm really pleased to see how Brockman has set up a kind of
modern-day version of the French salon.   


6) Edge: <A HREF = "http://www.edge.org">http://www.edge.org</A>


Okay.  Now... what's even more fashionable, trendy, and close to
the cutting edge than complexity theory?  You guessed it: homotopy
theory!  Currently known only to hippest of the hip, this is bound to
hit the bigtime as soon as they figure out how to make flashy color
graphics illustrating the Adams spectral sequence.

Last week I went to the Workshop on Higher Category Theory and Physics
at Northwestern University, and also, before that, part of a conference 
on homotopy theory they had there.   Actually these two subjects are 
closely related: homotopy theory is a highly algebraic way of studying
the topology of spaces of various dimensions, and lots of what we understand
about "higher dimensional algebra" comes from homotopy theory.  So it was
a nice combination.

Lots of the homotopy theory was over my head, alas, but what I
understood I enjoyed.  It may seem sort of odd, but the main thing I got
out of the homotopy theory conference was an explanation of why the
number 24 is so important in string theory!  In bosonic string theory
spacetime needs to be 26-dimensional, but subtracting 2 dimensions for
the surface of the string itself we get 24, and it turns out that it's
really the special properties of the number 24 that make all the magic
happen.

I began to delve into these mysteries in "<A HREF = "week95.html">week95</A>".  There, however, I
was mainly reporting on very fancy stuff that I barely understand, stuff
that seems like a pile of complicated coincidences.  Now, I am glad to
report, I am beginning to understand the real essence of this 24
business.  It turns out that the significance of the number 24 is woven
very deeply into the basic fabric of mathematics.  To put it rather
mysteriously, it turns out that every integer has some subtle "hidden
symmetries".  These symmetries have symmetries of their own, and in turn
THESE symmetries have symmetries of THEIR own - of which there are
exactly 24.

Hmm, mysterious.  Let me put it another way.  It probably won't be
obvious why this is another way of saying the same thing, but it has the
advantage of being more concrete.  Suppose that the integer n is
sufficiently large - 4 or more will do.  Then there are 24 essentially
different ways to wrap an (n+3)-dimensional sphere around an
n-dimensional sphere.  More precisely still, given two continuous
functions from an (n+3)-sphere to an n-sphere, let's say that they lie
in the same "homotopy class" if you can continuously deform one into
another.  Then when n is 4 or more, it turns out that there are exactly
24 such homotopy classes.

Now that I have all the ordinary mortals confused and all the homotopy
theorists snickering at me for making such a big deal out of something
everyone knows, I should probably go back and explain what the heck I'm
getting at, and why it has to do with string theory.  But I'm getting
worn out, and your attention is probably flagging, so I'll do this next
time.  I'll say a bit about homotopy theory, stable homotopy theory, the
sphere spectrum, and why Andre Joyal says we should call the sphere
spectrum the "integers" (thus explaining my mysterious remark above).








\par\noindent\rule{\textwidth}{0.4pt}
<em>Deep, deep infinity!  Quietness.  To dream away from the tensions
of daily living; to sail over a calm sea at the prow of a ship, toward
a horizon that always recedes; to stare at the passing waves and listen
to their monotonous soft murmur; to dream away into unconsciousness....</em> 
- Maurits Escher

<HR>

% </A>
% </A>
% </A>


% parser failed at source line 288
