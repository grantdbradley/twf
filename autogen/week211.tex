
% </A>
% </A>
% </A>
\week{March 6, 2005 }



The last time I wrote an issue of this column, the Huyghens probe was
bringing back cool photos of Titan.  Now the European "Mars Express"
probe is bringing back cool photos of Mars!  

1) Mars Express website, <A HREF = "http://www.esa.int/SPECIALS/Mars_Express/index.html">http://www.esa.int/SPECIALS/Mars_Express/index.html</A>

There are some tantalizing pictures of what might be a "frozen sea" - 
water ice covered with dust - near the equator in the Elysium Planitia region:

<BR>
<A HREF = "http://www.esa.int/SPECIALS/Mars_Express/SEMCHPYEM4E_0.html">
<DIV ALIGN = CENTER>
<IMG WIDTH = 400 HEIGHT = 340 SRC = mars_packice.jpg>
</DIV>
% </A>
<BR>

2) Mars Express sees signs of a "frozen sea", 
<A HREF = "http://www.esa.int/SPECIALS/Mars_Express/SEMCHPYEM4E_0.html">
http://www.esa.int/SPECIALS/Mars_Express/SEMCHPYEM4E_0.html</A>

Ice has already been found at the Martian poles - it's easily visible there, 
and Mars Express is getting some amazing closeups of it now - here's a
here's a view of some ice on sand at the north pole:


<br>
<A HREF = "http://www.esa.int/SPECIALS/Mars_Express/SEMLF6D3M5E_1.html">
<DIV ALIGN = CENTER>
<IMG WIDTH = 400 HEIGHT = 340 SRC = mars_pole.jpg>
</DIV>
% </A>
<br>
3) Glacial, volcanic and fluvial activity on Mars: latest images,
<A HREF = "http://www.esa.int/SPECIALS/Mars_Express/SEMLF6D3M5E_1.html">
http://www.esa.int/SPECIALS/Mars_Express/SEMLF6D3M5E_1.html</A>

What's new is the possibility of large amounts of water in warmer parts of 
the planet. 

Now for some math.  It's always great when two subjects you're interested in 
turn out to be bits of the same big picture.  That's why I've been really 
excited lately about Bott periodicity and the "super-Brauer group".  

I wrote about Bott periodicity in "<A HREF = "week105.html">week105</A>", and about the Brauer group
in "<A HREF = "week209.html">week209</A>", but I should remind you about them before putting them together.

Bott periodicity is all about how math and physics in n+8-dimensional space 
resemble math and physics in n-dimensional space.  It's a weird and wonderful
pattern that you'd never guess without doing some calculations.  It shows up
in many guises, which turn out to all be related.  The simplest one to verify 
is the pattern of Clifford algebras.

You're probably used to the complex numbers, where you throw in just 
\emph{one}
square root of -1, called i.  And maybe you've heard of the quaternions, where 
you throw in \emph{two} 
square roots of -1, called i and j, and demand that they
anticommute:

\begin{verbatim}
ij = -ji
\end{verbatim}
    
This implies that k = ij is another square root of -1.   Try it and see!

In the late 1800s, Clifford realized there's no need to stop here.  He invented 
what we now call the "Clifford algebras" by starting with the real numbers and 
throwing in n square roots of -1, all of which anticommute with each other.
The result is closely related to rotations in n+1 dimensions, as I explained in 
"<A HREF = "week82.html">week82</A>".
  
I'm not sure who first worked out all the Clifford algebras - perhaps it was
Cartan - but the interesting fact is that they follow a periodic pattern. 
If we use C_{n} to stand for the Clifford algebra generated by n anticommuting
square roots of -1, they go like this:

\begin{verbatim}
C<sub>0</sub>  R
C<sub>1</sub>  C
C<sub>2</sub>  H
C<sub>3</sub>  H + H
C<sub>4</sub>  H(2) 
C<sub>5</sub>  C(4)
C<sub>6</sub>  R(8)
C<sub>7</sub>  R(8) + R(8)
\end{verbatim}
    
where:
<UL>
<LI>
 R(n) means n x n real matrices, 
<LI>
 C(n) means n x n complex matrices, and
<LI>
 H(n) means n x n quaternionic matrices.  
</UL>
All these become algebras with the usual addition and multiplication of 
matrices.  Finally, if A is an algebra, A + A consists of pairs of guys 
in A, with pairwise addition and multiplication.

What happens next?  Well, from then on things sort of "repeat" with period 8: 
C_{n+8} 
consists of 16 x 16 matrices whose entries lie in C_{n}!  

So, you can remember all the Clifford algebras with the help of this 
eight-hour clock:


<PRE><b>
                                    0
                                 
                                    R

                 7                                    1
                                                   
                   R+R                             C





             6   R                                       H   2
 




                    C                             H+H
                 
                  5                                    3

                                    H

                                    4
</b></PRE>
To use this clock, you have to remember to use matrices of the right size to 
get C_{n} to have dimension 2^{n}.  
So, when I write "R + R" next to the "7" on 
the clock, I don't mean C_{7} is really R + R.  To get 
C_{7}, you have to take 
R + R and beef it up until it becomes an algebra of dimension 2^{7} 
= 128.  You 
do this by taking R(8) + R(8), since this has dimension 8 x 8 + 8 x 8 = 128.  

Similarly, to get C_{10}, you note that 10 is 2 modulo 8, so you look at
"2" on the clock and see "H" next to it, meaning the quaternions.  But to get 
C_{10}, 
you have to take H and beef it up until it becomes an algebra of 
dimension 2^{10} = 1024.  You do this by taking H(16), since this 
has dimension 4 x 16 x 16 = 1024.   

This "beefing up" process is actually quite interesting.  For any associative
algebra A, the algebra A(n) consisting of n x n matrices with entries in A 
is a lot like A itself.  The reason is that they have equivalent categories 
of representations!   

To see what I mean by this, remember that a "representation" of an algebra
is a way for its elements to act as linear transformations of some vector
space.  For example, R(n) acts as linear transformations of R^{n} 
by matrix 
multiplication, so we say R(n) has a representation on R^{n}.  
More generally,
for any algebra A, the algebra A(n) has a representation on A^{n}.  

More generally still, if we have any representation of A on a vector space V, 
we get a representation of A(n) on V^{n}.  It's less obvious,
but true, that \emph{every} 
representation of A(n) comes from a representation of
A this way.  

In short, just as n x n matrices with entries in A form an algebra A(n) 
that's a 
beefed-up version of A itself, every representation of A(n) is a beefed-up
version of some representation of A.  

Even better, the same sort of thing is true for maps between representations 
of A(n).  This is what we mean by saying that A(n) and A have equivalent
categories of representations.  If you just look at the categories of 
representations of these two algebras as abstract categories,
there's no way to tell them apart!
We say two algebras are "Morita equivalent" when this happens.

It's fun to study Morita equivalence classes of algebras - say algebras over 
the real numbers, for example.  The tensor product of algebras gives us a way 
to multiply these classes.  If we just consider the invertible classes, we get 
a \emph{group}.  
This is called the "Brauer group" of the real numbers.   

The Brauer group of the real numbers is just Z/2, consisting of the classes 
[R] and [H].  These correspond to the top and bottom of the Clifford clock!  
Part of the reason is that 

\begin{verbatim}
H tensor H = R(4)
\end{verbatim}
    
so when we take Morita equivalence classes we get

\begin{verbatim}
[H] x [H] = [R]
\end{verbatim}
    
But, you may wonder where the complex numbers went!  Alas, the Morita
equivalence class [C] isn't invertible, so it doesn't live in the Brauer 
group.  In fact, we have this little multiplication table for tensor product
of algebras:


\begin{verbatim}
        tensor       R       C      H
                  ----------------------
          R      |   R       C      H
                 |
          C      |   C      C+C    C(2)
                 |
          H      |   H      C(2)   R(4)

\end{verbatim}
    
Anyone with an algebraic bone in their body should spend an afternoon
figuring out how this works!  But I won't explain it now.

Instead, I'll just note that the complex numbers are very aggressive and
infectious - tensor anything with a C in it and you get more C's.  That's
because they're a field in their own right - and that's why they don't 
live in the Brauer group of the real numbers.  

They do, however, live in the \emph{super-Brauer} group of the real numbers,
which is Z/8 - the Clifford clock itself!

But before I explain that, I want to show you what the categories of
representations of the Clifford algebras look like:

\begin{verbatim}
                                       0

                               real vector spaces
                                
      7                                                                1 
         split real vector spaces               complex vector spaces
 
                     

 
6  real vector spaces                          quaternionic vector spaces  2




        complex vector spaces         split quaternionic vector spaces
     5                                                                   3

                             
                           quaternionic vector spaces

                                      4
\end{verbatim}
    
You can read this information off the 8-hour Clifford clock I showed you 
before, at least if you know some stuff:

<UL>
<LI> 
A real vector space is just something like R^{n}
<LI>
A complex vector space is just something like C^{n}
<LI>
A quaternionic vector space is just something like H^{n}
</UL>

and a "split" vector space is a vector space that's been written 
as the direct sum of two subspaces.  

Take C_{4}, 
for example - the Clifford algebra generated by 4 anticommuting 
square roots of -1.  The Clifford clock tells us this is H + H.  And if you
think about it, a representation of this is just a pair of representations of
H.  So, it's two quaternionic vector spaces - or if you 
prefer, a "split" quaternionic vector space.

Or take C_{7}.  
The Clifford clock says this is R + R... or at least Morita 
equivalent to R + R: it's actually R(8) + R(8), but that's just a beefed-up
version of R + R, with an equivalent category of representations.  So, the 
category of representations of C_{7} is \emph{equivalent} 
to the category of split 
real vector spaces.

And so on.  Note that when we loop all the way around the clock, our 
Clifford algebra becomes 16 x 16 matrices of what it was before, but this
is Morita equivalent to what it was.   So, we have a truly period-8 clock 
of categories!

But here's the really cool part: there are also arrows going clockwise and
counterclockwise around this clock!  Arrows between categories are called
"functors".   

Each Clifford algebra is contained in the next one, since they're built 
by throwing in more and more square roots of -1.  So, if we have a 
representation of C_{n}, it gives us a 
representation of C_{n-1}.  Ditto 
for maps between representations.  So, we get a functor from the category 
of representations of C_{n} to the category of 
representations of C_{n-1}.  
This is called a "forgetful functor", since it "forgets" 
that we have 
representations of C_{n} and just thinks of them as 
representations of C_{n-1}.

So, we have forgetful functors cycling around counterclockwise!

Even better, all these forgetful functors have "left adjoints" going 
back the other way.   I talked about left adjoints in "<A HREF = "week77.html">week77</A>",
so I won't say much about them now.  I'll just give an example.  

Here's a forgetful functor:

$$
                        forget complex structure
complex vector spaces ---------------------------> real vector spaces
$$
    
which is one of the counterclockwise arrows on the Clifford clock.
This functor takes a complex vector space and forgets your ability to multiply 
vectors by i, thus getting a real vector space.  When you do this to 
C^{n}, you get R^{2n}.  

This functor has a left adjoint:

$$
                               complexify
complex vector spaces <-------------------------- real vector spaces
$$
    
where you take a real vector space and "complexify" it by tensoring it with
the complex numbers.  When you do this to R^{n}, you get C^{n}.

So, we get a beautiful version of the Clifford clock with forgetful functors 
cycling around counterclockwise and their left adjoints cycling around 
clockwise! When I realized this, I drew a big picture of it in my math 
notebook - I always carry around a notebook for precisely this sort of thing.  
Unfortunately, it's a bit hard to draw this chart in ASCII, so I won't 
include it here. 

Instead, I'll draw something easier.  For this, note the following mystical
fact.  The Clifford clock is symmetrical under reflection around the 
3-o'clock/7-o'clock axis:


\begin{verbatim}
                                       0

                               real vector spaces
                                
      7                                                                 1                            
       split real vector spaces                   complex vector spaces
                             \ 
                               \
                                 \
                                   \
6  real vector spaces                \           quaternionic vector spaces  2
                                       \
                                         \
                                           \
                                             \
        complex vector spaces             split quaternionic vector spaces
     5                                                                     3

                             
                           quaternionic vector spaces

                                      4
\end{verbatim}
    
It seems bizarre at first that it's symmetrical along \emph{this} 
axis instead 
of the more obvious 0-o'clock/4-o'clock axis.  But there's a good reason,
which I already mentioned: the Clifford algebra C_{n} is related to rotations in 
n+1 dimensions.  

I would be very happy if you had enough patience to listen to a full 
explanation of this fact, along with everything else I want to say.  But 
I bet you don't... so I'll hasten on to the really cool stuff.

First of all, using this symmetry we can fold the Clifford clock in half... 
and the forgetful functors on one side perfectly match their left adjoints 
on the other side!    

So, we can save space by drawing this "folded" Clifford clock: 


\begin{verbatim}

                   split real vector spaces
  
                             | ^
            forget splitting | | double
                             v |
                   
                    real vector spaces

                             | ^
                  complexify | | forget complex structure
                             v |

                   complex vector spaces

                             | ^
               quaternionify | | forget quaternionic structure
                             v |

                  quaternionic vector spaces
  
                             | ^
                      double | | forget splitting
                             v |

                split quaternionic vector spaces

\end{verbatim}
    
The forgetful functors march downwards on the right, and their
left adjoints march back up on the left!

The arrows going between 7 o'clock and 0 o'clock look a bit weird:

       
\begin{verbatim}

                  split real vector spaces
  
                             | ^
            forget splitting | | double
                             V |
                   
                    real vector spaces

\end{verbatim}
    

Why is "forget splitting" on the left, where the left adjoints belong, when 
it's obviously an example of a forgetful functor?  

One answer is that this is just how it works.  Another answer is that it 
happens when we wrap all the way around the clock - it's like how going from 
midnight to 1 am counts as going forwards in time even though the number is 
getting smaller.  A third answer is that the whole situation is so symmetrical 
that the functors I've been calling "left adjoints" are also "right adjoints" of
their partners!   So, we can change our mind about which one is "forgetful",
without getting in trouble.

But enough of that: I really want to explain how this stuff is related
to the super-Brauer group, and then tie it all in to the \emph{topology} 
of Bott 
periodicity.  We'll see how far I get before giving up in exhaustion....

What's a super-Brauer group?  It's just like a Brauer group, but where we
use superalgebras instead of algebras!  A "superalgebra" is just physics 
jargon for a Z/2-graded algebra - that is, an algebra A that's a direct 
sum of an "even" or "bosonic" part A_{0} 
and an "odd" or "fermionic" part A_{1}:

\begin{verbatim}
A = A<sub>0</sub> + A<sub>1</sub>
\end{verbatim}
    
such that multiplying a guy in A_{i} 
and a guy in A_{j} gives a guy in A_{i+j},
where we add the subscripts mod 2.  
 
The tensor product of superalgebras is defined differently than for algebras.
If A and B are ordinary algebras, when we form their tensor product, we
decree that everybody in A commutes with everyone in B.   For superalgebras
we decree that everybody in A "supercommutes" with everyone in B - meaning 
that

ab = ba

if either a or b are even (bosonic) while 

ab = -ba

if a and b are both odd (fermionic).

Apart from these modifications, the super-Brauer group works almost like the
Brauer group.  We start with superalgebras over our favorite field - here 
let's use the real numbers.  We say two superalgebras are "Morita equivalent" 
if they have equivalent categories of representations.  We can multiply
these Morita equivalence classes by taking tensor products, and if we just 
keep the invertible classes we get a group: the super-Brauer group.

As I've hinted already, the super-Brauer group of the real numbers is Z/8 - 
just the Clifford algebra clock in disguise!

Here's why:

The Clifford algebras all become superalgebras if we decree that all the
square roots of -1 that we throw in are "odd" elements.  And if we do this,
we get something great:

\begin{verbatim}
C<sub>n</sub> tensor C<sub>m</sub> = C<sub>n + m</sub>
\end{verbatim}
    
The point is that all the square roots of -1 we threw in to get C_{n} 
\emph{anticommute} with those we threw in to get C_{m}.   

Taking Morita equivalence classes, this mean

\begin{verbatim}
[C<sub>n</sub>] [C<sub>m</sub>] = [C<sub>n+m</sub>] 
\end{verbatim}
    
but we already know that 

\begin{verbatim}
[C<sub>n+8</sub>] = [C<sub>n</sub>]
\end{verbatim}
    
so we get the group Z/8.  It's not obvious that this is 
\emph{all} the super-Brauer
group, but it actually is - that's the hard part.

Now let's think about what we've got.   We've got the super-Brauer group, 
Z/8, which looks like an 8-hour clock.  But before that, we had the categories 
of representations of Clifford algebras, which formed an 8-hour clock with 
functors cycling around in both directions.

In fact these are two sides of the same coin - or clock, actually.  The 
super-Brauer group consists of Morita equivalence classes of Clifford algebras, 
where Morita equivalence means "having equivalent categories of 
representations".  But, our previous clock just shows their categories of 
representations!

This suggests that the functors cycling around in both directions are secretly 
an aspect of the super-Brauer group.  And indeed they are!  The functors going 
clockwise are just "tensoring with C_{1}", since you can 
tensor a representation 
of C_{n} with C_{1} 
and get a representation of C_{n+1}.  And the functors going
counterclockwise are "tensoring with C_{-1}"... or 
C_{7} if you insist, since C_{-1} doesn't strictly 
make sense, but 7 equals -1 mod 8, so it does the
same job.

Hmm, I think I'm tired out.  I didn't even get to the topology yet!  Maybe
that'll be good as a separate little story someday.  If you can't wait, 
just read this: 

4) John Milnor, Morse Theory, Princeton U. Press, Princeton, New Jersey, 1963.

You'll see here that a representation of C_{n} is just the same as a vector
space with n different anticommuting ways to "rotate vector by 90 degrees",
and that this is the same as a real inner product space equipped with a map
from the n-sphere into its rotation group, with the property that the north 
pole of the n-sphere gets mapped to the identity, and each great circle 
through the north pole gives some action of the circle as rotations.  Using 
this, and stuff about Clifford algebras, and some Morse theory, Milnor gives a
beautiful proof that 

\Omega ^{8}(SO(\infty )) ~ SO(\infty )

or in English: the 8-fold loop space of the infinite-dimensional rotation
group is homotopy equivalent to the infinite-dimensional rotation group!

The thing I really like, though, is that Milnor relates the forgetful functors
I was talking about to the process of "looping" the rotation group.  That's
what these maps from spheres into the rotation group are all about... but I 
want to really explain it all someday!

I learned about the super-Brauer group here:

5) V. S. Varadarajan, Supersymmetry for Mathematicians: An Introduction,
American Mathematical Society, Providence, Rhode Island, 2004.

though the material here on this topic is actually a summary of some
lectures by Deligne in another book I own:

6) P. Deligne, P. Etingof, D.S. Freed, L. Jeffrey, D. Kazhdan, J. Morgan, D.R. 
Morrison and E. Witten, Quantum Fields and Strings: A Course For Mathematicians 
2 vols., American Mathematical Society, Providence, 1999.  Notes also available
at <A HREF = "http://www.math.ias.edu/QFT/">http://www.math.ias.edu/QFT/</A>

Varadarajan's book doesn't go as far, but it's much easier to read, so I 
recommend it as a way to get started on "super" stuff.


\par\noindent\rule{\textwidth}{0.4pt}
% </A>
% </A>
% </A>


% parser failed at source line 701
