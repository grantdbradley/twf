
% </A>
% </A>
% </A>
\week{March 6, 2005 }



The last time I wrote an issue of this column, the Huyghens probe was
bringing back cool photos of Titan.  Now the European "Mars Express"
probe is bringing back cool photos of Mars!  

1) Mars Express website, <A HREF = "http://www.esa.int/SPECIALS/Mars_Express/index.html">http://www.esa.int/SPECIALS/Mars_Express/index.html</A>

There are some tantalizing pictures of what might be a "frozen sea" - 
water ice covered with dust - near the equator in the Elysium Planitia region:

<BR>
<A HREF = "http://www.esa.int/SPECIALS/Mars_Express/SEMCHPYEM4E_0.html">
<DIV ALIGN = CENTER>
<IMG WIDTH = 400 HEIGHT = 340 SRC = mars_packice.jpg>
</DIV>
% </A>
<BR>

2) Mars Express sees signs of a "frozen sea", 
<A HREF = "http://www.esa.int/SPECIALS/Mars_Express/SEMCHPYEM4E_0.html">
http://www.esa.int/SPECIALS/Mars_Express/SEMCHPYEM4E_0.html</A>

Ice has already been found at the Martian poles - it's easily visible there, 
and Mars Express is getting some amazing closeups of it now - here's a
here's a view of some ice on sand at the north pole:


<br>
<A HREF = "http://www.esa.int/SPECIALS/Mars_Express/SEMLF6D3M5E_1.html">
<DIV ALIGN = CENTER>
<IMG WIDTH = 400 HEIGHT = 340 SRC = mars_pole.jpg>
</DIV>
% </A>
<br>
3) Glacial, volcanic and fluvial activity on Mars: latest images,
<A HREF = "http://www.esa.int/SPECIALS/Mars_Express/SEMLF6D3M5E_1.html">
http://www.esa.int/SPECIALS/Mars_Express/SEMLF6D3M5E_1.html</A>

What's new is the possibility of large amounts of water in warmer parts of 
the planet. 

Now for some math.  It's always great when two subjects you're interested in 
turn out to be bits of the same big picture.  That's why I've been really 
excited lately about Bott periodicity and the "super-Brauer group".  

I wrote about Bott periodicity in "<A HREF = "week105.html">week105</A>", and about the Brauer group
in "<A HREF = "week209.html">week209</A>", but I should remind you about them before putting them together.

Bott periodicity is all about how math and physics in n+8-dimensional space 
resemble math and physics in n-dimensional space.  It's a weird and wonderful
pattern that you'd never guess without doing some calculations.  It shows up
in many guises, which turn out to all be related.  The simplest one to verify 
is the pattern of Clifford algebras.

You're probably used to the complex numbers, where you throw in just 
\emph{one}
square root of -1, called i.  And maybe you've heard of the quaternions, where 
you throw in \emph{two} 
square roots of -1, called i and j, and demand that they
anticommute:


\begin{verbatim}

ij = -ji
\end{verbatim}
    
This implies that k = ij is another square root of -1.   Try it and see!

In the late 1800s, Clifford realized there's no need to stop here.  He invented 
what we now call the "Clifford algebras" by starting with the real numbers and 
throwing in n square roots of -1, all of which anticommute with each other.
The result is closely related to rotations in n+1 dimensions, as I explained in 
"<A HREF = "week82.html">week82</A>".
  
I'm not sure who first worked out all the Clifford algebras - perhaps it was
Cartan - but the interesting fact is that they follow a periodic pattern. 
If we use C_{n} to stand for the Clifford algebra generated by n anticommuting
square roots of -1, they go like this:


$$

C_{0}  R
C_{1}  C
C_{2}  H
C_{3}  H + H
C_{4}  H(2) 
C_{5}  C(4)
C_{6}  R(8)
C_{7}  R(8) + R(8)
$$
    
where:
<UL>
<LI>
 R(n) means n x n real matrices, 
<LI>
 C(n) means n x n complex matrices, and
<LI>
 H(n) means n x n quaternionic matrices.  
</UL>
All these become algebras with the usual addition and multiplication of 
matrices.  Finally, if A is an algebra, A + A consists of pairs of guys 
in A, with pairwise addition and multiplication.

What happens next?  Well, from then on things sort of "repeat" with period 8: 
C_{n+8} 
consists of 16 x 16 matrices whose entries lie in C_{n}!  

So, you can remember all the Clifford algebras with the help of this 
eight-hour clock:





% parser failed at source line 246
