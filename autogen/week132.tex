
% </A>
% </A>
% </A>
\week{April 2, 1999 }

Today I want to talk about n-categories and quantum gravity again.  For
starters let me quote from a paper of mine about this stuff:

1) John Baez, Higher-dimensional algebra and Planck-scale physics,
to appear in Physics Meets Philosophy at the Planck Scale, eds. 
Craig Callender and Nick Huggett, Cambridge U. Press.  Preprint
available as 
<A HREF = "http://xxx.lanl.gov/abs/gr-qc/9902017">gr-qc/9902017</A>.

By the way, this book should be pretty fun to read - it'll contain papers 
by both philosophers and physicists, including a bunch who have already 
graced the pages of This Week's Finds, like Barbour, Isham, Rovelli, 
Unruh, and Witten.   I'll say more about it when it comes out.

Okay, here are some snippets from this paper.  It starts out talking
about the meaning of the Planck length, why it may be important in 
quantum gravity, and what a theory of quantum gravity should be like:


\begin{quote}
Two constants appear throughout general relativity: the speed of light
c and Newton's gravitational constant G.   This should be no
surprise, since Einstein created general relativity to reconcile the
success of Newton's theory of gravity, based on instantaneous action at
a distance, with his new theory of special relativity, in which no
influence travels faster than light.  The constant c also appears in
quantum field theory, but paired with a different partner: Planck's
constant &#295;.   The reason is that quantum field theory takes
into account special relativity and quantum theory, in which &#295;
sets the scale at which the uncertainty principle becomes important.  

It is reasonable to suspect that any theory reconciling general
relativity and quantum theory will involve all three constants c, G,
and &#295;.  Planck noted that apart from numerical factors there
is a unique way to use these constants to define units of length, time, 
and mass.  For example, we can define the unit of length now
called the `Planck length' as follows:

                       L = sqrt(&#295; G /c^3)

This is extremely small: about 1.6 x 10^{-35} meters.   Physicists have
long suspected that quantum gravity will become important for
understanding physics at about this scale.   The reason is very simple:
any calculation that predicts a length using only the constants c, G and
&#295; must give the Planck length, possibly multiplied by an unimportant
numerical factor like 2&pi;.  

For example, quantum field theory says that associated to any mass m
there is a length called its Compton wavelength, L_C, such that
determining the position of a particle of mass m to within one Compton
wavelength requires enough energy to create another particle of that
mass.  Particle creation is a quintessentially quantum-field-theoretic
phenomenon.  Thus we may say that the Compton wavelength sets the
distance scale at which quantum field theory becomes crucial for
understanding the behavior of a particle of a given mass.  On the other
hand, general relativity says that associated to any mass m there is a
length called the Schwarzschild radius, L_S, such that compressing
an object of mass m to a size smaller than this results in the
formation of a black hole.  The Schwarzschild radius is roughly the
distance scale at which general relativity becomes crucial for
understanding the behavior of an object of a given mass.  Now, ignoring
some numerical factors, we have  

                    L_C = &#295;/mc

and  

                    L_S = Gm/c^2. 

These two lengths become equal when m is the Planck mass.  And 
when this happens, they both equal the Planck length!

At least naively, we thus expect that both general relativity and
quantum field theory would be needed to understand the behavior of an
object whose mass is about the Planck mass and whose radius is about the
Planck length.  This not only explains some of the importance of the
Planck scale, but also some of the difficulties in obtaining
experimental evidence about physics at this scale.  Most of our
information about general relativity comes from observing heavy objects
like planets and stars, for which L_S &gt;&gt; L_C.  Most of our information
about quantum field theory comes from observing light objects like
electrons and protons, for which L_C &gt;&gt; L_S.  The Planck mass is
intermediate between these: about the mass of a largish cell.  But the
Planck length is about 10^{-20} times the radius of a proton!   To study
a situation where both general relativity and quantum field theory are
important, we could try to compress a cell to a size 10^{-20} times that
of a proton.   We know no reason why this is impossible in principle. 
But we have no idea how to actually accomplish such a feat.  

There are some well-known loopholes in the above argument.  The
`unimportant numerical factor' I mentioned above might actually be very
large, or very small.  A theory of quantum gravity might make testable
predictions of dimensionless quantities like the ratio of the muon and
electron masses.  For that matter, a theory of quantum gravity might
involve physical constants other than c, G, and &#295;.  The latter two
alternatives are especially plausible if we study quantum gravity as
part of a larger theory describing other forces and particles.  However,
even though we cannot prove that the Planck length is significant for
quantum gravity, I think we can glean some wisdom from pondering the
constants c,G, and &#295; - and more importantly, the physical insights
that lead us to regard these constants as important.

What is the importance of the constant c?   In special relativity,
what matters is the appearance of this constant in the Minkowski
metric 

                ds^2 = c^2 dt^2 - dx^2 - dy^2 - dz^2  

which defines the geometry of spacetime, and in particular the lightcone
through each point.  Stepping back from the specific formalism here, we
can see several ideas at work.  First, space and time form a unified
whole which can be thought of geometrically.  Second, the quantities
whose values we seek to predict are localized.  That is, we can measure
them in small regions of spacetime (sometimes idealized as points).  
Physicists call such quantities `local degrees of freedom'.  And third,
to predict the value of a quantity that can be measured in some region
R, we only need to use values of quantities measured in regions that
stand in a certain geometrical relation to R.  This relation is called
the `causal structure' of spacetime.  For example, in a relativistic
field theory, to predict the value of the fields in some region R, it
suffices to use their values in any other region that intersects every
timelike path passing through R.  The common way of summarizing this
idea is to say that nothing travels faster than light.  I prefer to say
that a good theory of physics should have *local degrees of freedom
propagating causally*. 

In Newtonian gravity, G is simply the strength of the gravitational
field.  It takes on a deeper significance in general relativity, where
the gravitational field is described in terms of the curvature of the
spacetime metric.  Unlike in special relativity, where the Minkowski 
metric is a `background structure' given a priori, in general relativity
the metric is treated as a field which not only affects, but also is
affected by, the other fields present.  In other words, the geometry of
spacetime becomes a local degree of freedom of the theory.
Quantitatively, the interaction of the metric and other fields is
described by Einstein's equation

                 G_{ab} = 8 &pi; G T_{ab}

where the Einstein tensor G_{ab} depends on the curvature of the
metric, while the stress-energy tensor T_{ab} describes the flow
of energy and momentum due to all the other fields.  The role of the
constant G is thus simply to quantify how much the geometry of
spacetime is affected by other fields.   Over the years, people have
realized that the great lesson of general relativity is that a good
theory of physics should contain no geometrical structures that affect
local degrees of freedom while remaining unaffected by them.  Instead,
all geometrical structures - and in particular the causal structure -
should themselves be local degrees of freedom.  For short, one says
that the theory should be background-free.
 
The struggle to free ourselves from background structures began long
before Einstein developed general relativity, and is still not complete.
The conflict between Ptolemaic and Copernican cosmologies, the dispute
between Newton and Leibniz concerning absolute and relative motion, and
the modern arguments concerning the `problem of time' in quantum gravity
- all are but chapters in the story of this struggle.  I do not have
room to sketch this story here, nor even to make more precise the
all-important notion of `geometrical structure'.  I can only point the
reader towards the literature, starting perhaps with the books by
Barbour and Earman, various papers by Rovelli, and the many references 
therein.  

Finally, what of &#295;?  In quantum theory, this appears most 
prominently in the commutation relation between the momentum p and
position q of a particle: 

                        pq - qp = -i &#295;, 

together with similar commutation relations involving other pairs of
measurable quantities.   Because our ability to measure two quantities
simultaneously with complete precision is limited by their failure to
commute, &#295; quantifies our inability to simultaneously know
everything one might choose to know about the world.  But there is far
more to quantum theory than the uncertainty principle.  In practice,
&#295; comes along with the whole formalism of complex Hilbert spaces
and linear operators.  

There is a widespread sense that the principles behind quantum theory
are poorly understood compared to those of general relativity.  This has
led to many discussions about interpretational issues.  However, I do
not think that quantum theory will lose its mystery through such
discussions.   I believe the real challenge is to better understand why
the mathematical formalism of quantum theory is precisely what it is. 
Research in quantum logic has done a wonderful job of understanding the
field of candidates from which the particular formalism we use has been
chosen.  But what is so special about this particular choice?  Why, for
example, do we use complex Hilbert spaces rather than real or
quaternionic ones?  Is this decision made solely to fit the experimental
data, or is there a deeper reason?  Since questions like this do not yet
have clear answers, I shall summarize the physical insight behind &#295;
by saying simply that a good theory of the physical universe should be a
<em>quantum theory</em> - leaving open the possibility of eventually saying
something more illuminating.

Having attempted to extract the ideas lying behind the constants c, G,
and &#295;, we are in a better position to understand the task of
constructing a theory of quantum gravity.  General relativity
acknowledges the importance of c and G but idealizes reality by treating
&#295; as negligibly small.  From our discussion above, we see that this
is because general relativity is a background-free classical theory with
local degrees of freedom propagating causally. On the other hand,
quantum field theory as normally practiced acknowledges c and &#295; but
treats G as negligible, because it is a background-dependent quantum
theory with local degrees of freedom propagating causally.

The most conservative approach to quantum gravity is to seek a theory
that combines the best features of general relativity and quantum field
theory.    To do this, we must try to find a *background-free
quantum theory with local degrees of freedom propagating causally*.
While this approach may not succeed, it is definitely worth pursuing.
Given the lack of experimental evidence that would point us towards 
fundamentally new principles, we should do our best to understand
the full implications of the principles we already have!

From my description of the goal one can perhaps see some of the
difficulties.  Since quantum gravity should be background-free, the
geometrical structures defining the causal structure of spacetime should
themselves be local degrees of freedom propagating causally.  This much
is already true in general relativity.  But because quantum gravity
should be a quantum theory, these degrees of freedom should be treated
quantum-mechanically.  So at the very least, we should develop a quantum
theory of some sort of geometrical structure that can define a causal
structure on spacetime.   
\end{quote}
    


Then I talk about topological quantum field theories, which are
background-free quantum theories \emph{without} local degrees of freedom, and
what we have learned from them.  Basically what we've learned is that
there's a deep analogy between the mathematics of spacetime
(e.g. differential topology) and the mathematics of quantum theory.
This is interesting because in background-free quantum theories we
expect that spacetime, instead of serving as a "stage" on which events
play out, actually becomes part of the play of events itself - and
must itself be described using quantum theory.  So it's very interesting
to try to connect the concepts of spacetime and quantum theory.  The 
analogy goes like this:
\begin{verbatim}

     DIFFERENTIAL TOPOLOGY                  QUANTUM THEORY            

  (n-1)-dimensional manifold                Hilbert space             
          (space)                             (states)                  

cobordism between (n-1)-dimensional           operator    
          manifolds                           (process)                 
         (spacetime)                         

  composition of cobordisms            composition of operators  

      identity cobordism                  identity operator         

\end{verbatim}
    
And if you know a little category theory, you'll see what we have
here are two categories: the category of cobordisms and the category
of Hilbert spaces.  A topological quantum field theory is a functor
from the first to the second....

Okay, now for some other papers:

2) Geraldine Brady and Todd H. Trimble. A string diagram calculus for
predicate logic, and C. S. Peirce's system Beta, available at
<A HREF = "http://people.cs.uchicago.edu/~brady">http://people.cs.uchicago.edu/~
brady</A>
Geraldine Brady and Todd H. Trimble, A categorical interpretation
of Peirce's propositional logic Alpha, Jour. Pure and
Appl. Alg. 149 (2000), 213-239.
Geraldine Brady and Todd H. Trimble, The topology of relational
calculus.



Charles Peirce is a famously underappreciated American philosopher who
worked in the late 1800s.  Among other things, like being the father of
pragmatism, he is also one of the fathers of higher-dimensional algebra.
As you surely know if you've read me often enough, part of the point of
higher-dimensional algebra is to break out of "linear thinking".  By
"linear thinking" I mean the tendency to do mathematics in ways that are
easily expressed in terms of 1-dimensional strings of symbols.  In his
work on logic, Peirce burst free into higher dimensions.  He developed a
way of reasoning using diagrams that he called "existential graphs".  
Unfortunately this work by Peirce was never published!  One reason is 
that existential graphs were difficult and expensive to print.  As a
result, his ideas languished in obscurity.

By now it's clear that higher-dimensional algebra is useful in physics:
examples include Feynman diagrams and the spin networks of Penrose.  
The theory of n-categories is beginning to provide a systematic language
for all these techniques.  So it's worth re-evaluating Peirce's work and
seeing how it fits into the picture.  And this is what the papers by
Brady and Trimble do!

3) J. Scott Carter, Louis H. Kauffman, and Masahico Saito, Structures
and diagrammatics of four dimensional topological lattice field theories,
to appear in Adv. Math., preprint available as 
<A HREF = "http://xxx.lanl.gov/abs/math.GT/9806023">math.GT/9806023</A>.

We can get 3-dimensional topological quantum field theories from certain
Hopf algebras.  As I described in "<A HREF = "week38.html">week38</A>", Crane and Frenkel made the 
suggestion that by categorifying this construction we should get 
4-dimensional TQFTs from certain Hopf categories.  This paper makes the
suggestion precise in a certain class of examples!  Basically these are
categorified versions of the Dijkgraaf-Witten theory.

4) J. Scott Carter, Daniel Jelsovsky, Selichi Kamada, Laurel Langford
and Masahico Saito, Quandle cohomology and state-sum invariants of
knotted curves and surfaces, preprint available as 
<A HREF = "http://xxx.lanl.gov/abs/math.GT/9903135">math.GT/9903135</A>.

Yet another attack on higher dimensions!  This one gets invariants of
2-links - surfaces embedded in R^4 - from the cohomology groups of
"quandles".  I don't really understand how this fits into the overall
scheme of higher-dimensional algebra yet.  They show their invariant
distinguishes between the 2-twist spun trefoil (a certain sphere knotted
in R^4) and the same sphere with the reversed orientation.

5) Tom Leinster, Structures in higher-dimensional category theory,
preprint available at <A HREF = "http://www.dpmms.cam.ac.uk/~leinster">http://www.dpmms.cam.ac.uk/~leinster</A>

This is a nice tour of ideas in higher-dimensional algebra.  Right now
one big problem with the subject is that there are lots of approaches
and not a clear enough picture of how they fit together.  Leinster's
paper is an attempt to start seeing how things fit together.  

6) Claudio Hermida, Higher-dimensional multicategories, slides of
a lecture given in 1997, available at <A HREF = "http://www.math.mcgill.ca/~hermida">http://www.math.mcgill.ca/~hermida</A>

This talk presents some of the work by Makkai, Power and Hermida on
their definition of n-categories.   For more on their work see "<A HREF = "week107.html">week107</A>".

7) Carlos Simpson, On the Breen-Baez-Dolan stabilization hypothesis for
Tamsamani's weak n-categories, preprint available as 
<A HREF = "http://xxx.lanl.gov/abs/math.CT/9810058">math.CT/9810058</A>.

For quite a while now James Dolan and I have been talking about something
we call the "stabilization hypothesis".  I gave an explanation of this in
"<A HREF = "week121.html">week121</A>", but briefly, it says that the nth column of the following chart
(which extends to infinity in both directions) stabilizes after 2n+2 rows:

$$
                   k-tuply monoidal n-categories 

              n = 0           n = 1             n = 2

k = 0         sets          categories         2-categories
     

k = 1        monoids         monoidal           monoidal
                            categories        2-categories

k = 2       commutative      braided            braided
             monoids         monoidal           monoidal
                            categories        2-categories 


k = 3         " "           symmetric            weakly
                             monoidal          involutory
                            categories          monoidal
                                              2-categories

k = 4         " "             " "               strongly 
                                               involutory
                                                monoidal
                                              2-categories

k = 5         " "             " "                "  "


$$
    
Carlos Simpson has now made this hypothesis precise and proved it using
Tamsamani's definition of n-categories!  And he did it using the same
techniques that Graeme Segal used to study k-fold loop spaces... 
exploiting the relation between n-categories and homotopy theory.  This 
makes me really happy.

8) Mark Hovey, Model Categories, American Mathematical Society Mathematical
Surveys and Monographs, vol 63., Providence, Rhode Island, 1999.  Preprint
available as <A HREF = "http://www.math.uiuc.edu/K-theory/0278/index.html">
http://www.math.uiuc.edu/K-theory/0278/index.html</A>

Speaking of that kind of thing, the technique of model categories is 
really important for homotopy theory and n-categories, and this book
is a really great place to learn about it.  

9) Frank Quinn, Group-categories, preprint available as 
<A HREF = "http://xxx.lanl.gov/abs/math.GT/9811047">math.GT/9811047</A>.

This one is about the algebra behind certain topological quantum field 
theories.  I'll just quote the abstract:

\begin{quote}
     A group-category is an additively semisimple category with a
     monoidal product structure in which the simple objects are
     invertible. For example in the category of representations of a
     group, 1-dimensional representations are the invertible simple
     objects. This paper gives a detailed exploration of &quot;topological
     quantum field theories&quot; for group-categories, in hopes of finding
     clues to a better understanding of the general situation.
     Group-categories are classified in several ways extending results
     of Frohlich and Kerler. Topological field theories based on homology
     and cohomology are constructed, and these are shown to include
     theories obtained from group-categories by Reshetikhin-Turaev
     constructions. Braided-commutative categories most naturally give
     theories on 4-manifold thickenings of 2-complexes; the usual
     3-manifold theories are obtained from these by normalizing them
     (using results of Kirby) to depend mostly on the boundary of the
     thickening. This is worked out for group-categories, and in
     particular we determine when the normalization is possible and when
     it is not.
\end{quote}
    
10) Sjoerd Crans, A tensor product for Gray-categories, 
Theory and Applications of Categories, Vol. 5, 1999, No. 2, pp 12-69, 
available at <A HREF = "http://www.tac.mta.ca/tac/volumes/1999/n2/abstract.html">http://www.tac.mta.ca/tac/volumes/1999/n2/abstract.html</A>
 
A Gray-category is what some people call a semistrict 3-category: not as
general as a weak 3-category, but general enough.  Technically,
Gray-categories are defined as categories enriched over the category of
2-categories equipped with a tensor product invented by John Gray.  To
define semistrict 4-categories one might similarly try to equip
Gray-categories with a suitable tensor product.  And this is what Crans
is studying.  Let me quote the abstract:

\begin{quote}
     In this paper I extend Gray's tensor product of 2-categories to a
     new tensor product of Gray-categories. I give a description in
     terms of generators and relations, one of the relations being an
     ``interchange'' relation, and a description similar to Gray's
     description of his tensor product of 2-categories. I show that this
     tensor product of Gray-categories satisfies a universal property
     with respect to quasi-functors of two variables, which are defined
     in terms of lax-natural transformations between
     Gray-categories. The main result is that this tensor product is
     part of a monoidal structure on Gray-Cat, the proof requiring
     interchange in an essential way.  However, this does not give a
     monoidal {(bi)closed} structure, precisely because of interchange
     And although I define composition of lax-natural transformations,
     this composite need not be a lax-natural transformation again,
     making Gray-Cat only a partial Gray-Cat-cateegory.
\end{quote}
    


<p> <hr>

% </A>
% </A>
% </A>


% parser failed at source line 548
