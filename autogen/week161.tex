
% </A>
% </A>
% </A>
\week{December 10, 2000 }


I'm in the middle of reading this book, so I don't know how it ends
yet, but it's good:

1) Dava Sobel, Galileo's Daughter, Penguin Books, London, 2000.

Galileo had two daughters and a son with a beautiful woman whom never
married - Marina Gamba of Venice.  The son was a wastrel, and the
younger daughter was very shy, but the older daughter, Virginia, loved
Galileo very much and wrote him many letters.  Of these, 124 have been
preserved, which serve as the basis of this book.  At the age of 13 she
was sent to a convent, and she later became a nun.  She took on the
name Suor Maria Celeste - Sister Mary of the Heavens.   Unfortunately,
all of Galileo's letters to her were destroyed by her abbess after
his trial by the Inquisition.  Thus, what was really a dialog has come
down to us as a monolog.  Nonetheless it is fascinating, especially since
Sobel elegantly fills in many of the holes using other sources.  

Since I haven't read much about Galileo, I didn't know that this man,
often considered the father of experimental physics and telescope-aided
astronomy, was officially the "Chief Mathematician of the University of
Pisa".  Now I can add him to my list of mathematicians who have done good 
physics.  

Two later figures standing on the border of math and physics are Kelvin
and Stokes:

2) David B. Wilson, Kelvin and Stokes: A Comparative Study in Victorian
Physics, Adam Hilger, Bristol, 1987.

One thing I like about this book is the debunking of the popular image
of quantum mechanics and relativity as "bolts from the blue"
shattering the complacent serenity of 19th-century physics.  In physics,
the 19th century was also a century of drastic change!  To quote:

\begin{quote}
     Science in Victorian Britain underwent revolutionary conceptual
     and institutional changes.  Together, thermodynamics and the 
     electromagnetic theory of light, for example, transformed a bundle
     of only partially linked, largely experimental sciences into a 
     coherent, unified, mathematical physics of energy and ether.  In
     the 1890s one could contemplate reducing the phenomena of matter,
     electricity, magnetism, heat and light to an underlying reality of
     potential and kinetic energy in an all-pervading ether.  The pursuit
     of scientific research, largely avocational early in the century, 
     was a full-fledged profession by the century's end.  Science became
     important to university curricula, and the universities expanded
     their science faculties.  Institutions like the British Association
     for the Advancement of Science, founded in 1831, and Royal Society
     of London, reformed at mid-century, provided organizational support
     for a growing community of scientists.  And that community of late-
     Victorian scientists resided in a community which, on balance, was
     much more scientific and less religious than it had been only two
     or three generations earlier.  In sum late-Victorian society 
     endorsed the imporance of scientific knowledge and research, and
     late-Victorian physics affirmed the primary significance of the 
     ideal of unification and the language of mathematics.  In these
     respects, there was an essential <em>similarity</em> between late-Victorian
     Britain and both the &quot;big science&quot; and the modern physics of the
     twentieth century.  The metamorphosis that created this state of
     affairs was the context of the the careers of G G Stokes and William
     Thomson, Lord Kelvin.
\end{quote}
    
Marching forwards into the 20th century, we find Einstein as another
physicist with a special tie to mathematics.  Certainly he was no
mathematician, but his search for a theory of general relativity was
a curious combination of philosophical and mathematical reasoning, 
with very little support from experiment.  How did he really figure
it out?  This book is a good place to learn the details:

3) Don Howard and John Stachel eds., Einstein and the History of General
Relativity, Birkhauser, Boston, 1989.  

There are a number of essays exploring the interesting period between
1912, when Einstein recognized that gravity was caused by spacetime
curvature, and 1915, when he found his field equations and used them to
compute the anomalous precession of the perihelion of Mercury.  Why did
it take him so long?  According to Einstein himself, "The main reason
lies in the fact that it is not easy to free oneself from the idea that
co-ordinates must have an immediate metrical significance". 

Indeed, in 1913 he noticed that generally covariant field equations
could not uniquely determine the gravitational field generated by a
fixed mass distribution.  The reason - apart from the existence of
gravitational waves, which he was not concerned with here - is that one
can take any solution, apply an arbitrary change of coordinates, and get
a new solution.  This seemed to suggest a conflict between general
covariance and the principle that every effect should have a sufficient
cause.  

Before he solved it, this conceptual problem aggravated the technical
problem of getting the right field equations: there aren't that many
good candidates for these equations if one demands general covariance,
but during the period when he distrusted this principle, Einstein and
his collaborator Grossman put a lot of work into other candidates.  The
main one they tried gave Mercury an anomalous precession of 18" per
century instead of the correct value of 45" per century.  Einstein only
discarded this theory in November, 1915.  

On November 11th he tried a theory where the Ricci tensor was proportional 
to the stress-energy tensory.  On November 25th he tried a better one,
where what we now call the Einstein tensor is proportional to the
stress-energy tensor.  He quickly used this to derive the correct
precession for Mercury.  And so general relativity was born!  In January
1916 he explained in letters to Ehrenfest and Besso how he had
reconciled general covariance with causality: two solutions of the
field equations that differ only by a change of coordinates should be
regarded as physically the same.

Now I'd like to switch to something else: a couple of emails I got.   
A while back I wrote up a webpage about the end of the universe:

4) John Baez, The end of the universe, <a href = "http://math.ucr.edu/home/baez/end.html">http://math.ucr.edu/home/baez/end.html</a>

I got a lot of the numbers out of a book I bet you've already read:

5) John D. Barrow and Frank J. Tipler, The Cosmological Anthropic 
Principle, Oxford U. Press, Oxford, 1988.

What - you haven't read it?  Yikes!  Hurry up and give it to a friend
for Christmas - and then make them lend it to you.  Regardless of what
you think about the anthropic principle, you're bound to enjoy the cool
facts this book is stuffed with!  Anyway, I got an email from Barrow
saying that he's coming out with a new book.  Like the previous one,
it's sure to be full of interesting things.  You can tell from the title:

6) John D. Barrow, The Book of Nothing, to be published.

My other email was from Bert Schroer, an expert on the C*-algebraic 
approach to quantum field theory.  He has written a paper about the
"AdS-CFT correspondence" which is bound to stir up controversy:

7) Bert Schroer, Facts and fictions about Anti de Sitter spacetimes
with local quantum matter, available as <A HREF = "http://xxx.lanl.gov/abs/hep-th/9911100">hep-th/9911100</A>.

Let me just quote the beginning:

\begin{quote}
      There has been hardly any problem in particle physics which has
      has attracted as much attention as the problem if and in what way
      quantum matter in the Anti de Sitter spacetime and the one dimension
      lower conformal field theories are related and whether this could
      possibly contain clues about the meaning of quantum gravity.

      In more specific quantum physical terms the question is about
      a conjectured (and meanwhile in large parts generically and 
      rigorously understood) correspondence between two quantum field 
      theories in different spacetime dimensions; the lower-dimensional
      conformal one being the &quot;holographic image&quot; or projection 
      of the AdS theory.
     
      The entire globalized community of string physicists has placed
      this problem in the centre of their interest and treated it as
      the dominating problem of theoretical particle physics with the
      result that there have been approximately around 100-150 papers
      per month during a good part of 1999.  Even if one takes into 
      account the increase in the number of particle physicists during
      the last decades and compares it with the relative number of
      participants in previous fashionable topics (the S-matrix bootstrap,
      Regge theory, the SU(6) - U(12) symmetric and the so-called
      relativistic quark theory, to name some of them) which also led
      to press-conferences, interviews and articles in the media (but
      not to awards and prizes), it remains still an impressive sociological
      phenomenon.  Just imagine yourslef working on this kind of problem
      and getting up every morning turning nervously to the hep-th server 
      in order to check that nobody has beaten you to similar results.  
      What a life in an area which used to required a contemplative 
      critical attitude!
  
      This is clearly a remarkable situation in the exact sciences
      which warrants an explanation.  This is particularly evident to
      somebody old enough to have experienced theoretical particle
      physics at times of great conceptual and calculational achievements,
      e.g. the derivation of scattering theory and dispersion theory
      from local fields, achievements with which the name of Harry
      Lehmann (to whose memory this article is dedicated) is inexorably
      linked.  In those times the acceptance of a theoretical proposal
      in particle physics was primarily coupled to its experimental
      verifiability and its conceptual standing within physics and not
      yet to the beauty of its differential-geometric content.  There 
      were also fashions, but if they did not deliver what they promised
      they were allowed to die. 

      In the opinion of Roger Penrose, the new totalitarian attitude in
      particle physics is the result of the rapid and propagandistic 
      communication through the new electronic media which favors 
      speedy calculations with no or only insufficient superficial
      physical interpretation to more contemplative and not instantly
      profitable conceptual investments.  He cites supersymmetry and
      inflation cosmology as examples of theories which achieved a 
      kind of monopolistic dominance despite a total lack of experimental
      fact (or even convincing theoretical arguments).  It seems to me
      that this phenomenon receives an even stronger illustration from
      string theory, and I am not the only one who thinks this way
      [here he cites a paper by I. Todorov].

      Leaving the final explanation of this phenomenon to historians
      or sociologists of the exact sciences, I will limit myself to
      analyzing the particle physics content of the so-called Anti
      de Sitter - conformal QFT correspondence from the conservative
      point of view of a quantum field theorist with a 30 year 
      professional experience who, although having no active ambitions
      outside QFT, still nourishes a certain curiosity about present
      activities in particle physics, e.g. string theory or the use
      of noncommutative geometry.  Some of the consistency calculations
      one finds there are really surprising and if one could consider
      them in the critical Bohr-Sommerfeld spirit as ciphers encoding
      possibly new principles in fundamental physics and not as a theory
      (let alone a theory of everything), these observations may have
      an enigmatic use.  But for this to be successful one would have
      to make a much more serious attempt at confronting the new
      mathematical consistency observations with local quantum physics
      on a more conceptual level beyond the standard formalism.  Only 
      in this way can one be sure to confront something new and not
      just a new formalism which implements the same principles in a
      different way.  

      The AdS model of a curved spacetime has a long history as a
      theoretical laboratory of what can happen with particle physics
      in a universe which is the extreme opposite of globally hyperbolic
      in that it possesses a self-closing time, whereas the proper 
      de Sitter spacetime was once considered among the more realistic
      models of the universe.  The recent surge of interest about AdS
      came from string theory and is different in motivation and more 
      related to the hope (or dream) to attribute a meaning to &quot;Quantum
      Gravity&quot; from a string theory viewpoint.

      Fortunately for the curious outsider (otherwise I would have to
      quit right here), this motivation has no bearing on the conceptual
      and mathematical problems posed by the would-be AdS-conformal QFT
      correspondence, which turned out to be one of those properties 
      discovered in the setting of string theory which allow an interesting
      and rigorous formulation in QFT which confirms some but not all of
      the conjectured properties.  The rigorous treatment however requires
      a reformulation of (conformal) QFT.  The standard formalism based
      on pointlike &quot;field coordinatizations&quot; which underlies the    
      Lagrangian
      (and Wightman) formulations does not provide a natural setting for 
      the study of isomorphisms between models in different spacetime 
      dimensions, even though the underlying principles are the same.  
      One would have to introduce too many additional concepts and 
      auxiliary tricks into the standard framework.  The important 
      aspects in this isomorphism are related to space and time-like 
      (Einstein, Huyghens) causality, localization of corresponding 
      objects and problems of degree of freedom counting.  All these 
      issues belong to real-time physics and in most cases their meaning
      in terms of Euclidean continuation (statistical mechanics) remains
      obscure; but this of course does not make them less physical.

      This note is organized as follows.  In the next section I elaborate
      on the kinematical aspects of the AdS<sub>d+1</sub>-CQFT<sub>d</sub> 
      situation as a 
      collateral of the old (1974/75) compactification formalism for the 
     &quot;conformalization&quot; of the d-dimensional Minkowski spacetime.    
      For
      this reason the seemingly more demanding problem of studying QFT
      directly in AdS within a curved spacetime formalism can be bypassed.
      The natural question whose answer would have led directly from 
      CQFT<sub>4</sub> to AdS<sub>5</sub> in the particle physics setting (without string
      theory as a midwife) is: Does there exist a quantum field theory
      which has the same SO(4,2) symmetry and just reprocesses the 
      CQFT<sub>4</sub>
      matter content in such a way that the &quot;conformal hamiltonian&quot;
      (the timelike generator of rotations of conformally compactified
      Minkowski space) becomes the true hamiltonian?  The theory 
      exists and is an AdS theory with a specific local matter content 
      computable from the CQFT matter content.  The answer is unique,
      but as a result of the different dimensionality one cannot describe
      this unique relation between matter contents in terms of pointlike
      fields.  This will be treated in Section 3, where we will also
      compare the content of Rehren's isomorphism with Maldacena, Witten
      et al conjectures and notice some subtle but potentially serious
      differences.  Whoever is aware of the fact that subtle differences
      have often been the enigmatic motor of progress in good physics
      times will not dismiss such observations.

      The last section presents some results of algebraic QFT on degrees
      of freedom counting and holography.  Closely connected is the idea
      of &quot;chiral scanning&quot;, i.e. the encoding of the full content of
      a higher dimensional (massive) QFT into a finite number of copies
      of one chiral theory in a carefully selected position within a 
      common Hilbert space.  In this case the price one has to pay for 
      this more generic holography (light-front holography) is that some
      of the geometrically acting spacetime symmetry transformations 
      become &quot;fuzzy&quot; in the holographic projection and some of the 
      geometrically acting symmetries on the holographic image are not 
      represented by diffeomorphisms if pulled back to the original QFT.
\end{quote}
    
As you can see, there is some interesting mathematical physics in here, 
as well as some serious criticism of how particle physics is done these 
days.  

By the way, Schroer has recently written a paper about the braid group 
and quantum field theory.  Everyone knows how the braid group shows
up in 3d quantum field theory, but this is about \emph{4d} 
quantum field theory:

8) Bert Schroer, Braided structure in 4-dimensional conformal quantum
field theory, available as <A HREF =
"http://xxx.lanl.gov/abs/hep-th/0012021">hep-th/0012021</A>.

<p> <hr>

% </A>
% </A>
% </A>


% parser failed at source line 387
