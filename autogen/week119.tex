
% </A>
% </A>
% </A>
\week{April 13, 1998 }

I've been slacking off on This Week's Finds lately because I was
busy getting stuff done at Riverside so that I could visit the 
Center for Gravitational Physics and Geometry here at Penn State
with a fairly clean slate.  Indeed, sometimes my whole life seems
like an endless series of distractions designed to prevent me from
writing This Week's Finds.  However, now I'm here and ready to have
some fun....

Recently I've been trying to learn about grand unified theories, or
"GUTs".  These were popular in the late 1970s and early 1980s, when
the Standard Model of particle interactions had fully come into its
own and people were looking around for a better theory that would
unify all the forces and particles present in that model - in short,
everything except gravity.  

The Standard Model works well but it's fairly baroque, so it's natural
to hope for some more elegant theory underlying it.  Remember how it
goes: 

\par\noindent\rule{\textwidth}{0.4pt}
\begin{verbatim}
                            GAUGE BOSONS
                           
    ELECTROMAGNETIC FORCE          WEAK FORCE         STRONG FORCE
        
         photon                     W+                  8 gluons 
                                    W-
                                    Z  
\end{verbatim}
    
\par\noindent\rule{\textwidth}{0.4pt}
\begin{verbatim}
                            FERMIONS

        LEPTONS                                     QUARKS

electron        electron neutrino         down quark        up quark
muon            muon neutrino             strange quark     charm quark
tauon           tauon neutrino            bottom quark      top quark
\end{verbatim}
    
\par\noindent\rule{\textwidth}{0.4pt}
\begin{verbatim}
                        HIGGS BOSON (not yet seen)

\end{verbatim}
    
\par\noindent\rule{\textwidth}{0.4pt}

The strong, electromagnetic and weak forces are all described by
Yang-Mills fields, with the gauge group SU(3) x SU(2) x U(1).  In what
follows I'll assume you know the rudiments of gauge theory, or at
least that you can fake it.

SU(3) is the gauge group of the strong force, and its 8 generators
correspond to the gluons.  SU(2) x U(1) is the gauge group of the
electroweak force, which unifies electromagnetism and the weak force.
It's \emph{not} true that the generators of SU(2) corresponds to the W+, W-
and Z while the generator of U(1) corresponds to the photon.  Instead,
the photon corresponds to the generator of a sneakier U(1) subgroup
sitting slantwise inside SU(2) x U(1); the basic formula to remember
here is:

Q = I_{3} + Y/2

where Q is ordinary electric charge, I_{3} is the 3rd component of
"weak isospin", i.e. the generator of SU(2) corresponding to the
matrix

\begin{verbatim}
(1/2   0)
(0  -1/2)
\end{verbatim}
    
and Y, "hypercharge", is the generator of the U(1) factor.  The role
of the Higgs particle is to spontaneously break the SU(2) x U(1)
symmetry, and also to give all the massive particles their mass.
However, I don't want to talk about that here; I want to focus on the
fermions and how they form representations of the gauge group SU(3) x
SU(2) x U(1), because I want to talk about how grand unified theories
attempt to simplify this picture - at the expense of postulating more
Higgs bosons.

The fermions come in 3 generations, as indicated in the chart above.
I want to explain how the fermions in a given generation are grouped
into irreducible representations of SU(3) x SU(2) x U(1).  All the
generations work the same way, so I'll just talk about the first
generation.  Also, every fermion has a corresponding antiparticle, but
this just transforms according to the dual representation, so I will
ignore the antiparticles here.

Before I tell you how it works, I should remind you that all the
fermions are, in addition to being representations of SU(3) x SU(2) x
U(1), also spin-1/2 particles.  The massive fermions - the quarks and
the electron, muon and tauon - are Dirac spinors, meaning that they
can spin either way along any axis.  The massless fermions - the
neutrinos - are Weyl spinors, meaning that they always spin
counterclockwise along their axis of motion.  This makes sense
because, being massless, they move at the speed of light, so everyone
can agree on their axis of motion!  So the massive fermions have two
helicity states, which we'll refer to as "left-handed" and
"right-handed", while the neutrinos only come in a "left-handed" form.

(Here I am discussing the Standard Model in its classic form.  I'm
ignoring any modifications needed to deal with a possible nonzero
neutrino mass.  For more on Standard Model, neutrino mass and
different kinds of spinors, see "<A HREF = "week93.html">week93</A>".)

Okay.  The Standard Model lumps the left-handed neutrino and the
left-handed electron into a single irreducible representation of 
SU(3) x SU(2) x U(1):

\begin{verbatim}
(&nu;<sub>L</sub>, e<sub>L</sub>)                                 (1,2,-1)
\end{verbatim}
    
This 2-dimensional representation is called (1,2,-1), meaning
that it's the tensor product of the 1-dimensional trivial rep
of SU(3), the 2-dimensional fundamental rep of SU(2), and the
1-dimensional rep of U(1) with hypercharge -1.  

Similarly, the left-handed up and down quarks fit together as:

\begin{verbatim}
(u<sub>L</sub>, u<sub>L</sub>, u<sub>L</sub>, d<sub>L</sub>, d<sub>L</sub>, d<sub>L</sub>)              (3,2,1/3)
\end{verbatim}
    
Here I'm writing both quarks 3 times since they also come in 3 color
states.  In other words, this 6-dimensional representation is the
tensor product of the 3-dimensional fundamental rep of SU(3), the
2-dimensional fundamental rep of SU(2), and the 1-dimensional rep of
U(1) with hypercharge 1/3.  That's why we call this rep (3,2,1/3).  

(If you are familiar with the irreducible representations of U(1) you
will know that they are usually parametrized by integers.  Here we are
using integers divided by 3.  The reason is that people defined the
charge of the electron to be -1 before quarks were discovered, at
which point it turned out that the smallest unit of charge was 1/3 as
big as had been previously believed.)

The right-handed electron stands alone in a 1-dimensional rep, since
there is no right-handed neutrino:

\begin{verbatim}
e<sub>R</sub>                                         (1,1,-2)
\end{verbatim}
    
Similarly, the right-handed up quark stands alone in a 3-dimensional
rep, as does the right-handed down quark:

\begin{verbatim}
(u<sub>R</sub>, u<sub>R</sub>, u<sub>R</sub>)                             (3,1,4/3) 

(d<sub>R</sub>, d<sub>R</sub>, d<sub>R</sub>)                             (3,1,-2/3)
\end{verbatim}
    
That's it.  If you want to study this stuff, try using the formula

Q = I_{3} + Y/2

to figure out the charges of all these particles.  For example, since
the right-handed electron transforms in the trivial rep of SU(2), it
has I_{3} = 0, and if you look up there you'll see that it has Y = -2.
This means that its electric charge is Q = -1, as we already knew.  

Anyway, we obviously have a bit of a mess on our hands!  The Standard
Model is full of tantalizing patterns, but annoyingly complicated.
The idea of grand unified theories is to find a pattern lurking in all
this data by fitting the group SU(3) x SU(2) x U(1) into a larger
group.  The smallest-dimensional "simple" Lie group that works is
SU(5).  Here "simple" is a technical term that eliminates, for
example, groups that are products of other groups - these aren't very
"unified".  Georgi and Glashow came up with their "minimal" SU(5)
grand unified theory in 1975.  The idea is to stick SU(3) x SU(2) into
SU(5) in the obvious diagonal way, leaving just enough room to cram in
the U(1) if you are clever.

Now if you add up the dimensions of all the representations above you
get 2 + 6 + 1 + 3 + 3 = 15.  This means we need to find a
15-dimensional representation of SU(5) to fit all these particles.
There are various choices, but only one that really works when you
take all the physics into account.  For a nice simple account of the
detective work needed to figure this out, see:

1) Edward Witten, Grand unification with and without supersymmetry,
Introduction to supersymmetry in particle and nuclear physics, edited by
O. Castanos, A. Frank, L. Urrutia, Plenum Press, 1984.

I'll just give the answer.  First we take the 5-dimensional fundamental
representation of SU(5) and pack fermions in as follows:

\begin{verbatim}
(d<sub>R</sub>, d<sub>R</sub>, d<sub>R</sub>, e+<sub>R</sub>, nubar<sub>R</sub>)             5 = (3,1,-2/3) + (1,2,-1)

\end{verbatim}
    
Here e+_{R} is the right-handed positron and nubar_{R} is the right-handed
antineutrino - curiously, we need to pack some antiparticles in with
particles to get things to work out right.  Note that the first 3
particles in the above list, the 3 states of the right-handed down
quark, transform according to the fundamental rep of SU(3) and the
trivial rep of SU(2), while the remaining two transform according to
the trivial rep of SU(3) and the fundamental rep of SU(2).  That's how
it has to be, given how we stuffed SU(3) x SU(2) into SU(5).  

Note also that the charges of the 5 particles on this list add up to
zero.  That's also how it has to be, since the generators of SU(5) are
traceless.  Note that the down quark must have charge -1/3 for this to
work!  In a sense, the SU(5) model says that quarks \emph{must} have
charges in units of 1/3, because they come in 3 different colors!
This is pretty cool.

Then we take the 10-dimensional representation of SU(5) given by
the 2nd exterior power of the fundamental representation - i.e.,
antisymmetric 5x5 matrices - and pack the rest of the fermions in
like this:

\begin{verbatim}
 (     0      ubar<sub>L</sub>  ubar<sub>L</sub>    u<sub>L</sub>    d<sub>L</sub>   )       10 = (3,2,1/3) + 
 (  -ubar<sub>L</sub>     0     ubar<sub>L</sub>    u<sub>L</sub>    d<sub>L</sub>   )            (1,1,2)   +
 (  -ubar<sub>L</sub>  -ubar<sub>L</sub>    0       u<sub>L</sub>    d<sub>L</sub>   )            (3,1,-4/3)
 (   -u<sub>L</sub>     -u<sub>L</sub>    -u<sub>L</sub>       0     e+<sub>L</sub>  )
 (   -d<sub>L</sub>     -u<sub>L</sub>    -d<sub>L</sub>     -e+<sub>L</sub>    0    )
                     
\end{verbatim}
    
Here the u-bar is the antiparticle of the up quark - again we've 
needed to use some antiparticles.  However, you can easily check
that these two representations of SU(5) together with their duals
account for all the fermions and their antiparticles.

The SU(5) theory has lots of nice features.  As I already noted, it
explains why the up and down quarks have charges 2/3 and -1/3,
respectively.  It also gives a pretty good prediction of something
called the Weinberg angle, which is related to the ratio of the masses
of the W and Z bosons.  It also makes testable new predictions!  Most
notably, since it allows quarks to turn into leptons, it predicts that
protons can decay - with a halflife of somewhere around 10^{29} or
10^{30} years.  So people set off to look for proton decay....

However, even when the SU(5) model was first proposed, it was regarded
as slightly inelegant, because it didn't unify all the fermions of a
given generation in a \emph{single} irreducible representation (together
with its dual, for antiparticles).  This is one reason why people
began exploring still larger gauge groups.  In 1975 Georgi, and
independently Fritzsch and Minkowski, proposed a model with gauge
group SO(10).  You can stuff SU(5) into SO(10) as a subgroup in such a
way that the 5- and 10-dimensional representations of SU(5) listed
above both fit into a single 16-dimensional rep of SO(10), namely the
chiral spinor rep.  Yes, 16, not 15 - that wasn't a typo!  The SO(10)
theory predicts that in addition to the 15 states listed above there
is a 16th, corresponding to a right-handed neutrino!  I'm not sure yet
how the recent experiments indicating a nonzero neutrino mass fit into
this business, but it's interesting.  

Somewhere around this time, people noticed something interesting about
these groups we've been playing with.  They all fit into the "E series"!

I don't have the energy to explain Dynkin diagrams and the ABCDEFG
classification of simple Lie groups here, but luckily I've already
done that, so you can just look at "<A HREF = "week62.html">week62</A>" - "<A HREF = "week65.html">week65</A>" to learn about
that.  The point is, there is an infinite series of simple Lie groups
associated to rotations in real vector spaces - the SO(n) groups, also
called the B and D series.  There is an infinite series of them
associated to rotations in complex vector spaces - the SU(n) groups,
also called the A series.  And there is infintie series of them
associated to rotations in quaternionic vector spaces - the Sp(n)
groups, also called the C series.  And there is a ragged band of 5
exceptions which are related to the octonions, called G2, F4, E6, E7,
and E8.  I'm sort of fascinated by these - see "<A HREF = "week90.html">week90</A>", "<A HREF = "week91.html">week91</A>", and
"<A HREF = "week106.html">week106</A>" for more - so I was extremely delighted to find that the E
series plays a special role in grand unified theories.

Now, people usually only talk about E6, E7, and E8, but one can work
backwards using Dynkin diagrams to define E5, E4, E3, E2, and E1.
Let's do it!  Thanks go to Allan Adler and Robin Chapman for helping
me understand how this works....

E8 is a big fat Lie group whose Dynkin diagram looks like this:

\begin{verbatim}
      o      
      |      
o--o--o--o--o--o---o

\end{verbatim}
    
If we remove the rightmost root, we obtain the Dynkin diagram of
a subgroup called E7:

\begin{verbatim}
      o      
      |      
o--o--o--o--o--o

\end{verbatim}
    
If we again remove the rightmost root, we obtain the Dynkin diagram
of a subgroup of E7, namely E6:

\begin{verbatim}
      o      
      |      
o--o--o--o--o

\end{verbatim}
    
This was popular as a gauge group for grand unified models, and 
the reason why becomes clear if we again remove the rightmost root,
obtaining the Dynkin diagram of a subgroup we could call E5:

\begin{verbatim}
      o      
      |      
o--o--o--o

\end{verbatim}
    
But this is really just good old SO(10), which we were just
discussing!  And if we yet again remove the rightmost root, we 
get the Dynkin diagram of a subgroup we could call E4:

\begin{verbatim}
      o      
      |      
o--o--o

\end{verbatim}
    
This is just SU(5)!  Let's again remove the rightmost root, 
obtaining the Dynkin diagram for E3.  Well, it may not be clear
what counts as the rightmost root, but here's what I want to 
get when I remove it:

\begin{verbatim}
      o      
            
o--o

\end{verbatim}
    
This is just SU(3) x SU(2), sitting inside SU(5) in the way we just
discussed!  So for some mysterious reason, the Standard Model and
grand unified theories seem to be related to the E series!  

We could march on and define E2:

\begin{verbatim}
      o      
            
o

\end{verbatim}
    
which is just SU(2) x SU(2), and E1:

\begin{verbatim}
      o      
            


\end{verbatim}
    
which is just SU(2)... but I'm not sure what's so great about these
groups.  

By the way, you might wonder what's the real reason for removing the
roots in the order I did - apart from getting the answers I wanted to
get - and the answer is, I don't really know!  If anyone knows, please
tell me.  This could be an important clue.

Now, this stuff about grand unified theories and the E series is one
of the reasons why people like string theory, because heterotic string
theory is closely related to E8 (see "<A HREF =
"week95.html">week95</A>").  However, I must now tell you the
\emph{bad} news about grand unified theories.  And it is \emph{very} bad.

The bad news is that those people who went off to detect proton
decay never found it!  It became clear in the mid-1980s that the
proton lifetime was at least 10^{32} years or so, much larger than
what the SU(5) theory most naturally predicts.  Of course, if one is
desperate to save a beautiful theory from an ugly fact, one can resort
to desperate measures.  For example, one can get the SU(5) model to
predict very slow proton decay by making the grand unification mass
scale large.  Unfortunately, then the coupling constants of the strong
and electroweak forces don't match at the grand unification mass
scale.  This became painfully clear as better measurements of the
strong coupling constant came in.

Theoretical particle physics never really recovered from this crushing
blow.  In a sense, particle physics gradually retreated from the goal
of making testable predictions, drifting into the wonderland of pure
mathematics... first supersymmetry, then supergravity, and then
superstrings... ever more elegant theories, but never yet a verified
experimental prediction.  Perhaps we should be doing something
different, something better?  Easy to say, hard to do!  If we see a
superpartner at CERN, a lot of this "superthinking" will be
vindicated - so I guess most particle physicists are crossing their
fingers and praying for this to happen.

The following textbook on grand unified theories is very nice,
especially since it begins with a review of the Standard Model:

2) Graham G. Ross, Grand Unified Theories, Benjamin-Cummings, 1984.

This one is a bit more idiosyncratic, but also good - Mohapatra
is especially interested in theories where CP violation arises via
spontaneous symmetry breaking:

3) Ranindra N. Mohapatra, Unification and Supersymmetry: The Frontiers
of Quark-Lepton Physics, Springer-Verlag, 1992.

I also found the following articles interesting:

4) D. V. Nanopoulos, Tales of the GUT age, in Grand Unified Theories
and Related Topics, proceedings of the 4th Kyoto Summer Institute,
World Scientific, Singapore, 1981.

5) P. Ramond, Grand unification, in Grand Unified Theories and Related
Topics, proceedings of the 4th Kyoto Summer Institute, World
Scientific, Singapore, 1981.

Okay, now for some homotopy theory!  I don't think I'm ever gonna get
to the really cool stuff... in my attempt to explain everything
systematically, I'm getting worn out doing the preliminaries.  Oh well,
on with it... now it's time to start talking about loop spaces!  These
are really important, because they tie everything together.  However, 
it takes a while to deeply understand their importance.  
\par\noindent\rule{\textwidth}{0.4pt}
<STRONG> O.</STRONG> The loop space of a topological space.  Suppose we have a "pointed
space" X, that is, a topological space with a distinguished point
called the "basepoint".  Then we can form the space LX of all "based
loops" in X - loops that start and end at the basepoint.

One reason why LX is so nice is that its homotopy groups are the same
as those of X, but shifted:

\pi _{i}(LX) = \pi _{i+1}(X) 

Another reason LX is nice is that it's almost a topological group,
since one can compose based loops, and every loop has an "inverse".
However, one must be careful here!  Unless one takes special care,
composition will only be associative up to homotopy, and the "inverse"
of a loop will only be the inverse up to homotopy.

Actually we can make composition strictly associative if we work with
"Moore paths".  A Moore path in X is a continuous map

f: [0,T] \to  X

where T is an arbitrary nonnegative real number.  Given a Moore path
f as above and another Moore path

g: [0,S] \to  X

which starts where f ends, we can compose them in an obvious way to
get a Moore path

fg: [0,T+S] \to  X

Note that this operation is associative "on the nose", not just up to
homotopy.  If we define LX using Moore paths that start and end at the
basepoint, we can easily make LX into a topological monoid - that is,
a topological space with a continuous associative product and a unit
element.  (If you've read section L, you'll know this is just a monoid
object in Top!)  In particular, the unit element of LX is the path 
i: [0,0] \to  X that just sits there at the basepoint of X.

LX is not a topological group, because even Moore paths don't have
strict inverses.  But LX is \emph{close} to being a group.  We can make
this fact precise in various ways, some more detailed than others.
I'm pretty sure one way to say it is this: the natural map from LX to
its "group completion" is a homotopy equivalence.

\par\noindent\rule{\textwidth}{0.4pt}
<STRONG>P.</STRONG>  The group completion of a topological monoid.  Let TopMon be the
category of topological monoids and let TopGp be the category of
topological groups.  There is a forgetful functor

F: TopGp \to  TopMon

and this has a left adjoint

G: TopMon \to  TopGp

which takes a topological monoid and converts it into a topological
group by throwing in formal inverses of all the elements and giving
the resulting group a nice topology.  This functor G is called "group
completion" and was first discussed by Quillen (in the simplicial
context, in an unpublished paper), and independently by Barratt and
Priddy:

6) M. G. Barratt and S. Priddy, On the homology of non-connected
monoids and their associated groups, Comm. Math. Helv. 47 (1972),
1-14.

For any topological monoid M, there is a natural map from M to
F(G(M)), thanks to the miracle of adjoint functors.  This is the
natural map I'm talking about in the previous section!






\par\noindent\rule{\textwidth}{0.4pt}
% </A>
% </A>
% </A>


% parser failed at source line 605
