
% </A>
% </A>
% </A>
\week{March 19, 2008 }


Sorry for the long pause!  I've been busy writing.  For example: 
a gentle introduction to category theory, focusing on its role 
as a "Rosetta Stone" that helps us translate between four languages:

1) John Baez and Mike Stay, Physics, topology, logic and 
computation: a Rosetta Stone, to appear in New Structures in 
Physics, ed. Bob Coecke.  Available at 
<a href = "http://math.ucr.edu/home/baez/rosetta.pdf">http://math.ucr.edu/home/baez/rosetta.pdf</a>

The idea is to take this chart and make it really precise:


\begin{verbatim}

       PHYSICS        TOPOLOGY    LOGIC        COMPUTATION
       Hilbert space  manifold    proposition  data type
       operator       cobordism   proof        program
\end{verbatim}
    

In each case we have a kind of "thing" and a kind of
"process" going between things.  But it turns out we can
make the analogies much sharper and more detailed than that.

The hard work has already been done by many researchers. 
People working on topological quantum field theory have seen
how cobordisms - spacetimes going from one slice of space to
another - are analogous to operators between Hilbert spaces.
The "Curry-Howard correspondence" makes the analogy between
proofs and programs precise.  Girard's work on "linear logic" 
sets up an analogy between operators and proofs.  And so on....

We're just trying to present these analogies in an easy-to-read 
form, all in one place.  I hope that pondering them will help us
break down some walls separating disciplines.  In more optimistic 
moments, I even thnk they represent the first steps toward a 
general theory of systems and processes!  Then I remember that 
scientists are trained to distrust such grand visions, and for 
good reasons.  Time will tell.

But enough of that.  This Week will be an ode to the number 3. 

First, though... here's the nebula of the week!
<br>
<br>
<div align = "center">
<img src = "hourglass_nebula.jpg">
</div>

2) Hubble finds an hourglass nebula around a dying star,
<a href = "http://hubblesite.org/newscenter/archive/releases/nebula/planetary/1996/07/">http://hubblesite.org/newscenter/archive/releases/nebula/planetary/1996/07/</a>

It looks like the eye of Sauron in Tolkien's \emph{Lord of the Rings} 
trilogy.  It's not.  It's a planetary nebula 8000 light years away, called
MyCn 18 - or, more romantically, the Engraved Hourglass Nebula.

The colors look unreal.  They are.

<UL>
<LI>
 H&alpha; light is shown as as green, but it's actually red.  This is 
 the light hydrogen emits when its one electron jumps from its n = 3 
 state to its n = 2 state.

<LI>
 N II light is shown as red, and it actually is.  This is light 
 from singly ionized nitrogen.

<LI>
 O III light is shown as blue, but it's actually green.  This is 
 light from doubly ionized oxygen.  

<LI>
 Furthermore, the colors have been adjusted so that regions where 
 H&alpha; and O II overlap are orange.
</UL>

Okay, so the colors are fake.  But how did this weird nebula form?  
You can see a clue if you pay attention: the bright white dwarf 
star isn't located exactly at the center.  It's a bit to the left!  
This paper, written by the folks who took the photograph, argues 
that it has an unseen companion:

3) Raghvendra Sahai et al, The Etched Hourglass Nebula MyCn 18.  
I: Hubble space telescope observations, The Astronomical Journal
118 (1999), 468-476.  Also available at
<a href = "http://www.iop.org/EJ/article/1538-3881/118/1/468/990080.text.html">http://www.iop.org/EJ/article/1538-3881/118/1/468/990080.text.html</a>

This paper tackles the difficult problem of modelling the nebula:

4) Raghvendra Sahai et al, The Etched Hourglass Nebula MyCn 18.  
II: A spatio-kinematic model, The Astronomical Journal
110 (2000), 315-322.  Also available at
<a href = "http://www.iop.org/EJ/article/1538-3881/119/1/315/990248.text.html">http://www.iop.org/EJ/article/1538-3881/119/1/315/990248.text.html</a>

It doesn't seem that the white dwarf alone could have produced all
the glowing gas we see here.   A red giant companion could help.
But, there are lots of mysteries.  

That shouldn't be surprising.  Even the simplest things can be quite
rich in complexity if you look at them hard enough.  I'll illustrate 
this with a little ode to the number 3.  I'll start off slow, and 
ramp up to a discussion of how all these mathematical entities are 
locked in a tight embrace:

<UL>
<LI>
  the trefoil knot
<LI>
  cubic polynomials
<LI>
  the group of permutations of 3 things
<LI>
  the three-strand braid group
<LI>
  modular forms and cusp forms
</UL>

As I kind of intermezzo, I'll talk about how to solve the cubic 
equation.  We all learn about quadratic equations in school: 
they're the bread and butter of algebra, right after linear 
equations.  Cubics are trickier, but studying them can give you
a lifetime's worth of fun.

Let's start with the trefoil knot.  This is the simplest of knots:

<div align = "center">
<a href = "http://en.wikipedia.org/wiki/Trefoil_knot">
<img style="border:none;" src = "http://upload.wikimedia.org/wikipedia/commons/d/df/TrefoilKnot-01.png">
% </a>
</div>

You can even draw it on the surface of a doughnut!  Just take a pen
and draw a curve that winds around your doughnut three time in one
direction as it winds twice in the other direction:

<div align = "center">
<a href = "http://www.popmath.org.uk/sculpmath/pagesm/torus2.html">
<img style="border:none;" src = "http://www.popmath.org.uk/sculpmath/imagesm/tortef2j.gif">
% </a>
</div>

5) Center for the Popularisation of Mathematics, Torus knots,
<a href = "http://www.popmath.org.uk/sculpmath/pagesm/torus2.html">
http://www.popmath.org.uk/sculpmath/pagesm/torus2.html</a>

Mathematically, the surface of a doughnut is called a "torus".
We can describe a point on the torus by two angles running
from 0 to 2\pi  - the "latitude" and "longitude".  But another name 
for such an angle is a "point on the unit circle".  If we think
of the unit circle in the complex plane, this gives us a nice
equation for the trefoil: 

u^{2} = v^{3}

Here u and v are complex numbers with absolute value 1.  The equation 
says that as u moves around the unit circle, v moves around 2/3 as 
fast.  So, the set of solutions is a curve on the torus that winds 
around thrice in one direction while it winds around twice in the
other direction - a trefoil knot!

We can also drop the restriction that u and v have absolute
value 1.  Then the equation u^{2} = v^{3} is famous for other reasons -
it's related to cubic equations!   

As you've probably heard, there's a formula for solving cubic equations,
sort of like the quadratic formula, but bigger and badder.  It goes 
back to some Italians in the 1500s who liked to challenge each other 
with equations and make bets on who could solve them: Scipione del 
Ferro, Niccolo Tartaglia and Gerolamo Cardano.

Imagine we're trying to solve a cubic equation.   We can always divide 
by the coefficient of the cubic term, so it's enough to consider 
equations like this:

z^{3} + Az^{2} + Bz + C = 0

If we could solve this and find the roots a, b, and c, we could 
write it as:

(z - a)(z - b)(z - c) = 0

But this means

A = -(a + b + c)
<br/>
B = ab + bc + ca
<br/>
C = -abc

Note that A, B, and C don't change when we permute a, b, and c.
So, they're called "symmetric polynomials" in the variables a, b, 
and c.  

You see this directly, but there's also a better explanation: 
the coefficients of a polynomial depend on its roots, but they 
don't change when we permute the roots.

I can't resist mentioning a cool fact, which is deeply related to
the trefoil: \emph{every} symmetric polynomial of a, b, and c can 
be written as a polynomial in A, B, and C - and in a unique way!

In fact, this sort of thing works not just for cubics, but for 
polynomials of any degree.  Take a general polynomial of degree n 
and write the coefficients as functions of the roots.  Then these 
functions are symmetric polynomials, and \emph{every} symmetric polynomial 
in n variables can be written as a polynomial of these - and in a 
unique way.  

But, back to our cubic.  Note that -A/3 is the average of the three 
roots.  So, if we slide z over like this:

x = z + A/3
we get a new cubic equation for which the average of the three
roots is zero.  This new cubic equation will be of this form:

x^{3} + Bx + C = 0

for some new numbers B and C.  In other words, the "A" in this new
cubic is zero, since we translated the roots to make their average 
zero.

So, to solve cubic equations, it's enough to solve cubics like
x^{3} + Bx + C = 0.  This is a great simplification.  When you
first see it, it's really exciting.  But then you realize you have no
idea what to do next!  This must be why it's called a "depressed
cubic".

In fact, Scipione del Ferro figured out how to solve the "depressed
cubic" shortly after 1500.  So, you might think he could solve any 
cubic.  But, \emph{negative numbers hadn't been invented yet}.  This 
prevented him from reducing any cubic to a depressed one!  

It's sort of hilarious that Ferro was solving cubic equations before 
negative numbers were worked out.  It should serve as a lesson: we 
mathematicians often work on fancy stuff before understanding the 
basics.  Often that's why math seemss hard!  But often it's impossible 
to discover the basics except by working on fancy stuff and getting 
stuck.

Here's one trick for solving the depressed cubic x^{3} + Bx + C = 0.
Write

x = y - B/(3y)

Plugging this in the cubic, you'll get a quadratic equation in y^{3}, 
which you can solve.  From this you can figure out y, and then x.  

Alas, I have no idea what this trick means.  Does anyone know?  Ferro 
and Tartaglia used a more long-winded method that seems just as sneaky. 
Later Lagrange solved the cubic yet another way.  I like his way
because it contains strong hints of Galois theory.  

You can see all these methods here:

6) Wikipedia, Cubic function, 
<a href = "http://en.wikipedia.org/wiki/Cubic_equation">http://en.wikipedia.org/wiki/Cubic_equation</a>.

So, I won't say more about solving the cubic now.  Instead, I want to 
explain the "discriminant".  This is a trick for telling when two 
roots of our cubic are equal.  It turns out to be related to the 
trefoil knot.

For a quadratic equation ax^{2} + bx + c = 0, the two roots are equal 
precisely when b^{2} - 4ac = 0.  That's why b^{2} - 4ac is called the 
"discriminant" of the quadratic.  The same idea works for other 
equations; let's see how it goes for the cubic.  

Suppose we were smart enough to find the roots of our cubic

x^{3} + Bx + C = 0

and write it as

(x - a)(x - b)(x - c) = 0

Then two roots are equal precisely when 

(a - b)(b - c)(c - a) = 0

The left side isn't a symmetric polynomial in a, b, and c; it changes 
sign whenever we switch two of these variables.  But if we square it,
we get a symmetric polynomial that does the same job:

D = (a - b)^{2} (b - c)^{2} (c - a)^{2}

This is the discriminant of the cubic!  By what I said about symmetric 
polynomials, it has to be a polynomial in B and C (since A = 0).  If 
you sweat a while, you'll see 

D = -4B^{3} - 27C^{2}

So, here's the grand picture: we've got a 2-dimensional space of 
cubics with coordinates B and C.   Sitting inside this 2d space is a 
curve consisting of "degenerate" cubics - cubics with two roots the 
same.  This curve is called the "discriminant locus", since it's where 
the discriminant vanishes:

4B^{3} + 27C^{2} = 0

If we only consider the case where B and C are real, the discriminant 
locus looks like this:


\begin{verbatim}

                   |C
           o       |      
            o      |     
               o   |  
        -----------o-------------
               o   |           B
            o      |     
           o       |
                   |
\end{verbatim}
    
It's smooth except at the origin, where it has a sharp point called
a "cusp".

Now here's where the trefoil knot comes in.  The equation for the 
discriminant locus:

4B^{3} + 27C^{2} = 0

should remind you of the equation for the trefoil:

u^{2} = v^{3} 

Indeed, after a linear change of variables they're the same!   
But, for the trefoil we need u and v to be \emph{complex} numbers.  
We took them to be unit complex numbers, in fact.

So, the story is this: we've got a 2-dimensional \emph{complex} space 
of complex cubics.  Sitting inside it is a \emph{complex} curve, the
discriminant locus.  In our new variables, it's this:

u^{2} = v^{3}

If we intersect this discriminant locus with the torus 

|u| = |v| = 1

we get a trefoil knot.  But that's not all!

Normal folks think of knots as living in ordinary 3d space, but 
topologists often think of them as living in a 3-sphere: a sphere 
in 4d space.  That's good for us.  We can take this 4d space to be
our 2d complex space of complex cubics!  We can pick out spheres in 
this space by equations like this:

|u|^{2} + |v|^{3} = c     &nbsp; &nbsp;            (c > 0)

These are not round 3-spheres, thanks to that annoying third power.
But, they're topologically 3-spheres.  If we take any one of them 
and intersect it with our discriminant locus, we get a trefoil knot!  
This is clear when c = 2, since then we have

|u|^{2} + |v|^{3} = 2

and 

u^{2} = v^{3}

which together imply 

|u| = |v| = 1

But if you think about it, we also get a trefoil knot for any other
c > 0.  This trefoil shrinks as c \to  0, and at c = 0 it reduces to a 
single point, which is also the cusp here:


\begin{verbatim}

                     |u         
                     |      o
                     |     o
                     |   o
          -----------o-------------
                     |   o        v
                     |     o
                     |      o
                     |          
\end{verbatim}
    
We don't see trefoil knots in this picture because it's just a
real 2d slice of the complex 2d picture.  But, they're lurking in
the background!  

Now let me say how the group of permutations of three things
gets into the game.  We've already seen the three things: they're 
the roots a, b, and c of our depressed cubic!  So, they're three
points on the complex plane that add to zero.  Being a physicist
at heart, I sometimes imagine them as three equal-mass planets, 
whose center of mass is at the origin.

The space of possible positions of these planets is a 2d complex
vector space, since we can use any two of their positions as 
coordinates and define the third using the relation

a + b + c = 0

So, there are three coordinate systems we can use: the (a,b)
system, the (b,c) system and the (c,a) system.  We can draw all 
three coordinate systems at once like this:


\begin{verbatim}

                b         
                 \       /
                  \     /
                   \   /
                    \ /
             --------o--------a
                    / \
                   /   \
                  /     \
                 /       \
                c
\end{verbatim}
    
The group of permutations of 3 things acts on this picture 
by permuting the three axes.  Beware: I've only drawn a 2-dimensional 
\emph{real} vector space here, just a slice of the full 2d complex space.  

Now suppose we take this 2d complex space and mod out by the 
permutation symmetries.  What do we get?  It turns out we get \emph{another}
2d complex vector space!   In this new space, the three coordinate axes
shown above become just one thing... but this thing is a curve, like
this:


\begin{verbatim}

            o
             o
                o
                   o
                o
             o
            o
\end{verbatim}
    
Look familiar?  Sure!  It's just the discriminant locus we've
seen before.

Why does it work this way?  The explanation is sitting before us.
We've got two 2d complex vector spaces: the space of possible
\emph{ordered triples of roots} of a depressed cubic, and the space
of possible \emph{coefficients}.  There's a map from the first space
to the second, since the coefficients are functions of the roots:

B = ab + bc + ca <br/>
C = -abc

These functions are symmetric polynomials: they don't change when
we permute a, b, and c.  And, it follows from what I said earlier 
that we can get \emph{any} symmetric polynomial as a function of these -
under the assumption that a+b+c = 0, that is.  

So, the map where we mod out by permutation symmetries of the roots 
is exactly the map from roots to coefficients.

The lines in this picture are places where two roots are equal:


\begin{verbatim}

            c=a        
              \       /
               \     /
                \   /
                 \ /
          --------o-------- b=c
                 / \
                /   \
               /     \
              /       \
            a=b
\end{verbatim}
    
So, when we apply the map from roots to coefficients, these lines
get mapped to the discriminant locus:


\begin{verbatim}

                  |
          o       |      
           o      |     
              o   |  
       -----------o-------------
              o   |           
           o      |     
          o       |
                  |
\end{verbatim}
    
You should now feel happy and quit reading... unless you
know a bit of topology.  If you \emph{do} know a little topology, 
here's a nice spinoff of what we've done.   Though I didn't say
it using so much jargon, we've already seen that space of 
nondegenerate depressed cubics is C^{2} minus a cone on the 
trefoil knot.  So, the fundamental group of this space is the
same as the fundamental group of S^{3} minus a trefoil knot.  
This is a famous group: it has three generators x,y,z, and three 
relations saying that:

<ul>
<li>
x conjugated by y is z 
<li>
y conjugated by z is x 
<li>
z conjugated by x is y 
</ul>

On the other hand, we've seen this space is the space of triples
of distinct points in the plane, centered at the origin, mod 
permutations.   The condition "centered at the origin"
doesn't affect the fundamental group.  So, this fundamental 
group is another famous group: the "braid group on 3 strands".  
This has two generators:


\begin{verbatim}

\ /  |
 /   |          X
/ \  |
\end{verbatim}
    
and


\begin{verbatim}

|  \ /
|   /           Y
|  / \
\end{verbatim}
    
and one relation, called the "Yang-Baxter equation" or "third 
Reidemeister move":


\begin{verbatim}

\ /  |        |  \ /
 /   |        |   /
/ \  |        |  / \
|  \ /        \ /   |
|   /     =    /    |           XYX = YXY
|  / \        / \   |
\ /  |        |  \ /
 /   |        |   /
/ \  |        |  / \
\end{verbatim}
    
So: the 3-strand braid group is \emph{isomorphic} to the fundamental 
group of the complement of the trefoil!  You may enjoy checking 
this algebraically, using generators and relations, and then 
figuring out how this algebraic proof relates to the geometrical 
proof.

I find all this stuff very pretty...

... but what's really \emph{magnificent} is that most of it
generalizes to any Dynkin diagram, or even any Coxeter diagram!  (See
"<a href = "week62.html">week62</A>" for those.)

Yes, we've secretly been studying the Coxeter diagram A_{2}, whose 
"Coxeter group" is the group of permutations of 3 things, and whose 
"Weyl chambers" look like this:

                         

\begin{verbatim}

                 \       /
                  \     /
                   \   /
                    \ /
             --------o--------
                    / \
                   /   \
                  /     \
                 /       \
\end{verbatim}
    
Let me just sketch how we can generalize this to A_{n-1}.   Here 
the Coxeter group is the group of permutations of n things, which
I'll call n!.

Let X be the space of n-tuples of complex numbers summing to 0.  
X is a complex vector space of dimension n-1.  We can think of
any point in X as the ordered n-tuple of roots of some depressed polynomial
of degree n.  Here "depressed" means that the leading coefficient 
is 1 and the sum of the roots is zero.  This condition makes polynomials
sad.

The permutation group n! acts on X in an obvious way.  The 
quotient X/n! is isomorphic (as a variety) to another complex 
vector space of dimension n-1: namely, the space of depressed 
polynomials of degree n.  The quotient map

X \to  X/n!

is just the map from roots to coefficients!  

Sitting inside X is the set D consisting of n-tuples of roots
where two or more roots are equal.  D is the union of a bunch of hyperplanes,
as we saw in our example:

                         

\begin{verbatim}

                  \       /
                   \     /
                    \   /
                     \ /
              --------o--------
                     / \
                    /   \
                   /     \
                  /       \
\end{verbatim}
    
Sitting inside X/n! is the "discriminant locus" D/n!, consisting
of \emph{degenerate} depressed polynomials of degree n - that is, those
with two or more roots equal.  This is a variety that's smooth except for 
some sort of "cusp" at the origin:


\begin{verbatim}

              o
               o
                  o
                      o
                  o
               o
              o
\end{verbatim}
    
The fundamental group of the complement of the discriminant locus
is the braid group on n strands.  The reason is that this group
describes homotopy classes of ways that n points in the plane can
move around and come back to where they were (but possibly permuted).
These points are the roots of our polynomial.

On the other hand, the discriminant locus is topologically the cone 
on some higher-dimensional knot sitting inside the unit sphere in 
C^{n-1}.  So, the fundamental group of the complement of this knot
is the braid group on n strands.

This relation between higher-dimensional knots and singularities 
was investigated by Milnor, not just for the A_{n} series of Coxeter 
diagrams but more generally:

7) John W. Milnor, Singular Points of Complex Hypersurfaces, 
Princeton U. Press, 1969.

The other Coxeter diagrams give generalizations of braid groups
called Artin-Brieskorn groups.  Algebraically you get them by taking
the usual presentations of the Coxeter groups and dropping the 
relations saying the generators (reflections) square to 1.  

If you like braid groups and Dynkin diagrams, Artin-Brieskorn groups 
are irresistible!  For a fun modern account, try:

8) Daniel Allcock, Braid pictures for Artin groups, available as
<a href = "http://arxiv.org/abs/math.GT/9907194">arXiv:math.GT/9907194</a>.

But I'm digressing!  I must return and finish my ode to the number 3.
I need to say how modular forms get into the game!

I'll pick up the pace a bit now - if you're tired, quit here.

Any cubic polynomial P(x) gives something called an "elliptic
curve".  This consists of all the complex solutions of

y^{2} = P(x)

together with the point (\infty , \infty ), which we include to
make things nicer.

Clearly this elliptic curve has two points (x,y) for each value of x
\emph{except} for x = \infty  and the roots of P(x), where it just has one.
So, it's a "branched double cover" of the Riemann sphere,
with branch points at the roots of our cubic and the point at infinity.

In fact, this elliptic curve has the topology of a torus, at least 
when all the roots of our cubic are different.  If you have trouble
imagining a torus that's a branched double cover of a sphere, ponder
this:

<div align = "center">
<img src = "quincuncial_tiled.jpg">
</div>

9) Carlos Furuti, Peirce's quincuncial map, 
<a href = "http://www.progonos.com/furuti/MapProj/Normal/ProjConf/projConf.html">http://www.progonos.com/furuti/MapProj/Normal/ProjConf/projConf.html</a>

This square map of the Earth is an unwrapped torus; each point of the
Earth shows up lots of times.  If we wrap it up just right, we get a
branched double cover of the sphere!  Can you spot the branch points?
For a lot more explanation, read "<a href =
"week229.html">week229</A>".

Now, way back in "<a href = "week13.html">week13</A>", I turned this story around.  I started with 
a torus formed as the quotient of the complex plane by a lattice -
and showed how to get an elliptic curve out of it.  I wrote the 
equation for this elliptic curve in "Weierstrass form":

y^{2} = 4x^{3} - g_{2} x - g_{3}

By a simple change of variables, this is equivalent to a depressed 
cubic:

y^{2} = x^{3} + Bx + C

So, we can think of g_{2} and g_{3} as coordinates on
our 2d space of depressed cubics!  They're just rescaled versions of
our coordinate functions B and C.

What's the big deal?  Well, g_{2} and g_{3} are famous
examples of "modular forms" - whatever those are.  In fact,
it's a famous fact that every modular form is a polynomial in
g_{2} and g_{3}.

I defined modular forms back in "<a href = "week142.html">week142</A>", where I summarized the
Taniyama-Shimura-Weil theorem: the big theorem about modular forms
that implies Fermat's Last Theorem.  So, you can reread the definition
there if you're curious.  But if you've never seen it before, it's
a bit intimidating.  A modular form of weight w is a function on the 
space of lattices that transforms in a certain bizarre way, satisfying a 
certain growth condition... blah blah blah.

It's important stuff, and incredibly cool once you get a feel for it.
But suppose we're trying to explain modular forms more simply.
Then we can avoid a lot of technicalities if we just say 
a modular form is a polynomial on the space of depressed cubics!  In 
other words, a polynomial in our friends B and C.  

Then we can make some definitions.  The "weight" of the modular form

B^{i} C^{j}

is 4i+6j.  Okay, I admit this sounds arbitrary and weird without a lot
more explanation.  But better: a "cusp form" is a modular
form that vanishes on the discriminant locus.  Then we can see every
cusp form is the product of the discriminant 4B^{3} +
27C^{2} and some other modular form... and we can use this to
work out lots of basic stuff about modular forms.

So, I hope you now see how tightly entwined all these ideas are:

<UL>
<LI>
  the trefoil knot
<LI>
  cubic polynomials
<LI>
  the group of permutations of 3 things
<LI>
  the three-strand braid group
<LI>
  modular forms and cusp forms
</UL>

At this point I should give credit where credit is due.  As usual,
I've been talking to Jim Dolan, and many of these ideas come from
him.  But also, you can think of this Week as an expansion of the
remarks by Joe Christy and Swiatowslaw Gal in the Addenda to 
"<a href = "week233.html">week233</A>".  And, it was
Chris Hillman who first told Jim and me that SL(2,R)/SL(2,Z) looks
like S^{3} minus a trefoil knot.

Finally, I should say that my low-budget approach to modular forms
mostly just handles so-called "level 0" modular forms - the basic 
kind, defined using the group 

\Gamma  = PSL(2,Z)

More exciting are modular forms that transform nicely only for a 
\emph{subgroup} of \Gamma .   Jim and I are just beginning to understand
these.  But the modular forms for \Gamma (2) fit nicely into today's 
ode!  Here \Gamma (2) is the subgroup of \Gamma  consisting of matrices
congruent to the identity matrix mod 2.  What does this have to do
with my ode to the number 3?  Well,

\Gamma /\Gamma (2) &cong; PSL(2,F_{2}) 

and this is isomorphic to the group of permutations of 3 things! 

So, as a final flourish, I claim that:

Modular forms for \Gamma (2) are polynomials on the space X
consisting of roots of depressed cubics:

X = {(a,b,c): a,b,c complex with a + b + c = 0}

Modular forms for \Gamma  are polynomials on the space X/3!
consisting of coefficients of depressed cubics:

X/3! = {(B,C): B,C complex}

The obvious quotient map X \to  X/3! sends roots to coeffficients:

(a,b,c) |\to  (B,C) = (ab + bc + ca, abc)

and this induces the inclusion of modular forms for \Gamma  into
modular forms for \Gamma (2):

B |\to  ab + bc + ca  <br/>
C |\to  abc

I hope this is all true!

Modular forms for \Gamma (2) are particularly nice.   A good example
is the \emph{cross-ratio}, much beloved in complex analysis.  If you want
to learn more about this stuff, try:

10) Igor V. Dolgachev, Lectures on modular forms, Fall 1997/8, 
available at <a href = "http://www.math.lsa.umich.edu/~idolga/modular.pdf">http://www.math.lsa.umich.edu/~idolga/modular.pdf</a>

especially chapter 9 for level 2 modular forms.  Also:

11) Henry McKean and Victor Moll, Elliptic Curves: Function Theory,
Geometry, Arithmetic, Cambridge U. Press, 1999.

especially chapter 4.

\par\noindent\rule{\textwidth}{0.4pt}
\textbf{Addendum:} For more discussion, go to the
<a href = "http://golem.ph.utexas.edu/category/2007/12/this_weeks_finds_in_mathematic_20.html">\emph{n}-Category Caf&eacute;</a>.

\par\noindent\rule{\textwidth}{0.4pt}
<em>It is difficult to give an idea of the vast extent of modern mathematics.
The word "extent" is not the right one: I mean extent crowded with 
beautiful detail - not an extent of mere uniformity such as an objectless
plain, but a tract of beautiful country to be rambled through and studied
to every detail of hillside and valley, stream, rock, wood and flower.</em> 
- Arthur Cayley
\par\noindent\rule{\textwidth}{0.4pt}

% </A>
% </A>
% </A>
