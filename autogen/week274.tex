
% </A>
% </A>
% </A>
\week{March 12, 2009 }


Whew!  It's been a long time since I wrote my last Week's Finds.
I've been too busy.  But luckily, I've been too busy writing papers 
about math and physics.  So, let me talk about one of those.  

First, the astronomy picture of the week:

<div align = "center">
<a href = "http://photojournal.jpl.nasa.gov/catalog/PIA08813">
<img src = "mars_victoria_crater_overhead.jpg">
% </a>
</div>

1) NASA Photojournal, 'Victoria Crater' at Meridiani Planum,
<a href = "http://photojournal.jpl.nasa.gov/catalog/PIA08813">http://photojournal.jpl.nasa.gov/catalog/PIA08813</a>

This is the crater that NASA's rover called Opportunity has been
exploring.  It's 800 meters across.  I like this picture just 
because it's beautiful.  It was taken by the High Resolution 
Imaging Science Experiment on NASA's Mars Reconnaissance Orbiter. 

Now, on to business!  I want to talk about this paper, which
took over 2 years to write:

2) John Baez, Aristide Baratin, Laurent Freidel and Derek Wise,
Representations of 2-groups on infinite-dimensional 2-vector spaces,
available as <a href = "http://arxiv.org/abs/0812.4969">arXiv:0812.4969</a>.

We can dream up the notion of "2-vector space" by pondering
this analogy chart:


\begin{verbatim}

numbers              vector spaces
addition             direct sum
multiplication       tensor product       
0                    the 0-dimensional vector space
1                    the 1-dimensional vector space
\end{verbatim}
    

Just as you can add and multiply numbers, you can add and multiply
vector spaces - but people call these operations "direct sum" and
"tensor product", to make them sound more intimidating.  These new
operations satisfy axioms similar to the old ones.  However, what 
used to be equations like this:

x + y = y + x

now become isomorphisms like this:

X + Y \cong  Y + X.

This means we're "categorifying" the concepts of plus and times.

The unit for addition of vector spaces is the 0-dimensional
vector space, and the unit for multiplication of vector spaces 
is the 1-dimensional vector space.  

But here's the coolest part.  Our chart is like a snake eating its own
tail.  The first entry of the first column matches the last entry of
the second column!  The set of all "numbers" is the same as
"the 1-dimensional vector space".  If by "numbers"
we mean complex numbers, these are both just C.

This suggests continuing the chart with a third column, like this:


$$

numbers (C)          vector spaces (Vect)     2-vector spaces (2Vect)
addition             direct sum               direct sum
multiplication       tensor product           tensor product  
0                    C^{0}                       Vect^{0}
1                    C^{1}                       Vect^{1}
$$
    

Here C^{0} is short for the 0-dimensional vector space, while
C^{1} is short for the 1-dimensional vector space - in other
words the complex numbers, C.  Vect is the category of all vector
spaces.  So, whatever a "2-vector space" is, to make the
chart nice we'd better have Vect be the 1-dimensional 2-vector space.
We can emphasize this by calling it Vect^{1}.

In fact, about 15 years ago Kapranov and Voevodsky invented a 
theory of 2-vector spaces that makes all this stuff work:

3) Mikhail Kapranov and Vladimir Voevodsky, 2-categories and
Zamolodchikov tetrahedra equations, in Algebraic Groups and
Their Generalizations: Quantum and Infinite-Dimensional Methods,
Proc. Sympos. Pure Math. 56, Part 2, AMS, Providence,
RI, 1994, pp. 177-259.

They mainly considered \emph{finite-dimensional} 2-vector spaces.
Every finite-dimensional vector space is secretly just C^{n},
or at least something isomorphic to that.  Similarly, every
finite-dimensional 2-vector space is secretly just Vect^{n},
or at least something equivalent to that.

(You see, when we categorify once, equality becomes isomorphism.
When we do it again, isomorphism becomes "equivalence".)

What's Vect^{n}, you ask?  Well, what's C^{n}?  It's the set where
an element is an n-tuple of numbers:

(x_{1}, ..., x_{n})

So, Vect^{n} is the category where an object is an n-tuple of 
vector spaces:

(X_{1}, ..., X_{n})

It's all pathetically straightforward.  Of course we also need
to know what's a morphism in Vect^{n}.  What's a morphism from 

(X_{1}, ..., X_{n})

to

(Y_{1}, ..., Y_{n})?

It's just the obvious thing: an n-tuple of linear operators

(f_{1}: X_{1} \to  Y_{1}, ..., f_{n}: X_{n} \to  Y_{n})

And we compose these in the obvious way, namely "componentwise".

This may seem like an exercise in abstract nonsense, extending formal 
patterns just for the fun of it.  But in fact, 2-vector spaces are
all over the place once you start looking.  For example, take
the category of representations of a finite group, or the category
of vector bundles over a finite set.  These are finite-dimensional
2-vector spaces!  

Here I can't resist a more sophisticated digression, just to impress
you.  The whole theory of Fourier transforms for finite abelian groups 
categorifies nicely, using these examples.  Any finite abelian group 
G has "Pontryagin dual" G* which is again a finite abelian group.   
I explained how this works back in "<a href = "week273.html">week273</A>".  The Fourier transform 
is a map from functions on G to functions on G*.  So, it's a map 
between vector spaces.  But, lurking behind this is a map between 
2-vector spaces!  It's a map from representations of G to vector 
bundles over G*.

You can safely ignore that last paragraph if you like.  But if you 
want more details, try section 6.1 of this old paper:

4) John Baez, Higher-dimensional algebra II: 2-Hilbert spaces,
Adv. Math. 127 (1997), 125-189.  Also available as 
<a href = "http://arXiv.org/abs/arXiv:q-alg/9609018">arXiv:q-alg/9609018</A>.

As you can see from the title, I was trying to go beyond 2-vector
spaces and think about "2-Hilbert spaces".  That's because
in quantum physics, we use Hilbert spaces to describe physical
systems.  Recent work on physics suggests that we categorify this idea
and study 2-Hilbert spaces, 3-Hilbert spaces and so on - see "<a
href = "week58.html">week58</A>" for details.  In the above paper
I defined and studied finite-dimensional 2-Hilbert spaces.  But a lot
of the gnarly fun details of Hilbert space theory show up only for
infinite-dimensional Hilbert spaces - and we should expect the same
for 2-Hilbert spaces.

After these old papers on 2-vector spaces and 2-Hilbert spaces,
various people came along and improved the whole story.  For example:

5) Martin Neuchl, Representation Theory of Hopf Categories,
Ph.D. dissertation, University of Munich, 1997.  Chapter 2:
2-dimensional linear algebra.  Available at <a href =
"http://math.ucr.edu/home/baez/neuchl.ps">http://math.ucr.edu/home/baez/neuchl.ps</a>

6) Josep Elgueta, A strict totally coordinatized version of Kapranov
and Voevodsky 2-vector spaces, to appear in Math. Proc.  Cambridge
Phil. Soc.  Also available as <a href =
"http://arXiv.org/abs/math/0406475">arXiv:math/0406475</A>.

7) Bruce Bartlett, The geometry of unitary 2-representations of finite
groups and their 2-characters, available as <a href =
"http://arXiv.org/abs/0807.1329">arXiv/0807.1329</a>.

In the last of these, Bruce worked out how finite-dimensional 
2-Hilbert spaces arise naturally in certain topological 
quantum field theories! 

Just as we can study representations of groups on vector spaces, we
can study representations of "2-groups" on 2-vector spaces:

8) Magnus Forrester-Barker, Representations of crossed modules and
cat^{1}-groups, Ph.D. thesis, Department of Mathematics,
University of Wales, Bangor, 2004.  Available at <a href =
"http://www.maths.bangor.ac.uk/research/ftp/theses/forrester-barker.pdf">
http://www.maths.bangor.ac.uk/research/ftp/theses/forrester-barker.pdf</a>

9) John W. Barrett and Marco Mackaay, Categorical representations of
categorical groups, Th. Appl. Cat. 16 (2006), 529-557.  Also available
as <a href =
"http://arXiv.org/abs/math/0407463">arXiv:math/0407463</a>.

10) Josep Elgueta, Representation theory of 2-groups on finite
dimensional 2-vector spaces, available as <a href =
"http://arXiv.org/abs/math.CT/0408120">arXiv:math.CT/0408120</a>.

A group is a category with one object, all of whose morphisms are
invertible.  Similarly, a 2-group is a 2-category with one object, all
of whose morphisms and 2-morphisms are invertible.  Just as we can
define "Lie groups" to be groups where the group operations
are smooth, we can define "Lie 2-groups" to be 2-groups
where all the 2-group operations are smooth.  Lie groups are wonderful
things, so we can hope Lie 2-groups will be interesting too.  There
are already lots of examples known.  You can see a bunch here:

11) John Baez and Aaron Lauda, Higher-dimensional algebra V: 2-groups,
Theory and Applications of Categories 12 (2004), 423-491.  Available
at <a href =
"http://www.tac.mta.ca/tac/volumes/12/14/12-14abs.html">http://www.tac.mta.ca/tac/volumes/12/14/12-14abs.html</a>
and as <a href =
"http://arXiv.org/abs/arXiv:math/0307200">arXiv:math/0307200</A>.

However, Barrett and Mackaay discovered something rather upsetting.
While Lie groups have lots of interesting representations on
finite-dimensional vector spaces, Lie 2-groups don't have many
representations on finite-dimensional 2-vector spaces!  

In fact, the problem already shows up for representations of
plain old Lie \emph{groups} on 2-vector spaces.  A Lie group can be
seen as a special sort of Lie 2-group, where the only 2-morphisms 
are identity morphisms.

The problem is that unlike a vector space, a 2-vector space has 
a unique basis - at least up to isomorphism.  In C^{n} there's 
an obvious basis consisting of vectors like

(1,0,0,...)<br/>
(0,1,0,...)<br/>
(0,0,1,...)

and so on, but there are lots of other bases too.  But in
Vect^{n} the only basis goes like this:

(C^{1},C^{0},C^{0},...)<br/>
(C^{0},C^{1},C^{0},...)<br/>
(C^{0},C^{0},C^{1},...)

Well, I'm exaggerating slightly: we could replace C^{1} here
by any other 1-dimensional vector space, and C^{0} by any
other 0-dimensional vector space.  That would give other bases - but
they'd still be \emph{isomorphic} to the basis shown above.

So, if we have a group acting on a finite-dimensional 2-vector space, 
it can't do much more than permute the basis elements.  So, any 
representation of a group on a finite-dimensional 2-vector space gives 
an action of this group as permutations of a finite set!

That's okay for finite groups, since these can act in interesting ways
as permutations of finite sets.  But it's no good for Lie groups.  Lie
groups are usually infinite: they're manifolds.  So, they have lots of
actions on \emph{manifolds}, but not many actions on finite sets.

This suggests that to study representations of Lie groups (or more
general Lie 2-groups) on 2-vector spaces, we should invent some notion
of "infinite-dimensional 2-vector space", where the basis
can be not a finite set but an infinite set - indeed, something more
like a manifold.

Luckily, such a concept was already lurking in the mathematical 
literature! 

In the categorification game, it's always good when 
the concepts you invent shed light on existing issues in mathematics.  
And it's especially fun when you categorify a concept and get a 
concept that turns out to have been known - or at least partially 
known - under some other name.  Then you're not just making up new 
stuff: you're seeing that existing math already had categorification 
built into it!  This happens surprisingly often.  That's why I take
categorification so seriously.

The concept I'm talking about here is called a "field of Hilbert 
spaces".   Roughly speaking, the idea is that you pick a set X, 
possibly infinite.   X could be the real line, for example.  Then a 
"field of Hilbert spaces" assigns to each point x in X a Hilbert 
space H_{x}.  

As I've just described it, a measurable field of Hilbert spaces is an
object in what we might call Hilb^{X} - a hairier, scarier
relative of the tame little Vect^{n} that I've been talking
about.

Let's think about how Hilb^{X} differs from Vect^{n}.
First, the the finite number n has been replaced by an infinite set X.
That's why Hilb^{X} deserves to be thought of as an
\emph{infinite-dimensional} 2-vector space.

Second, Vect has been replaced by Hilb - the category of Hilbert
spaces.  This suggests that Hilb^{X} is something more than a
mere infinite-dimensional 2-vector space.  It's closer to an
infinite-dimensional \emph{2-Hilbert} space!  So, we've departed
somewhat from our original goal of inventing a notion of
infinite-dimensional vector space.  But that's okay, especially if
we're interested in applications to quantum physics that involve
analysis.

And here I must admit that I've left out some important details.  When
studying fields of Hilbert spaces, people usually bring in some
analysis to keep the Hilbert space H_{x} from jumping around
too wildly as x varies.  They restrict attention to
"measurable" fields of Hilbert spaces.  To do this, they
assume X is a "measurable space": a space with a
sigma-algebra of subsets, like the Borel sets of the real line.  Then
they assume H_{x} depends in a measurable way on x.

The last assumption must be made precise.  I won't do that here - you
can see the details in our paper.  But, here's an example of what I
mean.  Take X and partition it into countably many disjoint measurable
subsets.  For each one of these subsets, pick some Hilbert space H and
let H_{x} = H for points x in that subset.  So, the dimension
of the Hilbert space H_{x} can change as x moves around, but
only in a "measurable way".  In fact, every measurable field
of Hilbert spaces is isomorphic to one of this form.

So, a measurable field of Hilbert spaces on X is like a vector bundle
over X, except the fibers are Hilbert spaces and there's no smoothness
or continuity - the dimension of the fiber can "jump" in a measurable
way.  

If you've studied algebraic geometry, this should remind you of a
"coherent sheaf".  That's another generalization of a vector
bundle that allows the dimension of the fiber to jump - but in an
algebraic way, rather than a measurable way.  One reason algebraic
geometers like categories of coherent sheaves is because they need a
notion of infinite-dimensional 2-vector space.  Similarly, one reason
analysts like measurable fields of Hilbert spaces is because they want
\emph{their own} notion of infinite-dimensional 2-vector space.  Of
course, they don't know this - if you ask, they'll strenuously deny
it.

We learned most of what we know about measurable fields of Hilbert 
spaces from this classic book:

12) Jacques Dixmier, Von Neumann Algebras, North-Holland, Amsterdam, 
1981.

This book was also helpful:

13) William Arveson, An Invitation to C*-Algebra, Chapter 2.2,
Springer, Berlin, 1976.

As you might guess from the titles of these books, measurable fields
of Hilbert spaces show up when we study representations of operator
algebras that arise in quantum theory.  For example, any commutative
von Neumann algebra A is isomorphic to the algebra
L^{\infty }(X) for some measure space X, and every
representation of A comes from a measurable field of Hilbert spaces on
X.

The following treatment is less detailed, but it explains how 
measurable fields of Hilbert spaces show up in group representation 
theory:

14) George W. Mackey, Unitary Group Representations in Physics, 
Probability and Number Theory, Benjamin-Cummings, New York, 1978.

I'll say a lot more about this at the very end of this post, but
here's a quick, rough summary.  Any sufficiently nice topological
group G has a "dual": a measure space G* whose points are
irreducible representations of G.  You can build any representation of
G from a measurable field of Hilbert spaces on G* together with a
measure on G*.  You build the representation by taking a "direct
integral" of Hilbert spaces over G*.  This is a generalization of
writing a representation as a direct sum of irreducible
representation.  Direct integrals generalize direct sums - just as
integrals generalize sums!

By the way, Mackey calls measurable fields of Hilbert spaces
"measurable Hilbert space bundles".  Those who like vector
bundles will enjoy his outlook.

But let's get back to our main theme: representations of 2-groups
on infinite-dimensional 2-vector spaces.  

We don't know the general definition of an infinite-dimensional 
2-vector space.  However, for any measurable space X, we can 
define measurable fields of Hilbert spaces on X.  We can also define
maps between them, so we get a category, called Meas(X).  Crane and 
Yetter call these "measurable categories".

I believe someday we'll see that measurable categories are a 
halfway house between infinite-dimensional 2-vector spaces and 
infinite-dimensional 2-Hilbert spaces.  In fact, when we move 
up to n-vector spaces, it seems there could be n+1 different levels 
of "Hilbertness".  

The conclusions of our paper include a proposed definition of
2-Hilbert space that can handle the infinite-dimensional case.  
So, why work with measurable categories?  One reason is that they're 
they're well understood, thanks in part to the work of Dixmier - but 
also thanks to Crane and Yetter:

15) David Yetter, Measurable categories, Appl. Cat. Str. 13 (2005), 
469-500.  Also available as <a href = "http://arXiv.org/abs/math/0309185">arXiv:math/0309185</A>.

16) Louis Crane and David N. Yetter, Measurable categories and 
2-groups, Appl. Cat. Str. 13 (2005), 501-516.  Also available as
<a href = "http://arXiv.org/abs/math/0305176">arXiv:math/0305176</A>.

The paper by Crane and Yetter studies representations of discrete
2-groups on measurable categories.  Our paper pushes forward by
studying representations of \emph{topological} 2-groups, including
Lie 2-groups.  Topology really matters for infinite-dimensional
representations.  For example, it's a hopeless task to classify the
infinite-dimensional unitary representations of even a little group
like the circle, U(1).  But it's easy to classify the
\emph{continuous} unitary representations.

A group has a category of representations, but a 2-group has a
2-category of representations!  So, as usual, we have representations
and maps between these , which physicists call "intertwining
operators" or "intertwiners" for short.  But we also
have maps between intertwining operators, called
"2-intertwiners".

This is what's really exciting about 2-group representation theory.
Indeed, intertwiners between 2-group representations resemble group
representations in many ways - a fact noticed by Elgueta.  It turns
out one can define direct sums and tensor products not only for
2-group representations, but also for intertwiners!  One can also
define "irreducibility" and "indecomposability",
not just for representations, but also for intertwiners.

Our paper gives nice geometrical descriptions of these notions.
Some of these can be seen as generalizing the following paper 
of Crane and Sheppeard:

17) Louis Crane and Marnie D. Sheppeard, 2-categorical Poincare
representations and state sum applications, available as
<a href = "http://arXiv.org/abs/arXiv:math/0306440">arXiv:math/0306440</A>.

Crane and Sheppeard studied the 2-category of representations of the
"Poincare 2-group".  It turns out that we can get
representations of the Poincare 2-group from 
<a href = "http://en.wikipedia.org/wiki/Fuchsian_group">discrete 
subgroups of the Lorentz group</a>.  Since the Lorentz group acts 
as symmetries of the hyperbolic plane, these subgroups come from 
symmetrical patterns like these:


<div align = "center">
<a href = "http://www.plunk.org/~hatch/HyperbolicTesselations/">
<img src = "7_3.gif">
% </a>
</div>


18) Don Hatch, Hyperbolic planar tesselations, 
<a href = "http://www.plunk.org/~hatch/HyperbolicTesselations/">http://www.plunk.org/~hatch/HyperbolicTesselations/</a>

But Crane and Sheppeard weren't just interested in beautiful geometry.
They developed their example as part of an attempt to build new
"spin foam models" in 4 dimensions.  I've talked about such
models on and off for many years here.  The models I've discussed were
usually based on representations of groups or quantum groups.  Now we
can build models using 2-groups, taking advantage of the fact that we
have not just representations and intertwiners, but also
2-intertwiners.  You can think of these models as discretized path
integrals for gauge theories with a "gauge 2-group".  To
compute the path integral you take a 4-manifold, triangulate it, and
label the edges by representations, the triangles by intertwiners, and
the tetrahedra by 2-intertwiners.  Then you compute a number for each
4-simplex, multiply all these numbers together, and sum the result
over labellings.

Baratin and Freidel have done a lot of interesting computations 
in the Crane-Sheppeard model.  I hope they publish their results
sometime soon.

To wrap up, I'd like to make a few technical remarks about group
representation theory and measurable fields of Hilbert spaces.  In
"<a href = "week272.html">week272</A>" I talked about a
class of measurable spaces called standard Borel spaces.  Their
definition was frighteningly general: any measurable space X whose
measurable subsets are the Borel sets for some complete separable
metric on X is called a "standard Borel space".  But then I
described a theorem saying these are all either countable or
isomorphic to the real line!  They are, in short, the "nice"
measurable spaces - the ones we should content ourselves with studying.

In our work on 2-group representations, we always assume our 
measurable spaces are standard Borel spaces.  We need this to get
things done.  But standard Borel spaces also show up ordinary group 
representation theory.  Let me explain how!

To keep your eyes from glazing over, I'll write "rep" to
mean a strongly continuous unitary representation of a topological
group on a separable Hilbert space.  And, I'll call an irreducible one
of these guys an "irrep".

Mackey wanted to build all the reps of a topological group G starting
from irreps.  This will only work if G is nice.  Since Haar measure is
a crucial tool, he assumed G was locally compact and Hausdorff.  Since
he wanted L^{2}(G) to be separable, he also assumed G was
second countable.
 
For a group with all these properties - called an "lcsc
group" by specialists wearing white lab coats and big horn-rimmed
glasses - Mackey was able to construct a measure space G* called the
"unitary dual" of G.

The idea is simple: the points of G* are isomorphism classes of 
irreps of G.   But let's think about some special cases....

When G is a finite group, G* is a finite set.  

When G is abelian group, not necessarily finite, G* is again an
abelian group, called the "Pontryagin dual" of G.  I talked
about this a lot in "<a href = "week273.html">week273</A>".

When G is both finite and abelian, so of course is G*. 

But the tricky case is the general case, where G can be infinite and
nonabelian!  Here Mackey described a procedure which is a grand
generalization of writing a rep as a direct sum of irreps.

If we choose a sigma-finite measure \mu  on G* and a measurable field
H_{x} of Hilbert spaces on G*, we can build a rep of G.
Here's how.  Each point x of G* gives an irrep of G, say
R_{x}.  These form another measurable field of Hilbert spaces
on G*.  So, we can tensor H_{x} and R_{x}, and then
form the "direct integral"

&int;_{x}  (H_{x} \otimes  R_{x}) d\mu (x)

As I already mentioned, a direct integral is a generalization of a
direct sum.  The result of doing this direct integral is a Hilbert
space, and in this case it's a rep of G.  The Hilbert spaces
H_{x} specify the "multiplicity" of each irrep
R_{x} in the representation we are building.

The big question is whether we get \emph{all} the reps of G this way.

And the amazing answer, due to James Glimm, is: yes, <i>if G* is a
standard Borel space!</i>

In this case we say G is "type I".  People know lots of
examples.  For example, an lcsc group will be type I if it's compact,
or abelian, or a connected real algebraic group, or a connected
nilpotent Lie group.  That covers a lot of ground.  However, there are
plenty of groups, even Lie groups, that aren't type I.  The
representation theory of these is more tricky!

If you want to know more, either read Mackey's book listed above, 
or this summary:

19) George W. Mackey, Infinite-dimensional group representations,
Bull. Amer. Math. Soc. 69 (1963), 628-686.  Available from Project
Euclid at <a href =
"http://projecteuclid.org/euclid.bams/1183525453">http://projecteuclid.org/euclid.bams/1183525453</a>

\par\noindent\rule{\textwidth}{0.4pt}

<em>The most fascinating thing about algebra and geometry is the way 
they struggle to help each other to emerge from the chaos of non-being, 
from those dark depths of subconscious where all roots of intellectual 
creativity reside.</em> - <a href = "http://arxiv.org/abs/math.AG/0201005">Yuri
Manin</a>

\par\noindent\rule{\textwidth}{0.4pt}

% </A>
% </A>
% </A>
