
% </A>
% </A>
% </A>
\week{May 18, 1995 }

Near the end of April I was invited by Ronnie Brown to 
Bangor, Wales for a very exciting get-together.  Readers of
"This Week's Finds" will know I'm interested in n-categories
and higher-dimensional algebra these days.  Brown is
the originator of the term "higher-dimensional algebra"
and has been sort of a prophet of the subject for many years.
Tim Porter at Bangor also works on this subject; 
I'll try to say a bit more about his stuff next week.  And visiting
Bangor at the time were John Power and Ross Street, two category 
theorists who do a lot of work on n-categories.   So I had a chance 
to learn some more higher-dimensional algebra and category theory
and see what these folks thought of my crazy ideas.

1) Ronald Brown, Out of line, Royal Institution Proceedings
64, 207-243.


Brown is very interested in explaining mathematics to the
public, and this paper is based on a talk he gave to a 
general audience.  It is a very accessible introduction to what
mathematics is really all about, and what higher-dimensional
algebra is about in particular.   "Out of line" is a pun, of
course, because not only does higher-dimensional algebra seek
to burst free of certain habits of "linear thinking" that tend
to go along with symbol string manipulation, it also has been
a bit outside the mainstream of mathematics until recently.  

Now, when I speak of "linear thinking" I am not indulging in
some vague new-agey complaint against rationality.  I mean
something very precise: the tendency to focus ones energy
on operations that are easily modelled by the juxtaposition
of symbols in a line.  The primordial example is addition:
we have a bunch of sticks in a row:


\begin{verbatim}

                        |||||
\end{verbatim}
    

and we say there are "5" sticks and write


\begin{verbatim}

                     1+1+1+1+1 = 5.
\end{verbatim}
    

Fine.  But when we have a 2-dimensional array of sticks:


\begin{verbatim}

                        ||||
                        ||||
\end{verbatim}
    
  
we are in a hurry to bring the situation to linear form by making
up a new operation, "multiplication", and saying we have 2 x 4
sticks.  This isn't so bad for plenty of purposes; it's not as
if I'm against times tables!   But certain things, particular in topology, 
can get obscured by neglecting operations that correspond most
naturally to higher-dimensional forms of juxtaposition, and Brown's 
paper explains some of these, and how to deal with these problems.  
The point is not to avoid linear notation; it's to avoid falling 
into certain mental traps it can lead you into if you're not careful!
  
2) A. J. Power, Why tricategories?, preprint available
as ECS-LFCS-94-289 from Laboratory for Foundations of Computer
Science, University of Edinburgh.  Also available at
<A HREF = "http://www.ima.umn.edu/talks/workshops/SP6.7-18.04/power/power.pdf">http://www.ima.umn.edu/talks/workshops/SP6.7-18.04/power/power.pdf</A>


When I mentioned this paper to a friend, she puzzledly asked
"`Why try categories?'?"  And indeed, one must have tried
categories and enjoyed them before moving on to bicategories,
tricategories and that great beckoning terra incognita of
mathematics, n-category theory.

In a sense I already know "why tricategories".  I think they're 
great, and in a paper with James Dolan --- summarized in "<A HREF = "week49.html">week49</A>" --- 
I did my best to get everyone else interested in general 
n-categories.  For me, the great thing about n-category theory 
is that it strives to formalize the notion of "process" in 
all its recursive splendor.   An n-category is a mathematical
structure containing not only objects, which one might think of as 
"things", and morphisms, which one might think of as "processes leading
from one thing to another", but also 2-morphisms, which are
"processes leading from one process to another", and then
3-morphisms, etc., on up to n-morphisms.   

In physics and topology applications, the k-morphisms can often be 
thought of as k-dimensional geometrical objects, since (as the Greeks knew)
the process of motion of a point traces out a 1-dimensional figure, 
and similarly the motion of a 1-dimensional figure traces out
a 2-dimensional surface... and n-dimensional spacetime is in some
rough sense "traced out" by the motion of (n-1)-dimensional spacelike slices 
through time.  If you think this is vague, you're right --- and
that's why one needs n-category theory, to make it precise!   When 
one understands n-categories (which so far we really do only
up to n = 3) one sees that the possibilities inherent in n-dimensional
topology match up very nicely with one might have stumbled on
not knowing topology at all, but just playing around with this
iterated notion of processes between processes between processes...
This "natural correspondence" between purely algebraic
concepts and topological ones is what makes topological quantum field 
theory tick, and I can't help but feel that it will have quite
a bit to say about physics eventually.

Power, however, gives a quite different set of reasons for being
interested in tricategories.  These are drawn from computer science
and logic, and they make me realize yet again how poor and outdated
my education in logic was, and how much interesting stuff there is
going on in the subject! 

At a completely naive level, one might already expect that
relation between "processes" and "things" is subtle and interesting
in computer science.  For after all, we can think of a program
either as a process that turns one thing into another, or as
data, a sort of thing, which other programs can act on.  Power does
not really emphasize this issue explicitly, but I can't help
remarking on it, especially because it reminds me of the curious fact 
that in mathematical physics, "time is the last dimension".  

That is, in topological quantum field theory, the n-morphisms in an n-category, 
which are the processes having no further processes going between them, 
represent the passage of time.  And indeed, for many practical purposes 
it appears that n = 4 is where things leave off, since spacetime appears 
4-dimensional.  On the other hand, nobody knows any mathematical reason why 
one has to stop at any given n.  So ultimately we should try to 
understand "\omega -categories", having n-morphisms for all n greater than or
equal to zero (0-morphisms being simply objects, and 1-morphisms
being morphisms).  This corresponds philosophically to the notion
that every "process" can also be regarded as a "thing" which
other processes can transform.  Moreover, we should also try to understand 
"Z-categories", having n-morphisms for all integers n, even negative ones!  
In this world, where there is no bottom as well as no top, every "thing" 
can also be regarded as a "process".  

But I digress.  Power is actually more interested in a different
(though perhaps related) hierarchy.   Sometimes people like to 
say computers just do stuff with bunches of numbers, but that's 
pretty misleading.  Of course computers \emph{can} do things with numbers,
but that's one of the simpler mathematical things they can do.
A number is an element of a set (the set of real numbers, or some
set of more computer-manageable numbers.)   And computers
have no problem dealing with elements of sets.  But computers can also 
deal with sets themselves --- and more fancy mathematical
objects.

Many mathematical objects are sets, or bunches of sets, equipped with 
operations satisfying equational laws.  For example, a group is a set equipped 
with a product and inverse operation satisfying various laws.   Sometimes
these operations are only defined if certain conditions hold, of course.
For example, a category is a set of "objects" and a set of "morphisms",
together with various operations like composition of morphisms, but one i
can only compose two morphisms f: x \to  y and g: w \to  z if y = w.  
Other examples might include graphs, partially ordered sets... and all 
sorts of things computer scientists know and love.

We could call all of these "sets with essentially algebraic structure." 
Mathematically sophisticated computer scientists want to be able 
define data types corresponding to arbitrary sorts of sets with 
essentially algebraic structure, and to play around with them easily.  
So they need to ponder such things in considerable generality.

Note that in all cases, there is not just a bunch of objects to play with ---
like "groups" or "partially ordered sets" --- but a \emph{category} in which 
the morphisms are structure-preserving maps between the objects in question.  
For example, there is a category Grp whose objects are groups and whose 
morphisms are group homomorphisms.  

The categories one gets this way are of a certain sort.  Power calls 
them "categories of models of finite limit theories".  To define this 
requires a bit of know-how, but it's basically simple.  For example, suppose
I were trying to explain the definition of a category to a computer
scientist; I might say, every category has a set ob of objects
and a set mor of morphisms; every morphism has an object called
its source (or domain), so there is a function


$$

                  source: mor \to  ob
$$
    

and similarly every morphism has an object called its target
(or codomain) so there is a function


$$

                  target: mor \to  ob
$$
    

Now, we can compose a morphism f and a morphism g to get fg
if target(f) = source(g), so we have a composition function


$$

              composition: C \to  mor
$$
    

defined only on the subset C of mor x mor that is the biggest
subset making the following diagram commute:


\begin{verbatim}

                   C  ------p1----> mor
                   |                 |             p1(f,g) = f
                  p2               target          p2(f,g) = g
                   |                 |
                   V                 V
                  mor ----source---> ob 
\end{verbatim}
    

Now category theorists have a slick way of dealing with these
functions defined only a subset satisfying equational conditions;
instead of talking about the "biggest subset" C they would say
that C is the "limit" of the diagram 


\begin{verbatim}

                                 mor
                                  |
                               target
                                  |
                                  V
                  mor -source---> ob 
\end{verbatim}
    

If you don't get this, don't worry; in a sense it's just another way 
of talking about the same thing, with the advantage of being 
infinitely more general, since one can talk about the limit of 
any diagram, though here we will only need limits of \emph{finite}
diagrams.

Then, after having lined up these ingredients (and I have left some 
out!), I could go ahead and say what equational laws they need 
to satisfy, like associativity of composition; and if I wanted I 
could write all these laws out using commutative diagrams, too!  
Then I would have laid out the "theory of categories" --- 
a complete description of the operations in a category and the laws they obey.  

The theory of categories is a typical example of a "finite limit 
theory", because what I really did above, in describing
the "theory of categories", is describe a CATEGORY, say Th, having
objects called ob and mor, and morphisms called source, target, composition,
and so on, such that various diagrams commute!   Moreover, we should
think of Th as a category with all finite limits, that is, one in which
all finite diagrams have limits.  That allows us to deal with things
like the object C above, which are defined as limits of finite
diagrams.

So we have this thing Th, the "theory of categories".  And then, 
any \emph{particular} category is a "model" of this theory Th.
A "model" assigns to each object in Th a particular
set --- for example, "mor" above gets assigned
the set of morphisms in some particular category C --- and assigns to each
morphism in Th a particular function --- for example,  
"composition" above gets assigned the function representing composition
in C.  Moreover, this assignment satisfies a bunch of utterly obvious 
consistency conditions which one summarizes by saying that a 
"model of the theory Th is a functor from Th to Set that preserves 
finite limits".  In logic, you know, a model of a theory is something 
that assigns to each thingie in the theory an actual thingie, in such a 
way that all the stuff the theory SAYS is true about these thingies, 
IS true!

Now if you are with me thus far you either know this stuff better
than I do, or else I congratulate you, because the example I 
picked was deliberately self-referential and confusing --- I was
using category theory to describe the theory of categories, and also,
the theory Th itself was a category!  But the world of thought does have
a funny way of wrapping back on itself like that... so it's good
to get used to it.

In fact there is a big literature on "sets with essentially
algebraic structure" and "categories of models of finite limit
theories"... this is a branch of logic they never taught me about 
in school, but it definitely exists, and Power gives some references
to it:

3) P. Gabriel and F. Ulmer, Lokal praesentierbare Kategorien, in
Springer Lecture Notes in Math 221 (1971).

G. Kelly, Structures defined by finite limits in the enriched
context I, Cahiers de Top. et. Geom. Diff. 23 (1982), 3-41.

Michael Makkai and Robert Pare, Accessible categories: the foundations
of categorical model theory, in Contemp. Math. 104 (1989).


But let's dig in a bit further, since really the fun is just 
starting.  Now, I told you what a model of one of these finite
limit theories Th was, but not what a morphism between models
is!  Well, if a model is a sort of functor, a morphism between
them should be a sort of natural transformation between functors; 
that's how it usually goes.  So there is really a category
Mod(Th) of models of one of these theories Th.  If Th were the
theory of categories as above, Mod(Th) would be the category of 
(small) categories, which is called Cat.  To take a less fiendish 
example, if Th were the theory of groups, Mod(Th) would the category 
Grp.

But now suppose one wanted to build a computer language that could
not only deal with all sorts of data types corresponding to different
"sets with essentially algebraic structure", but also various 
"categories with essentially algebraic structure".  For if
one likes category theory well enough to do a lot of computer
science using it, it makes sense to let the computer itself
join the fun, by creating a language in which it's easy to 
talk about categories.  After all, our eventual goal with computers 
is to have them completely replace computer scientists, right?  

Well, in a way "categories with essentially algebraic
structure" aren't terribly different from sets with essentially 
algebraic structure.  Roughly, the idea is that instead of having a
data type that consists of a bunch of sets with functions between
them satisfying some equational laws, we have a data type consisting
of a bunch of categories, functors between them, and natural
transformations between THEM satisfying equational laws.  
What this means is that if we try to copy the above
stuff, instead of a "theory" we will have a "2-theory" Th, which
is some sort of 2-category, and then a model of this would be
a 2-functor from Th to Cat.  We want to wind up getting a 2-category
Mod(Th) of models of Th.   

But actually carrying this out is a bit tricky, and much of Power's
paper goes into the details of various proposed schemes.  Of
course there is no reason in principle to stop here, other
than our limited understanding of n-categories, sheer bewilderment,
or boredom.  Reasoning about n-categories always tends to
drag in (n+1)-categories, because the collection of all n-categories with 
some particular structure (such as the "essentially algebraic structures" 
I've focussed on here, but also other sorts) typically forms an 
(n+1)-category.  This is how Power motivates tricategories.  
Right now we are stuck at n = 3, but there are good 
reasons to expect that pretty soon we'll go beyond that.  In fact, 
Power and Street showed me a sketch of their ideas on tetracategories....
\par\noindent\rule{\textwidth}{0.4pt}

% </A>
% </A>
% </A>
