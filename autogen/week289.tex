
% </A>
% </A>
% </A>
\week{January 8, 2010 }

This week I'll tell you some news about E_{8}.  Then I'll continue
expanding the grand analogy between different kinds of physics.  We'll
get into a bit of thermodynamics - and chemistry too!  And then we'll
continue our exploration of rational homotopy theory, this time
entering the world of "differential graded Lie algebras".

But first: what's going on here? 

<div align = "center">
<img border = "2" src = "cryptic_terrain.jpg">
</div>

As usual, the answer is at the end.  

Now for the news:

Hurrah!  Yippee-yay!  They've discovered the exceptional group
E_{8} in nature!  And they've found the golden ratio lurking
in a quantum system!  At least, that's what the headlines are blaring:

1) Joe Palca, Finding 'beautiful' symmetry near absolute zero,
All Things Considered, National Public Radio, 
<a href = "http://www.npr.org/templates/story/story.php?storyId=122365883">http://www.npr.org/templates/story/story.php?storyId=122365883</a>

Here's the paper that started all the fuss:

2) R. Coldea, D. A. Tennant, E. M. Wheeler, E. Wawrzynska,
D. Prabhakaran, M. Telling, K. Habicht, P. Smeibidl and K. Kiefer,
Quantum criticality in an Ising chain: experimental evidence for
emergent E_{8} symmetry, Science (327), 177 - 180.  Available at
<a href = "http://www.sciencemag.org/cgi/content/abstract/327/5962/177">http://www.sciencemag.org/cgi/content/abstract/327/5962/177</a>

Unfortunately you can only see the paper if you subscribe to Science
or have magical superpowers.  And I don't understand the details,
which apparently go back to a theoretical paper by the guy who
invented the "Zamolodchikov tetrahedron equation",
much beloved by people who study knotted surfaces in 4 dimensions:

3) A. B. Zamolodchikov, Integrals of motion and S-matrix of the
(scaled) T=T_{c} Ising model with magnetic field, Int. J. Mod. Phys. A 
4 (1989), 4235.

But the rough idea is this.  You can simulate a 1-dimensional magnet
using a chemical called cobalt niobate.  The molecules form a chain,
and each one has a spin, and the spins want to line up: either all
pointing up, or all pointing down.  And at zero temperature, that's
what they'd do.

At any slightly higher temperature, random fluctuations spoil things.
This is different than in higher dimensions, where a single atom whose
spin points the wrong way will be surrounded by conformists on all
sides and eventually fall into line, as long as it's cool enough.  In
one dimension, once one spin flips, its neighbors don't know which
side to go with: the guy on their left or the guy on the right.
Rebellious provinces of flipped spins can grow every larger without an
ever-lengthening frontier that takes a lot of energy to maintain!

But at temperatures close to zero, we'll see big stretches of spins
pointing the same way, like this:

<div align = "center">
&darr;&darr;&darr;&darr;&darr;&darr;&darr;&darr;&darr;&darr;&darr;&darr;&darr;&darr;&darr;&darr;&darr;&darr;&darr;&darr;&uarr;&uarr;&uarr;&uarr;&uarr;&uarr;&uarr;&uarr;&uarr;&uarr;&uarr;&uarr;&uarr;&uarr;&uarr;&uarr;&uarr;&uarr;&uarr;&uarr;&darr;&darr;&darr;&darr;&darr;&darr;&darr;&darr;&darr;&darr;&darr;&darr;&darr;&darr;&darr;&darr;&darr;
</div>

The points at which the spin flips can move around like particles.  
There are two kinds of particles like this, which we could call "kinks":

<div align = "center">
&darr;&darr;&darr;&darr;&darr;&darr;&darr;&darr;&darr;&darr;&darr;&darr;&darr;&darr;&darr;&darr;&darr;&darr;&darr;&darr;&uarr;&uarr;&uarr;&uarr;&uarr;&uarr;&uarr;&uarr;&uarr;&uarr;&uarr;&uarr;&uarr;&uarr;&uarr;&uarr;&uarr;&uarr;&uarr;&uarr;
</div>

and "antikinks":

<div align = "center">
&uarr;&uarr;&uarr;&uarr;&uarr;&uarr;&uarr;&uarr;&uarr;&uarr;&uarr;&uarr;&uarr;&uarr;&uarr;&uarr;&uarr;&uarr;&uarr;&uarr;&darr;&darr;&darr;&darr;&darr;&darr;&darr;&darr;&darr;&darr;&darr;&darr;&darr;&darr;&darr;&darr;&darr;&darr;&darr;&darr;
</div>

When a kink meets an antikink, they can annihilate each other.  In the following picture, each line is a moment of time; as time proceeds we march down the
page.  We see a kink moving right and an antikink moving left.  When they
collide, they annihilate!

<div align = "center">
&darr;&darr;&darr;&darr;&darr;&darr;&darr;&darr;&darr;&darr;&darr;&darr;&darr;&darr;&darr;&darr;&darr;&uarr;&uarr;&uarr;&uarr;&uarr;&uarr;&uarr;&darr;&darr;&darr;&darr;&darr;&darr;&darr;&darr;&darr;&darr;&darr;&darr;&darr;&darr;&darr;&darr; <br/>
&darr;&darr;&darr;&darr;&darr;&darr;&darr;&darr;&darr;&darr;&darr;&darr;&darr;&darr;&darr;&darr;&darr;&darr;&uarr;&uarr;&uarr;&uarr;&uarr;&darr;&darr;&darr;&darr;&darr;&darr;&darr;&darr;&darr;&darr;&darr;&darr;&darr;&darr;&darr;&darr;&darr; <br/>
&darr;&darr;&darr;&darr;&darr;&darr;&darr;&darr;&darr;&darr;&darr;&darr;&darr;&darr;&darr;&darr;&darr;&darr;&darr;&uarr;&uarr;&uarr;&darr;&darr;&darr;&darr;&darr;&darr;&darr;&darr;&darr;&darr;&darr;&darr;&darr;&darr;&darr;&darr;&darr;&darr; <br/>
&darr;&darr;&darr;&darr;&darr;&darr;&darr;&darr;&darr;&darr;&darr;&darr;&darr;&darr;&darr;&darr;&darr;&darr;&darr;&darr;&uarr;&darr;&darr;&darr;&darr;&darr;&darr;&darr;&darr;&darr;&darr;&darr;&darr;&darr;&darr;&darr;&darr;&darr;&darr;&darr; <br/>
&darr;&darr;&darr;&darr;&darr;&darr;&darr;&darr;&darr;&darr;&darr;&darr;&darr;&darr;&darr;&darr;&darr;&darr;&darr;&darr;&darr;&darr;&darr;&darr;&darr;&darr;&darr;&darr;&darr;&darr;&darr;&darr;&darr;&darr;&darr;&darr;&darr;&darr;&darr;&darr; <br/>
</div>

Kinks and antikinks can also be created in pairs.

So far this is pretty simple.  
But next thing this experiment did is turn on a "transverse magnetic 
field" - in other words, a field at <em>right
angles</em> to the direction I'm drawing as vertical.  If the field is
strong enough, most of the spins will line up in this new direction:

<div align = "center">
\to \to \to \to \to \to \to \to \to \to \to \to \to \to \to \to \to \to \to \to \to <br/>
</div>

But at nonzero temperature, random thermal fluctuations will make
a few spins flip and point the wrong way.  And these wrongheaded spins
can again move around like particles:

<div align = "center">
\to \to \to \to \to &larr;\to \to \to \to \to \to \to \to \to \to \to \to \to \to \to <br/>
\to \to \to \to \to \to &larr;\to \to \to \to \to \to \to \to \to \to \to \to \to \to <br/>
\to \to \to \to \to \to \to &larr;\to \to \to \to \to \to \to \to \to \to \to \to \to <br/>
\to \to \to \to \to \to \to \to &larr;\to \to \to \to \to \to \to \to \to \to \to \to <br/>
\to \to \to \to \to \to \to \to \to &larr;\to \to \to \to \to \to \to \to \to \to \to <br/>
\to \to \to \to \to \to \to \to \to \to &larr;\to \to \to \to \to \to \to \to \to \to <br/>
</div>

So, this is what we see if the transverse magnetic field is strong
enough.  On the other hand, if the field is small, it won't have much
effect.  But right in between, at some critical value, some very
complicated things can happen.  This is called a "critical point".

The people doing the experiment found this critical point.  And then
they did something really tricky.  They turned on an <i>extra</i>
magnetic field pointing in the direction that I'm drawing as
<i>vertical</i>.  Under these conditions, Zamolodchikov claimed that
something amazing should happen.  Namely: a kink and an antikink
should be able to stick together in 8 different ways!

When two or more particles stick together and form a new one, we call
the result a "<a href =
"http://en.wikipedia.org/wiki/Bound_state">bound state</a>".  So,
we should get 8 different kinds of bound states, which can zip along
our spin chain like particles!  I wish I could draw them, but I don't
know how.

If you know particle physics, these bound states should remind you a
bit of mesons.  A <a href =
"http://en.wikipedia.org/wiki/Meson">meson</a> is a bound state made
of a <a href = "http://en.wikipedia.org/wiki/Quark">quark</a> and an
<a href = "http://en.wikipedia.org/wiki/Antiquark">antiquark</a>.  In
fact there are even 8 kinds of mesons made from up, down and strange
quarks and their corresponding antiquarks.  That's why Gell-Mann
called his theory of quarks the "<a href =
"http://en.wikipedia.org/wiki/Eightfold_Way_%28physics%29">Eightfold
Way</a>" when he came up with it back around 1961.  In this
theory, the number 8 shows up because the relevant symmetry group,
called SU(3), is 8-dimensional.

However, the math surrounding these 8 kink-antikink bound states is a
lot more sophisticated.  It's related to the exceptional Lie group
E_{8}, which is 248-dimensional!

Well, actually it's not quite that bad.  What really matters in this
game is not the group E_{8} but rather its "root
lattice", which is just 8-dimensional.  This is the pattern you
get when you pack equal-sized balls in 8 dimensions in the unique way
such that each ball touches the maximum number of others - namely,
248.  I sort of understand this pattern, and I explained it in back in
"<a href = "week193.html">week193</a>".  But I don't understand
why it shows up when you're studying a chain of spins in a magnetic
field!

What did the experiment actually measure?  Among other things,
they measured the ratio of masses of two of the 8 "particles"
formed as kink-antikink bound states - namely, the lightest two.
According to Zamolodchikov's calculation, it should be the golden
ratio!  You know:

\Phi  = (1 + \sqrt 5)/ 2

And that's what the experiment saw!

Now, I know plenty of relationships between E_{8}, and the
golden ratio - see "<a href =
"http://math.ucr.edu/home/baez/week270.html">week270</a>" - so
this connection doesn't shock me.  But I sure hope someone explains
the details!

If you're an amateur looking for a few clues, I suggest starting with
this very readable expository paper:

4) Paul A. Pearce, Phase transitions, critical phenomena and exactly
solvable lattice models.  Also available at 
<a href = "http://mac0916.ms.unimelb.edu.au/~pap/Publications_pdf/1997Pearce_VicRoyalSociety.pdf">http://mac0916.ms.unimelb.edu.au/~pap/Publications_pdf/1997Pearce_VicRoyalSociety.pdf</a>.

You'll notice he talks about various models of <i>2-dimensional</i> magnets,
leading up to a model that involves something called the E_{8}
Dynkin diagram:


\begin{verbatim}

   o----o---o----o----o----o----o
                      |
                      |
                      o
\end{verbatim}
    
which encodes the structure of the E_{8} lattice.  
These 2-dimensional magnets are related to the 1-dimensional
magnetic "spin chain" we've been discussing - but in a sneaky
way.  The 1-dimensional spin chain is 2-dimensional if we think of time
as an extra dimension!  Indeed I've already been drawing some 
2-dimensional pictures, like the picture of a kink colliding with an
antikink.   So, that's part of the story.  

But the story is much deeper - and for this, I really must thank 
Will Orrick, who also caught some mistakes in an earlier version
of my story here!  Orrick is a mathematician at Indiana University
who works on statistical mechanics and combinatorics.

For starters, at a critical point, a 2-dimensional magnet is related
to a kind of quantum field theory called a "conformal field
theory".  And the particular conformal theory this experiment is
studying is a so-called "minimal model" by the name of
M(3,4).  This conformal field theory can be built from E_{8}
using something called a "coset construction" - but it does
not have 8 bound states.  To see those, we really need to turn on that
<i>extra</i> magnetic field I mentioned: the field pointing in the
vertical direction.  This gives an "integrable massive
perturbation" of the conformal field theory.  That's what we need
to understand to see those 8 bound states, and compute their masses.

If you know nothing of conformal field theory, minimal models and the
coset construction, it can't hurt to look at my outline of Di
Francesco, Mathieu, and Senechal's book in "<a href =
"week124.html">week124</a>".  To dig a bit deeper, try:

5) Scholarpedia, A-D-E classification of conformal field theories,
<a href = "http://www.scholarpedia.org/article/Cappelli-Itzykson-Zuber_A-D-E_Classification">http://www.scholarpedia.org/article/Cappelli-Itzykson-Zuber_A-D-E_Classification</a>
 
And for more details on how the extra magnetic field creates those 8
bound states, try these:

6) Giuseppe Mussardo, Off-critical statistical models: factorized
scattering theories and bootstrap program, Physics Reports 218
(1992), 215-379.  

7) Giuseppe Mussardo, Statistical Field Theory, Oxford, 2010.

Okay, back down to earth.  Last week I began to sketch an analogy
between various kinds of physical systems, based on general concepts
of "displacement" and "momentum", and their time
derivatives: "flow" and "effort":



\begin{verbatim}


                displacement    flow          momentum      effort
                     q           q'              p            p'

Mechanics       position       velocity       momentum      force
(translation)

Mechanics       angle          angular        angular       torque
(rotation)                     velocity       momentum

Electronics     charge         current        flux          voltage
                                              linkage

Hydraulics      volume         flow           pressure      pressure
                                              momentum

\end{verbatim}
    

Today I want to make this chart even bigger!  There are more systems that
fit into this collection of analogies.

For a really good analogy, we want "effort" times
"flow" to have dimensions of power - that is, energy per
time.  Indeed, we want it to be true that:

pq   has dimensions of action (= energy \times  time) <br/>
p'q  has dimensions of energy <br/>
pq'  has dimensions of energy <br/>
p'q' has dimensions of power (= energy / time) <br/>


 If any one of these is true, they're all true.  And they're true
in the four examples I've listed so far.  

For example, suppose we have a circuit with one wire coming in and one
going out, and a complicated black box in the middle.  Then at any
given time, the power it takes to run this circuit equals the voltage
across the circuit times the current flowing through it.  That's
effort times flow.

Note the wording here.  Engineers say that voltage is an "across"
variable, while current is a "through" variable.  

I hope the idea of current flowing "through" a circuit
is reasonably intuitive: think of water flowing through a pipe.  But
the idea of voltage "across" a circuit may be a bit less
intuitive.  Crudely speaking, at any point of spacetime there's a
number called the "voltage".  And at any given time, the
voltage "across" our circuit is the voltage on the wire
coming in, minus the voltage on the wire coming out.

To be a bit less crude, it's important to note that only
\emph{differences} between voltages are measurable:

4) John Baez, Torsors made easy, <a href = "http://math.ucr.edu/home/baez/torsors.html">http://math.ucr.edu/home/baez/torsors.html</a>

But the voltage across a circuit is precisely such a difference.

Anyway, what are some other examples of physical systems where we
have a notion of "effort" and a notion of "flow",
such that effort times flow equals power?

Here are two:



\begin{verbatim}


                displacement    flow         momentum      effort
                     q           q'              p           p'

Thermodynamics  entropy        entropy       temperature   temperature
                               flow          momentum

Chemistry       moles          molar         chemical      chemical
                               flow          momentum      potential

\end{verbatim}
    

I made up the phrases "temperature momentum" and
"chemical momentum" since these quantities don't have
standard names, as far as I know.  But that's not so important.  What
really matters is that we've brought two more subjects into our circle
of analogies.

The example of thermodynamics works like this.  Say you have a
physical system in thermal equilibrium and all you can do is heat it
up or cool it down "reversibly" - that is, while keeping
it in thermal equilibrium all along.  For example, imagine a box of gas
that you can heat up or cool down.  If you put a tiny amount dE of
energy into the system in the form of heat, then its <a href =
"http://en.wikipedia.org/wiki/Entropy">entropy</a> increases by a tiny
amount dS.  And it works like this:

dE = TdS

where T is the temperature.  

Another way to say this is

dE/dt = T dS/dt

where t is time.  On the left we have the power put into the system in
the form of heat.  But since power should be "effort" times
"flow", on the right we should have "effort" times
"flow".  It makes some sense to call dS/dt the "entropy
flow".  So temperature, T, must play the role of
"effort".

This is a bit weird.  I don't usually think of temperature as a form
of "effort" analogous to force or torque.  Stranger still,
our analogy says that "effort" should be the time derivative
of some kind of "momentum".  So, we need to introduce
"temperature momentum": namely, the integral of temperature
over time.  I've never seen people talk about this concept, so it
makes me nervous.

But when we have a more complicated physical system like a piston full
of gas in thermal equilibrium, we can see the analogy working.  Now
we have

dE = TdS - PdV

The change in energy dE of our gas now has two parts.  There's the
change in heat energy TdS, which we saw already.  But now there's
also the change in energy due to compressing the piston!  When we
change the volume of the gas by a tiny amount dV, we put in energy
-PdV.

Now look back at the first chart I drew!  It says that pressure is a
form of "effort", while volume is a form of
"displacement".  If you believe that, the equation above
should help convince you that temperature is also a form of
"effort", while entropy is a form of
"displacement".

But what about the minus sign?  That's no big deal: it's the result of
some arbitrary conventions.  P is defined to be the \emph{outwards}
pressure of the gas on our piston.  If this is \emph{positive},
\emph{reducing} the volume of the gas takes a \emph{positive}
amount of energy - so we need to stick in a minus sign.  I could
eliminate this minus sign by changing some conventions - but if I did,
the chemistry professors at UCR would haul me away and increase my
heat energy by burning me at the stake.

Speaking of chemistry: here's how we can extend our table of analogies
to include chemistry!  Suppose we have a piston full of gas made of
different kinds of molecules, and there can be chemical reactions that
change one kind into another.  Now our equation gets fancier:

dE = TdS - PdV + &sum;_{i} \mu _{i} dN_{i}

Here N_{i} is the number of molecules of the ith kind, while
\mu _{i} is a quantity called a "chemical potential".
The chemical potential simply says how much energy it takes to
increase the number of molecules of a given kind:

8) Wikipedia, Chemical potential, 
<a href = "http://en.wikipedia.org/wiki/Chemical_potential">http://en.wikipedia.org/wiki/Chemical_potential</a>

So, we see that "chemical potential" is another form of
"effort", while "number of molecules" is another
form of "displacement".

Chemists are too busy to count molecules one at a time, so they count
them in big bunches called "moles".  A <a href =
"http://en.wikipedia.org/wiki/Mole_%28unit%29">mole</a> is the number
of atoms in 12 grams of carbon-12.  That's roughly

<div align = "center">
                 602,214,150,000,000,000,000,000
</div>

atoms.  This is called <a href =
"http://en.wikipedia.org/wiki/Avogadro_constant">Avogadro's
number</a>.

So, instead of saying that the displacement in chemistry is called
"number of molecules", you'll sound more like an expert if
you say "moles".  And the corresponding flow is called
"molar flow".  I don't know a name for the thing whose time
derivative is chemical potential, so let's call it "chemical
momentum".

For more on this, try the following book on network theory:

9) Francois E. Cellier, Continuous System Modelling, Chap. 9: Modeling
chemical reaction kinetics, Springer, Berlin, 1991.

So, we've added two more items to our list of analogies: thermodynamics
and chemistry.  But, we've seen that they're intimately interlinked.

There are also weaker analogies to subjects where effort times flow
doesn't have dimensions of power.  The two most popular are these:


\begin{verbatim}

    
                 displacement    flow          momentum      effort
                      q           q'              p            p'

Heat Flow        heat            heat          temperature   temperature 
                                 flow          momentum

Economics        inventory       flow of       economic      price of
                                 product       momentum      product

\end{verbatim}
    

The heat flow analogy comes up because people like to think of heat
flow as analogous to electrical current, and temperature as analogous
to voltage.  Why?  Because an insulated wall acts a bit like a
resistor!  The current flowing through a resistor is a function the
voltage across it.  Similarly, the heat flowing through an insulated
wall is about proportional to the difference in temperature between
the inside and the outside.

However, at least according to most engineers, there's a big
difference.  Current times voltage has dimensions of power, which is
what we want.  Heat flow times temperature does not have dimensions of
power.  In fact, heat flow by itself already has dimensions of power!
So, engineers feel somewhat guilty about this analogy.

Being a mathematical physicist, a possible way out presents itself to
me: use units where temperature is dimensionless!  In fact such units
are pretty popular in some circles.  But I don't know if this solution
is a real one, or whether it causes some sort of trouble.

In the economic example, "energy" has been replaced by
"money".  So other words, "inventory" times
"price of product" has units of money.  And so does
"flow of product" times "economic momentum"!  I'd
never heard of "economic momentum" before, and I have
absolutely no intuition for it, but I didn't make it up.  It's the
thing whose time derivative is "price of product".

I'm suspicious of any attempt to make economics seem like physics.
Unlike elementary particles or rocks, people don't seem to be very
well modelled by simple differential equations.  However,
some economists have used the above analogy to model economic systems.
And I can't help but find that interesting - even if intellectually
dubious when taken too seriously.

Now... what can we do with all these analogies?  I'll explain that 
in detail in the Weeks to come.  But maybe you want a quick answer
now.  

First of all, engineers use these analogies to systematically model
all sorts of gadgets using "bond graphs".  Bond graphs were invented
by an engineer named Henry Paynter.  His original book goes way back
to 1961:

10) Henry M. Paynter, Analysis and Design of Engineering Systems, MIT
Press, Cambridge, Massachusetts, 1961.

I haven't gotten ahold of this book yet, but I've learned a bit about
Paynter.  He got a bachelor's degree in civil engineering, a master's in
mathematics, and then a doctorate in hydroelectric engineering, all
from MIT.  He then became a professor at MIT and taught there until he
retired in 1985.  I can easily imagine that this diverse background
made him the perfect guy to unify lots of different subjects.

I want to explain bond graphs, how they differ from circuit diagrams,
and how they're both examples of "string diagrams" in
category theory.  But it will take me a while to get there - since
while abstract generalities are always fun, this is a great
opportunity to talk about lots of basic physics.

In particular, you'll note how all these analogies rely on a pair of
variables q and p: displacement and momentum.  In classical mechanics
we call these "<a href =
"http://en.wikipedia.org/wiki/Conjugate_variables">conjugate
variables</a>".  The importance of such pairs is explained in the
"<a href =
"http://en.wikipedia.org/wiki/Hamiltonian_mechanics">Hamiltonian</a>"
approach to classical mechanics, which in turn leads to a branch of
math called "<a href =
"http://en.wikipedia.org/wiki/Symplectic_geometry">symplectic
geometry</a>".  So, I should try to explain a bit of that, though
probably just the basics.

One more thing.  If you've studied your physics, you've seen how
"<a href =
"http://en.wikipedia.org/wiki/Legendre_transform">Legendre
transforms</a>" show up in both classical mechanics and
thermodynamics.  The Legendre transform lets you start with a function
of q and q' and turn it into a function of q and p.  Mathematically,
the idea is that given a function on the tangent bundle of a manifold:

f: TM \to  R

you get a map from the tangent bundle to the cotangent bundle:

\lambda : TM \to  T*M

which records the derivative of f in the "vertical directions".
In nice cases, this map \lambda  is one-to-one and onto.

In classical mechanics, this lets us pass from the "Lagrangian"
formalism, where everything is a function of position and velocity, to
the "Hamiltonian" formalism, where everything is a function of
position and momentum.  The idea is that position and velocity (q,q')
are represented by a point in TM, while position and momentum (q,p)
are represented by a point in T*M.  In our discussion of analogies so
far, we considering the simplest case, where M is the real line.
That's why I've been treating q, p, q' and p' as mere \emph{numbers} that
depend on time.  But it's good to generalize to an arbitrary manifold M.

For an elementary yet insightful introduction to the physics of Legendre
transforms, try this:

11) R. K. P. Zia, Edward F. Redish and Susan R. McKay, Making sense of
the Legendre transform, available as <a href =
"http://arxiv.org/abs/0806.1147">arXiv:0806.1147</a>.


I've spent decades thinking about the Legendre transform in the
context of classical mechanics, but not so much in thermodynamics.  I
think its appearance in both subjects should be a big part of the
analogy I'm talking about here.  But if anyone knows a clear, detailed
treatment of the analogy between classical mechanics and
thermodynamics, focusing on the Legendre transform, please let me
know!

The above article helps a bit.  But it seems to be using a slightly
different analogy than the one I was just explaining... so my
confusion is not eliminated.

I'm also curious about lots of other things.  For example: in
classical mechanics it's really important that we can define 
"Poisson brackets" of smooth real-valued functions 
on the cotangent bundle.  So: how about in thermodynamics?  
Does anyone talk about the Poisson bracket of temperature and 
entropy, for example?

And Poisson brackets are related to quantization - see "<a href =
"week282.html">week282</a>" for more on that.  So: does anyone
try to quantize thermodynamics by taking seriously the analogies I've
described?  I'm not sure it makes physical sense, but it seems
mathematically possible.

These are just a few of the strange ways you can try to extend the
analogies I've listed.

Anyway, stay tuned for more on this.  But for now, let me turn to a
different story: rational homotopy theory!


\begin{verbatim}

                      RATIONAL SPACES
                         /      \  
                        /        \  
                       /          \  
                      /            \
                     /              \
      DIFFERENTIAL GRADED ------- DIFFERENTIAL GRADED
      COMMUTATIVE ALGEBRAS           LIE ALGEBRAS
\end{verbatim}
    

Last time I explained how we can turn a rational space into a
differential graded commutative algebra, or DGCA.  Now I want to tell
you how to turn a rational space into a differential graded Lie
algebra, or DGLA.

But first: why should we care?  

A differential graded Lie algebra is a generalization of a Lie
algebra.  Usually we get Lie algebras from Lie groups.  But now
we'll get one of these generalized Lie algebra from any rational
space.  

So, we're massively generalizing Lie theory!

This should seem odd at first.  It's easy to imagine generalizing Lie
theory from Lie groups to other groups, like "infinite-dimensional Lie
groups".  But how can we generalize it to \emph{spaces}?

The answer is this: there's a way to turn any pointed space X into a
topological group called \Omega (X).  Roughly, this is the group of
"based loops" in X: maps from an interval into X that start
and end at the basepoint.  There are some technicalities involved in
getting an honest group this way.  We'll talk about them later.  But
\emph{roughly}, the idea is that we multiply two loops by forming a new
loop that runs first along one and then the other.  And \emph{roughly},
the inverse of a loop is the same loop run backwards.

So here's the plan.  We're going to generalize Lie theory from Lie
groups to topological groups.  Just as a Lie group has a Lie algebra,
any topological group will have a "differential graded Lie algebra".
Whenever we have a pointed space X, we can turn it into a topological
group \Omega (X), and then apply this construction.

And when X is a \emph{rational} space, the resulting differential graded
algebra will know \emph{everything} about X!

Well, I shouldn't get carried away in my enthusiasm.  The differential
graded Lie algebra will only know everything about the "homotopy
type" of X - a concept I defined last week.  But that's still
amazing.  It means that at least for rational spaces, we can reduce
homotopy theory to a souped-up version of the theory of Lie algebras.

It's like a dream come true: reducing a largish chunk of homotopy
theory to linear algebra!

But now let's see how it works.

First of all, what's a differential graded Lie algebra?  It's a 
Lie algebra in the world of chain complexes.   A "chain complex",
for us, will be a list of vector spaces and linear maps


$$

     d       d      d
C_{0} <--- C_{1} <--- C_{2} <---
$$
    

with d^{2} = 0.  And a vector space, for us, will be vector
space over the rationals.

Just as you can tensor vector spaces, you can tensor chain complexes.
And just as you can define a Lie algebra to be a vector space V with
a bracket operation

[.,.] : V \otimes  V \to  V

satisfying antisymmetry and the Jacobi identity, so you can define a
"differential graded Lie algebra" to be a chain complex C with a
bracket operation

[.,.]: C \otimes  C \to  C

satisfying graded antisymmetry and the graded Jacobi identity.  By
"graded", I mean you need to remember to put in a sign
(-1)^{jk} whenever you switch a guy in C_{j} and a guy in 
C_{k}.

Differential graded Lie algebras are often called DGLAs for short.  A
DGLA where only C_{0} is nonzero is just a plain old Lie
algebra.  So, DGLAs really are a generalization of Lie algebras.
Whenever anyone tells you something about DGLAs, you should check to
see what it says about Lie algebras.

Next let me tell you how to turn our rational homotopy type X into a
DGLA.  I'll quickly sketch this process, which consists of 3 steps,
and then go over the steps more slowly.  Don't get scared if none
of them make sense yet:

<ul>
<li>
 Let \Omega (X) the space of based loops in X.  You should think 
 of this as a topological group, with the group operation being 
 concatenation of loops.  
</li>
<li>
 Let C_{<sub>*}</sub>(\Omega (X)) be the chain complex of
 singular chains on \Omega (X) taking values in the rational numbers.
 This is a differential graded cocommutative Hopf algebra, or
 "DGCHA" for short.
</li>
<li>
 Let P(C_{<sub>*}</sub>(\Omega (X))) consist of the
 "primitive" elements of our DGCHA.  This is a differential
 graded Lie algebra, or DGLA!  
</li> 
</ul>

Each step is interesting in itself.  And each step is actually a
functor.  So I need to explain 3 different functors:

\Omega : [path-connected pointed spaces] \to  [topological groups]

C_{<sub>*}</sub>: [topological groups] \to  [DGCHAs]

P: [DGCHAs] \to  [DGLAs]

One thing that excites me about this subject is getting to know the
last two functors.  I've been in love with the first one for years,
and also the functor going back:

B: [topological groups] \to  [path-connected pointed spaces]

which sends any topological group G to its "classifying space" BG.

Indeed, it was a life-changing experience to realize that as far as
homotopy theory goes, pointed path-connected spaces are just the same
as topological groups, thanks to these functors going back and forth.
Both these things seemed fundamental and fascinating: spaces and
symmetry groups!  To realize they were "the same" was mindblowing.

It's the next two steps that are exciting me now.  Let me try to
explain what simpler, perhaps more familiar constructions they
generalize.

If you have a plain old group G, it has a "<a href =
"http://en.wikipedia.org/wiki/Group_ring">group algebra</a>" Q[G]
consisting of formal rational linear combinations of elements of G.
Its multiplication comes from the multiplication in G.  But it's
better than an algebra: it's a "cocommutative <a href =
"http://en.wikipedia.org/wiki/Hopf_algebra">Hopf algebra</a>".
This means it has a bunch of extra operations that completely encode
the group structure on G.

For example, in a Hopf algebra you can "comultiply" as well
as multiply.  In the group algebra Q[G], the comultiplication map

\Delta : Q[G] \to  Q[G] \otimes  Q[G]

is defined on elements g of G by the equation

\Delta (g) = g \otimes  g

We say a Hopf algebra is "cocommutative" if comultiplying 
is the same as comultiplying and then switching the two outputs.
You can see that's true here.  

A Hopf algebra also has a "counit" as well as a unit, 
and the counit in a group algebra is a map

\epsilon  : Q[G] \to  Q 

defined by

\epsilon (g) = 1

In fact, given any cocommutative Hopf algebra, the elements satisfying
both of the above two equations form a group!  These elements are
called "<a href =
"http://planetmath.org/encyclopedia/GrouplikeElementsInHopfAlgebras.html">grouplike
elements</a>".  If we take the grouplike elements of Q[G], we get the
group G back.

The functor

C: [topological groups] \to  [DGCHAs]

generalizes this idea from groups to topological groups.  Instead of
just taking formal linear combinations of \emph{elements} of G, we
now take formal linear combinations of \emph{simplices} in G.  The
0-simplices in G are just elements of G.  But the higher-dimensional
simplices keep track of the topology of G.

Now let's turn to the next functor:

P: [DGCHAs] \to  [DGLAs]

This generalizes a simpler procedure that takes cocommutative Hopf
algebras and gives Lie algebras.

To understand this, it's best to think about the reverse procedure
first.  If you have a plain old Lie algebra L, it has a "<a href
=
"http://en.wikipedia.org/wiki/Universal_enveloping_algebra">universal
enveloping</a>" algebra UL.  This is the free associative algebra
on L mod relations saying that

xy - yx = [x,y]

for any x,y in L.

But UL is better than an algebra: it's a cocommutative Hopf algebra!
The point is that Lie algebras are a lot like groups, and \emph{both}
can be encoded in cocommutative Hopf algebras.  

In the universal enveloping algebra UL, comultiplication is a map

\Delta : UL \to  UL \otimes  UL

defined on elements x of L by the equation

\Delta (x) = x \otimes  1 + 1 \otimes  x

The counit is a map

\epsilon : UL \to  Q

defined by the equation

\epsilon (x) = 0

And conversely, given any cocommutative Hopf algebra, the elements
satisfying both these equations form a Lie algebra!  These elements
are called "primitive elements".  If we take the primitive
elements of UL, we get the Lie algebra L back.

Let's summarize this using a bit more jargon.  There's a
"universal enveloping algebra" functor:

U: [Lie algebras] \to  [cocommutative Hopf algebras]

and this has a right adjoint, the "primitive elements"
functor:

P: [cocommutative Hopf algebras] \to  [Lie algebras]

Even better, if L is any Lie algebra, P(UL) is isomorphic to L.  

Today we're generalizing all this to the world of chain complexes!  
There's a universal enveloping algebra for differential graded 
Lie algebras:

U: [DGLAs] \to  [DGCHAs]

and it has a right adjoint, the "primitive elements" functor:

P: [DGCHAs] \to  [DGLAs]

Even better, if L is any DGLA, P(UL) is isomorphic to L.  

So now I hope you understand the strategy for generalizing Lie theory
to rational spaces.  We can take any path-connected pointed space X and
form its group of loops:

\Omega : [path-connected pointed spaces] \to  [topological groups]

Then we can form a differential graded analogue of its group
algebra:

C: [topological groups] \to  [DGCHAs]

Finally, we can turn that into a differential graded Lie algebra:

P: [DGCHAs] \to  [DGLAs]

So, just as we could study a Lie group "infinitesimally" by looking at
its Lie algebra, we can now study any path-connected pointed space
"infinitesimally" by looking at the differential graded algebra Lie
algebra of its group of loops!  And for \emph{rational} spaces, this
"infinitesimal" description knows everything about the homotopy type
of our space.

This is probably a good place to stop if you just want the basic
idea.  But now I want to tell the tale of three functors in a bit
more detail.  There are some subtleties that are worth knowing if 
you want to be an expert on algebraic topology.  (I'm always hoping
someday I'll be one, but it never seems to happen.)

I listed a bunch of fundamental concepts in homotopy theory starting
in "<a href = "week115.html">week115</A>" and going through
"<a href = "week119.html">week119</A>".  I listed them with
letters A, B, C, and so on up to the letter P.  Then I decided to slack
off and take a ten-year break.  Now I'll continue...

\par\noindent\rule{\textwidth}{0.4pt}
Q.  The "based loop space" functor:

\Omega : [path-connected pointed spaces] \to  [topological groups]

Suppose X is a path-connected pointed space.  Often people define
\Omega (X) to be the space of all based loops

f: [0,1] \to  X

where f(0) = f(1) is the basepoint of X.  There's an obvious way to
compose these loops, spending half your time on the first loop and
half your time on the second, but it's not associative!  It's just
associative up to homotopy.  So, we don't get a topological monoid,
just a topological monoid "up to homotopy".  Similarly, the "reverse"
of a loop, where we run it backwards in time, is only an inverse up to
homotopy.

The concept of a topological monoid "up to homotopy" can be made
precise using Stasheff's theory of A_{\infty } spaces.  So, we can
learn to love those - and we should.  But we can also fight harder to
get an honest topological group!

For starters, let's try to make the associative and unit laws hold as
equations, instead of just up to homotopy.  For this, we can just use
"Moore loops", which are maps

f: [0,T] \to  X 

where f(0) = f(T) is the basepoint of X, and T is any nonnegative real
number.  Composing a Moore loop of length T and one of length T'
naturally gives one of length T+T'.  This way of composing loops
satisfies the associative and unit laws "on the nose", since we don't
need to do any reparametrization.  So, if we let \Omega (X) be the space
of based Moore loops on X, it's a topological monoid!

Even better, the space of based Moore loops is homotopy equivalent to
the space of ordinary based loops.  They're even equivalent "as
A_{\infty } spaces" - that is, topological spaces with a
multiplication that's associative up to a homotopy that satisfies some
equation up to homotopy... and so on to infinity.

So, we're not really changing the subject by switching from ordinary
loops to Moore loops - at least, not as far as homotopy theory goes.

But what about inverses?  Sadly, Moore loops still only have inverses
"up to homotopy".  But here we can play another trick.  

Namely: we can always take a topological monoid, throw in formal
inverses, and put on a suitable topology to get a topological group.
This process is called "group completion".  It's a functor:

G: [topological monoids] \to  [topological groups]

and it's the left adjoint of the forgetful functor

F: [topological groups] \to  [topological monoids]

I described group completion in item P of "<a href =
"week119.html">week119</A>", and gave the classic reference.

Now, if we start with a \emph{path-connected} topological monoid M, its
group completion GM is homotopy equivalent to M.  They're even
equivalent as A_{\infty } spaces, I think.  So in this case we're just
improving M slightly to make it into a group.  But if M has lots of
connected components, GM can be drastically different.  For example,
if we start with the natural numbers, its group completion is the
integers!  

So, to improve our topological monoid \Omega (X) into a topological
group, I think this is what we should do.  Take the path component of
the identity and group complete that, getting a group G.  Then build a
topological group with the same group of path components as \Omega (X),
but with each component replaced by the group G.  

I'm pretty sure this trick lets us turn the monoid of based Moore
loops in X into a topological group that's equivalent as an
A_{i}nfinity space.  I'd love to be corrected if I'm wrong
here, or doing something suboptimal.

Henceforth, let's use \Omega (X) to stand for the group completion of
the monoid of based Moore loops.  These are what we naively \emph{want}
from our based loops in X: an honest topological group!

\par\noindent\rule{\textwidth}{0.4pt}
R.  The "singular chains" functor from topological groups to
differential graded cocommutative Hopf algebras:

C_{<sub>*}</sub>: [topological groups] \to  [DGCHAs]

To get this, let's line up some functors I mentioned last week:

Sing: [topological spaces] \to  [simplicial sets]

F: [simplicial sets] \to  [simplicial vector spaces]

N: [simplicial vector spaces] \to  [chain complexes]

Composing these is how we take any space and get a chain complex!

C_{<sub>*}</sub>: [topological spaces] \to  [chain complexes]

Namely, the chain complex whose homology is the rational homology 
of that space.   This is often called the "singular chain complex"
of our space.

And now we want to tackle this puzzle: if our topological space is a
topological \emph{group}, why does its chain complex become a DGCHA?

The argument is an easy downhill slide... but alas, there's a big bump
near the end that throws me off.

You see, all the categories above have a tensor product that makes
them symmetric monoidal.  For topological spaces this is the usual
cartesian product; for simplicial sets it's also the cartesian
product, and for chain complexes it's the tensor product I already
mentioned.

And, \emph{almost} all the functors listed above are symmetric monoidal
functors.  The first two actually are.  The third one:

N: [simplicial vector spaces] \to  [chain complexes]

is not quite.  I talked about this problem last week.  

If all three functors \emph{were} symmetric monoidal, they would
send cocommutative Hopf monoids to cocommutative Hopf monoids.  And
every topological group G is a cocommutative Hopf monoid.  So, if we
didn't have this slight problem, we would instantly know that
C_{<sub>*}</sub>(G) is a cocommutative Hopf monoid in [chain
complexes].  And that's precisely a DGCHA!

But alas, it's not quite so easy.  We get stuck at the second stage:
our group G becomes a cocommutative Hopf monoid in [simplicial
abelian groups], and then we get stuck.  

Let me remind you a bit about the annoying properties of the third
functor on my list:

N: [simplicial vector spaces] \to  [chain complexes]

It's called the "normalized chain complex" or
"normalized Moore complex" functor.

As I said last time, this functor is not monoidal.  But it's "lax
monoidal".  So, there's a natural transformation

EZ: N(X) \otimes  N(Y) \to  N(X \times  Y)

And it's also "oplax monoidal".  So, there's also a
natural transformation going back:

AW: N(X \times  Y) \to  N(X) \otimes  N(Y)

But they're not inverses.

These natural transformations are called the Eilenberg-Zilber and
Alexander-Whitney maps - it took 4 great mathematicians to invent
them.  Maybe too many cooks spoil the broth: it's really annoying that
these maps aren't inverses!  As I said last time, they come very
close.  EZ followed by AW is the identity.  AW followed by EZ is not.
But, it's chain homotopic to the identity!

Let's see how far we can get with just this.

In any monoidal category, we can define "monoids".  I
explained how back in "<a href = "week89.html">week89</A>",
so let's pretend you know this.  The great thing about a lax monoidal
functor is that it sends monoids to monoids.

A monoid object in topological spaces is called a "topological
monoid" - an example is a topological group.  On the other hand,
a monoid object in chain complexes is called a "differential
graded algebra".  Since C is a composite of functors that are
either monoidal or (ahem) just lax monoidal, pure abstract nonsense
tells us that C sends topological groups to differential graded
algebras!

In any monoidal category, we can also define "comonoids".
The great thing about an oplax monoidal functor is that it sends
comonoids to comonoids.

As I mentioned last week, in a category with finite products, every
object is a comonoid in exactly one way!  The comultiplication

\Delta : X \to  X \times  X

is the diagonal map, and the counit

\epsilon : X \to  1

is the unique map to the terminal object.  This, by the way, is
why people don't talk about comonoids in the category of sets:
every set is a comonoid in exactly one way.

The category of topological spaces has finite products, so every
topological space is a comonoid in just one way.  On the other hand, a
comonoid object is chain complexes is called a "differential
graded coalgebra".

Since C is a composite of functors that are either monoidal or (ahem)
just oplax monoidal, pure abstract nonsense tells us that C sends
topological spaces to differential graded coalgebras!  

So, without breaking a sweat, we have seen that for a topological
group G, the chain complex C_{<sub>*}</sub>(G) is both a
differential graded algebra and a differential graded coalgebra.  But
why do these fit together neatly to make a differential graded Hopf
algebra?  I don't know.  Somehow we just luck out.

I also don't know why C_{<sub>*}</sub>(G) gets to be
\emph{cocommutative}.  It would be automatic all 3 functors on my
list were symmetric monoidal.  But again, the third one is not.
Somehow we just luck out.

So, there are some formal properties of the normalized chain complex
functor

N: [simplicial vector spaces] \to  [chain complexes]

that I still need to understand!  

I'll conclude with some wisdom from Kathryn Hess, just so you can
get an expert's take on this situation.  Note that she says "lax
comonoidal" instead of "oplax monoidal":

\begin{quote}

    The normalized chains functor from simplicial sets to chain
    complexes (with any coefficients) is both lax monoidal and lax
    comonoidal.  The Eilenberg-Zilber equivalence, from the tensor
    product of the chains on X and on Y to the chains on the cartesian
    product of X and Y, provides the natural transformation that shows
    that the chain functor is lax monoidal. The Alexander-Whitney
    equivalence goes in the opposite direction and shows that the chain
    functor is lax comonoidal.

    Since the chain functor is lax comonoidal, the normalized chains on
    any simplicial set is a dg coalgebra, where the comultiplication is
    given by the composite of the chain functor applied to the diagonal
    map, followed be the Alexander-Whitney transformation.  It turns
    out that the Eilenberg-Zilber equivalence is actually itself a
    morphism of coalgebras with respect to this comultiplication.  On
    the other hand, the Alexander-Whitney map is a morphism of
    coalgebras up to strong homotopy.

    The A-W/E-Z equivalences for the normalized chains functor are a
    special case of the strong deformation retract of chain complexes
    that was constructed by Eilenberg and MacLane in their 1954 Annals
    paper "On the groups H(\pi ,n). II".  For any commutative
    ring R, they defined chain equivalences between the tensor
    product of the normalized chains on two simplicial R-modules and
    the normalized chains on their levelwise tensor product.

    Steve Lack and I observed recently that the normalized chains
    functor is actually even Frobenius monoidal.  We then discovered
    that Aguiar and Mahajan already had a proof of this fact in their
    recent monograph. :-)

 \end{quote}

\par\noindent\rule{\textwidth}{0.4pt}

Finally: what about the picture at the top of the page?  It was taken
in spring, near the south pole of Mars:

6) HiRISE (High Resolution Imaging Science Experiments), Cryptic
terrain on Mars, <a href = "http://hirise.lpl.arizona.edu/PSP_003179_0945">http://hirise.lpl.arizona.edu/PSP_003179_0945</a>

Candy Hansen writes:

\begin{quote}
  There is an enigmatic region near the south pole of Mars known as
  the "cryptic" terrain. It stays cold in the spring, even as its
  albedo darkens and the sun rises in the sky.

  This region is covered by a layer of translucent seasonal carbon
  dioxide ice that warms and evaporates from below. As carbon dioxide
  gas escapes from below the slab of seasonal ice it scours dust from
  the surface. The gas vents to the surface, where the dust is carried
  downwind by the prevailing wind.

  The channels carved by the escaping gas are often radially organized
  and are known informally as "spiders."
\end{quote}

Sounds spooky!  I love how these photos of Mars are revealing it to be
a complex and varied place.  They dispel the common impression that
it's uniformly red, dusty and dull.  I thank Jim Stasheff for pointing
them out!

\par\noindent\rule{\textwidth}{0.4pt}

\textbf{Addenda:} 
I thank John Armstrong and Tim Silverman for catching small
mistakes, and Kathryn Hess and Will Orrick for catching big ones.

Forrest W. Doss reassured me somewhat about the thing I
called "temperature momentum" - the thing whose time
derivative is temperature.  He wrote:

\begin{quote}
Hello, I am a grad student who reads your 'weekly' posts.  I research
shock waves in radiation-hydrodynamic regimes where the usual models
fail, and amuse myself by studying QFT and other things on the side.
I just wanted to reply to your statement that you were 'nervous' that
nobody seemed to talk about temperature as the time-derivative of a
quantity.

I actually once ran into this concept while looking for work on
extremal-action formalisms of thermodynamics/gas dynamics.  I found it
in A. Taub, "On Hamilton's principle for perfect compressible fluids",
in the Proceedings of the First Symposium in Applied Mathematics of
the American Mathematical Society, 1947.  He references the idea from
Helmholtz's \emph{Wissenschaftliche Abhandlungen}, I don't know in
what context it appeared there. He also says it is written as "a" in
Von Laue's \emph{Relativit&auml;tstheorie}, a convention which he
follows.  So the concept does exist out there!  
\end{quote}

For more discussion, visit the 
<a href = "http://golem.ph.utexas.edu/category/2010/01/this_weeks_finds_in_mathematic_50.html">\emph{n}-Category Caf&eacute;</a>.


\par\noindent\rule{\textwidth}{0.4pt}
<em>If to any homogeneous mass... we suppose an infinitesimal quantity
of any substance to be added, the mass remaining homogeneous and its
entropy and volume remaining unchanged, the increase of the energy of
the mass divided by the quantity of the substance added is the
potential for that substance in the mass considered.</em> - J. Willard
Gibbs

<em>A vague discomfort at the thought of the chemical potential is
still characteristic of a physics education.  This intellectual gap is
due to the obscurity of the writings of J. Willard Gibbs who
discovered and understood the matter 100 years ago.</em> - Charles
Kitell, \emph{Introduction to Solid State Physics}

<em>A nightmare... The prose is both laconic and imprecise - a
combination that spells very poor readability.</em> - J. Zrake, review
of Kitell's \emph{Introduction to Solid State Physics}

\par\noindent\rule{\textwidth}{0.4pt}

% </A>
% </A>
% </A>
