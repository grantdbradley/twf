
% </A>
% </A>
% </A>
\week{March 15, 1997}


Life here at the Center for Gravitational Physics and Geometry is
tremendously exciting.  In two weeks I have to return to U. C.
Riverside and my mundane life as a teacher of calculus, but right now
I'm still living it up.  I'm working with Ashtekar, Corichi, and
Krasnov on computing the entropy of black holes using the loop
representation of quantum gravity, and also I'm talking to lots of
people about an interesting 4-dimensional formulation of the loop
representation in terms of "spin foams" --- roughly speaking,
soap-bubble-like structures with faces labelled by spins.

Here are some papers I've come across while here:

1) Lee Smolin, The future of spin networks, in
The Geometric Universe: Science, Geometry, and the Work of Roger
Penrose, eds. S. Hugget, Paul Tod, and L.J. Mason, Oxford University
Press, 1998.  Also available as
<A HREF = "http://xxx.lanl.gov/abs/gr-qc/9702030">gr-qc/9702030</A>.

I've spoken a lot about spin networks here on This Week's Finds.  They
were first invented by Penrose as a radical alternative to the usual
way of thinking of space as a smooth manifold.  For him, they were
purely discrete, purely combinatorial structures: graphs with edges
labelled by "spins" j = 0, 1/2, 1, 3/2, etc., and with three edges
meeting at each vertex.  He showed how when these spin networks become
sufficiently large and complicated, they begin in certain ways to
mimic ordinary 3-dimensional Euclidean space.  Interestingly, he never
got around to publishing his original paper on the subject, so it
remains available only if you know someone who knows someone who has it:

2) Roger Penrose, Theory of quantized directions, unpublished manuscript.

In case you're wondering, I don't have a copy.  Someone here has
an nth-generation xerox copy, which I read, but n was sufficiently large
that the (n+1)st generation copy would have been unreadable.  I will
get ahold of it somehow, though!  

Anyway, Smolin's paper is a kind of tribute to Penrose, and it traces
the curiously twisting history of spin networks from their origin up
to the present day, where they play a major role in topological
quantum field theory and the loop representation - now more
appropriately called the spin network representation! - of quantum
gravity.  (See "<A HREF = "week55.html">week55</A>" for more
on spin networks.)

Note however that the title of the paper refers to the \emph{future} of
spin networks.  Smolin argues that spin networks are a major clue
about the future of physics, and he paints a picture of what this
future might be... which I urge you to look at.

For more on this, try:


 3) Fotini Markopoulou and Lee Smolin, Causal evolution of spin
networks, preprint available as <A HREF =
"http://arxiv.org/abs/gr-qc/9702025">gr-qc/9702025</A>.

Fotini Markopoulou is a student of Chris Isham at Imperial College,
but now she's visiting the CGPG and working with Lee Smolin on spin
networks.  In this paper they describe some theories in which spin
networks evolve in time in discrete steps.  The evolution is "local"
in the sense that in a given step, any vertex of the spin network
changes in a way that only depends on its immediate neighbors -
vertices connected to it by an edge.  
It is also "causal" in the sense
that history of spin network evolving according to their rules gives
a causal set, i.e. a set equipped with a partial ordering which we think
of as saying which points come "before" which other points.  
This ties their work to the work of Rafael Sorkin on causal sets, e.g.:

4) Luca Bombelli, Joohan Lee, David Meyer and Rafael D. Sorkin, 
Space-time as a causal set, Phys. Rev. Lett. 59 (1987), 521.

Unlike the related work of Reisenberger and Rovelli (see "<A HREF
= "week96.html">week96</A>"), Markopolou and Smolin do not
attempt to "derive" their rules from general relativity by
standard quantization techniques.  Instead, they hope that some theory
of the sort they consider will approximate general relativity in the
large-scale limit.  To check this will require some new techniques
akin to the "renormalization group" approach to studying the
large-scale limits of statistical mechanical systems defined on a
lattice.  This is a bit daunting, but it seems likely that no matter
how one proceeds to pursue a spin-network-based theory of quantum
gravity, one will need to develop such techniques at some point.

% <A NAME = "tale">
Now I'd like to switch gears and return to...

THE TALE OF N-CATEGORIES!


Recall that in our last episode, in "<A HREF =
"week92.html">week92</A>", we had worked our way up to
2-categories, and we were beginning to see what they had to do with
2-dimensional physics and toplogy.  I described how to get monads from
adjunctions, and what this has to do with matrix multiplication,
Yang-Mills theory, and the 4-color theorem.

Next week I want to get serious and start talking about n-categories
for arbitrary n.  One reason is that at the end of this month there's
a conference on n-categories and physics that I want to report on:

5) Workshop on Higher Category Theory and Physics, March 28-30, 1997,
Northwestern University, Evanston, Illinois.  Organized by Ezra Getzler and 
Mikhail Kapranov; program available at 
<A HREF = "http://math.nwu.edu/~getzler/conf97.html">http://math.nwu.edu/~getzler/conf97.html</A>

But before doing this, I want to say a bit about what category theory
has to do with quantum mechanics!

First remember the big picture: n-category theory is a language to
talk about processes that turn processes into other processes.
Roughly speaking, an n-category is some sort of structure with
objects, morphisms between objects, 2-morphisms between morphisms, and
so on up to n-morphisms.  A 0-category is just a set, with its objects
usually being called "elements".  Things get trickier as n increases.
For a precise definition of n-categories for n = 1 and 2, see "<A HREF = "week73.html">week73</A>"
and "<A HREF = "week80.html">week80</A>", respectively.

Most familiar mathematical gadgets are sets equipped with extra bells
and whistles: groups, vector spaces, Hilbert spaces, and so on all have
underlying sets.  This is why set theory plays an important role in
mathematics.  However, we can also consider fancier gadgets that are
\emph{categories} equipped with extra bells and whistles.  Some of the
most interesting examples are just "categorifications" of
well-known gadgets.

For example, a "monoid" is a simple gadget, just a set equipped with
an associative product and multiplicative identity.  An example we all
know and love is the complex numbers: the product is just ordinary
multiplication, and the multiplicative identity is the number 1.


We may categorify the notion of "monoid" and define a
"monoidal category" to be a \emph{category} equipped with an
associative product and multiplicative identity.  I gave the precise
definition back in "<A HREF = "week89.html">week89</A>"; the
point here is that while they may sound scary, monoidal categories are
actually very familiar.  For example, the category of Hilbert spaces is
a monoidal category where the product of Hilbert spaces is the tensor
product and the multiplicative identity is C, the complex numbers.

If one systematically studies categorification one discovers an
amazing fact: many deep-sounding results in mathematics are just
categorifications of stuff we all learned in high school.  There is a
good reason for this, I believe.  All along, mathematicians have been
unwittingly "decategorifying" mathematics by pretending that
categories are just sets.  We "decategorify" a category by forgetting
about the morphisms and pretending that isomorphic objects are equal.
We are left with a mere set: the set of isomorphism classes of
objects.

I gave an example in "<A HREF = "week73.html">week73</A>".  There is a category FinSet whose
objects are finite sets and whose morphisms are functions.  If we
decategorify this, we get the set of natural numbers!  Why?  Well, two
finite sets are isomorphic if they have the same number of elements.
"Counting" is thus the primordial example of decategorification.  


 I like to think of it in terms of the following fairy tale.  Long
ago, if you were a shepherd and wanted to see if two finite sets of
sheep were isomorphic, the most obvious way would be to look for an
isomorphism.  In other words, you would try to match each sheep in herd
A with a sheep in herd B.  But one day, along came a shepherd who
invented decategorification.  This person realized you could take each
set and "count" it, setting up an isomorphism between it and
some set of "numbers", which were nonsense words like
"one, two, three, four,..." specially designed for this
purpose.  By comparing the resulting numbers, you could see if two herds
were isomorphic without explicitly establishing an isomorphism!

According to this fairy tale, decategorification started out as the
ultimate stroke of mathematical genius.  Only later did it become a
matter of dumb habit, which we are now struggling to overcome through
the process of "categorification".

Okay, so what does this have to do with quantum mechanics?


Well, a Hilbert space is a set with extra bells and whistles, so maybe
there is some gadget called a "2-Hilbert space" which is a
\emph{category} with analogous extra bells and whistles.  And maybe if
we figure out this notion we will learn something about quantum
mechanics.

Actually the notion of 2-Hilbert space didn't arise in this
simple-minded way.  It arose in some work by Daniel Freed on
topological quantum field theory:

5) Higher algebraic structures and quantization, by Dan Freed,
Comm. Math. Phys. 159 (1994), 343-398, preprint available as
<A HREF = "http://xxx.lanl.gov/ps/hep-th/9212115">hep-th/9212115</A>; see also <A HREF = "week48.html">week48</A>.

Later, Louis Crane adopted this notion as part of his program to 
reduce quantum gravity to n-category theory:

6) Louis Crane: Clock and category: is quantum gravity algebraic?,
Jour. Math. Phys. 36 (1995), 6180-6193, preprint available as 
<A HREF = "http://xxx.lanl.gov/ps/gr-qc/9504038">gr-qc/9504038</A>.

These papers made is clear that 2-Hilbert spaces are interesting
things and that one should go further and think about "n-Hilbert
spaces" for higher n, too.  However, neither of them gave a precise
definition of 2-Hilbert space, so at some point I decided to do this.
It took a while for me to learn enough category theory, but eventually
I wrote something about it:

7) John Baez, Higher-dimensional algebra II: 2-Hilbert spaces, 
to appear in Adv. Math., preprint available as <A HREF = "http://xxx.lanl.gov/ps/q-alg/9609018">q-alg/9609018</A> or
at <A HREF = "http://math.ucr.edu/home/baez/">http://math.ucr.edu/home/baez/</A>

To understand this requires a little category theory, so let
me explain the basic ideas here.  

I'll concentrate on finite-dimensional Hilbert spaces, since the
infinite-dimensional case introduces extra complications.  To define
2-Hilbert spaces we need to start by categorifying the various
ingredients in the definition of Hilbert space.  These are:
<OL>
<LI>
the zero element, 
<LI>
addition, 
<LI>
subtraction, 
<LI>
scalar multiplication,
and 
<LI>
the inner product. 
</OL>
The first four have well-known categorical
analogs.  The fifth one, which is really the essence of a Hilbert
space, may seem a bit more mysterious at first, but as we shall see,
it's really the key to the whole business.

1) The analog of the zero vector is a `zero object'.  A zero object in
a category is an object that is both initial and terminal.  That is,
there is exactly one morphism from it to any object, and exactly one
morphism to it from any object.  Consider for example the category
Hilb having finite-dimensional Hilbert spaces as objects, and linear
maps between them as morphisms.  In Hilb, any zero-dimensional Hilbert
space is a zero object.

Note: there isn't really a unique zero object in the "strict"
sense of the term.  Instead, any two zero objects are canonically
isomorphic.  The reason is that if you have two zero objects, say 0 and
0', there is a unique morphism f: 0 \to  0' and a unique morphism g: 0'
-> 0.  These morphisms are inverses of each other so they are
isomorphisms.  Why are they inverses?  Well, fg: 0 \to  0' must be the
identity morphism 1_{0}: 0 \to  0, because there is only one
morphism from 0 to 0!  Similarly, gf is the identity on 0'.  (Note that
I am using category theorist's notation, where the composite of f: x
-> y and g: y \to  z is denoted fg: x \to  z.)

This is typical in category theory.  We don't expect stuff to be
unique; it should only be unique up to a canonical isomorphism.

2) The analog of adding two vectors is forming the "coproduct" of two
objects.  Coproducts are just a fancy way of talking about direct
sums.  Any decent quantum mechanic knows about the direct sum of
Hilbert spaces.  But in fact, we can define this notion very generally
in any category, where it goes under the name of a "coproduct".  (I
give the definition below; if I gave it here it would scare people
away.)  As with zero objects, coproducts are typically not unique, but
they are always unique up to canonical isomorphism, which is what
matters.  It's a good little exercise to show this.

3) The analog of subtracting vectors is forming the "cokernel" of a
morphism f: x \to  y.  If x and y are Hilbert spaces, the cokernel of f
is just the orthogonal complement of the range of f.  In other words,
for Hilbert spaces we have "direct differences" as well as direct
sums.  However, the notion of cokernel makes sense in any category
with a zero object.  I won't burden you with the precise definition
here.

An important difference between zero, addition and subtraction and
their categorical analogs is that these operations represent extra
\emph{structure} on a set, while having a zero object, coproducts of two
objects, or cokernels of morphisms is merely a \emph{property} of a
category.  Thus these concepts are in some sense more intrinsic to
categories than to sets.  On the other hand, we've seen one pays a
price for this: while the zero element, sums, and differences are
unique in a Hilbert space, the zero object, coproducts, and cokernels
are typically unique only up to canonical isomorphism.

4) The analog of multiplying a vector by a complex number is tensoring
an object by a Hilbert space.  Besides its additive properties (zero
object, binary coproducts, and cokernels), Hilb is also a monoidal
category: we can multiply Hilbert space by tensoring them, and there
is and a multiplicative identity, namely the complex numbers C.  In
fact, Hilb is a "ring category", as defined by Laplaza and Kelly.

We expect Hilb it to play a role in 2-Hilbert space theory analogous
to the role played by the ring C of complex numbers in Hilbert space
theory.  Thus we expect 2-Hilbert spaces to be "module categories"
over Hilb, as defined by Kapranov and Voevodsky.

An important part of our philosophy here is that C is the primordial
Hilbert space: the simplest one, upon which the rest are modelled.  By
analogy, we expect Hilb to be the primordial 2-Hilbert space.  This is
part of a general pattern pervading higher-dimensional algebra; for
example, there is a sense in which the (n+1)-category of all (small)
n-categories, nCat, is the primordial (n+1)-category. The real
significance of this pattern remains mysterious.

5) Finally, what is the categorification of the inner product in a
Hilbert space?  It's the `hom functor'!  The inner product in a
Hilbert space eats two vectors v and w and spits out a complex number

<v,w>

Similarly, given two objects v and w in a category, the hom functor 
gives a \emph{set}

hom(x,y)

namely the set of morphisms from x to y.  Note that the inner product
<v,w> is linear in w and conjugate-linear in y, and similarly, the hom
functor hom(x,y) is covariant in y and contravariant in x.  This hints
at the category theory secretly underlying quantum mechanics.  In
quantum theory the inner product <v,w> represents the
\emph{amplitude} to
pass from v to w, while in category theory hom(x,y) is the \emph{set} of
ways to get from x to y.  In Feynman path integrals, we do an integral
over the set of ways to get from here to there, and get a number, the
amplitude to get from here to there.  So when physicists do Feynman 
path integration - just like a shepherd counting sheep - they are engaged
in a process of decategorification!


To understand this analogy better, note that any morphism f: x \to  y
in Hilb can be turned around or "dualized" to obtain a
morphism f*: y \to  x.  This is usually called the "adjoint"
of f, and it satisfies

<fv,w> = <v,f*w> 

for all v in x, and w in y.  This ability to dualize morphisms is
crucial to quantum theory.  For example, observables are represented
by self-adjoint morphisms, while symmetries are represented by unitary
morphisms, whose adjoint equals their inverse.

However, it should now be clear - at least to the categorically minded -
that this sort of adjoint is just a decategorified version of the
"adjoint functors" so important in category theory.  As I explained in
"<A HREF = "week79.html">week79</A>", a functor F*: D \to  C is a "right adjoint" of F: C \to  D if
there is, not an equation, but a natural isomorphism

hom(Fc,d) ~ hom(c,F*d) 

for all objects c in C, and d in D.  

Anyway, in the paper I proceed to use these ideas to give a precise
definition of 2-Hilbert spaces, and then I prove all sorts of stuff
which I won't describe here.

Let me wrap up by explaining the definition of "coproduct".  This is
one of those things they should teach all math grad students, but for
some reason they don't.  It's a bit dry but it'll be good for you.  A
coproduct of the objects x and y is an object x+y equipped with
morphisms

i: x \to  x+y  

and 

j: y \to  x+y,

that is universal with respect to this property.  In other words,
if we have any \emph{other} object, say z, and morphisms

i': x \to  z

and 

j': y \to  z,

then there is a unique morphism f: x+y \to  z such that 

i' = if  

and

j' = jf.

This kind of definition automatically implies that the coproduct is
unique up to canonical isomorphism.  To understand this abstract
nonsense, it's good to check that the coproduct of sets or topological
spaces is just their disjoint union, while the coproduct of vector
spaces or Hilbert spaces is their direct sum.


<A HREF = "week100.html#tale">To continue reading the `Tale of
n-Categories', click here.</A>


\par\noindent\rule{\textwidth}{0.4pt}

% </A>
% </A>
% </A>
