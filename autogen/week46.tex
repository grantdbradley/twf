
% </A>
% </A>
% </A>
\week{December 12, 1994}

I will be on sabbatical during the first half of 1995.  I'll be roaming
hither and thither, and also trying to get some work done on n-categories,
quantum gravity and such, so this will be the last "This Week's Finds"
for a while.  I have also taken a break from being a co-moderator of
sci.physics.research.

So, let me sign off with a roundup of diverse and sundry things!  I'm
afraid I'll be pretty terse about describing some of them.  First for
some news of general interest, then a little update on Seiberg-Witten
theory, then some neat stuff on TQFTs, n-categories, quantum gravity and
all that, and then various other goodies....

1)  The speed of write, by Gary Stix, Scientific American, Dec. 1994,
106-111.  

Goodbye, Gutenberg, by Jacques Leslie, WiReD 2.10, Oct. 1994, 
available via WWW as http://www.hotwired.com/Lib/Wired/2.10/departments
/electrosphere/ejournals.html  


Among other things, the above articles show that Paul Ginsparg is
starting to get the popular recognition he deserves for starting up
hep-th.  In case anyone out there doesn't know yet, hep-th is the
"high-energy physics - theoretical" preprint archive, which
revolutionized communications within this field by making preprints
easily available world-wide, thus rendering many (but not all) aspects
of traditional journals obsolete.  The idea was so good it quickly
spread to other subjects.  Within physics it went like this:


\begin{verbatim}

   High Energy Physics - Theory (hep-th), started 8/91 
   High Energy Physics - Lattice (hep-lat), started 2/92 
   High Energy Physics - Phenomenology (hep-ph), started 3/92 
   Astrophysics (astro-ph), started 4/92 
   Condensed Matter Theory (cond-mat), started 4/92 
   General Relativity \text{\&}  Quantum Cosmology (gr-qc), started 7/92 
   Nuclear Theory (nucl-th), started 10/92 
   Chemical Physics (chem-ph), started 3/94 
   High Energy Physics - Experiment (hep-ex), started 4/94 
   Accelerator Physics (acc-phys), started 11/94 
   Nuclear Experiment (nucl-ex), started 11/94 
   Materials Theory (mtrl-th), started 11/94 
   Superconductivity (supr-con), started 11/94 
\end{verbatim}
    

Similar archives are sprouting up in mathematics (see below - but also
note the existence of the American Mathematical Society preprint server,
described later in this week's finds).  

There are many ways to access these preprint archives, since Ginsparg
has kept up very well with the times - indeed, so much better than I
that I'm afraid to go into any details for fear of making a fool of
myself.  The dernier cri, I suppose, is to access the archives
using the World-Wide Web, which is conveniently done by opening the
document

http://xxx.lanl.gov

If this makes no sense to you, my first and very urgent piece of advice
is to learn about the World-Wide Web (WWW), Mosaic, and the like, since
they are wonderful and very simple to use!  In the meantime, however,
you can simply send mail to various addresses with subject header

help

and no message body, in order to get information.  Some addresses are:


\begin{verbatim}

acc-phys@xxx.lanl.gov               (accelerator physics)
astro-ph@xxx.lanl.gov               (astrophysics)
chem-ph@xxx.lanl.gov                (chemical physics)
cond-mat@xxx.lanl.gov               (condensed matter)
funct-an@xxx.lanl.gov               (functional analysis)
gr-qc@xxx.lanl.gov                  (general relativity / quantum cosmology)
hep-lat@ftp.scri.fsu.edu            (computational and lattice physics)
hep-ph@xxx.lanl.gov                 (high energy physics phenomenological)
hep-th@xxx.lanl.gov                 (high energy physics formal)
hep-ex@xxx.lanl.gov                 (high energy physics experimental)
nucl-th@xxx.lanl.gov                (nuclear theory)
nucl-ex@xxx.lanl.gov                (nuclear experiment)
mtrl-th@xxx.lanl.gov                (materials theory)
supr-con@xxx.lanl.gov               (superconductivity)

alg-geom@publications.math.duke.edu (algebraic geometry)
auto-fms@msri.org                   (automorphic forms)
cd-hg@msri.org                      (complex dynamics \text{\&}  hyperbolic geometry)
dg-ga@msri.org                      (differential geometry \text{\&}  global analysis)

nlin-sys@xyz.lanl.gov               (non-linear systems)
cmp-lg@xxx.lanl.gov                 (computation and language)
e-mail@xxx.lanl.gov                 (e-mail address database)
\end{verbatim}
    

One might also want to check out the:

Directory of Electronic Journals, Newsletters, and Academic Discussion
Lists: Send e-mail to ann@cni.org at the Association of Research
Libraries, +1 (202) 296-2296, fax +1 (202) 872 0884.


2) Monopoles and four-manifolds, by Edward Witten, preprint available as
<A HREF = "http://xxx.lanl.gov/abs/hep-th/9411102">hep-th/9411102</A>.

The genus of embedded surfaces in the projective plane, by 
P. B. Kronheimer and T. S. Mrowka, preprint number #19941128001,
available from the AMS preprint server under subject 57 in the
Mathematical Reviews Subject Classification Scheme. 


I don't have anything interesting to say about these papers that wasn't in
"<A HREF = "week44.html">week44</A>" and "<A HREF = "week45.html">week45</A>", but anyone interested in following the revolution
in Donaldson theory initiated by Seiberg and Witten will have to read
them.  

Let me say a bit about how the AMS preprint server works.  Assuming you
are hip to the WWW, just go to

http://e-math.ams.org

You will then see a menu, and you can click on "Mathematical Preprints",
and then "AMS Preprint Server", where preprints are classified by subject.  
Alternatively, click on "New Items This Month (all Subjects)".

On a related note, you can also get some AMS stuff using telnet by doing

telnet e-math.ams.org

and using 

e-math

as login and password.   This doesn't seem to get you to the preprints,
though.  For gopher fans,

gopher e-math.ams.org
 
has roughly similar effects.  

 
3) Spin networks in quantum gravity, by Carlo Rovelli and Lee Smolin, to
appear.


This paper is closely related to the earlier one in which Rovelli and
Smolin argue that discreteness of area and volume arise naturally in the
loop representation of quantum gravity, and also to my own paper on spin
networks.  (See "<A HREF = "week43.html">week43</A>" for more on these, and a brief intro to spin
networks.)  Basically, while my paper shows that spin networks give a
kind of basis of states for gauge theories with arbitrary (compact,
connected) gauge group, in this paper Rovelli and Smolin concentrate on
the gauge groups SL(2,C) and SU(2), which are relevant to quantum
gravity, and work out a lot of aspects particular to this case, in more
of a physicist's style.  This makes spin networks into a practical
computational tool in quantum gravity, used to great effect in the
paper on the discreteness of area and volume.


4) Recent mathematical developments in quantum general relativity, by
Abhay Ashtekar, 14 pages in TeX format available as <A HREF = "http://xxx.lanl.gov/abs/gr-qc/9411055">gr-qc/9411055</A>
(discussed in "<A HREF = "week37.html">week37</A>").

Coherent state transforms for spaces of connections, by 
Abhay Ashtekar, Jerzy Lewandowski, Donald Marolf, Jose Mourao and
Thomas Thiemann, 38 pages in LaTeX format, available as <A HREF = "http://xxx.lanl.gov/abs/gr-qc/9412014">gr-qc/9412014</A>
(discussed in "<A HREF = "week43.html">week43</A>")


These are two papers on the loop representation of quantum gravity which
I talked about in earlier "finds", and are out now.  The former is a
nice review of recent mathematically rigorous work; the latter takes a
tremendous step towards handling the infamous "reality conditions"
problem.  


5) Differential geometry on the space of connections via graphs and
projective limits, by Abhay Ashtekar and Jerzy Lewandowski, 54 pages in
LaTeX format, available as <A HREF = "http://xxx.lanl.gov/abs/hep-th/9412073">hep-th/9412073</A>


I've spoken quite a bit about doing rigorous functional \emph{integration} in
gauge theory using ideas from the loop representation; this paper treats
functional \emph{derivatives} and other things that are more differential
than integral in nature.  This is crucial in quantum gravity because the
main remaining mystery, the Wheeler-DeWitt equation or Hamiltonian
constraint, involves a differential operator on the space of connections.
(For a wee bit more, try "<A HREF = "week11.html">week11</A>" or "<A HREF = "week43.html">week43</A>", where the Hamiltonian
constraint is simply written as G_{00} = 0.)  

Let me quote their abstract:

\par\noindent\rule{\textwidth}{0.4pt}
In a quantum mechanical treatment of gauge theories (including general
relativity), one is led to consider a certain completion, A, of the
space of gauge equivalent connections. This space serves as the quantum
configuration space, or, as the space of all Euclidean histories over
which one must integrate in the quantum theory.  A is a very large space
and serves as a ``universal home'' for measures in theories in which the
Wilson loop observables are well-defined. In this paper, A is considered
as the projective limit of a projective family of compact Hausdorff
manifolds, labelled by graphs (which can be regarded as ``floating
lattices'' in the physics terminology). Using this characterization,
differential geometry is developed through algebraic methods. In
particular, we are able to introduce the following notions on A:
differential forms, exterior derivatives, volume forms, vector fields
and Lie brackets between them, divergence of a vector field with respect
to a volume form, Laplacians and associated heat kernels and heat kernel
measures.  Thus, although A is very large, it is small enough to be
mathematically interesting and physically useful. A key feature of this
approach is that it does not require a background metric. The
geometrical framework is therefore well-suited for diffeomorphism
invariant theories such as quantum general relativity.
\par\noindent\rule{\textwidth}{0.4pt}

6)  Edge states in gravity and black hole physics, by A. P.
Balachandran, L. Chandar, Arshad Momen, 22 pages in RevTeX format,
available as <A HREF = "http://xxx.lanl.gov/abs/gr-qc/9412019">gr-qc/9412019</A>.  


Ever since it started seeming that black holes have an entropy closely
related to (and often proportional to) the area of their event horizons,
many physicists have sought a better understanding of this entropy.  In
many ways, the nicest sort of explanation would say that the entropy was
due to degrees of freedom living on the event horizon.  A concrete
calculation along these lines was recently made by Steve Carlip (see
"<A HREF = "week41.html">week41</A>") in the context of 2+1-dimensional gravity.  The mechanism 
is mathematically very similar to what happens in (a widely popular
theory of) the fractional quantum Hall effect!  In both cases, 
Chern-Simons theory on a 3d manifold with boundary gives rise to an
interesting field theory on the boundary, or "edge".  The above paper
clarifies this, especially for those of us who don't understand the
fractional quantum Hall effect too well.  Let me quote:

\par\noindent\rule{\textwidth}{0.4pt}
Abstract: We show in the context of Einstein gravity that the removal of
a spatial region leads to the appearance of an infinite set of
observables and their associated edge states localized at its boundary.
Such a boundary occurs in certain approaches to the physics of black
holes like the one based on the membrane paradigm. The edge states can
contribute to black hole entropy in these models. A "complementarity
principle" is also shown to emerge whereby certain "edge" observables
are accessible only to certain observers. The physical significance of
edge observables and their states is discussed using their similarities
to the corresponding quantities in the quantum Hall effect.  The
coupling of the edge states to the bulk gravitational field is
demonstrated in the context of (2+1) dimensional gravity.
\par\noindent\rule{\textwidth}{0.4pt}

I can't resist adding that I have a personal stake in the notion that a
lot of interesting things about quantum gravity will only show up when
we consider it on manifolds with boundary, including the area-entropy
relations.  The loop representation of quantum gravity has a lot to do
with knots and links, but on a manifold with boundary it has a lot to do
with \emph{tangles}, which can contain arcs that begin and end at the
boundary.   I wrote a paper on this a while back:


7)  Quantum gravity and the algebra of tangles, by John Baez, 
Jour. Class. Quantum Grav. 10 (1993), 673 - 694.


and I'll be coming out with another in a while, co-authored with Javier
Muniain and Dardo Piriz.  The importance of manifolds with boundary for
cutting-and-pasting constructions is also well-known in the theory of
"extended" TQFTs (topological quantum field theories).  These cutting
and pasting operations should allow one to describe extended TQFTs in n
dimensions purely algebraically using "higher-dimensional algebra".  So
part of the plan here is to understand better the relation between
quantum gravity, TQFTs, and higher-dimensional algebra.  Along these
lines, a very interesting new paper has come out:


8)  On algebraic structures implicit in topological quantum field
theories, by Louis Crane and David Yetter, 13 pages in LaTeX format
available as <A HREF = "http://xxx.lanl.gov/abs/hep-th/9412025">hep-th/9412025</A>, figures available by request.  


This makes more precise some of Louis Crane's ideas on "categorification".  
Nice TQFTs in 3 dimensions have a lot to do with Hopf algebras (like
quantum groups), or alternatively, their categories of representations,
which are certain braided monoidal categories.  In this paper it's shown
that nice TQFTs in 4 dimensions have a lot to do with Hopf categories,
or alternatively, their categories of representations, which are certain
braided monoidal 2-categories.


9) On the definition of 2-category of 2-knots, by V. M. Kharlamov and V.
G. Turaev, preprint.


This preprint, which I obtained through my network of spies, \emph{seems} to
be implicitly claiming that the work of Fischer describing 2d surfaces
in R^4 via on braided monoidal 2-categories (see "<A HREF = "week12.html">week12</A>") is a bit
wrong, but they don't come out and say quite what if anything is really
wrong.


10) Non-involutory Hopf algebras and 3-manifold invariants, by Greg
Kuperberg, preprint #19941128002, available from the AMS preprint server
under subject 57 or 16 in the Mathematical Reviews Subject
Classification Scheme.


I noted the existence of a draft of this paper, and related work, in
"<A HREF = "week38.html">week38</A>".  Let me quote:
 
\par\noindent\rule{\textwidth}{0.4pt}
Abstract: We present a definition of an invariant #(M,H), defined
for every finite-dimensional Hopf algebra (or Hopf superalgebra or Hopf
object) $H$ and for every closed, framed 3-manifold M.  When H is a
quantized universal envloping algebra, #(M,H) is closely related to
well-known quantum link invariants such as the HOMFLY polynomial, but
it is not a topological quantum field theory.
\par\noindent\rule{\textwidth}{0.4pt}

Okay, now for some miscellaneous interesting things...


11) If Hamilton had prevailed: quaternions in physics, by J. Lambek,
McGill University preprint, Nov. 1994.

Lambek is mainly known for work in category theory, but he has a
strong side-interest in mathematical physics.  This paper is, first of
all, a "nostalgic account of how certain key results in modern
theoretical physics (prior to World War II) can be expressed concisely
in the labguage of quaternions, thus suggesting how they might have been
discovered if Hamilton's views had prevailed."  But there is a very
interesting new thing, too: a way in which the group SU(3) X SU(2) X
U(1) shows up naturally when considering Dirac's equation a la
quaternions.   This group is the gauge group of the Standard Model!
Lambek modestly says that there does not appear to be any significance
to this coincidence... but it \emph{would} be nice, wouldn't it?



12) The life and times of Emmy Noether; contributions of E. Noether to
particle physics, by Nina Byers, 32 pages in RevTeX format, available as
<A HREF = "http://xxx.lanl.gov/abs/hep-th/9411110">hep-th/9411110</A>.

Reminiscences about many pitfalls and some successes of QFT within the last
three decades, by B. Schroer, 52 pages, 'shar'-shell-archiv, consisting of 5
files, available as <A HREF = "http://xxx.lanl.gov/abs/hep-th/9410085">hep-th/9410085</A>.  

My encounters - as a physicist - with mathematics, R. Jackiw, 13
pages in LaTeX format, available as <A HREF = "http://xxx.lanl.gov/abs/hep-th/9410151">hep-th/9410151</A>. 


These are some interesting historical/biographical pieces.


13) Speedup in quantum computation is associated with attenuation of
processing probability, by Karl Svozil, available as <A HREF = "http://xxx.lanl.gov/abs/hep-th/9412046">hep-th/9412046</A>.

The subject of quantum computation has become more lively recently.
I haven't had time to look at this paper, but quoting the abstract:

\par\noindent\rule{\textwidth}{0.4pt}
Quantum coherence allows the computation of an arbitrary number of
distinct computational paths in parallel. Based on quantum parallelism
it has been conjectured that exponential or even larger speedups of
computations are possible. Here it is shown that, although in principle
correct, any speedup is accompanied by an associated attenuation of
detection rates. Thus, on the average, no effective speedup is obtained
relative to classical (nondeterministic) devices.
\par\noindent\rule{\textwidth}{0.4pt}
% </A>
% </A>
% </A>
