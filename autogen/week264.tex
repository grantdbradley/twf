
% </A>
% </A>
% </A>
\week{May 18, 2007 }


Here's a puzzle.  Guess the next term of this sequence:

<div align = "center">
1, 1, 2, 3, 4, 5, 6, ...
</div>

and then guess the \emph{meaning} of this sequence!  I'll give away
the answer after telling you about Coleman's videos on quantum
field theory and an amazing result on the homotopy groups of spheres.

But first... the astronomy picture of the day.

The Eaton Collection at UC Riverside may be the world's best
library of science fiction: 

1) The Eaton Collection of Science Fiction, Fantasy, Horror 
and Utopian Literature, <a href = "http://eaton-collection.ucr.edu/">http://eaton-collection.ucr.edu/</a>.

Right now my wife Lisa Raphals is attending a conference there on the
role of Mars in SF, called "Chronicling Mars".  Gregory Benford, 
Frederik Pohl, Greg Bear, David Brin, Kim Stanley Robinson and 
even Ray Bradbury are all there!  But for some reason I'm staying 
home working on This Week's Finds.  I'd say that shows true devotion - 
or maybe just stupidity.

Anyway, in honor of the occasion, here's an incredible closeup of a 
crater on Mars' moon Phobos:

<div align = "center">
<a href = "http://apod.nasa.gov/apod/ap080410.html">
<img width = "500" src = "phobos.jpg">
% </a>
</div>

2) Astronomy Picture of the Day, Stickney Crater
<a href = "http://apod.nasa.gov/apod/ap080410.html">http://apod.nasa.gov/apod/ap080410.html</a>

It's another great example of how machines in space now deliver many
more thrills per buck than the old-fashioned approach using canned
primates.  This photo was taken by HiRISE, the High Resolution Imaging
Science Experiment - the same satellite that took the stunning photos
of Martian dunes which graced "<a href =
"week262.html">week262</A>".

Mars has two moons, Phobos and the even tinier Deimos.  Their names
mean "fear" and "dread" in Greek, since in Greek
mythology they were sons of Mars (really Ares), the god of war.

Interestingly, Kepler predicted that Mars had two moons before
they were seen.  This sounds impressive, but it was simple 
interpolation, since Earth has 1 moon and Jupiter has 4.  Or at
least Galileo saw 4 - now we know there are a lot more.

Phobos is only 21 kilometers across, and the big crater you
see here - Stickney Crater - is about 9 kilometers across.
That's almost half the size of the whole moon!  The collision 
that created it must have almost shattered Phobos.

Phobos is so light - just twice the density of water - that people 
once thought it might be hollow.  This now seems unlikely, though
it's been the premise of a few SF stories.  It's more likely that 
Phobos is a loosely packed pile of carbonaceous chondrites captured 
from the asteroid belt.

Phobos orbits so close to Mars that it zips around once every
8 hours, faster than Mars itself rotates!  Oddly, in
1726 Jonathan Swift wrote about two moons of Mars in his novel
"Gulliver's Travels" - and he guessed that the inner one orbited
Mars every 10 hours.

Gravitational tidal forces are dragging Phobos down, so in only 10 
million years it'll either crash or - more likely - be shattered by 
tidal forces and form a ring of debris. 

So, enjoy it while it lasts.

Anyone who's seriously struggled to master quantum field theory is
likely to have profited from this book:

3) Sidney Coleman, Aspects of Symmetry: Selected Erice Lectures,
Cambridge U. Press, Cambridge, 1988.

It's brimming with wisdom and humor.  You should have already
encountered quantum field theory before trying it: what you'll
get are deeper insights.  

But what if you're just getting started?

Sidney Coleman, recently deceased, was one of the best quantum field
theorists from the heyday of particle physics.  As a grad student I
took a course on quantum field theory from Eddie Farhi, who said he
based his class on the notes from Coleman's class at Harvard.  So,
I've always been curious about these notes.  Now they're available
online in handwritten form:

4) Sidney Coleman, lecture notes on quantum field theory,
transcribed by Brian Hill,
<a href = "http://www.damtp.cam.ac.uk/user/dt281/qft/col1.pdf">http://www.damtp.cam.ac.uk/user/dt281/qft/col1.pdf</a>
and
<a href = "http://www.damtp.cam.ac.uk/user/dt281/qft/col2.pdf">http://www.damtp.cam.ac.uk/user/dt281/qft/col2.pdf</a>

Someone should LaTeX them up!

Even more fun, you can now see \emph{videos} of Coleman teaching quantum
field theory:

5) Sidney Coleman, Physics 253: Quantum Field Theory, 50 lectures
recorded 1975-1976, <a href = "http://www.physics.harvard.edu/about/Phys253.html">http://www.physics.harvard.edu/about/Phys253.html</a>

This is a younger, hipper Coleman than I'd ever seen: long-haired,
sometimes puffing on a cigarette between sentences.  He begins by
saying "Umm... this is Physics 253, a course in relativistic quantum
mechanics.  My name is Sidney Coleman.  The apparatus you see around 
you is part of a CIA surveillance project."

I wish I'd had access to these when I was a kid!

Now for some miraculous math.  Daniel Moskovich kindly pointed out a
paper that describes all the homotopy groups of the 2-sphere, and 
I want to summarize the main result.

I explained the idea of homotopy groups back in "<a href =
"week102.html">week102</A>".  Very roughly, the nth homotopy
group of a space X, usually denoted \pi _{n}(X), is the set of ways you
can map an n-sphere into that space, where we count two ways as the
same if you can continuously deform one to the other.  If a space has
holes, homotopy groups are one way to detect those holes.

Homotopy groups are notoriously hard to compute - so even for so
humble a space as the 2-sphere, S^{2}, there's a sense in which
"nobody knows" all its homotopy groups.  People know the
first 64, though.  Here are a few:

\pi _{1}(S^{2})  = 0    <BR/>
\pi _{2}</sub>(S^{2})  = Z <BR/>
\pi _{3}(S^{2})  = Z <BR/>
\pi _{4}(S^{2})  = Z/2 <BR/>
\pi _{5}(S^{2})  = Z/2 <BR/>
\pi _{6}(S^{2})  = Z/4 \times  Z/3 <BR/>
\pi _{7}(S^{2})  = Z/2 <BR/>
\pi _{8}(S^{2})  = Z/2 <BR/>
\pi _{9}(S^{2})  = Z/3 <BR/>
\pi _{10}(S^{2}) = Z/3 \times  Z/5 <BR/>
\pi _{11}(S^{2}) = Z/2 <BR/>
\pi _{12}(S^{2}) = Z/2 \times  Z/2 <BR/>
\pi _{13}(S^{2}) = Z/2 \times  Z/2 \times  Z/3 <BR/>
\pi _{14}(S^{2}) = Z/2 \times  Z/2 \times  Z/4 \times  Z/3 \times  Z/7 <BR/>
\pi _{15}(S^{2}) = Z/2 \times  Z/2 <BR/>

Apart from the fact that they're all abelian groups, all finite 
except for the first two, it's hard to spot any pattern!  

In fact there's a majestic symphony of patterns in the homotopy 
groups of spheres, starting from ones that are easy to explain 
and working on up to those that push the frontiers of mathematics,
like elliptic cohomology.  But, many of these patterns are too
complex for present-day mathematics until we use some tricks to
"water down" or simplify the homotopy groups.

So, what people often do first is take the limit of
\pi _{n+k}(S^{n}) as n \to  \infty , getting what's
called the kth "stable" homotopy group of spheres.  It's a
wonderful but well-understood fact that these limits really exist.
But so far, even these are too complicated to understand until we work
"at a prime p".  This means that we take the kth stable
homotopy group of spheres and see which groups of the form Z/p^{n} show
up in it.  For example,

\pi _{14}(S^{2}) = Z/2 \times  Z/2 \times  Z/4 \times  Z/3 \times  Z/7

but if we work "at the prime 2" we just see the Z/2 \times  Z/2 \times  Z/4.

After all this data processing, we get some astounding pictures:

<div align = "center">
<a href = "http://www.math.cornell.edu/~hatcher/stemfigs/stems.html">
<img style = "border:none;" src = "stable_homotopy_groups_hatcher.gif">
% </a>
</div>

This picture summarizes the first 999 stable homotopy groups of 
spheres at the prime 5.  To understand exactly what it means, read this:

6) Allen Hatcher, Stable homotopy groups of spheres,
<a href = "http://www.math.cornell.edu/~hatcher/stemfigs/stems.html">http://www.math.cornell.edu/~hatcher/stemfigs/stems.html</a>

Order teetering on the brink of chaos!  If you're brave, you can
learn more about this stuff here:

7) Douglas C. Ravenel, Complex Cobordism and Stable Homotopy Groups
of Spheres, AMS, Providence, Rhode Island, 2003.

If you're less brave, I strongly suggest starting here:

8) Wikipedia, Homotopy groups of spheres, 
<a href = "http://en.wikipedia.org/wiki/Homotopy_groups_of_spheres">http://en.wikipedia.org/wiki/Homotopy_groups_of_spheres</a>

But now, I want to talk about an amazing paper that pursues a
very different line of attack.  It gives a beautiful description 
of <i>all</i> the homotopy groups of S^{2}, in terms of braids:

9) A. Berrick, F. R. Cohen, Y. L. Wong and J. Wu, Configurations, 
braids and homotopy groups, J. Amer. Math. Soc., 19 (2006), 265-326.
Also available at <a href = "http://www.math.nus.edu.sg/~matwujie/BCWWfinal.pdf">http://www.math.nus.edu.sg/~matwujie/BCWWfinal.pdf</a>

For this you need to realize that for any n, there's a group B_{n}
whose elements are n-strand braids.  For example, here's an element
of B_{3}:

\begin{verbatim}
|   |  |  
 \ /   | 
  /    |       
 / \   |
|   \ /    
|    /         
|   / \   
 \ /   | 
  /    |      
 / \   |
|   \ /    
|    /        
|   / \   
 \ /   | 
  /    |      
 / \   |
|   \ /    
|    /         
|   / \   
|  |   |
\end{verbatim}
    

I actually talked about this specific braid back in "<a href =
"week233.html">week233</A>".  But anyway, we count two braids as
the same if you can wiggle one around until it looks like the other
without moving the ends at the top and bottom - which you can think of
as nailed to the ceiling and floor.

How do braids become a group?  Easy: we multiply them by putting 
one on top of the other.  For example, this braid:

\begin{verbatim}
     |   |   |
      \ /    |
A  =   /     |
      / \    |
     |   |   |
\end{verbatim}
    
times this one:

\begin{verbatim}
     |   |   |   
     |    \ /    
B =  |     /    
     |    / \  
     |   |   |
\end{verbatim}
    
equals this:

\begin{verbatim}
     |   |   |
      \ /    |
       /     |
      / \    |
     |   |   |
AB = |   |   |   
     |    \ /    
     |     /    
     |    / \  
     |   |   |
\end{verbatim}
    
and in fact the big one I showed you earlier is (AB)^{3}.  

As you let your eye slide from the top to the bottom of a braid, the
strands move around.  We can visualize their motion as a bunch of
points running around the plane, never bumping into each other.  This
gives an interesting way to generalize the concept of a braid!
Instead of points running around the plane, we can have points running
around S^{2}, or some other surface X.  So, for any surface
X and any number n of strands, we get a "surface braid
group", called B_{n}(X).

As I hinted in "<a href = "week261.html">week261</A>", these
surface braid groups have cool relationships to Dynkin diagrams.  I
urged you to read this paper, and I'll urge you again:


10) Daniel Allcock, Braid pictures for Artin groups, 
available as <a href = "http://arxiv.org/abs/math.GT/9907194">arXiv:math.GT/9907194</a>.

But for now, we just need the "spherical braid group" B_{n}</sub>(S^{2})
together with the usual braid group B_{n}.  

Let's say a braid is "Brunnian" if when you remove any one strand,
the remaining braid becomes the identity: you can straighten out 
all the remaining strands to make them vertical.  It's a fun little
exercise to check that Brunnian braids form a subgroup of all braids.
So, we have an n-strand Brunnian braid group BB_{n}.  

The same idea works for braids on other surface, like the 2-sphere.
So, we also have an n-strand <i>spherical</i> Brunnian braid group
BB_{n}</sub>(S^{2}).

Now, there's obvious map

B_{n} \to  B_{n}</sub>(S^{2})

Why?  An element of B_{n} describes the motion of a bunch of points 
running around the plane, but the plane sits inside the 2-sphere:
the 2-sphere is just the plane with an extra point tacked on.  So,
an ordinary braid gives a spherical braid.

This map clearly sends Brunnian braids to spherical Brunnian braids, 
so we get a map

f: BB_{n} \to  BB_{n}</sub>(S^{2})

And now we're ready for the shocking theorem of Berrick, Cohen, 
Wong and Wu:

\textbf{Theorem:} For n > 3, \pi _{n}(S^{2}) is
BB_{n}</sub>(S^{2}) modulo the image of f.


% parser failed at source line 412
