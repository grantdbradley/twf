
% </A>
% </A>
% </A>
\week{March 7, 1999}

I've been thinking more about neutrinos and their significance for grand
unified theories.  The term "grand unified theory" sounds rather
pompous, but in its technical meaning it refers to something with
limited goals: a quantum field theory that attempts to unify all the
forces \emph{except} gravity.  This limitation lets you pretend spacetime
is flat.  

The heyday of grand unified theories began in the mid-1970s, shortly
after the triumph of the Standard Model.  As you probably know, the
Standard Model is a quantum field theory describing all known particles
and all the known forces except gravity: the electromagnetic, weak and
strong forces.  The Standard Model treats the electromagnetic and weak
forces in a unified way - so one speaks of the "electroweak" force -
but it treats the strong force seperately.

In 1975, Georgi and Glashow invented a theory which fit all the known
particles of each generation into two irreducible representations of
SU(5).  Their theory had some very nice features: for example, it
unified the strong force with the electroweak force, and it explained
why quark charges come in multiples of 1/3.  It also made some new
predictions, most notably that protons decay with a halflife of
something like 10^{29} or 10^{30} years.  Of course, 
it's slightly
inelegant that one needs \emph{two} 
irreducible representations of SU(5) to
account for all the particles of each generation.  Luckily SU(5) fits
inside SO(10) in a nice way, and Georgi used this to concoct a slightly
bigger theory where all 15 particles of each generation, AND ONE MORE,
fit into a single irreducible representation of SO(10).  I described the
mathematics of all this in 
"<A HREF = "week119.html">week119</A>", so I won't do so again here.

What's the extra particle?  Well, when you look at the math, one obvious
possibility is a right-handed neutrino.  As I explained last week, the
existence of a right-handed neutrino would make it easier for neutrinos
to have mass.  And this in turn would allow "oscillations" between
neutrinos of different generations - possibly explaining the mysterious
shortage of electron neutrinos that we see coming from the sun.

This "solar neutrino deficit" had already been seen by 1975, so 
everyone
got very excited about grand unified theories.  The order of the day
was: look for neutrino oscillations and proton decay!

A nice illustration of the mood of the time can be found in a talk
Glashow gave in 1980:

1) Sheldon Lee Glashow, The new frontier, in First Workshop on Grand
Unification, eds. Paul H. Frampton, Sheldon L. Glashow and Asim Yildiz,
Math Sci Press, Brookline Massachusetts, 1980, pp. 3-8.

I'd like to quote some of his remarks because it's interesting to
reflect on what has happened in the intervening two decades:

\begin{quote}
     Pions, muons, positrons, neutrons and strange particles were
     found without the use of accelerators.  More recently, most 
     developments in elementary particle physics depended upon these
     expensive artificial aids.  Science changes quickly.  A time may
     come when accelerators no longer dominate our field: not yet, but
     perhaps sooner than some may think.

     Important discoveries await the next generation of accelerators.
     QCD and the electroweak theory need further confirmation.  We need 
     know how b quarks decay.  The weak interaction intermediaries must
     be seen to be believed.  The top quark (or perversions needed by
     topless theories) lurks just out of range.  Higgs may wait to be 
     found.  There could well be a fourth family of quarks and leptons.
     There may even be unanticipated surprises.  We need the new machines.

\end{quote}

Of course we now know how the b (or "bottom") quark decays,
we've seen the t (or "top") quark, we've seen the weak
interaction intermediaries, and we're quite sure there is not a fourth
generation of quarks and leptons.  There have been no unanticipated
surprises.  Accelerators grew ever more expensive until the
U.S. Congress withdrew funding for the Superconducting Supercollider in
1993.  The Higgs is still waiting to be found or proved nonexistent.
Experiments at CERN should settle that issue by 2003 or so.

\begin{quote}
     On the other hand, we have for the first time an apparently 
     correct \emph{theory} of elementary particle physics.  It may be, in
     a sense, phenomenologically complete.  It suggests the possibility
     that there are no more surprises at higher energies, at least for
     energies that are remotely accessible.  Indeed, PETRA and ISR have
     produced no surprises.  The same may be true for PEP, ISABELLE, and
     the TEVATRON.  Theorists do expect novel higher-energy phenomena,
     but only at absurdly inacessible energies.  Proton decay, if it is 
     found, will reinforce the belief in the great desert extending from
     100 GeV to the unification mass of 10^{14} GeV.  Perhaps 
     the desert
     is a blessing in disguise.  Ever larger and more costly machines
     conflict with dwindling finances and energy reserves.  All frontiers
     come to an end.  

     You may like this scenario or not; it may be true or false.  But,
     it is neither impossible, implausible, or unlikely.  And, do not 
     despair nor prematurely lament the death of particle physics.  We
     have a ways to go to reach the desert, with exotic fauna along the
     way, and even the desolation of a desert can be interesting.  The
     end of the high-energy frontier in no ways implies the end of 
     particle physics.  There are many ways to skin a cat.  In this talk
     I will indicate several exciting lines of research that are well 
     away from the high-energy frontier.  Important results, perhaps even
     extraordinary surprises, await us.  But, there is danger on the way.

     The passive frontier of which I shall speak has suffered years of
     benign neglect.  It needs money and manpower, and it must compete
     for this with the accelerator establishment.  There is no labor
     union of physicists who work at accelerators, but sometimes it seems
     there is.  It has been argued that plans for accelerator construction
     must depend on the "needs" of the working force: several thousands
     of dedicated high-energy experimenters.  This is nonsense.  Future
     accelerators must be built in accordance with scientific, not 
     demographic, prioriries.  The new machines are not labor-intensive, 
     must not be forced to be so.  Not all high energy physicsts can be
     accomodated at the new machines.  The high-energy physicist has no
     guaranteed right to work at an accelerator, he has not that kind of
     job security.  He must respond to the challenge of the passive
     frontier
.
\end{quote}
Of course, the collapse of the high-energy physics job market and the
death of the Superconducting Supercollider give these words a certain
poignancy.  But what is this "passive frontier" Glashow mentions?  It
means particle physics that doesn't depend on very high energy particle
accelerators.  He lists a number of options:

A) CP phenomenology.  The Standard Model is not symmetrical under
switching particles and their antiparticles - called "charge
conjugation", or "C".  Nor is it symmetrical under
switching left and right - called "parity", or "P".
It's almost, but not quite, symmetrical under the combination of both
operations, called "CP".  Violation of CP symmetry is evident
in the behavior of the neutral kaon.  Glashow suggests looking for CP
violation in the form of a nonzero magnetic dipole moment for the
neutron.  As far as I know, this has still not been seen.

B) New kinds of stable matter.  Glashow proposes the search for new
stable particles as "an ambitious and risky field of scientific
endeavor".  People have looked and haven't found anything.

C) Neutrino masses and neutrino oscilllations.  Glashow claims that
"neutrinos should have masses, and should mix".  He now appears to 
be right.  It took almost 20 years for the trickle of experimental
results to become the lively stream we see today, but it happened.
He urges "Let us not miss the next nearby supernova!"  Luckily we
did not.

D) Astrophysical neutrino physics.  In addition to solar neutrinos and
neutrinos from supernovae, there are other interesting connections
between neutrinos and astrophysics.  The background radiation from the
big bang should contain neutrinos as well as the easier-to-see photons.
More precisely, there should be about 100 neutrinos of each generation
per cubic centimeter of space, thanks to this effect.  These "relic
neutrinos" have not been seen, but that's okay: by now they would be too
low in energy to be easily detected.   Glashow notes that if neutrinos
had a nonzero mass, relic neutrinos could contribute substantially to
the total density of the universe.  The heaviest generation weighing 30
eV or so might be enough to make the universe eventually recollapse! On
the other hand, for neutrinos to be gravitationally bound to galaxies,
they'd need to be at least 20 eV or so.   

E) Magnetic monopoles.  Most grand unified theories predict the
existence of magnetic monopoles due to "topological defects"
in the Higgs fields.  Glashow urges people to look for these.  This has
been done, and they haven't been seen.

F) Proton decay.  As Glashow notes, proton decay would be the "king
of the new frontier".  Reflecting the optimism of 1980, he notes
that "to some, it is a foregone conclusion that proton decay is
about to be seen by experiments now abuilding".  But alas, people
looked very hard and did not find it!  This killed the SU(5) theory.
Many people switched to supersymmetric theories, which are more
compatible with very slow proton decay.  But with the continuing lack of
new experiments to explain, enthusiasm for grand unified theories
gradually evaporated, and theoretical particle physics took refuge in
the elegant abstractions of string theory.

But now, 20 years later, interest in grand unified theories seems
to be coming back.  We have a rich body of mysterious experimental
results about neutrino oscillations.  Somebody should explain them!

On a slightly different note, one of my little side hobbies is to study
the octonions and dream about how they might be useful in physics.  One
place they show up is in the E6 grand unified theory - the next theory
up from the SO(10) theory.  I said a bit about this in "<A HREF =
"week119.html">week119</A>", but I just bumped into another paper
on it in the same conference proceedings that Glashow's paper appears
is:

2) Feza Gursey, Symmetry breaking patterns in E_6, in First Workshop on
Grand Unification, eds. Paul H. Frampton, Sheldon L. Glashow and Asim
Yildiz, Math Sci Press, Brookline Massachusetts, 1980, pp. 39-55.

He says something interesting that I want to understand someday - maybe
someone can explain why it's true.  He says that E6 is a "complex"
group, E7 is a "pseudoreal" group, and E8 is a
"real" group.  This use of terminology may be nonstandard, but
what he means is that E6 admits complex representations that are not
their own conjugates, E7 admits complex reps that are their own
conjugates, and that all complex reps of E8 are complexifications of
real ones (and hence their own conjugates).  This should have something
to do with the symmetry of the Dynkin diagram of E6.

Octonions are also prominent in string theory and in the grand unified
theories proposed by my friends Geoffrey Dixon and Tony Smith - see
"<A HREF = "week59.html">week59</A>", "<A HREF =
"week91.html">week91</A>", and "<A HREF =
"week104.html">week104</A>".  I'll probably say more about this
someday....

The reason I'm interested in neutrinos is that I want to learn what
evidence there is for laws of physics going beyond the Standard Model
and general relativity.  This is also why I'm trying to learn a bit of
astrophysics.  The new hints of evidence for a nonzero cosmological
constant, the missing mass problem, the large-scale structure of the
universe, and even the puzzling \gamma -ray bursters - they're all food
for thought along these lines.  

The following book caught my eye since it looked like just what I 
need - an easy tutorial in the latest developments in cosmology:

3) Greg Bothun, Modern Cosmological Observations and Problems, Taylor \text{\&} 
Francis, London, 1998.

On reading it, some of the remarks about particle physics made me
unhappy.  For example, Bothun says "the observed entropy S of the
universe, as measured by the ratio of baryons to photons, is ~ 5 x
10^{-10}." But as Ted Bunn explained to me, the entropy is
actually correlated to the ratio of photons to baryons - the reciprocal
of this number.  Bothun also calls the kinetic energy density of the
field postulated in inflationary cosmology, "essentially an entropy
field that currently drives the uniform expansion and cooling of the
universe".  This makes no sense to me.  There are also a large
number of typos, the most embarrassing being "virilizing" for
"virializing".

But there's a lot of good stuff in this book!  The author's specialty is
large-scale structure, and I learned a lot about that.  Just to set the
stage, recall that the Milky Way has a disc about 30 kiloparsecs in
diameter and contains roughly 100 or 200 billion stars.  But our galaxy
is one of a cluster of about 20 galaxies, called the Local Group.  In
addition to our galaxy and the Large and Small Magellanic Clouds which
orbit it, this contains the Andromeda Galaxy (also known as M31),
another spiral galaxy called M33, and a bunch of dwarf irregular
galaxies.  The Local Group is about a megaparsec in radius.

This is typical.  Galaxies often lie in clusters which are a few
megaparsecs in radius, containing from a handful to hundreds of big
galaxies.  Some famous nearby clusters include the Virgo cluster (about
20 megaparsecs away) and the Coma cluster (about 120 megaparsecs away).
Thousands of clusters have been cataloged by Abell and collaborators.

And then there are superclusters, each typically containing 3-10
clusters in an elongated "filament" about 50 megaparsecs in
diameter.  I don't mean to make this sound more neat than it actually
is, because nobody is very sure about intergalactic distances, and the
structures themselves are rather messy.  But there are various discernible
patterns.  For example, superclusters tend to occur at the edges of
larger roundish "voids" which have few galaxies in them.
These voids are very large, about 100 or 200 megaparsecs across.  In
general, galaxies tend to be falling into denser regions and moving away
from the voids.  For example, the Milky Way is falling towards the
center of the Local Supercluster at about 300 kilometers per second, and
the Local Supercluster is also falling towards the next nearest one -
the Hydra-Centaurus Supercluster - at about 300 kilometers per second.

Now, if the big bang theory is right, all this stuff was once very 
small, and the universe was much more homogeneous.  Obviously gravity
tends to amplify inhomogeneities.  The problem is to understand in a
quantitative way how these inhomogeneities formed as the universe grew.

Here are a couple of other books that I'm finding useful - they're
a bit more mathematical than Bothun's.  I'm trying to stick to new
books because this subject is evolving so rapidly:

4) Jayant V. Narlikar, Introduction to Cosmology, Cambridge U. 
Press, Cambridge, 1993.

5) Peter Coles and Francesco Lucchin, Cosmology: The Origin and
Evolution of Cosmic Structure, Wiley, New York, 1995.

While I was looking around, I also bumped into the following book on
black holes:

6) Sandip K. Chakrabarti, ed., Observational Evidence for Black Holes
in the Universe, Kluwer, Dordrecht, 1998.

It mentioned some objects I'd never heard of before.  I want to tell you
about them, just because they're so cool!

X-ray novae: First, what's a nova?  Well, remember that a white dwarf is
a small, dense, mostly burnt-out star.  When one member of a binary star
is a white dwarf, and the other dumps some of its gas on this white
dwarf, the gas can undergo fusion and emit a huge burst of energy - as
much as 10,000 times what the sun emits in a year.  To astronomers it
may look like a new star is suddenly born - hence the term
"nova".  But not all novae emit mostly visible light - some
emit X-rays or even \gamma  rays.  A "X-ray nova" is an X-ray
source that suddenly appears in a few days and then gradually fades away
in a year or less.  Many of these are probably neutron stars rather than
white dwarfs.  But a bunch are probably black holes!

Blazars: A "blazar" is a galactic nucleus that's shooting out
a jet of hot plasma almost directly towards us, exhibiting rapid
variations in power.  Like quasars and other active galactic nuclei,
these are probably black holes sucking in stars and other stuff and
forming big accretion disks that shoot out jets of plasma from both
poles.

Mega masers: A laser is a source of coherent light caused by stimulated
emission - a very quantum-mechanical gadget.  A maser is the same sort
of thing but with microwaves.  In fact, masers were invented before
lasers - they are easier to make because the wavelength is longer.  In
galaxies, clouds of water vapor, hydroxyl, silicon monoxide, methanol
and other molecules can form enormous natural masers.  In our galaxy the
most powerful such maser is W49N, which has a power equal to that of the
Sun.  But recently, still more powerful masers have been bound in other
galaxies, usually associated with active galactic nuclei.  These are
called "mega masers" and they are probably powered by black
holes.  The first mega maser was discovered in 1982; it is a hydroxyl
ion maser in the galaxy IC4553, with a luminosity about 1000 times that
of our sun.  Subsequently people have found a bunch of water mega
masers.  The most powerful so far is in TXFS2226-184 - it has a
luminosity of about 6100 times that of the Sun!

\par\noindent\rule{\textwidth}{0.4pt}
Addendum: Here is something from Allen Knutson in response to my remark
that E6 has complex representations that aren't their own conjugates.
I hoped that this is related to the symmetry of the Dynkin diagram
of E6, and Allen replied:


\begin{verbatim}

It does. The automorphism G\to G that exchanges representations with their
duals, the Cartan involution, may or may not be an inner automorphism.
The group of outer automorphisms of G (G simple) is iso to the diagram
automorphism group. So no diagram auts, means the Cartan involution is inner,
means all reps are iso to their duals, i.e. possess invariant bilinear forms.

(Unfortunately it's not iff - the D_{n}'s alternate between whether the
Cartan involution is inner, much as their centers alternate between
Z_{4} and Z_{2}^{2}.)

Any rep either has complex trace sometimes, or a real, or a quaternionic
structure, morally because of Artin-Wedderburn applied to the real
group algebra of G. Given a rep one can find out which by looking at
the integral over G of Tr(g^{2}), which comes out 0, 1, or -1 (respectively).  
This is the "Schur indicator" and can be found in Serre's LinReps of 
Finite Groups.
								Allen K.
\end{verbatim}
    





\par\noindent\rule{\textwidth}{0.4pt}
% </A>
% </A>
% </A>
