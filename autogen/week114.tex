
% </A>
% </A>
% </A>
\week{January 12, 1998}

Classes have started!  But I just flew back yesterday from the Joint
Mathematics Meetings in Baltimore - the big annual conference organized
by the AMS, the MAA, SIAM, and other societies.  Over 4000
mathematicians could be seen wandering in clumps about the glitzy harbor
area and surrounding crime-ridden slums, arguing about abstractions,
largely oblivious to the world around them.  Everyone ate the obligatory
crab cakes for which Baltimore is justly famous.  Some of us drank a bit
too much beer, too.  

Witten gave a plenary talk on "M-theory", which was great fun even
though he didn't actually say what M-theory is.  Steve Sawin and I ran a
session on quantum gravity and low-dimensional topology, so I'll say a
bit about what went on there.  There was also a nice session on homotopy
theory in honor of J. Michael Boardman.   I'll talk about that and
various other things next week.

A lot of the buzz in our session concerned the new "spin foam" approach
to quantum gravity which I discussed in "<A HREF = "week113.html">week113</A>".  The big questions
are: how do you test this approach without impractical computer
simulations?  Lee Smolin's paper below suggests one way.  Should you
only sum over spin foams that are dual to a particular triangulation of
spacetime, or should you sum over all spin foams that fit in a
particular 4-dimensional spacetime manifold, or should you sum over
\emph{all} spin foams?  There was a lot of argument about this.  In addition
to the question of what is physically appropriate, there's the
mathematical problem of avoiding divergent infinite sums.  Perhaps the
sum required to answer any truly physical question only involves
finitely many spin foams - that's what I hope.  Finally, should the
time evolution operators constructed using spin foams be thought of as
describing true time evolution, or merely the projection onto the kernel
of the Hamiltonian constraint?  While it sounds a bit technical, this
question is crucial for the interpretation of the theory; it's part of
what they call "the problem of time".  

Carlo Rovelli spoke about how spin foams arise in canonical quantum
gravity, while John Barrett and Louis Crane discussed them in the
context of discretized path integrals for quantum gravity, also known as
state sum models.  As in the more traditional "Regge calculus" approach,
these models start by chopping spacetime into simplices.  The biggest
difference is that now \emph{areas of triangles} play a more important role
than lengths of edges.  But Barrett, Crane and others are starting to
explore the relationships:

1) John W. Barrett, Martin Rocek, Ruth M. Williams, A note on area 
variables in Regge calculus, preprint available as <A HREF = "http://xxx.lanl.gov/abs/gr-qc/9710056">gr-qc/9710056</A>.

2) Jarmo Makela, Variation of area variables in Regge calculus
preprint available as <A HREF = "http://xxx.lanl.gov/abs/gr-qc/9801022">gr-qc/9801022</A>.  

Also, there's been some progress on extracting Einstein's equation for
general relativity as a classical limit of the Barrett-Crane state
sum model.  Let me quote the abstract of this paper:

3) Louis Crane and David N. Yetter, On the classical limit of the
balanced state sum, preprint available as <A HREF = "http://xxx.lanl.gov/abs/gr-qc/9712087">gr-qc/9712087</A>.

"The purpose of this note is to make several advances in the
interpretation of the balanced state sum model by Barrett and Crane in
<A HREF = "http://xxx.lanl.gov/abs/gr-qc/9709028">gr-qc/9709028</A> as a quantum theory of gravity. First, we outline a
shortcoming of the definition of the model pointed out to us by Barrett
and Baez in private communication, and explain how to correct
it. Second, we show that the classical limit of our state sum reproduces
the Einstein-Hilbert lagrangian whenever the term in the state sum to
which it is applied has a geometrical interpretation. Next we outline a
program to demonstrate that the classical limit of the state sum is in
fact dominated by terms with geometrical meaning. This uses in an
essential way the alteration we have made to the model in order to fix
the shortcoming discussed in the first section. Finally, we make a brief
discussion of the Minkowski signature version of the model."

Lee Smolin talked about his ideas for relating spin foam models 
to string theory.  He has a new paper on this, so I'll just
quote the abstract:

4) Lee Smolin, Strings as perturbations of evolving spin-networks,
preprint available as <A HREF = "http://xxx.lanl.gov/abs/hep-th/9801022">hep-th/9801022</A>.

"A connection between non-perturbative formulations of quantum gravity
and perturbative string theory is exhibited, based on a formulation of
the non-perturbative dynamics due to Markopoulou.  In this formulation
the dynamics of spin network states and their generalizations is
described in terms of histories which have discrete analogues of the
causal structure and many fingered time of Lorentzian spacetimes.
Perturbations of these histories turn out to be described in terms of
spin systems defined on 2-dimensional timelike surfaces embedded in the
discrete spacetime. When the history has a classical limit which is
Minkowski spacetime, the action of the perturbation theory is given to
leading order by the spacetime area of the surface, as in bosonic string
theory.  This map between a non-perturbative formulation of quantum
gravity and a 1+1 dimensional theory generalizes to a large class of
theories in which the group SU(2) is extended to any quantum group or
supergroup.  It is argued that a necessary condition for the
non-perturbative theory to have a good classical limit is that the
resulting 1+1 dimensional theory defines a consistent and stable
perturbative string theory."

Fotini Markopolou spoke about her recent work with Smolin on
formulating spin foam models in a manifestly local, causal
way.  

5) Fotini Markopoulou and Lee Smolin, Quantum geometry with intrinsic
local causality, preprint available as <A HREF = "http://xxx.lanl.gov/abs/gr-qc/9712067">gr-qc/9712067</A>.

"The space of states and operators for a large class of background
independent theories of quantum spacetime dynamics is defined.  The SU(2)
spin networks of quantum general relativity are replaced by labelled
compact two-dimensional surfaces.  The space of states of the theory is
the direct sum of the spaces of invariant tensors of a quantum group G_q
over all compact (finite genus) oriented 2-surfaces.  The dynamics is
background independent and locally causal. The dynamics constructs
histories with discrete features of spacetime geometry such as causal
structure and multifingered time.  For SU(2) the theory satisfies the
Bekenstein bound and the holographic hypothesis is recast in this
formalism."

The main technical idea in this paper is to work with "thickened" or
"framed" spin networks, which amounts to replacing graphs by
solid handlebodies.  One expects this "framing" business to be
important for quantum gravity with nonzero cosmological constant.  
This framing business also appears in the q-deformed version of 
Barrett and Crane's model and in my "abstract" version of 
their model, which assumes no background spacetime manifold.
Markopoulou and Smolin don't specify a choice of dynamics; instead,
they describe a \emph{class} of theories which has my model as a
special case, though their approach to causality is better suited
to Lorentzian theories, while mine is Euclidean.   

As I've often noted, spin foams are about spacetime geometry,
or dynamics, while spin networks are a way of describing the
geometry of space, or kinematics.  Kinematics is always easier
than dynamics, so the spin network approach to the quantum geometry
of space has been much better worked out than the new spin foam stuff.
Abhay Ashtekar gave an overview of these kinematical issues in his
talk on "quantum Riemannian geometry", and Kirill Krasnov described
how our understanding of these already allows us to compute the entropy 
of black holes (see "<A HREF = "week112.html">week112</A>").  
Here it's worth mentioning that the second part of Ashtekar's paper 
with Jerzy Lewandowski is finally out:

6) Abhay Ashtekar and Jerzy Lewandowski, Quantum theory of geometry II:
volume operators, preprint available as <A HREF = "http://xxx.lanl.gov/abs/gr-qc/9711031">gr-qc/9711031</A>.  

"A functional calculus on the space of (generalized) connections was
recently introduced without any reference to a background metric.  It is
used to continue the exploration of the quantum Riemannian geometry.
Operators corresponding to volume of three-dimensional regions are
regularized rigorously.  It is shown that there are two natural
regularization schemes, each of which leads to a well-defined operator.
Both operators can be completely specified by giving their action on
states labelled by graphs.  The two final results are closely related
but differ from one another in that one of the operators is sensitive to
the differential structure of graphs at their vertices while the second
is sensitive only to the topological characteristics.  (The second
operator was first introduced by Rovelli and Smolin and De Pietri and
Rovelli using a somewhat different framework.)  The difference between
the two operators can be attributed directly to the standard
quantization ambiguity.  Underlying assumptions and subtleties of
regularization procedures are discussed in detail in both cases because
volume operators play an important role in the current discussions of
quantum dynamics."

Before spin foam ideas came along, the basic strategy in the loop
representation of quantum gravity was to start with general relativity
on a smooth manifold and try to quantize it using the "canonical 
quantization" approach.  Here the most important
and difficult thing is to implement the "Hamiltonian constraint" 
as an operator on the Hilbert space of kinematical states, so you can write
down the Wheeler-deWitt equation, which is, quite roughly speaking, the
quantum gravity analog of Schrodinger's equation.  (For a summary of
this approach, try "<A HREF = "week43.html">week43</A>".)

The most careful attempt to do this so far is the work of Thiemann:

7) Thomas Thiemann, Quantum spin dynamics (QSD), preprint 
available as <A HREF = "http://xxx.lanl.gov/abs/gr-qc/9606089">gr-qc/9606089</A>.

Quantum spin dynamics (QSD) II, preprint available as
<A HREF = "http://xxx.lanl.gov/abs/gr-qc/9606090">gr-qc/9606090</A>.

QSD III: Quantum constraint algebra and physical scalar
product in quantum general relativity, preprint available as
<A HREF = "http://xxx.lanl.gov/abs/gr-qc/9705017">gr-qc/9705017</A>.

QSD IV: 2+1 Euclidean quantum gravity as a model to test
3+1 Lorentzian quantum gravity, preprint available as <A HREF = "http://xxx.lanl.gov/abs/gr-qc/9705018">gr-qc/9705018</A>.

QSD V: Quantum gravity as the natural regulator of matter
quantum field theories, preprint available as <A HREF = "http://xxx.lanl.gov/abs/gr-qc/9705019">gr-qc/9705019</A>.

QSD VI: Quantum Poincare algebra and a quantum positivity of energy
theorem for canonical quantum gravity, preprint available as
<A HREF = "http://xxx.lanl.gov/abs/gr-qc/9705020">gr-qc/9705020</A>

Kinematical Hilbert spaces for fermionic and Higgs quantum
field theories, <A HREF = "http://xxx.lanl.gov/abs/gr-qc/9705021">gr-qc/9705021</A>

If everything worked as smoothly as possible, the Hamiltonian constraint
would satisfy nice commutation relations with the other constraints of
the theory, giving a representation of something called the "Dirac
algebra".  However, as Don Marolf explained in his talk, this doesn't
really happen, at least in a large class of approaches including
Thiemann's:

8) Jerzy Lewandowski and Donald Marolf, Loop constraints: A habitat and
their algebra, preprint available as <A HREF = "http://xxx.lanl.gov/abs/gr-qc/9710016">gr-qc/9710016</A>.  

9) Rodolfo Gambini, Jerzy Lewandowski, Donald Marolf, and Jorge Pullin, 
On the consistency of the constraint algebra in spin network quantum
gravity, preprint available as <A HREF = "http://xxx.lanl.gov/abs/gr-qc/9710018">gr-qc/9710018</A>.

This is very worrisome... as everything concerning quantum gravity
always is.  Personally these results make me want to spend less time on
the Hamiltonian constraint, especially to the extent that it assumes a
the old picture of spacetime as a smooth manifold, and more time on
approaches that start with a discrete picture of spacetime.  However,
the only way to make serious progress is for different people to push on
different fronts simultaneously.  

There were a lot of other interesting talks, but since I'm concentrating
on quantum gravity here I won't describe the ones that were mainly about
topology.  I'll wrap up by mentioning Steve Carlip's talk on spacetime
foam.  He gave a nice illustration to how hard it is to "sum over
topologies" by arguing that this sum diverges for negative values of the
cosmological constant.  He has a paper out on this:

10) Steven Carlip, Spacetime foam and the cosmological constant,
Phys. Rev. Lett. 79 (1997) 4071-4074, preprint available as 
<A HREF = "http://xxx.lanl.gov/abs/gr-qc/9708026">gr-qc/9708026</A>.

Again, I'll quote the abstract:

"In the saddle point approximation, the Euclidean path integral for
quantum gravity closely resembles a thermodynamic partition function,
with the cosmological constant \Lambda  playing the role of temperature
and the ``density of topologies'' acting as an effective density of
states. For \Lambda  < 0, the density of topologies grows
superexponentially, and the sum over topologies diverges. In
thermodynamics, such a divergence can signal the existence of a maximum
temperature. The same may be true in quantum gravity: the effective
cosmological constant may be driven to zero by a rapid rise in the
density of topologies."








 \par\noindent\rule{\textwidth}{0.4pt}

% </A>
% </A>
% </A>
