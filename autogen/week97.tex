
% </A>
% </A>
% </A>
\week{February 8, 1997 }



I've taken a break from writing This Week's Finds in order to
finish up a paper with James Dolan in which we give a definition of
"weak n-categories" for all n.  This paper is now available
on my website, and I'm immodestly eager to talk about it, and I will,
but a lot of stuff has accumulated in the meantime which I want to
discuss first.

First, I'm sure you remember a while back when atoms were first coaxed
to form true Bose-Einstein condensates.  The basic idea is that
particles come in two basic kinds, fermions and bosons, and while the
fermions have half-integer spin and obey the Pauli exclusion principle
saying that no two identical fermions can be in the same state at the
same time, bosons have integer spin and are gregarious: they \emph{love} to
be in the same state at the same time.

Why is spin related to what happens when you try to put a bunch of
particles in the same state?  Well, it all has to do with the relation
between twisting something around:


\begin{verbatim}

|
|
 \
  \  /\
    /  \
   /   /
  /  \/
 /
|
|

\end{verbatim}
    
and switching two things:


\begin{verbatim}

\     /
 \   /
  \ /
   /
  / \
 /   \
/     \

\end{verbatim}
    
To understand this, try

1) Spin, statistics, CPT and all that jazz,  <A HREF = "http://math.ucr.edu/home/baez/spin.stat.html">http://math.ucr.edu/home/baez/spin.stat.html</A>

But let's consider some examples.  Since photons have spin 1 they are
bosons.  In laser light one has a bunch of photons all in the
same state.  Thanks to the Heisenberg uncertainty principle, of
course, we can't know both their position and momentum.  In a laser we
don't know the position of the photons: each photon is all over the
laser beam in a spread-out sort of way.  However, we do know the
momentum of the photons and they all have the same momentum.  This
means that we have "coherent light" in which all the photons are like
waves wiggling perfectly in phase.  One can demonstrate this by
interfering two beams of laser light and seeing beautifully perfect
interference fringes, bright and dark stripes in places where the two
beams are either in phase with each other and adding up, or out of
phase and cancelling out.

Now, other particles are bosons as well, and they can do similar
tricks.  Bose and Einstein predicted that when any gas of
noninteracting bosons gets sufficiently cold, all --- or at least a
sizable fraction --- of them will be found in the same state: the
state of least possible energy.  Unfortunately, when things get cold
they are usually liquids or solids rather than a gas, and the
particles in a liquid or gas interact a lot, so true Bose-Einstein
condensation is hard to achieve.

Some related things have been studied for decades.  If you get an even
number of fermions together they act approximately like a boson, at
least if the density is not too high.  Helium stays liquid at
temperatures arbitrarily close to absolute zero, when the pressure is
low enough.  Since helium 4 has 2 protons, 2 neutrons, and 2
electrons, and all these particles are fermions, helium 4 acts like a
boson.  At really low temperatures, helium 4 becomes "superfluid" ---
a substantial fraction of the atoms fall into the same state and the
liquid acquires shocking properties, like zero viscosity.  Similarly,
in certain metals at low temperatures electrons will, by a subtle
mechanism, form "Cooper pairs", and these act like bosons.  When a
bunch of these fall into the same state, you have a "superconductor".

But neither of these is a Bose-Einstein condensate in the technical
sense of the term, because the helium atoms interact a lot in superfluid
helium, and the Cooper pairs interact a lot in a superconductor.
Only recently have people been able to get dilute gases of bosonic atoms
cold enough to study true Bose-Einstein condensation.  

The fist team to do it, the "JILA" team in Boulder, Colorado got a
Bose-Einstein condensate of about 2000 rubidium atoms to form in a
magnetic trap at less than 2 x 10^{-7} degrees above absolute zero.  A
team at Rice University did it with lithium soon after, followed by a
team at MIT, who did it with sodium.

Check out:

2) Physicists create new state of matter, <A HREF = "http://jilav1.colorado.edu/www/bose-ein.html">http://jilav1.colorado.edu/www/bose-ein.html</A>

Atomcool home page, <A HREF = "http://atomcool.rice.edu/">http://atomcool.rice.edu/</A>

Neutral sodium ion trap at MIT, <A HREF = "http://bink.mit.edu/dallin/nat.html">http://bink.mit.edu/dallin/nat.html</A>

So what's the news?  Well, recently the team at MIT, led by Wolfgang
Ketterle, made two blobs of Bose-Einstein condensate out of sodium
atoms.  Ramming these into each other, they were able to see
interference fringes just as in a laser!  In other words, they is
seeing interference of matter waves, just as quantum mechanics
predicts, but involving lots of atoms in a coherent state rather than
a single electron as in the famous double slit experiment.  For
pictures and even movies, try:

3) Matter-wave interference of two Bose condensates,
<A HREF = "http://bink.mit.edu/dallin/news.html#matterwave">http://bink.mit.edu/dallin/news.html#matterwave</A>

In honor of this event, I hereby present the following limerick
composed by the poet Lisa Raphals, with myself serving as science
consultant.  It may aid your appreciation if I note first that
"Squantum" is an actual town in Massachusetts.  With no further ado:



\begin{verbatim}


       A metaphysician from Squantum
       Was asked, what's the state of the quantum?
       It's all reciprocity:
       Position, velocity -
       They're never both there when you want 'em!

\end{verbatim}
    
Now on to some more technical stuff....

I am now visiting the Center for Gravitational Physics and Geometry
here at Penn State, which is a delightful place for people interested
in the loop representation of quantum gravity (see "<A HREF = "week77.html">week77</A>").  Right
now everyone is working on using the loop representation to derive
Hawking's formula which says that the entropy of a black hole is
proportional to the surface area of its event horizon.  

When I arrived, Jorge Pullin handed me a copy of his book:

4) Rodolfo Gambini and Jorge Pullin, "Loops, knots, gauge theories,
and quantum gravity", Cambridge U. Press, Cambridge, 1996.

Here is the table of contents:
<OL>
<LI> Holonomies and the group of loops
<LI> Loop coordinates and the extended group of loops
<LI> The loop representation
<LI> Maxwell theory
<LI> Yang-Mills theories
<LI> Lattice techniques
<LI> Quantum gravity
<LI> The loop representation of quantum gravity
<LI> Loop representation: further developments
<LI> Knot theory and physical states of quantum gravity
<LI> The extended loop representation of quantum gravity
<LI> Conclusions, present status and outlook
</OL>

This is presently the most complete introduction available to the "loop
representation" concept, as applied to electromagnetism, Yang-Mills
theory, and quantum gravity.  Gambini was one of the original
inventors of this notion, and this book covers the whole sweep of its
ramifications, with a special emphasis on a particular technical form,
the "extended loop representation", which he has been developing with
Pullin and other collaborators.  

What the heck is the loop representation, anyway?  Well, all the
forces we know are described by gauge theories, and gauge theories
all describe the "phase", or generalization thereof, that a particle
acquires when you carry it around a loop.  In the case of electromagnetism,
for example, a charged quantum particle carried around a loop in space
acquires a phase equal to 

exp(-iqB/\hbar )

where q is the particle's charge, \hbar  is Planck's constant, and B is
the magnetic flux through the loop: i.e., the integral of the magnetic
field over any surface spanning the loop.  Knowing these phase for all
loops is the same as knowing the magnetic field.  Similarly, if we
knew the phase for all loops in SPACETIME instead of just space, we
would know both the electric and magnetic fields throughout spacetime.

General relativity is similar except that instead of a phase one
gets a rotation, or more generally a Lorentz transformation, when
one parallel transports a little arrow around a loop.  

The theories of the electroweak and strong forces are similar but
the analog of the "phase" is a bit more abstract: an element of the
group SU(2) x U(1) or SU(3), respectively.

The idea of the loop representation is to take these "phases acquired
around loops" as basic variables for describing the laws of physics.

That's the idea in a nutshell.  It turns out, not surprisingly, 
that there are many interesting relationships with such topics
involving loops, such as string theory and knot theory.

Gambini and Pullin's book develops this theme in many directions.  Let
me say a bit about one fascinating topic that they mention, which I
would like to understand better: Gerard 't Hooft's work on confinement
in chromodynamics using his "order and disorder operators".

I explained some basic ideas about confinement and asymptotic freedom
in "<A HREF = "week84.html">week84</A>" and "<A HREF = "week94.html">week94</A>", so I'll assume you've read that stuff.
Remember, the basic idea of confinement is that if you take a meson
and try to pull the quark and antiquark it contains apart, the force
required does not decrease with distance like 1/r^2, because the
chromoelectric field --- the strong force analog of the electric field
--- does not spread out in all directions like an ordinary electric
field does.  Instead, all the field lines are confined to a "flux
tube", so the force is roughly independent of the distance.

This means that the energy is roughly proportional to the distance.
Since action has dimensions of energy times time, this means that if
we consider the creation and subsequent annihilation of
a virtual quark-antiquark pair: 


\begin{verbatim}

                  /\
                 /  \
                /    \
                \    /
                 \  /
                  \/

\end{verbatim}
    
the total action is proportional to the \emph{area} of the loop traced out
in spacetime.  Here I am neglecting the action due to the kinetic
energy of the quark and antiquark, and only worrying about the
potential energy due to the flux tube joining them.  This amounts to
treating the quark and antiquark as "test particles" to study the
behavior of the strong force.

Now, when we study quantum physics using Euclidean path integrals the
basic principle is that the probability of the occurence of any
process is proportional to

exp(-S/\hbar ) 

where S is the action of that process and \hbar  is Planck's constant
again.  So in this framework the \emph{probability} of a particular virtual 
quark-antiquark pair tracing out a loop like the above one is proportional
to 

exp(-cA)

where c > 0 is some constant and A is the area of the loop.  This
"area law" was first proposed by Kenneth Wilson in his pioneering work
on confinement; he proposed it as a way to tell, mathematically, if
confinement was happening in some theory.  Just compute the
probability of a virtual quark-antiquark pair tracing out a particular
loop and see if it decreases exponentially with the area!

Deriving confinement from chromodynamics is something that people have
worked on for quite a while, and it's not easy: there is still no
rigorous proof, even though there are a bunch of heuristic arguments
for it, and computer simulations seem to demonstrate that it's bound
to occur.  One approach to studying the puzzle is due to 't Hooft and
involves "order" and "disorder" operators.  

I'll explain what these are, and what they have to do with knot
theory, but not how 't Hooft actually uses them in his argument for
confinement.  For the actual argument, try Gambini and Pullin's book,
or else 't Hooft's paper:

5) Gerard 't Hooft, Nucl. Phys. B138, (1978) 1.

Let us work in space at a given time, rather than in the Euclidean
path integral approach.  We'll do "canonical quantization", meaning that
now observables will be operators on some Hilbert space.  

If we have any loop g in space, there is an observable called the
"Wilson loop" W(g), which is the trace of the holonomy of the
connection around g.  The precise way of stating Wilson's area law for
confinement in this context is that 

<W(g)> ~ exp(-cA)

where <W(g)> is the vacuum expectation value of the Wilson loop, and
A is the area spanned by the loop g.  The point is that <W(g)> is
the same as what I was (a bit sloppily) calling the probability of the
quark-antiquark pair tracing out the loop g.  

't Hooft calls the Wilson loops "order operators".  We don't really
need to worry why he calls them this, but if you know how physicists
think, you may know that the Wilson loops are keeping track of a kind
of "order parameter" of the vacuum state.  Anyway, his idea was to
study the Wilson loops by introducing some other operators he called
"disorder operators".

Chromodynamics is an SU(3) gauge theory but it's a little clearer if
we work with any SU(N) gauge theory.  Notice that the center of the
group SU(N) consists of the matrices of the form 

exp(2 \pi  i n/N)

where n is an integer.  So if we have a loop h, we can imagine an
operator that does the following thing: it modifies the connection, or
vector potential, in such a way that if you do parallel transport
around a tiny loop linking h once, the holonomy changes to exp(2 \pi  i/N)
times what it had been.  Note: this is a gauge-invariant thing to do,
because that exp(2 \pi  i/N) is in the center of SU(N)!  So just as the
Wilson loop observables are gauge-invariant, we can hope for some some
"disorder operators" V(h) that modify the connection in this way.

If you think about it, what this means is that the following
commutation relations hold:

W(g) V(h) = V(h) W(g) exp(2 \pi  i L(g,h)/N)

where L(g,h) is the linking number of the loops g and h, which counts
how many times g wraps around h.  

There is an obvious symmetry or "duality" between the V's and the W's
going on here.  In fact, just as W's satisfy an area law where there
is confinement of chromoelectric field lines into flux tubes, I
believe the V's satisfy an area law when there is confinement of
chromomagnetic field lines into flux tubes.  The simplest case of this
kind of thing occurs in plain old electromagnetism, where plain old
magnetic field lines are confined into flux tubes in type II
superconductors.  For this reason confinement of electric field lines
is sometimes called "dual superconductivity".

Perhaps the simplest way of beginning to understand this stuff more
deeply is to understand the wonderful formula proved by Ashtekar and
Corichi in the following paper:

6) Abhay Ashtekar and Alejandro Corichi, Gauss linking number and
electro-magnetic uncertainty principle, preprint available as 
<A HREF = "http://xxx.lanl.gov/ps/hep-th/9701136">hep-th/9701136</A>.

This formula applies to plain old electromagnetism, or more precisely,
quantum electrodynamics.  If we work in units where \hbar  = 1, and
consider a particle of charge 1, the Wilson loop operator W(g) in
electromagnetism is just

W(g) = exp(-iB(g)) 

where B is the magnetic flux flowing through the loop g.  But instead
we can just work with B(g) directly.  Similarly, instead of V(h)'s
we can work with the operator E(h) corresponding to the electric flux
through the loop h.  Then we have

B(g) E(h) - E(h) B(g) = i L(g,h)

In other words, the electric and magnetic fields don't commute in
quantum electrodynamics, and the Heisenberg uncertainty of the
electric field flowing through a loop g and the magnetic field flowing
through a loop h is proportional to the linking number of g and h!

Quantum mechanics, electromagnetism, and knot theory are clearly quite
tangled up here.  Since the linking number was first discovered by
Gauss in his work on magnetism, it's all quite fitting.  

And that leads me to the last paper I want to mention this week.  It
should be of great interest to Vassiliev invariant fans; see "<A HREF = "week3.html">week3</A>"
if you don't know what a Vassiliev invariant is.

7) Dror Bar-Natan and Alexander Stoimenow, The fundamental theorem of 
Vassiliev invariants, preprint available as <A HREF = "http://xxx.lanl.gov/ps/q-alg/9702009">q-alg/9702009</A>.

Let me just quote the abstract here:

  The "fundamental theorem of Vassiliev invariants" says that every weight
system can be integrated to a knot invariant. We discuss four different
approaches to the proof of this theorem: a topological/combinatorial approach
following M. Hutchings, a geometrical approach following Kontsevich, an
algebraic approach following Drinfel'd's theory of associators, and a physical
approach coming from the Chern-Simons quantum field theory. Each of these
approaches is unsatisfactory in one way or another, and hence we argue that we
still don't really understand the fundamental theorem of Vassiliev invariants.

\par\noindent\rule{\textwidth}{0.4pt}

% </A>
% </A>
% </A>
