
% </A>
% </A>
% </A>
\week{August 10, 2006 }



This Week I'd like to talk about math books in Shanghai, 
and Urs Schreiber's blog entry on 
the gauge 3-group of M-theory.   But first:

<DIV ALIGN = CENTER>
<IMG SRC = "klein_quartic_egan.gif">
</DIV>

1) Greg Egan, Klein's quartic equation,
<A HREF = "http://gregegan.customer.netspace.net.au/SCIENCE/KleinQuartic/KleinQuarticEq.html">http://gregegan.customer.netspace.net.au/SCIENCE/KleinQuartic/KleinQuarticEq.html</A>

I discussed Klein's quartic curve in "<A HREF = "week214.html">week214</A>" and "<A HREF = "week215.html">week215</A>".
The idea is to take the nontrivial complex solutions of

u^{3} v + v^{3} w + w^{3} u = 0

and "projectivize" them - in other words, count two 
as the same if one is just a multiple of the other:

(u',v',w') = c(u,v,w)

The result is a 3-holed Riemann surface with the maximum number of
symmetries!  Here by a "symmetry" I mean a conformal
transformation mapping the surface to itself.  Back in 1893 Hurwitz
proved something quite bizarre: an n-holed Riemann surface can't
have more than 84(n-1) symmetries if n > 1.  So, a 3-holed Riemann
surface can't have more than 168 symmetries - and Klein's quartic
curve has exactly that many!

These 168 symmetries were constructed by Klein way back in 1879, 
but Egan gives an elementary proof that uses only algebra and a 
bit of calculus...  and a lot of cleverness.  And, his page has 
a wonderful spinning picture of the \emph{real} solutions of 

u^{3} v + v^{3} w + w^{3} u = 0.

This is what you see above.

As you can see, it consists of lots of lines through the origin,
including the u, v, and w axes.  When we "projectivize", we get 
one point for each of these lines, so we get a curve which is the 
real version of Klein's quartic curve.  This curve has an obvious 
3-fold symmetry, from cyclically permuting the coordinate axes.
The rest of the 168 symmetries are only easy to visualize when we go to 
the complex version - as Egan explains.  

It's great that Egan can draw this thing in Perth and I can easily
see it my apartment here in Shanghai - I feel like I'm living in a 
futuristic world, and I'm only 45.  What it'll be like when I'm 64?

Another pleasant thing about life in Shanghai, at least for a 
well-off visitor from America, is how cheap everything is.  
It's clear why the US is running an enormous trade deficit: 
there's a vast economic differential.  Stopping the flow of 
goods one way and dollars the other would be like damming the 
Niagara Falls.  

For example, last night I saw this excellent hardcover book 
on sale for 78 yuan, or about $10:

2) Yu. I. Manin and A. A. Panchishkin, Introduction to Modern
Number Theory, second edition, Science Press, 2005.

It's a great overview of number theory, from the basics through
class field theory to L-functions, modular forms and the Langlands 
program!  It's wisely divided into three sections: "problems and
tricks", "ideas and theories", and "analogies and visions".  
Back when I used to hate number theory, I thought it was all
problems and tricks.  Now I'm beginning to learn some of the 
ideas and theories, and I hope eventually to grasp the analogies 
and visions discussed here - for example, the analogy between 
Arakelov geometry and noncommutative geometry.

$10 is a nice price for a math book.   If you buy this one from 
Springer Verlag, you'll pay ten times that.  Illegal knockoffs 
of Western books are common in China, but I think the one I saw 
is legal, since Springer has signed an agreement with Science 
Press, which is run by the Chinese Academy of Sciences.  In 
exchange for letting Science Press publish Springer books in
China at affordable prices, Springer gets to publish translations 
of Chinese journals in the West at unaffordable prices.

By the way - after checking out the bookstore, I went out to the 
street vendors and bought an excellent dinner of rice, sausage 
and vegetables for 35 yuan - about 40 cents US.  It was cooked 
by a husband and wife in a wok on a cart.  

Just after I bought it, someone yelled the Chinese equivalent 
of "cops!", and all the street vendors suddenly dashed away 
with their carts, leaving only the woman, who kindly handed 
me my dinner in a styrofoam pack before walking off.  They
clearly had this down to a fine art: it all happened faster 
than my brain could process.  I guess the cops don't allow
street vendors there.  

I only wish I'd noticed: did the street vendors turn off their
gas stoves before running, or run while still cooking?  

Anyway, on to some serious math and physics.

You've probably heard of some mysterious thing called "M-theory"
that lives in 11 dimensions.  Back in "<A HREF = "week158.html">week158</A>" and "<A HREF = "week159.html">week159</A>" 
I took a stab at understanding this.  Now I'll try again, with
a lot of help from Urs Schreiber:

3) Urs Schreiber, Castellani on free differential algebras
in supergravity: gauge 3-group of M-theory, 
<A HREF = "http://golem.ph.utexas.edu/string/archives/000840.html">http://golem.ph.utexas.edu/string/archives/000840.html</A>

Calling M-theory a "theory" is a bit misleading, because nobody 
knows what this theory is!  There's just a lot of clues pointing
to its existence.  It seems to be the quantum version of a 
well-defined classical field theory called "11-dimensional 
supergravity".   And, it seems to involve 2-branes and 5-branes:
2- and 5-dimensional membranes that trace out 3- and 6-dimensional 
surfaces in spacetime, just like strings trace out 2-dimensional 
surfaces.  

Back in "<A HREF = "week158.html">week158</A>" I wrote down a Lagrangian for 11d supergravity.
This is a truly monstrous thing involving three fields:

A) a frame field e - the "graviton",

B) a field \psi  taking values in the real spin-3/2 
representation of the 11d Lorentz group - the "gravitino",

C) a 3-form A. 

When it was discovered back in 1978, people were
interested in 11d supergravity mainly because it was the
highest-dimensional theory they could concoct that includes 
general relativity and supersymmetry - a symmetry that interchanges 
bosons and fermions, in this case gravitons and gravitinos - 
without including any particles of spin > 2.  So, the fact that 
it looked like a mess wasn't such a big deal.  But now that 
some people are taking it very seriously, it's worth trying 
to understand the math behind it more deeply, to see what makes 
it tick.

For example: what's so great about 11 dimensions?  And: 
what's the reason for that 3-form?

I'm not a huge fan of string theory, but I like puzzles of
this sort - finding patterns that make certain things work
only in certain dimensions, and stuff like that.  So, I got 
intrigued when I learned that super-Yang-Mills theory and
superstring theory are nice in dimension 10 because of special 
properties of the octonions - see "<A HREF = "week104.html">week104</A>".  Maybe a little
extra stretch could bring us to dimension 11?   

I got even more intrigued when I ran across two competing 
explanations for that 3-form in 11d supergravity.  One was 
that it's a connection on a twice categorified version of 
a U(1) bundle.  The other was that it's the Chern-Simons
form for an E_{8} gauge theory. 

Let me say a bit about what these means.  I talked about categorified
U(1) bundles in "<A HREF = "week210.html">week210</A>", so
I'll be sort of brief about those....

A connection on a U(1)-bundle looks locally like a 1-form, 
so we can integrate it along a path and compute how the 
phase of charged particle changes when we move it along that 
path:


$$

             f
  x------------>-----y    a path f from the point x to the point y:
                                 we write this as f: x \to  y

$$
    
Believe it or not, this is the basis of all modern ideas on 
electromagnetism!

If we categorify this whole idea once, we get a kind of
connection that looks locally like a 2-form.  Folks call this
a "connection on a U(1) gerbe", but don't let the use of French 
here intimidate you: they just do that so they can charge more 
for the wine.  It's just a gadget that you can integrate over 
a surface, to compute how the phase of a charged \emph{string} moves 
when we slide it along that surface:


$$

              f
    ----------->-----
   /        ||       \
  x         ||F       y   a path-of-paths F from the path f to the path g:
   \        \/       /            we write this as F: f => g
    ----------->-----
               g
$$
    
And, if we categorify once more, we get a "connection on a 
U(1) 2-gerbe".  This is something that looks locally like a 
3-form, which describes what happens when we move \emph{2-branes} 
around!

If you're wondering why I'm talking about "categorifying", 
it's because this:


$$

             f
  x------------>-----y    
$$
    
is also a picture of a morphism in a category, while this:


$$

              f
    ----------->-----
   /        ||       \
  x         ||F       y  
   \        \/       /  
    ----------->-----
               g
$$
    
is a picture of a 2-morphism in a 2-category and so on.  We're
talking about processes between processes between processes... 
so we're climbing up the ladder of n-categories.

Anyway: since 11d supergravity has a 3-form in it, and M-theory 
apparently has 2-branes in it, maybe we need to categorify 
the concept of a U(1) bundle twice to understand what's going
on here!

I came up with this crazy idea on my own back in "<A HREF =
"week158.html">week158</A>", but it's an obvious guess after you
learn that the 2-form field called B in 10d superstring theory really
\emph{is} a connection on a U(1) gerbe:

4) Alan L. Carey, Stuart Johnson and Michael K. Murray,
Holonomy on D-branes, available as <A HREF = "http://arxiv.org/abs/hep-th/0204199">arXiv:hep-th/0204199</A>.

Unfortunately there are some problems with naively pushing
this idea up a dimension.  For example, a crucial factor of 
1/6 in the Lagrangian for 11d supergravity is not explained 
by thinking of A this way.

Another possible explanation was that this 3-form is the 
Chern-Simons form of an E_{8} bundle over spacetime:

5) Emanuel Diaconescu, Gregory Moore and Edward Witten,
E_{8} gauge theory, and a derivation of K-theory from M-theory,
Adv. Theor. Math. Phys. 6 (2003) 1031-1134.  Also available as 
<A HREF = "http://arxiv.org/abs/hep-th/0005090">arXiv:hep-th/0005090</A>.

6) Emanuel Diaconescu, Daniel S. Freed and Gregory Moore,
The M-theory 3-form and E_{8} gauge theory, available as
<A HREF = "http://arxiv.org/abs/hep-th/0312069">arXiv:hep-th/0312069</A>.

This idea explains that factor of 1/6.  And, it might move towards an
explanation of how the octonions get into the act, because the group
E_{8} is deeply related to the octonions.  But as the authors
of the above paper say, "the E_{8} gauge field plays a
purely topological role and appears, in some sense, to be a
'fake'." In particular, you don't see any E_{8}
connection staring you in the face in the Lagrangian for 11d
supergravity that I wrote down in "<A HREF =
"week158.html">week158</A>".

Later, it started becoming clear that both ideas - the twice 
categorified U(1) connection and the E_{8} gauge theory - fit 
together in some way:

7) Paolo Aschieri and Branislav Jurco, Gerbes, M5-brane 
anomalies and E_{8} gauge theory, JHEP 0410 (2004), 068.
Also available as <A HREF = "http://arxiv.org/abs/hep-th/0409200">arXiv:hep-th/0409200</A>.

It all became a lot clearer to me when Urs Schreiber read 
these papers and translated them into a language I like:

8) Leonardo Castellani, Lie derivatives along antisymmetric 
tensors, and the M-theory superalgebra, available as <A HREF = "http://arxiv.org/abs/hep-th/0508213">arXiv:hep-th/0508213</A>.

9) Pietro Fr&eacute; and Pietro Antonio Grassi, Pure spinors, 
free differential algebras, and the supermembrane, available
as <A HREF = "http://arxiv.org/abs/hep-th/0606171">arXiv:hep-th/0606171</A>.

The idea is to think of 11d supergravity as a twice categorifed
gauge theory - not just the 3-form field in 11d supergravity,
but all the fields, in a unified way!  

For this, we need to do something much more clever than taking
11d spacetime and slapping a U(1) 2-gerbe on top of it.  We 
need to combine the graviton, the gravitino and the 2-form field 
into a connection on a \emph{nonabelian} 2-gerbe.  

Here things get a bit technical, but Urs has covered the technical
points quite nicely in his blog, so right now I'll just try to
give you some hand-wavy intuitions.

Very roughly speaking, an connection on a bundle takes any path 
in spacetime


$$

             f
  x------------>-----y    
$$
    
and gives you an element of some \emph{group}, which says how a particle
would transform if you moved it along this path.  This group could
be U(1) - the group of phases - or it could be something more fun,
like a \emph{nonabelian} group.  

If we categorify this concept, we get the concept of a connection
on a "2-bundle" (which is more or less the same as a gerbe).  Such
a connection takes any path and gives you an \emph{object} in some 
\emph{2-group}, 
but it also takes any surface like this:


$$

              f
    ----------->-----
   /        ||       \
  x         ||F       y  
   \        \/       /  
    ----------->-----
               g
$$
    
and gives you a \emph{morphism} in this 2-group.  You see, 2-group
is a kind of category that acts like a group, and a category 
has "objects" and "morphisms".  The morphisms go between objects.  
For more on 2-groups, try:

10) Higher-dimensional algebra V: 2-Groups, with Aaron D. Lauda, 
Theory and Applications of Categories 12 (2004), available at
<A HREF = "http://www.tac.mta.ca/tac/volumes/12/14/12-14abs.html">http://www.tac.mta.ca/tac/volumes/12/14/12-14abs.html</A> &nbsp;
Also available as <A HREF = "http://arxiv.org/abs/math.QA/0307200">math.QA/0307200</A>.

If we categorify once more, we get connections on a "3-bundle",
which is more or less the same thing as a "2-gerbe" - unfortunately
the numbering systems are off by one.   This gives us objects,
morphisms and 2-morphisms in a \emph{3-group}, which describe what 
happens when we move particles, strings and 2-branes.

And so on:


\begin{verbatim}

group         point particles
2-group       point particles and strings
3-group       point particles, strings and 2-branes
4-group       point particles, strings, 2-branes and 3-branes
\end{verbatim}
    
etc.  

So, if 11d supergravity is a twice categorified gauge theory,
we need to know its symmetry \emph{3-group}.  

But actually, since we're doing geometry, this 3-group should
be a "Lie 3-group".  In other words, very roughly speaking, 
a 3-group that has a \emph{manifold} of objects, a manifold of 
morphisms, and a manifold of 2-morphisms, where all the 
operations are smooth.

But actually, since we're doing supersymmetric geometry, we
need a "Lie 3-supergroup"!  In other words, very roughly
speaking, a 3-group that has a \emph{supermanifold} of objects, 
a supermanifold of morphisms, and a supermanifold of 2-morphisms, 
where all the operations are smooth.  (Maybe I should say 
"supersmooth", just to be consistent.)

If you don't know what a supermanifold is, now is probably 
not the time to learn.  I mean, not right this second.
The point is just this: supersymmetry 
infests everything once you let it in the door, just like 
n-categories, and just like manifolds - and now we're doing all three.   

In fact, nobody has even written down a rigorous definition
of a Lie 3-supergroup yet!  But, Lie algebras are in some ways 
simpler than Lie groups, and they're a good start, so we can
be glad that people \emph{do} know what a Lie 3-superalgebra is!

And Urs describes, in his blog, the relevant Lie 3-superalgebra
for 11d supergravity!

I would like to say more about this, but it's getting a bit 
tough trying to talk about this stuff in a fun, easily accessible 
style, and I have the feeling I'm no longer succeeding.  In fact,
I don't think I can give a "fun, easily accessible" description 
of this specific Lie 3-superalgebra - at least not yet.  So, 
now I'll completely give up trying to be comprehensible, and simply 
state some facts.

As shown here:

11) Higher-dimensional algebra VI: Lie 2-Algebras, with Alissa
Crans, Theory and Applications of Categories 12 (2004),
available at <A HREF = "http://www.tac.mta.ca/tac/volumes/12/15/12-15abs.html">http://www.tac.mta.ca/tac/volumes/12/15/12-15abs.html</A> &nbsp;
Also available as <A HREF = "http://arxiv.org/abs/math.QA/0307200">math.QA/0307200</A>.

the category of Lie n-algebras is equivalent to the category
of L_{\infty } algebras which as chain complexes have only n
nonvanishing terms, the 0th to the (n-1)st.  L_{\infty } algebras
are just algebras of Stasheff's L_{\infty } operad in the category
of chain complexes of vector spaces - see "<A HREF = "week191.html">week191</A>"
and especially these:

12) Martin Markl, Steve Schnider and Jim Stasheff, Operads in Algebra,
Topology and Physics, AMS, Providence, Rhode Island, 2002.

James Stasheff, Hartford/Luminy talks on operads, available at 
<A HREF = "http://www.math.unc.edu/Faculty/jds/operadchik.ps">http://www.math.unc.edu/Faculty/jds/operadchik.ps</A>

But, we can replace vector spaces by Z/2-graded vector spaces and
everything still works.  Physicists call Z/2-graded vector spaces
"super vector spaces".  So, a "Lie n-superalgebra"
is an algebra of the L_{\infty } operad in the category of
chain complexes of super vector spaces.

Given this, to specify a Lie 3-superalgebra we first need to
specify the 0-chains, then the 1-chains, then the 2-chains.  

For the particular one Urs mentions, we have


\begin{verbatim}

{0-chains} = 11d Poincar&eacute; Lie superalgebra
{1-chains} = {0}
{2-chains} = R
\end{verbatim}
    

Here R is the real numbers, and this 1-dimensional thing is 
what ultimately gives the 3-form field A in 11d supergravity.   
As a vector space, the 11d Poincar&eacute; Lie superalgebra is the 
direct sum of an even part, which is the usual Poincar&eacute; Lie 
algebra iso(11,1), and an odd part, which is the 32-dimensional 
real spinor rep of so(11,1).  These give the graviton (or
more precisely the Levi-Civita connection) and the gravitino
in 11d supergravity.  

Next we need to make this stuff into a chain complex.  That's
easy: the differential \emph{has} to be zero.

Next, we need to specify the L_{\infty } structure on this
chain complex.  First, we need a binary bracket operation, 
like in an ordinary Lie superalgebra.
The bracket of 0-chains is the usual bracket in the 11d Poincar&eacute; 
Lie superalgebra.  All the other binary brackets are zero.

Then, we need a ternary bracket operation, which expresses how the
Jacobi identity holds only up to chain homotopy.  This is zero.

Then, we need a quaternary bracket operation (since that
chain homotopy satisfies its own identity only up to chain
homotopy).  This is nonzero: when we take the quaternary
bracket of four 0-chains we get a 2-chain, and there's a
nontrivial way to define this!  This is the interesting bit,
since ultimately it relates the graviton/gravitino to the 
3-form field.

How do we get that quaternary bracket?  Well, here's where
things get funky: D'Auria and Fr&eacute; dreamt up a formula that 
gives a number from 2 spinors and 2 vectors: 

(\psi , \phi , v, w) |\to  \psi * \Gamma ^{ab} \phi  v_{a} w_{b}

where that "\psi *" should really be "\psi -bar".
And, magically, in 11 dimensions this gives a 4-cocycle on 
the Poincar&eacute; Lie superalgebra!  The proof of this uses some
Fierz identity in 11 dimensions:

12) R. D'Auria and Pietro Fr&eacute;, Geometric supergravity in D = 11
and its hidden supergroup, Nucl. Phys. B201 (1982), 101-140.
Also available at <A HREF = "http://www.math.uni-hamburg.de/home/schreiber/sdarticle.pdf">http://www.math.uni-hamburg.de/home/schreiber/sdarticle.pdf</A>

And, from HDA6 we know that the 4-cocycle condition is just 
what's needed to make the quaternary bracket satisfy the identity
we need for a Lie 3-superalgebra.  (Alissa and I just did the
calculation for Lie n-algebras, but the "super" stuff should work
too with a few signs thrown in.)

So, this is all very cool, but I need to understand Fierz identities
in different dimensions to see what if anything is special to
11d here - or, alternatively, work out the cohomology of Poincar&eacute;
Lie superalgebras, to see when they can be deformed to Lie 
n-superalgebras.  Sounds like a lot of work - maybe someone already
did it.  Actually D'Auria and Fre make it look like a matter of
understanding tensor products of irreps of so(n,1), which is not bad.
A worthwhile project in any event.

I also need to understand what all this has to do with E_{8}.
For that the paper by Diaconescu, Freed and Moore should help.

Well, this is just the beginning, but Urs explains the rest.

\par\noindent\rule{\textwidth}{0.4pt}
\textbf{Addenda}: I thank Noam Elkies for a correction.  Aaron
Bergman has this to say about E_{8} and M-theory:

\begin{quote}
John Baez wrote:


$$

> I also need to understand what all this has to do with E_{8}.
$$
    

E_{8} might not have much of anything to do with this. As mentioned in
Diaconescu, Freed and Moore, E_{8} appears to function solely as a
convenient stand-in for K(Z,3).

On the other hand, the split form of the E-series (up to
E_{11} if you're feeling particularly speculative) is known to
show up in describing the fields of 11D SUGRA, but the I don't think
anyone knows of a connection between the two E_{8}s. Just to
add to the fun, E_{8} gauge fields also show up on the fixed
points of M-theory on S^{1}/Z_{2} giving the
E_{8} \times  E_{8} heterotic string. 

Aaron
\end{quote}

Urs Schreiber replied:

\begin{quote}
 Aaron Bergman wrote:

$$

 > E_{8} gauge fields also show up on the fixed points of 
 > M-theory on S^{1}/Z_{2} giving the E_{8} \times  E_{8} heterotic string.
$$
    

 The topological part of the membrane action involves the integral
 of the sugra 3-form over the worldvolume.  By DFW, part of that
 3-form can be thought of as an E_{8} CS 3-form.

 So part of the membrane action looks similar to an E_{8} CS-theory
 over the worldvolume.

 Now let the membrane have a boundary. A bulk E_{8} CS-theory is
 well known to induce an E_{8} WZW theory on the boundary.

 Could this be the connection between the DWF E_{8} and the
 Horava-Witten E_{8}?

 I have asked this question before: <A HREF =
 "http://golem.ph.utexas.edu/string/archives/000791.html">http://golem.ph.utexas.edu/string/archives/000791.html</A>. Jarah
 then agreed that this must be about right. But it is not completely
 clear to me yet.

 One problem is that in CS-theory we vary the connection, while in
 the topological membrane the E_{8} connection on the background is fixed
 and we vary the embedding by which we pull it back to the worldvolume.
 Under suitable assumptions that might be equivalent to varying an E_{8}
 connection on the worldvolume itself? 
\end{quote}

Aaron Bergman replied:

\begin{quote} 
 Urs Schreiber wrote:

$$

 > Now let the membrane have a boundary. A bulk E_{8} CS-theory is
 > well known to induce an E_{8} WZW theory on the boundary.
 >
 > Could this be the connection between the DWF E_{8} and the
 > Horava-Witten E_{8}?
$$
    
 A similar observation was made by Horava way back in <A HREF = "http://arxiv.org/abs/hep-th/9712130">arXiv:hep-th/9712130</A> (in
 the final section).
 Aaron 
\end{quote}


\par\noindent\rule{\textwidth}{0.4pt}
% </A>
% </A>
% </A>
