
% </A>
% </A>
% </A>
\week{December 17, 2000 }

Since the winter solstice is coming soon, I'll start with some gift 
suggestions... for the physicist who has everything.

1) The Universe Map, National Geographic Society, 2000, NSG #602011.

I've only seen a picture of this 20 x 31 inch map, but I know I want
one!  In a series of different 3d views, it shows the solar system, 
nearby stars, the Milky Way, the Local Group and the observable universe
as a whole.  I'll put it outside my office so my students can figure
out just where they stand in the grand scheme of things.

2) Wil Tirion and Roger W. Sinnot, Sky Atlas 2000.0, 2nd edition,
Cambridge U. Press, 1999.

This is a favorite sky atlas among amateur astronomers.  It comes in
lots of versions, but Kevin Kelly of Whole Earth says that the most
useful is the "deluxe version, spiralbound".  

3) Lee Smolin, Three Roads to Quantum Gravity, Weidenfeld and Nicholson,
2000.

This is a nontechnical guide to quantum gravity and the different
approaches people have taken to this problem: string theory, loop
quantum gravity, and the more radical lines of thought pursued by 
people whom Smolin calls "the true heroes of quantum gravity", like 
Alain Connes, David Finkelstein, Chris Isham, Roger Penrose and 
Raphael Sorkin.  I haven't gotten ahold of this book, so I can't 
describe it in detail yet, but it should be lots of fun.

That's enough gift suggestions.  Now I want to talk about Jordan
algebras and how they show up in projective geometry, quantum logic,
special relativity and so on.  I'll start by reminding you of some stuff
from "<A HREF = "week106.html">week106</A>"  and "<A HREF = "week145.html">week145</A>".   Then I'll charge ahead and show you
how a Jordan algebra built from the octonions is related to 10-dimensional 
Minkowski spacetime....

Projective geometry is a venerable subject that has its origins in the
study of perspective by Renaissance painters.  As seen by the eye, any
pair of parallel lines - e.g., train tracks - appear to meet at a "point
at infinity".  Furthermore, when you change your viewpoint, distances
and angles appear to change, but points remain points and lines remain
lines.  This suggests a modification of Euclidean plane geometry based
on a set of points, a set of lines, and relation whereby a point "lies
on" a line, satisfying the following axioms:

A) For any two distinct points, there is a unique line on which they
both lie.

B) For any two distinct lines, there is a unique point which lies on
both of them.

C) There exist four points, no three of which lie on the same line.

D) There exist four lines, no three of which have the same point lying
on them.  

Any structure satisfying these axioms is called a "projective plane".
But projective geometry is also interesting in higher dimensions.  One
can define a "projective space" by the following axioms:

A) For any two distinct points p and q, there is a unique line
pq on which they both lie.

B) For any line, there are at least three points lying on this line.

C) If a,b,c,d are distinct points and there is a point lying
on both ab and cd, then there is a point lying on both ac and bd.


Given a projective space and a set S of points in this space, we define
the "span" of S to be the set of all points lying on lines ab
where a,b are distinct points in S.  The "dimension" of a
projective space is defined to be one less than the smallest number of
points that span the whole space.  As you would hope, a 2-dimensional
projective space is the same thing as a projective plane!  It's a fun
exercise to show this straight from the above axioms.  If you give up,
read this book:

4) Lynn E. Garner, An Outline of Projective Geometry, North Holland, New
York, 1981.

How can we get our hands on some projective spaces?  Well, if K is any
field, there is an n-dimensional projective space called
KP^{n} where the points are lines through the origin in
K^{n+1}, the lines are planes through the origin in K^{n+1},
and the relation of "lying on" is inclusion.  The example
relevant to perspective is the real projective plane, RP^{2}.
But it's good to follow Polya's advice:

"Be wise - generalize!" 

and study KP^{n} for any field and any n.  In fact, we can
define KP^{n} even when K is a mere "skew field": a
ring such that every nonzero element has a left and right
multiplicative inverse.  We just need to be a bit careful about
defining lines and planes through the origin in K^{n+1}.  To
do this, we just take a line through the origin to be any set

L = {ax: a in K}

where x is nonzero element of K^{n+1}, and take a plane through the
origin to be any set

P = {ax + by: a,b in K}

where x,y are elements of K^{n+1} such that ax + by = 0
implies a and b are zero.


 Around now, you might be wondering whether \emph{every}
projective n-space is of the form KP^{n} for some skew field
K.  If so, you must have forgotten "<A HREF =
"week145.html">week145</A>", where I gave the answer: yes, but
only if n > 2.  Projective planes are more subtle!  A projective
plane comes from a skew field if and only if it satisfies an extra
axiom, the "axiom of Desargues".  I described this axiom in
"<A HREF = "week145.html">week145</A>" so I won't do it
again here.  The main point is that a projective plane coming from a
skew field has some extra geometrical properties that a
"non-Desarguesian" projective plane will not.

Projective geometry was very fashionable in the 1800s, with such worthies 
as Poncelet, Brianchon, Steiner and von Staudt making important
contributions.  Later it was overshadowed by other forms of geometry.  
However, work on the subject continued, and in 1933 Ruth Moufang
constructed a remarkable example of a non-Desarguesian projective plane
using the octonions:

5) Ruth Moufang, Alternativkoerper und der Satz vom vollstaendigen
Vierseit, Abhandlungen Math. Sem. Hamburg 9, (1933), 207-222.

It turns out that this projective plane deserves the name OP^{2}, 
where O stands for the octonions.  

The 1930s also saw the rise of another reason for interest in projective
geometry: quantum mechanics!  Quantum theory is distressingly different
from the classical Newtonian physics we have learnt to love.  In
classical mechanics, observables are described by real-valued functions.
In quantum mechanics, they are often described by hermitian n x n
complex matrices.   In both cases, observables are closed under addition
and multiplication by real scalars.  However, in quantum mechanics,
observables do not form an associative algebra.  Still, one can raise an
observable to any power, and from squaring one can define a commutative 
product:

x o y = [(x+y)^{2} - x^{2} - y^{2}]/2 = (xy + yx)/2

This product is not associative, but it satisfies the weaker identity

x o (y o x^{2}) = (x o y) o x^{2}

In 1932, Pascual Jordan attempted to understand this situation better by
isolating the bare minimum axioms that an "algebra of observables"
should satisfy:

6) Pascual Jordan, Ueber eine Klasse nichtassociativer hyperkomplexer
Algebren, Nachr. Ges. Wiss. Goettingen (1932), 569-575.  

He invented the definition of what is now called a "formally real Jordan
algebra": a commutative (but not necessarily associative) unital algebra
over the real numbers such that:

x o (y o x^{2}) = (x o y) o x^{2}

and also:

a^{2} + b^{2} + c^{2} + ... = 0   implies   a = b = c = ... = 0.

The last condition gives our algebra a partial ordering: if we say that
x is "less than or equal to" y when the element y - x is a sum of
squares, this condition says that if x is less than or equal to y and 
y is less than or equal to x, then x = y.   If we drop this last
condition, we get the definition of what is now called a "Jordan
algebra".  

In 1934, one year after Moufang published her paper on OP^{2}, Jordan
published a paper with von Neumann and Wigner classifying all formally
real Jordan algebras:

7) Pascual Jordan, John von Neumann, Eugene Wigner, On an algebraic
generalization of the quantum mechanical formalism, Ann. Math. 35
(1934), 29-64.  

Their classification is nice and succinct.  An "ideal" in the
Jordan algebra A is a subspace B such that if b is in B, a o b lies in B
for all a in A.  A Jordan algebra A is "simple" if its only
ideals are {0} and A itself.  Every formally real Jordan algebra is a
direct sum of simple ones.  The simple formally real Jordan algebras
consist of 4 infinite families and one exception:

<UL>
<LI> The algebra of n x n self-adjoint real matrices with the product 

x o y = (xy + yx)/2.

<LI> The algebra of n x n self-adjoint complex matrices with the product 

x o y = (xy + yx)/2.

<LI> The algebra of n x n self-adjoint quaternionic matrices with the product 

x o y = (xy + yx)/2.

<LI> The algebra R^{n} + R with the product 

(v,a) o (w,b) = (aw + bv, <v,w> + ab)

where <v,w> is the usual inner product of vectors in R^{n}.  This sort of
Jordan algebra is called a "spin factor".

<LI>
The algebra of 3 x 3 self-adjoint octonionic matrices with the product 

x o y = (xy + yx)/2.  

This is called the "exceptional Jordan algebra".
</UL>

This classification raises some obvious questions.  Why does nature
prefer the Jordan algebras h_{n}(C) over all the rest?  Or does
it?  Could the other Jordan algebras - even the exceptional one - have
some role to play in quantum physics?  Despite much research, these
questions remain unanswered to this day.

The paper by Jordan, von Neumann and Wigner appears to have been
uninfluenced by Moufang's discovery of OP^{2}, but in fact the two 
are related!  A "projection" in a formally real Jordan algebra is
defined to be an element p with p^{2} = p.  In the usual case of
h_{n}(C), these correspond to hermitian matrices with eigenvalues 
0 and 1, so they are used to describe observables that assume only 
two values - e.g., "true" and "false".  

This suggests treating projections in a formally real Jordan algebra as
propositions in a kind of "quantum logic".  The partial order helps us
do this: given projections p and q, we say that p "implies" q if p is
less than or equal to q.  We can then go ahead and define "and", "or"
and "not" in this context, and most of the familiar rules of Boolean logic 
continue to hold.  However, we no longer have the distributive laws:

p and (q or r) = (p and q) or (p and r)

p or (q and r) = (p or r) and (q or r)

The failure of these distributive laws is the hallmark of quantum logic.

Now, the relation between Jordan algebras and quantum logic is already
interesting in itself:

8) G. Emch, Algebraic Methods in Statistical Mechanics and Quantum Field
Theory, Wiley-Interscience, New York, 1972.

... but the real fun starts when we note that projections in the Jordan
algebra of n x n self-adjoint complex matrices correspond to subspaces of
C^{n}.  This sets up a relationship to projective geometry, since the
projections onto 1-dimensional subspaces correspond to points in CP^{n},
while the projections onto 2-dimensional subspaces correspond to lines.
Even better, we can work out the dimension of a subspace V from the
corresponding projection p: C^{n} \to  V using only the partial order on
projections: V has dimension d iff the longest chain of distinct
projections 

   p_{0} < p_{1} < ... < p_{i} = p  

has length i = d.  In fact, we can use this to define the "dimension"
of any projection in \emph{any} formally real Jordan algebra.  We can then try
to construct a projective space whose points are the 1-dimensional
projections and whose lines are the 2-dimensional projections, with the
relation of "lying on" given by the partial order in our Jordan algebra.

If we try this starting with the Jordan algebra of n x n self-adjoint
matrices with real, complex or quaternionic entries, we succeed when n
is 2 or more - and we obtain the projective spaces RP^{n},
CP^{n} and HP^{n}, respectively.  If we try this
starting with the spin factor R^{n} + R we succeed when n is 2
or more - and we obtain a series of 1-dimensional projective spaces
related to Minkowskian geometry, which I'll talk about in a minute.
Finally, in 1949 Jordan discovered that if we try this construction
starting with the exceptional Jordan algebra, we get the projective
plane discovered by Ruth Moufang - OP^{2}!

9) Pascual Jordan, Ueber eine nicht-desarguessche ebene projektive
Geometrie, Abhandlungen Math. Sem. Hamburg 16 (1949), 74-76.

Physicists have tried for a long time to find some use for the quantum
logic corresponding to the exceptional Jordan algebra.  So far they have
not succeeded.  Jordan hoped this stuff would be related to nuclear
physics.  Feza Gursey and Murat Gunaydin hoped it was related to
quarks, since 3 x 3 hermitian octonionic matrices should describe 
observables in some 3-state quantum system:

10) Murat Gunaydin and Feza Gursey, An octonionic representation of
the Poincare group, Lett. Nuovo Cim. 6 (1973), 401-406.

11) Murat Gunaydin and Feza Gursey, Quark structure and octonions,
Jour. Math. Phys. 14 (1973), 1615-1667.

12) Murat Gunaydin and Feza Gursey, Quark statistics and octonions,
Phys. Rev. D9 (1974), 3387-3391.

13) Murat Gunaydin, Octonionic Hilbert spaces, the Poincare group and
SU(3), Jour. Math. Phys. 17 (1976), 1875-1883.

14) M. Gunaydin, C. Piron and H. Ruegg, Moufang plane and octonionic 
quantum mechanics, Comm. Math. Phys. 61 (1978), 69-85.

Alas, these ideas never quite worked out, so most physicists discarded
the exceptional Jordan algebra as a lost cause.

However, the exceptional Jordan algebra is secretly related to string
theory, so there's a sense in which it's still lurking in the collective
subconscious.   Now, you probably want me to explain this, but I'm not
ready to.  So I won't say what 3 x 3 hermitian octonionic matrices have 
to do with string theory.  If you want to know that, read these:

15) E. Corrigan and T. J. Hollowood, The exceptional Jordan algebra and the 
superstring, Commun. Math. Phys., 122 (1989), 393.   Also available 
at <a href = "http://projecteuclid.org/DPubS?service=UI&version=1.0&verb=Display&handle=euclid.cmp/1104178468">http://projecteuclid.org/</a>

16) E. Corrigan and T. J. Hollowood, A string construction of a commutative 
nonassociative algebra related to the exceptional Jordan algebra, Phys. 
Lett. B203 (1988), 47. 

17) G. Sierra, An application of the theories of Jordan algebras
and Freudenthal triple systems to particles and strings, Class. 
Quant. Grav. 4 (1987), 227.

Instead, I'll just say what 2 x 2 hermitian octonionic matrices have to 
do with 10-dimensional Minkowski spacetime.  Since superstrings live in 
10 dimensions, that's at least a start.  

First, we need to think about spin factors.

In case you forgot, spin factors were the fourth infinite family of
simple formally real Jordan algebras on my list up there.  I gave a
lowbrow definition of these guys, but now let's try a highbrow one. 
Given an n-dimensional real inner product space V, the "spin factor"
J(V) is the Jordan algebra generated by V with the relations 

v o w = <v,w> 

This should remind you of the definition of a Clifford algebra, and
indeed, they're related - they have the same representations!   This
sets up a connection to spinors, which is why these Jordan algebras 
are called "spin factors".  

But anyway: if you think about it a while, you'll see that J(V) is
isomorphic to the direct sum V + R equipped with the product 

(v,a) o (w,b) = (aw + bv, <v,w> + ab)

which is basically the lowbrow definition of a spin factor.

Though Jordan algebras were invented to study quantum mechanics, the
spin factors are also deeply related to special relativity: we can think
of J(V) = V + R as "Minkowski spacetime", with V as space and
R as time.  The reason is that J(V) is naturally equipped with a dot
product:

(v,a) . (w,b) = <v,w> - ab

which is just the usual Minkowski metric in slight disguise.  This
makes it tempting to borrow an idea from special relativity and
define the "lightcone" to consist of all nonzero x in J(V) with

x . x = 0

A 1-dimensional subspace of J(V) spanned by an element of the lightcone
is called a "light ray", and the space of all light rays is
called the "heavenly sphere" S(V).  We can identify the
heavenly sphere with the sphere of unit vectors in V, since every light
ray is spanned by an element of J(V) of the form (v,1) where v is a unit
vector in V.

What's the physical meaning of the heavenly sphere?  Well, if you were
a resident of the spacetime J(V) and gazed up at the sky at night, the
stars would seem to lie on this sphere.  If you took off in a spaceship
and whizzed along at close to the speed of light, all the constellations
would look distorted, but all \emph{angles} would be preserved, since the
Lorentz group acts as conformal transformations of the heavenly sphere. 

Now, when V is at least 2-dimensional, we can build a projective space
from J(V) using the construction I described for any simple formally 
real Jordan algebra.  If we do this, what do we get?  

Well, you can easily check that aside from the elements 0 and 1, all
projections in J(V) are of the form p = (1/2)(v,1) where v is a unit
vector in V.  These projections will be the points of our projective
space, but as we've seen, they also correspond to points of the heavenly
sphere.   So our projective space is really just the heavenly sphere!
This is cool, because it means points on the heavenly sphere can also
be thought of as \emph{propositions} in a certain sort of quantum logic.

Now, what does this have to do with the exceptional Jordan algebra? 
Well, we have to sneak up carefully on this wild beast, so first let's 
think about a smaller Jordan algebra: the 2 x 2 hermitian octonionic 
matrices.  In fact, we can kill four birds with one stone, and think 
about 2 x 2 hermitian matrices with entries in any n-dimensional 
normed division algebra, say K.  There are not that many normed 
division algebras, so I really just mean:

the real numbers,    R, if n = 1, 
the complex numbers, C, if n = 2,
the quaternions,     H, if n = 4, 
the octonions,       O, if n = 8.  

The space h_{2}(K) of hermitian 2 x 2 matrices with entries in K is a
Jordan algebra with the product 

x o y = (xy + yx)/2

Moreover, this Jordan algebra is secretly a spin factor!  There is an 
isomorphism 

f: h_{2}(K) \to  J(K + R) = K + R + R

which sends the hermitian matrix


\begin{verbatim}

( a+b   k  )
(  k*  a-b )
\end{verbatim}
    

to the element (k,b,a) in K + R + R.

Furthermore, the determinant of matrices in h_{2}(K) is just the
Minkowski metric in disguise, since the determinant of


\begin{verbatim}

( a+b   k  ) 
(  k*  a-b )
\end{verbatim}
    
is


$$

a^{2} - b^{2} - <k,k>

$$
    
These facts have a number of nice consequences.  First of all, since the
Jordan algebras J(K + R) and h_{2}(K) are isomorphic, so are
their associated projective spaces.  We have seen that the former space
is the heavenly sphere S(K + R); unsurprisingly, the latter is the
projective line KP^{1}.  It follows that these are the same!
This shows that:

h_{2}(R) is 3d Minkowski spacetime, and RP^{1} is
the heavenly sphere S^{1};

h_{2}(C) is 4d Minkowski spacetime, and CP^{1} is the heavenly
sphere S^{2}; 

h_{2}(H) is 6d Minkowski spacetime, and HP^{1} is the
heavenly sphere S^{4};

h_{2}(O) is 10d Minkowski spacetime, and OP^{1} is the
heavenly sphere S^{8}.



Secondly, it follows that the determinant-preserving linear
transformations of h_{2}(K) form a group isomorphic to O(n+1,1).
How can we find some transformations of this sort?  For K = R, it's
easy: when g lies in SL(2,R) and x is in h_{2}(R), we have gxg*
in h_{2}(R) again, and

det(gxg*) = det(x)

This gives a homomorphism from SL(2,R) to O(2,1).  It's easy to see that
this makes SL(2,R) into a double cover of the Lorentz group
SO_{0}(2,1). The exact same construction works for K = C, so
SL(2,C) is a double cover of the Lorentz group SO_{0}(3,1) -
which you probably knew already, if you made it this far!

For the other two normed division algebras the above calculation
involving determinants breaks down, and it even becomes tricky to define
the group SL(2,K), so we'll start by working at the Lie algebra level.
We say a 2 x 2 matrix with entries in the normed division algebra K is
"traceless" if the sum of its diagonal entries is zero.  Any
such traceless matrix acts as a real-linear operator on K^{2}.
When K is commutative and associative, the space of operators coming
from 2 x 2 traceless matrices with entries in K is closed under
commutators, but otherwise it is not, so we'll define sl(2,K) to be the
Lie algebra of operators on K^{2} \emph{generated} by
operators of this form.  This Lie algebra in turn generates a Lie group
of real-linear operators on K^{2}, which we call SL(2,K).

Now, sl(2,K) has an obvious representation on K^{2}, called the
"fundamental representation".  If we tensor this
representation with its dual we get a representation of sl(2,K) on the
space of 2 x 2 matrices with entries in K, which is given by

a: x |-> ax + xa*         

whenever a is actually a 2 x 2 traceless matrix with entries in K.
Since ax + xa* is hermitian whenever x is, this representation restricts
to a representation of sl(2,K) on h_{2}(K).  This in turn gives
a rep of the group SL(2,K).  A little calculation at the Lie algebra
level shows that this action of SL(2,K) on h_{2}(K) preserves
the determinant, so we have a homomorphism

SL(2,K) \to  SO_{0}(n+1,1)  

This is two-to-one and onto, so it follows pretty easily that:

SL(2,R) is the double cover of the Lorentz group SO_{0}(2,1);   
SL(2,C) is the double cover of the Lorentz group SO_{0}(3,1);  
SL(2,H) is the double cover of the Lorentz group SO_{0}(5,1); 
SL(2,O) is the double cover of the Lorentz group SO_{0}(9,1).

and thus:

SL(2,R) acts as conformal transformations of the sphere S^{1} = RP^{1};

SL(2,C) acts as conformal transformations of the sphere S^{2} = CP^{1};

SL(2,H) acts as conformal transformations of the sphere S^{4} = HP^{1};

SL(2,O) acts as conformal transformations of the sphere S^{8} = OP^{1}.

In the complex case, these conformal transformations are often
called "Moebius transformations".   For more on the octonionic
case, try this:

15) Corinne A. Manogue and Tevian Dray, Octonionic Moebius transformations,
Mod. Phys. Lett. A14 (1999) 1243-1256, available as <A HREF = "http://xxx.lanl.gov/abs/math-ph/9905024">math-ph/9905024</A>.  

To round off the story, it helps to bring in spinors:

16) Anthony Sudbery, Division algebras, (pseudo)orthogonal groups and
spinors, Jour. Phys. A17 (1984), 939-955.  

The fundamental rep of SL(2,K) on K^{2} is secretly one of the
spinor reps of the double cover of the Lorentz group
SO_{0}(n+1,1).  Moreover, we can get points on the heavenly
sphere from these spinors!  This has been nicely explained by Penrose in
the complex case, but it works the same way for the other normed
division algebras.  It goes like this:

Suppose 


$$

|\psi > = (x,y) 

$$
    
is a unit spinor, i.e. an element of K^{2} with norm one.  Then


$$

|\psi > <\psi | =    ( xx*   xy* )
                 ( yx*   yy* )
$$
    

is a projection in h_{2}(K) which is not 0 or 1 - or in other
words, a point on the heavenly sphere.  If we identify the heavenly
sphere with KP^{1}, this point corresponds to the line through
the origin in K^{2} containing the spinor |\psi >.

To go further, I would want to say more about why this connection
between quantum logic, Lorentzian geometry, and spinors is interesting,
and what you can do with it.  And then I would want to take everything
we've seen about OP^{1} and h_{2}(O) and see how it fits
inside the bigger, more interesting story of OP^{2} and
h_{3}(O).  But alas, I'm running out of steam here, so I'll just
give you a little reading list about the octonionic projective plane and
the exceptional Jordan algebra:

20) Hans Freudenthal, Zur ebenen Oktavengeometrie, Indag. Math. 15
(1953), 195-200.  
  
Hans Freudenthal, Beziehungen der e_{7} und e_{8} 
zur Oktavenebene:
I, II, Indag. Math. 16 (1954), 218-230, 363-368.  
III, IV, Indag. Math. 17 (1955), 151-157, 277-285.  
V - IX, Indag. Math. 21 (1959), 165-201, 447-474.  
X, XI, Indag. Math. 25 (1963) 453-471, 472-487.  
  
Hans Freudenthal, Lie groups in the foundations of geometry, Adv.
Math. 1 (1964), 145-190.  
  
Hans Freudenthal, Oktaven, Ausnahmegruppen und Oktavengeometrie,
Geom. Dedicata 19 (1985), 7-63.   

21) Jacques Tits, Le plan projectif des octaves et les groupes de Lie
exceptionnels, Bull. Acad. Roy. Belg. Sci. 39 (1953), 309-329.

Jacques Tits, Le plan projectif des octaves et les groupes exceptionnels
E_{6} et E_{7}, Bull. Acad. Roy. Belg. Sci. 40 (1954), 29-40.

22) Tonny A. Springer, The projective octave plane, I-II, Proc. Koninkl.
Akad. Wetenschap. A63 (1960), 74-101.
 
Tonny A. Springer, On the geometric algebra of the octave planes, I-III,
Proc. Koninkl. Akad. Wetenschap. A65 (1962), 413-451.

23) J. R. Faulkner and J. C. Ferrar, Exceptional Lie algebras and related
algebraic and geometric structures, Bull. London Math. Soc. 9 (1977),
1-35.  

Finally, for a really good overview of Jordan algebras and related
things like "Jordan pairs" and "Jordan triple
systems", try this:

24) Kevin McCrimmon, Jordan algebras and their applications, AMS Bulletin
84 (1978), 612-627.










 \par\noindent\rule{\textwidth}{0.4pt}

% </A>
% </A>
% </A>
