
% </A>
% </A>
% </A>
\week{February 23, 2003 }



This is my last week in Sydney.  The year-long drought in Australia 
has finally been broken by a series of rainstorms, but the sky was 
clear as I walked to my office tonight, and I saw the Milky Way really 
well!  It's so much more prominent in the southern sky, since you can 
see the center of the Galaxy better.

Some issues of This Week's Finds are mainly for explaining things
to other people, while others are mainly for myself.  I'm afraid this 
Week is one of the latter.  But I'll try to start by explaining what 
I'm up to.

Conversations with Tony Smith and Thomas Larsson have been making me
think more about the biggest exceptional Lie group, that magnificent
248-dimensional monstrosity called E_{8}.  This plays a significant role in
string theory and some other attempts to wrap everything we know about
physics into a big, glorious Theory of Everything.  None of these
attempts have succeeded in predicting anything new that's actually
been observed (ahem), but I still think it's worth pondering the group E_{8}.

Why?  First of all, it's a beautiful thing in itself.  Second, it has
strong ties to many "exotic" things in mathematics, including:

<UL>
<LI>
 the dodecahedron (see "<A HREF = "week20.html">week20</A>" and "<A HREF = "week65.html">week65</A>")

<LI>
 the octonions (see "<A HREF = "week141.html">week141</A>" and "<A HREF = "week168.html">week168</A>")

<LI>
 the Poincare homology 3-sphere (see "<A HREF = "week163.html">week163</A>" and "<A HREF = "week164.html">week164</A>")

<LI>
 the 4-manifold called K3 and exotic smooth structures on R^{4} 
(see "<A HREF = "week67.html">week67</A>")

<LI>
 exotic spheres in 7 and 11 dimensions (see "<A HREF = "week164.html">week164</A>")

</UL>
... in short, a whole zoo of strange creatures!  Third, if the laws of 
physics are indeed structures of "exceptional beauty" rather than 
"classical beauty" - see "<A HREF = "week106.html">week106</A>" for an explanation of what I mean 
by that - then it's natural to hope that E_{8} plays an important role.

How do we get our hands on E_{8}?  It's a bit tricky.  To understand a
group, it's always best to see it as the \emph{symmetries of something}.
Often we try to see it as the symmetries of some vector space equipped
with extra structure.  But for E_{8}, the smallest vector space that will
do the job is 248-dimensional - it's the Lie algebra of E_{8} itself!  In
mathspeak, the smallest nontrivial irrep of E_{8} is the adjoint rep.  

But in normal Engish, the problem is this: it's hard to construct E_{8} as
the symmetries of anything simpler than \emph{itself}.  It reminds me of
Baron von Munchausen pulling himself out of the swamp by his own
bootstraps.  

One possible way around this is to construct E_{8} as the symmetries of
something other than a vector space - for example, some \emph{manifold}
equipped with extra structure.  Here there is some hope: the compact
real form of E_{8} is the isometry group of a 128-dimensional Riemannian
manifold called the "octooctonionic projective plane".  The reason for
this name is that around 1956, Boris Rosenfeld claimed that you can
construct this manifold as a projective plane over the "octooctonions":
the octonions tensored with themselves.  Unfortunately, while there's
definitely something to this idea, I don't think anyone knows how to
make it precise without first constructing E_{8}.  Maybe someday....

Recently, some mathematical physicists have been studying a construction
of E_{8} as the symmetries of a 57-dimensional manifold equipped with extra
structure:

1) Murat Gunaydin, Koepsell and Hermann Nicolai, Conformal and
quasiconformal realizations of exceptional Lie groups,
Commun. Math. Phys. 221 (2001), 57-76, also available as <A HREF = "http://xxx.lanl.gov/abs/hep-th/0008063">hep-th/0008063</A>

2) Thomas A. Larsson, Structures preserved by exceptional Lie algebras,
available as math-ph/0301006.

When I heard this, the number 57 instantly intrigued me - and not just
because Heinz advertises "57 varieties", either!  No, the reason is that
the smallest nontrivial of irrep of E_{8}'s little brother E_{7} is
56-dimensional: it's a vector space equipped with extra structure making
it into the so-called "Freudenthal algebra".  When you study this
subject long enough, you realize that strange numbers can serve as clues
to hidden relationships... and guess what: there's one here!  I'll
say a bit more about it later.

(By the way, the story behind Heinz's "57 varieties" is that Henry John
Heinz saw an ad for 21 styles of shoe, and liked the gimmick - but the
numbers 5 and 7 held a special significance for him and his wife.  If
you don't believe me, send a letter to Heinz Consumer Affairs, P.O. Box 57, 
Pittsburgh, PA 15230 and ask them!)

Another way to get ahold of the group E_{8} is starting with its "root
lattice", the so-called E_{8} lattice.  There are different ways to
describe this.  Perhaps the most efficient is to say that it's the
densest lattice packing of spheres in 8 dimensions!  If I were about to
drown and needed to define the E_{8} lattice before I went under, this is
how I'd do it.  Unfortunately this leaves the recipient of the message
with a lot of work: they have to \emph{find} the lattice meeting this
description.


A more user-friendly description is this.  In any dimension we can
make a "checkerboard" with alternating red and black
hypercubes, and we get a lattice by taking the centers of all the red
ones.  In n dimensions this is called the D_{n} lattice.  We
can pack spheres by centering one at each point of this lattice and
making them just big enough so they touch.  There will of course be
some space left over.  But when we get up to dimension 8, there's
enough room left over so we can slip another identical array of
spheres in the gaps between the ones we've got!  This gives the
E_{8} lattice.

We can translate this into formulas without too much work.  The
D_{n} lattice consists of all n-tuples of integers that sum to
an even integer: requiring that they sum to an even integer picks out
the center of every other hypercube in our checkerboard.  Then, to get
E_{8}, we take the union of two copies of the D_{8}
lattice: the original one and another one shifted by (1/2, ..., 1/2).

(Actually this "doubled D_{n}" is interesting in any
dimension, and it's called D_{n}^{+}.  In 3 dimensions
this is how carbon atoms are arranged in a diamond!  In any dimension,
the volume of the unit cell of D_{n}^{+} is 1, so we
can say it's "unimodular".  But D_{n}^{+} is
only a lattice in even dimensions.  In dimensions that are multiples
of 4, it's an "integral" lattice, meaning that the dot
product of any two vectors in the lattice is an integer.  And in
dimensions that are multiples of 8, it's also "even",
meaning that the dot product of any vector with itself is even.  In
fact, even unimodular lattices are only possible in Euclidean space
when the dimension is a multiple of 8.  D_{8}^{+} =
E_{8} is the only even unimodular lattice in 8 dimensions; in
16 dimensions there are just two: E_{8} \times  E_{8}
and D_{16}^{+}.  As explained in "<A HREF =
"week95.html">week95</A>", these give two versions of heterotic
string theory.)
 
Summarizing, we can say E_{8} consists of all 8-tuples of real
numbers (x_{1}, ..., x_{8}) that sum to an even
integer and that are either \emph{all} integers or \emph{all}
integers plus 1/2.

Using this description it's easy to see that when you pack spheres in 
an E_{8} lattice, each sphere touches 240 others.  The reason is that the
shortest nonzero vectors in this lattice, the so-called "roots", have
length-squared equal to 2, and there are 240 of them:


\begin{verbatim}

(1,1,0,0,0,0,0,0) and all permutations thereof:
there are 8 choose 2 = 28 of these

(-1,-1,0,0,0,0,0,0) and all permutations thereof:
there are 8 choose 2 = 28 of these

(1,-1,0,0,0,0,0,0) and all permutations thereof:
there are twice 8 choose 2 = 56 of these

(1/2,1/2,1/2,1/2,1/2,1/2,1/2,1/2):
there is 1 of these 

(-1/2,-1/2, 1/2,1/2,1/2,1/2,1/2,1/2):
there are 8 choose 2 = 28 of these

(-1/2,-1/2,-1/2,-1/2, 1/2,1/2,1/2,1/2):
there are 8 choose 4 = 70 of these

(-1/2,-1/2,-1/2,-1/2,-1/2,-1/2, 1/2,1/2): 
there are 8 choose 2 = 28 of these

(-1/2,-1/2,-1/2,-1/2,-1/2,-1/2,-1/2,-1/2): 
there is 1 of these
\end{verbatim}
    
for a total of 


\begin{verbatim}


28 x 6 + 70 + 2 = 168 + 72 = 240
\end{verbatim}
    
roots.  

There's also another description of the E_{8} lattice, which I've been
meaning to understand for \emph{ages}, but which always scared me.  You can
think of 8-dimensional space as the octonions.  The unit octonions are
closed under multiplication and taking inverses.  If you take the E_{8}
lattice, rescale it so the roots have length one, and rotate it
correctly, you get a collection of 240 unit octonions that are closed
under multiplication!  It then follows that the octonions in the E_{8}
lattice are closed under addition and multiplication; these are called
the "Cayley integral octonions".


This sounds like just the sort of thing I'd like; the problem is the
phrase "rotate it correctly".  First, you have to rotate the
rescaled E_{8} lattice so that it contains the octonion 1.  That already
means that the coordinate system used above is not the one we usually
use for octonions, where

(x_{0},...,x_{7}) = 
x_{0} + x_{1} e_{1} + ... + x_{7} e_{7}


with e_{1},...,e_{7} being the unit imaginary octonions,
which we multiply using the standard octonion multiplication table.  And
just rotating the lattice any old way so that it contains 1 =
(1,0,0,0,0,0,0,0) is not good enough; you have to do it the <em>right
way</em> to get a lattice closed under multiplication.


The right way is described in Conway and Sloane's book (see "<A
HREF = "week20.html">week20</A>").  These days you can even look it
up on the web:

3) Neil J. A. Sloane, Index of Lattices, the E_{8} lattice: coding version,
<A HREF = "http://www.research.att.com/~njas/lattices/E_{8}_code.html">http://www.research.att.com/~njas/lattices/E_{8}_code.html</A>

However, it always scared me, because the description involved the
"Hamming code H(8,4,4)".  You see, lattices are closely connected to
coding theory - not coding in the sense of cryptography, but coding in
the sense of efficient data transmission.  In a code like this you want
to pack information as efficiently as possible while keeping some
error-correction ability, and mathematically this is related to the
problem of densely packing spheres in higher-dimensional space!  This is
all very cool, but I don't understand it very well...  and more
importantly, whenever I looked at the description of the Hamming code
H(8,4,4), I could "understand it" in the sense of nodding in mute
assent, but not in the sense of seeing how it was related to anything.

Luckily, I now see how to get around this.  Instead of describing
the Cayley integral octonions using the theory of codes, I now see
how to describe them using the octonion multiplication table!  
I'm sure everyone else already knew this - but they never told me.

Here's how it goes.  First you have to remember your multiplication
table - the octonion multiplication table, that is.  Draw an equilateral
triangle, draw a line from each corner to the midpoint of the opposite
side, and inscribe a circle in the triangle.  Then label the corners,
the midpoints of the edges and the center of the triangle with the unit
imaginary octonions, any way you like:

<DIV ALIGN="CENTER">

<A NAME="Fano"></A></P><IMG SRC="fano.jpg">
</DIV>

There are 6 straight lines and a circle here: we call these all
"lines", and call this gadget the "Fano plane".
There are 7 points and 7 lines: each point lies on 3 lines, and each
line goes through 3 points... very nice.

I won't describe how to use this picture to multiply octonions, since I
already did that in "<A HREF = "week104.html">week104</A>",
and we won't need that here.

Now let me describe the Cayley integral octonions.  I'll actually
describe all 240 of them that have length 1.  Integer linear
combinations of these give the Cayley integral octonions - or in 
other words, a rescaled version of the E_{8} lattice.

First, we include &#177;e_{i} for i=0,...,7.  Second, we include 


$$

(&#177; 1 &#177; e_{i} &#177; e_{j} &#177; e_{k})/2
$$
    
whenever e_{i}, e_{j} and e_{k}
are imaginary octonions that all lie on the
same line in the above chart.  Third, we include


$$

(&#177; e_{i} &#177; e_{j} &#177; e_{k} &#177; e_{l})/2
$$
    
whenever
e_{i}, e_{j}, e_{k} and e_{l}
are imaginary octonions that all lie \emph{off}
the same line in the above chart.


It's easy to see that all these octonions have length 1.  It's also easy
to count them!  There are 2 x 8 = 16 of the first form, 2^{4} x
7 = 112 of the second form, and 2^{4} x 7 = 112 of the third
form, for a total of 240.

It's harder to check that these 240 guys are closed under
multiplication.  You can save some work by noticing that each line in
the Fano plane gives a copy of the quaternions sitting inside the
octonions.  Moreover, the 24 quaternions of the form


\begin{verbatim}

&#177;1, &#177;i, &#177;j, &#177;k,   (&#177; 1 &#177; i &#177; j &#177; k)/2
\end{verbatim}
    

are closed under multiplication - these are just the unit vectors
among the "Hurwitz integral quaternions", which form a
D_{4} lattice in the quaternions (see "<A HREF =
"week91.html">week91</A>").  So, each line in the Fano plane
gives a copy of the integral quaternions sitting inside the integral
octonions.  Even better - I'm sorry, this is getting a bit technical,
but I need to write it down or I'll forget! - if we do the
Cayley-Dickson construction (see "<A HREF =
"week59.html">week59</A>") to any of these copies of the integral
quaternions, we get a bigger set of integral octonions that's also
closed under addition and multiplication.  Unfortunately, this bunch
is just a copy of D_{4} \times  D_{4} sitting inside
E_{8}, not the whole E_{8}.  E_{8} is the
union of all these D_{4} \times  D_{4}'s, one for line in
the Fano plane.  So, I have to calculate more to finish convincing
myself that the Cayley integral octonions are closed under
multiplication - or equivalently, that the 240 guys listed above are
closed under multiplication.

[Note: I later realized that they are \emph{not} closed under 
multiplication!  We have a perfectly fine E_{8} lattice, so
everything that follows is okay... but it's not the Cayley integral 
octonions!  I'll explain this next week.]

Anyway: this probably makes no sense to you, but \emph{I'm} happy as a clam!
So what can I do with them, for example?

Well, I can see some ways to make E_{8} into a \emph{graded} Lie algebra!

I guess I should start by saying some general stuff about graded Lie
algebras, which explains why this is interesting.


For starters, I'm not talking about Z/2-graded Lie algebras, also known
as "Lie superalgebras"; I'm talking about taking a plain old
Lie algebra L and writing it as a direct sum of subspaces L(i), one for
each integer i, such that


\begin{verbatim}

[L(i), L(j)] is contained in L(i+j).
\end{verbatim}
    
If only the middle 3 of these subspace are nonzero, like so:


$$

L = L(-1) \oplus  L(0) \oplus  L(1)
$$
    
we say L is "3-graded". If only the middle 5 are nonzero, like so:


$$

L = L(-2) \oplus  L(-1) \oplus  L(0) \oplus  L(1) \oplus  L(2)
$$
    

we say L is "5-graded".  And so on.  In these situations,
some nice things happen.

First of all, L(0) is always a Lie subalgebra of L.  Second of all, 
it acts on each other space L(i) by means of the bracket.  Third of all,
if L is 3-graded, we can give L(1) a product by picking any element k of
L(-1) and defining


\begin{verbatim}

x o y = [[x,k],y]
\end{verbatim}
    
This product automatically satisfies two of identities defining
a Jordan algebra: 


\begin{verbatim}

x o y = y o x

x o ((x o x) o y) = (x o x) o (x o y)
\end{verbatim}
    
so 3-graded Lie algebras are a great source of Jordan algebras.  Fourth
of all, in this situation L(0) acts on L(1) by means of the bracket
operation, so we get a Lie algebra of "infinitesimal
symmetries" of our Jordan algebra, too.  Fifth of all, if L is
5-graded, we get a more fancy algebraic structure called a "Kantor
triple system", but I'm not ready to talk about these, and you're
probably not ready to listen, either!

There's a lot more to say about this stuff, but let's just see a bit
about how it works for E_{8}.  We've got two nice pictures of the 240 roots
of the E_{8} lattice; you should imagine these as the dazzling vertices of
a beautiful diamond in 8 dimensions.  To get a grading on E_{8}, all we
need to do is slice this diamond with evenly spaced parallel hyperplanes
in such a way that each vertex of the diamond, as well as its center,
lies on one of these hyperplanes.  There are different ways to do this,
so you should imagine yourself as a gem cutter, turning around this
diamond, looking for nice ways to slice it.

For example, if we use our picture of the E_{8} lattice as 8-tuples that
sum to an even integer are either all integers or all half-integers, one
obvious way to slice the diamond is to let each slice go through those
roots where the first coordinate takes on some fixed value.  The first
coordinate can be 1, 1/2, 0, -1/2, or -1, so we get a 5-grading.  Let's
work out how many roots there are of each kind:


\begin{verbatim}

The number of roots with a "1" as the first component is
7 + 7 = 14.

The number of roots with a "1/2" as the first component is
1 + (7 choose 5) + (7 choose 3) + (7 choose 1) = 1 + 21 + 35 + 7 = 64.

The number of roots with a "0" as the first component is 84.

The number of roots with a "-1/2" as the first component is
1 + (7 choose 5) + (7 choose 3) + (7 choose 1) = 1 + 21 + 35 + 7 = 64.

The number of roots with a "-1" as the first component is
7 + 7 = 14.
\end{verbatim}
    
Since I'm lazy, I figured out the number of roots with a "0" 
as the first component by totalling up all the rest and subtracting
that from 240.  That's how I got the number 84.

Now, whenever you have a simple Lie algebra it's a direct sum of
"root spaces", one for each root, together with an
n-dimensional subspace called the Cartan algebra, where n is the called
the "rank" of the Lie algebra.  The rank of E_{8} is 8, so its
dimension is 240 + 8 = 248.  When we taking our way of slicing the
diamond and convert it into a grading of E_{8}, the roots in the ith slice
form a basis of L(i), except we also have to count the Cartan as part of
L(0).  Thus in this example the dimension of L(0) is not just 84 but 84
+ 8 = 92.  Some basic stuff about simple Lie algebra guarantees that
this trick always works: we get


\begin{verbatim}

[L(i), L(j)] is contained in L(i+j)
\end{verbatim}
    
as desired.  

So, in this example we get a 5-grading where


$$

E_{8} =  L(-2) \oplus  L(-1) \oplus  L(0) \oplus  L(1) \oplus  L(2)
248 =  114  +  64   +  92   +  64  +  14
$$
    
where I'm writing the dimension of each vector space direct below it.

Now, L(0) is a Lie algebra, but which one?  To figure this out we need
to think about how this diamond-cutting trick worked.  At least in this
case - and in fact it often works like this - the roots in the 0th slice
are just the roots of a simple Lie algebra of rank one less than the one
we started with.  Since the Cartan of this smaller Lie algebra is one
dimension smaller, it turns out that L(0) equals this smaller Lie
algebra plus a one-dimensional abelian subalgebra - namely u(1).

In this example this smaller Lie algebra is so(14), which has dimension
91.  L(1) is a 64-dimensional chiral spinor rep of so(14), and L(2) is
the 14-dimensional vector rep... and similarly for L(-1) and L(-2).  
So we get a very "14-dimensional" picture of E_{8}:


$$

E_{8} =  [vectors] \oplus  [spinors] \oplus  [so(14) \oplus  u(1)] \oplus  [spinors] \oplus  [vectors]
$$
    

But we get a more exciting way of slicing the diamond if we use the
picture of E_{8} as the Cayley integral octonions!  Let's do this, and let
each slice go through those roots where the "real part"
x_{0} of our octonion


$$

x_{0} + x_{1} e_{1} + ... + x_{7} e_{7}
$$
    
takes on some fixed value.  This value can be 1, 1/2, 0, -1/2,
or -1, so we again get a 5-grading.  Let's count the number of roots in
each slice:


\begin{verbatim}

The number of roots with real part 1 is 1.

The number of roots with real part 1/2 is 56.

The number of roots with real part 0 is 126.

The number of roots with real part -1/2 is 56.

The number of roots with real part -1 is 1.
\end{verbatim}
    
Here I got 56 roots with real part 1/2 by multiplying the number of lines
in the Fano plane by the number of sign choices in


$$

(1 &#177; e_{i} &#177; e_{j} &#177; e_{k})/2
$$
    
Similarly for the roots with real part -1/2.  I got 126 roots with real
part 0 by subtracting all the other numbers on my list from 240.

So, we get a 5-grading of E_{8} like this:


$$

E_{8} =  L(-2) \oplus  L(-1) \oplus  L(0) \oplus  L(1) \oplus  L(2)
248 =   1   +  56   +  134  +  56  +  1
$$
    
since 126 + 8 = 134.   

This shows how to get E_{8} to act on
a 57-dimensional manifold: we form the group E_{8}, and form
the subgroup G whose Lie algebra is L(-2) \oplus  L(-1) \oplus  L(0), and
the quotient E_{8}/G will be a 57-dimensional space on which E_{8}
acts!   In fact this space is the smallest "Grassmannian" of
E_{8}, as explained in "<A HREF = "week141.html">week181</A>" 
- look at the picture of the E_{8} Dynkin diagram near the end.

My goal in life is now to define a set of
algebraic varieties, one for each root in L(1) and L(2), so I 
can write a paper entitled "57 Varieties" and get sued for trademark 
infringement by Heinz.

In the above grading of E_{8}, the Lie algebra L(0) is the direct sum of E_{7} 
and u(1).  This is no surprise if you know that the dimension of E_{7} is 133... 
but the reason it's \emph{true} is that if you take the roots of E_{8} that are
orthogonal to any one root, you get the roots of E_{7}.  So, we get a very
E_{7}-ish description of E_{8}:


$$

E_{8} =  [trivial] \oplus  [Freudenthal] \oplus  [E_{7} \oplus  u(1)] \oplus  [Freudenthal] \oplus  [trivial]
$$
    

Here the "Freudenthal algebra" is the 56-dimensional irrep
of E_{7}, which has an invariant symplectic structure and
ternary product satisfying some funky equations which get turned into
the definition of... a Freudenthal algebra!

There are a lot of other games we can play like this, but like solitaire
they're not too fun to watch, so I'll just mention one more, and then
give a bunch more references.

Above we have seen the roots of E_{7} as the imaginary Cayley
integral octonions of norm 1.  These form a 7-dimensional gemstone
with 126 vertices, and we can repeat the same "gem-slicing"
trick on a smaller scale to get gradings of the Lie algebra
E_{7}.  If we do this in a nice way, we get a 3-grading of
E_{7}:


$$

E_{7}  =  L(-1) \oplus  L(0) \oplus  L(1)
133 =    27  +  79  +  27
$$
    

Since E_{7}'s baby brother E_{6} is 78-dimensional,
it's no surprise that the Lie algebra L(0) is E_{6} plus u(1).
Since 3-gradings tend to give us Jordan algebras, it's no suprise that
L(1) is the exceptional Jordan algebra h_{3}(O) consisting of
all 3x3 hermitian octonionic matrices.  E_{6} acts as the
group of all transformations of h_{3}(O) preserving the
determinant, and in fact h_{3}(O) is an irrep of
E_{6}.  L(-1) is just the dual of this rep.  So, we get a very
octonionic description of E_{7}:


$$

E_{7} = h_{3}(O)* \oplus  [E_{6} \oplus  u(1)] \oplus  h_{3}(O)
$$
    

Now, since E_{6} sits in E_{7} which sits in
E_{8}, just like nested Russian dolls, we can take our
previous description of E_{8}:


$$

E_{8}  =  [trivial] \oplus  [Freudenthal] \oplus  [E_{7} + u(1)] &oplus [Freudenthal] \oplus  [trivial]
$$
    

and decompose everything in sight as irreps of E_{6}.  If we
do this, the only new exciting thing that happens is that the
Freudenthal algebra decomposes into a copy of the exceptional Jordan
algebra, a copy of its dual, and two copies of the trivial rep:


$$

[Freudenthal] = [trivial] \oplus  h_{3}(O)* \oplus  h_{3}(O) \oplus  [trivial] 
$$
    
At least I \emph{think} this is right: people sometimes write elements
of the Freudenthal algebra as 2\times 2 matrices


\begin{verbatim}

a x 
y b
\end{verbatim}
    
where a,b are real and x,y lie in h_{3}(O), but I suspect they're
"cheating" a bit and identifying h_{3}(O) with its dual.  

In short, E_{8} contains a lot of other
"exceptional" structures, all arranged in a very nice way.

Now for some references and apologies.

I didn't do justice to the stuff about Jordan algebras and 3-graded
Lie algebras, because I'm still confused about certain aspects.  For
example, where does the unit in the Jordan algebra come from?  I also
didn't explain precisely what sort of "infinitesimal
symmetries" we get from the action of L(0) on L(1).  If we
exponentiate these infinitesimal symmetries, we don't usually get
automorphisms of L(1), since there's no reason for the element
"k" to preserved - remember that


\begin{verbatim}

x o y = [[x,k],y]
\end{verbatim}
    
Instead, we get transformations that tend to preserve a "determinant"
on L(1).  People call L(0) the "structure algebra" of L(1) and call 
the corresponding group the "structure group".  There's a pretty 
readable explanation here:

4) Kevin McCrimmon, Jordan Algebras and their applications, Bull. 
AMS 84 (1978) 612-627.

and hopefully even more here:

5) Kevin McCrimmon, A Taste of Jordan Algebras, Springer, Berlin,
perhaps to appear in March 2003.  Available for free online at
<A HREF = "http://math1.uibk.ac.at/mathematik/jordan/archive/atoja/">http://math1.uibk.ac.at/mathematik/jordan/archive/atoja/</A> - but watch out,
it's 545 pages long!

In fact, all this is part of a bigger relationship between 3-graded 
Lie algebras and so-called "Jordan triple systems" known as the
Tits-Kantor-Koecher construction.  Jordan triple systems are a
generalization of Jordan algebras - and I'm sort of confused about 
why this generalization also turns up here.  I guess I should read
these too:

6) J. Tits, Une class d'algebres de Lie en relations avec les algebres
de Jordan, Ned. Akad. Wet., Proc. Ser. A 65 (1962), 530.

7) M. Koecher, Imbedding of Jordan algebras into Lie algebra I,
Am. J. Math. 89 (1967), 787. 

8) Soji Kaneyuki, Graded Lie algebras, related geometric structures,
and pseudo-hermitian symmetric spaces, in Analysis and Geometry on 
Complex Homogeneous Domains, by Faraut, Kaneyuki, Koranyi, Lu, and 
Roos, Birkhauser, New York, 2000.

Kaneyuki has made some nice tables of 3-gradings on simple Lie
algebras, and you can see some of these here:

9) Tony Smith, Graded Lie algebras, 
<A HREF = "http://www.innerx.net/personal/tsmith/GLA.html">http://www.innerx.net/personal/tsmith/GLA.html</A>
Thomas Larsson has made a nice table of all the formally real
simple Jordan algebras you get from 3-graded simple Lie algebras, 
and here it is, slightly modified:


\begin{verbatim}

Lie algebra L   L'(0)           dim(L(1))      Jordan algebra L(1)

sl(n+1)         sl(n)           n               R^{n-1} \oplus  R
so(n+2)         so(n)           n               R^{n-1} \oplus  R
sp(2n)          sl(n)           (n^{2}+n)/2        h_{n}(R)
so(2n)          sl(n)           (n^{2}-n)/2        h_{n-1}(R)
sl(2n)          sl(n)+sl(n)     n^{2}              h_{n}(C)
so(4n)          sl(2n)          2n^{2}-n           h_{n}(H)
E_{7}               E_{6}             27              h_{3}(O)
E_{6}               so(10)         16              h_{4}(C)
\end{verbatim}
    
Since L(0) always contains a u(1) summand in these cases, we
write 


\begin{verbatim}

L(0) = L'(0) + u(1)
\end{verbatim}
    
so that L'(0) is the interesting part of L(0).  The formally
real simple Jordan algebras appearing here are all those listed
in "<A HREF = "week162.html">week162</A>" - we get all of them!  
In particular, R^{n-1} + R is 
the so-called "spin factor" Jordan algebra, which appears
in special relativity.  

For the more intricate relationship between 5-graded Lie algebras, 
Freudenthal algebras and Kantor triple systems, I should reread these: 

10) I. Kantor, I. Skopets, Some results on Freudenthal triple systems,
10el. Math. Sov. 2 (1982), 293.

11) K. Meyberg, Eine Theorie Der Freudenthalschen Tripelsysteme, I, II,
Ned. Akad. Wet., Proc. Ser. A 71 (1968), 162-190.

12) R. Skip Garibaldi, Structurable algebras and groups of types E_{6} and 
E_{7}, available at <A HREF = "http://www.arXiv.org/abs/math.RA/9811035">math.RA/9811035</A>.

13) R. Skip Garibaldi, Groups of type E_{7} over arbitrary fields,
available at
<A HREF = "http://www.arXiv.org/abs/math.RA/9811056">math.RA/9811056</A>.

14) G. Sierra, An application of the theories of Jordan algebras and 
Freudenthal triple systems to particles and strings, Class. Quant. Grav. 
4 (1987), 227-236.  

Also, I didn't say anything yet about the connection of Lie triple
systems, Jordan algebras, and Jordan triple systems to the geometry 
of symmetric spaces!  There is in fact a dictionary relating these 
funny algebraic structures to very nice kinds of geometry, which
motivates the Tits-Kantor-Koecher construction and its generalizations.
Someday I may understand this well enough to explain it.  For now, 
you should try to get ahold of these:

15) W. Bertram, The Geometry of Jordan and Lie structures, 
Lecture Notes in Mathematics 1754, Springer, Berlin, 2001.

16) Ottmar Loos, Jordan triple systems, R-spaces and bounded symmetric 
domains, Bull. AMS 77 (1971), 558-561. 

17) Ottmar Loos, Symmetric Spaces I: General Theory, W. A. Benjamin, 
New York, 1969.  Symmetric Spaces II: Compact Spaces and Classification,
W. A. Benjamin, New York, 1969.

Unfortunately of the last two books I can get only volume I at U.C. 
Riverside, and only volume II here at Macquarie University!  Someone 
should reprint both of these books: they're nice.  Loos has also
written a book on "Jordan pairs", but in my current state of
development I find that unreadable.

\par\noindent\rule{\textwidth}{0.4pt}
Addendum: Blichfeldt proved in 1935 that E_{8} is a maximally dense
lattice packing of spheres in 8 dimensions, and Vetcinkin proved
in 1980 that it's the \emph{unique} lattice packing that 
achieves this density in 8 dimensions.
Now Cohn and Kumar have shown that the E_{8} packing is darn close to the 
densest of \emph{all} sphere packings in 8 dimensions, lattice or not.
No other can be more than 1 + 10^{-14} as dense as this one!

They also showed that in 24 dimensions no packing can be more than
1 + 10^{-29} times as dense as the Leech lattice, and that
this is the unique best lattice packing.  Of course the 
E_{8} and Leech lattices are probably the best of all sphere packings 
in their dimensions, but it's very hard to understand the set of 
all sphere packings, so even these partial results are amazing.

Here are their papers:
18) H. Cohn and A. Kumar, Optimality and uniqueness of the 
Leech lattice among lattices, available at <A HREF = 
"http://www.arxiv.org/abs/math.MG/0403263">math.MG/0403263</A>.

H. Cohn and A. Kumar, The densest lattice in twenty-four
dimensions, Elec. Res. Ann. 10 (2004), 58-67.   Available online
at <A HREF = "http://www.ams.org/era/2004-10-07/S1079-6762-04-00130-1/home.html">http://www.ams.org/era/2004-10-07/S1079-6762-04-00130-1/</A>

There's also a really nice overview of this topic in the 
American Mathematical Society Notices, which explains how
people manage to prove results about \emph{all} packings:

19) Florian Pfender and G&uuml;nter M. Ziegler, 
Kissing numbers, sphere packings, and some unexpected
proofs, AMS Notices 51 (September 2004), 873-883.
Available online at 
<A HREF = "http://www.ams.org/notices/200408/200408-toc.html">
http://www.ams.org/notices/200408/200408-toc.html</A>

And while you're at it, read this article, which
studies a question mentioned
in "<A HREF = "week20.html">week20</A>":

20) Bill Casselman, The difficulties of kissing in three 
dimensions, AMS Notices 51 (September 2004), 884-885.
Available online at <A HREF = "http://www.ams.org/notices/200408/200408-toc.html">
http://www.ams.org/notices/200408/200408-toc.html</A>

namely, how to roll twelve balls in 3 dimensions around the surface 
of a thirteenth ball of equal size. 


\par\noindent\rule{\textwidth}{0.4pt}
<em>The essential thing was that Serre each time strongly sensed
the rich meaning behind a statement that, on the page, would doubtless
have left me neither hot nor cold - and that he could "transmit"
this perception of a rich, tangible and mysterious substance - this
perception that is at the same time the \textbf{desire} to understand
this substance, to penetrate it.</em> - Alexandre Grothendieck, 
\emph{R&eacute;coltes et Semailles}, p. 556.
 
\par\noindent\rule{\textwidth}{0.4pt}

% </A>
% </A>
% </A>
