
% </A>
% </A>
% </A>
\week{May 9, 2005 }

This week I'd like to report on some cool things people have been
explaining to me.  The science fiction writer Greg Egan has been 
helping me understand Klein's quartic curve, and the mathematician
Darin Brown has been explaining the analogy between geodesics and 
prime numbers.  The two subjects even overlap slightly!

Last week I talked about Klein's quartic curve.  This led
Gerard Westendorp and Mike Stay to draw some pictures of it, 
and their ideas helped Greg Egan create this really nice picture:

1) Greg Egan, Klein's quartic curve,
<A HREF = "http://math.ucr.edu/home/baez/mathematical/KleinDual.png">
http://math.ucr.edu/home/baez/mathematical/KleinDual.png</A>

<DIV ALIGN = CENTER>
<IMG SRC = "http://math.ucr.edu/home/baez/mathematical/KleinDual.png">
</DIV>


It looks sort of tetrahedral at first glance, but if you look 
carefully you'll see that topologically speaking, it's a 3-holed
torus.  It's tiled by triangles, with 7 meeting at each vertex.  
So, it's the Klein quartic curve!

Perhaps I should explain.  Last week I talked about a tiling of the 
hyperbolic plane by regular heptagons with 3 heptagons meeting at 
each vertex.  Dual to this is a tiling of the hyperbolic plane by 
equilateral triangles with 7 triangles meeting at each vertex.  
We can take a quotient space of this by a certain symmetry group 
and get a 3-holed torus tiled by 56 triangles with 7 meeting at each
vertex.  This is what Egan drew!

With this picture you can almost \emph{see} the 168 symmetries of Klein's 
quartic curve.

First, you can take any vertex and twist it, causing the 7 triangles
that meet at this vertex to cycle around.  It's not obvious that this
is a symmetry of the whole tiled surface, but it is.  This gives a 
7-element symmetry group.  

Second, the whole thing looks like a tetrahedron, so it inherits the
rotational symmetries of a tetrahedron.  This gives a more obvious
12-element symmetry group.  

7 \times  12 = 84, so how do we get a total of 168 symmetries?

Well, there's also a 2-fold symmetry that corresponds to turning 
the tetrahedron inside out!  And Egan made a wonderful \emph{movie} of 
this.  If a picture is worth a thousand words, this is worth about 
a million:

2) Greg Egan, Turning Klein's quartic curve inside out,
<A HREF = "http://math.ucr.edu/home/baez/mathematical/KleinDualInsideOut.gif">
http://math.ucr.edu/home/baez/mathematical/KleinDualInsideOut.gif</A>


<DIV ALIGN = CENTER>
<IMG SRC = "http://math.ucr.edu/home/baez/mathematical/KleinDualInsideOut.gif">
</DIV>
So, we get a total of 7 \times  24 = 168 symmetries. 

Even better, if you watch carefully, you'll see that the tetrahedron 
in Egan's movie gets \emph{reflected} as it turns inside out.  More 
precisely, if you follow the four corners of the tetrahedron, 
you'll see that two come back to where they were, while the other 
two get switched.  So, this symmetry acts as a reflection, or odd 
permutation, of the 4 corners.  The rotations act as even permutations 
of the corners.

This means that the Klein quartic has 24 symmetries forming a group
isomorphic to the rotation/reflection symmetry group of a tetrahedron.
Algebraically speaking, this group is S_{4}: 
the permutations of 4 things.

This group is also the rotational symmetry group of a cube.  In fact, 
Egan was able to spot a hidden cube lurking in his picture!  Can you?

<DIV ALIGN = CENTER>
<IMG HEIGHT = 300 WIDTH = 300 
SRC = "http://math.ucr.edu/home/baez/mathematical/KleinDual.png">
</DIV>

If you look carefully, you'll see each corner of his tetrahedral
gadget is made of a little triangular prism with one triangle facing 
out and one facing in: for example, the pink triangle
staring you right in the face, or the light blue one on top.
Since 4 \times  2 = 8, there are 8 of these triangles.  
Abstractly, we can think of these as the 8 corners of a cube!  They 
aren't really, but we can pretend.  The way these 8 triangles come 
in pairs corresponds to how the vertices of a cube come in diagonally 
opposite pairs.  

Using this, you can see that the group S_{4} acts on these 8 triangles 
in precisely the same way it acts via rotations on the vertices of a 
cube.

In fact, you can even draw a \emph{picture} of a cube on the Klein quartic by 
drawing suitable curves that connect the centers of these 8 triangles!
It's horribly distorted, but topologically correct.  Part of the 
distortion is caused by embedding the Klein quartic in ordinary
3d Euclidean space.  If we gave the Klein quartic the metric it 
inherits from the hyperbolic plane, the edges of the cube would be 
geodesics.

This remark also helps us see something else.  The Klein quartic is
tiled by 56 triangles.  8 of them give the cube we've just been 
discussing.  In Egan's picture these triangles look special, since 
they lie at the corners of his tetrahedral gadget.  But this is
just an illusion caused by embedding the Klein quartic in 3d space.  
In reality, the Klein quartic is perfectly symmetrical: every 
triangle is just like every other.  So in fact there are lots of 
these cubes, and every triangle lies in some cube.

But this is where it gets really cool.  In fact, each triangle lies
in just \emph{one} cube.  So, there's precisely one way to take the 56 
triangles and divide them into 7 bunches of 8 so that each bunch forms 
a cube.   

So: the symmetry group of the Klein quartic acts on the set of cubes,
which has 7 elements.

But as I explained last week, this symmetry group also acts on the 
Fano plane, which has 7 points.

This suggests that cubes in the Klein quartic naturally correspond 
to points of the Fano plane.  And Egan showed this is true!

He showed this by showing more.  The Fano plane also has 7 lines.
What 7 things in the Klein quartic do these lines correspond to?

\emph{Anticubes!}

You see, the cubes in the Klein quartic have an inherent handedness
to them.  You can go between the 8 triangles of a given cube by 
following certain driving directions, but these driving directions 
involve some left and right turns.  If you follow the mirror-image
driving directions with "left" and "right" switched, 
you'll get an \emph{anticube}.
 
Apart from having the opposite handedness, anticubes are just like
cubes.  In particular, there's precisely one way to take the 56 
triangles and divide them into 7 bunches of 8 so that each bunch forms 
an anticube.

Here's a picture:

3) Greg Egan, Cubes and anticubes in the Klein quartic curve,
<A HREF = "http://math.ucr.edu/home/baez/KleinFigures.gif">
http://math.ucr.edu/home/baez/KleinFigures.gif</A>

<BR>
<DIV ALIGN = CENTER>
<IMG SRC ="http://math.ucr.edu/home/baez/KleinFigures.gif">
</DIV>
<BR>

Each triangle has a colored circle and a colored square on it.
There are 7 colors.  The colored circle says which of the 7 \emph{cubes}
the triangle belongs to.  The colored square says which of the 7
\emph{anticubes} it belongs to.  

If you stare at this picture for a few hours, you'll see that each 
cube is completely disjoint from precisely 3 anticubes.  Similarly, 
each anticube is completely disjoint from precisely 3 cubes.  

This is just like the Fano plane, where each point lies on 3 lines, 
and each line contains 3 points!

So, we get a vivid way of seeing how every figure in the Fano 
plane corresponds to some figure in the Klein quartic curve. 
This is why they have the same symmetry group.  

This is an excellent example of Klein's Erlangen program for reducing 
geometry to group theory, which I discussed in "<A HREF = "week213.html">week213</A>".  Here we are 
beginning to see how two superficially different geometries are secretly 
the same:


\begin{verbatim}

    FANO PLANE                      KLEIN'S QUARTIC CURVE
 
    7 points                        7 cubes
    7 lines                         7 anticubes
    incidence of points and lines   disjointness of cubes and anticubes
\end{verbatim}
    
However, we're only half done!  We've seen how to translate simple 
figures and indicence relations in the Fano plane to complicated ones
in Klein's quartic curve.  But, we haven't figured out translate back!
 

\begin{verbatim}

    KLEIN'S QUARTIC CURVE            FANO PLANE
       
    24 vertices                      ??? 
    84 edges                         ???
    56 triangular faces              ???
    incidence of vertices and edges  ???
    incidence of edges and faces     ???
\end{verbatim}
    
Here I'm talking about the tiling of Klein's quartic curve by 56
equilateral triangles.  We could equally well talk about its tiling
by 24 regular heptagons, which is the Poincare dual.  Either way, the
puzzle is to fill in the question marks.  I don't know the answer!

To conclude - at least for now - I want to give the driving directions 
that define a "cube" or an "anticube" in Klein's quartic curve.  Say 
you're on some triangle and you want to get to a nearby triangle that 
belongs to the same cube.  Here's what you do:

\begin{quote}
hop across any edge, <br>
turn right, <br>
hop across the edge in front of you, <br>
turn left, <br>
then hop across the edge in front of you.  

\end{quote}
Or, suppose you're on some triangle and you want to get to another
that's in the same anticube.  Here's what you do:

\begin{quote}
hop across any edge, <br>
turn left, <br>
hop across the edge in front of you, <br>
turn right, <br>
then hop across the edge in front of you.  
 
\end{quote}
(If you don't understand this stuff, look at the picture above and 
see how to get from any circle or square to any other circle or 
square of the same color.) 

You'll notice that these instructions are mirror-image versions of 
each other.  They're also both 1/4 of the "driving directions from 
hell" that I described last time.  In other words, if you go 
LRLRLRLR or RLRLRLRL, you wind up at the same triangle you started
from.  You'll have circled around one face of a cube or anticube!

In fact, your path will be a closed geodesic on the Klein quartic 
curve... like the long dashed line in Klein and Fricke's original 
picture:

4) Klein and Fricke, Klein's quartic curve with geodesic,
<A HREF = "http://math.ucr.edu/home/baez/Klein168.gif">http://math.ucr.edu/home/baez/Klein168.gif</A>

<DIV ALIGN = CENTER>
<IMG SRC = "http://math.ucr.edu/home/baez/Klein168.gif">
</DIV>
Next, a little about geodesics and prime numbers.   I've just been
talking a little about geodesics in the Klein quartic, which is the
quotient 

H/G

of the hyperbolic plane H by a certain group G which I explained  
last week.  This group, usually called \Gamma (7), is a nice example 
of a "Fuchsian group" - that is, a discrete subgroup of the isometries
of the hyperbolic plane.  

Darin Brown and his thesis advisor Jeff Stopple at U. C. Santa
Barbara have been thinking about geodesics in H/G for other Fuchsian
groups G, and their relation to number theory:

5) Jeff Stopple, A reciprocity law for prime geodesics, J. Number 
Theory 29 (1988), 224-230.

6) Darin Brown, Lifting properties of prime geodesics on hyperbolic
surfaces, Ph.D. thesis, U. C. Santa Barbara, 2004.

I'd really like to learn about this, because it connects all sorts
of stuff I dream of understanding someday, especially quantum chaos 
("<A HREF = "week190.html">week190</A>"), zeta functions in physics and number theory ("<A HREF = "week199.html">week199</A>"), 
and Galois theory as a theory of covering spaces ("<A HREF = "week205.html">week205</A>").  Also, 
it involves a big mysterious analogy, and I always like those!

I don't understand this stuff well enough to try a full-fledged
explanation yet, so I'll just give a vague sketch.  A "prime geodesic"
in a Riemannian manifold X is a closed geodesic 

f: S^{1} \to  X

that cycles around just once.  In other words, f should be one-to-one.

We say a closed geodesic is the "nth power" of a prime one if it's
just like the prime one but it cycles around n times.  Every closed
geodesic is the nth power of a prime one in a unique way.

If we have a Fuchsian group G, H/G is a surface with a Riemannian
metric.  It looks locally like the hyperbolic plane, so it's called
a "hyperbolic surface".  And, we can look at prime geodesics in it.  

If G' is a subgroup of G, we get a covering map

H/G' \to  H/G

so we can ask about lifting prime geodesics in H/G to closed geodesics
in H/G'.   There can be a bunch of ways to do this, so we say a
prime geodesic in H/G "splits" into powers of prime geodesics up in
H/G'.  

If you know any number theory - reading "<A HREF = "week205.html">week205</A>" should be enough -
this should remind you of how a prime ideal in some algebraic number 
field can "split" into prime ideals in an extension of this field, 
and/or "ramify" into powers of prime ideals.

And indeed, Darin Brown has found a big mysterious analogy that goes 
like this:

\begin{quote}

$$

Number field K                   Hyperbolic surface H/G
Field extension K' of K          Covering p: H/G' \to  H/G
Galois group Gal(K'/K)           Deck transformation group Aut(p)
Prime ideal Q of K               Prime geodesic f in H/G
Prime ideal Q' lying over Q      Prime geodesic f' lying over f
Splitting of prime ideal Q of K' Lifting of prime geodesic f to H/G'
Norm N(Q) of ideal Q             Norm N(f) of closed geodesic f
Frobenius conjugacy class of Q   Frobenius conjugacy class of f
Artin L-function                 Selberg zeta function
$$
    
\end{quote}

(Here by "prime ideal of K" we mean a prime ideal in the 
ring of algebraic integers of K.)

But this is more than an analogy: there's even a way to associate number
fields to certain hyperbolic surfaces!  The reason is that often
Fuchsian groups will consist of matrices whose entries lie in some
number field. 

I would like to understand the Selberg zeta function and its relation
to quantum mechanics.  The Selberg zeta function is related to closed 
geodesics, which are periodic classical trajectories, while the zeta 
function of a Laplacian is related to periodic \emph{quantum} trajectories 
(namely eigenfunctions of the Laplacian).  So, the two are related.  
I know there's a lot of cool stuff going on here - especially since 
the motion of a particle on a hyperbolic surface tends to be chaotic,
so "quantum chaos" rears its ugly head.  But, I don't understand any of the 
details.

In some notes on quantum chaos, Gutzwiller wrote:

\begin{quote}
  The classical periodic orbits are a crucial stepping stone in the 
  understanding of quantum mechanics, in particular when then classical 
  system is chaotic.  This situation is very satisfying when one 
  thinks of Poincar&eacute; who emphasized the importance of periodic orbits 
  in classical mechanics, but could not have had any idea of what 
  they could mean for quantum mechanics.  The set of energy levels and
  the set of periodic orbits are complementary to each other since 
  they are essentially related through a Fourier transform.  Such a 
  relation had been found earlier by the mathematicians in the study 
  of the Laplacian operator on Riemannian surfaces with constant 
  negative curvature.  This led to Selberg's trace formula in 1956 
  which has exactly the same form, but happens to be exact.  The 
  mathematical proof, however, is based on the high degree of symmetry
  of these surfaces which can be compared to the sphere, although the 
  negative curvature allows for many more different shapes.

\end{quote}
When I get serious, I'll read these:

7) M. C. Gutzwiller, Chaos in Classical and Quantum Mechanics,
Springer, Berlin, 1990.

8) Predrag Cvitanovic, Roberto Artuso, Per Dahlqvist, Ronnie Mainieri, 
Gregor Tanner, Gabor Vattay, Niall Whelan and Andreas Wirzba, Chaos: 
Classical and Quantum, available at <A HREF = "http://www.nbi.dk/ChaosBook/">
http://www.nbi.dk/ChaosBook/</A>

9) Svetlana Katok, Fuchsian Groups, U. Chicago Press, Chicago, 1992.

10) J. Elstrodt, F. Grunewald, and J. Mennicke, Groups Acting on 
Hyperbolic Space, Springer, Berlin, 1998.

11) Peter Sarnak, Quantum chaos, symmetry and zeta functions, in 
Current Developments in Mathematics, 1997, eds R. Bott et al., 
International Press, Boston, 1999, pp. 127-159.

12) C. Schmit, Quantum and classical properties of some billiards 
on the hyperbolic plane, in Chaos and Quantum Physics, eds.
M.-J. Giannoni et al., Elsevier, New York, 1991, pp. 333-369.

For a nice pop treatment of quantum chaos and the Riemann hypothesis,
try this:

13) Martin Gutzwiller, Quantum chaos, Scientific American, January
1992.  Also available at <A HREF = "http://www.maths.ex.ac.uk/~mwatkins/zeta/quantumchaos.html">http://www.maths.ex.ac.uk/~mwatkins/zeta/quantumchaos.html</A>

\par\noindent\rule{\textwidth}{0.4pt}
\textbf{Addendum:} Here is some email I got from Greg Egan and Toby Bartels,
and a post from Darin Brown which corrects some mistakes and answers
some questions raised by my pal Squark.

Greg Egan wrote me the following after I suggested a relation
between the Klein quartic curve and 3d Minkowski spacetime over the
field Z/7 - a relation that he later exploited in some fascinating
ways.

\begin{quote}

$$

Hi

Thanks for all the Lorentz group stuff!  This will take me a while to 
digest.

In the meantime, here are some more translations between the geometries.

Every cube intersects 4 anticubes, and any \emph{pair} of cubes, between them, 
intersect 6 anticubes (two of the 4 for each will always be shared).  So 
together the pair of cubes single out one anticube:  the 7th one that 
neither of them intersect.  This is analogous to the fact that any two 
Fano points single out a Fano line.

I'll write anti({c1,c2}) for the anticube singled out by a pair of cubes, 
and similarly cube({a1,a2}) for the cube singled out by a pair of 
anticubes.  In the scheme used in this diagram:

<DIV ALIGN = CENTER>
<IMG SRC ="http://math.ucr.edu/home/baez/KleinFigures.gif">
</DIV>

both functions have identical outputs for the same input colours:


    anti({c1,c2})  and  cube({a1,a2})
    =================================

    R   O   Y   G   LB  P   DB
   ----------------------------
R  | -   DB  R   DB  Y   Y   R
O  | DB  -   P   DB  P   O   O
Y  | R   P   -   LB  P   LB  R
G  | DB  DB  LB  -   G   LB  G
LB | Y   P   P   G   -   Y   G
P  | Y   O   LB  LB  Y   -   O
DB | R   O   R   G   G   O   -
   ----------------------------

Now for some actual translations.

Klein's Quartic Curve      Fano plane
---------------------      ----------

28 pairs of opposite       28 choices of a point
triangular faces           and a non-incident line,
                           {p,l}.

                              p1
                              (*)

                           ----------- l1

                           7 x 4 = 28

In Klein's quartic curve, we specify a pair of opposite triangular faces 
by picking one of seven cubes, then one of four anticubes that intersect 
it.  The intersection is a pair of triangular faces which are diagonally 
opposite each other both on the cube and on the anticube.  The 56 order-3 
elements of G preserve these pairs of triangular faces, and consist of 
rotations by 1/3 and 2/3 turns for each such pair.

Triangular faces           Pairings of points and
sharing an edge            non-incident lines {p1,l1} and
                           {p2,l2} having p1 incident on l2 and
                           p2 incident on l1.

                               p1
                           ----(*)----- l2

                           ----(*)----- l1
                               p2

In Klein's quartic curve, whenever two triangular faces share an edge, 
the cube each face belongs to will be disjoint from the anticube that 
the other face belongs to.  This can be checked by noting that the colour 
of the anticube appears in the row for anti(c,.).

If you inspect a triangle and the three neighbours that share edges with 
it, the neighbours will always belong to the three anticubes disjoint 
from the cube the central triangle belongs to, i.e. they will have 
exactly the three colours appearing in the row for anti(c,.)

84 edges                   84 choices of {p1,l1} and {p2,l2}
                           non-incident, but {p1,l2} and {p2,l1}
                           incident.

                               p1
                           ----(*)----- l2

                           ----(*)----- l1
                               p2

                           or equivalently:

                           84 choices of 3 non-colinear points,
                           with one point singled out.  In this
                           definition, the special 3rd point is
                           the one point shared by l1 and l2
                           of the previous definition.

                           (*) p1
                              \
                               \ l2
                                \
                                 (*) p3
                                /
                               / l1
                              /
                           (*) p2

                           We can count this as (7 choose 3) triples,
                           minus 7 triples that are colinear, times
                           three for three choices of distinguished
                           point:

                           ((7 choose 3) - 7) x 3 = 28 x 3 = 84

In Klein's quartic curve, we specify an edge by picking a pair of cubes 
{c1,c2} and then a distinguished third one, c3, so that the three aren't 
all disjoint from any one anticube.  This means that, between them, they 
must intersect all seven anticubes.  So the third cube must be one that 
intersects anti({c1,c2}).  There are exactly 4 of these (and c1 and c2 
aren't among them, by definition).  So another way of counting the total 
is (7 choose 2) x 4 = 21 x 4 = 84 choices.

To identify the particular edge, suppose we've chosen {{c1,c2},c3} as our 
cubes.  Then {c1, anti({c2,c3})} is a cube and an anticube that 
intersects it, which specifies a pair of diagonally opposite triangular 
faces, and the same is true of {c2, anti({c1,c3})}.  There is a unique 
edge where two of these triangles meet.

For example, if we pick {{red, orange}, yellow} then we have {red, 
anti-purple} and {orange, anti-red}.  Both cube/anticube choices specify 
two triangles each, but there is only one edge that belongs to both a 
{red, anti-purple} and an {orange, anti-red} triangle.

To reverse the process, if we look at the cube/anticube markings on the 
triangles either side of some edge, and they are {c1,a1} and {c2,a2}, 
then we can describe this edge by {{c1,c2},cube({a1,a2})}.

Triangular faces             Pairings of points and non-incident
each sharing an              lines {p1,l1} and {p2,l2} having
edge with a common           \emph{either} p1 incident on l2 or p2
neighbour, but not           incident on l1, but \emph{not} both.
each other.  (This
is sufficient, but                p1 [or p2]
not necessary, for           -----(*)-------- l2 [or l1]
them to share a
vertex.)                     ---------------- l1 [or l2]

                                  (*)
                                  p2 [or p1]

In Klein's quartic curve, as you go around a triangle anticlockwise and 
look at its three (edge-sharing) neighbours, the cube a triangle belongs 
to will be disjoint from the anticube of the triangle that follows, but 
the anticube it belongs to will intersect the cube of the triangle that 
follows.  (But what the sense of the rotation means in the Fano plane 
depends on whether we map cubes to points and anticubes to lines or vice 
versa!) 


24 vertices               168 pairings of points and non-incident
                          lines {p1,l1} and {p2,l2} having
                          \emph{either} p1 incident on l2 or p2
                          incident on l1, but \emph{not} both.

                               p1 [or p2]
                          -----(*)-------- l2 [or l1]

                          ---------------- l1 [or l2]

                               (*)
                               p2 [or p1]

                          There are:
                          28 choices for {p1, l1}
                        x  3 choices for l2 passing through p1
                        x (7-5)=2 choices for p2 not in l1 or l2

                          This count identifies each vertex
                          as shared by common neighbours of
                          a particular triangle, so we expect
                          to count each vertex 7 times for the
                          seven triangles.

                          We could double this to count for
                          swapping the role of p1 and p2, and then
                          we'd be counting each vertex twice
                          as often:  once going anticlockwise
                          between each pair of neighbours, and
                          once going clockwise.

This is all a bit strange and inconvenient!  I can pin down an edge, but 
I haven't really pinned down a single face, or a way to count a vertex 
just once.  I guess the answer for a vertex is to talk about an 
equivalence class of the structures:


                               p1 [or p2]
                          -----(*)-------- l2 [or l1]

                          ---------------- l1 [or l2]

                               (*)
                               p2 [or p1]

where we mod out by Z/7 and "gauge fix" l1.  Every vertex is surrounded 
by 7 triangular faces encompassing all seven cubes and all seven 
anticubes, so these equivalence classes do fix a single vertex.

Best wishes

Greg
$$
    
\end{quote}



<br>
<br>

Toby Bartels wrote:

<br>
<br>
\begin{quote}

$$

In Week 215, you wrote:

>We say a closed geodesic is the "nth power" of a prime one
>if it's just like the prime one but it cycles around n times.
>Every closed geodesic is the nth power of a prime one in a unique way.

The latter sentence is not quite true; you've forgotten n = 0 again!

Some manifolds, like the real line, have no prime geodesics,
but every (pointed) manifold has a unique \emph{unit} closed geodesic,
which is the geodesic that just sits at the basepoint the whole time.
Given any prime geodesic f, this unit geodesic is f^{0}.

Thinking about this, I noticed that multiplication of closed geodesics,
which involves (the often technically tricky) concatenation of paths,
has a unique definition that's associative \emph{on the nose}.
(Parametrise by arclength, concatenate, then parametrise to unit length;
since the paths are geodesics, the last step is also unique.)

Unfortunately, this gives no way to define multiplication
of closed geodesics that are (positive) powers of \emph{different} primes.
We could generalise to piecewise geodesics that may turn corners
at the basepoint, but this seems somewhat artificial,
and it doesn't have very nice properties.

-- Toby

$$
    
\end{quote}

Darin Brown wrote, in response to some questions by Squark
on sci.physics.research:

\begin{quote}

$$


Squark wrote:

  John Baez wrote:

   > If G' is a subgroup of G, we get a covering map
   >
   > H/G' \to  H/G
   >
   > so we can ask about lifting prime geodesics in H/G to closed
   > geodesics in H/G'.   There can be a bunch of ways to do this, so we 
   > say a prime geodesic in H/G "splits" into powers of prime geodesics 
   > up in H/G'.

   I don't quite understand how can the lift be a power, rather than just
   a prime.

===

Quite true. When you lift a geodesic, once you get back to the starting
basepoint, you've gone around once up above, corr. to a prime above, so
it doesn't make sense to go around more than once! (I think this is what
the author of this comment meant.) In fact, I think it's true (I can ask
Jeff) that in a sense, there are no "ramified primes" in the geodesic
context. (There are only finitely many in the number theory context.
Actually, ramified primes are bad behaviour in a sense.) It is true,
when you lift a prime, the geodesic above has length a multiple of the
prime below, this is the analogue of the _inertial degree_, not the
ramification degree. It seems all the ramification degrees are 1, and
the magic equation reduces to degree of extension = sum(inertia
degrees).

===

   > Norm N(Q) of ideal Q             Norm N(f) of closed geodesic f

   What is a norm of a geodesic? The length or the energy or... ?

===

Explicitly, the length of a geodesic is the (natural) log of the norm,
or equivalently, the norm is exp(length). For closed geodesics on
\Gamma \H, you find the norm explicitly as follows: consider the
associated hyp. conj. class {\gamma }, take an eigenvalue \epsilon  of an
element of this conj. class, then the norm is \epsilon ^2. The length of
the geodesic is then 2log(\epsilon ). This is independent of the choice of
\gamma  in the conj. class.

This is why I now like to think of the norm of an ideal as a kind of
"length function on ideals".

===

   > Frobenius conjugacy class of Q   Frobenius conjugacy class of f

   Again, what is the Frobenius on the right side here?

===

I can give 2 answers. The first answer is a cop-out, because it would
just give the concrete definition given in Jeff's paper or my thesis,
e.g. Namely, you take the associated matrix \gamma , and reduce entries
mod the prime Q, where Q determines the covering surface \Gamma (Q)\H.
This is a very concrete definition that doesn't hint at the connection
to number theory. Remember, secretly, PSL(2,q), q = norm(Q) is really
(isomorphic to) the deck transformation group of \Gamma (Q)\H over
\Gamma (H), and the Frob conj. class of a geodesic f should be a conj.
class in this deck transformation group. Conceptually, it should be an
element of the decomposition group, those deck transformations that fix
the prime geodesic above. Choosing different primes above the prime
below should give elements of the deck transformation group which are
conjugate to each other. At least, that should be the idea.

darin

$$
    
\end{quote}

Darin's description of the Frobenius associated to a prime
geodesic in H/G is a bit technical.  Here's my guess as to a simpler
description:

We have a covering space of a Riemannian manifold.  A geodesic down
below gives an element of the fundamental group of the base.  This
acts as deck transformations of the cover.  So, it acts on the set
of prime geodesics in the cover!  Indeed, it acts on the set of
prime geodesics which are lifts of the geodesic down below.
This is the "Frobenius automorphism" associated to the
geodesic.

It's just a guess, but I feel sure it's right, or at least
close.  It's just like the Frobenius automorphisms in number
theory - at least if we realize that a Galois group is secretly
a fundamental group, as explained in &quot<A HREF = "week213">week213</A>".


 



\par\noindent\rule{\textwidth}{0.4pt}
\emph{Wherever there is number, there is beauty.} - Proclus

\par\noindent\rule{\textwidth}{0.4pt}

% </A>
% </A>
% </A>
