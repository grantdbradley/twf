
% </A>
% </A>
% </A>
\week{March 24, 2004 }


The star we know as GRB030329 was named after the day the news of its 
death reached Earth.  About 2,650 million years ago, this star exploded. 
For thirty seconds it put out more power in the form of \gamma  rays 
than everything else in the visible universe combined.  These \gamma  
rays reached us on March 3rd, 2003, and they were detected by a 
satellite called HETE-II: the High-Energy Transient Explorer.  

The detection of GRB03029 set off a frenzy of activity among astronomers
all over the world.  As the closest \gamma -ray burster to be seen 
by well-prepared earthlings, GRB030329 taught us a lot.  We're not 
completely sure what it was - but we have a pretty good guess, and it 
makes a nice story, so I'll recount it as if it were a fact.  

As far as we can tell, GRB030329 was a Wolf-Rayet star before it
exploded.    Wolf-Rayet stars are very rare: only 200 have been seen
in our galaxy.  They're huge and very bright - up to a million times
as bright as the Sun.   They're surrounded by enormous bluish-purple 
nebulae like the one in this picture:

1) NGC 2359, the nebula around the Wolf-Rayet star HD56925,
picture at <A HREF = "http://cfa-www.harvard.edu/cfa/hotimage/n2359.html">
http://cfa-www.harvard.edu/cfa/hotimage/n2359.html</A>

But what makes them really special is that their spectral lines show 
\emph{little or no hydrogen}.  
Since most of the universe is made of hydrogen, 
a star without hydrogen is like a dry fish.  How can this be? 

Well, the life of a star is largely determined by its mass.  Small 
stars last a long time and fade away inconspicuously, while big stars 
live fast and die with a bang.  Wolf-Rayet stars are among the biggest, 
about 60 times as heavy as the Sun.   Like the sun, they begin life as 
cloud of gas that collapses and heats up until the hydrogen in its 
core "catches fire" and starts fusing into helium, like a gigantic 
H-bomb held together by its own gravity.  The core is surrounded by a 
envelope of cooler gas that transmits energy to the surface by 
convection and radiation, but doesn't actually do any fusion itself.

This stage of a star's life is called the "core burning phase".  But 
after a while helium builds up and sinks to the center, forming 
an inert helium core, with all the fusion going on in the layer of 
hydrogen right next to the core.   This is called the "shell burning 
phase".  

What next?  Well, for the Sun, as its hydrogen gradually runs out it'll 
become a "red giant", expanding to engulf the Earth... while meanwhile 
its helium core shrinks to a ball twice the size of the Earth and about 
100 times the density of water, turning from ordinary plasma into 
something called a "degenerate electron gas", where the Pauli exclusion 
principle limits further compression.  As the core shrinks it'll heat up, 
and when it reaches a temperature of 100 million kelvin the helium will 
catch fire and start fusing - mainly into carbon.  Models predict that 
this happens in a runaway reaction called the "helium flash", which puts 
out about 100 billion times the power of the present-day sun for a few 
hours - zounds! - but gradually settles down into a more stable phase 
of helium burning that lasts for tens of millions of years.  During 
this phase, the Sun will not be a red giant anymore, but instead a 
hotter "yellow giant".  

The Sun will never get hot enough to burn elements heavier than helium,
so eventually it'll develop an inert core of carbon and other junk, 
surrounded by a helium burning shell, surrounded by a hydrogen burning
shell.  Then the outer layers will peel off and expand to form a huge 
nebula, leaving the core as a tiny "white dwarf"... which will cool, 
after eons, to a "black dwarf".  Here's a nice chart of the whole story:

2) Sloan Digital Sky Survey, Evolutionary track of a sun-like star,
<A HREF = "http://skyserver.sdss.org/dr1/en/astro/stars/images/starevol.jpg">
http://skyserver.sdss.org/dr1/en/astro/stars/images/starevol.jpg</A>

Bigger stars do more exciting things.  In particular, stars heavier than
about 5 solar masses undergo a "carbon flash" when the carbon-rich 
core reaches 600 million kelvin and starts fusing into heavier elements.
Heavier stars then go on to an oxygen-burning phase.  Even heavier ones 
go on to a silicon-burning phase.  

But when silicon fuses, it forms highly stable nuclei like iron that 
don't want to fuse any further.   So, silicon burning is the end of the 
line.  And it doesn't last long!  For example, a star 25 times the mass
of the Sun is expected to spend about 5 to 10 million years burning 
hydrogen, 0.5 to 1 million years burning helium, 500 to 1000 years 
burning carbon, 6 to 12 months burning oxygen... but just a day or so 
burning silicon! 

Then what?  Well, the details depend on the star's mass.  But when a
star of at least 8 solar masses runs out of fuel, its core is made 
mainly of iron, and heavier than our Sun.  When it cools, it reaches a 
point where all of a sudden it collapses - in about a tenth of a second.   
When it crashes in on itself, it gets so hot that the iron nuclei 
disintegrate and the whole mess explodes in a "type II supernova".  
The star's outer layers get thrown off at high speeds, while the core 
itself gets crushed into a neutron star... or, for truly heavy stars, 
a black hole!

Type II supernovae are among the most violent events in the cosmos.  
They can easily reach a temperature of about 50 billion kelvin and 
emit 10^{46} joules of energy, which is what our galaxy puts out in 
10 years!   99% of this energy is in the form of neutrinos, emitted 
when protons in the iron core absorb electrons and turn into neutrons.
But, the remaining 1% in the form of electromagnetic radiation is
still enough to fry anything in the vicinity.  The supernova in the 
Crab Nebula was about 6,300 light years away, but when its light reached
us in 1054 AD, Chinese astronomers could see it in the daytime for 23 
days!  

You may think I've forgotten about GRB030329 and Wolf-Rayet stars,
but I haven't.  This big digression was just to set the stage.  
I've sketched what stars of up to 25 solar masses will do, but remember,
Wolf-Rayets are a lot bigger: they begin life at about 60 solar masses.
And astronomy resembles opera in this way: the bigger the star, the more 
noise they make in their final scene.  So, the stuff about supernovae 
was just to whet your appetite.

So, let's sit back and watch the thrilling life story of GRB030329... 
assuming that it began its days as most Wolf-Rayets do.  

As a child, it burnt hydrogen in a huge core of about 50 solar 
masses.  After a while helium "ashes" built up in this core, so it 
moved on to burning hydrogen in a shell.  But this process put out 
so much energy that the envelope started getting blown away in a 
powerful stellar wind!   

Since the helium wasn't burning, the core contracted until the 
temperature hit 40 million kelvin and the helium caught fire.  It 
started burning into carbon-12, but some hydrogen got into the core 
and made carbon-13 and nitrogen-14, and later - when the helium was 
almost all burnt - oxygen-16.  

All the while the stellar wind was increasing, and eventually almost
all the hydrogen was blown away, leaving only a bluish-white core 
full of helium, carbon, nitrogen and a little oxygen.  Now you see 
how a star gets rid of its hydrogen!  At this point GRB030329 was a
classic Wolf-Rayet star: almost no hydrogen in the star itself, 
lots of stellar wind, and surrounded by a big nebula of gas and dust 
that had been blown off.  

When all its helium was burnt, our hero's days were numbered.
In an ever-accelerating frenzy, it spent its last centuries burning
carbon, then oxygen, then silicon.  Meanwhile its stellar wind kept 
picking up speed, up to 5 or 10 thousand kilometers per second, blowing 
away more and more gas and dust.  By the time all the silicon had burnt
to iron, the core had shrunk down to about 10 solar masses.  

And when the fuel ran out, the core cooled down and collapsed.  

The core was so big, and its collapse so drastic, that it didn't 
"bounce back" and explode outwards, as in a supernova.  Instead, 
gravity triumphed!  A black hole formed, sucking down a hefty amount 
of the core in less than a tenth of a second.  

As several solar masses of iron rapidly spiralled down the throat of  
this growing black hole, it formed a pancake-like "accretion disk", 
which emitted powerful jets of radiation and matter in both directions
along its axis of rotation.  In a few seconds, these jets passed 
through the outer shell of the star and, together with a blast of 
newly created radioactive nickel-56, shattered it completely.  Our 
star became a "hypernova"!

Meanwhile, the jets plowed into the material surrounding the star
and created highly directional beams of \gamma  rays shooting in opposite
directions... one of which just happened to be pointed directly at
the Earth. 

2,650 million years later, the \gamma  rays reached us, and were
detected on March 23, 2003 by HETE-II.  Hundreds of such bursts are 
detected each year, but this one was closer than most, and a whole 
system had recently been devised for quickly turning the attention 
of the world's telescopes to the spot where a \gamma -ray burster had
been seen - in this case, within the constellation Leo.

So, within 90 min, a 40-inch telescope at the Siding Spring Observatory 
in Australia was looking at this spot.  So was a telescope in Japan. 
They saw the optical afterglow of the \gamma -ray burster and watched
how its brightness changed with time.  And within 24 hours, a spectrograph 
on a telescope in Chile made detailed readings of the spectrum, measuring 
the redshift (z = 0.1685) and thus the distance of the burst, and seeing
signs of radioactive nickel - about 1/3 of a solar mass of the stuff,
according to one estimate!  Later, more telescopes probed the event in
different ways.

The details of what was seen gave a lot of credence to the "hypernova" 
or "collapsar" model of \gamma -ray bursters, championed by Stan Woosley 
of U. C. Santa Cruz, among others.  But much remains mysterious about 
\gamma -ray bursters.  Nobody knows exactly how the energy from the jet 
gets turned into \gamma  rays!  And, the hypernova model only fits "long" 
\gamma -ray bursters, where the burst lasts about 2 seconds or more.  
There are also "short" ones, which may work some other way.

So, the hows and whys of \gamma  ray bursts remain one of the most 
fascinating mysteries in physics.  And since we can't actually peek
inside a star, a lot of the attempts to study these things involve 
complicated mathematical models... very technical stuff, when you 
actually try to read it.  So, I really \emph{am} talking about mathematical 
physics.  Honest!

Here are some ways to learn more, starting with the fun easy stuff.

The online version of the Messier Catalog - a famous old catalog of
galaxies and nebulae - is a really fun way to learn some astronomy:

3) The Messier Catalog, <A HREF = "http://www.maa.agleia.de/Messier/">
http://www.maa.agleia.de/Messier/</A>

It's packed with interesting stuff.  For example, here's a great 
page about nebulae like the one the sun will form after it becomes 
a yellow giant:  

4) The Messier Catalog, Planetary nebulae, 
<A HREF = "http://www.maa.agleia.de/Messier/planetar.html">
http://www.maa.agleia.de/Messier/planetar.html</A>

(They're misleadingly called "planetary nebulae", though they
don't have anything to do with planets.)  And here's a nice page
about the Crab Nebula, which is now a pulsar - a rapidly spinning
neutron star left over from a supernova:

5) The Messier Catalog, The crab nebula (M1), 
<A HREF = "http://www.maa.agleia.de/Messier/E/m001.html">
http://www.maa.agleia.de/Messier/E/m001.html</A>

This website is good for Wolf-Rayet stars and other things:

6) Chris Clowes' Astronomy Page,
<A HREF = "http://www.peripatus.gen.nz/Astronomy/">http://www.peripatus.gen.nz/Astronomy/</A>

I learned a lot about GRB030329 from this page:

7) European Southern Observatory (ESO), Cosmological \gamma -ray 
bursts and hypernovae conclusively linked, June 18, 2003,
<A HREF = "http://www.eso.org/outreach/press-rel/pr-2003/pr-16-03.html">http://www.eso.org/outreach/press-rel/pr-2003/pr-16-03.html</A>

For another key moment in the history of \gamma -ray bursters, 
try this:

8) Burst and Transient Source Experiment (BATSE), 
GOTCHA! - The big one that didn't get away, January 27, 1999,
<A HREF = "http://www.batse.com/jan27.html">http://www.batse.com/jan27.html</A>

For more on \gamma -ray bursters, try these:

9) NASA, \gamma  ray bursts, 
<A HREF = "http://imagine.gsfc.nasa.gov/docs/introduction/bursts.html">
http://imagine.gsfc.nasa.gov/docs/introduction/bursts.html</A>

10) Edo Berger, Gamma-ray burst FAQ, 
<A HREF = "http://www.astro.caltech.edu/~ejb/faq.html">http://www.astro.caltech.edu/~ejb/faq.html</A>

If you get more serious, there are lots of conference proceedings
to read, like this:

11) M. Livio, N. Panagia and K. Sahu, editors, Supernovae and Gamma-Ray
Bursts: The Greatest Explosions since the Big Bang, Cambridge U. Press,
2001.

There must be a bunch of conference proceedings written after the March 
2003 burster, but maybe they haven't been published yet, since I haven't 
been able to find them!

If you're looking for a more general background in astrophysics, this 
hefty tome is supposed to be a good intro, though I haven't tried it 
yet:

12) Bradley W. Carroll and Dale A. Ostlie, Introduction to Modern 
Astrophysics, Addison Wesley, 1996.

Personally I've found these helpful in writing the above stuff, though 
they're full of equations, so I find myself yearning for some purple 
prose here and there:

13) R. J. Tayler, The Stars: Their Structure and Evolution, 
2nd edition, Cambridge U. Press, Cambridge, 1994.

14) R. Kippenhahn and A. Weigert, Stellar Structure and Evolution,
Springer Verlag, Berlin, 1991.

By the way, I don't know much about astrophysics, so I'd love to hear
from any experts out there who'd like to correct or add detail to my
description of Wolf-Rayet stars, the "hypernova" scenario, or \gamma -
ray bursters in general.  I've been fond of Wolf-Rayet stars ever
since I wrote a few little articles on weird kinds of stars:

15) John Baez, Stuff about Stars, 
<A HREF = "http://math.ucr.edu/home/baez/stars.html">http://math.ucr.edu/home/baez/stars.html</A>

back before anyone suspected they were related to \gamma -ray bursters!
My interest in them was rekindled while revising the physics FAQ on 
open questions in physics:

16) John Baez, Open Questions in Physics, 
<A HREF = "http://math.ucr.edu/home/baez/open.questions.html">http://math.ucr.edu/home/baez/open.questions.html</A>

It hadn't been rewritten since 1997, and it was interesting to see
how outdated it had become, particularly in the area of cosmology
and astrophysics!  Here's the current list of problems:

\begin{quote}
<H4> Condensed Matter and Nonlinear Dynamics</H4>

 1) What causes sonoluminescence?

 2) What causes high temperature superconductivity?
  
 3) How can turbulence be understood and its effects calculated?

 4) The Navier-Stokes equations are the basic equations describing 
    fluid flow.  Does these equations have solutions that last for 
    all time, given arbitrary sufficiently nice initial data?

<H4> Quantum Mechanics</H4>

 1) How should we think about quantum mechanics?

 2) Can we build a working quantum computer big enough to do things
    ordinary computers can't easily do?

<H4> Cosmology and Astrophysics</H4>

 1) What happened at or before the Big Bang?

 2) Are there really three dimensions of space and one of time?
    If so, why?  Or is spacetime higher-dimensional, or perhaps 
    not really a manifold at all when examined on a short enough 
    distance scale?

 3) Why is there an arrow of time; that is, why is the future so 
    much different from the past?

 4) Is the Universe infinite in spatial extent?  More generally:
    what is the topology of space?

 5) Will the future of the Universe go on forever or not?

 6) Is the universe really full of "dark energy"?  If so, what 
    causes it?

 7) Why does it seem like the gravitational mass of galaxies exceeds
    the mass of all the stuff we can see, even taking into account 
    our best bets about invisible stuff like brown dwarfs, "Jupiters", 
    and so on?

 8) The Horizon Problem: why is the Universe almost, but not quite,
    homogeneous on the very largest distance scales? 

 9) Why are the galaxies distributed in clumps and filaments?

 10) When were the first stars formed, and what were they like?

 11) What are \gamma  ray bursters?

 12) What is the origin and nature of ultra-high-energy cosmic rays?

 13) Do gravitational waves really exist?  If so, can we detect them?  
     If so, what will they teach us about the universe?  Will they 
     mainly come from expected sources, or will they surprise us?

 14) Do black holes really exist?  (It sure seems like it.)
     Do they really radiate energy and evaporate the way Hawking 
     predicts?  If so, what happens when, after a finite amount of 
     time, they radiate completely away?  What's left?  Do black holes 
     really violate all conservation laws except conservation of 
     energy, momentum, angular momentum and electric charge?  What 
     happens to the information contained in an object that falls 
     into a black hole?

 15) Is the Cosmic Censorship Hypothesis true?  Roughly, for generic 
     collapsing isolated gravitational systems are the singularities 
     that might develop guaranteed to be hidden beyond a smooth event 
     horizon?

<H4> Particle Physicss</H4>

 1) Why are the laws of physics not symmetrical between left and 
    right, future and past, and between matter and antimatter?

 2) Why is there more matter than antimatter, at least around here?

 3) Are there really just three generations of leptons and quarks? 
    If so, why?

 4) Why does each generation of particles have precisely this 
    structure: two leptons and two quarks?   

 5) Do the quarks or leptons have any substructure, or are they
    truly elementary particles?

 6) Is there really a Higgs boson, as predicted by the Standard Model 
    of particle physics?   If so, what is its mass? 

 7) What is the correct theory of neutrinos?  Why are they almost but
    not quite massless?  Do all three known neutrinos - electron, muon,
    and \tau  - all have a mass? 

 8) Is quantum chromodynamics (QCD) a precise description of the 
    behavior of quarks and gluons?  Can we prove that quarks are 
    gluons are confined at low temperatures using QCD?  Is it possible 
    to calculate masses of hadrons (such as the proton, neutron, pion, 
    etc.) correctly from the Standard Model, with the help of QCD? 
    Does QCD predict that quarks and gluons become deconfined and form
    plasma at high temperature?  If so, what is the nature of the 
    deconfinement phase transition?
  
 9) Is there a mathematically rigorous formulation of a relativistic 
    quantum field theory describing interacting (not free) fields in 
    four spacetime dimensions?  For example, is the Standard Model 
    mathematically consistent?  How about Quantum Electrodynamics?

 10) Is the proton really stable, or does it eventually decay?

 11) Why do the particles have the precise masses they do?
 
 12) Why are the strengths of the fundamental forces (electromagnetism, 
     weak and strong forces, and gravity) what they are?

 13) Are there important aspects of the Universe that can only be 
     understood using the Anthropic Principle?  Or is this principle 
     unnecessary, or perhaps inherently unscientific?

 14) Do the forces really become unified at sufficiently high energy?

 15) Does some version of string theory or M-theory give specific
     predictions about the behavior of elementary particles?   If 
     so, what are these predictions?  Can we test these predictions 
     in the near future?  And: are they correct?

 <H4>The Big Question</H4>

 1) How can we merge quantum theory and general relativity to create 
    a quantum theory of gravity?  How can we test this theory?

\end{quote}

A bunch of these questions could turn out to be a bit silly - a good
answer might require changing the question.  But that's always how it
goes for really interesting puzzles.  I should also warn you that the
statements above are deliberately a bit naive-sounding: as the example 
of \gamma  ray bursters shows, we actually do know a lot about all these 
questions - we're just not sure about the answers!  So, see the webpage 
itself for a bit more information on these questions, and the links for 
even more...

Hmm.  I was going to say something about number theory, but I'm out of 
time!

\par\noindent\rule{\textwidth}{0.4pt}
<em>Of course: abstraction, irrelevance, purity, formalism make for 
good mathematics....  But sadly, they make for bad mathematics education.
Each one of these concepts - abstract, irrelevance, purity, formalism -
pushes mathematics further away from a growing human being, a being
whose psyche is in the phase of it development that no soft-brained
psychologist but a great mathematician, Alfred North Whitehead, calls
the Romantic Phase.</em> - <A HREF = "http://www.apostolosdoxiadis.com/files/essays/embeddingmath.pdf">Apostolos Doxiadis</A>

\par\noindent\rule{\textwidth}{0.4pt}

% </A>
% </A>
% </A>
