
% </A>
% </A>
% </A>
\week{ October 17, 2005 }


Last week there was a big conference on quantum gravity at
the Albert Einstein Institute near Berlin:

1) Loops '05, 
<A HREF = "http://loops05.aei.mpg.de">http://loops05.aei.mpg.de</A>

The focus was loop quantum gravity and spin foams, but there 
were also talks about other approaches, so it was much bigger than
last year's get-together in Marseille.  Last year about 100 people
attended; this time about 160 did!  It was strange seeing old pals 
like Ashtekar, Lewandowski, Loll, Rovelli and Smolin almost lost 
in a sea of new faces.  But, it was great to talk to everyone, 
both old and new.

I'll say more about this conference, but first let's talk about 
\gamma  ray bursters, a black hole without a host galaxy, the newly 
discovered moon of planet Xena, and lots of other transneptunian 
objects.

Actually, just for fun, let's start with this science fiction novel
I picked up in Heathrow en route to Berlin:

2) Charles Stross, Accelerando, Ace Books, New York.
Also available at <A HREF = "http://www.accelerando.org/book/">
http://www.accelerando.org/book/</A>

This is one of the few tales I've read that does a good job of 
fleshing out Verner Vinge's "Singularity" scenario, where the 
accelerating development of technology soars past human 
comprehension and undergoes a phase transition to a thoroughly 
different world.   This is a real possibility, and it's been 
discussed a lot:

3) Wikipedia, Technological singularity, 
<A HREF = "http://en.wikipedia.org/wiki/Technological_singularity">
http://en.wikipedia.org/wiki/Technological_singularity</A>

Ray Kurzweil, The Singularity,
<A HREF = "http://www.kurzweilai.net/meme/frame.html?m=1">http://www.kurzweilai.net/meme/frame.html?m=1</A>

Anders Sandberg, The Singularity, 
<A HREF = "http://www.aleph.se/Trans/Global/Singularity/">
http://www.aleph.se/Trans/Global/Singularity/</A>

However, it's not an easy subject for fiction - at least not for 
mere human readers!  Stross makes it gripping: sometimes goofy, sometimes
thrilling, and sometimes rather sad.  Characters include a robot cat
with ever-growing powers and some space-faring uploaded lobsters.  

The hero, Manfred Macx, starts out as a freeware developer, futurist 
and all-purpose wheeler-dealer.  Here's a scene from the beginning
of the book, before all hell breaks loose:

\begin{quote}
     Manfred's mood of dynamic optimism is gone, broken by the 
     knowledge that his vivisectionist stalker has followed him 
     to Amsterdam  to say nothing of Pamela, his dominatrix, 
     source of so much yearning and so many morning-after weals. 
     He slips his glasses on, takes the universe off hold, and 
     tells it to take him for a long walk while he catches up on 
     the latest on the tensor-mode gravitational waves in the 
     cosmic background radiation (which, it is theorized, may be 
     waste heat generated by irreversible computational processes 
     back during the inflationary epoch; the present-day universe 
     being merely the data left behind by a really huge calculation). 
     And then there's the weirdness beyond M31: according to the 
     more conservative cosmologists, an alien superpower - maybe 
     a collective of Kardashev Type Three galaxy-spanning 
     civilizations - is running a timing channel attack on the 
     computational ultrastructure of space-time itself, trying to 
     break through to whatever's underneath.  The tofu-Alzheimer's 
     link can wait.
\end{quote}
    

An idea a minute - and the book is free online: what more could you 
want?  

But right now, the big news in astronomy is \emph{not} about a type III 
civilization lurking beyond M31 (otherwise known as the Andromeda
Galaxy).  It's some evidence that short \gamma  ray bursts are caused 
by collisions involving neutron stars and black holes!

\Gamma  ray bursts are among the most energetic events known in the
heavens.   They happen in galaxies throughout the universe; we see 
about one a day, and each releases somewhere between 10^{45} and 
10^{47} joules of energy.  The larger figure is what you'd get by 
turning the entire mass of the Sun into energy.  

There could be several kinds of \gamma  ray bursts, but there seem to
be at least two: short and long.   Short bursts last between 
40 milliseconds and 10 seconds - imagine the whole Sun turning into 
energy that fast!  Long ones last between 10 and 100 seconds.  The
two kinds seem to be qualitatively different: for example, the short 
ones consist of higher-frequency \gamma  rays.  The big news is that 
they happen in different kinds of galaxies!

In "<A HREF = "week204.html">week204</A>", I described how 
people caught a long \gamma  ray burst 
in the act in March 2003.  A \gamma  ray detector aboard a satellite 
relayed information to telescopes in Australia and Japan, allowing 
them to spot a visible afterglow right after the burst.  The details 
of this glow fit the "hypernova" theory of long \gamma  ray bursts.

The hypernova theory says that when a star more than 25 times heavier
than the Sun runs out of fuel and collapses, it forms a black hole 
that sucks down the star's iron core before a normal supernova 
explosion can occur.  In just a few seconds, about a solar mass of 
iron spirals into the black hole, forming a pancake-shaped disk as 
it goes down.  In the process, this disk becomes incredibly hot and 
shoots out jets of radiation in the transverse directions.  As they 
plow through the star's outer layers, these jets create beams of 
\gamma  rays.  

The short bursts have been harder to catch.  By the time a telescope 
on Earth could be aimed at the spot where the \gamma  rays were seen, 
no afterglow could be seen!

So, in October 2004 NASA launched Swift: a \gamma -ray detecting 
satellite equipped with an X-ray telescope and an ultraviolet/optical 
telescope that can respond quickly whenever a burst is
seen:  

4) Official NASA Swift homepage,
<A HREF = "http://swift.gsfc.nasa.gov/docs/swift/swiftsc.html">
http://swift.gsfc.nasa.gov/docs/swift/swiftsc.html</A>

5) Gamma-ray burst real-time sky map, <A HREF = "http://grb.sonoma.edu/">
http://grb.sonoma.edu/</A>

On May 9th, 2005, Swift detected a short burst and caught 11 photons 
of the burst's X-ray afterglow.  Another short burst detected by 
HETE-II had its X-ray afterglow caught by the Chandra X-ray satellite.  
Analysis of these and two more short bursts has convinced some 
scientists that they're caused by collisions between neutron stars 
and/or black holes:

6) D. B. Fox et al, The afterglow of GRB050709 and the nature of the
short-hard \gamma -ray bursts, Nature 437 (October 2005), 849-850.
Also available at <A HREF = "http://www.nasa.gov/pdf/135397main_nature_fox_final.pdf">http://www.nasa.gov/pdf/135397main_nature_fox_final.pdf</A>

Despite what the news media are saying, I don't see that this paper 
"proves" the short \gamma -ray bursts are caused by such collisions.  
Instead, I see some good pieces of evidence.  

The faintness of the afterglows suggests some mechanism other than a 
hypernova.  But as far as I can tell, the best evidence is that short
\gamma  ray bursts tend to happen near the edges of old galaxies, while 
the long ones happen near the centers of young galaxies.  

The center of a young galaxy is where you'd expect to find a really 
huge Wolf-Rayet star, the sort that dies in a hypernova.  The edge of 
an old galaxy is where you'd expect to see black holes and neutron 
stars collide.  Why?  Because such collisions can only happen long 
after stars are first formed.  First you need an orbiting pair of 
giant stars to go supernova and collapse into neutron stars and/or 
black holes.  Then you need plenty more time for this pair to spiral 
down thanks to gravitational radiation, and eventually collide.
By then the pair may sail off to the edge of the galaxy, thanks to 
the "kick" delivered by the supernova explosions. 

I hope astronomers can clinch the case for the collision theory of
short \gamma  ray bursts.  After all, these collisions involving 
neutron stars and black holes are precisely what gravitational wave 
detectors like LIGO and VIRGO are hoping to see!  If we know to look 
for gravitational waves precisely when we see short \gamma  ray bursts, 
and we know where they're coming from, we'll have a better chance of 
finding them.

(Of course, we'll also have a better chance of \emph{fooling} ourselves
into \emph{thinking} we found them, until we do some double-blind tests.)

By the way, LIGO is already analysing data to look for gravitational
waves.  I talked about this in 
"<A HREF = "week189.html">week189</A>", but here's something new:
now you can help them by running a cool screensaver called 
Einstein@Home on your computer!  Check it out:

7) Einstein@Home, <A HREF = "http://einstein.phys.uwm.edu/">
http://einstein.phys.uwm.edu/</A>

Speaking of black holes, last month the Hubble Space Telescope and
the Very Large Telescope in Chile detected a quasar that seems
to have no host galaxy:

8) European Southern Observatory, Black hole in search of a home,
<A HREF = "http://www.eso.org/outreach/press-rel/pr-2005/pr-23-05.html">
http://www.eso.org/outreach/press-rel/pr-2005/pr-23-05.html</A>

HubbleSite, Quasar without host galaxy compared with normal quasar,
<A HREF = "http://hubblesite.org/newscenter/newsdesk/archive/releases/2005/13/image/a">http://hubblesite.org/newscenter/newsdesk/archive/releases/2005/13/image/a</A>

Quasars are thought to be super massive black holes; they're usually 
found in the centers of galaxies, where they devour stars and shoot 
out enormously powerful jets of radiation.  However, the quasar 
HE0450-2958 is surrounded only by a blob of ionized gas.  Nearby, a
wildly disturbed spiral galaxy can be seen.  

Compare HE0450-2958 (at left) with a normal quasar (at right):

<DIV ALIGN = center>
<A HREF = "http://hubblesite.org/newscenter/newsdesk/archive/releases/2005/13/image/a">
<IMG SRC = 
"HE0450-2958.jpg">
% </A>
</DIV>

The quasar HE0450-2958 is in the middle of the left-hand picture;
the disturbed galaxy is above and a completely irrelevant foreground
star is below.  For more details on what this image means, click on it.

Did this quasar begin 
life in the middle of a galaxy and then get kicked out when that 
galaxy collided 
with something containing a super-massive black hole?  What could that
something be?

Puzzles, puzzles, in the sky....

Closer to home, astronomers at the Keck Observatory in Hawaii have 
discovered that planet Xena has a moon! 

<DIV ALIGN = center>

<A HREF =" http://www2.keck.hawaii.edu/optics/staff/mvandam/gabrielle">
<IMG HEIGHT = 300 WIDTH = 300 SRC ="xena.jpg">
% </A>
</DIV>

They nicknamed it Gabrielle, 
after this famous TV character's sidekick:

9) Michael E. Brown, The moon of the 10th planet, 
<A HREF = "http://www.gps.caltech.edu/~mbrown/planetlila/moon/index.html">
http://www.gps.caltech.edu/~mbrown/planetlila/moon/index.html</A>

If you hadn't heard about planet Xena, or you don't like the idea 
of naming a planet after a TV character - even a "warrior princess" - 
don't get worked up just yet.  Xena's official name is currently 
2003 UB_{313}, and though she's larger than Pluto, the International 
Astronomical Union has not decided whether she'll officially be 
considered a planet.  

If Xena becomes a planet, she'll probably be renamed Persephone, 
after the reluctant queen of the underworld in Greek mythology.  
But, she may have to settle for the status of a mere "transneptunian 
object", like Quaoar and Sedna.  Indeed, if Pluto had been discovered
more recently, folks probably wouldn't have called him a planet 
either.

If you haven't even heard of Quaoar and Sedna... well, you must be
too absorbed by mundane concerns to keep track of the burgeoning
population of our Solar System.  But it's not too late to mend your
ways!  Impress your friends by casually dropping some of this jargon:

<UL>
<LI>
  \textbf{<A HREF = "http://en.wikipedia.org/wiki/Trans-Neptunian_object">Transneptunian object</A>} - any object that orbits the Sun at an average 
  distance greater than that of Neptune.  Neptune is about 30 AU from 
  the Sun, meaning it's 30 times farther from the Sun than we are.  
  Transneptunian objects can be roughly divided into three kinds: 
  Kuiper belt objects, scattered disc objects, and Oort cloud objects.

<DIV ALIGN = CENTER>
<A HREF = "http://en.wikipedia.org/wiki/Kuiper_belt"> 
<IMG HEIGHT = 400 WIDTH = 500 SRC = "kuiper_oort.jpg">
% </A>
</DIV>


<UL>
<LI>
   <b><A HREF = "http://en.wikipedia.org/wiki/Kuiper_belt">Kuiper 
   belt object</A></b> - 
   any object whose orbit lies in the Kuiper belt. 
   This is the region in the ecliptic (the plane of the planets' orbits) 
   between 30 and 50 AU from the Sun.  There are a bunch of planetoids
   in this belt.  Beyond 50 AU there seems to be a sharp dropoff in 
   their density.  Three main kinds of Kuiper belt objects have been 
   found so far: cubewanos, plutinos and twotinos.



<UL>
<LI>
    \textbf{<A HREF = "http://en.wikipedia.org/wiki/Cubewano">Cubewano</A>} - 
    A cubewano is a Kuiper belt object whose orbit is not
    in resonance with any of the outer planets.  The curious name
    comes from "QB1", since the first example was named 
    <A HREF = "http://en.wikipedia.org/wiki/%2815760%29_1992_QB1">1992 
    QB_{1}</A>.
 
    One of the biggest cubewanos is 
    <A HREF = "http://en.wikipedia.org/wiki/Quaoar">Quaoar</A>, with a 
    diameter of about 
    1200 kilometers.  This is about half the diameter of Pluto, or a 
    third the size of the Moon: much bigger than anything in the 
    asteroid belt!  


<DIV ALIGN = CENTER>
<A HREF = "http://en.wikipedia.org/wiki/90377_Sedna"> 
<IMG HEIGHT = 400 WIDTH = 500 SRC = "sedna.jpg">
% </A>
</DIV>

    Folks believe Quaoar is a mixture of ice and rock.  
    It's very dark in color, but last year crystalline water ice was 
    detected on its surface, using infrared spectroscopy.   This came 
    as a surprise, because cosmic rays and solar wind should convert 
    exposed ice crystals to the amorphous form of ice within about 
    10 million years.  Could there have been liquid water volcanos 
    active on Quaoar this recently??  Or maybe meteor impacts melt 
    amorphous ice and then it crystallizes?

    Other big cubewanos include 
    <A HREF = "http://en.wikipedia.org/wiki/%2819521%29_Chaos">Chaos</A>, 
    <A HREF = "http://en.wikipedia.org/wiki/20000_Varuna">Varuna</A>, 
   and 
   <A HREF = "http://en.wikipedia.org/wiki/53311_Deucalion">Deucalion</A>.  
    2003 EL61 and 2005 FY9 are even bigger, but they haven't got nice names 
    yet.

<LI>
    \textbf{<A HREF = "http://en.wikipedia.org/wiki/Plutino">Plutino</A>} - A plutino is a Kuiper belt object whose orbit is in 
    3:2 resonance with Neptune: they go around the Sun twice while Neptune
    goes around three times.  About a quarter of Kuiper belt objects 
    are plutinos.  

    The most famous plutino is 
    <A HREF = "http://en.wikipedia.org/wiki/Pluto">Pluto</A> itself, 
    though some pedants argue 
    that Pluto can't be a "little Pluto".  Pluto is quite different 
    than anything else we call a planet: it has an eccentric orbit that 
    ranges between 30 and 50 AU, and its orbit is tilted 17 degrees to
    the ecliptic.  Its surface is light brown, consisting mainly of 
    frozen nitrogen and carbon monoxide.  When it comes near the sun, 
    as it recently did, it also gets a thin atmosphere made of these 
    gases.

    Other plutinos include 
<A HREF = "http://en.wikipedia.org/wiki/28978_Ixion">Ixion</A>, 
<A HREF = "http://en.wikipedia.org/wiki/90482_Orcus">Orcus</A>, 
<A HREF = "http://en.wikipedia.org/wiki/38083_Rhadamanthus">Rhadamanthus</A>, 
and Pluto's moon 
<A HREF = "http://en.wikipedia.org/wiki/Charon_%28moon%29">Charon</A>.  
If you know Greek mythology, you'll know these guys are all 
    named after deities of the underworld.

<LI>
   \textbf{<A HREF = "http://en.wikipedia.org/wiki/Twotino">Twotino</A>} - 
    A twotino is a Kuiper belt object whose orbit is in 2:1
    resonance with Neptune.  These are rare compared to plutinos, and
    they're smaller, so they're stuck with boring names like 1996 TR66.
    There are also a couple of Kuiper belt objects in 4:3 and 5:3
    resonances with Neptune.

</UL>
<LI>
   <b><A HREF = "http://en.wikipedia.org/wiki/Scattered_disc">Scattered disc 
   object</A></b> - A scattered disc object is a Kuiper belt
   object that has been perturbed by interactions with Neptune into
   an orbit that is more eccentric or more tilted from the ecliptic. 

   <A HREF = "http://en.wikipedia.org/wiki/2003_UB313">Xena</A> 
   (or more properly 2003 UB_{313}) 
   is a highly eccentric scattered 
   disc object whose orbit carries it between 40 to 100 AU from the sun. 
   Its orbit is inclined a whopping 44 degrees, and it's locked in a
   complicated 17:5 resonance with Neptune.  It's probably larger than 
   Pluto - a reasonable rough guess is 2900 kilometers in diameter, as 
   compared with 2400 for Pluto.  Its surface has methane ice, and we 
   now know it has a moon.
 
   It's quite possible that scattered disc objects are related to 
   <A HREF = "http://en.wikipedia.org/wiki/Centaur_%28planetoid%29">centaurs</A>,   
   which are planetoids orbiting the Sun between Jupiter
   and Neptune.  The centaurs may be Kuiper belt objects that got
   knocked towards the Sun instead of away from it!  Centaurs have
   chaotic orbits and will probably either collide with something or
   be ejected from the Solar System.

<LI>
  \textbf{<A HREF = "http://en.wikipedia.org/wiki/Oort_cloud">Oort cloud object</A>} - the Oort cloud is a hypothesized spherical cloud
   of comets, perhaps between 50,000 and 100,000 AU from the Sun.  The idea 
   is this cloud consists of leftovers from the original nebula that
   collapsed to form our Solar system, and comets come from this region 
   when they are perturbed from their orbits by the gravity of other 
   stars.  
 
   Nobody has seen a certified Oort cloud object.  The best candidate
   so far is <A HREF = "http://en.wikipedia.org/wiki/90377_Sedna">Sedna</A>, 
   an object roughly 1500 kilometers in diameter with
   a wildly eccentric orbit taking it between 80 to 930 AU from the Sun.  

<DIV ALIGN = CENTER>
<A HREF = "http://en.wikipedia.org/wiki/90377_Sedna">
<IMG HEIGHT = 600 WIDTH = 600 SRC = "sedna_orbit.jpg">
% </A>
</DIV>

   Sedna was discovered in 2004 when it was 90 AU from the Sun. 
   It's redder than Mars, its temperature never rises above 23 Kelvin,
   and its year lasts 11,250 years.  It's
   the farthest known object in our Solar System, but still much closer
   than the Oort cloud was supposed to be.  Maybe it's a drastic example
   of a scattered disc object, maybe it's part of an "inner Oort 
   cloud"... or maybe the Oort cloud isn't as far out as people thought.

   The closest people have come to seeing the Oort cloud is seeing
   a "Bok globule":  

<DIV ALIGN = CENTER>
<A HREF = "http://www.eso.org/outreach/press-rel/pr-2001/pr-01-01.html">
<IMG HEIGHT = 400 WIDTH = 370 SRC = "bok_globule.jpg">
% </A>
</DIV>
    A Bok globule is a cloud of dust and gas that's collapsing to
    form a star.  This one is about 12,500 AU across.  The scientists
    who observed it say it's about the size of the Oort cloud.  This
    just goes to show how little we know about the Oort cloud!
   
</UL>
</UL>

For a great introduction to the Kuiper belt and related topics, try
this:

10) David C. Jewitt, Kuiper belt, 
<A HREF = "http://www.ifa.hawaii.edu/faculty/jewitt/kb.html">
http://www.ifa.hawaii.edu/faculty/jewitt/kb.html</A>

For transneptunian objects in general, try:
 
11) William Robert Johnston, Transneptunian objects,
<A HREF = "http://www.johnstonsarchive.net/astro/tnos.html">
http://www.johnstonsarchive.net/astro/tnos.html</A>

Also check out this newsletter:

12) Distant EKOs: the Kuiper Belt Electronic Newsletter,
<A HREF = "http://www.boulder.swri.edu/ekonews/">
http://www.boulder.swri.edu/ekonews/</A>

Quaoar was discovered in 2002 by Chad Trujillo and Mike Brown of 
Caltech:

13) Chad Trujillo, Quaoar, 
<A HREF = "http://www.gps.caltech.edu/~chad/quaoar/">
http://www.gps.caltech.edu/~chad/quaoar/</A>

For evidence of crystalline water ice on Quaoar, see:

14) David C. Jewitt and Jane Luu, Crystalline water ice on the Kuiper
belt object (50000) Quaoar, Nature 432 (2004), 731-733.  
Also available at
<A HREF = "http://www.ifa.hawaii.edu/faculty/jewitt/quaoar.html">
http://www.ifa.hawaii.edu/faculty/jewitt/quaoar.html</A>

Xena was discovered in 2003 by Trujillo, Brown and a colleague of
theirs at Yale University:

15) Michael E. Brown, Chad A. Trujillo and David L. Rabinowitz,
Discovery of a planetary-sized object in the scattered Kuiper belt, 
submitted to ApJ Letters, available at
<A HREF = "http://www.gps.caltech.edu/%7Embrown/papers/ps/xena.pdf">
http://www.gps.caltech.edu/%7Embrown/papers/ps/xena.pdf</A>

Brown has a nice webpage about Xena and Gabrielle:

16) Michael E. Brown, The discovery of UB313, the 10th planet,
<A HREF = "http://www.gps.caltech.edu/~mbrown/planetlila/">
http://www.gps.caltech.edu/~mbrown/planetlila/</A>

The same gang of three also discovered Sedna in 2003:

17) Michael E. Brown, Chad A. Trujillo and David L. Rabinowitz,
Discovery of a candidate inner Oort cloud planetoid, to appear
in ApJ Letters, available at
<A HREF = "http://www.gps.caltech.edu/~mbrown/papers/ps/sedna.pdf">
http://www.gps.caltech.edu/%7Embrown/papers/ps/sedna.pdf</A>

... and Brown has a fun Sedna webpage too:

18) Michael E. Brown, Sedna (2003 VB12), 
<A HREF = "http://www.gps.caltech.edu/~mbrown/sedna/">
http://www.gps.caltech.edu/~mbrown/sedna/</A>

How all these transneptunian objects got where they are is a wonderful
puzzle in celestial mechanics, but you can read more about that in
the references above, especially Jewitt's Kuiper belt webpage.

Now I want to talk about Loops '05!

Instead of trying to review all the talks - a hopeless task, since
there were 86 - I'll just mention
the \emph{two} strands of work I find most exciting.

First, there's new evidence that a quantum theory of pure gravity
(meaning gravity without matter) makes sense in 4-dimensional spacetime.  

To understand why this is exciting, you have to realize that in some 
quarters, the conventional wisdom says a quantum theory of pure gravity
can't possibly make sense, except as a crude approximation at large
distance scales, because this theory is "perturbatively 
nonrenormalizable".

Very roughly, this means that as we zoom in and look at the theory at
shorter and shorter distance scales, it looks less and less like a 
"free field theory" where gravitons zip about without interacting.   
Instead, the interactions get stronger and more complicated!  

So, in the jargon of the trade, we don't get a "Gaussian ultraviolet 
fixed point".   

Huh?

Well, roughly, an "ultraviolet fixed point" is a quantum field theory
that keeps looking the same as you keep viewing it on shorter and
shorter distance scales.  A "Gaussian" ultraviolet fixed point is one
that's also a free quantum field theory: one where particles don't 
interact.  

If quantum gravity approached a Gaussian ultraviolet fixed point as 
we zoomed in, we could calculate what gravitons do at arbitrarily 
high energies (at least perturbatively, as power series in Newton's 
constant - no guarantee that these series converge).  Particle 
physicists would then be happy and say the theory was "perturbatively 
renormalizable".  

But, it's not.

The conventional wisdom concludes that to save quantum gravity, we 
must include matter of precisely the right sort to \emph{make} it 
perturbatively renormalizable.  This is the quest that led people 
first to supergravity and ultimately to superstring theory - see
"<A HREF = "week195.html">week195</A>" for more of this story.

But, as far back as 1979, the particle physicist Weinberg raised the 
possibility that pure quantum gravity is "nonperturbatively 
renormalizable", or "asymptotically safe".  This means that 
as we 
zoom in and look at the theory at shorter and shorter distance scales, 
it approaches some theory \emph{other than} that of noninteracting gravitons.

In other words, Weinberg was suggesting that pure quantum gravity 
approaches a non-obvious ultraviolet fixed point - possibly a 
"non-Gaussian" one.  

The big news is that this seems to be true!  

Even cooler, in this theory spacetime seems to act \emph{2-dimensional}
at very short distance scales.  

This idea has been brewing for a long time - I talked about it 
extensively back in "<A HREF = "week139.html">week139</A>".  
But now there's more solid 
evidence for it, coming from two quite different approaches.

First, people doing numerical quantum gravity in the "causal dynamical 
triangulations" approach are seeing this effect in their computer 
calculations.  This is what Renate Loll explained at Loops '05.  The 
best place to read the details is here:

19) Jan Ambj&oslash;rn, J. Jurkiewicz and Renate Loll, Reconstructing 
the universe, Phys. Rev. D72 (2005) 064014.  Also available as
<A HREF = "http://xxx.lanl.gov/abs/hep-th/0505154">hep-th/0505154</A>.

but if you need something less technical, try this:

20) Jan Ambj&oslash;rn, J. Jurkiewicz and Renate Loll, The universe from
scratch, available as 
<A HREF = "http://xxx.lanl.gov/abs/hep-th/0509010">hep-th/0509010</A>.

The titles of their papers are a bit grandiose, but their calculations 
are solid stuff - truly magnificent.  I described their basic strategy
in my report on the Marseille conference in 
<A HREF = "week206.html">week206</A>.  So, I won't explain that again.
I'll just mention their big new result: in pure 
quantum gravity, spacetime has a spectral dimension of 4.02 
&plusmn; 0.1 on large distance scales, but 1.80 &plusmn; 0.25 in the 
limit of very short distance scales! 

Zounds!  What does that mean?
The "spectral dimension" of a spacetime is the dimension as measured 
by watching heat spread out on this spacetime: the short-time behavior of 
the heat equation probes the spacetime at short distance scales, while its 
large-time behavior probes large distance scales.  Spectral dimensions
don't need to be integers - for fractals they're typically not.  But,
Loll and company believe they're seeing spacetimes that are \emph{exactly}
2-dimensional in the limit of very small distance scales, 
\emph{exactly} 4-dimensional in the limit of very large scales, 
with a continuous change in dimension in between.   The error bars in 
the above figures come from doing Monte Carlo simulations.  They're 
just using ordinary computers, not supercomputers.  So, with 
more work one could shrink their error bars and test their result.

My main worry about their work is that it uses a fixed slicing of
spactime by timelike slices.  So, there's a danger that their 
procedure breaks Lorentz-invariance, even in the continuum limit 
which they are attempting to compute.   I would like to find a way
around this problem!

Luckily, some other people are getting similar results from a second
procedure that definitely does \emph{not} break Lorentz invariance:

21) Oliver Lauscher and Martin Reuter, Fractal spacetime structure in 
asymptotically safe gravity, available as <A HREF = "http://xxx.lanl.gov/abs/hep-th/0508202">hep-th/0508202</A>.

Reuter spoke about all this work at Loops '05.  The idea is to 
investigate Weinberg's original idea in excruciatingly precise
detail using "renormalization group flow" ideas.  The above paper 
is a review of lots of others, and you need to read a bunch to get 
what's really going on.  The upshot, however, is that they find 
evidence for a non-Gaussian ultraviolet fixed point in pure quantum
gravity.  Moreover, the spectral dimension of spacetime approaches
2 in the limit of very short distance scales.  

Suppose this is all true.  What does it mean?

Nobody knows yet; there are lots of attitudes one could take.

Ambj&oslash;rn, Jurkiewicz and Loll could probably just plunge ahead and
use computers to calculate lots of things about quantum gravity.
(Right now they want to test their results in lots of ways.)  One
good thing would be to include matter of various sorts and see how
it affects the conclusions.

Similarly, Lauscher and Reuter could just plunge ahead and compute,
if they wanted.

This is excellent.  But personally, I'd like to find a beautiful theory
in which spacetime is 2-dimensional at short distance scales, which
reduces to general relativity at large scales.  In other words, to
redo all these calculations "from the bottom up".  

Unsurprisingly, I hope this beautiful theory is a spin foam model, 
since spin foams are 2-dimensional and I like them a lot.  I presented 
some rough ideas on how one might invent such a model:

22) John Baez, Towards a spin foam model of quantum gravity, 
talk at Loops '05, available at
<A HREF = "http://math.ucr.edu/home/baez/loops05/">http://math.ucr.edu/home/baez/loops05/</A>

But, these ideas are very tentative and only time will tell if they amount to
anything.  What's more important is that pure quantum gravity seems
to exist - as a theory, that is - and people seem to be learning
actual facts about it, instead of just arguing endlessly about it.
That's progress!

The second most exciting thing at Loops '05, in my biased opinion,
was the work of John Barrett, Laurent Freidel, Karim Noui and others 
on "matter without matter" in 3d quantum gravity.  Simply by carving a 
Feynman-diagram-shaped hole in 3d spacetime and doing quantum gravity
on the spacetime that's left over, you get a good theory of quantum
gravity coupled to matter!  You can even take the limit as Newton's
gravitational constant goes to zero and get ordinary quantum field
theory on flat spacetime!  

Check these out:

23) John Barrett, Feynman diagams coupled to three-dimensional quantum 
gravity, available as <A HREF = "http://xxx.lanl.gov/abs/gr-qc/0502048">gr-qc/0502048</A>.

John Barrett, Feynman loops and three-dimensional quantum gravity,
Mod. Phys. Lett. A20 (2005) 1271.  Also available as <A HREF = "http://xxx.lanl.gov/abs/gr-qc/0412107">gr-qc/0412107</A>.

24) Laurent Freidel and David Louapre, Ponzano-Regge model revisited
I: gauge fixing, observables and interacting spinning particles, 
Class. Quant. Grav. 21 (2004) 5685-5726.  Also available as 
<A HREF = "http://xxx.lanl.gov/abs/hep-th/0401076">hep-th/0401076</A>.

Laurent Freidel and David Louapre, Ponzano-Regge model revisited 
II: equivalence with Chern-Simons, available as <A HREF = "http://xxx.lanl.gov/abs/gr-qc/0410141">gr-qc/0410141</A>

Laurent Freidel and Etera R. Livine, Ponzano-Regge model 
revisited III: Feynman diagrams and effective field theory,
available as <A HREF = "http://xxx.lanl.gov/abs/hep-th/0502106">hep-th/0502106</A>.

25) Laurent Freidel, Daniele Oriti, and James Ryan, A group field 
theory for 3d quantum gravity coupled to a scalar field, available
as <A HREF = "http://xxx.lanl.gov/abs/gr-qc/0506067">gr-qc/0506067</A>.

26) Karin Noui and Alejandro Perez, Three dimensional loop quantum 
gravity: coupling to point particles, available as <A HREF = "http://xxx.lanl.gov/abs/gr-qc/0402111">gr-qc/0402111</A>.

This is mindblowingly beautiful, especially because lots of it is 
already mathematically rigorous, and we can easily make more so.  
It's even related to n-categories: my student Jeffrey Morton
presented a poster on this aspect.  

Together with my student Derek Wise, Jeffrey Morton and I plan to have 
a lot of fun studying this stuff.  So, I won't talk about it more now - 
I'll probably get around to saying more someday, especially about how 
the whole story generalizes to 4 dimensions.

There's a lot more to say about Loops '05, but this will have to do.
In a while, a bunch of the talks should be visible on the conference 
homepage.... that should give you a better idea of what happened.
 
\par\noindent\rule{\textwidth}{0.4pt}

\textbf{Addendum}: Here are some comments on this Week's Finds by Gene Partlow,
Phillip Helbig and Robert Helling, and my replies - as well as a replies
by Jonathan Thornburg and Arnold Neumaier.

Gene Partlow writes: 



% parser failed at source line 876
