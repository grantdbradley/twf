
% </A>
% </A>
% </A>
\week{August 27, 2007 }


My European wanderings continue.  I'm in Greenwich again, just back 
from a mind-blowing conference in Vienna, part of a bigger program 
that's still going on:

1) Poisson sigma models, Lie algebroids, deformations, and higher 
analogues, Erwin Schr&ouml;dinger Institute, August - September 2007,
organized by Thomas Strobl, Henrique Bursztyn, and Harald Grosse.
Program at <a href = "http://w3.impa.br/~henrique/esi.html">http://w3.impa.br/~henrique/esi.html</a>

I learned a huge amount, both from the talks and from conversations
with Urs Schreiber and others.  Mainly, I learned that I've really 
been falling behind the times when it comes to classical mechanics 
and quantization!

I could easily spend several Weeks trying to assimilate the 
half-digested information I acquired and explain it all to you.  
But, I want to get back to the Tale of Groupoidification!  So, 
I'll only say a little about this wonderful conference.

You may know that in classical mechanics, the space of states of a
physical system is called its "phase space".  Often this is
described by a "symplectic manifold" - a manifold equipped
with a nondegenerate closed 2-form.  Sometimes it's described by a
"Poisson manifold" - a manifold equipped with a bracket
operation on its smooth functions, making the smooth functions into a
Lie algebra and also satisfying the product rule:

{f,gh} = {f,g}h + g{f,h}

Every symplectic manifold gives a Poisson manifold, but not vice
versa.  A good example of a Poisson manifold that's not symplectic
is the phase space of a spinning point particle, which has angular
momentum but no other properties.

Every mathematical physicist should know some symplectic geometry 
and Poisson geometry!  To get started on symplectic geometry, try 
these, in rough order of increasing difficulty:

1) Vladimir I. Arnold, Mathematical Methods of Classical Mechanics,
Springer, Berlin, 1997.
 
2) Ralph Abraham and Jerrold E. Marsden, Foundations of Mechanics, 
Benjamin-Cummings, New York, 1978.

3) Victor Guillemin and Shlomo Sternberg, Symplectic Techniques
in Physics, Cambridge U. Press, Cambridge, 1990.

4) Ana Cannas da Silva, Symplectic geometry, available as 
<a href = "http://arxiv.org/abs/math.SG/0505366">arXiv:math.SG/0505366</a>.

5) Sergei Tabachnikov, Introduction to symplectic topology, 
available at <a href = "http://www.math.psu.edu/tabachni/courses/symplectic.pdf">http://www.math.psu.edu/tabachni/courses/symplectic.pdf</a>

For Poisson geometry, try the above but also:

6) Alan Weinstein, Poisson geometry, available at
<a href = "http://galileo.stmarys-ca.edu/bdavis/poisson.pdf">http://galileo.stmarys-ca.edu/bdavis/poisson.pdf</a>

7) Darryl Holm, Applications of Poisson geometry to physical 
problems, available as <a href = "http://arxiv.org/abs/0708.1585">arXiv:0708.1585</a>.

8) I. Vaisman, Lectures on the Geometry of Poisson Manifolds, 
Birkhaeuser, Boston, 1994.

All this stuff is great.  But lately, people have started thinking 
about generalizations of the idea of phase space that go far beyond 
Poisson manifolds!  In fact there seems to be an infinite sequence,
which begins like this:

\begin{quote}
 symplectic manifolds, <br/>
 Poisson manifolds, <br/>
 Courant algebroids,<br/>
 ...
\end{quote}

I'd heard of Courant algebroids before, but they always seemed
like a scary and arbitrary concept - until I came across this 
paper in Vienna:


 9) Pavol Severa, Some title containing the words
"homotopy" and "symplectic", e.g. this one,
available as <a href =
"http://arxiv.org/abs/math/0105080">arXiv:math/0105080</a>.

The title is goofy, but the paper itself contains some truly
visionary speculations.  Among other things, it argues that the 
above sequence of concepts really goes like this:
 
\begin{quote}
 symplectic manifolds,<br/>
 symplectic Lie algebroids,<br/>
 symplectic Lie 2-algebroids,<br/>
 symplectic Lie 3-algebroids,<br/>
 ...
\end{quote}

These, in turn, are infinitesimal versions of perhaps more fundamental
concepts:

\begin{quote}
 symplectic manifolds, <br/>
 symplectic Lie groupoids, <br/>
 symplectic Lie 2-groupoids,<br/>
 symplectic Lie 3-groupoids,<br/>
 ...
\end{quote}

These concepts take the basic concept of classical phase space and
build in symmetries, symmetries between symmetries, and so on!

So, we may be starting to see the "periodic table of n-categories"
show up in classical mechanics.
Back in "<A HREF = "week49.html">week49</A>" I explained the most basic version of this 
table.  Here's a tiny portion of it:

$$
                  k-tuply monoidal n-categories 

              n = 0           n = 1             n = 2

k = 0         sets          categories         2-categories
     

k = 1        monoids         monoidal           monoidal
                            categories        2-categories


k = 2       commutative      braided            braided
             monoids         monoidal           monoidal
                            categories        2-categories 

k = 3         " "           symmetric           sylleptic
                             monoidal           monoidal
                            categories         2-categories

k = 4         " "             " "               symmetric
                                                monoidal
                                              2-categories

k = 5         " "             " "                "  "
$$
    

An n-category has objects, 1-morphisms betwen objects, 2-morphisms
between 1-morphisms, and so on up to the nth level.  A "k-tuply 
monoidal" n-category is an (n+k)-category that's trivial on the 
bottom k levels.  It masquerades as an n-category with extra bells
and whistles.  As you can see, we get lots of fun structures this 
way.  

The concept of n-category is very general: it describes things,
processes that go between things, metaprocesses that go between
processes and so on.   But, in classical mechanics we may want 
to demand that all these morphisms be invertible, and that all the 
ways of composing them be smooth functions.  Then we should get some 
table like this:

$$
                   k-tuply groupal Lie n-groupoids 

              n = 0           n = 1              n = 2

k = 0       manifolds     Lie groupoids       Lie 2-groupoids
     

k = 1       Lie groups     Lie 2-groups        Lie 3-groups


k = 2        abelian         braided             braided
            Lie groups     Lie 2-groups        Lie 3-groups


k = 3         " "           symmetric           sylleptic
                           Lie 2-groups        Lie 3-groups


k = 4         " "             " "               symmetric
                                               Lie 3-groups


k = 5         " "             " "                "  "
$$
    

There are lots of technical issues to consider - for example, whether
manifolds are a sufficiently general notion of "smooth space" to
make this chart really work.  But for now, the key thing is to 
understand what we're shooting for, so we can set up definitions that 
accomplish it.

For example, it would be nice if we could "differentiate" any of the 
gadgets on the above table, just as we differentiate a Lie group
and get a Lie algebra.  This should give another table, like this:

$$
                   k-tuply groupal Lie n-algebroids 

              n = 0           n = 1                n = 2

k = 0     vector bundles?  Lie algebroids      Lie 2-algebroids
     

k = 1      Lie algebras    Lie 2-algebras      Lie 3-algebras


k = 2        abelian         braided              braided
           Lie algebras   Lie 2-algebras       Lie 3-algebras


k = 3         " "           symmetric            sylleptic
                         Lie 2-algebras        Lie 3-algebras


k = 4         " "             " "               symmetric
                                               Lie 3-algebras


k = 5         " "             " "                "  "
$$
    

The n = k = 0 corner is a bit puzzling - it's sort of degenerate.
Everyone knows how to get Lie algebras from Lie groups.  So, the
real fun starts in getting Lie algebroids from Lie groupoids!  
If you want to see how it works, start here:

10) Alan Weinstein, Groupoids: unifying internal and external 
symmetry, AMS Notices, 43 (1996), 744-752.  Also available as
<a href = "http://arxiv.org/abs/math/9602220">arXiv:math/9602220</a>.

For more details, try this:

11) Kirill Mackenzie, General Theory of Lie Groupoids and Lie 
Algebroids, Cambridge U. Press, 2005.

There's also the question of going back.  We can integrate any
finite-dimensional Lie algebra to get a simply-connected Lie group - 
that's called Lie's 3rd theorem.  But getting from Lie algebroids 
to Lie groupoids is harder... in fact, according to the standard 
definitions, it's often impossible!

That's bad enough, but the really weird part is this: you can 
get something like a Lie \emph{2-groupoid} from a Lie algebroid!  
This throws a serious monkey wrench into the whole periodic
table.

Luckily, one of the people who really understands this stuff
was at this conference in Vienna - Chenchang Zhu.  And, she 
explained what's going on.  So now I'm busily reading her papers:

12) Hsian-Hua Tseng and Chenchang Zhu, Integrating Lie algebroids 
via stacks, available as <a href = "http://arxiv.org/abs/math/0405003"> arXiv:math/0405003</a>.

13) Chenchang Zhu, Lie n-groupoids and stacky Lie groupoids, available
as <a href =
"http://arxiv.org/abs/math/0609420">arXiv:math/0609420</a>.

14) Chenchang Zhu, Lie II theorem for Lie algebroids via stacky 
Lie groupoids, available as <a href = "http://arxiv.org/abs/math/0701024">arXiv:math/0701024</a>.

(Lie's 2nd theorem says that all Lie algebra homomorphisms 
integrate to give homomorphisms between the corresponding 
simply-connected Lie groups.)

I'm optimistic that the patterns will be very beautiful when we 
fully understand them.  In particular, problems also arise
when trying to integrate Lie n-algebras to get Lie n-groups, but
a lot of progress has been made on these problems:
 
15) Ezra Getzler, Lie theory for nilpotent L_{\infty }-algebras,
available as <a href = "http://arxiv.org/abs/math/0404003">arXiv:math/0404003</a>.

16) Andre Henriques, Integrating L_{\infty }-algebras, available
as <a href = "http://arxiv.org/abs/math/0603563">arXiv:math/0603563</a>.

The really wonderful part is that there's already a functioning
theory of Lie n-algebroids, carefully disguised under the name of 
"NQ-manifolds of degree n".  For a great introduction to these, 
see section 2 of this paper:

17) Dmitry Roytenberg, On the structure of graded symplectic 
supermanifolds and Courant algebroids, in Quantization, Poisson 
Brackets and Beyond, ed. Theodore Voronov, Contemp. Math. 315, 
AMS, Providence, Rhode Island, 2002.  Also available as 
<a href = "http://arxiv.org/abs/math.SG/0203110">arxiv:math.SG/0203110</a>.

Using these, people are already busy extending the ideas of
classical mechanics across the top row of the periodic table!

The details are currently rather baroque.  The best way to 
see the big picture, I think, is to simultaneously read the 
above papers by Pavol Severa and Dmitry Roytenberg.  For example,
Roytenberg's paper proves that: 

\begin{quote}
symplectic NQ-manifolds of degree 0 = symplectic manifolds

symplectic NQ-manifolds of degree 1 = Poisson manifolds
 
symplectic NQ-manifolds of degree 2 = Courant algebroids
\end{quote}

If we follow his advice and define Lie n-algebroids to be 
NQ-manifolds of degree n, we can express this by saying:

\begin{quote}
symplectic Lie 0-algebroids = symplectic manifolds

symplectic Lie 1-algebroids = Poisson manifolds
 
symplectic Lie 2-algebroids = Courant algebroids
\end{quote}

And ultimately, Lie n-algebroids should be just a technical
tool for studying Lie n-groupoids - modulo the tricky problems with
the generalizations of Lie's 2nd theorem, mentioned above.

Though I met both Roytenberg and Severa in Vienna, I was 
just beginning to grasp the basics of NQ-manifolds, Courant 
algebroids and the like, so I couldn't take full advantage of this 
opportunity.  I will need to pester them some other time.  In fact, 
I was stuggling to cope with the fact that everything I just mentioned 
is just part of an even bigger story...

This bigger story involves Batalin-Vilkovisky quantization, 
Poisson sigma models, the proof by Kontsevich that every Poisson 
manifold admits a deformation quantization, its interpretation 
by Cattaneo and Felder in the language of 2d TQFTs, and its 
generalization by Hofman and Park to the quantization of Courant 
algebroids using 3d TQFTs... which should itself be the tip of a big
iceberg.  To quantize symplectic Lie n-algebroids, it seems we
need to use (n+1)-dimensional TQFTs! There are some truly 
mind-boggling ideas afoot here, which will turn out to be quite 
simple when properly understood.  For a taste of the underlying 
simplicity, try this:

18) Urs Schreiber, That shift in dimension,
<a href = "http://golem.ph.utexas.edu/category/2007/08/john_baez_and_i_spent.html">http://golem.ph.utexas.edu/category/2007/08/john_baez_and_i_spent.html</a>

But, I'd better learn more before trying to explain these things.

Now, let me return to the Tale of Groupoidification!  When I left
off, I was about to discuss an example: Hecke operators in the 
special case of symmetric groups.  But, one reader expressed 
unease with what I'd done so far, saying it was too informal and 
hand-wavy to understand.  

So, this Week I'll fill in some details about "degroupoidification" - 
the process that sends groupoids to vector spaces and spans of 
groupoids to linear operators.  

How does this work?  For starters, each groupoid X gives a vector 
space [X] whose basis consists of isomorphism classes of objects 
of X.   

Given an object x in X, let's write its isomorphism class 
as [x].  So: x in X gives [x] in [X].

Next, each span of groupoids 

\begin{verbatim}
                       S
                      / \
                     /   \
                    /     \
                   v       v
                  X         Y
\end{verbatim}
    

gives a linear operator 

[S]: [X] \to  [Y]

Note: this operator [S] depends on the whole span, not just the
groupoid S sitting on top.  So, I'm abusing notation here.

More importantly: how do we get this operator [S]?  The recipe is
simple, but I think you'll profit much more by seeing where the 
recipe comes from.

To figure out how it should work, we insist that degroupoidification 
be something like a functor.  In other words, it should get along 
well with composition:

[TS] = [T] [S]

and identities:

[1_{X}] = 1_{[X]}

(Warning: today, just to confuse you, I'll write composition in 
the old-fashioned backwards way, where doing S and then T is 
denoted TS.)

How do we compose spans of groupoids?  We do it using a "weak 
pullback".  In other words, given a composable pair of spans:

\begin{verbatim}
                       S         T
                      / \       / \
                    f/   \g   h/   \i
                    /     \   /     \
                   v       v V       v
                  X         Y         Z
\end{verbatim}
    
we form the weak pullback in the middle, like this:

\begin{verbatim}
                            TS
                           / \
                         j/   \k
                         /     \
                        v       v
                       S         T
                      / \       / \
                    f/   \g   h/   \i
                    /     \   /     \
                   v       v v       v
                  X         Y         Z
\end{verbatim}
    
Then, we compose the arrows along the sides to get a big span 
from X to Z:

\begin{verbatim}
                            TS
                           /  \
                          /    \
                         /      \
                     fj /        \ ik
                       /          \
                      /            \
                     /              \
                    /                \
                   v                  v
                  X                    Z
\end{verbatim}
    
Never heard of "weak pullbacks"?  Okay: I'll tell you what an 
object in the weak pullback TS is.  It's an object t in T and an 
object s in S, together with an isomorphism between their images in Y.  
If we were doing the ordinary pullback, we'd demand that these
images be \emph{equal}.  But that would be evil!  Since t and s are living 
in groupoids, we should only demand that their images be \emph{isomorphic} 
in a specified way.  

(Exercise: figure out the morphisms in the weak pullback.  Figure 
out and prove the universal property of the weak pullback.)

So, how should we take a span of groupoids 

\begin{verbatim}
                       S
                      / \
                     /   \
                    /     \
                   v       v
                  X         Y
\end{verbatim}
    
and turn it into a linear operator 

[S]: [X] \to  [Y] ?

We just need to know what this operator does to a bunch of
vectors in [X].  How do we describe vectors in [X]?  

I already said how to get a basis vector [x] in [X] from any object
x in X.  But, that's not enough for what we're doing now, since a 
linear operator doesn't usually send basis vectors to \emph{basis}
vectors.  So, we need to generalize this idea.

An object x in X is the same as a functor from 1 to X:

\begin{verbatim}
                      1
                      |
                      |p
                      |
                      v
                      X
\end{verbatim}
    
where 1 is the groupoid with one object and one morphism.  So, 
let's generalize this and figure out how \emph{any} functor from 
\emph{any} finite groupoid V to X:

\begin{verbatim}
                      V
                      |
                      |p
                      |
                      v
                      X
\end{verbatim}
    
picks out a vector in [X].  Again, by abuse of notation we'll
call this vector [V], even though it also depends on p.

First suppose V is a finite set, thought of as a groupoid with
only identity morphisms.  Then to define [V], we just go through 
all the points of V, figure out what p maps them to - some bunch 
of objects x in X - and add up the corresponding basis vectors 
[x] in [X].

I hope you see how pathetically simple this idea is!  It's especially
familiar when V and X are both sets.  Here's what it looks like then:

\begin{verbatim}
            ------------------
     V     |    o             |
           |    o  o          |
     |     |    o  o  o     o |   
     |p    | o  o  o  o     o |
     |      ------------------
     v
            ------------------
     X     | o  o  o  o  o  o |
            ------------------
\end{verbatim}
    
I've drawn the elements of V and X as little circles, and shown how 
each element in X has a bunch of elements of V sitting over it.  When 
degroupoidify this to get a vector in the vector space [X], we get:

\begin{verbatim}
     [V] = (1, 4, 3, 2, 0, 2)
\end{verbatim}
    
This vector is just a list of numbers saying how many points of V 
are sitting over each point of X!

Now we just need to generalize a bit further, to cover the case
where V is a groupoid:

\begin{verbatim}
                      V
                      |
                      |p
                      |
                      v
                      X
\end{verbatim}
    
Sitting over each object x in X we have its "essential preimage",
which is a groupoid.  To get the vector [V], we add up basis 
vectors [x] in [X], one for each isomorphism class of objects 
in X, multiplied by the "cardinalities" of their essential preimages.

Now you probably have two questions: 

\textbf{A)} Given a functor p: V \to  X between groupoids and an object
x in X, what's the "essential preimage" of x?

and 

\textbf{B)} what's the "cardinality" of a groupoid?

Here are the answers:

\textbf{A)} An object in the essential preimage of x is an object 
v in V equipped with an isomorphism from p(v) to x.                        

(Exercise: define the morphisms in the essential preimage.
Figure out and prove the universal property of the essential
preimage.  Hint: the essential preimage is a special case of a
weak pullback!)

\textbf{B)} To compute the cardinality of a groupoid, we pick one object
from each isomorphism class, count its automorphisms, take the 
\emph{reciprocal} of this number, and add these numbers up.

(Exercise: check that the cardinality of the groupoid of finite
sets is e = 2.718281828...  If you get stuck, read "<A HREF = "week147.html">week147</A>".)

Also: define the morphisms in the essential preimage.  Figure out 
and prove the universal property of the essential preimage.  Hint: 
the essential preimage is a special case of a weak pullback!)

Okay.  Now in principle you know how any groupoid over X, say

\begin{verbatim}
                      V
                      |
                      |
                      |
                      v
                      X
\end{verbatim}
    
determines a vector [V] in [X].  You have to work some examples
to get a feel for it, but I want to get to the punchline.  We're 
unpeeling an onion here, and we're almost down to the core, where 
you see there's nothing inside and wonder why you were crying so much.

So, let's finally figure out how a span of groupoids

\begin{verbatim}
                       S
                      / \
                     /   \
                    /     \
                   v       v
                  X         Y
\end{verbatim}
    
gives a linear operator 

[S]: [X] \to  [Y] 

It's enough to know what this operator does to vectors 
coming from groupoids over X:

\begin{verbatim}
                       V
                       |
                       |
                       |
                       v
                       X
\end{verbatim}
    
And, the trick is to notice that such a diagram is the same as
a silly span like this:

\begin{verbatim}
                       V
                      / \
                     /   \
                    /     \
                   v       v
                  1         X
\end{verbatim}
    
1 is the groupoid with one object and one morphism, so there's 
only one choice of the left leg here!

So here's what we do.  To apply the operator [S] coming from 
the span

\begin{verbatim}
                       S
                      / \
                     /   \
                    /     \
                   v       v
                  X         Y
\end{verbatim}
    
to the vector [V] corresponding to the silly span

\begin{verbatim}
                       V
                      / \
                     /   \
                    /     \
                   v       v
                  1         X
\end{verbatim}
    
we just compose these spans, and get a silly span

\begin{verbatim}
                       SV
                      /  \
                     /    \
                    /      \
                   v        v
                  1          Y
\end{verbatim}
    
which picks out a vector [SV] in [Y].  Then, we define

[S] [V] = [SV]

Slick, eh?  Of course you need to check that [S] is well-defined.

Given that, it's trivial to prove that [-] gets along with 
composition of spans:

[TS] = [T] [S]

At least, it's trivial once you know that composition of spans is 
associative up to equivalence, and equivalent spans give the same 
operator!  But your friendly neighborhood category theorist can check
such facts in a jiffy, so let's just take them for granted.  Then 
the proof goes like this.  We have:

\begin{verbatim}
[TS] [V] = [(TS)V]       by definition 
         = [T(SV)]       by those facts I just mentioned
         = [T] [SV]      by definition 
         = [T] [S] [V]   by definition 
\end{verbatim}
    
Since this is true for all [V], we conclude

[TS] = [T] [S]

Voil&agrave;!

By the way, if "<A HREF = "week47.html">week47</A>" doesn't satisfy your hunger for information on
groupoid cardinality, try this: 

19) John Baez and James Dolan, From finite sets to Feynman diagrams, in 
Mathematics Unlimited - 2001 and Beyond, vol. 1, eds. Bjorn Engquist 
and Wilfried Schmid, Springer, Berlin, 2001, pp. 29-50.  Also 
available as <A HREF = "http://arxiv.org/abs/math.QA/0004133">math.QA/0004133</A>.

For more on turning spans of groupoids into linear operators, and
composing spans via weak pullback, try these:

20) Jeffrey Morton, Categorified algebra and quantum mechanics, 
TAC 16 (2006), 785-854.  Available at 
<a href = "http://www.emis.de/journals/TAC/volumes/16/29/16-29abs.html">http://www.emis.de/journals/TAC/volumes/16/29/16-29abs.html</a>; 
also available as <A HREF = "http://arxiv.org/abs/math.QA/0601458">math.QA/0601458</A>.

21) Simon Byrne, On Groupoids and Stuff, honors thesis, Macquarie 
University, 2005, available at 
<a href = "http://www.maths.mq.edu.au/~street/ByrneHons.pdf">http://www.maths.mq.edu.au/~street/ByrneHons.pdf</a> and 
<a href = "http://math.ucr.edu/home/baez/qg-spring2004/ByrneHons.pdf">http://math.ucr.edu/home/baez/qg-spring2004/ByrneHons.pdf</a>

For a more leisurely exposition, with a big emphasis on applications
to combinatorics and the quantum mechanics of the harmonic oscillator,
try:

22) John Baez and Derek Wise, Quantization and Categorification, 
lecture notes available at: 
<a href = "http://math.ucr.edu/home/baez/qg-fall2003/">http://math.ucr.edu/home/baez/qg-fall2003/</a> <br/>
<a href = "http://math.ucr.edu/home/baez/qg-winter2004/">http://math.ucr.edu/home/baez/qg-winter2004/</a> <br/>
<a href = "http://math.ucr.edu/home/baez/qg-spring2004/">http://math.ucr.edu/home/baez/qg-spring2004/</a>

Finally, a technical note.  Why did I say the degroupoidification
process was "something like" a functor?  It's because spans of 
groupoids don't want to be a category!  

Already spans of sets don't naturally form a category.  They form a 
weak 2-category!   Since pullbacks are only defined up to canonical 
isomorphism, composition of spans of sets is only associative 
up to isomomorphism... but luckily, this "associator" isomorphism 
satisfies the "pentagon identity" and all that jazz, so we get a 
weak 2-category, or bicategory.

Similarly, spans of groupoids form a weak 3-category.  Weak pullbacks
are only defined up to canonical equivalence, so composition of spans
of groupoids are associative up to equivalence... but luckily this
"associator" equivalence satisfies the pentagon identity up to an
isomorphism, and this "pentagonator" isomomorphism satisfies a 
coherence law of its own, governed by the 3d Stasheff polytope.

So, we're fairly high in the ladder of n-categories.  But, if we
want a mere category, we can take groupoids and \emph{equivalence classes}
of spans.  Then, degroupoidification gives a functor

[-]: [finite groupoids, spans] \to  [vector spaces, linear maps]

That's the fact whose proof I tried to sketch here.  

While I'm talking about annoying technicalities, note we need 
some sort of finiteness assumption on our spans of groupoids
to be sure all the necessary sums converge.   If we go all-out
and restrict to spans where all groupoids involved are finite,
we'll be very safe.  The cardinality of a finite groupoid is a
nonnegative rational number, so we can take our vector spaces to 
be defined over the rational numbers.

But, it's also fun to consider "tame" groupoids, as defined that 
paper I wrote with Jim Dolan.  These have cardinalities that can 
be irrational numbers, like e.   So, in this case we should use
vector spaces over the real numbers - or complex numbers, but that's
overkill.

Finding a class of groupoids or other entities whose cardinalities
are complex would be very nice, to push the whole groupoidification
program further into the complex world.  In the above paper by 
Jeff Morton, he uses sets over U(1), but that's probably not the
last word.


\par\noindent\rule{\textwidth}{0.4pt}
<em>Viewed superficially, mathematics is the result of centuries of effort by
thousands of largely unconnected individuals scattered across continents, 
centuries and millennia.  However the internal logic of its development
much more closely resembles the work of a single intellect developing its
thought in a continuous and systematics way - much as in an orchestra playing
a symphony written by some composer the theme moves from one instrument 
to another, so that as soon as one performer is forced to cut short his part,
it is taken up by another player, who continues with due attention to the
score.</em> - I. R. Shavarevich

\par\noindent\rule{\textwidth}{0.4pt}

% </A>
% </A>
% </A>
