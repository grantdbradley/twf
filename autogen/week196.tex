<html>

% </A>
% </A>
% </A>
\week{June 1, 2003}


Today I'd like to talk about the Big Bang and Pythagorean spinors.
But first, a book!  If you want to start learning general relativity
without first mastering the intricacies of tensors, here's a way to
get going:

1) James B. Hartle, Gravity: an Introduction to Einstein's General
Relativity, Addison-Wesley, San Francisco, 2003.

Hartle is an expert on general relativity, but here he avoids showing
off.  He gets to the physics of general relativity as quickly and
simply as possible, avoiding the usual route of doing huge amounts of
math first.  In particular, he works out the physics of specific
solutions of Einstein's equations - like those describing black holes
and the Big Bang - before he introduces the equations.  This puts off
the hard abstract stuff until later, when the student has more feeling
for it.  The purists may grumble, but I have a feeling it's pedagogically 
sound.

Now... what happened in the first second after the Big Bang?

This may sound like an insanely ambitious question, but in fact we
seem to have a fairly good idea of what happened all the way back to
the first microsecond - unless, of course, there's some important
physics we're missing.  This paper tells the story quite nicely:

2) Dominik J. Schwarz, The first second of the universe,
available as <A HREF = "http://xxx.lanl.gov/abs/astro-ph/0303574">astro-ph/0303574</A>.

But, since the physics gets weirder as we approach the Big Bang, 
I'll tell this story in reverse order - and I'll start from \emph{now},
just to set the stage.  

So, here's a quick reverse history of the universe:


<UL>
<LI>
13.7 billion years after the Big Bang: now.  <br>
Temperature: 2.726 K 
</UL>
According to data from the Wilkinskon Microwave Anisotropy Probe
(WMAP), our best estimate of the age of the universe is 13.7 billion
years, plus or minus 150 million years or so.  Previous estimates
were similar, but with an uncertainty of about half a billion years.

The temperature listed here is that of the cosmic microwave background
radiation.  Slight deviations from this average figure were first
detected in 1992 by the Cosmic Background Explorer (COBE).  This
satellite-based experiment found hot and cold patches in the microwave
background that differ from the mean temperature by an amount on the
order of 30 microkelvin.  WMAP is a more refined experiment along the
same lines, which came out with a lot of exciting new results in
February, 2003.


<UL>
<LI>
200 million years after the Big Bang: reionization.<BR>
Temperature: roughly 50 K.
</UL>
"Reionization" is the name for when the hydrogen in the universe,
which had cooled after the Big Bang, became hot and ionized again.
The most likely cause is radiation from the very first stars.  So, we
now think the first stars ignited about 200 million years after the
Big Bang.

Again this is a result from WMAP.  The error bars are quite large, so
the numbers above could be off by a factor of two or so.  Nonetheless,
a lot of people were surprised by this result, since they thought that
the clumping of matter due to gravity would have taken \emph{much} longer to
form stars!

There's a lot we don't know about star and galaxy formation, because
the current conventional wisdom requires that the gravitational
clumping was seeded by "cold dark matter" - but nobody knows what this
stuff is.


<UL>
<LI>
380 thousand years after the Big Bang: recombination.  <BR>
Temperature: 3000 K, or .25 eV of energy per particle.
</UL>
"Recombination" is the usual stupid name for when the
universe cooled down enough for electrons and protons to stick
together and form hydrogen atoms.  It should really be called
"combination", because the electrons and protons were never
combined \emph{before} this time.  Before this time the universe was always
full of plasma - that is, electrically charged particles running
around loose.  Afterwards it was full of electrically neutral
hydrogen... at least until the stars lit up and reionized a lot of
this hydrogen.

Plasma absorbs light of all frequencies, while electrically neutral
gases tend to be transparent except for certain frequency bands.
Thus, recombination was the first time when light could start
travelling for long distances without getting absorbed!  For this
reason, the cosmic background radiation we see now consists of the
photons emitted right at the time of recombination.  When emitted it
had a temperature of about 3000 kelvin, but it has cooled with the
expansion of the universe.

The era between recombination and the ignition of the first stars goes
by a romantic name: the Dark Ages.  Adding to the romance, Pfenniger
and Puy hypothesized that hydrogen could have frozen into crystalline
flakes before the stars lit up and began warming the universe again.
A cold, dark, eerie universe full of hydrogen snowflakes... the
thought sends shivers right up my spine!  Unfortunately, if stars
formed as early as WMAP says they did, the universe would probably 
not get cold enough for these crystals to form.  

By the way, the figure of 380 thousand years for the time of
recombination is another result from WMAP, consistent with previous
estimates, but presumably more accurate.

<UL>
<LI>
10 thousand years after the Big Bang: end of the radiation-dominated era.<BR>
Temperature: 12,000 K, or 1 eV per particle.
</UL>
Before this time, the energy density due to light exceeded that due to
matter, so we say the universe was "radiation-dominated".  Afterwards
the universe became "matter-dominated" - at least until considerably
later, when matter spread out so thin that the dominant form of energy
became "dark energy", as it seems to be now.

(The best estimate due to WMAP says that currently the energy in the
universe is 4% ordinary matter, 23% cold dark matter and 73% dark
energy.  For more on the latter two concepts, see "<A HREF = "week167.html">week167</A>".)

The end of the radiation-dominated era is important because this is
when gravity began to amplify small fluctuations in the density of
matter.  In other words, this is when stuff began to form clumps of
various sizes, eventually leading to stars, galaxies, galaxy clusters,
and so on.  During the radiation-dominated era, density fluctuations
were mainly made of \emph{light}, and these could not grow because the
light was moving too fast to form clumps.  People believe that as soon
as this era ended, cold dark matter began clumping up under its own
gravity.  Ordinary matter started clumping up later, after
recombination - since before that it was in the form of plasma, 
which stayed smoothed out by its interaction with light.

<UL>
<LI>
1000 seconds after the Big Bang: decay of lone neutrons.   <BR>
Temperature: roughly 500 million K, or about 50 keV per particle.
</UL>

A lone neutron is not a stable particle: with a mean lifetime of 886
seconds, it will decay into a proton, electron and antineutrino.  So,
any neutrons created early in the history of the universe must fuse
with protons to form nuclei by roughly this time, or they are doomed
to decay.

<UL>
<LI>
180 seconds after the Big Bang: nucleosynthesis begins.<BR>
Temperature: roughly 1 billion K, or about 100 keV per particle.
</UL>

At about this time, the temperature dropped to the point where a proton
and neutron could stick together forming a deuterium nucleus, and the
process of "nucleosynthesis" began, in which deuterium nuclei stick
together to form nuclei of helium.  This is responsible for the fact
that even before the stars started processing hydrogen into heavier
elements, the universe was about 25% helium, the rest being almost all
hydrogen.

<UL>
<LI>
10 seconds after the Big Bang: annihilation of electron-positron pairs.<BR>
Temperature: roughly 5 billion K, or about 500 keV per particle.
</UL>

Apart from the neutrinos and the photon, the lightest particle in
nature is the electron.  The rest mass of an electron corresponds to
an energy of 511 keV, so it only takes twice that much energy to
create an electron-positron pair.  If we multiply 511 MeV by
Boltzmann's constant, we get a temperature of roughly 5 billion
kelvin.  That means that at this temperature, two particles colliding
head-on will often have enough kinetic energy to create a
electron-positron pair.  So, when it's this hot or hotter, collisions
between particles generate a thick stew of electrons and positrons!

But as temperatures cool below this point, the density of this stew
drops off exponentially: electron-positron pairs annihilate each
other, leaving radiation.  This happened roughly 10 seconds after the
Big Bang.

<UL>
<LI>
1 second after the Big Bang: decoupling of neutrinos.<BR>
Temperature: roughly 10 billion K, or about 1 MeV per particle.
</UL>

Neutrinos can easily zip through light-years of lead, but the very
early universe was so compressed that they interacted vigorously with
other forms of matter.  But around a second after the Big Bang, the
density of the universe decreased to about 400,000 times that of
water, and neutrinos "decoupled" from other matter.  

Since these neutrinos were not reheated by nucleosynthesis, they
should now be cooler than the cosmic microwave background radiation -
about 2 kelvin instead of 2.726 kelvin.  We are currently unable to
detect such unenergetic neutrinos, but detecting them would be a major
confirmation that our theories of the early universe are correct.

<UL>
<LI>
100 microseconds after the Big Bang: annihilation of pions.<BR>
Temperature: roughly 1 trillion K, or about 100 MeV per particle.
</UL>

Particles made of quarks and antiquarks are called
"hadrons", and they interact via the strong nuclear force.
The only hadrons we encounter in daily life are protons and neutrons,
made of 3 quarks each.  But the lightest hadrons are the pions, which
come in positive, negative and neutral forms.  The positive and
negative ones are antiparticles of each other, while the neutral one
is its own antiparticle.  They all have mass on the order of 100 MeV.

Just as I described for electron-positron pairs, at a high enough
temperature everything is always awash in a sea of pions, while below
this temperatures the pions quickly disappear by annihilation.  To
estimate the relevant temperature, we can just convert its mass to a
temperature following the rough rule 1 MeV ~ 10 billion kelvin.  So,
when the temperature of the early universe dropped below 1 trillion
kelvin, pions went away.  This happened around 100 microseconds after
the Big Bang.  Before this, hadrons ruled!

<UL>
<LI>
50 microseconds after the Big Bang: QCD phase transition.<BR>
Temperature: 1.7-2.1 trillion K, corresponding to 150-180 MeV per particle.
</UL>

At normal temperatures, quarks and antiquarks are confined within
hadrons by the strong force.  The strong force is carried by gluons,
so you can vaguely visualize a hadron as a bag-like thing in which
quarks and antiquarks wiggle about, constantly exchanging virtual
gluons, which also exchange virtual gluons, quarks and antiquarks of
their own.  The details are described by "quantum chromodynamics", or
QCD.  Since QCD says the strong force gets stronger with increasing
distance, if you try to pull a quark out of this bag, it takes enough
energy to create a whole new bag!

But if you have a bunch of hadrons at temperatures above 2 trillion
kelvin or so, they'll be smashing into each other so furiously that
the distinction between the "bags" and the "space
between the bags", never completely sharp, dissolves entirely.
At this point, all you've got is a bunch of quarks, antiquarks and
gluons zipping around.  This is a new state of matter: a
"quark-gluon plasma".  In "<A HREF =
"week76.html">week76</A>" and "<A HREF =
"week117.html">week117</A>" I described how how people at the
Relativistic Heavy Ion Collider in Brookhaven are making quark-gluon
plasmas by smashing nuclei at each other at high speeds.

A lot of Dominik Schwarz's paper is about the "QCD phase transition"
which happened about 50 microseconds after the Big Bang, when the
universe cooled down enough for the quark-gluon plasma to condense
into the confined phase.  Though the subject is controversial, most
people think this phase transition is a "first-order" transition,
meaning that heat is emitted as the transition happens, just as when
water vapor condenses to form liquid droplets.  If so, the quark-gluon
plasma would probably supercool until small bubbles of hadron phase
formed.  As these bubbles grew, latent heat would be emitted.  This
would tend to reheat the quark-gluon plasma, limiting the speed at
which the bubbles expand.  Heat would mainly be dispersed by means of
neutrinos and acoustic waves - i.e., sound.

<UL>
<LI>
10 picoseconds after the Big Bang: electroweak phase transition.<BR>
Temperature: 1-2 quadrillion K, corresponding to 100-200 GeV per particle.
</UL>

At high enough temperatures, there should be no difference between the
electromagnetic force and weak force.  This difference should only
arise when things cool down enough for the Higgs field to settle into
a fixed position, breaking the symmetry between these forces - a bit
like how ice crystallizes, breaking the rotational symmetry of liquid
water.  At higher temperatures the Higgs field wiggles around too
much to settle down.  Or in the language of particles rather than
fields: collisions between particles create a stew of Higgs bosons!

The mass of Higgs seems to be somewhere around 130 GeV.  If so, the
electroweak phase transition would have occured roughly 10^{-11}
seconds after the Big Bang.  But, since we haven't actually gotten
direct evidence for the Higgs boson yet, this is still a bit
speculative.  Right now people say the Large Hadron Collider at CERN
will come online and start looking for the Higgs in 2007.  But the LHC
project has gotten some nasty budget cuts recently, so I wouldn't be
surprised if there were delays.  If and when the Higgs is found, maybe
I'll return to this topic and say what people think happened \emph{before}
the electroweak phase transition.  People have thought about this a
lot.  But for now, I'll quit here!

If you want to learn more about the early universe, start with
this classic:

3) Steven Weinberg, The First Three Minutes, Basic Books, New York,
1977.

Then catch up with recent developments by reading these websites:

3) Ned Wright's Cosmology Tutorial, 
<A HREF = "http://www.astro.ucla.edu/~wright/cosmolog.htm">http://www.astro.ucla.edu/~wright/cosmolog.htm</A>

4) Martin White, The Cosmic Rosetta Stone, 
<A HREF = "http://astron.berkeley.edu/~mwhite/rosetta/rosetta.html">http://astron.berkeley.edu/~mwhite/rosetta/rosetta.html</A>

To dig deeper, try these books:

5) P. Coles and F. Lucchin, Cosmology: The Origin and Evolution
of Cosmic Structure, Wiley, New York, 1995.

6) Edward W. Kolb and Michael Turner, The Early Universe, 
Addison-Wesley, Reading, Massachusetts, 1990.

For a detailed description of some of WMAP's results, try these:

7) C. L. Bennett et al, First Year Wilkinson Microwave Anisotropy
Probe (WMAP) Observations: Preliminary Maps and Basic Results, 
available as <A HREF = "http://xxx.lanl.gov/abs/astro-ph/0302207">astro-ph/0302207</A>.

8) D. N. Spergel et al, First Year Wilkinson Microwave Anisotropy
Probe (WMAP) Observations: Determination of Cosmological Parameters,
available as <A HREF = "http://xxx.lanl.gov/abs/astro-ph/0302209">astro-ph/0302209</A>.

These are two of thirteen related papers produced by the WMAP team!
Both of them have lots of coauthors, one of which is Ned Wright, author
of the nice website mentioned above.

Here is Pfenniger and Puy's paper on hydrogen "snowflakes":

9) D. Pfenniger and D. Puy, Possible flakes of molecular hydrogen in
the early Universe, available as <A HREF = "http://xxx.lanl.gov/abs/astro-ph/0211393">astro-ph/0211393</A>.

I should also thank Ted Bunn for telling me some stuff about
converting between times, temperatures and redshifts.  Cosmologists of
the early universe use "redshift z" to stand for the time when the
universe was 1/(z+1) times as big as it is now - by which I mean,
distances between faraway objects were multiplied by this factor.
Equivalently, this is the time when the temperature of the background
radiation was z+1 times as big as it is now.  So, converting between
temperatures and redshifts is easier.  Converting these to times is
less trivial, and indeed the times listed above are more likely to
suffer from inaccuracies than the temperatures.

By the way, some people say it's confusing to use numbers like
"billion", "trillion" and "quadrillion"
to mean 10^{9}, 10^{12} and 10^{15},
respectively - because these are American usages, and Europeans (they
claim) use "milliard", "billion" and
"billiard" for these numbers.  These people say that, for
example "gigakelvin", "terakelvin" and
"exakelvin" are less ambiguous than "billion
kelvin", "trillion kelvin" and "quadrillion
kelvin".

This is probably true, but it's also true that fewer people know
what an "exakelvin" is than a "quadrillion kelvin".  Since I was
trying to explain cosmology to the unwashed masses, I opted to 
use number words above, and explain them here.  I was using the
American system... and I'm sort of betting this system will take 
over, because I've \emph{never} 
heard anyone use the word "billiard" 
to mean 10^{15}.   More importantly, I really hope that \emph{some}
system takes over, because it's a bit sad not to be able to use 
words for numbers.

Speaking of numbers... now for some math!

The volume in honor of Penrose's 65th birthday is full of fun stuff 
about spin networks, twistors, and so on - but I particularly liked 
this paper by Trautman on "Pythagorean spinors":

10) Andrzej Trautman, Pythagorean spinors and Penrose twistors, in 
The Geometric Universe: Science Geometry and the Work of Roger Penrose,
eds. Huggett, Mason, Tod, Tsou and Woodhouse, Oxford U. Press,
Oxford, 1998.  Also available at <A HREF = "http://www.fuw.edu.pl/~amt/amt.html">http://www.fuw.edu.pl/~amt/amt.html</A>

If you're a physicist you'll have heard about Dirac spinors, Weyl
spinors, Majorana spinors, and maybe even Majorana-Weyl spinors.  
I've you haven't, you can read my explanations in
"<A HREF = "week93.html">week93</A>".  But 
what in the world are "Pythagorean" spinors?  The basic idea is that 
from two spinors you can make a vector, and Trautman points out
that a special case of this idea gives a famous old formula for
getting Pythagorean triples - that is, integers a,b,c with

a^{2} + b^{2} = c^{2}.

I think I'll explain this in detail....

Spinors are used to describe spin-1/2 particles, so-called
because they don't come back to where they were when you 
turn them around 360 degrees - you have to rotate them \emph{twice} 
to get back where you started!  Thus, mathematically, spinors 
are representations of the double cover of the rotation group,
or the double cover of the Lorentz group if you take special 
relativity into account.  

In 4d spacetime, the double cover of the Lorentz group is 
SL(2,C), the group of 2\times 2 complex matrices with determinant 1.  
We can take a spinor to be just a pair of complex numbers, 
but there are actually two ways such a thing can transform 
under SL(2,C).  One way is obvious, but for the other we take 
the \emph{complex conjugate} of the matrix before letting it act 
on the spinor.  We get two sorts of spinors, called left- 
and right-handed "Weyl spinors".  In physics, we use these 
to describe massless particles that spin either clockwise or 
counterclockwise along their line of motion as they zip 
along at the speed of light.

In 3d spacetime, the double cover of the Lorentz group is SL(2,R),
the group of 2\times 2 \emph{real} matrices with determinant 1.  In this
dimension, we can take a spinor to be a pair of \emph{real} numbers.
But since we don't have complex conjugation at our disposal, we 
don't get left- and right-handed versions of these spinors, and
we don't call them Weyl spinors.  Since they are real, we call them
"Majorana spinors".

Since Pythagoras had a strong fondness for number theory, if he 
were alive today he might want to simplify things even further 
and consider SL(2,Z), the group of 2\times 2 \emph{integer} matrices with 
determinant 1.  This acts on "Pythagorean spinors", namely pairs 
of integers.

We could also go up to higher dimensions using the quaternions
and octonions: SL(2,H) is the double cover of the Lorentz
group in 6d spacetime, and SL(2,O) is the double cover of the
Lorentz group in 10d spacetime.  But I explained this in my
octonion webpage:

11) John Baez, OP^{1} and Lorentzian geometry, 
<A HREF = "http://math.ucr.edu/home/baez/octonions/node11.html">http://math.ucr.edu/home/baez/octonions/node11.html</A>

so I won't talk about it now.

In each case, there's a trick for turning a spinor into a 
lightlike vector.  In 4 dimensions we do it like this: we 
take a left-handed spinor \psi , take its conjugate transpose 
to get a right-handed spinor \psi *, and form 


$$

\psi  \otimes  \psi *
$$
    
which we can think of as a 2\times 2 hermitian matrix.   If you're
a fancy mathematical physicist you know that the space of 
2\times 2 hermitian matrices is the same as 4d Minkowski spacetime, 
with the matrices of determinant zero corresponding to the 
lightlike vectors, so you're done!  Otherwise, you can work 
out the above matrix explicitly:


\begin{verbatim}

\psi  = (a)                                         (a column vector)
    (b)

\psi  = (a*, b*)                                    (a row vector)


\psi  \otimes  \psi * = (aa* ab*)                              (a 2\times 2 matrix)
         (ba* bb*)                  
\end{verbatim}
    
This matrix is hermitian, so you can write it as a real 
linear combination of Pauli matrices:


$$

\psi  \otimes  \psi * = t \sigma _{t} + x \sigma _{x} + y \sigma _{y} + z \sigma _{z}
$$
    
where


\begin{verbatim}

\sigma _{t} = (1  0)
     (0  1)

\sigma _{x} = (0  1)
     (1  0)

\sigma _{y} = (0 -i)
     (i  0)

\sigma _{z} = (1  0)
     (0 -1)
\end{verbatim}
    
You get a vector in Minkowski spacetime, (t,x,y,z).   
If you check that this vector is lightlike:


$$

t^{2} = x^{2} + y^{2} + z^{2} 
$$
    
you'll be done.

The trick in 3 dimensions is just the same except that now
the components of \psi  are real numbers, so things simplify:
we don't need complex conjugation, and \psi  \otimes  \psi * 
will be a \emph{real} hermitian matrix.  Real hermitian matrices
are the same as vectors in 3d Minkowski spacetime, since we 
can write them as linear combinations of the three Pauli 
matrices without i's in them - namely, all of them except
\sigma _{y}.  So, we get a lightlike vector in 3d Minkowski 
spacetime: say, (t,x,z) with 

t^{2} = x^{2} + z^{2} 

Now for the really fun part: the trick works the same with 
Pythagorean spinors except now everything in sight is an integer... 

... so (t,x,z) is a Pythagorean triple!!!   And in fact,
we get every Pythagorean triple this way, at least up to
an integer multiple.  And in fact, this trick was already 
known by Euclid.  

Explicitly, if 


$$

\psi  = (a)
    (b)
$$
    
then 


\begin{verbatim}

2 \psi  \otimes  \psi * = (2a^{2}  2ab) 
           (2ab  2b^{2})                  

         = (a^{2} + b^{2}) \sigma _{t} + 2ab \sigma _{x}  + (a^{2} - b^{2}) \sigma _{z}
\end{verbatim}
    
so we get the Pythagorean triple


$$

(t,x,z) = (a^{2} + b^{2}, 2ab, a^{2} - b^{2})
$$
    
For example, if we take our spinor to be


$$

\psi  = (2)
    (1)
$$
    
we get the famous triple 


\begin{verbatim}

(t,x,z) = (5,4,3)  
\end{verbatim}
    

By the way, you'll notice I had to insert a fudge factor of
"2" in that formula up there to get things to work.  I'm not
sure why.

\par\noindent\rule{\textwidth}{0.4pt}
\textbf{Addendum:} Thanks to Andy Everett for catching a typo.
Noam Elkies sent me the following:

\begin{quote}

\begin{verbatim}

Dear J. Baez,

In article <bbcc58$olh$1@glue.ucr.edu> you write:

>[......] so (t,x,z) is a Pythagorean triple!!!   And in fact,
>we get every Pythagorean triple this way, at least up to
>an integer multiple.  And in fact, this trick was already
>known by Euclid.

Are you sure of this?  The formula was surely known by Euclid's time --
I've even seen claims that it must have been known by the Babylonians
(perhaps not a coincidence, since Pythagoras spent some time in Babylonia)
-- but did Euclid have anything like this interpretation, or the proof
that every "Pythagorean triple" is proportional to one of this form?

[For that matter there's apparently some controversy about just who
Pythagoras might have been and what he might have known, believed,
and/or proved, since secrecy was one of the Pythagoreans' tenets.]

>By the way, you'll notice I had to insert a fudge factor of "2"
>in that formula up there to get things to work.  I'm not sure why.

This is presumably an artifact of the ambiguity in the
"symmetric square of Z".  In general, the symmetric square
of a module M can be formed as either a submodule or a quotient
module of the tensor square of M.  These two symmetric squares
are (canonically) isomorphic if 2 is invertible, but not in general.

The parametrization of Pythagorean triples is also closely related
with the "half-angle substitution" of elementary calculus.
For yet another interpretation, see

<A HREF = "http://www.math.harvard.edu/~elkies/Misc/hilbert.dvi">http://www.math.harvard.edu/~elkies/Misc/hilbert.dvi</A>

[also <A HREF = "http://www.math.harvard.edu/~elkies/Misc/hilbert.pdf">.pdf</A> instead of .dvi].

The dissections that illustrate the Pythagorean theorem
can be generalized to the law of cosines; see

<A HREF = "http://www.math.harvard.edu/~elkies/Misc/cos1.ps">http://www.math.harvard.edu/~elkies/Misc/cos1.ps</A>
<A HREF = "http://www.math.harvard.edu/~elkies/Misc/cos2.ps">http://www.math.harvard.edu/~elkies/Misc/cos2.ps</A>

for the acute and obtuse cases respectively.

Enjoy,
--Noam D. Elkies

\end{verbatim}
    
\end{quote}
I replied saying that by Euclid knowing "this trick", I only 
meant he knew this formula for Pythagorean triples:

$$

(a^{2} + b^{2}, 2ab, a^{2} - b^{2})
$$
    
I don't know who proved it gives \emph{all} of them (up to multiples).  
Rob Johnson then noted:
\begin{quote}

\begin{verbatim}


There is not any high powered math involved in showing that these are
all the pythagorean triples up to scalar multiples.  See

    <A HREF = "http://www.whim.org/nebula/math/pythag.html">http://www.whim.org/nebula/math/pythag.html</A>

Nothing more than the fundamental theorem of arithmetic is used (to
justify the statement "Since b^2 = 4MN and gcd(M,N) = 1, both M and N
must be perfect squares."), and Euclid knew that.

So my guess is that Euclid probably knew that this formula gives all
pythagorean triples up to scalar multiples, but it is just a guess.

Rob Johnson
rob@whim.org
\end{verbatim}
    
\end{quote}

\par\noindent\rule{\textwidth}{0.4pt}
% </A>
% </A>
% </A>
