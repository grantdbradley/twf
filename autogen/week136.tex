
% </A>
% </A>
% </A>
\week{August 21, 1999 }

I spent most of last month in Portugal, spending time with Roger Picken
at the Instituto Superior Tecnico in Lisbon and attending the category
theory school and conference in Coimbra, which was organized by Manuela
Sobral:

1) Category Theory 99 website, with abstracts of talks, 
<A HREF ="http://www.mat.uc.pt/~ct99/">http://www.mat.uc.pt/~ct99/</A>

The conference was a big deal this year, because it celebrated the
90th birthday of Saunders Mac Lane, who with Samuel Eilenberg invented
category theory in 1945.  Mac Lane was there and in fine fettle.  He
gave a nice talk about working with Eilenberg, and after the banquet
in his honor, he even sang a song about Riemann while wrapped in a
black cloak!

(In case you're wondering, the cloak was contributed by some musicians.
In Coimbra, the folks who play fado music tend to wear black cloaks. 
A few days ago we'd seen them serenade a tearful old man and then wrap 
him in a cloak, so one of our number suggested that they try this trick on 
Mac Lane.  Far from breaking into tears, he burst into song.)

The conference was exquisitely well-organized, packed with top 
category theorists, and stuffed with so many cool talks I scarcely 
know where to begin describing them... I'll probably say a bit 
about a random sampling of them next time, and the proceedings 
will appear in a special issue of the Journal of Pure and Applied
Algebra honoring Mac Lane's 90th birthday, so keep your eye out for 
that if you're interested.  The school featured courses by Cristina 
Pedicchio, Vaughan Pratt, and some crazy mathematical physicist who 
thinks the laws of physics are based on n-categories.  The notes can 
be found in the following book: 

2) School on Category Theory and Applications, Coimbra, July
13-17, 199, Textos de Matematica Serie B No. 21, Departamento De 
Matematica da Universidade de Coimbra.  Contains: "n-Categories" 
by John Baez, "Algebraic theories" by M. Cristina Pedicchio, and 
"Chu Spaces: duality as a common foundation for computation and 
mathematics" by Vaughan Pratt.  

Pedicchio's course covered various generalizations of Lawvere's
wonderful concept of an algebraic theory.  Recall from "<A HREF = "week53.html">week53</A>" 
that we can think of a category C with extra properties or structure 
as a kind of "theory", and functors F: C \to  Set preserving this 
structure as "models" of the theory.  For example, a "finite
products theory" C is just a category with finite products.  In
this case, a model is a functor F: C \to  Set preserving finite 
products, and a morphism of models is a natural transformation 
between such functors.  This gives us a category Mod(C) of models 
of C.

To understand what this really means, let's restrict attention
the simplest case, when all the objects in C are products of a
given object x.  In this case Pedicchio calls C an "algebraic
theory".  A model F is then really just a set F(x) together 
with a bunch of n-ary operations coming from the morphisms in C, 
satisfying equational laws coming from the equations between 
morphisms in C.  Any sort of algebraic gadget that's just
a set with a bunch of n-ary operations satisfying equations can
be described using a theory of this sort.  For example: monoids, 
groups, abelian groups, rings... and so on.  We can describe
any of these using a suitable algebraic theory, and in each case, 
the category Mod(C) will be the category of these algebraic gadgets.

Now, what I didn't explain last time I discussed this was the 
notion of theory-model duality.  Fans of "duality" in all its 
forms are sure to like this!  There's a functor

R: Mod(C) \to  Set

which carries each model F to the set F(x).  We can think of this
as a functor which forgets all the operations of our algebraic
gadget and remembers only the underlying set.  Now, if you know
about adjoint functors (see "<A HREF = "week77.html">week77</A>"-"<A HREF = "week79.html">week79</A>"), this should 
immediately make you want to find a left adjoint for R, namely 
a functor 

L: Set \to  Mod(C)

sending each set to the "free" algebraic gadget on this set.
Indeed, such a left adjoint exists!

Given this pair of adjoint functors we can do all sorts of fun 
stuff.  In particular, we can talk about the category of "
finitely generated free models" 
of our theory.  The objects here are objects of Mod(C)
of the form L(S) where S is a finite set, and the morphisms are the usual 
morphisms in Mod(C).  Let me call this category fgFree-Mod(C).

Now for the marvelous duality theorem: fgFree-Mod(C) is equivalent
to the opposite of the category C.   In other words, you can 
reconstruct an algebraic theory from its category of finitely
generated free algebras in the simplest manner imaginable: just 
reversing the direction
of all the morphisms!  This is so nice I won't explain why it's
true... I don't want to deprive you of the pleasure of looking
at some simple examples and seeing for yourself how it works.  
For example, take the theory of groups, and figure out how every
operation appearing in the definition of "group" corresponds to
a homomorphism between finitely generated free groups.

There are lots of other interesting questions related to
theory-model duality.  For example: what kinds of categories
arise as categories of models of an algebraic theory?  Pedicchio
calls these "algebraic categories", and she told us some nice 
theorems characterizing them.  Or: given the category of free
models of an algebraic theory, can you fatten it up to get 
the category of \emph{all} models?   Pedicchio mentioned a process
called "exact completion" that does the job.  Or: starting 
from just the category of models of a theory, can you tell 
which are the free models?  Alas, I don't know the answer
to this... but I'm sure people do.

Even better, all of this can be generalized immensely, to 
theories of a more flexible sort than the "algebraic theories"
I've been talking about so far.   For example, we can study
"essentially algebraic theories", which are just categories
with finite limits.  Given one of these, say C, we define a
model to be a functor F: C \to  Set preserving finite limits.
This allows one to study algebraic structures with 
partially defined operations.  I already gave an example in 
"<A HREF = "week53.html">week53</A>" - there's a category with finite limits called 
"the theory of categories", whose models are categories!  
One can work out theory-model duality in this bigger context, 
where it's called Gabriel-Ulmer duality:

3) P. Gabriel and F. Ulmer, Lokal praesentierbare Kategorien,
Springer Lecture Notes in Mathematics, Berlin, 1971. 

But this stuff goes far beyond that, and Pedicchio led us at a 
rapid pace all the way up to the latest work.  A lot of the basic 
ideas here came from Lawvere's famous thesis on algebraic
semantics, so it was nice to see him attending these lectures, and 
even nicer to hear that 26 years after he wrote it, his thesis 
is about to be published:

4) William Lawvere, Functorial Semantics of Algebraic Theories,
Ph.D. Dissertation, University of Columbia, 1963.   Summary 
appears under same title in: Proceedings of the National 
Academy of Sciences of the USA 50 (1963), 869-872.

(Unfortunately I forget who is publishing it!)  It was also nice 
to find out that Lawvere and Schanuel are writing a book on 
"objective number theory"... which will presumably be more 
difficult, but hopefully not less delightful, than their
wonderful introduction to category theory for people who 
know \emph{nothing} about fancy mathematics:

5) William Lawvere and Steve Schanuel, Conceptual Mathematics:
A First Introduction to Categories, Cambridge U. Press, Cambridge 1997.

This is the book to give to all your friends who are wondering
what category theory is about and want to learn a bit without
too much pain.  If you've read this far and understood what I 
was talking about, you must have such friends!  If you \emph{didn't}
understand what I was talking about, read this book!

By the way, Lawvere told me that he started out wanting to do
physics, and wound up doing his thesis on algebraic semantics 
when he started to trying to formalize what a physical theory
was.   It's interesting that the modern notion of "topological
quantum field theory" is very much modelled after Lawvere's ideas,
but with symmetric monoidal categories with duals replacing the
categories with finite products which Lawvere considered!  I guess
he was just ahead of his time.  In fact, he has returned to physics
in more recent years - but that's another story.

Okay, let me change gears now....

Some n-category gossip.  Ross Street has a student who has defined a
notion of semistrict n-category up to n = 5, and Sjoerd Crans has
defined semistrict n-categories (which he calls "teisi") for n up to
6.  However, the notion still seems to resist definition for general
n, which prompted my pal Lisa Raphals to compose the following
limerick:

$$

     A theoretician of "n"
     Considered conditions on when
     Some mathematicians
     Could find definitions 
     For n even greater than ten.
$$
    
Interestingly, work on weak n-categories seems to be proceeding at
a slightly faster clip - they've gotten to n = infinity already.  
In fact, during the conference Michael Batanin came up to me and said
that a fellow named Penon had published a really terse definition of 
weak \omega -categories that seems equivalent to Batanin's own 
(see "<A HREF = "week103.html">week103</A>") - at least after some minor tweaking.  Batanin was quite 
enthusiastic and said he plans to write a paper about this stuff.  

Later, when I went to Cambridge England, Tom Leinster gave a talk
summarizing Penon's definition:

6) Jacques Penon, Approache polygraphique des $\infty$-categories
non strictes, in Cahiers Top. Geom. Diff. 40 (1999), 31-79.

It seems pretty cool, so I'd like to tell you what Leinster said -
using his terminology rather than Penon's (which of course is in
French).  To keep this short I'm going to assume you know a 
reasonable amount of category theory.

First of all, a "reflexive globular set" is a collection of sets and
functions like this:


$$

       <--s---       <--s---       <--s---
   X_{0}   ---i-->  X_{1}   ---i-->  X_{2}   ---i-->  .....
       <--t---       <--t---       <--t---
$$
    
going on to infinity, satisfying these equations:

s(s(x)) = s(t(x))
t(s(x)) = t(t(x)) 
s(i(x)) = t(i(x)) = x.

We call the elements of X_{n} are "n-cells", and call s(x) and t(x) the
"source" and "target" of the n-cell x, respectively.  If s(x) = a and
t(x) = b, we think of x as going from a to b, and write x: a \to  b.  

If we left out all the stuff about the maps i we would simply have a
"globular set".  These are important in n-category theory because
strict \omega -categories, and also Batanin's weak \omega -categories, are
globular sets with extra structure.   This also true of Penon's
definition, but he starts right away with "reflexive" globular sets,
which have these maps i that are a bit like the degeneracies in the
definition of a simplicial set (see "<A HREF = "week115.html">week115</A>").  In Penon's definition
i(x) plays the role of an "identity n-morphism", so we also write i(x)
as 1_{x}: x \to  x.

Let RGlob be the category of reflexive globular sets, where morphisms
are defined in the obvious way.  (In other words, RGlob is a presheaf
category - see "<A HREF = "week115.html">week115</A>" for an explanation of this notion.)

In this setup, the usual sort of strict \omega -category may be
defined as a reflexive globular set X together with various 
"composition" operations that allow us to compose n-cells x and y 
whenever t^j(x) = s^j(x), obtaining an n-cell 

x o_{j} y 

We get one such composition operation for each n and each j such 
that 1 <= j <= n.   We impose some obvious axioms of two
sorts:

A: axioms determining the source and target of a composite
B: strict associativity, unit and interchange laws

I'll assume you know these axioms or can fake it.  (If you
read the definition of strict 2-category in "<A HREF = "week80.html">week80</A>", perhaps
you can get an idea for what kinds of axioms I'm talking about.)

Now, strict \omega -categories are great, but we need to weaken this
notion.  So, first Penon defines an "\omega -magma" to be something
exactly like a strict \omega -category but without the axioms of type B.
You may recall that a "magma" is defined by Bourbaki to be a set with
a binary operation satisyfing no laws whatsoever - the primeval
algebraic object!  An \omega -magma is just as lawless, and a lot bigger
and meaner.

Strict \omega -categories are too strict: all laws hold as equations.
\Omega -magmas are too weak: no laws hold at all!  How do we get what
we want?  

We define a category Q whose objects are quadruples (M,p,C,[.,.])
where:

\begin{quote}
M is an \omega -magma 

C is a strict \omega -category

p: M \to  C is a morphism of \omega -magmas (i.e., a morphism of
reflexive globular sets strictly preserving all the \omega -magma
operations)

[.,.] is a way of lifting equations between n-morphisms in the 
image of the projection p to (n+1)-morphisms in M.  More precisely: 
given n-cells

f,g: a \to  b

in M such that p(f) = p(g), we have an (n+1)-cell

[f,g]: f \to  g

in M such that p([f,g]) = 1_{p(f)} = 1_{p(g)}.  
We require that [f,f] = 1_{f}.

\end{quote}
A morphism in Q is defined to be the obvious thing: a morphism
f: M \to  M' of \omega -magmas and a morphism f: C \to  C' of strict-\omega 
categories, strictly preserving all the structure in sight.

Okay, now we define a functor

U: Q \to  RGlob

by 

U(M,p,C,[.,.]) = M

where we think of M as just a reflexive globular set.  Penon
proves that U has a left adjoint

F: RGlob \to  Q

This adjunction defines a monad

T: RGlob \to  RGlob 

and Penon defines a "weak \omega -category" to be an algebra of this
monad.  

(See "<A HREF = "week92.html">week92</A>" and "<A HREF = "week118.html">week118</A>" for how you get monads from adjunctions.
Alas, I think I haven't gotten around to explaining the concept of an
algebra of a monad!  So much to explain, so little time!)

Now, if you know some category theory and think a while about this,
you will see that in a weak \omega -category defined this way, all
the laws like associativity hold \emph{up to equivalence}, with the
equivalences satisfying the necessary coherence laws *up to 
equivalence*, and so ad infinitum.  Crudely speaking, the 
lifting [.,.] is what turns equations into n-morphisms.  To get
a feeling for how this work, you have to figure out what the left
adjoint F looks like.  Penon works this out in detail in the second
half of his paper.  


 \par\noindent\rule{\textwidth}{0.4pt}

% </A>
% </A>
% </A>
