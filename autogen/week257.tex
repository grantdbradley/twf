
% </A>
% </A>
% </A>
\week{October 14, 2007 }

Time flies!  This week I'll finally finish saying what I did on 
my summer vacation.  After my trip to Oslo I stayed in London, 
or more precisely Greenwich.  While there, I talked with some good 
mathematicians and physicists: in particular, Minhyong Kim, Ray 
Streater, Andreas D&ouml;ring and Chris Isham.  I also went to a 
topology conference in Sheffield... and Eugenia Cheng explained
some cool stuff on the train ride there.  I want to tell you about 
all this before I forget.

Also, the Tale of Groupoidification has taken a shocking new
turn: it's now becoming available as a series of \emph{videos}.

But first, some miscellaneous fun stuff on math and astronomy.  

Math: if you haven't seen a sphere turn inside out, you've got 
to watch this classic movie, now available for free online:

1) The Geometry Center, Outside in,
<a href = "http://video.google.com/videoplay?docid=-6626464599825291409">http://video.google.com/videoplay?docid=-6626464599825291409</a>

Astronomy: did you ever wonder where dust comes from?  I'm 
not talking about dust bunnies under your bed - I'm talking 
about the dust cluttering our galaxy, which eventually clumps 
together to form planets and... you and me!

These days most dust comes from aging stars called <a href =
"http://www.daviddarling.info/encyclopedia/A/AGB.html">asymptotic
giant branch</a> stars: 

<div align = "center">
<a href = "http://www.noao.edu/outreach/press/pr03/sb0307.html">
<img style = "border:none;" src = "asymptotic_giant_branch.gif">
% </a>
</div>

The sun will eventually become one of these.
The story goes like this: first it'll keep burning until the hydrogen
in its core is exhausted.  Then it'll cool and become a <a href = "http://www.daviddarling.info/encyclopedia/R/redgiant.html">red giant</a>.
Eventually <a href = "http://www.daviddarling.info/encyclopedia/H/helium_flash.html">helium at the core will ignite</a>, and the Sun will heat up and
<a href = "http://www.daviddarling.info/encyclopedia/H/horizontal_branch.html">shrink again</a>... but its core will then become cluttered with even
heavier elements, so it'll cool and expand once more, moving onto the
"asymptotic giant branch".  At this point it'll have a layered
structure: heavier elements near the bottom, then a layer of helium,
then hydrogen on the top:

<div align = "center">
<a href = "http://www.noao.edu/outreach/press/pr03/sb0307.html">
<img style = "border:none;" src = "asymptotic_giant_branch_cutaway.gif">
% </a>
</div>

(A similar fate awaits any star between 0.6 and 10 solar masses,
though the details depend on the mass.  For the more dramatic
fate of heavier stars, see "<A HREF = "week204.html">week204</A>".)

Anyway: this layered structure is unstable, so asymptotic giant branch 
stars pulse every 10 to 100 thousand years or so.  And, they 
puff out dust!  Stellar wind then blows this dust out into space.  

A great example is the Red Rectangle:

<div align = "center">
<a href = "http://apod.nasa.gov/apod/ap040513.html">
<img style = "border:none;" src = "red_rectangle.jpg">
% </a>
</div>

2) Rungs of the Red Rectangle, Astronomy picture of the day, 
May 13, 2004, <a href = "http://apod.nasa.gov/apod/ap040513.html">http://apod.nasa.gov/apod/ap040513.html</a>

Here two stars 2300 light years from us are spinning around
each other while pumping out a huge torus of icy dust grains and
hydrocarbon molecules.  It's not really shaped like a rectangle 
or X - it just looks that way.  The scene is about 1/3 of a light 
year across.

Ciska Markwick-Kemper is an expert on dust.  She's an astrophysicist
at the University of Manchester.  Together with some coauthors, she
wrote a paper about the Red Rectangle:

3) F. Markwick-Kemper, J. D. Green, E. Peeters, Spitzer 
detections of new dust components in the outflow of the Red 
Rectangle, Astrophys. J. 628 (2005) L119-L122.  Also available
as <A HREF = "http://xxx.lanl.gov/abs/astro-ph/0506473">arXiv:astro-ph/0506473</A>.

They used the Spitzer Space Telescope - an infrared telescope on 
a satellite in earth orbit - to find evidence of magnesium and 
iron oxides in this dust cloud.  

But, what made dust in the early Universe?   It took about a
billion years after the Big Bang for asymptotic giant branch stars
to form.  But we know there was a lot of dust even before then!
We can see it in distant galaxies lit up by enormous black holes 
called "quasars", which pump out vast amounts of radiation as 
stuff falls into them.  

Markwick-Kemper and coauthors have also tackled that question:

4) F. Markwick-Kemper, S. C. Gallagher, D. C. Hines and J. Bouwman, 
Dust in the wind: crystalline silicates, corundum and periclase in 
PG 2112+059, Astrophys. J. 668 (2007), L107-L110. Also available
as <A HREF = "http://xxx.lanl.gov/abs/0710.2225">arXiv:0710.2225</A>.

They used spectroscopy to identify various kinds of dust in a distant
galaxy: a magnesium silicate that geologists call
"forsterite", a magnesium oxide called
"periclase&quot, and aluminum oxide, otherwise known as
"corundum" - you may have seen it on sandpaper.

And, they hypothesize that these dust grains were formed in the
hot wind emanating from the quasar at this galaxy's core!

So, besides being made of star dust, as in the Joni Mitchell
song, you also may contain a bit of black hole dust. 

Okay - now that we've got that settled, on to London!

Minhyong Kim is a friend I met back in 1986 when he was a grad 
student at Yale.  After dabbling in conformal field theory, he
became a student of Serge Lang and went into number theory.  He 
recently moved to England and started teaching at University 
College, London.  I met him there this summer, in front of the 
philosopher Jeremy Bentham, who had himself mummified and stuck
in a wooden cabinet near the school's entrance.

If you're not into number theory, maybe you should read this:

5) Minhyong Kim, Why everyone should know number theory,
available at <a href = "http://www.ucl.ac.uk/~ucahmki/numbers.pdf">http://www.ucl.ac.uk/~ucahmki/numbers.pdf</a>

Personally I never liked the subject until I realized it was
a form of \emph{geometry}.  For example, when we take an equation like
this

x^{2} + y^{3} = 1

and look at the real solutions, we get a curve in the plane - 
a "real curve".  If we look at the complex solutions, we get
something bigger.  People call it a "complex curve", because 
it's analogous to a real curve.  But topologically, it's 
2-dimensional.  

If we use polynomial equations with more variables, we get
higher-dimensional shapes called "algebraic varieties" -
either real or complex.  Either way, we can study these shapes using
geometry and topology.

But in number theory, we might study the solutions of these 
equations in some other number system - for example in Z/p, 
meaning the integers modulo some prime p.  At first glance there's 
no geometry involved anymore.  After all, there's just a <em>finite 
set</em> of solutions!  However, algebraic geometers have figured 
out how to apply ideas from geometry and topology, mimicking 
tricks that work for the real and complex numbers.  

All this is very fun and mind-blowing - especially when we reach
Grothendieck's idea of <a href =
"http://en.wikipedia.org/wiki/%C3%89tale_cohomology">&eacute;tale
topology</a>, developed around 1958.  This is a way of studying
"holes" in things like algebraic varieties over finite
fields.  Amazingly, it gives results that nicely match the results we
get for the corresponding complex algebraic varieties!  That's part of
what the <a href =
"http://en.wikipedia.org/wiki/Weil_conjectures">Weil conjectures</a>
say.


You can learn the details here:

6) J. S. Milne, Lectures on &Eacute;tale Cohomology, available at
<a href = "http://www.jmilne.org/math/CourseNotes/math732.html">http://www.jmilne.org/math/CourseNotes/math732.html</a>

Anyway, I quizzed about Minhyong about one of the big mysteries
that's been puzzling me lately.  I want to know why the integers 
resemble a 3-dimensional space - and how prime numbers give something
like "knots" in this space!  

I made a small step toward explaining this back in "<a href =
"week205.html">week205</a>".  There I sketched one of the basic
ideas of algebraic geometry: every commutative ring, for example the
integers or the integers modulo p, has a kind of space associated to
it, called its "spectrum".  We can think of elements of the
commutative ring as functions on this space.  I also explained why the
process turning a commutative ring into a space is "contravariant".
This implies that the obvious map from the integers to the integers
modulo p

Z \to  Z/p

gives rise to a map going \emph{the other way} between spectra:

Spec(Z/p) \to  Spec(Z)

In "<a href = "week218.html">week218</a>" I reviewed an
old argument saying that Spec(Z) is analogous to the complex
plane, and that Spec(Z/p) is analogous to a point.  
From this viewpoint, primes gives something like points in a plane.  

However, from a different viewpoint, primes give something like
circles in a 3d space!

The easy thing to see is how Spec(Z/p) acts more like a circle than a point.
In particular, its "&eacute;tale topology" resembles the topology of a
circle.  Oversimplifying a bit, the reason is that just as the circle
has one n-fold cover for each integer n > 0, so too does Spec(Z/p).
To get the n-fold cover of the circle, you just wrap it around itself
n times.  To get the n-fold cover of Spec(Z/p), we take the spectrum
of the field with p^{n} elements, which is called
F_{p^{n}}.  Z/p sits inside this larger field:

Z/p \to  F_{p^{n}}

so by the contravariance I mentioned, we get a map going the
other way:

Spec(F_{p^{n}}) \to  Spec(Z/p)

which is our n-fold cover.

I should explain this in much more detail someday - it involves
the relation between &eacute;tale cohomology, Galois theory and
covering spaces.  I began tackling this in 
"<a href = "week213.html">week213</a>", but I have a long
way to go.

Anyway, the basic idea here is that each prime p gives a "circle" 
Spec(Z/p) sitting inside Spec(Z).  
But the really nonobvious part is that according to &eacute;tale
cohomology, Spec(Z) is \emph{3-dimensional} - and the different circles
corresponding to different primes are \emph{linked!}

I've been fascinated by this ever since I heard about it, but I
got even more interested when I saw a draft of a paper by 
Kapranov and Smirnov.  I got it from Thomas Riepe, who got
it from Yuri Manin.  There's a version on the web:

7) M. Kapranov and A. Smirnov, Cohomology determinants and
reciprocity laws: number field case, available at
<a href = "http://wwwhomes.uni-bielefeld.de/triepe/F1.html">http://wwwhomes.uni-bielefeld.de/triepe/F1.html</a>

It begins:

\begin{quote}
     The analogies between number fields and function fields
     have been a long-time source of inspiration in arithmetic.
     However, one of the most intriguing problems in this
     approach, namely the problem of the absolute point, is
     still far from being satisfactorialy understood.  The
     scheme Spec(Z), the final object in the category of schemes,
     has dimension 1 with respect to the Zariski topology
     and at least 3 with respect to the etale topology.  This
     has generated a long-standing desire to introduce a more
     mythical object P, the "absolute point", with a natural
     morphism X \to  P given for any arithmetic scheme X [...]
\end{quote}

Even though I don't fully understand this, I can tell something
big is afoot here.  I think they're saying that
because Spec(Z) is so big and fancy from the viewpoint of
&eacute;tale topology, there should be some mysterious kind of "point"
that's much smaller than Spec(Z) - the "absolute point".

Anyway, in this paper the authors explain how the <a href =
"http://en.wikipedia.org/wiki/Legendre_symbol">Legendre symbol</a> of
primes is analogous to the <a href =
"http://en.wikipedia.org/wiki/Linking_number">linking number</a> of
knots.

The Legendre symbol depends on two primes: it's 1 or -1 depending 
on whether or not the first is a square modulo the second.  The 
linking number depends on two knots: it says how many times the 
first winds around the second.

The linking number stays the same when you switch the two knots.  
The Legendre symbol has a subtler symmetry when you switch the 
two primes: this symmetry is called <a href = "http://golem.ph.utexas.edu/category/2007/06/quadratic_reciprocity.html">quadratic reciprocity</a>, and 
it has lots of proofs, starting with a bunch by Gauss - all a bit 
tricky.  

I'd feel very happy if I truly understood why quadratic reciprocity 
reduces to the symmetry of the linking number when we think of 
primes as analogous to knots.  Unfortunately, I'll need to think a 
lot more before I really get the idea.  I got into number theory 
late in life, so I'm pretty slow at it.  

This paper studies subtler ways in which primes can be "linked":

8) Masanori Morishita, Milnor invariants and Massey products for 
prime numbers, Compositio Mathematica 140 (2004), 69-83.

You may know the <a href =
"http://en.wikipedia.org/wiki/Borromean_rings">Borromean rings</a>, a
design where no two rings are linked in isolation, but all three are
when taken together.  Here the linking numbers are zero, but the
linking can be detected by something called the "Massey triple
product".  Morishita generalizes this to primes!

But I want to understand the basics...

The secret 3-dimensional nature of the integers and certain other 
"rings of algebraic integers" seems to go back at least to the work 
of Artin and Verdier:

9) Michael Artin and Jean-Louis Verdier, Seminar on &eacute;tale cohomology 
of number fields, Woods Hole, 1964. 

You can see it clearly here, starting in section 2:

10) Barry Mazur, Notes on the &eacute;tale cohomology of number fields,
Annales Scientifiques de l'Ecole Normale Superieure Ser. 4, 
6 (1973), 521-552.  Also available at
<a href = "http://www.numdam.org/numdam-bin/fitem?id=ASENS_1973_4_6_4_521_0">http://www.numdam.org/numdam-bin/fitem?id=ASENS_1973_4_6_4_521_0</a>

By now, a big "dictionary" relating knots to primes has been 
developed by Kapranov, Mazur, Morishita, and Reznikov.  This 
seems like a readable introduction:

11) Adam S. Sikora, Analogies between group actions on 3-manifolds
and number fields, available as <a href = "http://arxiv.org/abs/math/0107210">arXiv:math/0107210</a>.

I need to study it.  These might also be good - I haven't looked
at them yet:

12) Masanori Morishita, On certain analogies between knots and 
primes, J. Reine Angew. Math. 550 (2002), 141-167.

Masanori Morishita, On analogies between knots and primes, 
Sugaku 58 (2006), 40-63.

After giving a talk on 2-Hilbert spaces at University College,
I went to dinner with Minhyong and some folks including Ray 
Streater.
Ray Streater and Arthur Wightman wrote the book "PCT, Spin, 
Statistics and All That".  Like almost every mathematician who 
has seriously tried to understand quantum field theory, I've 
learned a lot from this book.  So, it was fun meeting Streater, 
talking with him - and finding out he'd once been made an honorary 
colonel of the US Army to get a free plane trip to the Rochester 
Conference!  This was a big important particle physics conference, 
back in the good old days.

He also described Geoffrey Chew's Rochester conference talk on the 
analytic S-matrix, given at the height of the <a href = "http://en.wikipedia.org/wiki/Bootstrap_model">bootstrap model</a> fad. 
Wightman asked Chew: why assume from the start that the S-matrix was 
analytic?  Why not try to derive it from simpler principles?  Chew 
replied that "everything in physics is smooth".  Wightman asked about
smooth functions that aren't analytic.  Chew thought a moment and 
replied that there weren't any.

Ha-ha-ha...

What's the joke?   Well, first of all, Wightman had already succeeded
in deriving the analyticity of the S-matrix from simpler principles. 
Second, any good mathematician - but not necessarily every physicist, 
like Chew - will know examples of smooth functions that aren't 
analytic. 

Anyway, Streater has just finished an interesting book on "lost 
causes" in physics: ideas that sounded good, but never panned out.  
Of course it's hard to know when a cause is truly lost.  But a 
good pragmatic definition of a lost cause in physics is a topic 
that shouldn't be given as a thesis problem.  

So, if you're a physics grad student and some professor wants you to 
work on hidden variable theories, or octonionic quantum mechanics, 
or deriving laws of physics from Fisher information, you'd better 
read this:

13) Ray F. Streater, Lost Causes in and Beyond Physics, Springer 
Verlag, Berlin, 2007.

(I like octonions - but I agree with Streater about not inflicting 
them on physics grad students!  Even though all my students are in 
the math department, I still wouldn't want them working \emph{mainly} on 
something like that.  There's a lot of more general, clearly useful 
stuff that students should learn.) 

I also spoke to Andreas D&ouml;ring and Chris Isham about their work 
on topos theory and quantum physics.  Andreas D&ouml;ring lives near
Greenwich, while Isham lives across the Thames in London proper.
So, I talked to D&ouml;ring a couple times, and once we visited Isham
at his house.

I mainly mention this because Isham is one of the gurus of quantum
gravity, profoundly interested in philosophy... so I was surprised,
at the end of our talk, when he showed me into a room with a huge 
rack of computers hooked up to a bank of about 8 video monitors,
and controls reminiscent of an airplane cockpit.

It turned out to be his homemade flight simulator!  He's been a 
hobbyist electrical engineer for years - the kind of guy who 
loves nothing more than a soldering iron in his hand.  He'd just 
gotten a big 750-watt power supply, since he'd blown out his 
previous one.  

Anyway, he and D&ouml;ring have just come out with a series of papers:

14) Andreas D&ouml;ring and Christopher Isham, A topos foundation 
for theories of physics: I. Formal languages for physics, 
available as <A HREF = "http://xxx.lanl.gov/abs/quant-ph/0703060">arXiv:quant-ph/0703060</A>.

II. Daseinisation and the liberation of quantum theory, 
available as <A HREF = "http://xxx.lanl.gov/abs/quant-ph/0703062">arXiv:quant-ph/0703062</A>.

III.  The representation of physical quantities with arrows,
available as <A HREF = "http://xxx.lanl.gov/abs/quant-ph/0703064">arXiv:quant-ph/0703064</A>.

IV. Categories of systems, available as <A HREF = "http://xxx.lanl.gov/abs/quant-ph/0703066">arXiv:quant-ph/0703066</A>.

Though they probably don't think of it this way, you can think 
of their work as making precise Bohr's ideas on seeing the quantum
world through classical eyes.  Instead of talking about all
observables at once, they consider collections of observables that
you can measure simultaneously without the uncertainty principle
kicking in.  These collections are called "commutative subalgebras". 

You can think of a commutative subalgebra as a classical snapshot
of the full quantum reality.  Each snapshot only shows part of the
reality.  One might show an electron's position; another might show
it's momentum.

Some commutative subalgebras contain others, just like some open 
sets of a topological space contain others.  The analogy is a good 
one, except there's no one commutative subalgebra that contains
\emph{all} the others.  

Topos theory is a kind of "local" version of logic, but where the 
concept of locality goes way beyond the ordinary notion from 
topology.  In topology, we say a property makes sense "locally" 
if it makes sense for points in some particular open set.
In the D&ouml;ring-Isham setup, a property makes sense "locally" if
it makes sense "within a particular classical snapshot of reality" -
that is, relative to a particular commutative subalgebra.

(Speaking of topology and its generalizations, this work on topoi and
physics is related to the "&eacute;tale topology" idea I
mentioned a while back - but technically it's much simpler.  The
<a href = "http://en.wikipedia.org/wiki/Grothendieck_topology">&eacute;tale 
topology</a> lets you define a topos of <a href = "http://en.wikipedia.org/wiki/Grothendieck_topology#Sites_and_sheaves">sheaves</a>
on a certain category.  The D&ouml;ring-Isham work just uses the topos of
<a href = "http://en.wikipedia.org/wiki/Presheaf_(category_theory)">presheaves</a> on the poset of commutative subalgebras.  Trust
me - while this may sound scary, it's much easier.)

D&ouml;ring and Isham set up a whole program for doing physics 
"within a topos", based on existing ideas on how to do math in 
a topos.  You can do vast amounts of math inside any topos just 
as if you were in the ordinary world of set theory - but using 
intuitionistic logic instead of classical logic.  Intuitionistic
logic denies the principle of excluded middle, namely:

<div align = "center">
"For any statement P, either P is true or not(P) is true."
</div>

In D&ouml;ring and Isham's setup, if you pick a commutative subalgebra 
that contains the position of an electron as one of its observables,
it can't contain the electron's momentum.  That's because these
observables don't commute: you can't measure them both simultaneously.
So, working "locally" - that is, relative to this particular 
subalgebra - the statement

<div align = "center">
P = "the momentum of the electron is zero"
</div>

is neither true nor false!  It's just not defined.

Their work has inspired this very nice paper:

15) Chris Heunen and Bas Spitters, A topos for algebraic quantum
theory, available as <a href = "http://arxiv.org/abs/0709.4364">arXiv:0709.4364.</a>

so let me explain that too.

I said you can do a lot of math inside a topos.  In particular, 
you can define an algebra of observables - or technically, a
"C*-algebra".

By the Isham-D&ouml;ring work I just sketched, any C*-algebra of 
observables gives a topos.  Heunen and Spitters show that 
the original C*-algebra gives rise to a
C*-algebra \emph{in this topos},
which is \emph{commutative} even if the original one was 
noncommutative!  That actually makes sense, since in this setup each 
"local view" of the full quantum reality is classical.  

So, they get this sort of picture:

<div align = "center">
<a href = "http://arxiv.org/abs/0709.4364">
<img style = "border:none;" src = "heunen_spitters.jpg">
% </a>
</div>

I've been taking the "ambient topos" to be the familiar category
of sets, but it could be something else.  

What's 
really neat is that the Gelfand-Naimark theorem, saying commutative 
C*-algebras are always algebras of continuous functions on compact
Hausdorff spaces, can be generalized to work within any topos.
So, we get a space \emph{in our topos} such that observables of the 
C*-algebra \emph{in the topos} are just functions on this space.  

I know this sounds technical if you're not into this stuff.  But
it's really quite wonderful.  It basically means this: using topos 
logic, we can talk about a classical space of states for a quantum 
system!  However, this space typically has "no global points"
- that's called the "Kochen-Specker theorem".  In 
other words, there's no overall classical reality that matches all 
the classical snapshots.  

As you can probably tell, category theory is gradually seeping
into this post, though I've been doing my best to keep it
hidden.  Now I want to say what Eugenia Cheng explained on 
that train to Sheffield.  But at this point, I'll break down and
assume you know some category theory - for example, monads.

If you don't know about monads, never fear!  I defined them in 
"<A HREF = "week89.html">week89</A>", and studied them using string diagrams in "<A HREF = "week92.html">week92</A>". 
Even better, Eugenia Cheng and Simon Willerton have formed a 
little group called the Catsters - and under this name, they've 
put some videos about monads and string diagrams onto YouTube!  
This is a really great new use of technology.  So, you should 
also watch these:

16) The Catsters, Monads, 
<a href = "http://youtube.com/view_play_list?p=0E91279846EC843E">http://youtube.com/view_play_list?p=0E91279846EC843E</a>

The Catsters, Adjunctions, 
<a href = "http://youtube.com/view_play_list?p=54B49729E5102248">http://youtube.com/view_play_list?p=54B49729E5102248</a>

The Catsters, String diagrams, monads and adjunctions,
<a href = "http://youtube.com/view_play_list?p=50ABC4792BD0A086">http://youtube.com/view_play_list?p=50ABC4792BD0A086</a>

A very famous monad is the "free abelian group" monad

F: Set \to  Set

which eats any set X and spits out the free abelian group on X, 
say F(X).   A guy in F(X) is just a formal linear combination
of guys in X, with integer coefficients.

Another famous monad is the "free monoid" monad 

G: Set \to  Set

This eats any set X and spits out the free monoid on X, namely 
G(X).  A guy in G(X) is just a formal product of guys in X.

Now, there's yet another famous monad, called the "free 
ring" monad, which eats any set X and spits out the free ring on
this set.  But, it's easy to see that this is just F(G(X))!
After all, F(G(X)) consists of formal linear combinations of
formal products of guys in X.  But that's precisely what you find
in the free ring on X.  

But why is FG a monad?  There's more to a monad than just a 
functor.  A monad is really a kind of \emph{monoid} in the world of
functors from our category (here Set) to itself.  In particular, 
since F is a monad, it comes with a natural transformation called
a "multiplication":

m: FF => F

which sends formal linear combinations of formal linear combinations
to formal linear combinations, in the obvious way.  Similarly,
since G is a monad, it comes with a natural transformation

n: GG => G

sending formal products of formal products to formal products.
But how does FG get to be a monad?  For this, we need some 
natural transformation from FGFG to FG!

There's an obvious thing to try, namely


$$

                    mn 
FGFG ======> FFGG ======> FG
$$
    
where in the first step we switch G and F somehow, and in the
second step we use m and n.  But, how do we do the first step?

We need a natural transformation

d: GF => FG

which sends formal products of formal linear combinations
to formal linear combinations of formal products.  Such a
thing obviously exists; for example, it sends

(x + 2y)(x - 3z) 

to

xx + 2yx - 3xz - 6yz

It's just the distributive law!  

Quite generally, to make the composite of monads F and 
G into a new monad FG, we need something that people call a
"distributive law", which is a natural transformation

d: GF => FG

This must satisfy some equations - but you can work out
those yourself.  For example, you can demand that


$$

       FdG          mn 
FGFG ======> FFGG ======> FG

$$
    

make FG into a monad, and see what that requires.  (Besides the
"multiplication" in our monad, we also need the
"unit", so you should also think about that - I'm ignoring
it here because it's less sexy than the multiplication, but it's
equally essential.)

However: all this becomes more fun with string diagrams!  As the
Catsters explain, and I explained in "<A HREF =
"week89.html">week89</A>", the multiplication m: FF => F can
be drawn like this:


\begin{verbatim}

                     \               /
                      \             /
                      F\          F/
                        \         /
                         \       /
                          \     /
                           \   /
                            \ /
                             |m               
                             |
                             |
                             |
                             |
                             |
                            F|
                             |
\end{verbatim}
    
And, it has to satisfy the associative law, which says we
get the same answer either way when we multiply three things:


\begin{verbatim}

             \      /        /        \        \      /
              \    /        /          \        \    /
              F\  /F      F/           F\       F\  /F
                \/        /              \        \/
                m\       /                \       /m 
                  \     /                  \     /
                  F\   /                    \   /F
                    \ /                      \ /
                     |m                       |m
                     |                        |
                     |            =           |
                     |                        |
                     |                        |
                     |                        |
                    F|                       F|
                     |                        |
\end{verbatim}
    
The multiplication n: GG => G looks similar to m, and it too has
to satisfy the associative law.   

How do we draw the distributive law d: FG => GF?  Since it's a 
process of switching two things, we draw it as a \emph{braiding}:


\begin{verbatim}

              F\   /G
                \ /
                 / 
                / \
              G/   \F 
\end{verbatim}
    
I hope you see how incredibly cool this is: the good old 
distributive law is now a \emph{braiding}, which pushes our diagrams
into the third dimension!  

Given this, let's draw the multiplication for our would-be
monad FG, namely 


$$

       FdG          mn 
FGFG ======> FFGG ======> FG
$$
    
It looks like this:


\begin{verbatim}

                     \   \           /   /
                      \   \         /   /
                      F\  G\      F/   /G
                        \   \     /   /
                         \   \   /   /
                          \   \ /   /
                           \   /   /
                            \ / \ /
                             |m  |n             
                             |   |
                             |   |
                             |   |
                             |   |
                             |   |
                            F|   |G
                             |   |
\end{verbatim}
    
Now, we want \emph{this} multiplication to be associative!  So, 
we need to draw an equation like this:


\begin{verbatim}

             \      /        /        \        \      /
              \    /        /          \        \    /
               \  /        /            \        \  /
                \/        /              \        \/
                 \       /                \       / 
                  \     /                  \     /
                   \   /                    \   /
                    \ /                      \ /
                     |                        |
                     |                        |
                     |            =           |
                     |                        |
                     |                        |
                     |                        |
                     |                        |
                     |                        | 
\end{verbatim}
    
but with the strands \emph{doubled}, as above - I'm too lazy
to draw this here.  And then we need to find some nice conditions
that make this associative law true.  Clearly we should use the
associative laws for m and n, but the "braiding" - the distributive
law d: FG => GF - also gets into the act.

I'll leave this as a pleasant exercise in string diagram 
manipulation.  If you get stuck, you can peek in the back of 
the book:

17) Wikipedia, Distibutive law between monads, 
<a href = "http://en.wikipedia.org/wiki/Distributive_law_between_monads">http://en.wikipedia.org/wiki/Distributive_law_between_monads</a>

The two scary commutative rectangles on this page are the 
"nice conditions" you need.    They look nicer as 
string diagrams.  One looks like this:


\begin{verbatim}

         F\    G\   /G             F\    G/    /G
           \     \ /                 \   /    /
            \     |n                  \ /    /
             \   /                     /    /
              \ /             =       / \  /
               /                     /    /
              / \                   /    /  
             /   \                  \   /  \
            /     \                  \ /    \
          G/       \F                 |n     \F
          /         \                G|       \

\end{verbatim}
    
In words: 

<div align = "center">
 "multiply two G's and slide the result over an F" = <br/>
 "slide both the G's over the F and then multiply them"
</div>

If the pictures were made of actual string, this would be obvious!

The other condition is very similar.  I'm too lazy to draw it,
but it says 

<div align = "center">
 "multiply two F's and slide the result under a G" = <br/>
 "slide both the F's under a G and then multiply them"
</div>

All this is very nice, and it goes back to a paper by Beck:

18) Jon Beck, Distributive laws, Lecture Notes in Mathematics 
80, Springer, Berlin, 1969, pp. 119-140. 

This isn't what Eugenia explained to me, though - I already knew
this stuff.  She started out by explaining something in a paper 
by Street:

19) Ross Street, The formal theory of monads, J. Pure Appl. Alg.
2 (1972), 149-168.

which is reviewed at the beginning here:

20) Steve Lack and Ross Street, The formal theory of monads II,
J. Pure Appl. Alg. 175 (2002), 243-265.  Also available at
<a href = "http://www.maths.usyd.edu.au/u/stevel/papers/ftm2.html">http://www.maths.usyd.edu.au/u/stevel/papers/ftm2.html</a>

(Check out the cool string diagrams near the end!)  

Street noted that we can talk about monads, not just in the
2-category of categories, but in any 2-category.  I actually 
explained monads at this level of generality back in 
"<a href = "week89.html#tale">week89</a>".
Indeed, for any 2-category C, there's a 2-category Mnd(C)
of monads in C.  

And, he noted that a monad in Mnd(C) is a pair of monads in C
related by a distributive law!

That's already mindbogglingly beautiful.  According to Eugenia,
it's practically the last sentence in Street's paper.  But in her new work:

21) Eugenia Cheng, Iterated distributive laws, available as
<a href = "http://arxiv.org/abs/0710.1120">arXiv:0710.1120</a>.

she goes a bit further: she considers monads in Mnd(Mnd(C)), 
and so on.   Here's the punchline, at least for today: she shows 
that a monad in Mnd(Mnd(C)) is a triple of monads F, G, H related 
by distributive laws satisfying the Yang-Baxter equation:
  

\begin{verbatim}

             F\  G/   |H     F|  G\   /H
               \ /    |       |    \ /
                /     |       |     /
               / \    |       |    / \
              /   \   |       \   /   \
             |     \ /         \ /     |
             |      /     =     /      | 
             |     / \         / \     |
             |    /   \       /   \    |
             \   /    |       |    \   /
              \ /     |       |     \ /
               /      |       |      /
              / \     |       |     / \
            H/   \G   |F     H|   G/   \F
\end{verbatim}
    

This is also just what you need to make the composite FGH
into a monad!

By the way, the pathetic piece of ASCII art above is lifted 
from "<A HREF = "week1.html">week1</A>", where I first 
explained the Yang-Baxter equation.
That was back in 1993.  So, it's only taken me 14 years to learn
that you can derive this equation from considering monads in
the category of monads in the category of monads in a 2-category.

You may wonder if this counts as progress - but Eugenia
studies lots of \emph{examples} of this sort of thing, so it's far
from pointless.  

Okay... finally, the Tale of Groupoidification.  I'm a bit tired
now, so instead of telling you more of the tale, let me just say
the big news.

Starting this fall, James Dolan and I are running a seminar on
geometric representation theory, which will discuss:

<ul>
<li>
 Actions and representations of groups, especially symmetric groups
</li><li>
 Hecke algebras and Hecke operators
</li><li>
 Young diagrams
</li><li>
 Schubert cells for flag varieties
</li><li>
 q-deformation 
</li><li>
 Spans of groupoids and groupoidification
</li>
</ul>

This is the Tale of Groupoidification in another guise.

Moreover, the Catsters have inspired me to make videos of this 
seminar!  You can already find some here, along with course 
notes and blog entries where you can ask questions and talk about 
the material:

22) John Baez and James Dolan, Geometric representation theory seminar,
<a href = "http://math.ucr.edu/home/baez/qg-fall2007/">http://math.ucr.edu/home/baez/qg-fall2007/</a>

More will show up in due course.  I hope you join the fun.

\par\noindent\rule{\textwidth}{0.4pt}
\textbf{Addenda:} I thank Eugenia Cheng for some corrections.
Thomas Larsson points out that you can find some of Streater's 
"lost causes in physics" online:

23) Ray F. Streater, Various causes in physics and elsewhere,
<a href = "http://www.mth.kcl.ac.uk/~streater/causes.html">http://www.mth.kcl.ac.uk/~streater/causes.html</a>

For the proof of the Gelfand-Naimark theorem inside a topos, see:

24) Bernhard Banachewski and Christopher J. Mulvey, A globalisation
of the Gelfand duality theorem, Ann. Pure Appl. Logic 137 (2006), 
62-103.  Also available at 
<a href = "http://www.maths.sussex.ac.uk/Staff/CJM/research/pdf/globgelf.pdf">
http://www.maths.sussex.ac.uk/Staff/CJM/research/pdf/globgelf.pdf</a>

They show that any commutative C*-algebra A in a Grothendieck topos is
canonically isomorphic to the C*-algebra of continuous 
complex functions on the compact, completely regular locale that is its
maximal spectrum (that is, the space of homomorphisms f: A \to  C).
Conversely, they show any compact completely regular locale X gives
a commutative C*-algebra consisting of continuous complex functions
on X.  Even better, they explain what all this stuff means.

Jordan Ellenberg sent me the following comments about knots and primes:

\begin{quote}

<ol>
<li>
In the viewpoint of Deninger, very badly
oversimplified, Spec Z is to be thought of not just as a
3-manifold but as a 3-manifold with
a flow, in which the primes are not just knots, but are precisely the
closed orbits of the flow!
</li>
<li>
One thing to keep in mind about the analogy is that "the complement of
a knot or link in a 3-manifold" and "the complement of a prime or composite
integer in Spec Z" (which is to say Spec Z[1/N])
are both "things which have fundamental groups," thanks to Grothendieck in
the latter case.  And much of the concrete part of the analogy (like the
stuff about linking numbers) follows from this
fact.
</li>
<li>
On a similar note, a recent paper of Dunfield and Thurston which
I like a lot, "Finite covers
of random 3-manifolds," develops a model of "random 3-manifold" and
shows that the behavior of the first
homology of a random 3-manifold mod p is exactly the same as the
\emph{predicted} behavior of the mod p class group of a random number
field under the Cohen--Lenstra heuristics.  In other words, you should
not think of Spec Z or Spec Z[1/N] as being anything
like a \emph{particular} 3-manifold -- better to think of the class of
3-manifolds as being like the class of number fields.
</li>
</ol>

\end{quote}

Here's one of Deninger's papers:

25) Christopher Deninger, Number theory and dynamical systems on
foliated spaces, available as <a href = "http://arxiv.org/abs/math/0204110">arXiv:math/0204110</a>.

And here's the paper by Dunfield and Thurston:

26) Nathan M. Dunfield and William P. Thurston, Finite covers of
random 3-manifolds, available as <a href = "http://arxiv.org/abs/math/0502567">arXiv:math/0502567</a>.

On the \emph{n}-Category Caf&eacute;, a number theorist named James
corrected some serious mistakes in the original version of this Week's
Finds.  Here are his remarks on why Spec(Z) is 3-dimensional:

\begin{quote}

   So then why should there be the two dimensions of primes needed 
   to make Spec(Z) three-dimensional?  I don't think there is a 
   pure-thought answer to this question.  As you wrote, there is a 
   scientific answer in terms of Artin-Verdier duality, which is
   pretty much the same as class field theory.  There is also a
   pure-thought answer to an analogous question.  Let me try to 
   explain that.

   Instead of considering Z, let's consider F[x], where F is a finite 
   field.  They are both principal ideal domains with finite residue
   fields, and this makes them behave very similarly, even on a deep
   level.  I'll explain why F[x] is three-dimensional, and then by
   analogy we can hope Z is, too.  Now F[x] is an F-algebra.  In 
   other words, X = Spec(F[x]) is a space mapping to S = Spec(F).  
   I already explained why S is a circle from the point of view 
   of the &eacute;tale topology.  So, if X is supposed to be three-dimensional, 
   the fibers of this map better be two-dimensional.  What are the 
   fibers of this map?  Well, what are the points of S?  A point 
   in the &eacute;tale topology is Spec of some field with a trivial absolute 
   Galois group, or in other words, an algebraically closed field 
   (even better, a separably closed one).   Therefore a &eacute;tale point 
   of S is the same thing as Spec of an algebraic closure F^{&ndash;}
   of F.  
   What then is the fiber of X over this point?  It's Spec of the 
   ring F^{&ndash;}[x].  Now, \emph{this} is just the affine line over an 
   algebraically closed field, so we can figure out its cohomological 
   dimension.  The affine line over the complex numbers, another 
   algebraically closed field, is a plane and therefore has cohomological 
   dimension 2.  Since &eacute;tale cohomology is kind of the same as usual 
   singular cohomology, the &eacute;tale cohomological dimension of 
   Spec(F^{&ndash;}[x]) ought to be 2.

   Therefore X looks like a 3-manifold fibered in 2-manifolds over
   Spec(F), which looks like a circle.  Back to Spec(Z), we 
   analogously expect it to look like a 3-manifold, but absent a
   (non-formal) theory of the field with one element, Z is not
   an algebra over anything.  Therefore we expect Spec(Z) to
   be a 3-manifold, but not fibered over anything.
\end{quote}

For more discussion, go to the
<a href = "http://golem.ph.utexas.edu/category/2007/10/this_weeks_finds_in_mathematic_18.html">\emph{n}-Category Caf&eacute;</a>.







\par\noindent\rule{\textwidth}{0.4pt}
<em>It is a glorious feeling to discover the unity of a set of phenomena
that at first seem completely separate.</em> - Albert Einstein

\par\noindent\rule{\textwidth}{0.4pt}

% </A>
% </A>
% </A>
