
% </A>
% </A>
% </A>
\week{August 20, 1996}



Let me continue summarizing what happened during July at the
Mathematical Problems of Quantum Gravity workshop in Vienna.  The
first two weeks concentrated on the foundations of the loop
representation of quantum gravity; the next week was all about black
holes!

Tuesday, July 16th - Ted Jacobson gave an overview of "Issues of Black
Hole Thermodynamics".  There is a lot to say about this subject and I
won't try to repeat his marvelous talk here.  Let me just mention a
very interesting technical point he made.  The original
Bekenstein-Hawking formula for the entropy of a black hole is

S = A/(4 \hbar  G)

where A is the area of the event horizon, \hbar  is Planck's constant,
and G is Newton's constant.  One way to try to derive this is from the
partition function of a quantum field theory involving gravity and
other fields.  Jacobson sketched a heuristic calculation along these
lines.  When you do this calculation it's natural to worry why the
other fields, representing various forms of matter, don't seem to
contribute to the answer above.  Also, when we do quantum field
theory, there is often a difference between the "bare" coupling
constants we put into the theory and the "renormalized" coupling
constants that are what the theory predicts we'll observe
experimentally.  So it's natural to worry about whether it's the bare
or renormalized Newton's constant G that enters the above formula ---
even though quantum gravity is so unlike most other quantum field
theories that it's unclear that this worry makes sense, ultimately.

Anyway, the nice thing is that these two worries cancel each other
out.  In other words: yes, it's the renormalized Newton's constant G
--- the physically measured one --- that enters the above formula.
But at least to first order in \hbar , the difference between the bare G
and the renormalized G is precisely due to the interactions between
gravity and the matter fields (including the self-interaction of the
gravitational field).  In other words, the matter fields really \emph{do}
contribute to the black hole entropy, but this contribution is
absorbed into the definition of the renormalized G.

In the most extreme case, the bare 1/G is zero, and the renormalized
1/G is entirely due to interactions between matter and gravity.  This
is Andrei Sakharov's theory of "induced gravity".  According to
Jacobson, in this case all of the black hole entropy is "entanglement
entropy" --- this being standard jargon for the way that two parts of
a quantum system can each have entropy due to correlations, even
though the whole system has zero entropy.  Unfortunately my notes do
not allow me to reconstruct the wonderful argument whereby he showed
this.  (See "<A HREF = "week27.html">week27</A>" for a more detailed explanation of entanglement
entropy.)

Wednesday July 17th - There was a talk on "Colombeau theory" by a
mathematician whose name I unfortunately failed to catch.  Colombeau
theory is a theory that allows you to multiply distributions, just
like they said in school that you weren't allowed to do.  So if for
example you want to square the Dirac delta function, you can do it in
the context of Colombeau theory.  There has been a certain amount of
debate, however, on whether Colombeau theory allows you to this
multiplication in a \emph{useful} way.  There were a lot of physicists at
this talk who would be willing and eager to master Colombeau theory if
it let one solve the physics problems they wanted to solve.  However,
after much discussion, it appears that they didn't buy it.  I believe
that at best Colombeau theory provides a useful framework for
understanding the ambiguities one encounters when multiplying
distributions.

I say "ambiguities" rather than "disasters" because while the square
of the Dirac delta function has no sensible interpretation as a
distribution, there are many cases, such as when you try to multiply
the Dirac delta function and the Heaviside function, where you can
interpret the result as a distribution in a variety of ways.  These
ambiguous cases are the ones of greatest interest in physics.  A nice
place to see this in quantum field theory is in

1) G. Scharf, Finite quantum electrodynamics: the causal approach, 
Springer-Verlag, Berlin, 1995.  

If you want to learn about Colombeau theory you can try:

2) J. F. Colombeau, "Multiplication of Distributions: a Tool in
Mathematics, Numerical Engineering, and Theoretical Physics," Lecture
Notes in Mathematics 1532, Springer, Berlin, 1992.

Later that day I had nice conversation with Jerzy Lewandowski
on the approach to the loop representation where one uses smooth,
rather than analytic, loops.  (See "<A HREF = "week55.html">week55</A>" for more on this issue.)

Thursday, July 18th - Carlo Rovelli spoke on "Black Hole Entropy",
reporting some work he did with Kirill Krasnov.  They have a nice
approach to computing the black hole entropy using the loop
representation of quantum gravity.  A common goal among quantum
gravity folks is to recover the Bekenstein-Hawking formula from some
full-fledged theory of quantum gravity --- the original derivation
being a curious "semiclassical" hybrid of quantum and classical
reasoning.  In a statistical mechanical approach, entropy should be
the logarithm of the number of microstates some system can have in a
given macrostate.  So one wants to count states somehow.  Basically
what Rovelli and Krasnov do is count the number of ways a surface can
be pierced by a spin network so as to give it a certain area.  (This
uses the formula for the area operator I descrbed in "<A HREF = "week86.html">week86</A>".)  They
get an entropy proportional to the area, but not with the same constant
as in the Bekenstein-Hawking formula. 

There were some hopes that taking matter fields into account might
give the right constant.  But since everyone had been to Ted
Jacobson's talk, this led to much interesting wrangling over whether
Rovelli and Krasnov were using the bare or renormalized Newton's
constant G, and whether the concept of bare and renormalized G even
makes sense, ultimately!  Also, there are some extremely important
puzzles about what the right way to count states is, in these loop
representation computations.

For more, try:

3) Carlo Rovelli, Loop quantum gravity and black hole physics, preprint
available as <A HREF = "http://xxx.lanl.gov/abs/gr-qc/9608032">gr-qc/9608032</A>.  

Kirill Krasnov, The Bekenstein bound and non-perturbative quantum gravity,
preprint available as <A HREF = "http://xxx.lanl.gov/abs/gr-qc/9603025">gr-qc/9603025</A>.

Kirill Krasnov, On statistical mechanics of gravitational systems, 
preprint available as <A HREF = "http://xxx.lanl.gov/abs/gr-qc/9605047">gr-qc/9605047</A>.

Friday, July 19th - Don Marolf spoke on "Black hole entropy in string
theory".  He attempted valiantly to describe the string-theoretic
approach to computing black hole entropy to an audience only generally
familiar with string theory.  I will not try to summarize his talk,
except to note that he mainly discussed the case of a black hole in 5
dimensions, which was really a "black string" in 6 dimensions --- a
solution with translational symmetry in the 6th dimension, but where
the extra 6th dimension is so tiny that ordinary 5-dimensional folks
think they've just got a black hole.  (By the way, even the
6-dimensional approach is really just a way of talking about a string
theory that fundamentally lives in 10 dimensions.  This stuff is not
for the faint-hearted.)

Here are a few papers on this subject by Marolf and Horowitz:

4) Gary Horowitz, The origin of black hole entropy in string theory, 
preprint available as <A HREF = "http://xxx.lanl.gov/abs/gr-qc/9604051">gr-qc/9604051</A>.

Gary T. Horowitz and Donald Marolf, Counting states of black strings with 
traveling waves, preprint available as <A HREF = "http://xxx.lanl.gov/abs/hep-th/9605224">hep-th/9605224</A>.

Gary T. Horowitz and Donald Marolf, Counting states of black strings with 
traveling waves II, preprint available as <A HREF = "http://xxx.lanl.gov/abs/hep-th/9606113">hep-th/9606113</A>.

Monday, July 22nd - Kirill Krasnov spoke on "The Einstein-Maxwell
Theory of Black Hole Entropy".  This was a report on attempts to see how
his calculations of the black entropy in the loop representation changed
when he took the electromagnetic field into account.  The calculations
were very tentative, for certain technical reasons I won't go into here,
but they made even clearer the importance of the issue of how one counts
states when computing entropy in this approach.

Later, I had a nice conversation with Carlo Rovelli about my hopes
for thinking of fermions (e.g., electrons) as the ends of wormholes
in the loop representation of quantum gravity.  We came up with a nice
heuristic argument to get the right Fermi statistics for these wormhole
ends.  Hopefully we can make this all more precise at some later date.

Tuesday, July 23rd - Ted Jacobson gave informal talks on two subjects,
the first of which was "Transplanckian puzzle: origin of outgoing
black hole modes."  This dealt with the puzzling fact that in the
standard computation of Hawking radiation, the rather low-frequency
radiation which leaves the hole is the incredibly redshifted offspring
of high-frequency modes which swung past the horizon shortly after the
hole's formation --- modes whose wavelength is far smaller than the
Planck length!  

What if spacetime is "grainy" in some way at the Planck scale?
Jacobson studied this using an analogy introduced by Unruh.  If you
have fluid flowing down a narrowing pipe, and at some point the
velocity of the fluid flow exceeds the speed of sound in the fluid,
there will be a "sonic horizon".  In other words, there is a line
where the fluid flow exceeds the speed of sound, and no sound can work
its way upstream across that line.  Now if you quantize the theory of
sound in a simple-minded way you get "phonons" --- which have indeed
been observed in solid-state physics.  Unruh showed that in the case
at hand you would get "Hawking radiation" of phonons from the sonic
horizon, going upstream and getting shifted to lower frequencies as
they go.  

Jacobson considered what would happen if you actually took into
account the graininess of the fluid.  (He considered the theory of
liquid helium, to be specific.)  The graininess at the molecular scale
means that the group velocity of waves drops at very high frequencies.
So what happens instead of "Hawking radiation" is something rather
different.  Start with a high-frequency wave attempting to go
upstream, starting from upstream of the sonic horizon.  Its group
velocity is very slow so it fails miserably and gets swept toward the
sonic horizon, like a hapless poor swimmer getting pulled to the edge
of a waterfall despite trying to swim upstream.  But as it gets pulled
near the horizon its wavelength increases, and thus group velocity
increases, thus allowing it to shoot upstream at the last minute!  (An
analogous process is apparently familiar in plasma physics under the
name of "mode conversion".)  In this scenario, the Hawking radiation
winds up resulting from incoming modes through this process of mode
conversion --- modes that have short wavelength, but not as short as
the intermolecular spacing (or Planck length, in the gravitational case.)

Ted Jacobson's second talk was even more interesting to me, but
I'll postpone that for next Week.

Here, by the way, is a paper related to the talk by Pullin discussed in 
"<A HREF = "week86.html">week86</A>":

5) Hugo Fort, Rodolfo Gambini and Jorge Pullin, Lattice knot theory
and quantum gravity in the loop representation, preprint available as
<A HREF = "http://xxx.lanl.gov/abs/gr-qc/9608033">gr-qc/9608033</A>.

\par\noindent\rule{\textwidth}{0.4pt}

% </A>
% </A>
% </A>
