
% </A>
% </A>
% </A>
\week{March 18, 2006 }

Last week I showed you some pretty pictures of dunes on Mars.  
This week I'll talk about dunes called "barchans" and their relation
to self-organized criticality.  Then I'll say a bit about Lauscher
and Reuter's work on quantum gravity... and then I'll beg for help 
on a problem involving so-called "rational tangles".

But first, a demonstration of my psychic powers.  

Take any book off the shelf and look at its 10-digit ISBN number.  
Multiply the first digit by 1, the second digit by 2, the third 
digit by 3 and so on... up to the \emph{next to last} digit.   Add them up.  

Then take this sum and see what it equals mod 11.  At the end of 
this article, I'll say what you got.

Okay.  Here's a photo of the icy dunes of northern Mars.  I love it
because it shows that Mars is a lively place with wind and water:

<DIV ALIGN = CENTER>
<A HREF = "http://themis.asu.edu/features/polardunes"> 
<IMG HEIGHT = 774 WIDTH = 516 SRC = "mars_polardunes.jpg">
% </A>
</DIV>

1) North polar sand sea, Mars Odyssey Mission, THEMIS (Thermal 
emission imaging system), 
<A HREF = "http://themis.mars.asu.edu/features/polardunes">
http://themis.mars.asu.edu/features/polardunes</A>

These dunes, occupying a region the size of Texas, have been sculpted by 
wind into long lines with crests 500 meters apart.  Their hollows are 
covered with frost, which appears bluish-white in this infrared photograph.
The big white spot near the bottom is a hill 100 meters high.  

Where the dunes become sparser - for example, near that icy hill - they 
break apart into "barchans".  These are crescent-shaped formations 
whose horns point downwind.  Barchans are also found on the deserts of 
Earth, and surely on many other planets across the Universe.  They are 
one of several basic dune patterns, an inevitable consequence of the 
laws of nature under fairly common conditions.

The upwind slope of a barchan is gentle, while the downwind slope is 
between 32 and 34 degrees.  
This is the "angle of repose" for sand - the 
maximum angle it can tolerate before it starts slipping down:

<DIV ALIGN = CENTER>
<A HREF = "http://en.wikipedia.org/wiki/Barchan">
<IMG SRC = "barchan.jpg">
% </A>
</DIV>
2) Wikipedia, Barchan, 
<A HREF = "http://en.wikipedia.org/wiki/Barchan">
http://en.wikipedia.org/wiki/Barchan</A>

Wind-blown sand accumulates on the front of the barchan, and then
slides down the "slip face" on the back.

Barchans gradually migrate in the direction of the wind at speeds of
about 1-20 meters per year, with small barchans moving faster than big 
ones.  In fact, when they collide, the smaller barchans pass right 
through the big ones!  So, they act like solitons in some ways.    

It would be great to see one of these frosty Martian barchans close up.
We almost can do it now!  The European Space Agency's orbiter 
called Mars Express took this wonderful closeup, already shown 
in "<A HREF = "week211.html">week211</A>":

<DIV ALIGN = CENTER>
<A HREF = "http://www.esa.int/SPECIALS/Mars_Express/SEMLF6D3M5E_1.html">
<IMG SRC = "mars_pole_closeup.jpg">
% </A>
</DIV>

3) ESA/DLR/FU Berlin (G. Neukum), 
Glacial, volcanic and fluvial activity on Mars: latest images, 
<A HREF = "http://www.esa.int/SPECIALS/Mars_Express/SEMLF6D3M5E_1.html">
http://www.esa.int/SPECIALS/Mars_Express/SEMLF6D3M5E_1.html</A>

However, this is not a barchan - it's a lot bigger.   On top of 
the picture we see dunes, but then there's a cliff almost 2 kilometers
high leading down into what may be a volcanic caldera.  The white stuff 
is ice, while the dark stuff could be volcanic ash. 

It's actually a bit surprising that there's enough wind on Mars to
create dunes.  After all, the air pressure there is about 1% what it
is here on Earth!  But in fact the wind speed on Mars often exceeds 
200 kilometers per hour, with gusts up to 600 kilometers per hour.  
There are dust storms on Mars so big they were first seen from 
telescopes on Earth long ago.  So, wind is a big factor in Martian 
geology:

4) NASA, Mars exploration program: dust storms, 
<A HREF = "http://mars.jpl.nasa.gov/gallery/duststorms/">
http://mars.jpl.nasa.gov/gallery/duststorms/</A>

The Mars rover Spirit even got its solar panels cleaned by 
some dust devils, and it took some movies of them:

<DIV ALIGN = CENTER>
<A HREF = "http://marsrovers.nasa.gov/gallery/press/spirit/20050819a.html">
<IMG EEIGHT = 150 WIDTH = 600 SRC = "mars_dust_devil.gif">
% </A>
</DIV>

5) NASA, Exploration rover mission: dust devils at Gusev, Sol 525,
<A HREF = "http://marsrovers.nasa.gov/gallery/press/spirit/20050819a.html">
http://marsrovers.nasa.gov/gallery/press/spirit/20050819a.html</A>

Turning to mathematical physics per se, I can't resist pointing out
that sand piles became very fashionable in this subject a 
while back.

Why?  Well, for this I need to explain "self-organized criticality".

First, note that when a pile of sand is exactly at its angle of repose, 
it will suffer lots of little landslides - and a few of these will become 
big.  

The theory of "critical phenomena" suggests that in this situation,
the probability that a landslide grows to size L should satisfy a 
power law.  In other, it should be proportional to 

1/L^{c} 

for some number c called the "critical exponent".  At least, this 
type of behavior is seen in many other situations where a physical 
system is on the brink of some drastic change - or more precisely,
a "critical point".

When a system is not at a critical ponit, we typically see exponential 
laws, where the probability of a disturbance of size L is proportional 
to

exp(-L/L_{0})

where L_{0} is a fixed length scale.  This means that our system will 
look qualitatively different depending how much we zoom in with our 
microscope.  At length scales shorter than L_c, disturbances are
really common, while at larger length scales they're incredibly rare.

When a system \emph{is} at a critical point, it's self-similar: you can 
zoom in or zoom out, and it looks qualitatively the same!  It has
no specific length scale.  This is what the power law says.

Here's a good place to learn the basics of power laws and self-similarity:

6) Manfred R. Schroeder, Fractals, Chaos, Power Laws, W. H. Freeman, 
New York, 1992.

What makes sand dunes interesting is that as they seem to \emph{enjoy}
living on the brink of danger.  As the wind blows, they heap up until
their slip face is right at the angle of repose... ready for landslides!

This is the idea of "self-organized criticality": some physical 
systems 
seem to spontaneously bring themselves towards critical points, without 
any need for us to tune their parameters to special values.  

The paper that introduced this idea came out in 1987:

7) Per Bak, Chao Tang and Kurt Wiesenfeld, Self-organized criticality: 
an explanation of 1/f noise, Phys. Rev. Lett. 59 (1987) 381-384.  

They came up with a simple model of a sand pile that exhibits 
self-organized criticality.  In the words of Jos Thijssen:

\begin{quote}
     Bak and co-workers modelled the sand pile as a regular array 
     of columns consisting of cubic sand grains.  Addition of new 
     grains is simply performed by selecting a column at random and 
     increasing its height by one.  If the column then exceeds its 
     neighbours in height more than some threshold, it 
     will "collapse": 
     it will lose some grains which are distributed evenly over its 
     nearest neighbours.  As this collapse alters the height differences 
     involving those neighbours, there is the possibility that they 
     collapse in turn.  A cascade process sets in until all height 
     differences are below the threshold. The size of such an avalanche 
     is defined as the number of sand grains sliding as a result of 
     a single grain of sand being added to the pile.

     What is so interesting about the sand pile model?  It turns out 
     that the sides of the sand pile acquire a specific slope, which is 
     such that the distribution of avalanches as function of size scales 
     as a power law. Power laws indicate the absence of scale and indeed 
     avalanches on all scales are sustained for the equilibrium slope. 
     If the slope is changed artificially from its equilibrium value, 
     the distribution is no longer a power law, but it will have an 
     intrinsic scale (e.g. exponential).  Power laws and absence of scale 
     are the signature of a system being critical.  Because the sand pile 
     tends to adjust the slope of its sides until the power law scaling 
     sets in, the criticality is called "self-organised".
\end{quote}

If your computer runs Java applets, you can play with Thijssen's
simulation sand pile and see the avalanches yourself:

8) Jos Thijssen, The sand pile model and self organised criticality,
<A HREF = "http://www.tn.tudelft.nl/tn/People/Staff/Thijssen/sandexpl.html">
http://www.tn.tudelft.nl/tn/People/Staff/Thijssen/sandexpl.html</A>

And here's a cellular automaton sand pile you can play with:   

9) Albert Schueller, Cellular automaton sand pile model,
<A HREF = "http://schuelaw.whitman.edu/JavaApplets/SandPileApplet/">
http://schuelaw.whitman.edu/JavaApplets/SandPileApplet/</A>

This one is only 2-dimensional, so the avalanches are less dramatic,
but you can have some fun using the mouse to build structures that 
impede the motion of sand.

Like a speck of sand landing at the right place at the right time, 
the original paper of Bak \emph{et al} started a huge landslide of work on 
self-organized criticality, some of which has been popularized here:

10) Per Bak, How Nature Works: The Science of Self-Organized Criticality,
Copernicus, New York, 1996.

As you can guess from the title "How Nature Works", some people got a 
little carried away with the importance of self-organized criticality.
Then there was a kind of backlash, just as happened with fractals, 
chaos, and catastrophe theory.  These are all perfectly respectable and 
interesting topics in mathematical physics that suffered from being 
oversold.  People are always eager to find the secret key that will 
unlock all the mysteries of the universe.  So, if some new idea seems 
very general, people will run around trying to unlock all the mysteries 
of the universe with it - and become sorely disappointed when it only 
unlocks \emph{some}.

I'd be interested to see how well mathematical physicists can model
actual sand dunes.  These display an interesting complexity of behavior,
as the pictures here show:

11) US Army Corps of Engineers, Dunes, 
<A HREF = "http://www.tec.army.mil/research/products/desert_guide/lsmsheet/lsdune.htm">
http://www.tec.army.mil/research/products/desert_guide/lsmsheet/lsdune.htm</A>

I've only looked at a few papers on the subject, all dealing with
barchans:

12) V. Schwaemmle and H. J. Herrmann, Solitary wave behaviour of sand 
dunes, Nature 426 (Dec. 11, 2003), 619-620.

13) Klaus Kroy, Gerd Sauermann, and Hans J. Hermann, Minimal model for
sand dunes, Phys. Rev. Lett. 88 (2002), 054301.  Also available at
<A HREF = "http://arxiv.org/abs/cond-mat/0101380">cond-mat/0101380</A>.

14) H. Elbelrhiti, P. Claudin, and B. Andreotti, Field evidence for 
surface-wave-induced instability of sand dunes, Nature 437 (Sep. 29, 2005),
720-723.

The first paper describes how barchans pass through each other like
solitons, simulating them by an equation that's described in the second 
one.  (By the way, the term "minimal model" in the title of the 
second 
paper is not being used in the sense familiar in conformal field theory!)

The third paper reports the results of a 3-year field study: in reality,
barchans are not stable, and big ones (called "megabarchans") can 
break apart into smaller "elementary barchans".

If you're more interested in Mars than the mathematical physics of sand 
dunes, you'll be happy to hear that Google has just moved to drastically 
expand its customer base by introducing "Google Mars":

15) Google Mars, <A HREF = "http://www.google.com/mars/">http://www.google.com/mars/</A>

Using this you can explore many features of Mars, including its dunes.

I'm getting a little tired out, but there's one thing I've been
meaning to mention for a while.  It's actually related to renormalization,
which is secretly the same subject as this "critical point" business 
I just mentioned.  But, it's not about sand piles - it's about quantum 
gravity!

In "<A HREF = "week222.html">week222</A>" I spoke about the 
work of Lauscher and Reuter, who claim to 
have found evidence for an ultraviolet fixed point in quantum gravity
without matter.  In other words, as you zoom in closer and closer, they 
claim quantum gravity without matter acts more and more like some fixed
theory.  This would be big news: it would suggest that gravity without 
matter is a sensible theory, contrary to what everyone in string theory 
says!

Not surprisingly, the string theorist Jacques Distler examined Lauscher 
and Reuter's work with a critical eye.  And, he wrote up a nice 
explanation of the problems with their work:

16) Jacques Distler, Unpleasantness,
<A HREF = "http://golem.ph.utexas.edu/~distler/blog/archives/000648.html">
http://golem.ph.utexas.edu/~distler/blog/archives/000648.html</A>

Briefly, the problem is that Lauscher and Reuter make a drastic 
approximation.   They start with the "exact renormalization group 
equation", which is a beautiful thing: it says how a Lagrangian 
for a field theory at one length scale gives rise to an effective 
Lagrangian for the same theory at a larger length scale.   However,
then they truncate the incredibly complicated formula for a fully
general Lagrangian, restricting to Lagrangians with only an 
Einstein-Hilbert term and a cosmological constant.  Like Distler, 
I see no reason to think this approximation is valid.  So, their
claimed ultraviolet fixed point could be an artifact of their method.

Whether it's worth going further and checking this by considering 
a slightly less brutal approximation, using Lagrangians with a few 
more terms, is a matter of taste.  Distler doesn't think so.  I 
hope Lauscher and Reuter do.  If they don't, we may never know for
sure what happens.  I think it's actually rather amazing that they 
get an fixed point with their brutal approximation, instead of 
coupling constants that run to infinity or zero, which is what 
I would have naively expected.  But who knows?  Maybe this is 
easily understood if you think hard enough.


Today I was also going to talk about the 3-strand braid group, the 
group PSL(2,Z), and rational tangles, but now I don't have the energy.  
So instead, I'll just put out a request for help!  

There's a wonderful game invented by John Conway called "rational
tangles".  Here's how it works.  It involves two players and a referee.

The players, call them A and B, start by facing each other and holding 
ropes in each hand connecting them together like this:
  

\begin{verbatim}

  A   A
  |   |
  |   |
  B   B
\end{verbatim}
    

This is called "position 0".  
The referee then cries out either \emph{add one!} 
or \emph{take the negative reciprocal!}.   
If the referee yells \emph{add one!}, player
B has to switch which hand he's using to hold which rope, making sure to
pass the right one over the left, like this:


\begin{verbatim}

  A   A
   \ /
    /
   / \
  B   B
\end{verbatim}
    

This is called "position 1", 
since we started with "position 0" and
then did \emph{add one!}  But if the referee says <em>take the inverse 
reciprocal!</em>, both players must cooperate to move all four ends of
the ropes a quarter-turn clockwise, like this:


\begin{verbatim}

  A   A
   \_/
    _
   / \
  B   B
\end{verbatim}
    

This is called "position -1/0", since we started with 0 and then
did \emph{take the negative reciprocal!}

The referee keeps crying \emph{add one!}
or \emph{take the negative reciprocal!}
in whatever order she feels like, and players A and B keep doing the
same sort of thing: either player B switches the ropes right over left,
or both players rotate the whole tangle a quarter-turn clockwise.  It's
actually best if the referee doesn't start with <em>take the negative
reciprocal!</em>, since some people refuse to divide by zero, for religious
reasons.  But, it's perfectly legal in this game.

Anyway, after a while the ropes will be 
in "position p/q" for some complicated rational number p/q.  
The'll be all tangled up - but in a special way, called a 
"rational tangle".

Then the players have to \emph{undo} 
the tangling and get back to "position 0".
They may not remember the exact sequence of moves that got them into
the mess they are in.  In fact the game is much more fun if they 
\emph{don't}
remember.  It's best to do it at a party, possibly after a few drinks.

Luckily, any sequence of \emph{add one!} and 
\emph{take the negative reciprocal!} moves the players
make that carry their number back to 0, will carry their tangle back 
to "position 0".  So they just need to figure out how to get their 
number back to 0, and the tangle will automatically untangle itself.  
That's the cool part!  It's a highly nonobvious theorem due to Conway.

I'm vaguely aware of a few proofs of this fact.  As far as I know, 
Conway's original proof uses the Alexander-Conway polynomial:

16) John Horton Conway, An enumeration of knots and links and some
of their algebraic properties, in Computational Problems of Abstract
Algebra, ed. John Leech, Pergamon Press, Oxford, 1970, 329-358.

There's also a proof by Goldman and Kauffman using the Jones polynomial:

17) Jay R. Goldman and Louis H. Kauffman, Rational tangles, Advances in 
Applied Mathematics 18 (1997), 300-332.  Also available at 
<A HREF = "http://www.math.uic.edu/~kauffman/RTang.pdf">
http://www.math.uic.edu/~kauffman/RTang.pdf</A>

There are also two proofs in here:

18) Louis H. Kauffman and Sofia Lambropoulou, On the classification of
rational tangles, available as 
<A HREF = "http://arxiv.org/abs/math.GT/0311499">math.GT/0311499</A>.
 
But here's what I want to know: is there a proof that makes
extensive use of the group PSL(2,Z) and its relation to topology?

After all, the basic operations on rational tangles are "adding
one" and "negative reciprocal", and these generate all the 
fractional linear transformations


\begin{verbatim}

         az + b
z |->   --------
         cz + d
\end{verbatim}
    
with a,b,c,d integer and ad-bc = 1.  The group of these transformations
is PSL(2,Z).  It acts on rational tangles, and Conway's theorem says
this action is isomorphic to the obvious action of PSL(2,Z) as fractional
linear transformations of the "rational projective line", 
meaning the rational 
numbers together with a point at infinity.  Since PSL(2,Z) has lots of 
relations to topology, there should be some proof of Conway's theorem
that \emph{uses} these relations to get the job done.

Does anybody know one?
Finally, the answer to the psychic powers puzzle: if you did the 
calculation right, you got the last digit of the book's ISBN number - 
unless your answer was 10, in which case the ISBN number should end 
in the letter X.

This trick is called a "check sum" or "check digit": 
it's a way to spot 
errors.  The Universal Product Code, used in those bar codes you see
everywhere, also has a check digit.  So do credit cards.  

\par\noindent\rule{\textwidth}{0.4pt}
\textbf{Addendum:}  
Aaron Lauda and James Given had comments.  Lauda wrote:

\begin{quote}
Usually people describe the scheme in a different way, which is
actually equivalent to what you said.  Denote the 10 digit ISBN number 
as N_{i}, for 1\le i\le 10. Compute


\begin{verbatim}

       10
 Mod(  \sum  (11-i)N_{i} , 11)              (*)
      i=1
\end{verbatim}
    
which should give you zero.  That is take the first digit, multiply it by 
10, the second by 9, etc.  Add them up and compute the sum mod 11.  You 
will always get zero.

Some fun things you might like to add:

<OL>
<LI>
If you make a mistake writing down a single digit in the ISBN then the 
equation (*) will not equal zero.  

<LI>
 The equation (*) may fail to give you zero mod 11 if you make a mistake 
  with two of the digits, but it will never fail if you interchange two 
  adjacent digits.
</OL>

Regards,<br>
Aaron<br>
\end{quote}

Given wrote:

\begin{quote}
  Self-organized criticality (SOC) does in fact involve special settings of
  the parameters in a model.
  SOC occurs in sandpile models because one adds the sand extremely slowly,
  i.e., one grain at a time. Otherwise a critical state is not obtained.
  This makes SOC be a special example of dynamical critical phenomena in
  the case that the flux variable (here the rate of sand addition) is set
  to \epsilon +, i.e., an infinitesimal value greater than zero.  This
  formulation allows SOC to be studied using quantum field theory.

  Of course the model is built around an underlying instability, namely
  the fact that sand piles which are too steep will fall down.  Also, one
  must remove by idealization most sorts of friction between sand grains
  which will otherwise blur out the transition.  So SOC is no magic
  prescription for generating scale invariant phenomena.  SOC systems are
  "special" in the way that equilibrium critical points are 
  "special".
  As you note, theories of this kind are easily oversold among those
  eager to believe in magic formulas.

  Also, you may have been confusing SOC with a favorite concept of the
  chaos/dynamical systems people, namely the "edge of chaos".  
   It went
  through several incarnations.  Each one tried to formally specify the
  domain, intermediate between order and chaos, in which complex systems
  were most "interesting".

  Wikipedia summarizes this pretty well.  I append the listing for
  convenience.

  All My Best,<br>
  Jim Given
\end{quote}

The Wikipedia article is:

19) Edge of chaos, Wikipedia, 
<A HREF = "http://en.wikipedia.org/wiki/Edge_of_chaos">
http://en.wikipedia.org/wiki/Edge_of_chaos</A>


\par\noindent\rule{\textwidth}{0.4pt}
% </A>
% </A>
% </A>
