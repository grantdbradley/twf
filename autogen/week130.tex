
% </A>
% </A>
% </A>
\week{February 27, 1999}


All sorts of cool stuff is happening in physics - and I don't mean
mathematical physics, I mean real live experimental physics!  I feel
slightly guilty for not mentioning it on This Week's Finds.  Let me
atone.

Here's the big news in a nutshell: we may have been wrong about four
fundamental constants of nature.  We thought they were zero, but maybe
they're not!  I'm talking about the masses of the neutrinos and the
cosmological constant.   

Let's start with neutrinos.

There are three kinds of neutrinos: electron, muon, and \tau  neutrinos.
They are closely akin to the charged particles whose names they borrow -
the electron, muon and \tau  - but unlike those particles they are
electrically neutral and very light.  They are rather elusive, since
they interact only via the weak force and gravity.  I'm sure you've all
heard how a neutrino can easily make it through hundreds of light years
of lead without being absorbed.  

But despite their ghostly nature, neutrinos play a very real role in
physics, since radioactive decay often involves a neutron turning into a
proton while releasing an electron and an electron antineutrino.  (In
fact, Pauli proposed the existence of neutrinos in 1930 to account for a
little energy that went missing in this process.  They were only
directly observed in 1956.)  Similarly, in nuclear fusion, a proton may
become a neutron while releasing a positron and an electron neutrino.  
For example, when a type II supernova goes off, it emits so many
neutrinos that if you're anywhere nearby, they'll kill you before
anything else gets to you!  Indeed, in 1987 a supernova in the Large
Magellanic Cloud, about 100,000 light years away, was detected by four
separate neutrino detectors.  

I said neutrinos were "very light", but just how light?  So far most
work has only given upper bounds.  In the 1980s, the Russian ITEP group
claimed to have found a nonzero mass for the electron neutrino, but this
was subsequently blamed on problems with their apparatus.  As of now,
laboratory experiments give upper bounds of 4.4 eV for the electron
neutrino mass, .17 MeV for the muon neutrino, and 18 MeV for the \tau 
neutrino.  By contrast, the electron's mass is .511 MeV, the muon's is
106 MeV, and the \tau 's is a whopping 1771 MeV.  

For this reason, the conventional wisdom used to be that neutrinos were
massless.  After all, the electron neutrino is definitely far lighter
than any known particle except the photon - which is massless.   The
larger upper bounds on the other neutrino's masses are mainly due to 
the greater difficulty in doing the experiments.  

Having neutrinos be massless would also nicely explain their most
stunning characteristic, namely that they're only found in a left-handed
form.  What I mean by this is that they spin counterclockwise when
viewed head-on as they come towards you.  It turns out that this
violation of left-right symmetry comes fairly easily to massless
particles, but only with more difficulty to massive ones.  The reason is
simple: massless particles move at the speed of light, so you can't
outrun them.  Thus everyone, regardless of their velocity, agrees on
what it means for such a particle to be spinning one way or another as
it comes towards them.  This is not the case for a massive particle!

There was, however, a fly in the ointment.  Since the sun is powered by
fusion, it should emit lots of neutrinos.   In fact, the standard solar
model predicts that here on earth we are bombarded by 60 billion solar
neutrinos per square centimeter per second!  So in the late 1960s, a
team led by Ray Davis set out to detect these neutrinos by putting a
tank of 100,000 gallons of perchloroethylene down into a gold mine in
Homestake, South Dakota.  Lots of different nuclear reactions are going
on in the sun, producing neutrinos of different energies.  The Homestake
experiment can only detect the most energetic ones - those produced when
boron-8 decays into beryllium-8.  These neutrinos have enough energy to
turn chlorine-37 in the tank into argon-37.  Being a noble gas, the
argon can be separated out and measured.  This is not easy - one only
expects about 4 atoms of argon a day!  So the experiment required
extreme care and went on for decades.  

They only saw about a quarter as many neutrinos as expected.  

Of course, with an experiment as delicate as this, there are always many
possibilities for error, including errors in the standard solar model. 
So a Japanese group decided to use a tank of 2,000 tons of water  in a
mine in Kamioka to look for solar neutrinos.  This "Kamiokande"
experiment used photomultiplier tubes to detect the Cherenkov radiation 
formed by electrons that happen to be hit by neutrinos.  Again it was
sensitive only to high-energy neutrinos.  

After 5 years, they started seeing signs of a correlation between
sunspot activity and their neutrino count.  Interesting.  But more
interesting still, they didn't see as many neutrinos as expected.  
Only about half as many, in fact.  

Starting in the 1990s, various people began to build detectors that
could detect lower-energy neutrinos - including those produced in the 
dominant fusion reactions powering the sun.   For this it's good to use
gallium-71, which turns to germanium-71 when bombarded by neutrinos. 
The GALLEX detector in Italy uses 30 tons of gallium in the form of
gallium chloride dissolved in water.  The SAGE detector, located in a
tunnel in the Caucasus mountains, uses 60 tons of molten metallic
gallium.   This isn't quite as scary as it sounds, because gallium has a
very low melting point - it melts in your hand!  But still, of course,
these experiments are very difficult.  

Again, these experiments didn't see as many neutrinos as expected.

By this point, the theorists had worked themselves into a full head of
steam trying to account for the missing neutrinos.  Currently the most
popular theory is that some of the electron neutrinos have turned into
muon and \tau  neutrinos by the time they reach earth.  These other 
neutrinos would be not be registered by our detectors.  

Folks call this hypothetical process "neutrino oscillation".  For it
to happen, the neutrinos need to have a nonzero mass.  After all,
a massless particle moves at the speed of light, so it doesn't experience
any passage of time - thanks to relativistic time dilation.  Only particles
with mass can become something else while they are whizzing along minding
their own business.  

If in fact you posit a small mass for the neutrinos, oscillations happen
automatically as long as the "mass eigenstates" are different 
from the
"flavor eigenstates".  By "flavor" we mean whether 
the neutrino is an
electron, muon or \tau  neutrino.  For simplicity, imagine that the state 
of a neutrino at rest is given by a vector whose 3 components are the 
amplitudes for it to be one of the three different flavors.  If all 
but one of these components are zero we have a neutrino with a definite
flavor - a "flavor eigenstate".   On the other hand, the energy of a
particle at rest is basically just its mass.  Thus in the present
context the energy of the neutrino is described by a 3 x 3 self-adjoint
matrix H, the "Hamiltonian", whose eigenvectors are called "mass
eigenstates".  These may or may not be the same as the flavor
eigenstates!  Schroedinger's equation says that any state \psi  of the
neutrino evolves as follows:

                       d\psi /dt = -iH \psi 

Thus if \psi  starts out being a mass eigenstate it stays a mass eigenstate.
But if it starts out being a flavor eigenstate, it won't stay a flavor 
eigenstate - unless the mass and flavor eigenstates coincide!  Instead, it
will oscillate.

I bet you were wondering when the math would start.   Don't worry, there
won't be much this time.

Anyway, for other particles, like quarks, it's well-known that the mass
and flavor eigenstates \emph{don't} coincide.  So we shouldn't be surprised
at neutrino oscillations, at least if neutrinos actually have nonzero
mass.  

Actually things are more complicated than I'm letting on.  In addition
to oscillating in empty space, it's possible that neutrinos oscillate
\emph{more} as they are passing through the sun itself, thanks to something
called the MSW effect - named after Mikheyev, Smirnov and Wolfenstein.
And there are two different ways for neutrinos to have mass, depending
on whether they are Dirac spinors or Majorana spinors 
(see "<A HREF = "week93.html">week93</A>").

But I don't want to get caught up in theoretical nuances here!  I want
to talk about experiments, and I haven't even gotten to the new stuff
yet - the stuff that's getting everybody \emph{really} confused!

First of all, there's now some laboratory evidence for neutrino 
oscillations coming from the Liquid Scintillator Neutrino Detector at
Los Alamos.  What these folks do is let positively charged pions decay
into antimuons and muon neutrinos.  Then they check to see if any muon
neutrinos become electron neutrinos.   They claim that they do!  They
also claim to see evidence of muon antineutrinos becoming electron
antineutrinos.

Secondly, and more intriguing still, there are a bunch of experiments
involving atmospheric neutrinos: Super-Kamiokande, Soudan 2, IMB, and 
MACRO.   You see, when cosmic rays smack into the upper atmosphere, they
produce all sorts of particles, including electron and muon neutrinos
and their corresponding antineutrinos.  Cosmic ray experts think they
know how many of each sort of neutrino should be produced.  But the
experimenters down on the ground are seeing different numbers!  

Again, this could be due to neutrino oscillations.  But what's REALLY
cool is that the numbers seem to depend on where the neutrinos are
coming from: from the sky right above the detector, from right below the
detector - in which case they must have come all the way through the
earth - or whatever.  Neutrinos coming from different directions take
different amounts of time to get from the upper atmosphere to the
detector.  Thus an obvious explanation for the experimental results is
that we're actually seeing the oscillation process AS IT TAKES PLACE.

If this is true, we can try to get detailed information about the
neutrino mass matrix from the numbers these experiments are measuring!

And this is exactly what people have been doing.  But they're finding
something very strange.  If all the experiments are right, and nobody is
making any mistakes, it seems that NO choice of neutrino mass matrix
really fits all the data!  To fit all the data, folks need to do
something drastic - like posit a 4th kind of neutrino!  

Now, it's no light matter to posit another neutrino.  The known
neutrinos couple to the weak force in almost identical ways.  This
allows one to create equal amounts of neutrino-antineutrino pairs of
all 3 flavors by letting Z bosons decay - the Z being the neutral
carrier of the weak force.   When a Z boson seemingly decays into
"nothing", we can safely bet that it has decayed into a neutrino-
antineutrino pair.  In 1989, an elegant and famous experiment at CERN
showed that Z bosons decay into "nothing" at exactly the 
rate one would
expect if there were 3 flavors of neutrino.  Thus there can only be
extra flavors of neutrino if they are very massive, if they couple very
differently to the weak force, or if some other funny business is going
on.

Now, electron or muon neutrinos are unlikely to oscillate into a 
\emph{very massive} sort of neutrino - basically because of energy conservation.
So if we want an extra neutrino to explain the experimental results 
we find ourselves stuck with, it'll have to be one that couples to the
weak force very differently from the ones we know.  A simple, but
drastic, possibility is that it not interact via the weak force at all!
Folks call this a "sterile" neutrino.  

Now, sterile neutrinos would blow a big hole in the Standard Model,
much more so than plain old \emph{massive} neutrinos.  So things are getting
very interesting.

Wilczek recently wrote a nice easy-to-read paper describing arguments
that \emph{massive} neutrinos fit in quite nicely with the possibility that
the Standard Model is just part of a bigger, better theory - a "Grand
Unified Theory".  I sketched the basic ideas of the SU(5) and SO(10)
grand unified theories in 
"<A HREF = "week119.html">week119</A>".  

Recall that in the SU(5) theory, 
the left-handed parts of all fermions of a given generation fit into two 
irreducible representations of SU(5) - a 5-dimensional rep and a 
10-dimensional rep.  For example, for the first generation, the 
5-dimensional rep consists of the left-handed down antiquark (which
comes in 3 colors), the left-handed electron, and the left-handed electron
neutrino.  The 10-dimensional rep consists of the left-handed up quark,
down quark, and up antiquark (which come in 3 colors each), together
with the left-handed positron.   

In the SO(10) theory, all these particles AND ONE MORE fit into a single 
16-dimensional irreducible representation of SO(10).  What could this 
extra particle be?  

Well, since this extra particle transforms trivially under SU(5), it
must not feel the electromagnetic, weak or strong force!  Thus it's
tempting to take this missing particle to be the left-handed electron 
antineutrino.  Of course, we don't see such a particle - we only see
antineutrinos that spin clockwise.  But if neutrinos are massive Dirac 
spinors there must be such a particle, and having it not feel the
electromagnetic, weak or strong force would nicely explain \emph{why} we
don't see it.


Grotz and Klapdor consider this possibility in their book on the weak
interaction (see below), but unfortunately, it seems this theory would
make the electron neutrino have a mass of about 5 MeV - much too big!
Sigh.  So Wilczek, following the conventional wisdom, assumes the
missing particle is very massive - he calls it the "N".  And
he summarizes some arguments that this massive particle could help give
the neutrinos very small masses, via something called the "seesaw
mechanism".  Unfortunately I don't have the energy to describe this
now, so for more you should look at his paper (referred to below).

To wrap up, let me just say one final thing about the cosmic
significance of the neutrino.  Massive neutrinos could account for some
of the "missing mass" that cosmologists are worrying about.
So there's an indirect connection between the neutrino mass and the
cosmological constant!  The cosmological constant is essentially the
energy density of the vacuum.  It was long assumed to be zero, but now
there are some glimmerings of evidence that it's not.  In fact, some
people are quite convinced that it's not.  The fate of the universe
hangs in the balance....

Unfortunately I am too tired now to say much more about this.
So let me just give you a nice easy starting-point:

1) Special Report: Revolution in Cosmology, Scientific American, January
1999.  Includes the articles "Surveying space-time with
supernovae" by Craig J. Horgan, Robert P. Kirschner and
Nicholoas B.  Suntzeff, "Cosmological antigravity" by
Lawrence M. Krauss, and "Inflation in a low-density
universe" by Martin A. Bucher and David N.  Spergel.

How can you learn more about neutrinos?  It can't hurt to start here:

2) Nikolas Solomey, The Elusive Neutrino, Scientific American Library,
1997.

If you want to dig in deeper, you need to learn about the weak force, 
since we've only seen neutrinos via their weak interaction with other
particles.  The following book is a great place to start:

3) K. Grotz and H. V. Klapdor, The Weak Interaction in Nuclear, Particle
and Astrophysics, Adam Hilger, Bristol, 1990.

Then you'll be ready for this book, which examines every aspect of
neutrinos in detail - complete with copies of historical papers:

4) Klaus Winter, ed., Neutrino Physics, Cambridge U. Press, Cambridge,
1991.

And then, if you want to study the possibility of \emph{massive} neutrinos,
you should try this:

5) Felix Boehm and Petr Vogel, Physics of Massive Neutrinos, Cambridge 
U. Press, Cambridge, 1987.

But neutrino physics is moving fast, and lots of the new stuff hasn't
made its way into books yet, so you should also look at other stuff.
For links to lots of great neutrino websites, including websites for 
most of the experiments I mentioned, try:

6) The neutrino oscillation industry,  
<A HREF = "http://www.hep.anl.gov/NDK/hypertext/nu_industry.html">http://www.hep.anl.gov/NDK/hypertext/nu_industry.html</A>

For some recent general overviews, try these:

7) Paul Langacker, Implications of neutrino mass, 
<A HREF = "http://dept.physics.upenn.edu/neutrino/jhu/jhu.html">http://dept.physics.upenn.edu/neutrino/jhu/jhu.html</A>

8) Boris Kayser, Neutrino mass: where do we stand, and where are we 
going?, preprint available as 
<A HREF = "http://xxx.lanl.gov/abs/hep-ph/9810513">hep-ph/9810513</A>.

For information on various experiments, try these:

9) GALLEX collaboration, GALLEX solar neutrino observations: complete 
results for GALLEX II, Phys. Lett. B357 (1995), 237-247.

Final results of the CR-51 neutrino source experiments in GALLEX,
Phys. Lett. B420 (1998), 114-126.

GALLEX solar neutrino observations: results for GALLEX IV, Phys.
Lett. B447 (1999), 127-133.

11) SAGE collaboration, Results from SAGE, Phys. Lett B328 (1994), 
234-248.

The Russian-American gallium experiment (SAGE) CR neutrino source
measurement, Phys. Rev. Lett. 77 (1996), 4708-4711. 

12) LSND collaboration, Evidence for neutrino oscillations from 
muon decay at rest, Phys. Rev. C54 (1996) 2685-2708, preprint available
as <A HREF = "http://xxx.lanl.gov/abs/nucl-ex/9605001">nucl-ex/9605001</A>.

Evidence for anti-muon-neutrino \to  anti-electron-neutrino oscillations
from the LSND experiment at LAMPF, Phys. Rev. Lett 77 (1996), 3082-3085,
preprint available as 
<A HREF = "http://xxx.lanl.gov/abs/nucl-ex/9605003">nucl-ex/9605003</A>.

Evidence for muon-neutrino \to  electron-neutrino 
oscillations from LSND, Phys. Rev. Lett. 81 
(1998), 1774-1777, preprint available as
<A HREF = "http://xxx.lanl.gov/abs/nucl-ex/9709006">nucl-ex/9709006</A>.

Results on muon-neutrino \to  electron-neutrino
oscillations from pion decay in flight,
Phys. Rev. C58 (1998), 2489-2511.

13) Super-Kamiokande collaboration, Evidence for oscillation of 
atmospheric neutrinos, Phys. Rev. Lett 81 (1998), 1562-1567,
preprint available as
<A HREF = "http://xxx.lanl.gov/abs/hep-ex/9807003">hep-ex/9807003</A>.

14) MACRO collaboration, Measurement of the atmospheric neutrino-induced
upgoing muon flux, Phys. Lett. B434 (1998), 451-457, 
preprint available as
<A HREF = "http://xxx.lanl.gov/abs/hep-ex/9807005">hep-ex/9807005</A>.

15) IMB collaboration, A search for muon-neutrino oscillations with
the IMB detector, Phys. Rev. Lett 69 (1992), 1010-1013.

For a fairly model-independent attempt to figure out something about
neutrino masses from the latest crop of experiments, see:

16) V. Barger, T. J. Weiler, and K. Whisnant, Inferred 4.4 eV upper
limits on the muon- and tau-neutrino masses, preprint available as
<A HREF = "http://xxx.lanl.gov/abs/hep-ph/9808367">hep-ph/9808367</A>.

For a nice summary of the data, and an argument that it's evidence
for the existence of a sterile neutrino, see:

17) David O. Caldwell, The status of neutrino mass, preprint available
as 
<A HREF = "http://xxx.lanl.gov/abs/hep-ph/9804367">hep-ph/9804367</A>.

For a very readable argument that massive neutrinos are evidence for a
supersymmetric SO(10) grand unified theory, see

18) Frank Wilczek, Beyond the Standard Model: this time for real, 
preprint available as
<A HREF = "http://xxx.lanl.gov/abs/hep-ph/9809509">hep-ph/9809509</A>.

Finally, with all these cracks developing in the Standard Model, it's nice
to think again about the rise of the Standard Model.  The following book
is packed with the reminiscences of many theorists and experimentalists
involved in developing this wonderful theory of particles and forces,
including Bjorken, 't Hooft, Veltman, Susskind, Polyakov, Richter,
Iliopoulos, Gell-Mann, Weinberg, Lederman, Goldhaber, Cronin, and
Kobayashi:

19) Lilian Hoddeson, Laurie Brown, Michael Riordan and Max Dresden,
eds., The Rise of the Standard Model: Particle Physics in the 1960s and
1970s.  

It's a must for anyone with an interest in the history of physics!

\par\noindent\rule{\textwidth}{0.4pt}
By the way, here's a cool picture of the sun, taken using neutrinos
rather than light:

20) LSU Super-Kamiokande group homepage, 
<A HREF = "http://beavis.phys.lsu.edu/~superk/">http://beavis.phys.lsu.edu/~superk/</A>

Thanks go to Jim Carr for pointing this out.


 \par\noindent\rule{\textwidth}{0.4pt}

% </A>
% </A>
% </A>
