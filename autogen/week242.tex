
% </A>
% </A>
% </A>
\week{December 17, 2006 }

This week I'd like to talk about a paper by Jeffrey Morton.  Jeff is a
grad student now working with me on topological quantum field theory
and higher categories.  I've already mentioned his work on
categorified algebra and quantum mechanics in "<A HREF =
"week236.html">week236</A>".  He'll be be finishing his
Ph.D. thesis in the spring of 2007 - and as usual, that means he's
already busy applying for jobs.

As all you grad students reading this know, applying for jobs is
pretty scary the first time around: there are some tricks involved, 
and nobody prepares you for it.  I remember myself, wondering what 
I'd do if I didn't succeed.  Would I have to sell ice cream from one 
of those trucks that plays a little tune as it drives around the 
neighborhood?  A job in the financial industry seemed scarcely more 
appealing: less time to think about math, and less ice cream too.

Luckily things worked out for me... and I'm sure they'll work out
for Jeff and my other student finishing up this year - Derek Wise, 
who is working on Cartan geometry and MacDowell-Mansouri gravity. 

But, to help them out a bit, I'd like to talk about their work.  
This has been high on my list of interests for the last few years, 
of course, but I've mostly been keeping it under wraps.   

This time I'll talk about Jeff's thesis; next time Derek's.  But 
first, let's start with some cool astronomy pictures!

Here's a photo of Saturn, Saturn's rings, and its moon Dione, taken by 
the Cassini orbiter in October last year:

<div align = center>
<a href = "dione_ring.jpg">
<img width = 500 src = "dione_ring.jpg">
% </a>
</div>

1) NASA, Ringside with Dione, 
<a href = "http://solarsystem.nasa.gov/multimedia/display.cfm?IM_ID=4163">http://solarsystem.nasa.gov/multimedia/display.cfm?IM_ID=4163</a>

It's so vivid it seems like a composite fake, but it's not!  With
the Sun shining from below, delicate shadows of the B and C rings 
cover Saturn's northern hemisphere.   Dione seems to hover nearby.
Actually it's 39,000 kilometers away in this photo.  It's 1,200 
kilometers in diameter, about the third the size of our Moon.

Here's a photo of Saturn, its rings, and its moon Mimas, taken 
in November 2004:

<div align = center>
<a href = "mimas_ring_big.jpg">
<img width = 500 src = "mimas_ring.jpg">
% </a>
</div>

2) NASA, Nature's canvas, <a href = "http://saturn.jpl.nasa.gov/multimedia/images/image-details.cfm?imageID=1088">
http://saturn.jpl.nasa.gov/multimedia/images/image-details.cfm?imageID=1088</a>

It's gorgeous, but it takes some work to figure out what's going on! 

The blue stuff in the background is Saturn, with lines created by
shadows of rings.  The bright blue-white stripe near Mimas is sunlight
shining through a break in the rings called the "Cassini
division".  The brownish stuff near the bottom is the A ring -
you can see right through it.  Above it there's a break and a thinner
ring called the F ring.  Below it is the Cassini division itself.

This is just one of many photos taken by Cassini and Huyghens, the probe 
that Cassini dropped onto Saturn's moon Titan - see "<A HREF = "week210.html">week210</A>" for more on
that.  You can see more of these photos here:

3) NASA, Cassini-Huygens, <a href = "http://saturn.jpl.nasa.gov/">http://saturn.jpl.nasa.gov/</a>

I hope you see from these beautiful images, and others on This Week's
Finds, that we are \emph{already in space}.  We don't need people up there
for us to effectively \emph{be there}.

Alas, not everyone recognizes this.  An expensive American program to set 
up a base on the Moon, perhaps as a stepping stone to a manned mission to 
Mars, is starting to drain money from more exciting unmanned missions.  
NASA guesses this program will cost $104 billion up to the time when we 
land on the Moon - again - in 2020.  By 2024, the Government Accounting 
Office guesses the price will be $230 billion.  By comparison, the 
Cassini-Huygens mission cost just about $3.3 billion.

And what will be benefits of a Moon base be?  It's unclear: at best,
some vague dream of "space colonization".

Mind you, I'm in favor of space exploration, and even colonization.  
But, these are very different things!  

Colonies are usually about making money.  Governments support them 
in hopes of turning a profit: think Columbus and Isabella, or other
adventurers funded by colonial powers.  

Right now most of the money lies in near-earth orbit, not on the Moon
and Mars.  Telecommunication satellites and satellite photos are
established businesses.  The next step may be tourism.  Dennis Tito,
Gregory Olsen and Mark Shuttleworth have already paid the Russian
government $20 million each to visit the International Space Station.
This orbits at an altitude of about 350 kilometers, in the upper
"thermosphere" - the layer of the Earth's atmosphere where
gases get ionized by solar radiation.

If this is too pricey for you, wait a few years.  Richard Branson's 
company Virgin Galactic plans to give 500 people per year a 7-minute 
experience of weightlessness at a cost of just $200,000 each.  Alas, 
you'll only go up 100 kilometers, near the bottom of the thermosphere.  

Some competition may lower the price.  Jeff Bezos, the founder of Amazon, 
has bought a lot of land in Texas to a build space port for his company
Blue Origins.  He wants to do test flights by next year, and he eventually
wants 50 flights a year in a vehicle that holds 3.  If you've always looked 
forward to using your seat cushion as a flotation device in the event 
of a water landing, you'll love this:

\begin{quote}
  "During an abort situation, the crew capsule would separate, using 
  small solid-rocket motors to safely recover the space flight participants. 
  The abort module containing the solid-rocket motors would then jettison 
  from the crew capsule."  
\end{quote}

None of this stuff requires any taxpayer funding.  It's a bit
self-indulgent and silly, but it may eventually grow and merge with
other profit-making forms of space colonization.

Exploration is a bit different: seeing what's out there, mainly for the 
sake of adventure and understanding.  For this we should send machines, 
not people.  Machines can be designed to do well in vacuum.  People can't - 
not yet.  This will probably change when nanotech, AI and cyborg technologies
kick in.  But for now, unmanned probes are the way to go.

Here are some of the wonderful things we could do, all for
less than setting up a Moon base: 

4) The Laser Interferometer Space Antenna (LISA), <a href = "http://lisa.jpl.nasa.gov/">http://saturn.jpl.nasa.gov/</a>

The idea of LISA is to put 3 satellites in a huge equilateral triangle
following the Earth in its orbit around the Sun, and bounce lasers
between them to detect gravitational waves (see "<A HREF =
"week143.html">week143</A>"):

<div align = center>
<a href = "http://lisa.jpl.nasa.gov/STRATEGY/getThere.html">
<img style = "border:none;" width = 500 src = "LISA_orbit.gif">
% </a>
</div>

This would avoid the ground noise that plagues LIGO (see "<A HREF
= "week241.html">week241</A>"), and it could detect waves of much
lower frequencies.  If all works well, it could see gravitational
waves from the \emph{very} early Universe, long before the hot gas enough
cooled to let light through.  We're talking times like 10^{-38}
seconds after the Big Bang!  That's the biggest adventure I can
imagine... back to the birth pangs of the Universe.

Right now LISA is scheduled for launch around 2016.  But as you'll soon
see, this may not happen.


5) Constellation-X, <a href = "http://constellation.gsfc.nasa.gov/">
http://constellation.gsfc.nasa.gov</a>

This would be a team of X-ray telescopes, combining forces to be 100
times more powerful than any previous single one.  Among many other things,
Constellation-X could study the X-rays emitted by matter falling into things 
that look like black holes.  The redshift of these X-rays is our best test 
of general relativity for very strong gravitational fields.  So, it's our
best way of checking that these black hole candidates really do have event 
horizons!

In February 2006, when NASA put out their latest budget, they said
Constellation-X would be "delayed indefinitely".  And in
September 2006, a National Research Council committee was formed to
pick \emph{one} of NASA's five "Beyond Einstein" programs
for the first shot at funding: LISA, Constellation-X, the Joint Dark
Energy Mission, the Inflation Probe and the Black Hole Finder.
Currently the Joint Dark Energy Mission seems to be in the lead:

6) Steinn Sigur&eth;sson, NASA: double down on science, Dynamics of
Cats, September 16, 2006, <a href = "http://scienceblogs.com/catdynamics/2006/09/nasa_double_down_on_science.php">http://scienceblogs.com/catdynamics/2006/09/nasa_double_down_on_science.php</a>

A decision is expected around September 2007.

7) The Terrestrial Planet Finder (TPF),
<a href = "http://planetquest.jpl.nasa.gov/TPF/">http://planetquest.jpl.nasa.gov/TPF/</a>

This could study Earth-like planets orbiting stars up to 45 light
years away.  It would consist of two observatories: a visible-light
"coronagraph" that blocks out the light from a star so it
can see nearby fainter objects:

<div align = center>
<a href = "http://planetquest.jpl.nasa.gov/TPF/">
<img src = "tpf_coronagraph.jpg">
% </a>
</div>
<br>

and an infrared interferometer made of
several units flying in formation:

<br>
<div align = center>
<a href = "http://planetquest.jpl.nasa.gov/TPF/">
<img src = "tpf_formation_flying_infrared_interferometer.jpg">
% </a>
</div>

In February 2006, NASA halted work on the TPF.  In June 2006, thanks
to public pressure, Congress reinstated funding for this
program and also a mission to Jupiter's moon Europa, which could have
oceans underneath its icy crust.  However, at last report, NASA was
continuing to fight \emph{against} reinstating these missions:

8) Louis D. Friedman, Congressional inaction leaves science still
devastated, The Planetary Society, November 26, 2006, 
<a href = "http://planetary.org/programs/projects/sos/20061122.html">http://planetary.org/programs/projects/sos/20061122.html</a>

The constantly shifting situation makes it hard to know what's
going on.

9) The Nuclear Spectroscopic Telescope Array (NuStar),
<a href = "http://www.nustar.caltech.edu/">http://www.nustar.caltech.edu/</a>

This is an orbiting observatory with three telescopes, designed to see 
hard X-rays.   It could conduct a thorough survey of black hole candidates
throughout the universe.  It could study relativistic jets of particles 
from the cores of active galaxies (which are probably also black holes).
And, it could study young supernova remnants - hot new neutron stars.

NASA suddenly canceled work on NuStar in February 2006.

10) Dawn, <a href = "http://dawn.jpl.nasa.gov/">http://dawn.jpl.nasa.gov/</a>

The Dawn mission seeks to understand the early Solar System by probing
the asteroid belt and taking a good look at Ceres and Vesta.  <a href
= "http://en.wikipedia.org/wiki/Ceres_%28dwarf_planet%29">Ceres</a> is 
the largest asteroid of all, 950 kilometers in diameter.  It
seems have a rocky core, a thick mantle of water ice, and a thin dusty
outer crust.  <a href = "http://en.wikipedia.org/wiki/4_Vesta">Vesta</a> is the second
largest, about 530 kilometers in diameter.  It's very different from
Ceres: it's not round, and it's all rock.  A certain group of stony
meteorites called "<a href =
"http://en.wikipedia.org/wiki/HED_meteorite">HED meteorites</a>"
are believed to be pieces of Vesta!

NASA cancelled the Dawn mission in March 2006 - but later that month,
they changed their minds.  

It's depressing to contemplate all the wonderful things we could miss
while spending hundreds of billions to "send canned primates to
Mars", as Charles Stross so cleverly put it in his novel
"Accelerando" (see "<A HREF = "week222.html">week222</A>").
I'm all for humanity spreading through space.  I just don't think we
should do it in a clunky, low-tech way like setting up a base on the
Moon where astronauts sit around and... what, play golf?  It's like
something out of old science fiction!

To cheer myself up again, here's a picture of the Sun:

<div align = center>
<a href = "http://cosmicvariance.com/2006/10/13/sun-shots/">
<img width = 400 src = "sun_neutrinos.jpg">
% </a>
</div>

11) Joanne Hewett, Sun Shots, <a href = "http://cosmicvariance.com/2006/10/13/sun-shots/">http://cosmicvariance.com/2006/10/13/sun-shots/</a>

It was taken not with light, but with \emph{neutrinos}.  It was made at
the big neutrino observatory in Japan, called Super-Kamiokande.  It
took about 504 days and nights to make.  

That's right - nights! Neutrinos go right through the Earth.

As you probably know, neutrinos oscillate between three different 
kinds, but only electron neutrinos are easy to detect, so we see about
third as many neutrinos from the Sun as naively expected.  That's
the kind of thing they're studying at Super-Kamiokande.

But what I want to know is: what's the "glare" in this picture?  
Neutrinos are made by the process of fusion, which involves this 
reaction:

proton + electron \to  neutron + electron neutrino

Fusion mostly happens in the Sun's core, which has a density of 160 
grams per cubic centimeter (10 times denser than lead) and a temperature 
of 15 million kelvin (300 thousand times hotter than 
the "broil" setting on an American oven).  

So, what's the disk in this picture: the whole Sun, or the Sun's core?
And what's the glare?  

Okay, now for some serious mathematical physics:

12) Jeffrey Morton, A double bicategory of cobordisms with corners,
available as <A HREF = "http://arXiv.org/abs/math.CT/0611930v1">math.CT/0611930v1</A>. 

People have been talking a long time about topological quantum field 
theory and higher categories.  The idea is that categories, 2-categories, 
3-categories and the like can describe how manifolds can be chopped into 
little pieces - or more precisely, how these little pieces can be glued 
together to form manifolds.  Then the problem of doing quantum field 
theory on some manifold can be reduced to the problem of doing it on 
these pieces and gluing the results together.  This works easiest if 
the theory is "topological", not requiring a background metric.

There's a lot of evidence that this is a good idea, but getting the details
straight has proved tough, even at the 2-category level.  This is what 
Morton does, in a rather clever way.  Very roughly, his idea is to use 
something I'll call a "weak double category", and prove that these:

<ul>
<li>
 (n-2)-dimensional manifolds
</li>
<li>
 (n-1)-dimensional manifolds with boundary
</li>
<li>
  n-dimensional manifolds with corners
</li>
</ul>

give a weak double category called nCob_{2}.  The proof is a
cool mix of topology and higher category theory.  He then shows that
this particular weak double category can be reinterpreted as something
a bit more familiar - a "weak 2-category".

In the rest of his thesis, Jeff will use this formalism to construct
some examples of "extended TQFTs", which are roughly maps of
weak 2-categories

Z: nCob_{2} \to  2Vect

where 2Vect is the weak 2-category of "2-vector spaces".  He's
focusing on some extended TQFTs called the Dijkgraaf-Witten models,
coming from finite groups.  

But, he's also thought about the case where the finite group is
replaced by a compact Lie group.  In this case we get something 
called BF theory, which is a lot like an extended TQFT, but not quite,
because there are some divergences (infinities) that arise. 
In this case of 3d spacetime with the Lie group SU(2), BF theory gives
a nice theory of quantum gravity called the Ponzano-Regge model.
And, as I hinted back in "<A HREF =
"week232.html">week232</A>", we can let 2d space in this model be
a manifold with \emph{boundary} by poking little holes in space.
Then these holes wind up acting like particles!

So, we get a relation like this:


\begin{verbatim}

 (n-2)-dimensional manifolds                     MATTER

 (n-1)-dimensional manifolds with boundary       SPACE

  n-dimensional manifolds with corners           SPACETIME
\end{verbatim}
    

I like this a lot: it reminds me of the title of Weyl's famous
book "Raum, Zeit, Materie", meaning "Space, Time, Matter".  
He never guessed this trio was related to the objects, morphisms and
2-morphisms in a weak 2-category!  It's too bad we can't seem to
get something like this to work for full-fledged quantum gravity.

It would be fun to talk more about this.  However, to understand Morton's 
work more deeply, you need to understand a bit about "weak double 
categories".  He explains them quite nicely, but I think I'll spend 
the rest of this Week's Finds giving a less detailed introduction, just
to get you warmed up.

This chart should help:


\begin{verbatim}

                                   BIGONS                SQUARES

   LAWS HOLDING                    strict                strict 
   AS EQUATIONS                 2-categories        double categories

   LAWS HOLDING                    weak                  weak
 UP TO ISOMORPHISM              2-categories        double categories
\end{verbatim}
    

2-categories are good for describing how to glue together
2-dimensional things that, at least in some abstract sense, are shaped
like \emph{bigons}.  A "bigon" is a disc with its boundary
divided into two halves.  Here's my feeble ASCCI rendition of a bigon:


\begin{verbatim}

                      f 
                   --->---
                  /       \
                 /   ||    \
              X o    ||B    o Y
                 \   \/    /
                  \       /
                   --->---
                      g
                     
\end{verbatim}
    
The big arrow indicates that we think of the bigon B as "going from" 
the top semicircle, f, to the bottom semicircle, g.  Similarly, we 
think of the arcs f and g as going from the point X to the point Y.  

Similarly, double categories are good for describing how to glue together 
2-dimensional gadgets that are shaped like \emph{squares}:


\begin{verbatim}

                      f
               X o---->----o X'
                 |         |
               g v    S    v g'
                 |         |
               Y o---->----o Y' 
                      f'
\end{verbatim}
    

Both 2-categories and double categories come in "strict" and
"weak" versions.  The strict versions have operations
satisfying a bunch of laws "on the nose", as equations.  In
the weak versions, these laws hold up to isomorphism whenever
possible.

A few more details might help....

A 2-category has a set of objects, a set of morphisms f: X \to  Y going 
from any object X to to any object Y, and a set of 2-morphisms T: f => g 
going from any morphism f: X \to  Y to any morphism g: X \to  Y.  We can 
visualize the objects as dots: 


\begin{verbatim}

                   o 
                   X
\end{verbatim}
    
the morphisms as arrows:


$$

                   f           
            X o---->----o Y
$$
    
and the 2-morphisms as bigons:


\begin{verbatim}

                   f 
                --->---
               /       \
              /   ||    \
           X o    ||B   o Y
              \   \/    /
               \       /
                --->---
                   g
\end{verbatim}
    

We can compose morphisms like this:


\begin{verbatim}

            f         g                                  fg
       o---->----o---->----o             gives       o--->---o
       X         Y         Z                         X       Z
\end{verbatim}
    

We can also compose 2-morphisms vertically:


\begin{verbatim}

                 f                                          f
             ---->----                                 --->---  
            /  S      \                               /       \
           /     g     \                             /         \
        X o ----->----- o Y              gives    X o     ST    o Y
           \   T       /                             \         /
            \         /                               \       /   
             ---->----                                 --->---
                 h                                        h
\end{verbatim}
    
and horizontally:


\begin{verbatim}

             f           f'                              ff'
          --->---     --->---                          --->--- 
         /       \   /       \                        /       \
        /         \ /         \                      /         \
     X o     S     o     T     o Z       gives    X o    S.T    o Z
        \         / \         /                      \         /
         \       /   \       /                        \       / 
          --->---     --->---                          --->---
             g           g'                              gg'
\end{verbatim}
    

There are also a bunch of laws that need to hold.  I don't want to
list them; you can find them in Jeff's paper (also see "<A HREF =
"week80.html">week80</A>").  I just want to emphasize how a
strict 2-category is different from a weak one.

In a strict 2-category, the composition of morphisms is associative
on the nose:

(fg)h = f(gh)

and there are identity morphisms that satisfy these laws on the nose:

1f = f = f1 

In a weak 2-category, these equations are replaced by 2-isomorphisms - that 
is, invertible 2-morphisms.  And, these 2-isomorphisms need to satisfy new 
equations of their own!

What about double categories?

Double categories are like 2-categories, but instead of bigons, we have 
squares.  

More precisely, a double category has a set of objects:


\begin{verbatim}

                   o 
                   X
\end{verbatim}
    
a set of horizontal arrows:


$$

                   f           
            X o---->----o X'
$$
    
a set of vertical arrows:


\begin{verbatim}

            X o
              |
            g v
              |
            Y o
\end{verbatim}
    
and a set of squares:


\begin{verbatim}

                f
         X o---->----o X'
           |         |
         g v    S    v g'
           |         |
         Y o---->----o Y' 
                f'
\end{verbatim}
    
We can compose the horizontal arrows like this:


\begin{verbatim}

            f         f'                                  f.f'
       o---->----o---->----o             gives         o--->---o
       X         Y         Z                           X       Z
\end{verbatim}
    
We can compose the vertical arrows like this:


\begin{verbatim}

      X o
        |
      g v                                                X o
        |                                                  |
      Y o                                gives         gg' v
        |                                                  |
     g' v                                                Z o
        |
      Z o
\end{verbatim}
    
And, we can compose the squares both vertically: 


\begin{verbatim}

             f
      X o---->----o X'
        |         |                                        f
      g v    S    v g'                              X o---->----o X'
        |         |                                   |         |
      Y o---->----o Y'                  gives     gh  v   SS'   v g'h'
        |         |                                   |         |
      h v    S'   v h'                              Z o---->----o Z'
        |         |                                        f'
      Z o---->----o Z'
             f'
\end{verbatim}
    
and horizontally:


\begin{verbatim}

                f    Y    g                                 f.g
         X o---->----o---->----o Z                    X o---->----o Z
           |         |         |                        |         |
         h v    S    v    S'   v h'                   h v   S.S'  v h'  
           |         |         |                        |         |
        X' o---->----o---->----o Z'                  X' o---->----o Z' 
                f'   Y'   g'                               f'.g'
\end{verbatim}
    
In a strict double category, both vertical and horizontal composition 
of morphisms is associative on the nose:

(fg)h = f(gh)             

(f.g).h = f.(g.h)

and there are identity morphisms for both vertical and horizontal 
composition, which satisfy the usual identity laws on the nose.

In a weak double category, we want these laws to hold only up to 
isomorphism.  But, it turns out that this requires us to introduce
bigons as well!  The reason is fascinating but too subtle to explain
here.  I didn't understand it until Jeff pointed it out.  But, it 
turns out that Dominic Verity had already introduced the right concept 
of weak double category - a gadget with both squares and bigons - in 
\emph{his} Ph.D. thesis a while back:

13) Dominic Verity, Enriched categories, internal categories, and
change of base, Ph.D. dissertation, University of Cambridge, 1992.

Interestingly, if you weaken \emph{only} the laws for vertical
composition, you don't need to introduce bigons.  The resulting
concept of "horizontally weak double category" has been
studied by Grandis and Pare:

14) Marco Grandis and Bob Par&eacute;, Limits in double categories, Cah.
Top. Geom. Diff. Cat. 40 (1999), 162-220.  Also available at
<a href = "http://www.dima.unige.it/~grandis/Dbl.Cahiers.pdf">http://www.dima.unige.it/~grandis/Dbl.Cahiers.pdf</a>

Marco Grandis and Bob Par&eacute;, Adjoints for double categories, Cah. 
Top. Geom. Diff. Cat. 45 (2004), 193-240.  Also available at
<a href = "http://www.dima.unige.it/~grandis/Dbl.Adj.pdf">http://www.dima.unige.it/~grandis/Dbl.Adj.pdf</a>

and more recently by Martin Hyland's student Richard Garner:

15) Richard Garner, Double clubs, available as <A HREF =
"http://xxx.lanl.gov/abs/math.CT/0606733">math.CT/0606733</A>

and Tom Fiore:

16) Thomas M. Fiore, Pseudo algebras and pseudo double categories,
available as <A HREF = "http://xxx.lanl.gov/abs/math.CT/0608760">
math.CT./0608760</a>.

At this point I should admit that the terminology in this whole
field is a bit of a mess.  I've made up simplified terminology 
for the purposes of this article, but now I should explain how it
maps to the terminology most people use:


\begin{verbatim}

   ME                                  THEM

strict 2-category                   2-category
weak 2-category                     bicategory
strict double category              double category
weak double category                double bicategory
horizontally weak double category   pseudo double category
\end{verbatim}
    

Verity used the term "double bicategory" to hint that his gadgets
have both squares and bigons, so they're like a blend of double 
categories and bicategories.  It's a slightly unfortunate term, since
experts know that a double category is a category object in Cat, but
Verity's double bicategories are not bicategory objects in BiCat.  
Morton mainly uses Verity's double bicategories - but in the proof of
his big theorem, he also uses bicategory objects in BiCat.  

There's a lot more to say, but I'll stop here and let you read
the rest in Jeff's paper!

\par\noindent\rule{\textwidth}{0.4pt}
\textbf{Addenda:} I thank Charlie Clingen for catching some typos
in my diagrams, and Nathan Urban and Torbj&ouml;rn Larsson for helping me
update some of my information on the funding of NASA programs.  I
won't attempt to keep this information up to date, since it's changing
too often.  But, I'd like to it be correct as of the date I wrote it!

Sean Carroll writes:

\begin{quote}
Hi John-- 

Just a couple of comments on This Week's Finds--

You mention a bunch of missions that could "probably" be funded for
the cost of a Moon base.  That's being quite conservative!  Each of
those missions is about $1 billion or less, while the Moon base is
upwards of $200 billion.

And you asked about the neutrino image of the Sun.  The "haze" is just
an imaging problem, not a feature of the Sun; the resolution of this
image is worse than 10 degrees (I forget the exact number), so we're
certainly not looking at any substructure inside the Sun (whose entire
disk is only half a degree wide).

<a href = "http://preposterousuniverse.com/">Sean</a>
\end{quote}

I've deleted the word "probably".  According to
a comment on Joanne Hewett's blog entry, each pixel in the
neutrino picture of the Sun is one degree in size.  The 
Sun itself is just half a degree wide.

For more discussion, go to the <a href = "http://golem.ph.utexas.edu/category/2006/12/this_weeks_finds_in_mathematic_4.html">\emph{n}-Category
Caf&eacute;</a>.

\par\noindent\rule{\textwidth}{0.4pt}
% </A>
% </A>
% </A>
