
% </A>
% </A>
% </A>
\week{August 8, 2003}


I've been away from This Week's Finds for a long time, so I have 
a lot to talk about... so much that I scarcely know where to begin!  

In June I went to a big general relativity conference at Penn State, 
and I have a lot to say about that, but at the end of July I went to 
two conferences in Lisbon, and I want to talk about those a bit now.

One was a workshop on "categorification and higher-order geometry".  
This was run by Roger Picken and Marco Mackaay, and it brought together 
a bunch of people interested in how n-categories are affecting our 
notions of geometry.  If you're interested in this, you might enjoy
looking at the talk titles here:

1) Workshop on categorification and higher-order geometry, 
<A HREF = "http://www.math.ist.utl.pt/~rpicken/CHOG2003">
http://www.math.ist.utl.pt/~rpicken/CHOG2003</A>

The other was the "Young Researcher's Symposium", a section of 
the International Congress of Mathematical Physics.  This symposium 
allows old geezers to pass on their accumulated wisdom to young 
researchers before they go senile and forget it all.  The youngsters 
also give talks, but I was invited as one of the old geezers.  It's 
a bit scary!   

Anyway, at these conferences I learned some cool stuff about 
elliptic cohomology from Stephan Stolz, and also some cool stuff 
about "Monstrous Moonshine" from Terry Gannon.  It turns out they're
more related than I realized - and the relation involves string 
theory!  I always love it when two things I'm studying turn out 
to be related.  So, I'd like to tell you about this stuff... 
before I forget it. 

I gave a very sketchy introduction to elliptic cohomology in 
"<A HREF = "week149.html">week149</A>" and "<A HREF = "week150.html">week150</A>".  One reason I'm interested in this subject 
is that it seems to be a categorified version of something topologists
are already fond of: K-theory.  In K-theory, you study a space 
by looking at all the vector bundles over this space.  By trying to
categorify the concept of "vector space", Kapranov and Voevodsky 
were led to the concept of "2-vector space", which is a category
that acts sort of like a vector space.  You can think of elliptic
cohomology as a souped-up version of K-theory where you study a
space by looking at all the "2-vector bundles" on it! 

I'll warn you right away, this isn't how \emph{most} people think 
about elliptic cohomology - this is a fairly new approach due to 
Nils Baas, Bjorn Dundas and John Rognes.  Most people think of 
elliptic cohomology as being related to string theory.  But the 
two viewpoints seem to be compatible.

Here's why: if you have a connection on a vector bundle, it gives 
a way to parallel transport a vector along a curve.  People use this 
to study how the state of a point particle changes when you move it 
around in a "gauge field" - which is just physics talk for 
a connection.

So now let's imagine you categorified this whole story.  If you 
had a connection on a 2-vector bundle, and you believe that 
categorification increases the dimensions of things by one -
which it often does - you might hope that this connection would 
tell you how to do parallel transport over a \emph{2d} surface!  And 
this in turn might tell you how \emph{strings} change state when you 
move them around.

Well, nobody has worked out all the details yet, but something like
this seems to be going on... and I want to know what it is!

I'd like to explain what Stephan Stolz told me about this.  I have 
to warn you, though: this stuff applies to a \emph{new improved version}
of elliptic cohomology, which became popular after the one I was 
talking about in previous Weeks.  
Some of the old stuff I said no longer applies to 
this new version.  To minimize confusion, people call this new 
version the theory of "topological modular forms".   

So, what is this thing?

First of all, it's a generalized cohomology theory.  

Hmm.  To make sure you understand that sentence, I need to give
the world's quickest course on generalized cohomology theories.  
For a more leisurely introduction see 
"<A HREF = "week149.html">week149</A>".

Here goes: 

A "spectrum" is an infinite list of spaces E(n) where n ranges over 
all integers, such that each space in the list is the space of 
loops in the next space on the list.  Given any space X, we can 
define the "generalized cohomology groups" of X to be 


$$

h^{n}(X) = [X,E(n)]
$$
    
where [X,E(n)] is the set of all homotopy classes of maps from
X to E(n).  Thanks to the magic of loops, these sets are actually
abelian groups.  

If you know about the good old familar "ordinary" cohomology groups 
H^{n}(X) of a space X, you'll be pleased to know that these are an 
example of a generalized cohomology theory.  You'll also be happy 
to know that lots of the basic theorems about ordinary cohomology 
theory hold for these generalized ones.  The main one that \emph{doesn't}
hold is the one that says: 


$$

H^{n}(point) = Z  if n = 0
^{ }           0  otherwise
$$
    
For a generalized cohomology theory, the cohomology of a point
can be more interesting!  In particular, if E(n) is something called
a "ring spectrum", the groups h^{n}(point) will form a 
graded ring.
This happens in a lot of interesting examples.

Okay, now you're an expert on generalized cohomology theories.

As I said, the theory of "topological modular forms" is one of
these things.  So, to completely describe it, I just need to
give you an infinite list of spaces tmf(n) forming a spectrum.  
Then for any space X we can define a list of abelian groups 


$$

tmf^{n}(X) = [X,tmf(n)]
$$
    
and we're off and running.  By the way, don't be freaked out that 
now I'm using the same name for the spectrum and the generalized 
cohomology theory it gives - people do this a lot.

Unfortunately, at present it's a lot of work to define these 
spaces tmf(n).  Mike Hopkins and Haynes Miller figured out how,
and it was a great achievement:

2) Michael J. Hopkins, Topological modular forms, the Witten
genus, and the theorem of the cube, in Proceedings of the 
International Congress of Mathematicians (Zurich, 1994), 
Birkhauser, Basel, 1995, pp. 554-565.

But, they used a lot of heavy-duty algebraic topology that 
simple-minded folks like me have almost no chance of understanding.  

Fortunately, Stephan Stolz told me what people secretly think
these spaces must be!  Nobody has proved this yet or even
made it into a precise conjecture, but it's so audacious - 
and it would explain so much - that I can't resist saying it:

\begin{quote}<B>
  tmf(n) is the space of supersymmetric conformal field theories 
                   of central charge -n.
</B>\end{quote}
There's a lot of fine print here that I'm leaving out, and some 
that nobody even knows... but a "supersymmetric conformal field 
theory" is sort of roughly like a "superstring vacuum": a world
in which superstrings can romp and play.  This is oversimplified
and it will piss off string theorists, but never mind, right now 
I'm just trying to make a very crude point: the theory of 
topological modular forms is sort of like studying a space by 
mapping it into the space of all possible superstring vacua!

Zounds!

Before we blow our minds contemplating the space of all superstring
vacua, let me back off a bit and try to explain what any of this 
has to do with "modular forms".  Modular forms are a famous old 
concept 
from complex analysis.  These days people do complex analysis not just 
on the complex plane but on more general Riemann surfaces, and this 
turns out to be crucial for understanding modular forms.  We also use 
these surfaces to describe the "worldsheets" traced out in spacetime 
by the motion of a strings.  So, it should not come as a shock that 
modular forms should show up in a generalized cohomology theory involving 
strings!  But I'd like to make this connection considerably more precise.
  
To do this, I'll reveal that the spectrum for topological modular 
form theory is a ring spectrum, and the abelian groups


$$

tmf^{n}(point) 
$$
    
fit together in a very famous graded ring: it's the ring of 
MODULAR FORMS!  

Well, at least after we tensor it with the complex numbers,
it is... but before we worry about that, I should say what 
modular forms are.

I'll start with a quick but unenlightening definition.
First, a "modular function of weight n" is an analytic 
function on the upper half of the complex plane, say


$$

f: H \to  C                      
$$
    
where H is the upper half-plane, which transforms as follows:


$$

f((az+b)/(cz+d)) = (cz+d)^{n} f(z)
$$
    
for all matrices of integers


\begin{verbatim}

(a b) 
(c d)
\end{verbatim}
    
having determinant 1.  Then, we say a modular function is a
"modular form" if it doesn't blow up as you march up
the upper half-plane to the point at infinity.  

There are only nonzero modular forms when the weight
is a natural number.  It's easy to see that these form a
graded ring: if you add two modular forms of weight n 
you get another one of weight n, and if you multiply two
modular forms of weights n and n', you get one of weight n+n'. 

This graded ring is the same as what you get by tensoring the 
graded ring tmf^{n}(point) by the complex numbers!

In case you're wondering what this "tensoring with the complex 
numbers" business is all about: it's mainly just a way of killing 
off elements of a group that become zero when you multiply them by 
some integer.  If you're a topologist these so-called "torsion 
elements" are really interesting.  They make topological modular
forms a lot more subtle than traditional modular forms as defined 
above.  Topologists really go into raptures over torsion!  But if 
you're a lowly mathematical physicist such as myself, struggling to 
understand even a little of what's going on, you go ahead and kill 
the torsion by tensoring with C.  And, I'm pretty sure the new
"topological modular form" theory is the same as the old version
of elliptic cohomology except for stuff involving torsion.

So, ignoring these subtleties, let's just say that tmf is a 
generalization of cohomology theory in which the integers get 
replaced by the modular forms when we calculate the cohomology 
of a point... where modular forms are some weird functions that 
show up in complex analysis! 

But what does this have to do with the idea that tmf is related
to the space of all string theories?

To understand this, we need a better understanding of modular forms: 
we need to see how they're related to "elliptic curves", 
and we need to see how these are related to conformal field theory.  
Then things will start to make sense.    

To do this, let's start with the world's quickest course on 
elliptic curves.  For a more leisurely introduction, see "<A HREF = "week13.html">week13</A>", 
"<A HREF = "week125.html">week125</A>", and "<A HREF = "week126.html">week126</A>".

An "elliptic curve" is what you get when you take the
complex plane and mod out by a lattice, like this:

                  

\begin{verbatim}

                *       *      *      *

                             
                    *      0*      *                


                *       *      *      *
\end{verbatim}
    
Topologically you get a torus, of course.  But it also
has the structure of an abelian group, coming from addition
in the complex plane.  It also has the structure of a compact
Riemann surface - that is, a compact 1-dimensional complex manifold. 
So, a more precise definition of an elliptic curve is that it's 
an abelian group in the category of compact Riemann surfaces.

With this definition, it turns out that we can rotate or dilate
our lattice without changing the elliptic curve we get from it.  
More precisely, we get an \emph{isomorphic} elliptic curve.  So, any 
elliptic curve is isomorphic to one coming from a lattice like this:



\begin{verbatim}

                                   z         z + 1 
                      *            *           *
 
                                0           1
                   *            *           *
               
 
                *           *           *
\end{verbatim}
    
where z is in the upper half-plane.  

But, lots of different choices of z give the same elliptic 
curve!  For example, we can replace z by z + 1 and still get
the same lattice, hence the same elliptic curve.  We can also 
replace z by -1/z.  This turns the short squat right-leaning 
parallelogram in the above picture into a tall skinny left-leaning 
one - but after rotating and dilating this, we get back the 
parallelogram we started with, so we get the same elliptic curve.

In fact, though it's not obvious from \emph{this} way of thinking
about the problem, it's easy to show that all the different 
choices of z that give the same elliptic curve are related by 
these two transformations.

Now, the group of transformations of the upper half-plane 
generated by 


$$

z |-> z + 1
$$
    
and 


$$

z |-> -1/z
$$
    
is precisely the group of all transformations


$$

z |-> (az+b)/(cz+d)
$$
    
where the matrix 


\begin{verbatim}

(a b)
(c d)
\end{verbatim}
    
has determinant 1.  This group of such transformations is
called PSL(2,Z).  So, the space of all isomorphism classes 
of elliptic curves is


\begin{verbatim}

H/PSL(2,Z)
\end{verbatim}
    
where again H is the upper half-plane.  Folks call this
space the "moduli space of elliptic curves".  It's a Riemann 
surface, and I drew a picture of 
it in "<A HREF = "week125.html">week125</A>".

Okay, now you're an expert on elliptic curves.

A while back, I defined a "modular function of weight n" to be an 
analytic function on the upper half-plane


$$

f: H \to  C                  
$$
    
such that


$$

f((az+b)/(cz+d)) = (cz+d)^{n} f(z)
$$
    
for all transformations in PSL(2,Z).  Now we can see what
this equation really means.  When n = 0, it just says f is 
\emph{invariant} under PSL(2,Z), so it becomes a function on H/PSL(2,Z).
Thus, modular functions of weight 0 are just analytic functions 
on the moduli space of elliptic curves!  

So, if you're trying to explain modular functions to your friends,
just tell them they're functions that depend on the shape of a 
doughnut - what could be simpler than that?  Of course "shape"
needs to be interpreted in a subtle way to make this true.

Similarly, a modular function is a "modular form" if it doesn't 
blow up when Im(z) \to  +\infty , which means that it doesn't 
blow up when your doughnut gets really long and skinny, more
like a circle than an honest doughnut.   The 
circle is like the ultimate low-calorie doughnut.  In the language of 
string theory, where the surface of your doughnut is the worldsheet 
of a string, the limit Im(z) \to  +\infty  corresponds to 
the "particle limit", where 
the worldsheet of the string degenerates to the worldline of a particle.

Of course, when n is nonzero, modular forms of weight n aren't 
really invariant under PSL(2,Z): they're only invariant "up to 
a phase".  I put this physics jargon in quotes because the fudge 
factor (cz+d)^{n} isn't really a unit complex number.  But the moral 
principle is the same - and in string theory, this fudge factor 
really \emph{does} come from a quantum mechanical phase ambiguity, 
called the "conformal anomaly".  

(To make this "up to a phase" idea precise, we can think of 
modular forms of weight n as sections of some \emph{line bundle} 
on the \emph{moduli stack} of elliptic curves... but I explained this
already in "<A HREF = "week125.html">week125</A>", and I don't 
want to say more about it now.)

Now that we understand modular forms a bit better, we can
begin to vaguely see why 


$$

tmf^{n}(point) tensor C
$$
    
is the space of modular forms of weight n. 

Here's how.  If you know a little about the path-integral approach
to quantum field theory, you'll know that one of the basic things
you compute in any quantum field theory is a number called the
"partition function".  You'll also know that this number is often
infinite, or defined only up to some ambiguities... that's why
quantum field theory is tough.  

So, given that a conformal field theory is something like a string 
theory, and given that the worldsheet of a string is a Riemann 
surface, you shouldn't be surprised that given any compact Riemann 
surface and any conformal field theory we can try to compute a 
number called the "partition function".  Nor should you be surprised 
that this "number" is sometimes afflicted with ambiguities!  

So, restricting attention to the case where our Riemann surface
is an elliptic curve, you should not be surprised that the 
partition function of any conformal field theory is a MODULAR FORM!  

If this modular form has weight 0, the partition function is an 
honest-to-goodness function on the moduli space of elliptic curves: 
for any elliptic curve the partition function is an actual number.  
But if the modular form has nonzero weight, the partition function 
is afflicted with "phase ambiguities" - where "phase" 
is in quotes for the same reason as before.  

In particular, if the partition function is a modular form
of weight n, we say our conformal field theory has "central
charge -n".  The central charge just tells us how the phase
ambiguity works... though some jerk put in a minus sign to confuse us.  

Now think what this implies!   Remember that tmf(n) is space 
of conformal field theories with central charge -n.  Since
the partition function of any such thing is a modular form of
weight n, we get a map


$$

Z: tmf(n) \to  {modular forms of weight n}
$$
    
This is a step towards seeing that 


$$

tmf^{n}(point) tensor C = {modular forms of weight n} 
$$
    
since at least there's a relation between the two sides!

To go further, use the definition of generalized cohomology:


$$

tmf^{n}(point) = [point,tmf(n)]
$$
    
and note that
 

\begin{verbatim}

[point,tmf(n)] 
\end{verbatim}
    
is the set of \emph{connected components} of the space of supersymmetric
conformal field theories of central charge -n.  So, we'd like to 
see why this is an abelian group, and why tensoring it with
the complex numbers gives the space of modular forms of weight n.

To see this, we'd just need to show four amazing things:

<UL>
<LI> The partition function doesn't change as we trace out
a continuous path in the space of conformal field theories
of central charge -n.  Thus, the partition function defines
a map


$$

Z: [point,tmf(n)] \to  {modular forms of weight n}
$$
    

<LI> The set of connected components of the space of conformal
field theories of central charge -n forms an abelian group, 
and the above map is a group homomorphism.

<LI> The kernel of the above homomorphism consists precisely of
the torsion elements, so we get a 1-1 homomorphism
 

$$

Z: [point,tmf(n)] tensor C \to  {modular forms of weight n}
$$
    
<LI> Any modular form of weight n is a linear combination of
partition functions of conformal field theories of central
charge -n, so the homomorphism


$$

Z: [point,tmf(n)] tensor C \to  {modular forms of weight n}
$$
    
is also onto.

</UL>
Sorry, I'm getting a little carried away... it's not good to 
put in so much detail when you're explaining stuff, but I 
just realized that we need these four amazing things to be true, 
and I couldn't resist writing them down.  Learning by teaching
is great for the teacher; sometimes less so for the student.  

Anyway: 

The first amazing thing must come from "index theory" and how the
"index of a Fredholm operator" doesn't change when we deform 
it continuously.  It must also use the fact that the partition
function we're talking about can be written as such an index.  This 
only happens because we're considering supersymmetric theories!
Stephan Stolz emphasized to me that we really need to be using 
"N = 1/2 supersymmetric conformal field theories"; I haven't
gotten around to understanding the N = 1/2 part.

The second amazing thing is not really amazing.  In fact, it's
easy to see the whole graded ring structure of modular forms
coming from operations on conformal field theories.  I'll explain
that in a minute.

The third amazing thing is a total mystery to me.  It's
obvious that all torsion elements must lie in the kernel of a
homomorphism from a group to a vector space, but it's utterly
mysterious why the kernel consists \emph{precisely} of the torsion
elements.

The fourth amazing thing is presumably some sort of calculation:
you just need to find enough conformal field theories to make
sure their partition functions generate the ring of modular
forms.  In fact, the ring of modular forms is generated by one
of weight 4 and one of weight 6: these are both "Eisenstein 
series", which are well-understood, so we just need someone to 
cook up conformal field theories having these as partition 
functions.  Does anyone reading this know how to do it?

(Irrelevant digression: The previous paragraph implies that 
all nonzero modular forms have \emph{even weight}.  To correct for
this, some people stick in a factor of 1/2 when defining the 
weight of a modular form.  I mention this only so you're 
forewarned when you read the literature.)
 
Okay, let me round off this story by saying a little about 
how you add and multiply conformal field theories... and why.

A "conformal field theory" assigns a Hilbert space to any 
compact oriented 1-manifold, and a linear operator going
between Hilbert spaces to any Riemann surface with boundary 
going between such 1-manifolds.  There are a bunch of axioms
it needs to satisfy, invented by Graeme Segal.  I won't list 
these here, but the category theorists among you will quiver
with delight upon learning that the most important of these
axioms say a conformal field theory is a "symmetric monoidal 
functor".  

Anyway, it's easy to take direct sums and tensor products of 
Hilbert spaces and also operators.  This gives a way of 
defining the direct sum and tensor product of conformal 
field theories.  When we take the direct sum of conformal 
field theories their partition functions add.  When we take 
their tensor product the partition functions multiply.  So, 
these operations on conformal field theories correspond 
precisely to the graded ring structure on modular forms!

To see why this graded ring structure is interesting in
string theory, I should be more precise about the relation
between string theory and conformal field theory.  Perturbatively, 
string theory in a given background is described by a conformal 
field theory.  We can use this to calculate an operator for 
any Riemann surface with boundary: we think of this operator as
saying how the string changes state given the conformal structure
on its worldsheet.  When a conformal field theory plays this role 
we call it a "string vacuum".   

But, not any old conformal field theory will serve as a string vacuum!
It has to be one with central charge 0, in order to have a partition 
function without any ambiguities.   If the central charge is nonzero
we say there's a "conformal anomaly" and turn up our noses in 
disgust.  However, people often build conformal field theories with central 
charge 0 out of ones with nonzero central charge.  The simplest ways 
to build new conformal field theories from old are direct sums and 
tensor products.  So, the graded ring structure on modular forms is 
sort of lurking around in string theory!

To learn more about elliptic cohomology and its relation
to conformal field theory, you should read this paper that
Stephan Stolz is in the process of writing with Peter Teichner:

3) Stephan Stolz and Peter Teichner, What is an elliptic object?
Available at
<A HREF = "http://math.ucsd.edu/~teichner/papers.html">http://math.ucsd.edu/~teichner/papers.html</A>

This paper is almost 80 pages long and they aren't even
done yet!  The main goal is to define a concept of "elliptic
object" on a space X such that tmf^{n}(X) is built from
formal differences of elliptic objects with central charge n over X, just as
the K-theory of X is built from formal differences of vector
bundles over X.  In fact you can built K-theory using formal
differences of vector bundles \emph{equipped with a connection},
and an elliptic object is really a categorified version of
a vector bundle equipped with connection.  In particular, 
it lets you do "parallel transport" over 2d surfaces
in your space X.  The funny part is that these surfaces need
to be Riemann surfaces.  Indeed, an elliptic object
is very much like a conformal field theory, but where the surfaces
are mapped into X.  

The concept of elliptic object goes back to Graeme Segal.
His idea was roughly that an elliptic object should be a 
functor assigning a Hilbert space to any 
compact oriented 1-manifold \emph{mapped into X}, and 
a linear operator to any Riemann surface with boundary <em>
mapped into X</em>.
Stolz and Teichner's big realization is that an elliptic
object needs to be not just a functor, but a 2-functor!
In other words, it needs to assign data not just to Riemann
surfaces and 1-manifolds in X, but also to points in X!
Thus it's a lot like a 2d extended topological quantum field
theory, as explained in "<A HREF = "week35.html">week35</A>
".   The 
big difference is that the surfaces are Riemann surfaces,
and everything is happening "in X".

For how elliptic cohomology is related to 2-vector spaces,
read this:

4) Nils A. Baas, Bjorn Ian Dundas and John Rognes, Two-vector
bundles and forms of elliptic cohomology, available as 
<A HREF = "http://www.arXiv.org/abs/math.AT/0306027">math.AT/0306027</A>.

I'll quote the abstract because it will be enlightening to
a few of you:

\begin{quote}
  In this paper we define 2-vector bundles as suitable bundles of 
  2-vector spaces over a base space, and compare the resulting 
  2-K-theory with the algebraic K-theory spectrum K(V) of the 
  2-category of 2-vector spaces, as well as the algebraic 
  K-theory spectrum K(ku) of the connective topological K-theory 
  spectrum ku. We explain how K(ku) detects v_{2}-periodic phenomena 
  in stable homotopy theory, and as such is a form of elliptic cohomology. 
\end{quote}
One thing this means is that these folks have not gotten "the" 
theory of elliptic cohomology by studying 2-vector bundles.
They've gotten a theory which "detects v_{2}-periodic 
phenomena",
and is thus "a form" of elliptic cohomology.  

The point is, there's an infinite tower of generalized cohomology
theories, called the "chromatic filtration".  This has ordinary 
cohomology tensored with the complex numbers on the 0th level, 
complex K-theory on the 1st level, elliptic cohomology on the 2nd 
level, and so on up to infinity, where something called "complex 
cobordism theory" sits grinning down at us.  Theories on the nth 
level "detect v_{n}-periodic phenomena".  
Despite the best efforts
of several homotopy theorists, I still don't understand what this
means.  But, Bott periodicity for complex K-theory is the paradigm 
of a "v_{1}-periodic 
phenomenon", so we're talking about some heavy-duty 
generalization of that!  

Note that Baas, Dundas and Rognes don't talk about connections 
on their 2-vector bundles.  The closest thing to this that
people have used in elliptic cohomology is the notion of
"elliptic object", invented by Graeme Segal and improved by
Stolz and Teichner.  An elliptic object on a manifold M is 
like a way of moving strings around in M, so you can think of 
it as a recipe for 2d parallel transport.  The funny part is, 
you need a conformal structure on your surface before you can 
do parallel transport over it!  

Stolz and Teichner do a great job of working out the following analogy:


\begin{verbatim}

complex K-theory                       elliptic cohomology
connections on complex vector bundles  elliptic objects
supersymmetric 1d field theories       supersymmetric conformal field theories
\end{verbatim}
    
In particular, they show how the spectrum for complex K-theory
can be built from the space of supersymmetric 1d field theories,
just as the spectrum "tmf" is (conjecturally) built from some
space of supersymmetric conformal field theories.  Being an
optimist, I can't help but hope this pattern goes on something like this:


$$

some cohomology theory that detects v_{n}-periodic phenomena
connections on complex "n-vector bundles"
some supersymmetric field theories on n-dimensional spacetime
$$
    
Who knows?  

Next I should say a word about the "new" versus "old"
versions of elliptic cohomology.  At this
point things are going to get... ahem... a bit technical.
Then I'll talk about the connection to Monstrous Moonshine, and things will
get really vague, and downright bizarre.

The old version of elliptic cohomology was a specially nice sort
of generalized cohomology theory called a "complex oriented
cobordism theory".  I explained what these were in "<A HREF = "week149.html">week149</A>", and 
in "<A HREF = "week150.html">week150</A>" I explained how each of these things gives a "formal 
group law".  

If you want an easily understood example of a formal group law, 
just take a group, pick coordinates near the identity of this 
group, and write out the group operation in terms of these 
coordinates as a power series.  This works whenever your group is 
an analytic manifold and the group operations are analytic functions.
The result is a "formal group law".  The word "formal" comes from 
the fact that we'd actually be satisfies if the group operations
were described by \emph{formal} power series.

Anyway, now consider the torus.  A torus is a group in an obvious 
way - just a product of two copies of the group U(1) - but there 
are different ways to make it into a \emph{complex} manifold where 
the group operations are \emph{complex} analytic functions.  A way of 
doing this is nothing other than an "elliptic curve"!

In fact, each elliptic curve corresponds to a complex oriented 
cobordism theory, and we could call any one of these "an elliptic 
cohomology theory", if we wanted.

But it's better, actually, to glom all these different theories
into one big "universal" theory.  The most obvious way to attempt
this is to take the moduli space of elliptic curves and cook up 
a formal group law over the algebra of functions on this space 
by stitching together all the formal group laws for each specific 
elliptic curve.  This formal group law corresponds to a
complex oriented cobordism theory called Ell.  This is what 
I was calling the "old version" of elliptic cohomology. 

The "new version", namely "tmf", is a bit sneakier.  I think 
it's the "limit" - in the sense of category theory - of the 
elliptic cohomology theories for all specific elliptic curves.  
The reason this is different than Ell is that some elliptic 
curves have nontrivial symmetries!  Unlike Ell, tmf is \emph{not}
a complex oriented cobordism theory.  But the difference is 
very subtle, and only involves "2-torsion" and "3-torsion",
that is, elements that vanish when you multiply them by some
power of 2 times some power of 3.

The reason the numbers 2 and 3 show up is apparently because
the elliptic curves with nontrivial symmetries come from the 
square lattice:


\begin{verbatim}

             *     *     *     *


             *     *     *     *


             *     *     *     *
\end{verbatim}
    
and the hexagonal lattice:


\begin{verbatim}

            *       *      *      *


                *       *      *                


            *       *      *      *
\end{verbatim}
    
which have 4-fold and 6-fold symmetry, respectively.   I already 
expounded on these symmetries in "<A HREF = "week124.html">week124</A>" and "<A HREF = "week125.html">week125</A>", and showed 
that they're responsible for the mysterious role of the number 24 in 
string theory.  So, it's nice to see them showing up here!

In fact, they also show up in other devious ways, which I would   
love to understand better.  For starters, they give a certain
"period-12" pattern in the theory of modular forms, which becomes
a "period-24" pattern if you define weights using the convention 
that I'm using here.  Lots of people know about this - see any 
introduction to modular forms, like this one:

5) Neal Koblitz, Introduction to Elliptic Curves and Modular Forms, 
2nd edition, Springer-Verlag, 1993. 

I already vaguely explained this 
in "<A HREF = "week125.html">week125</A>".

But, more deviously, these symmetries are also related to a certain 
"period-576" pattern in topological modular form theory!  The number
576 is 24 x 24.  According to my vague memories of what Stephan Stolz
said, the first 24 is the usual one in bosonic string theory.  In
particular, if we ignored subtleties involving torsion, elliptic 
cohomology would have period 24, with the periodicity generated 
by a conformal field theory of central charge 24 having an enormous
group called the Monster as its symmetries!  This is where Monstrous 
Moonshine comes in, and especially the work of Borcherds.

(This can't be exactly right, because the most famous conformal
field theory whose symmetries form the Monster is not supersymmetric,
and its partition function is the j-function, which is modular
function of weight 0, not a modular form of weight 24.  So, 
my brain must have been a bit fried by the time we got to this
really far-out stuff.) 

Where does the extra 24 come from?  I don't know, but Stephan Stolz 
said it has something to do with the fact that while PSL(2,Z) doesn't 
act freely on the upper half-plane - hence these elliptic curves with 
extra symmetries - the subgroup "\Gamma (3)" does.  This subgroup consists 
of integer matrices


\begin{verbatim}

(a b)
(c d)
\end{verbatim}
    
with determinant 1 such that each entry is congruent to the
corresponding entry of 

\begin{verbatim}

(1 0)
(0 1)
\end{verbatim}
    
modulo 3.

So, if we form


$$

H/\Gamma (3)
$$
    
we get a nice space without any "points of greater symmetry".
To get the moduli space of elliptic curves from this, we just 
need to mod out by the group 


$$

SL(2,Z)/\Gamma (3) = SL(2,Z/3)
$$
    
But this group has 24 elements!

In fact, I think this is just another way of explaining the 
period-24 pattern in the theory of modular forms, but I like
it.  

I especially like it because SL(2,Z/3) is also known as the 
"binary tetrahedral group".  To get your hands on this group,
take the group of rotational symmetries of the tetrahedron,
also known as A_{4}.  This is a 12-element subgroup of SO(3).  
Using the fact that SO(3) has SU(2) as a double cover, take 
all the points in SU(2) that map to A_{4}.  You get a 24-element
subgroup of SU(2) which is the binary tetrahedral group.

In fact, if you think of SU(2) as the unit sphere in the
quaternions, the binary tetrahedral group becomes the vertices
of a 4-dimensional regular polytope called the 24-cell!

I'm very fond of this polytope, and have already extolled its 
charms in "<A HREF = "week91.html">week91</A>" and
 "<A HREF = "week155.html">week155</A>".   So, what 
pleases me now
is that I've found a trail directly from the 24-cell to the
appearance of the number 24 in string theory... and even the
fact that topological modular form theory has periodicity 24 x 24.

Of course I can barely follow this trail myself, and I probably
got some stuff wrong - I hope the experts correct
me!  But the trail seems to be real, not just a will o' the wisp, 
so I can now try to widen it and make it less twisty.

There's more to say but I'll stop here.  I have given other 
references to monstrous moonshine in "<A HREF = "week66.html">week66</A>", but here's a 
very pretty website about it:

6) Helena A. Verrill, Monstrous moonshine and mirror symmetry, 
<A HREF = "http://hverrill.net/pages~helena/seminar/seminar1.html">http://hverrill.net/pages~helena/seminar/seminar1.html</A>

and here is a nice easy paper by Terry Gannon about it:

7) Terry Gannon, Postcards from the edge, or Snapshots of the 
theory of generalised Moonshine, available as <A HREF = 
"http://www.arXiv.org/abs/math.QA/0109067">math.QA/0109067</A>.

I thank Allen Knutson and Peter Teichner for help with
this issue.      
\par\noindent\rule{\textwidth}{0.4pt}
\textbf{Addenda:} After posting this article, Aaron Bergman helped 
solve my puzzle about a supersymmetric conformal field theory with 
the Monster as symmetries, and Stephan Stolz explained why
topological modular form theory has period 24^{2}. 

Aaron Bergman writes:
\begin{quote}

\begin{verbatim}

John Baez wrote:

>(This can't be exactly right, because the most famous conformal
>field theory with the Monster as symmetries is not supersymmetric,
>and its partition function is the j-function, which is a modular
>function of weight 0, not a modular form of weight 24.  So, my
>brain must have been a bit fried by the time we got to this really
>far-out stuff.)

You might be interested in:

BEAUTY AND THE BEAST: SUPERCONFORMAL SYMMETRY IN A MONSTER MODULE. 
By Lance J. Dixon (Princeton U.), P. Ginsparg (Harvard U.),
Jeffrey A. Harvey (Princeton U.). HUTP-88-A013, PUPT-1088, Apr
1988. 30pp. 
Published in Commun. Math. Phys. 119 (1988), 221-241.

There's a scanned version on line. Note that they are working in
lightcone gauge so c=24.

Aaron
-- 
Aaron Bergman
<A HREF = "http://www.princeton.edu/~abergman/">http://www.princeton.edu/~abergman/</A>
\end{verbatim}
    
\end{quote}


In reply to an email of mine, Stephan Stolz wrote:

\begin{quote}

$$

 John Baez wrote:

 >Do you have a reference on the period-24^2 behavior of tmf?
 >That's one of the things I'm having trouble understanding,
 >even heuristically.  Actually I saw something about
 >it having period 192.  That's not 24^2.

 One reference is the course notes of a course Charles Rezk taught at 
 Northwestern University in 2001. You can find them on his home page
 <A HREF = "http://www.math.uiuc.edu/~rezk/papers.html">http://www.math.uiuc.edu/~rezk/papers.html</A>

 Let me make some remarks on periodicity: the ring M_{*} of 
 integral modular forms is 24-periodic with the discriminant \Delta 
 being the periodicity element.  Explicitly:

 M_{*} = Z[c_{4},c_{6},\Delta ]/(c_{4}^{3} - c_{6}^{2} - 12^{3} \Delta ).

 There is a ring homomorphism tmf_{*} \to  M_{*}; the periodicity 
 of tmf_{*} is then determined by the smallest power of \Delta  in 
 the image of this map.  After localizing at 2, this is \Delta ^{8} 
 (see Thm. 19.3 in Rezk's paper) which makes a period of 8 x 24 = 
 192.  However, this is \emph{only} after localizing at 2!  Localized 
 at the prime 3, the smallest power of \Delta  in the image is 
 \Delta ^{3} (see Thm. 17.2); hence after inverting all the primes not
 equal to than 2,3, the smallest power in the image is \Delta ^{24}! 
 Since localized at any other primes the above map is an isomorphism, 
 this shows that integrally tmf_{*} has period 24^{2}.

 Best regards,
 Stephan
 -- 
$$
    
\end{quote}

\par\noindent\rule{\textwidth}{0.4pt}
<em>The other Grand Canyon elder that I sought was George Stock.
He received his Ph.D. in theoretical math from the University of California
at Berkeley.  I first traveled with him when he was seventy-three years
old.  We carried a couple nights of gear through fields of boulders and
a few hand-over-hand ledges from the rim of the Grand Canyon to the river.
There we stripped naked and swam in the Colorado River.

George described his routes to me with a steady, comprehensive tone,
telling me about places of incredible hazard and reward.  He had walked
the entire length of the Grand Canyon when he was fifty-seven years old,
in eighty days, all of it done in the puzzling confines of the inner
reaches.  I had seen some of his routes before, and had used a number
of them, his meager catwalks and handholds.  They were like spider's silk,
lines across the landscape that were not visible until I touched them.
</em> - Craig Childs, Soul of Nowhere

\par\noindent\rule{\textwidth}{0.4pt}

% </A>
% </A>
% </A>
