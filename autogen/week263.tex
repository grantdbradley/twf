
% </A>
% </A>
% </A>
\week{April 5, 2008 }

Enough nebulae!  Today's astronomy picture is Saturn's moon Titan,
photographed by the Cassini probe.  Red and green represent methane
absorption bands, while blue represents ultraviolet.  Note the 
incredibly deep atmosphere &mdash; hundreds of kilometers deep.
That's because Titan has a dense atmosphere but not much gravity.
The pale feature in the center here is called Xanadu.  

<div align = "center">
<a href = "http://apod.nasa.gov/apod/ap041028.html">
<img width = "500" src = "titan.jpg">
% </a>
</div>

1) Astronomy Picture of the Day, Tantalizing Titan, 
<a href = "http://apod.nasa.gov/apod/ap041028.html">
http://apod.nasa.gov/apod/ap041028.html</a>

If you fell into Titan's atmosphere, here's what you'd see.  
Unlike the previous picture, this is in natural colors, taken
by the Cassini probe on March 31st, 2005 from a distance of 
just 9,500 kilometers: 

<div align = "center">
<a href = "http://en.wikipedia.org/wiki/Titan_(moon)#Atmosphere">
<img src = "titan_atmosphere.jpg">
% </a>
</div>

2) Wikipedia, Titan's atmosphere,
<a href = "http://en.wikipedia.org/wiki/Titan_(moon)#Atmosphere">
http://en.wikipedia.org/wiki/Titan_(moon)#Atmosphere</a>
 
The orange stuff is hydrocarbon "smog", perhaps made of
chemicals called <a href =
"http://en.wikipedia.org/wiki/Tholin">tholins</a> which I don't really
understand.  When you got further down the atmosphere would be so
thick, and the gravity so low, that you could fly through it by
flapping wings attached to your arms.  Unfortunately the atmosphere
would be very cold, and unbreathable: mostly nitrogen, with a little
methane and ethane.  (I wrote about the hydrocarbon rain on Titan back
in "<a href = "week160.html">week160</a>", and showed you
the first pictures of its lakes in "<a href =
"week210.html">week210</a>".)

<div align = "center">
<a href = "http://en.wikipedia.org/wiki/Titan_(moon)#Atmosphere">
<img src = "titan_atmosphere_chart.jpg">
% </a>
</div>

Astronomy is great, but today I want to talk about group theory.
As you may have heard, John Thompson and Jacques Tits won the
2008 Abel prize for their work on groups:

3) Abel Prize, 2008 Laureates,
<a href = "http://www.abelprisen.no/en/prisvinnere/2008/">http://www.abelprisen.no/en/prisvinnere/2008/</a>

If you want a fun, nontechnical book that gives a good taste of
the sort of things Thompson thought about, try this:

4) Marcus du Sautoy, Symmetry: a Journey into the Patterns of 
Nature, HarperCollins, 2008.

Mathematicians will enjoy this book for its many anecdotes about
the heroes of symmetry, from Pythagoras to Thompson and other
modern group theorists.  Nonmathematicians will learn a lot about
group theory in a fun easy way.

As a PhD student working under Saunders Mac Lane, Thompson began 
his career with a bang, by solving a 60-year-old conjecture posed 
by the famous group theorist Frobenius.  

5) Mactutor History of Mathematics Archive, John Griggs Thompson,
<a href = "http://www-history.mcs.st-andrews.ac.uk/Biographies/Thompson_John.html">http://www-history.mcs.st-andrews.ac.uk/Biographies/Thompson_John.html</a>

But, he's mainly famous for helping prove an even harder theorem 
that's even simpler to state - one of those precious nuggets of 
knowledge that mathematicians fight so hard to establish:


<div align = "center">
"Every finite group with an odd number of elements is solvable."
</div>

We say a group is "solvable" if it can be built out of abelian
groups in finitely many stages: the group at each stage mod the 
group at the previous stage must be abelian.  The term "solvable"
comes from Galois theory, since we can solve a polynomial equation
using radicals iff its Galois group is solvable.  

Way back in 1911, Burnside conjectured that every finite group
with an odd number of elements is solvable.  John Thompson and 
Walter Feit proved this in 1963.  Their proof took all 255 pages 
of an issue of the Pacific Journal of Mathematics!

The proof has been simplified a bit since then, but not much.
Versions can be found in two different books, and there is a 
project underway to verify it by computer:

6) Wikipedia, Feit-Thompson Theorem,
<a href = "http://en.wikipedia.org/wiki/Feit-Thompson_theorem">http://en.wikipedia.org/wiki/Feit-Thompson_theorem</a>

This theorem, also called the "odd order theorem", marked a trend 
toward really long proofs in finite group theory, as part of a 
quest to classify finite "simple" groups.  A simple group is one 
that has no nontrivial normal subgroups.  In other words: there's 
no way to find a smaller group inside it, mod out by that, and 
get another smaller group.   So, more loosely speaking, we can't 
build it up in several stages: it's a single-stage affair, a basic 
building block.

One reason finite simple groups are important is that \emph{every}
finite group can be built up in stages, where the group at each 
stage mod the group at the previous stage is a finite simple
group.   So, the finite simple groups are like the "prime 
numbers" or "atoms" of finite group theory.  

The first analogy is nice because \emph{abelian} finite simple groups 
practically \emph{are} prime numbers.  More precisely, every abelian 
finite simple group is Z/p, the group of integers mod p, for some 
prime p.  So, building a finite group from simple groups is a grand 
generalization of factoring a natural number into primes.

However, the second analogy is nice because just as you can build 
different molecules with the same collection of atoms, you can 
build different finite groups from the same finite simple groups.

I actually find a third analogy helpful.  As I hinted, for any finite 
group we can find an increasing sequence of subgroups, starting with the 
trivial group and working on up to the whole group, such that each subgroup 
mod the previous one is a finite simple group.  So, we're building our
group as a "layer-cake" with these finite simple groups as
"layers".

But: knowing the layers is not enough: each time we put on the next
layer, we also need some "frosting" or "jam" to
stick it on!  Depending on what kind of frosting we use, we can get
different cakes!

To complicate the analogy, stacking the layers in different 
orders can sometimes give the same cake.  This is reminiscent
of how multiplying prime numbers in different orders gives the 
same answer.  But, unlike multiplying primes, we can't \emph{always}
build our layer cake in any order we like. 

Apart from the order, though, the layers are uniquely determined -
just as every natural number has a unique prime factorization.  This
fact is called the "Jordan-H&ouml;lder theorem", and these
layer cakes are usually called "composition series".  For
more, try this:

7) Wikipedia, Composition series,
<a href = "http://en.wikipedia.org/wiki/Composition_series">http://en.wikipedia.org/wiki/Composition_series</a>

But let's see some examples! 

Suppose we want to build a group out of just two layers, where 
each layer is the group of integers mod 3, otherwise known as Z/3.  
There are two ways to do this.  One gives Z/3 \oplus  Z/3, the group
of pairs of integers mod 3.  The other gives Z/9, the group of
integers mod 9.  

We can think of Z/3 \oplus  Z/3 as consisting of pairs of digits 0,1,2
where we add each digit separately mod 3.  For example:

01 + 02 = 00<br/>
12 + 11 = 20<br/>
11 + 20 = 01

We can think of Z/9 as consisting of pairs of digits 0,1,2 where we
add each digit mod 3, but then carry a 1 from the 1's place to the
10's place when the sum of the digits in the 1's place exceeds 2 -
just like you'd do when adding in base 3.  I hope you remember your
early math teachers saying "don't forget to carry a 1!" It's
like that.  For example:

01 + 02 = 10<br/>
12 + 11 = 00<br/>
11 + 20 = 01

So, the "frosting" or "jam" that we use to stick
our two copies of Z/3 together is the way we carry some
information from one to the other when adding!  If we do it trivially,
not carrying at all, we get Z/3 \oplus  Z/3.  If we do it in a more
interesting way we get Z/9.

In fact, this how it always works when we build a layer cake
of groups.  The frosting at each stage tells us how to "carry"
when we add.  Suppose at some stage we've got some group G.  Then 
we want to stick on another layer, some group H.   An element of 
the resulting bigger group is just a pair (g,h).  But we add these
pairs like this:

(g,h) + (g',h') = (g + g' + c(h,h'), h + h')

where 

c: H \times  H \to  G

tells us how to "carry" from the "H place" to the
"G place" when we add.  So, information percolates down when
we add two guys in the new top layer of our group.

Of course, not any function c will give us a group: we
need the group laws to hold, like the associative law.  To 
make these hold, the function c needs to satisfy some equations.
If it does, we call it a "2-cocycle".  

These cocycles are studied in a subject called "group cohomology".
Usually people focus on the simplest case, when our original group 
G is abelian, and its elements commute with everything in the big 
new group we're building.  If this isn't true, we need something 
more general: <i>nonabelian</i> group cohomology, often
called "Schreier theory" (see "<a href = "week223.html">week223</a>").

I like this layer cake business because it's charming and it
generalizes in two nice ways.  First of all, it works for
lots of algebraic gadgets besides groups.  Second of all, it
works for \emph{categorified} versions of these gadgets.  

For example, a group is a category with one object, all of whose
morphisms are invertible.  Similarly, an "n-group" is an
n-category with one object, all of whose 1-morphisms, 2-morphisms and
so on are invertible.  We can build up n-groups as layer cakes where
the layers are groups.  It's a more elaborate version of what I just
described - and it uses not just "2-cocycles" but also
"3-cocycles" and so on.  I never really understand group
cohomology until I learned to see it this way.

But what's \emph{really} cool is that n-groups can also be thought
of as topological spaces.  This lets us build every space as a
"layer cake" where the layers are groups!  These groups are
called the "homotopy groups" of the space.  The nth homotopy
group keeps track of how many n-dimensional holes the space has - see
"<a href = "week102.html">week102</A>" for details.

But of course, they don't call the process of sticking these groups
together a "layer cake": that would be too undignified.
They call it a "Postnikov tower".  And instead of
"frosting", they speak of "Postnikov invariants".
Every space is the union of a bunch of connected pieces, each of which
is determined by its homotopy groups and its Postnikov invariants.

(At least this is true if you count spaces as the same when they're
"weakly homotopy equivalent".  This is a fairly sloppy equivalence
relation beloved by homotopy theorists.  You've probably heard
how a topologist is someone who can't tell the difference between
a doughnut and a coffee cup.  Actually they can tell: they just
don't care!  A homotopy theorist is a more relaxed sort of guy
who doesn't even care about the difference between a doughnut 
and a Moebius strip.  They're both just fattened up versions of a 
circle.)

Mike Shulman and I tried to explain this layer cake business here:

8) John Baez and Michael Shulman, Lectures on n-categories
and cohomology, to appear in n-Categories: Foundations and 
Applications, eds. John Baez and Peter May.  Also available
as <a href = "http://arxiv.org/abs/math/0608420">arXiv:math/0608420</a>

Whoops!  I see I've drifted from my supposed topic - the work of 
John Thompson - to something I actually understand.
It was a digression, but not a completely pointless one.  From 
what I've told you, it follows that every space with finite 
homotopy groups can be built as a fancy "layer cake" made of
finite simple groups.  

And even better, the finite simple groups have now been classified! -
we think.  There are 18 infinite series of these groups, and also 26
exceptions called "sporadic" groups, ranging in size from
the five Mathieu groups (see "<a href =
"week234.html">week234</A>") on up to the Monster (see "<a
href = "week20.html">week20</A>" and "<a href =
"week66.html">week66</A>").

9) Wikipedia, List of finite simple groups, 
<a href = "http://en.wikipedia.org/wiki/List_of_finite_simple_groups">http://en.wikipedia.org/wiki/List_of_finite_simple_groups</a>

Proving that these are all the possibilities took mathematicians 
about 10,000 pages of work!  The Feit-Thompson theorem is a small 
but crucial piece in this enormous pyramid of proofs.  There could 
still be some mistakes here and there, but experts are busy working 
through the details more carefully.

Among the 26 sporadic groups, one is called the Thompson group.  It
was discovered by Thompson, and it's a subgroup of a version of the
group E_{8} defined over F_{3}, the field with 3
elements.  It has about 9 \times  10^{16} elements, and it has
a 248-dimensional representation over F_{3}.  I don't know
much about it.  I mention it just to show what crazy possibilities had
to be considered to classify all finite simple groups - and how deeply
Thompson was involved in this work.

But what about Jacques Tits?

10) Mactutor History of Mathematics Archive, Jacques Tits
<a href = "http://www-history.mcs.st-andrews.ac.uk/Biographies/Tits.html">http://www-history.mcs.st-andrews.ac.uk/Biographies/Tits.html</a>

He's not mentioned in du Sautoy's book "Symmetry", which is a
pity, but not surprising, since too many mathematicians have
studied group theory to fit comfortably in one story.  He
has a sporadic finite simple group named after him, but his
work leaned in a different direction, more focused on the role
of groups in geometry.  He was an honorary member of Bourbaki,
and in that role he helped awaken interest in the work of Coxeter.

I've mentioned his work on the "magic square" of exceptional
Lie groups in "<a href = "week145.html">week145</A>" and
"<a href = "week253.html">week253</A>"... but he's more
famous for his work on "buildings", sometimes called
"Bruhat-Tits buildings".

The subject of buildings has a reputation for being intimidating,
perhaps because the \emph{definition} of a building looks scary and
unmotivated.  You can read these and decide for yourself:

11) Wikipedia, Building (mathematics),
<a href = "http://en.wikipedia.org/wiki/Building_%28mathematics%29">http://en.wikipedia.org/wiki/Building_%28mathematics%29</a>

12) Kenneth S. Brown, What is a building?, Notices AMS, 49 
(2002), 1244-1245.  Also available at 
<a href = "http://www.ams.org/notices/200210/what-is.pdf">http://www.ams.org/notices/200210/what-is.pdf</a>

13) Paul Garrett, Buildings and Classical Groups, CRC Press,
1997.  Preliminary version available at 
<a href = "http://www.math.umn.edu/~garrett/m/buildings/">http://www.math.umn.edu/~garrett/m/buildings/</a>

14) Kenneth S. Brown, Buildings, Springer, 1989.

15) Mark Ronan, Lectures on Buildings, Academic Press, 1989.

Personally I found it a lot easier to start with \emph{examples}.

So, start with any "finite reflection group" - a finite
group of transformations of R^{n} that's generated by reflections.  The
possibilities have been completely worked out, and I listed them back
in "<a href = "week62.html">week62</A>".  But let's do an
easy one: the symmetry group of an equilateral triangle.

I can't resist mentioning that this group is also S_{3}, the
group of all permutations of the three vertices of the triangle.  In
fact, this group was the star of "<a href =
"week261.html">week261</A>", where it showed up as the Galois
group of the cubic equation!  We can solve a cubic using radicals
since this group is solvable.  In other words, we can build this group
as a "layer cake" from the abelian groups Z/3 and Z/2.  The
bottom layer is Z/3, the subgroup of even permutations.  The top layer
is S_{3} modulo the even permutations, namely Z/2.  Galois
theory says you can solve a cubic by messing around a bit, then taking
a square root, and then taking a cube root.  Why a square root
\emph{first?}  Because you build this sort of layer cake from the bottom
up, but you eat it from the top down, slicing off one layer at a time.

But now we want to think about how this group is generated by 
reflections.  You can use just two, for example the reflections 
across the mirrors labelled r and s here:


\begin{verbatim}

                          s 
                 \       /
                  \     /
                   \   /
                    \ /
             --------o--------r
                    / \
                   /   \
                  /     \
                 /       \
\end{verbatim}
    
Let's call these reflections r and s.  They clearly satisfy

r^{2} = s^{2} = 1

but since the mirrors are at an angle of \pi /3 from each other,
they also satisfy

(rs)^{3} = 1

This gives a presesentation of our group S_{3}.  We can summarize 
this presentation with a little "Coxeter diagram":
       

\begin{verbatim}

      3
  r-------s
\end{verbatim}
    

where the dots r and s are the generators, and the edge labelled
"3" is the interesting relation (rs)^{3} = 1.  I
explained these diagrams more carefully back in "<a href =
"week62.html">week62</A>".  If you know about Dynkin diagrams,
these are pretty similar - see "<a href =
"week63.html">week63</A>" and "<a href =
"week64.html">week64</A>" for details.

Note that the mirrors in this picture:


\begin{verbatim}

                          s 
                 \       /
                  \     /
                   \   /
                    \ /
             --------o--------r
                    / \
                   /   \
                  /     \
                 /       \
\end{verbatim}
    
chop the plane into 
6 "chambers", and the group S_{3} has 6
elements.  This is no coincidence - it works like this for any finite
reflection group!  We can pick any chamber as our favorite and label
it 1:


\begin{verbatim}

                          s 
                 \       /
                  \     /  
                   \   /   1
                    \ /
             --------o--------r
                    / \
                   /   \
                  /     \
                 /       \
\end{verbatim}
    
Then, we can label any other chamber by the unique element of
our group that carries our favorite chamber to that one:


\begin{verbatim}

                          s 
                 \       /
                  \  s  /     
              sr   \   /   1
                    \ /
             --------o--------r
                    / \
        rsr = srs  /   \   r
                  / rs  \  
                 /       \
\end{verbatim}
    
If we start with chamber 1 and keep reflecting across mirrors, 
we keep getting products of more and more generators until we 
reach the diametrically opposite chamber, which corresponds to 
the so-called "long word" in our finite reflection group.  In 
this case, the long word is rsr = srs.

(Fanatical devotees will also note that this equation is the
"Yang-Baxter equation" mentioned in "<a href = "week261.html">week261</A>".)

Now, Coxeter thought about all this stuff, and he realized
that it was nice to introduce a polytope with one face for 
each chamber - in this case, just a hexagon:
 

\begin{verbatim}

                     s
                  o-----o
              rs /       \1 
                /         \
               o           o
                \         / 
             rsr \       /r
                  o-----o
                    sr
\end{verbatim}
    
This is called the "Coxeter complex" of our finite reflection
group.  Our finite reflection group acts on it, and it acts
on the faces in a free and transitive way. 

But, you'll note we started with the symmetry group of an 
equilateral triangle, and wound up with a hexagon!  What happened?

The quick way to say it is this: combinatorially speaking,
the hexagon is the "<a href = "http://en.wikipedia.org/wiki/Barycentric_subdivision">barycentric subdivision</a>" of our original
triangle.  Not the inside of the triangle - just its surface,
or boundary!  The boundary of the triangle is a simplicial 
complex made of 3 vertices and 3 edges:
                     

\begin{verbatim}

                  o
                  .  .   
                  .     . 
                  .        o
                  .     .  
                  .  .
                  o
\end{verbatim}
    
so if we barycentrically subdivide it, we get 6 vertices
and 6 edges:
                     

\begin{verbatim}

                  o-----o
                 /       \
                /         \
               o           o
                \         / 
                 \       /
                  o-----o
\end{verbatim}
    
and that's our hexagon - drawn puffed out a bit, just for the sake
of prettiness.

If this seems bizarre - and it probably does, given how lousy these
pictures are - I urge you to try the next example on your own.  Take
the symmetry group of the regular tetrahedron, also known as S_{4},
the group of permutations of 4 things.  Show it's generated by 
three reflections r,s,t with relations

r^{2} = s^{2} = t^{2} = 1

(rs)^{3} = (st)^{3} = 1

rt = tr

We can summarize these with the following Coxeter diagram:


\begin{verbatim}

      3       3
  r-------s-------t
\end{verbatim}
    
Draw all mirrors corresponding to reflections in S_{4}, and show 
they chop 3d space into 24 chambers, one for each element of S_{4}. 
Then, barycentrically subdivide the boundary of the tetrahedron 
and check that the resulting "Coxeter complex" has 24 faces, one 
inside each chamber.

Anyway, one thing Tits did is realize how these Coxeter complexes 
show up in the geometry of the \emph{Lie groups}, or more generally
<i>algebraic groups</i>, associated to Dynkin diagrams.  

For example, if I take this guy:


\begin{verbatim}

      3
  r-------s
\end{verbatim}
    

and remove some of the labels, I get the so-called A_{2}
Dynkin diagram:


\begin{verbatim}

  o-------o
\end{verbatim}
    
which corresponds to the Lie group PSL(3).  And, this is the group 
of symmetries of projective plane geometry!  Each dot in the Dynkin
corresponds to a "type of figure":


\begin{verbatim}

POINT    LINE
  o-------o
\end{verbatim}
    
and the edge corresponds to an "incidence relation": in projective
plane geometry, a point can lie on a line.   This shape, which we've
seen before:


\begin{verbatim}

                  o
                  .  .   
                  .     . 
                  .        o
                  .     .  
                  .  .
                  o
\end{verbatim}
    
is then revealed to stand for a configuration of 3 points and 3 
lines, satisfying incidence relations obvious from the picture.  
To put points and lines on an equal footing, we switch to the  
the Coxeter complex:


\begin{verbatim}

               POINT    LINE      
                  o-----o
                 /       \
                /         \
          LINE o           oPOINT
                \         / 
                 \       /
                  o-----o
              POINT     LINE
\end{verbatim}
    
where now the vertices represent "figures" and the edges represent
"incidence relations".  It turns out that inside any projective
plane, we can find lots of configurations like this: 3 points and
3 lines, each pair of points lying on one of the lines, and each 
pair of lines intersecting in a point.  Such a configuration is
called an "apartment".  

If we take all the apartments coming from a projective plane, they
form a simplicial complex called a "building".  And this generalizes
to any geometry corresponding to any sort of Dynkin diagram.  The
building knows everything about the geometry: all the figures, all
the incidence relations.


And that's all I have time for now, but it's just the beginning of
the marvelous theory Jacques Tits worked out.

\par\noindent\rule{\textwidth}{0.4pt}
\textbf{Addenda:} At least on Titan, tholins seem to be a complex 
brew of compounds made by irradiation of molecular nitrogen and 
methane in the upper atmosphere.  The same sort of compounds 
could be an early chemical step in the formation of life on Earth - 
that's one reason I'm interested.  They're related to PAHs, or
"polycyclic aromatic hydrocarbons", which are ubiquitous
in outer space - I wrote about those back in "<a href = 
"week258.html">week258</a>".  I guess the main difference
is that tholins contain nitrogen!

I found some more information on tholins here:

16) J. H. Waite, Jr., et al, The process of tholin formation in 
Titan's upper atmosphere, Science 316 (2007), 870-875.

Here's a picture of how tholins get made, from this paper:

<div align = "center">
<img src = "titan_tholins.gif">
% </a>
</div>

You can see more discussion and also <i>questions I'm dying to know
the answers to</i> over at the
<a href = "http://golem.ph.utexas.edu/category/2008/04/this_weeks_finds_in_mathematic_24.html">\emph{n}-Category Caf&eacute;</a>.  Whenever I 
write This Week's Finds, I come up with lots of questions.  If you
can help me with some of these, I'll be really grateful.

\par\noindent\rule{\textwidth}{0.4pt}
<em>It was technical - there was no way to avoid it.  But it was a 
wonderful thing.  We'd finally busted it.  But then, just before
we were about to submit the paper, Walter noticed a mistake.</em>
- John Thompson

\par\noindent\rule{\textwidth}{0.4pt}

% </A>
% </A>
% </A>
