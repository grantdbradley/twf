
% </A>
% </A>
% </A>
\week{December 5, 1999 }

I was recently infected by a meme - a self-propagating pattern of human 
behavior.  Now I want to pass it on to you!  I like this particular meme 
because it's so simple.  It's even simpler than the parasites described 
on my webpage:

1) John Baez, Subcellular life forms,
<A HREF = "http://math.ucr.edu/home/baez/subcellular.html">http://math.ucr.edu/home/baez/subcellular.html</A>

I wrote this webpage when I was trying to understand some of the
simplest self-reproducing entities: viruses, viroids, virusoids,
plasmids, prions, and various forms of junk DNA.  Viroids are 
especially simple.  Unlike a virus, a viroid doesn't even have a protein
coat: it's just a naked RNA molecule!  So instead of actively breaking
into the host cell, it must passively wait to be absorbed.  Then somehow
it hijacks the machinery of the cell nucleus to reproduce itself.  
Theodore Diener discovered the first viroid in 1971:  the potato spindle
tuber viroid, which makes potatoes abnormally long and sometimes
cracked.   At first people doubted the possibility of a life form
smaller than a virus.  But by now the complete molecular structure of
this viroid has been worked out.  It consists of only 359 nucleotides -
or in other words, about 12,000 atoms!  

But since a meme relies on the complex apparatus of human culture to 
reproduce itself, it can get away with being even simpler than a   
viroid.  It can even be the simplest sort of thing of all: an abstract
mathematical structure defined by a short list of axioms!

A good example is the game of tic-tac-toe.  It's not very interesting,
but it's just interesting enough to keep propagating itself through
human children, who are highly susceptible to the charm of simple games.
Most children soon develop an immunity to tic-tac-toe, just like 
measles and mumps - but only after passing it on to some other child.

Unfortunately, the meme that infected me is a lot harder to shake,
because it's a lot more interesting.  I'm talking about the game of
Go.  

This game is played on a 19 x 19 square grid.  Each player starts with a
large supply of stones - black for the first player, white for the
second.  They take turns putting a stone on a grid point.  A group of
stones of one color "dies" and is removed from the board when it is
surrounded by stones of the other color.  More precisely, we say a stone
is "dead" when none of its nearest neighbors of the same color have
nearest neighbors of the same color which have nearest neighbors of the
same color which... have nearest neighbors that are still vacant grid
points.

There are also two subsidary rules, designed to keep silly things from
happening.  

First, you are not allowed to put a stone someplace where it will
immediately die, \emph{unless} doing so immediately kills one or more of the
other player's stones - in which case their stones die, and yours lives.
 
Second, if putting down your stone kills a stone of the other player, but
they could immediately put that stone back and kill yours, leading to an
infinite loop, we say that "ko" has occurred.  In this case, the other
player is forbidden from putting their stone back right away.  

How do you win?  Simply put, the goal is to end up with as much
"territory" as possible.  Territory includes grid points occupied by
stones of your color, and also vacant grid points that the other player
could not occupy without their stones eventually dying.  (In practice, 
Go players do not fight to the bitter end, so territory also includes
stones of the other color that are "doomed to die".)

That's basically it!

The cool part is that starting from these simple rules, a whole world of
strategy unfolds, full of specific tricks - but also quite general
philosophical lessons about "power", "territory", and "threat".  In a
good game, both players start by efficiently marking out some territory,
putting stones down in a widely separated way that looks random to the 
beginner, but in fact is delicately balanced between being too
conservative and too ambitious.  The midgame starts when both players
start trying to surround each other and threaten to kill stones.   But
be careful: threatening to kill stones can be better than actually
killing them, and the difference between "surrounding" and "being
surrounded" is rather subtle!   The endgame comes when territory is
almost fully demarcated, with only a few squabbles around the edges.  
The endgame game proves unexpectedly difficult for beginners, since 
one can snatch defeat from the jaws of victory even at this stage.

A well-developed Go game is said to be like a work of art, with all
opposing forces neatly balanced in a harmonious pattern.  As a
mathematical physicist, it reminds me of the Ising model at a phase
transition, when there are as many black grid points as white ones, and
arbitrarily large clusters of both colors.  Perhaps there's even a
real relation to the theory of "self-organized criticality", in which
a system spontaneously works its way to the brink of a phase transition.

People say Go was developed in China between 4 and 6 thousand years ago.  
Its early history is obscure, but it is said to have started, not as a game, 
but as a tool for divination and the teaching of military strategy.   I'm 
no expert, but to me Go seems like a nice illustration of yin-yang 
philosophy - the idea that the dynamic complexity of the universe arises 
from the dialectic interplay of binary opposites.   For a good
introduction to what I'm talking about, you can't beat the I Ching - the
"Classic of Changes", a Chinese divination text compiled in the 9th century
B.C., but containing material that probably dates back at least a few
centuries earlier.   This book describes the significance of 64 "hexagrams", 
which are patterns built from 6 bits of information, like this:


\begin{verbatim}

                        ______
                        __  __
                        ______
                        ______
                        __  __
                        __  __
\end{verbatim}
    

The idea that complex patterns can be described using bits was borrowed
from the Chinese by Leibniz, who invented the concept of binary arithmetic
and dreamt of a purely mechanical approach to logic based on simple rules.
Now, of course, these ideas dominate modern technology!  So perhaps it's
not surprising that Go still holds an attraction for many mathematicians
and physicists.  

In fact, I bet some you are smirking and wondering why I didn't learn Go
much earlier!  The reason is that I've always avoided playing games,
except for the "great game" of mathematical physics.  I only tried
playing Go the weekend before last, while visiting my friend Bruce Smith
up in San Rafael after giving a talk on quantum tetrahedra at Stanford.  
Bruce explained Go to me and showed me how it was philosophically  
interesting.  But most importantly, he showed me a computer program
that plays Go.  Computers aren't great at Go, but they're good enough to
beat an amateur like me, so they're good to learn from at first, and for
some reason I prefer to play a computer than another person - perhaps because 
computers don't gloat.  

The computer program I played against is called "GNU Go".  You can download
it free from the internet, thanks to the Free Software Foundation:

2) GNU Go, <A HREF = "http://www.gnu.org/software/gnugo/devel.html">http://www.gnu.org/software/gnugo/devel.html</A>

You can adjust the size of the board and also the handicap - the number
of stones you get right away when you start.  To use this program in a
UNIX environment you need an interface program called
"cgoban", which is also free:

3) CGoban, <A HREF = "http://www.inetarena.com/~wms/comp/cgoban/">http://www.inetarena.com/~wms/comp/cgoban/</A>

On Windows you can use an interface available from the GNU Go webpage.

For more information on Go start here:

4) American Go Association, <A HREF = "http://www.usgo.org/resources/">http://www.usgo.org/resources/</A>

You can find lots of go books listed at this website.  Personally
I found these books to be a nice introduction to the game, but they
may be hard to find:

5) The Nihon Kiin, Go: The World's Most Fascinating Game, 2 volumes,
Sokosha Printing Co., Tokyo, 1973.

When you get more advanced, there are a lot of books to read, with 
fun titles like "Get Strong at Invading", "Reducing Territorial 
Frameworks", and "Utilizing Outward Influence".  It pays to study
"joseki", or openings:

6) Ishida Yoshio, Dictionary of Basic Joseki, 3 volumes, Ishi
Press International, San Jose, California, 1977.

It's also good to study "tsume-go", or "life and death problems", where
you figure out which player can win in various configurations.  A
mathematician would call this the "local" analysis of Go:

7) Cho Chikun, All About Life and Death, 2 volumes, Ishi Press 
International, San Jose, California, 1993.

Ishi Press puts out a lot of other books on Go, but I haven't been able
to get ahold of them yet.  I'm sort of fascinated by one that talks
about a difficult abstract concept called "thickness", since I suspect
this is a global rather than local concept:

8) Ishidea Yoshio, All About Thickness: Understanding Moyo and
Influence, Ishi Press International, San Jose, California.

If you want to get mathematical about Go endgames, try this:

9) Elwyn Berlekamp and David Wolfe, Mathematical Go: Chilling
Gets the Last Point, A. K. Peters, Wellesley Massachusetts, 1994.

If you want to get computational, try this:

10) Markus Enzenberger, The integration of a priori knowledge into
a Go playing neural network, 
<A HREF = "http://www.cgl.ucsf.edu/go/Programs/neurogo-html/NeuroGo.html">http://www.cgl.ucsf.edu/go/Programs/neurogo-html/NeuroGo.html</A>

If instead you prefer to curl up with a good novel based
on a game of Go, try this:

11) Yasunari Kawabata, The Master of Go, trans. Edward G. 
Seidensticker, Knopf, New York, 1972.  

On a different note, here are two good editions of the I Ching:

12) The I Ching or Book of Changes, trans. Richard Wilhelm and Cary F.
Baynes, Princeton U. Press, Princeton, 1969.

The Classic of Changes: A New Translation of the I Ching as Interpreted
by Wang Bi, trans. Richard John Lynn, Columbia U. Press, 1994.

Okay.  Enough culture - time for some math!

I was invited to Stanford University by David Carlton, who works on
modular forms, and I found out from him and his friends that the
Shimura-Taniyama-Weil conjecture has been proved!  This might have been
a nice scoop for This Week's Finds, but by now it's appeared in the
Notices of the AMS, so everyone knows about it:

13) Henri Darmon, A proof of the full Shimura-Taniyama-Weil conjecture
is announced, Notices of the American Mathematical Society, 46 no. 11
(December 1999), 1397-1401.

Andrew Wiles proved part of this conjecture in order to prove Fermat's
Last Theorem, but the conjecture is actually much more interesting than
Fermat's Last Theorem, and a proof of the whole thing was announced this
summer by Breuil, Conrad, Diamond and Taylor.  

What does the conjecture say?  

Well, first you have to know a bit about elliptic curves.  An 
"elliptic curve" is the space of solutions of an equation like 
this:

y^{2} = x^{3} + ax + b

They come up naturally in string theory, and I've talked about them
already in "<A HREF = "week13.html">week13</A>" and "<A
HREF = "week124.html">week124</A>" - "<A HREF =
"week127.html">week127</A>".  If all the variables in sight are
complex numbers, an elliptic curve looks like a torus, but number
theorists like to consider the case where the coefficients a and b are
rational.  By a simple change of variables you can then get the
coefficients to be integers.  Then it makes sense to work modulo a
prime number p: in other words, to think of all the variables as
living in the field of integers mod p, better known as Z/p.  If you're
smart, you can tell if an elliptic curve mod p is "singular"
or not: being nonsingular is like being a smooth manifold.  People say
an elliptic curve has "good reduction at p" if it's
nonsingular mod p.  For any given elliptic curve, this is true except
for finitely many primes.

Any elliptic curve E has finitely many points mod p.  Let's call the 
number of points N(E,p) and set

a(E,p) = p - N(E,p)

If this list of numbers satisfies a certain condition, which I'll
describe in a minute, we say our elliptic curve is "modular".  The
Shimura-Taniyama-Weil conjecture states that all elliptic curves
are modular.

Okay, so what does "modular" mean?  Well, for this we need a
little digression on modular forms.  In "<A HREF =
"week125.html">week125</A>" I described the moduli space of
elliptic curves, which is the space of all different shapes an
elliptic curve can have.  I showed that this space was H/SL(2,Z),
where H is the upper half of the complex plane and SL(2,Z) is the
group of 2x2 integer matrices with determinant 1.  A modular form is
basically just a holomorphic section of some line bundle over the
moduli space of elliptic curves.  But if this sounds too high-tech,
don't be scared!  We can also think of it as an analytic function on
the upper half-plane that transforms in a nice way under the action of
SL(2,Z).  Remember, any matrix


\begin{verbatim}

                          a b
                          c d

\end{verbatim}
    
in SL(2,Z) acts on the upper half-plane as follows:

                \tau  |\to  (a \tau  + b)/(c \tau  + d)
For an analytic function f: H \to  C to be a "modular form of
weight k", it must transform as follows:

f((a \tau  + b)/(c \tau  + d)) = (c \tau  + d)^{k} f(\tau )

for some integer k.   We also require that f satisfy some growth 
conditions as \tau  \to  \infty , so we can expand it as a Taylor series

f(\tau ) = &sum; a_{n} q^{n}

where 

q = exp(2 \pi  i \tau )

is a variable that equals 0 when \tau  = \infty .  The nicest modular
forms are the "cusp forms", which have a_{0} = 0,
and thus vanish at \tau  = \infty .

Next, we can straightforwardly generalize everything I just said if we
replace SL(2,Z) by various subgroups thereof.  (This amounts to
studying holomorphic sections of line bundles over some moduli space
of elliptic curves \emph{equipped with extra structure}.)  For
example, we can use the subgroup \Gamma _{0}(N) consisting of those
matrices in SL(2,Z) whose lower-left entries are divisible by N.  If
we use this group instead of SL(2,Z), we get what are called modular
forms of "level N".  We define "weight" of such a
modular form just as before, and ditto for "cusp forms".

And now we can say what it means for an elliptic curve to be modular!
We say an elliptic curve E is "modular" if for some N
there's a weight 2 level N cusp form

f(\tau ) = &sum; a_{n} q^{n}

normalized so that a_{1} = 1, with the property that

a_{p} = a(E,p)

for all primes p at which E has good reduction.  

So now you know what the Shimura-Taniyama-Weil conjecture says:
all elliptic curves are modular!  It's not obvious that this implies
Fermat's Last Theorem, but it does, thanks to a trick invented by
Gerhard Frey.

There turn out to be fascinating but mysterious relationships between
the Shimura-Taniyama-Weil conjecture, something called the Langlands
program, and topological quantum field theory:

14) Mikhail Kapranov, Analogies between the Langlands correspondence
and topological quantum field theory, in Functional Analysis on the 
Eve of the 21st Century, Vol. 1, Birkhaueser, Boston, pp. 119-151.

For this reason - and others - it's not so surprising that David
Carlton and some of his buddies are interested in n-categories.  In
fact, Carlton caught a small error in the definition of n-categories
due to James Dolan and myself - it turns out that the number
"1" should be the number "2" at one particular
place in the definition!  Anyone who can spot a problem like that is
friend of mine.

Even better, Carlton is now interested in understanding the (n+1)-category 
of all n-categories, which is crucial for really doing anything with 
n-categories.  Makkai has a new paper on this subject, and I realize now 
that I've never mentioned this paper on This Week's Finds, so let me 
conclude by quoting the abstract.  It's pretty long and detailed, and
probably only of interest to n-category addicts....

15) M. Makkai, The multitopic \omega -category of all multitopic 
\omega -categories, preprint available at <A HREF = "ftp://ftp.math.mcgill.ca/pub/makkai">ftp://ftp.math.mcgill.ca/pub/makkai</A>

\begin{quote}
"The paper gives two definitions: that of "multitopic \omega -category" and
that of "the (large) multitopic set of all (small) multitopic
\omega -categories". It also announces the theorem that the latter is a
multitopic \omega -category. (The proof of the theorem will be contained in
a sequel to this paper.)

The work has two direct sources. One is the paper [H/M/P] (for the
references, see at the end of this abstract) in which, among others, the
concept of "multitopic set" was introduced. The other is the present
author's work on FOLDS, First Order Logic with Dependent Sorts. The
latter was reported on in [M2]. A detailed account of the work on FOLDS is
in [M3]. For the understanding of the present paper, what is contained in
[M2] suffices. In fact, section 1 of the present paper gives the
definitions of all that's needed in this paper; so, probably, there won't
be even a need to consult [M2]. 

The concept of multitopic set, the main contribution of [H/M/P], was, in
turn, inspired by the work of J. Baez and J. Dolan [B/D]. Multitopic sets
are a variant of opetopic sets of loc. cit. The name "multitopic set"
refers to multicategories, a concept originally due to J. Lambek [L], and
given an only moderately generalized formulation in [H/M/P]. The earlier
"opetopic set" of [B/D] is based on a concept of operad. I should say that
the exact relationship of the two concepts ("multitopic set" and "opetopic
set") is still not clarified. The main aspect in which the theory of
multitopic sets is in a more advanced state than that of opetopic sets is
that, in [H/M/P], there is an explicitly defined category Mlt of
\textbf{multitopes}, with the property that the category of multitopic sets is
equivalent to the category of Set-valued functors on Mlt, a result given a
detailed proof in [H/M/P]. The corresponding statement on opetopic sets
and opetopes is asserted in [B/D], but the category of opetopes is not
described. In this paper, the category of multitopes plays a basic role.

Multitopic sets and multitopes are described in section 2 of this paper;
for a complete treatment, the paper [H/M/P] should be consulted.

The indebtedness of the present work to the work of Baez and Dolan goes
further than that of [H/M/P]. The second ingredient of the Baez/Dolan
definition, after "opetopic set", is the concept of "universal cell". The
Baez/Dolan definition of weak n-category achieves the remarkable feat of
specifying the composition structure by universal properties taking place
in an opetopic set. In particular, a (weak) opetopic (higher-dimensional)
category is an opetopic set with additional properties ( but with no
additional data), the main one of the additional properties being the
existence of sufficiently many universal cells. This is closely analogous
to the way concepts like "elementary topos" are specified by universal
properties: in our situation, "multitopic set" plays the "role of the
base" played by "category" in the definition of "elementary topos". In
[H/M/P], no universal cells are defined, although it was mentioned that
their definition could be supplied without much difficulty by imitating
[B/D]. In this paper, the "universal (composition) structure" is supplied
by using the concept of FOLDS-equivalence of [M2].

In [M2], the concepts of "FOLDS-signature" and "FOLDS-equivalence" are
introduced. A (FOLDS-) signature is a category with certain special
properties. For a signature L , an \textbf{L-structure} is a Set-valued functor
on L. To each signature L, a particular relation between two variable
L-structures, called L-equivalence, is defined. Two L-structures M, N, are
L-equivalent iff there is a so-called L-equivalence span M<---P--->N
between them; here, the arrows are ordinary natural transformations,
required to satisfy a certain property called "fiberwise surjectivity".

The slogan of the work [M2], [M3] on FOLDS is that *all meaningful
properties of L-structures are invariant under L-equivalence*. As with all
slogans, it is both a normative statement ("you should not look at
properties of L-structures that are not invariant under L-equivalence"),
and a statement of fact, namely that the "interesting" properties of
L-structures are in fact invariant under L-equivalence. (For some slogans,
the "statement of fact" may be false.) The usual concepts of "equivalence"
in category theory, including the higher dimensional ones such as
"biequivalence", are special cases of L-equivalence, upon suitable, and
natural, choices of the signature L; [M3] works out several examples of
this. Thus, in these cases, the slogan above becomes a tenet widely held
true by category theorists. I claim its validity in the generality stated
above.

The main effort in [M3] goes into specifying a language, First Order Logic
with Dependent Sorts, and showing that the first order properties
invariant under L-equivalence are precisely the ones that can be defined
in FOLDS. In this paper, the language of FOLDS plays no role. The concepts
of "FOLDS-signature" and "FOLDS-equivalence" are fully described in
section 1 of this paper. 

The definition of \textbf{multitopic \omega -category} goes, in outline, as
follows. For an arbitrary multitope SIGMA of dimension >=2, for a
multitopic set S, for a pasting diagram ALPHA in S of shape the domain of
SIGMA and a cell a in S of the shape the codomain of SIGMA, such that a
and ALPHA are parallel, we define what it means to say that a is a
\textbf{composite} of ALPHA. First, we define an auxiliary FOLDS signature
L<SIGMA> extending Mlt, the signature of multitopic sets. Next, we define
structures S<a> and S<ALPHA>, both of the signature L<SIGMA>, the first
constructed from the data S and a , the second from S and ALPHA, both
structures extending S itself. We say that a is a composite of ALPHA if
there is a FOLDS-equivalence-span E between S<a> and S<ALPHA> that
restricts to the identity equivalence-span from S to S . Below, I'll refer
to  E as an \textbf{equipment} for  a  being a composite of ALPHA. A multitopic
set is a \textbf{mulitopic \omega -category} iff every pasting diagram  ALPHA in it
has at least one composite.

The analog of the universal arrows in the Baez/Dolan style definition is
as follows. A \textbf{universal arrow} is defined to be an arrow of the form
b:ALPHA-----> a where  a  is a composite of ALPHA via an equipment E that
relates b with the identity arrow on  a : in turn, the identity arrow on
a  is any composite of the empty pasting diagram of dimension  dim(a)+1
based on  a . Note that the main definition does \emph{not} go through first
defining "universal arrow". 

A new feature in the present treatment is that it aims directly at weak
*\omega *-categories; the finite dimensional ones are obtained as truncated
versions of the full concept. The treatment in [B/D] concerns finite
dimensional weak categories. It is important to emphasize that a
multitopic \omega -category is still just a multitopic set with additional
properties, but with no extra data.

The definition of "multitopic \omega -category" is given is section 5; it
uses sections 1, 2 and 4, but not section 3.

The second main thing done in this paper is the definition of MltOmegaCat.
This is a particular large multitopic set. Its definition is completed
only by the end of the paper. The 0-cells of MltOmegaCat are the samll
multitopic \omega -categories, defined in section 5. Its 1-cells, which we
call 1-transfors (thereby borrowing, and altering the meaning of, a term
used by Sjoerd Crans [Cr]) are what stand for "morphisms", or "functors",
of multitopic \omega -categories. For instance, in the 2-dimensional case,
multitopic 2-categories correspond to ordinary bicategories by a certain
process of "cleavage", and the 1-transfors correspond to homomorphisms of
bicategories [Be]. There are n-dimensional transfors for each n in N . For
each multitope (that is, "shape" of a higher dimensional cell) PI, we
have the \textbf{PI-transfors}, the cells of shape PI in MltOmegaCat.

For each fixed multitope PI, a PI-transfor is a *PI-colored multitopic
set* with additional properties. "PI-colored multitopic sets" are defined
in section 3; when PI is the unique zero-dimensional multitope, PI-colored
multitopic sets are the same as ordinary multitopic sets. Thus, the
definition of a transfor of an arbitrary dimension and shape is a
generalization of that of "multitopic \omega -category"; the additional
properties are also similar, they being defined by FOLDS-based universal
properties. There is one new element though. For dim(PI)>=2 , the concept
of PI-transfor involves a universal property which is an \omega -dimensional, 
FOLDS-style generalization of the concept of right Kan-extension (right 
lifting in the terminology used by Ross Street).  This is a "right-adjoint" 
type universal property, in contrast to the "left-adjoint" type involved 
in the concept of composite (which is a generalization of the usual 
tensor product in modules). 

The main theorem, stated but not proved here, is that MltOmegaCat is a
multitopic \omega -category. 

The material in this paper has been applied to give formulations of
\omega -dimensional versions of various concepts of homotopy theory;
details will appear elesewhere.

References:

[B/D]	J. C. Baez and J. Dolan, Higher-dimensional algebra III.
n-categories and the algebra of opetopes. Advances in Mathematics 135
(1998), 145-206.

[Be]	J. Benabou, Introduction to bicategories. In: Lecture Notes in
Mathematics 47 (1967), 1-77 (Springer-Verlag). 

[Cr]	S. Crans, Localizations of transfors. Macquarie Mathematics
Reports no. 98/237. 

[H/M/P]	C. Hermida, M. Makkai and J. Power, On weak higher dimensional
categories I. Accepted by: Journal of Pure and Applied Algebra. Available
electronically (when the machines work ...).

[L]	J. Lambek, Deductive systems and categories II. Lecture Notes in
Mathematics 86 (1969), 76-122 (Springer-Verlag). 

[M2]	M. Makkai, Towards a categorical foundation of mathematics. In:
Logic Colloquium '95 (J. A. Makowski and E. V. Ravve, editors). Lecture
Notes in Logic 11 (1998) (Springer-Verlag). 

[M3]	M. Makkai, First Order Logic with Dependent Sorts. Research
monograph, accepted by Lecture Notes in Logic (Springer-Verlag). Under
revision. Original form available electronically (when the machines
work ...). 

\end{quote}


 \par\noindent\rule{\textwidth}{0.4pt}

% </A>
% </A>
% </A>
