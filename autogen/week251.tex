
% </A>
% </A>
% </A>
\week{May 5, 2007 }


Last week I mentioned the conference on "Philosophical and
Formal Foundations of Modern Physics" in Les Treilles, an estate
near Nice.  On our last night there, the chef showed us his 
telescope.  We saw the phase of Venus, mountains on the Moon, 
and - best of all - the rings of Saturn!  They were beautiful.  
I was reminded of Galileo, who had to make do with a much cruder
telescope.

Here's an even better view - a photo taken by the Cassini probe
on March 1st, from a distance of 1.2 million kilometers:

<div align = "center">
<img src = "saturn.jpg">
</div>

1) Cassini-Huyghens, Tourniquet shadows,
<a href = "http://saturn.jpl.nasa.gov/multimedia/images/image-details.cfm?imageID=2507">http://saturn.jpl.nasa.gov/multimedia/images/image-details.cfm?imageID=2507</a>

I learned some fun stuff about the foundations of quantum mechanics
at Les Treilles, so I want to mention that before I forget!
I'll take a little break from the Tale of Groupoidification... though
if you've been following carefully, you may see it lurking beneath the
surface.

Lately people have been developing "foils for quantum mechanics":
theories of physics that aren't classical, but aren't ordinary 
quantum theory, either.  These theories can lack some of the 
weird features of quantum theory... or, they may have "supra-quantum"
features, like the Popescu-Rohrlich box I mentioned last week.   

The idea is not to take these theories seriously as models of our 
universe - though one can always dream.  Instead, it's to explore 
the logical possibilities, so we can see quantum mechanics and 
classical mechanics as just two examples from a larger field of
options, and better understand what's special about them.

Rob Spekkens is a young guy who used to be at the Perimeter Institute;
now he's at DAMTP in Cambridge.  At Les Treilles he gave a cool talk
about a simple theory that mimics some of features of quantum mechanics:

2) Evidence for the epistemic view of quantum states: a toy theory,
Phys. Rev. A 75, 032110 (2007).  Also available as <A HREF = "http://xxx.lanl.gov/abs/quant-ph/0401052">quant-ph/0401052</A>.

The idea is to see how far you get using a very simple principle, 
namely: even when you know as much as you can, there's an equal 
amount you don't know.

In this setup, the complete description of a physical system involves
N bits of information, but you can only know N/2 of them.  When you do 
an experiment to learn more information than that, the system's state 
changes in a random way, so something you knew become obsolete.

The fraction "1/2" here is chosen for simplicity: it's just a toy 
theory.  But, it leads to some charming mathematical structures 
that I'd like to understand better.

In this theory, the simplest nontrivial system is one whose state
takes two bits to describe - but you can know at most one.  Two bits 
of information is enough to describe four states, say states 1, 2, 3, 
and 4.  But, since you can only know one bit of information, you can't 
pin down the system's state completely.  At most you can halve the 
possibilities, and know something like "the system is in state 1 or 3".  
You can also be completely ignorant - meaning you only know "the 
system is in state 1, 2, 3 or 4".

Since there are 3 ways to chop a 4-element set in half, there are 
3 "axes of knowledge", namely

Is the system's state in {1,2} or {3,4}?<br>
Is the system's state in {1,3} or {2,4}?<br>
Is the system's state in {1,4} or {2,3}?

You can only answer one of these questions.

This has a cute resemblance to how you can measure the angular
momentum of a spin-1/2 particle along the x, y, or z axis, in 
each case getting two choices.  Spekkens has a nice picture
in his paper:
 

\begin{verbatim}

                  {1,2}
                    |
                    |  {2,4}
                    | /
                    |/       
      {1,4}-----{1,2,3,4}-----{2,3}
                   /| 
                  / |
             {1,3}  |
                    |
                  {3,4} 
\end{verbatim}
    

This octahedron is a discrete version of the "Bloch ball" describing 
mixed states of a spin-1/2 particle in honest quantum mechanics.  If
you don't know about that, I should remind you:

A "pure state" of a spin-1/2 particle is a unit vector in
C^{2}, modulo phase.  The set of these is just the Riemann sphere!

In a pure state, we know as much as we can know.  In a "mixed state",
we know less.  Mathematically, a mixed state of a spin-1/2 particle 
is a 2\times 2 "density matrix" - a self-adjoint matrix with nonnegative 
eigenvalues and trace 1.  These form a 3-dimensional ball, the "Bloch 
ball", whose boundary is the Riemann sphere.  

The x, y, and z coordinates of a point in the Bloch ball are the 
expected values of the three components of angular momentum for a 
spin-1/2 particle in the given mixed state.  The center of the Bloch 
ball is the state of complete ignorance.

In honest quantum mechanics, the rotation group SO(3) acts as symmetries 
of the Bloch ball.  In Spekken's toy version, this symmetry group is 
reduced to the 24 permutations of the set {1,2,3,4}.  You can think 
of these permutations as acting on a tetrahedron whose corners are the
4 states of our system.  The 6 corners of the octahedron above are the
midpoints of the edges of this tetrahedron!

Since Spekkens' toy system resembles a qubit, he calls it a "toy
bit".  He goes on to study systems of several toy bits - and the
charming combinatorial geometry I just described gets even more
interesting.  Alas, I don't really understand it well: I feel there
must be some mathematically elegant way to describe it all, but I
don't know what it is.

Just as you can't duplicate a qubit in honest quantum mechanics - the
famous <a href = "http://en.wikipedia.org/wiki/No_cloning_theorem">no-cloning 
theorem</a> - it turns out you can't
duplicate a toy bit.  However, <a href = "http://en.wikipedia.org/wiki/Bell's_theorem">Bell's theorem</a> on nonlocality and the
<a href = "http://plato.stanford.edu/entries/kochen-specker/">Kochen-Specker 
theorem</a> on contextuality don't apply to toy bits.
Spekkens also lists other similarities and differences.

All this is fascinating.  It would be nice to find the mathematical
structure that underlies this toy theory, much as the category of 
Hilbert spaces underlies honest quantum mechanics.  

In my talk at Les Treilles, I explained how the seeming weirdness of
quantum mechanics arises from how the category of Hilbert spaces
resembles not the category of sets and functions, but a category with
"spaces" as objects and "spacetimes" as morphism.
This is good, because we're trying to unify quantum mechanics with our
best theory of spacetime, namely general relativity.  In fact, I think
quantum mechanics will make more sense when it's part of a theory of
quantum gravity!  To see why, try this:

3) John Baez, Quantum quandaries: a category-theoretic perspective,
talk at Les Treilles, April 24, 2007, <a href = "http://math.ucr.edu/home/baez/treilles/">http://math.ucr.edu/home/baez/treilles/</a>

For more details, see my paper with the same title (see "<A HREF
= "week247.html">week247</A>").  

This fun paper by Bob Coecke gives another view of categories and
quantum mechanics, coming from work on quantum information theory:

4) Bob Coecke, Kindergarten quantum mechanics, available as 
<A HREF = "http://xxx.lanl.gov/abs/quant-ph/0510032">quant-ph/0510032</A>.

To dig deeper, try these:

5) Samson Abramsky and Bob Coecke, A categorical semantics of quantum
protocols, <A HREF =
"http://xxx.lanl.gov/abs/quant-ph/0402130">quant-ph/0402130</A>.
 
6) Peter Selinger, Dagger compact closed categories and completely 
positive maps, available at 
<a href = "http://www.mscs.dal.ca/~selinger/papers.html#dagger">http://www.mscs.dal.ca/~selinger/papers.html#dagger</a>

Since the category-theoretic viewpoint sheds new light on the
no-cloning theorem, Bell's theorem, quantum teleportation, and the
like, maybe we can use it to classify "foils for quantum
mechanics".  Where would Spekkens' theory fit into this
classification?  I want to know!

Another mathematically interesting talk was by Howard Barnum, 
who works at Los Alamos National Laboratory.  Barnum works on a general
approach to physical theories using convex sets.  The idea is that 
in any reasonable theory, we can form a mixture or "convex linear 
combination" 

px + (1-p)y

of mixed states x and y, by putting the system in state x with
probability p and state y with probability 1-p.  So, mixed states
should form a "convex set".

The Bloch sphere is a great example of such a convex set.  Another 
example is the octahedron in Spekken's theory.  Another example is
the tetrahedron that describes the mixed states of a classical 
system with 4 pure states.  Spekken's octahedron is a subset of 
this tetrahedron, reflecting the limitations on knowledge in his 
setup.  

To learn about the convex set approach, try these papers:

7) Howard Barnum, Quantum information processing, operational 
quantum logic, convexity, and the foundations of physics, available
as <A HREF = "http://xxx.lanl.gov/abs/quant-ph/0304159">quant-ph/0304159</A>.

8) Jonathan Barrett, Information processing in generalized 
probabilistic theories, available as <A HREF = "http://xxx.lanl.gov/abs/quant-ph/0508211">quant-ph/0508211</A>.

9) Howard Barnum, Jonathan Barrett, Matthew Leifer and Alexander Wilce,
Cloning and broadcasting in generic probabilistic theories, 
available as <A HREF = "http://xxx.lanl.gov/abs/quant-ph/0611295">quant-ph/0611295</A>.

Actually I've been lying slightly: these papers also allow mixtures
of states

px + qy

where p+q is less than or equal to 1.   For example, if you prepare 
an electron in the "up" spin state with probability p and the 
"down" state with probability q, but there's also a chance that you
drop it on the floor and lose it, you might want p+q < 1.

I'm making it sound silly, but it's technically nice and maybe even 
conceptually justified.   Mathematically it means that instead 
of a convex set of states, you have a vector space equipped with 
a convex cone and a linear functional P such that the cone is 
spanned by the "normalized" states: those with P(x) = 1.  This
is very natural in both classical and quantum probability theory.

Quite generally, the normalized states form a convex set.  
Conversely, starting from a convex set, you can create a vector 
space equipped with a convex cone and a linear functional with 
the above properties. 

So, I was only lying slightly.  In fact, a complicated 
web of related formalisms have been explored; you can learn 
about them from Barnum's paper. 

For example, the convex cone formalism seems related to the Jordan
algebra approach described in "<A HREF =
"week162.html">week162</A>".  Barnum cites a paper by Araki that
shows how to get Jordan algebras from sufficiently nice convex cones:

10) H. Araki, On a characterization of the state space of quantum
mechanics, Commun. Math. Phys. 75 (1980), 1-24.

It's a very interesting paper but a wee bit too technical for me to
feel like summarizing here.

Some nice examples of Jordan algebras are the 2\times 2 self-adjoint matrices 
with real, complex, quaternionic or octonionic entries.  Each of these 
algebras has a cone consisting of the nonnegative matrices, and the trace 
gives a linear functional P.  The nonnegative matrices with trace = 1 
are the mixed states of a spin-1/2 particle in 3, 4, 6, and 10-dimensional 
spacetime, respectively!  In each case these mixed states form a convex 
set: a round ball generalizing the Bloch ball.  Similarly, the pure states 
form a sphere generalizing the Riemann sphere.

Back in "<A HREF = "week162.html">week162</A>" I explained
how these examples are related to special relativity and spinors in
different dimensions.  It's so cool I can't resist reminding you.

Our universe seems to like complex quantum mechanics.  And, the space
of 2\times 2 self-adjoint complex matrices - let's call it
h_{2}(C) - is isomorphic to 4-dimensional Minkowski spacetime!
The cone of positive matrices is isomorphic to the future lightcone.
The set of pure states of a spin-1/2 particle is the Riemann sphere
CP^{1}, and this is isomorphic to the "heavenly
sphere": the set of light rays through a point in Minkowski
spacetime.

This whole wonderful scenario works just as well in other dimensions
if we replace the complex numbers (C) by the real numbers (R), the 
quaternions (H) or the octonions (O):

<ul>
<li>
h_{2}(R) is 3d Minkowski spacetime, and RP^{1} is the heavenly sphere S^{1}.
</li>
<li>
h_{2}(C) is 4d Minkowski spacetime, and CP^{1} is the heavenly sphere S^{2}.
</li>
<li>
h_{2}(H) is 6d Minkowski spacetime, and HP^{1} is the heavenly sphere S^{4}.
</li>
<li>
h_{2}(O) is 10d Minkowski spacetime, and OP^{1} is the heavenly sphere S^{8}.
</li>
</ul>

So, it's all very nice - but a bit mysterious.  Why did our universe
choose the complex numbers?  We're told that scientists shouldn't ask
"why" questions, but that's not really true - the main thing
is to do it only to the extent that it leads to progress.  But, sometimes
you just can't help it.

String theorists occasionally think about 10d physics using the octonions,
but not much.  The strange thing about the octonions is that the 
self-adjoint nxn octonionic matrices h_{n}(O) only form a Jordan algebra 
when n = 1, 2, or 3.  So, it seems we can only describe very small 
systems in octonionic quantum mechanics!  Nobody knows what this means.

People working on the foundations of quantum mechanics have also
thought about real and quaternionic quantum mechanics.
h_{n}(R), h_{n}(C) and h_{n}(H) are Jordan
algebras for all n, so the strange limitation afflicting the octonions
doesn't affect these cases.  But, I wound up sharing a little cottage
with Lucien Hardy at Les Treilles, and he turns out to have thought
about this issue.  He pointed out that something interesting happens
when we try to combine two quantum systems by tensoring them.  The
dimensions of h_{n}(C) behave quite nicely:

dim(h_{nm}(C)) = dim(h_{n}(C)) dim(h_{m}(C))

But, for the real numbers we usually have

dim(h_{nm}(R)) > dim(h_{n}(R)) dim(h_{m}(R))

and for the quaternions we usually have

dim(h_{nm}(H)) < dim(h_{n}(H)) dim(h_{m}(H))

So, it seems that when we combine two systems in real quantum
mechanics, they sprout mysterious new degrees of freedom!  More
precisely, we can't get all density matrices for the combined system
as linear combinations of tensor products of density matrices for the
two systems we combined.  For the quaternions the opposite effect
happens: the combined system has fewer mixed states than we'd expect.

This observation lurks behind axiom 4 in this paper:

11) Lucien Hardy, Quantum theory from five reasonable axioms, 
available as <A HREF = "http://xxx.lanl.gov/abs/quant-ph/0101012">quant-ph/0101012</A>.

Another special way in which C is better than H or R is that
only for a complex Hilbert space is there a correspondence between
continuous 1-parameter groups of unitary operators and self-adjoint
operators.  We always get a <i>skew-adjoint</i> operator, but only
in the complex case can we convert this into a self-adjoint operator
by dividing by i.  

Here are some more references, kindly provided by Rob Spekkens.  The 
pioneering quantum field theorist St&uuml;ckelberg wrote a bunch of papers 
on real quantum mechanics.  Spekkens recommends this one:

12) E. C. G. St&uuml;ckelberg, Quantum theory in real Hilbert space,
Helv. Phys. Acta 33, 727 (1960).

This is a modern review:

13) Jan Myrheim, Quantum mechanics on a real Hilbert space, available
<A HREF = "http://xxx.lanl.gov/abs/quant-ph/9905037">quant-ph/9905037</A>.

What I find most fascinating is the connection between real quantum
mechanics and time reversal symmetry.  In ordinary complex quantum
mechanics, time reversal symmetry is sometimes described by a
conjugate-linear (indeed "antiunitary") operator T with
T^{2} = 1.  Such an operator is precisely a "real
structure" on our complex Hilbert space: it picks out a real
Hilbert subspace of which our complex Hilbert space is the
complexification.

It's worth adding that in the physics of fermions, another possibility
occurs: an antiunitary time reversal operator with T^{2} = -1.
This is precisely a "quaternionic structure" on our complex
Hilbert space: it makes it into a quaternionic Hilbert space!

For more on these ideas try:

14) Freeman J. Dyson, The threefold way: algebraic structure of 
symmetry groups and ensembles in quantum mechanics, Jour. Math. Phys. 3 
(1962), 1199-1215.

15) John Baez, Symplectic, quaternionic, fermionic, 
<a href = "http://math.ucr.edu/home/baez/symplectic.html">http://math.ucr.edu/home/baez/symplectic.html</a>

From all this one can't help but think that complex, real, and quaternionic
quantum mechanics fit together in a unified structure, with the complex
numbers being the most important, but other two showing up naturally
in systems with time reversal symmetry.  

Stephen Adler - famous for the Adler-Bell-Jackiw anomaly - spent 
a long time at the Institute for Advanced Studies working on 
quaternionic quantum mechanics:

16) S. L. Adler, Quaternionic Quantum Mechanics and Quantum Fields, 
Oxford U. Press, Oxford, 1995.

A problem with this book is that it defines a quaternionic vector
space to be a \emph{left} module of the quaternions, instead of a
\emph{bimodule}.  This means you can't naturally tensor two
quaternionic vector spaces and get a quaternionic vector space!  Adler
"solves" this problem by noting that any left module of the
quaternions becomes a right module, and in fact a bimodule, via

xq = q*x

But, when you're working with
a noncommutative ring, you really need to think about left modules,
right modules, and bimodules to understand the theory of tensor products.
And, the quaternions have more bimodules than you might expect: for
example, for any automorphism of the quaternions:

f: H \to  H 

there's a way to make H into an H-bimodule with the obvious left action 
and a "twisted" right action, where q acts on x to give

x f(q)

Since the automorphism 
group of the quaternions is SO(3), there turn out to be SO(3)'s worth of 
nonisomorphic ways to make H into an H-bimodule!

For an attempt to tackle this issue, see:

17) John Baez and Toby Bartels, Functional analysis with quaternions,
available at <a href = "http://toby.bartels.name/papers/#quaternions">http://toby.bartels.name/papers/#quaternions</a>

However, it's possible we'll only see what real and quaternionic
quantum mechanics are good for when we work in the 3-category Alg(R)
mentioned in "<A HREF = "week209.html">week209</A>", taking
R to be the real numbers.  Here:

<ul>
<li>
there's one object, the real numbers R.
</li>
<li>
the 1-morphisms are algebras A over R.
</li>
<li>
the 2-morphisms M: A \to  B are (A,B)-bimodules.
</li>
<li>
the 3-morphisms f: M \to  N are (A,B)-bimodule morphisms.
</li>
</ul>

This could let us treat real, complex and quaternionic quantum
mechanics as part of a single structure.

Dreams, dreams....

\par\noindent\rule{\textwidth}{0.4pt}
\textbf{Addenda:} In email, Scott Aaronson pointed out this nice webpage:

18) Scott Aaronson, Lecture 9: Quantum, 
<a href="http://www.scottaaronson.com/democritus/lec9.html">http://www.scottaaronson.com/democritus/lec9.html</a>

He wrote:

\begin{quote}

I talk all about the known differences between QM over the complex
numbers and QM over the reals and quaternions (including the
parameter-counting difference you mentioned, but also a couple you
didn't), and why the universe might've gone with complex numbers.

\end{quote}

His lecture also cites this paper:

19) Carlton M. Caves, Christopher A. Fuchs, and Ruediger Schack, 
Unknown quantum states: the quantum de Finetti representation,
available as 
<a href="http://www.arxiv.org/abs/quant-ph/0104088">
quant-ph/0104088</a>.

which Rob Spekkens also pointed out to me.

The quantum de Finetti theorem is a generalization of the 
<a href="http://en.wikipedia.org/wiki/De_Finetti's_theorem">classical 
de Finetti theorem</a>.   Both classical quantum de Finetti theorems 
are about n copies of a system sitting side by side in an 
"exchangeable" state: a state that's not only invariant under 
permutations of the copies, but lacking correlations between the 
different copies!

Here's the quantum de Finetti theorem.  Suppose you have an 
"exchangeable" density operator \rho _{n} 
on H^{\otimes  n} - that is, one such that for each 
N \ge  n, there's a density operator \rho _{N} 
on H^{\otimes  N} which 1) is invariant under permutations 
in S_{N} and 2) has \rho  as its marginal, meaning that

Tr (\rho _{N}) = \rho _{n}  

where Tr is the partial trace map sending operators on 
H^{\otimes  N} to operators on 
H^{\otimes  n}. 
Then, \rho _{n} is a mixture of density matrices of the form 
\rho  \otimes  ... \otimes  \rho : a tensor product of n copies of 
a density matrix on H.

This is completely plausible if you know what all this jargon means.

And now for the punch line: <i>This theorem would \textbf{fail} if we did 
quantum mechanics using the real numbers!</i>

Of course, this is related to the fact I mentioned this Week, 
namely that for real quantum mechanics, "the whole is more 
than the product of its parts" in a more severe way than for 
complex quantum mechanics.

Bob Coecke wrote:

\begin{quote}
The standard references on quaternionic QM are:

20) D. Finkelstein, J.M. Jauch, S. Schiminovich and D. Speiser, Foundations of 
quaternion quantum mechanics, Journal of Mathematical Physics 3, 207 (1962).

21)
D. Finkelstein, J.M. Jauch, S. Schiminovich and D. Speiser, Some physical consequences of general Q-covariance, Helvetica Physica Acta 35, 328-329 (1962).

22) D. Finkelstein, J.M. Jauch, S. Schiminovich and D. Speiser, Principle of 
general Q-covariance, Journal of Mathematical Physics 4, 788-796 (1963).

A standard structural result in the order-theoretic vein which separates Reals, Complex Numbers and Quaternions from ''non-classical fields'' is:

23) M. P. Soler (1995) Characterization of Hilbert spaces with orthomodular 
spaces, Comm. Algebra 23, pp. 219-243.

It does this relative to the order-theoretic characterization of Hilbert 
spaces:

24) C.  Piron (1964, French) Axiomatique Quantique, Helv. Phys. Acta 37, 
pp. 439-468. 

25) I. Amemiya and H. Araki (1966) A Remark on Piron's Paper, Publ. Res. Inst. Math. Sci. Ser. A 2, pp. 423&#x2013;427. 

24) C. Piron (1976) Foundations of Quantum Physics, W. A. Benjamin, Inc., 
Reading. 

A nicely written recent survey on this stuff is:

26) Isar Stubbe and B. van Steirteghem (2007) Propositional systems,
Hilbert lattices and generalized Hilbert spaces, chapter in: Handbook
Quantum Logic (edited by D. Gabbay, D. Lehmann and K. Engesser),
Elsevier, to appear.  Available at <a href =
"http://www.win.ua.ac.be/~istubbe/">http://www.win.ua.ac.be/~istubbe/</a>

It is not clear to me how exactly this order-theoretic stuff relates
to the \emph{thick} categorical axiomatics for QM John mentioned
above.  One key difference is that in the order-theoretic axiomatics
one failed to find an abstract counterpart to the Hilbert space tensor
product.  (ie without having to say that we are working in the lattice
of closed subspaces of a Hilbert space) On the other hand, the
categorical approach starting from symmetric monoidal categories takes
that description of compound systems as an a priori.  Singling out the
complex numbers is done in terms of two involutions on morphisms, one
covariant and one contravariant, where the covariant one capture
complex conjugation ie the unique non-trivial automorphism
characteristic of complex numbers.  The contravariant one captures
transposition and together they make up the adjoint.

\end{quote}

Here "thick" refers to working with categories which nice
big hom-sets, instead of mere posets or preorders, which are categories 
with at most one morphism from one object to another.

Rob Spekkens also gives some references on quantum computation
in real quantum mechanics.  He writes:

\begin{quote}
See also:

27) C. M. Caves, C. A. Fuchs, P. Rungta, Entanglement of formation of an
arbitrary state of two rebits, available as <a href = "http://arxiv.org/abs/quant-ph/0009063">quant-ph/0009063</a>.

It's also worth noting that quantum computation and quantum cryptography do
not require the complex field.  Have a look at:

28) T. Rudolph and L. Grover, A 2 rebit gate universal for quantum
computing, 
<a href = "http://arxiv.org/abs/quant-ph/0210187">quant-ph/0210187</a>.

I actually know of no information-theoretic task whose possibility is
contingent on the nature of the number field.
\end{quote}

More discussion (and pictures!) can be found at the <a href = "http://golem.ph.utexas.edu/category/2007/05/this_weeks_finds_in_mathematic_12.html">\emph{n}-Category
Caf&eacute;</a>.

\par\noindent\rule{\textwidth}{0.4pt}
\par\noindent\rule{\textwidth}{0.4pt}

% </A>
% </A>
% </A>
