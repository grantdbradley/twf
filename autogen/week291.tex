
% </A>
% </A>
% </A>
\week{January 22, 2010 }


This week I want to ask for references - references on a cool relationship
between Julia sets and the Mandelbrot set.  Then, we'll delve further
into electrical circuits and analogous systems.  No more rational
homotopy theory, I'm afraid!  There's a lot more to say, but I've been
thinking about other things.  These days I'm trying to crank out one
This Week's Finds every week.  I may give up on that soon... but I
want to finish this one today, and it's 9 pm, and I haven't had dinner.

First: if you're into n-categories, you have to check out Carlos Simpson's
new book:

1) Carlos Simpson, Homotopy theory of higher categories, 
draft available as <a href = "http://hal.archives-ouvertes.fr/hal-00449826/fr/">http://hal.archives-ouvertes.fr/hal-00449826/fr/</a>

It's very readable, with a long historical introduction that'll
help you understand the motivations behind current work, and a
warmup section on strict n-categories - which are relatively easy - before 
diving into the subtleties of weak ones.  It compares many approaches
to weak n-categories before explaining his own.  

This could be the book the world has been waiting for!  And he's
asking for comments and corrections, so you can help make it better.

Next: a little music.  Mike Stay pointed me to a great video
illustrating the first piece from Bach's Musical Offering.  Jos Leys
did the animation, while a physics blogger with the monicker
"Xantox" played the music:

<div align = "center">
<object width="425" height="344"><param name="movie" value="http://www.youtube.com/v/xUHQ2ybTejU&hl=en_US&fs=1&"></param><param name="allowFullScreen" value="true"></param><param name="allowscriptaccess" value="always"></param><embed src="http://www.youtube.com/v/xUHQ2ybTejU&hl=en_US&fs=1&" type="application/x-shockwave-flash" allowscriptaccess="always" allowfullscreen="true" width="425" height="344"></embed></object>
</div>

2) Jos Leys, <a href = "http://www.josleys.com/Canon/Canon.html">http://www.josleys.com/Canon/Canon.html</a> 

3) Xantox, Canon 1 a 2, at his blog Strange Paths,
<a href = "http://strangepaths.com/canon-1-a-2/2009/01/18/en/">http://strangepaths.com/canon-1-a-2/2009/01/18/en/M</a>

This is a "crab canon", meaning roughly a melody that sounds
good when you play it both forwards and backwards, simultaneously.
Bach wrote it after Frederick the Great invited him to the Prussian
court in Berlin.  When Bach arrived, he was asked to test the king's
new pianos.  The king proposed a musical theme and asked Bach to
improvise a fugue based on it.

Legend has it that Bach immediately improvised two: one for three 
voices, and one for six!  And later, after returning to his home in 
Leipzig, Bach composed a set of canons and a trio sonata featuring 
the king's theme, and sent the whole lot to Frederick as 
a "Musiche Opfer", or musical offering.

The whole Musical Offering is a \emph{tour de force} - the sort of
highly patterned thing you'd expect mathematicians to like.  It
consists mainly of "strict canons".  In a strict canon,
first you start playing one melody, called the "leader".
Then, while that melody is going on, you start playing another, the
"follower", which is an exact copy of the leader - except
perhaps transposed to a different pitch.

The hard part is to make the leader and follower fit beautifully when
they're both going on.  If you need to bend the rules to make your
canon sound better, that's okay - but then it's not
"strict".

A crab canon, which is very rare, bends the rules by letting the
follower be an upside-down version of the leader.  This style is
\emph{not} for wimps who can't write a good strict canon: it's for
people like Bach who find strict canons insufficiently challenging.  

The crab canon is not the only tricky feat in the Musical Offering.
For example, the fifth piece is a "spiral canon", designed
to sound good if you play it over and over, but going up a whole step
each time.  And the eighth piece is a "mirror canon" Here
the follower is an upside-down version of the leader!

I first learned this stuff here, back when I was a teenager:

4) Douglas Hofstadter, G&ouml;del, Escher, Bach: an Eternal Golden Braid,
Basic Books, 1979.

I feel sort of silly recommending this book.  You must have already
read it!  But maybe not.  I can imagine various good excuses.  Maybe
you were just recently born, or something.  Anyway: if you like logic,
self-reference, goofiness, puzzles and puns, and you haven't read this
book yet, do it now!  But if you hate such things, you're excused.
Hofstadter's humor might grate on some people's nerves.

While it's fun to read about crab canons, and fun to listen to them, 
you may have trouble fully appreciating them unless you see the score
while you're listening.  And that's one reason the video by Jos Leys
and Xantox is so great.  

For more on the Musical Offering, try these:

5) Timothy A. Smith, Canons of the Musical Offering, 
<a href = "http://jan.ucc.nau.edu/~tas3/musoffcanons.html">http://strangepaths.com/canon-1-a-2/2009/01/18/en/</a>

6) Tony Phillips, Math and the Musical Offering,
<a href = "http://www.ams.org/featurecolumn/archive/canons.html">http://www.ams.org/featurecolumn/archive/canons.html</a>

Next: there's an incredibly cool relationship between the Mandelbrot
set and all the Julia sets.  Somehow somebody neglected to tell me
about it when I was first learning about fractals.  They ought to be 
sued!  I just learned about it from Jesse McKeown over at the
\emph{n}-Category Caf&eacute;, and I want some good references on it.  
I don't understand it as well as I'd like!  But I can show it to you.

Consider this function of two complex variables:

z |\to  z^{2} + c

If we fix a number c, this function defines a map from the complex
plane to itself.  We can start with any number z and keep applying
this map over and over.  We get a sequence of numbers.  Sometimes this
sequence shoots off to infinity and sometimes it doesn't.  The 
boundary of the set where it doesn't is called the "Julia set" for 
this number c.

On the other hand, we can start with z = 0, and draw the set of
numbers c for which the resulting sequence doesn't shoot off to
infinity.  That's called the "Mandelbrot set".

Here's the cool relationship: in the vicinity of the number c, the
Mandelbrot set tends to look like the Julia set for that number c.
This is especially true right at the boundary of the Mandelbrot set.

For example, this is the Julia set for c = -0.743643887037151 + 0.131825904205330 i:

<div align = "center">
<img border = "2" src = "julia_-0.743643887037151+0.131825904205330i.jpg">
% </a>
</div>

while this:

<div align = "center">
<img border = "2" src = "mandelbrot_-0.743643887037151+0.131825904205330i.jpg">
% </a>
</div>

is a tiny patch of the Mandelbrot set centered at the
same value of c.  They're shockingly similar!

7) Wikipedia, Mandelbrot sets: relationship with Julia sets,
<a href = "http://en.wikipedia.org/wiki/Mandelbrot_set#Relationship_with_Julia_sets">http://en.wikipedia.org/wiki/Mandelbrot_set#Relationship_with_Julia_sets</a>

This is why the Mandelbrot set is so complicated.  Julia sets are
already very complicated.  But the Mandelbrot set looks like <i>a
lot</i> of Julia sets!

Here's a great picture illustrating this fact.  Click on it for a bigger
view:

<div align = "center">
<a href = "725_Julia_sets.png">
<img border = "2" src = "696px-725_Julia_sets.png">
% </a>
</div>

8) Wikimedia Commons, 725 Julia sets,
<a href = "http://commons.wikimedia.org/wiki/File:725_Julia_sets.png">
http://commons.wikimedia.org/wiki/File:725_Julia_sets.png</a>

It's a big picture made of lots of little pictures of Julia sets for
various values of c...  but it mimics the Mandelbrot set.  You'll
notice that the Mandelbrot set is the set of numbers c whose Julia
sets are connected.  Those Julia sets are the black blobs.  When c
leaves the Mandelbrot set, its Julia set falls apart into dust: that's
the white stuff.

For an even better view of this phenomenon, try this:

9) David Joyce, Mandelbrot and Julia set explorer, 
<a href = "http://aleph0.clarku.edu/~djoyce/julia/explorer.html">http://aleph0.clarku.edu/~djoyce/julia/explorer.html</a>

You can zoom into the Mandelbrot set and see the corresponding Julia
set at various values of c.  For example, here's the Julia set at
c = -0.689494949 - 0.462323232 i:

<div align = "center">
<img border = "2" src = "julia_-0.689494949-0.462323232i.gif">
</div>

and here's a tiny piece of Mandelbrot set near that point:

<div align = "center">
<img border = "2" src = "mandelbrot_-0.689494949-0.462323232i.gif">
</div>

Does anyone know a good introduction to this phenomenon?  Apparently
it's the key to all deep work on the Mandelbrot set.

Last week I explained five kinds of circuit elements: resistances,
capacitances, inertances, effort sources and flow sources.  All these
are "1-ports", meaning they have one wire coming in and one
going out:


\begin{verbatim}

      |
      V
      |
    -----
   |     |
   |     |
    -----
      |
      V
      |
\end{verbatim}
    

Today I want to talk about 2-ports and 3-ports.  From these, we can
build all the more complicated circuits we'll be wanting to study.
But first, just for fun, here's some very basic stuff about one of the
1-ports I just listed.  Namely: effort sources.

We see plenty of effort sources in everyday life.  Indeed, all the
technology in a modern home relies on them!  

For starters, batteries try to act like constant voltage sources.  For
example, a 9-volt battery tries to provide

V(t) = 9

Why do I say "tries"?  Because this is an idealization.  If
you take a perfect constant voltage source and connect its input and
output with a perfectly conductive wire:


\begin{verbatim}

        ________
       /        \
      |          |
      V          |
      |          |
    -----        |
   |     |       |
   |     |       |
    -----        |
      |          |
      V          |
      |          |
       \________/

\end{verbatim}
    
you'll get an infinite current!  In reality, if you connect the two
terminals of battery with a highly conductive copper wire, you'll get
a short circuit: a large amount of current which winds up destroying
the battery.

(Particle physicists should look at the above diagram and think about
how Feynman diagrams with closed loops in them lead to infinities.
Category theorists should think about "traces" and how sometimes
traces diverge.  It is my job to make these analogies precise.  But
not today.)

Electrical outlets also do their best to act like voltage sources.
But they put out alternating current, so the voltage wiggles like a
sine wave.  In America, from Canada down to Ecuador, outlets mostly
try to produce this voltage:

V(t) = \sqrt 2 120 sin(2&pi 60t + c)
  
where c is some undetermined constant.  People say they put out 120
volts at a frequency of 60 hertz.  But this 120 volts is the
"root-mean-square" voltage.  To get the "peak"
voltage we need to multiply by the square root of 2, for reasons
explained here:

10) Wikipedia, Root mean square: average electrical power, at
<a href = "http://en.wikipedia.org/wiki/Root_mean_square#Average_electrical_power">http://en.wikipedia.org/wiki/Root_mean_square#Average_electrical_power</a>

That's where the square root of 2 comes from.  Also, in electrical
engineering, a frequency of 60 hertz means you've got a wave that
makes 60 full cycles per second, so we need a 2\pi  in the above
formula.  Physicists often define frequency a different way, that
doesn't require the 2\pi .  This causes violent fistfights when
engineers meet physicists.

In most of the rest of the world, outlets try to produce 240 volts 
at a frequency of 50 hertz, so

V(t) = \sqrt 2 240 sin(2\pi  50t + c)

But humans can never agree on anything.  So, there are also countries
that do lots of other things - and countries like Brazil that do a
mixture of things: 115 volts, 127 volts or 220 volts at 60 hertz,
depending on where you are!  

Why does Brazil use three voltages?  Why did Australia convert from 
240 volts to 230 in the year 2000?  Why do some parts of Japan use 50 
hertz current while others use 60 hertz, forcing Japanese appliances 
to have a switch that lets you pick which one you're using?  I don't 
know... but now I want to.  I have an endless capacity to find these
puzzles electrifying, once I let go of a certain mental resistance, 
which impedes me.

And let's not even get \emph{started} on the various types of plugs used 
in different countries!

11) Wikipedia, Mains electricity, 
<a href = "http://en.wikipedia.org/wiki/Mains_electricity">http://en.wikipedia.org/wiki/Mains_electricity</a>

12) Wikipedia, Mains power around the world,
<a href = "http://en.wikipedia.org/wiki/Mains_power_around_the_world">http://en.wikipedia.org/wiki/Mains_power_around_the_world</a>
 
Okay, now let's talk about 2-ports and 3-ports.  Remember, a 1-port
looks like this:


\begin{verbatim}

      |
      V
      |
    -----
   |     |
   |     |
    -----
      |
      V
      |
\end{verbatim}
    
If all we have is 1-ports, we can only build circuits by stringing
them together in series:


\begin{verbatim}

      |
      V
      |
    -----
   |     |
   |     |
    -----
      |
      V
      |
    -----
   |     |
   |     |
    -----
      |
      V
      |
    -----
   |     |
   |     |
    -----
      |
      V
      |
\end{verbatim}
    

or perhaps forming a closed loop:


\begin{verbatim}

        ___________
       /           \
      |             |
      V             |
      |             |
    -----           |
   |     |          |
   |     |          |
    -----           |
      |             |
      V             |
      |             |
    -----           |
   |     |          |
   |     |          |
    -----           |
      |             |
      V             |
      |             |
       \___________/

\end{verbatim}
    

This is sort of dull, though still worth understanding.  To have more
fun, we need some 2-ports or 3-ports!

A 2-port looks like this:


\begin{verbatim}

      |  |
      V  V
      |  |
    --------
   |        |
   |        |
    --------
      |  |
      V  V
      |  |
\end{verbatim}
    

The current flowing in the left wire on top must equal the current
flowing out the left wire on bottom - that's just a rule in this game.
And similarly for the wires on the right.  So, a 2-port has just two
flows, say q'_{1} and q'_{2}.  Similarly, it has two
efforts p'_{1} and p'_{2}.

Mathematically, we specify a 2-port by giving 2 equations involving 
these two efforts and flows, the corresponding momenta and 
displacements, and perhaps the time variable t.

The most popular 2-ports are very simple.  They are:

<ol>
<li>
A "transformer".  A transformer multiplies effort and divides
  flow:

  p'_{2} =  \  m \   p'_{1}<br/>
  q'_{2} = (1/m) q'_{1}

 If you bought some electrical equipment in Europe and you try to
 use it in the US, you need a transformer - although your equipment 
 may have one built in.  The transformer multiplies the voltage
 by the right number.  But thanks to some sad fact of life, it must
 also divide the current by that same number.  

  In mechanics, a lever acts as a transformer.  If you push on the 
  long end, the short end pushes with a force that's been multiplied
  by some number.  But thanks to some sad fact of life, the short end
  moves at a velocity that's been divided by that very same number!
</li>
<li>
  2. A "gyrator".  A gyrator trades effort for flow:

  p'_{2} =  \  r \   q'_{1} <br/>
  q'_{2} = (1/r) p'_{1}

  An example is a spinning gyroscope that's leaning completely 
  horizontally.  If you push it down slightly, its axis turns 
  at a rate proportional to your push.  So, it's trading angular
  velocity for torque!
</li>
</ol>

Both these 2-ports "conserve energy" in the sense I described last
week.  Of course we need to generalize that notion a bit, since we've 
got more ports now!  But it's easy.  In the conventions we're using
right now, the power absorbed by a 2-port equals

p'_{1} q'_{1} - p'_{2} q'_{2}

The minus sign here is one of many that plague this subject, like
flies in an impoverished, unsanitary tropical village.  I would like
to exterminate them all by a better choice of conventions, but I
haven't figured out the best way.  Luckily the signs don't really
matter much.  Here they seem to arise from treating the first port as
an "input" and the second as an "output".  
In other words, instead of this:


\begin{verbatim}

      |  |
      V  V
      |  |
    --------
   |        |
   |        |
    --------
      |  |
      V  V
      |  |
\end{verbatim}
    

people sometimes think of the 2-port this way:


\begin{verbatim}

  |           |
  |           |
  V           ^
  |   -----   |
  |  |     |  |
   --|     |--
   --|     |--
  |  |     |  |
  |   -----   |
  V           ^
  |           |
  |           |
\end{verbatim}
    

Anyway, if we use vectors and write 

p = (p_{1},p_{2})

q = (q_{1},q_{2})

then the power is some funny dot product of these vectors, namely

p' &middot; q' = p'_{1} q'_{1} - p'_{2} q'_{2}

for short.  And we say the 2-port "conserves energy" if we can find 
some function H(p,q) such that

dH(p,q)/dt = p' &middot; q'

Remember, H is the energy or "Hamiltonian".  So, this equation means
that when you pour power into the 2-port, its energy rises at exactly
the rate you'd expect.  And, you can check that both the transformer
and gyrator conserve energy according to this definition.

Next: 3-ports!  To build interesting circuits, we need the ability to
hook up two 1-ports in parallel, like this:


\begin{verbatim}

        |
        |
        |
       / \
      /   \
     /     \
   ---     ---
  |   |   |   |
   ---     ---
     \     /
      \   /
       \ /
        |
        |
        |
\end{verbatim}
    
But this gizmo, made of just wire:


\begin{verbatim}

        |
        |
        |
       / \
      /   \
     /     \

\end{verbatim}
    
is not an n-port of any kind, since it has an odd number of wires
coming out.

So, how can we connect 1-ports in parallel using just n-ports?

This puzzle had me stumped for a while.  But the answer is simple.  To
connect 1-ports in parallel, we need \emph{two} gizmos of the above
sort!  And taken together, they can be viewed as a 3-port!

In other words, there's a 3-port like this:


\begin{verbatim}

    |   |   |
  ooooooooooooo
 o             o
 o             o
 o             o
 o             o
 o             o
 o             o
 o             o
 o             o
 o             o
  ooooooooooooo
   |   |   |
\end{verbatim}
    

which you can use to connect two 1-ports in parallel.  You just attach
them like this:


\begin{verbatim}

           _____________________
          /   ___________       \
         /   /           \       \
    |   |   |             |       |
  ooooooooooooo           |       |
 o             o          |       |
 o             o          |       |
 o             o         ---     ---
 o             o        |   |   |   |
 o             o         ---     ---
 o             o          |       |
 o             o          |       |
  ooooooooooooo           |       |
    |   |   |             |       |
         \   \___________/       /  
          \_____________________/    
\end{verbatim}
    

What's in this 3-port?  Nothing but wires:



\begin{verbatim}

    |   |   |
  oo|ooo|ooo|oo
 o  |   |   |  o
 o  |   |   |  o
 o  |___|___|  o
 o             o
 o   ___ ___   o
 o  |   |   |  o
 o  |   |   |  o
  oo|ooo|ooo|oo
    |   |   |

\end{verbatim}
    

The little circles don't actually do anything here - they're just the
"packaging" that makes our 3-port seem impressive.   Inside, it's
just two three-pronged gizmos made of wire.  But if the customer can't 
see inside, we can sell it for a lot of money!  See how it works?


\begin{verbatim}

           _____________________
          /   ___________       \
         /   /           \       \
    |   |   |             |       |
  ooooooooooooo           |       |
 o  |   |   |  o          |       |
 o  |   |   |  o          |       |
 o  |___|___|  o         ---     ---
 o             o        |   |   |   |
 o   ___ ___   o         ---     ---
 o  |   |   |  o          |       |
 o  |   |   |  o          |       |
  ooooooooooooo           |       |
    |   |   |             |       |
         \   \___________/       /  
          \_____________________/    
\end{verbatim}
    

Current flows in at the upper left.  It gets split, goes through our
two 1-ports at right, gets rejoined, and exits at the lower left!

This 3-port is called a "parallel junction".  Henry Paynter,
who invented bond graphs - which we're gradually getting ready to
discuss - also called this 3-port a "0-junction".  And it's also
called a "flow junction", which makes some sense, since this
3-port takes the flow coming in and divides it in two.

Just as the mathematical description of a 1-port requires 1 equation,
while a 2-port requires 2, the description of a 3-port requires 3.
For the parallel junction they are:

  q'_{1} + q'_{2} + q'_{3} = 0<br/>
  p'_{1} = p'_{2} = p'_{3}

The first equation says that the total flow through is zero.  That's
obvious from the design: current can't flow from the top to the
bottom.  The other equations say that the voltage difference between
points 1 and 1' equals the voltage difference between points 2 and 2',
and also that between points 3 and 3':


\begin{verbatim}

    1   2   3
    |   |   |
  oo|ooo|ooo|oo
 o  |   |   |  o
 o  |   |   |  o
 o  |___|___|  o
 o             o
 o   ___ ___   o
 o  |   |   |  o
 o  |   |   |  o
  oo|ooo|ooo|oo
    |   |   |
    1'  2'  3'

\end{verbatim}
    
This is clear if you know a tiny bit about electrical circuits:
the voltage on each connected component of wire is constant, at
least in the idealization we're using.  That's because our wires
have zero electrical resistance.  They're like resistors with 
resistance 0, and we've seen that the voltage difference across
a resistor is the current times the resistance.  

Our second kind of 3-port is called a "series junction".  It's a
different sort of black box, which you can use to connect two 1-ports
in series.  You just attach them like this:


\begin{verbatim}

           _____________________
          /   ___________       \
         /   /           \       \
    |   |   |             |       |
  ooooooooooooo           |       |
 o             o          |       |
 o             o          |       |
 o             o         ---     ---
 o             o        |   |   |   |
 o             o         ---     ---
 o             o          |       |
 o             o          |       |
  ooooooooooooo           |       |
    |   |   |             |       |
         \   \___________/       /  
          \_____________________/    
\end{verbatim}
    

What's in this 3-port?  Just wires, but now arranged a different way:



\begin{verbatim}

    |   |   |
  oo|ooo|ooo|oo
 o  |   |   |  o
 o  |   |   |  o
 o   \   \ /   o
 o    \---\    o
 o   / \   \   o
 o  |   |   |  o
 o  |   |   |  o
  oo|ooo|ooo|oo
    |   |   |
\end{verbatim}
    

See how it works?


\begin{verbatim}

           _____________________
          /   ___________       \
         /   /           \       \
    |   |   |             |       |
  oo|ooo|ooo|oo           |       |
 o  |   |   |  o          |       |
 o  |   |   |  o          |       |
 o   \   \  /  o         ---     ---
 o    \---\    o        |   |   |   |
 o   / \   \   o         ---     ---
 o  |   |   |  o          |       |
 o  |   |   |  o          |       |
  oo|ooo|ooo|oo           |       |
    |   |   |             |       |
         \   \___________/       /  
          \_____________________/    
\end{verbatim}
    

The series junction is also called a "1-junction" or "effort
junction".  This makes sense, since the equations defining this 3-port
are exactly like the equations for the previous one, but with effort
and flow switched!

  p'_{1} + p'_{2} + p'_{3} = 0<br/>
  q'_{1} = q'_{2} = q'_{3}

I'll let you figure out why these are true.  

By the way: this "duality" between the series junction and
parallel junction - the way they're the same, but with the roles of
effort and flow switched - is actually the tip of a big iceberg!
There's a duality between effort and flow.  This duality is related to
Fourier duality, since in quantum physics the Fourier transform
interchanges momentum and position - the quantities whose time
derivatives are the effort and flow variables in classical mechanics.
But this duality is also related to Poincar&eacute; duality.  For any
circuit whose underlying graph is planar, there's a
"Poincar&eacute; dual" circuit where we replace edges by
vertices, vertices by edges - and also switch efforts and flows!

I hope to say more about this duality when I reach the more cosmic,
grandiose aspects of the long story I'm telling.  But if I forget,
you'll have to read this:

13) Istvan Vago, Graph Theory: Application to the Calculation of
Electrical Networks, Elsevier, 1985.  

See the section called "The Principal of Duality", on page 77.
Also, look on the web for stuff about the "\Delta -Y transformation",
which is a special case.

If you want to learn more about the 1-ports, 2-ports and 3-ports I've
been discussing, let me again recommend this book:

14) Dean C. Karnopp, Donald L. Margolis and Ronald C. Rosenberg,
System Dynamics: a Unified Approach, Wiley, New York, 1990.

It's good on the abstract concepts, it clearly lays out most of the
basic analogies, and it's not very long.  It seems to be a modernized
version of this earlier book, which has its own homegrown charm:

15) Dean C. Karnopp and Ronald C. Rosenberg, Analysis and Simulation of
Multiport Systems, MIT Press, Cambridge, Massachusetts, 1968.

For something vastly more detailed, try:

16) Forbes T. Brown, Engineering System Dynamics: a Unified 
Graph-Centered Approach, Taylor and Francis, 2007.

This mammoth tome is 1058 pages long, mainly because it's packed with
examples.  So, some of the big ideas are a bit hard to spot.  But it
proves these ideas are useful in many different fields!

\par\noindent\rule{\textwidth}{0.4pt}
\textbf{Addenda:} 
I thank Tim van Beek for correcting my German spelling.
David Roberts says it's questionable whether Bach really composed a
six-part fugue on the spot in Frederick's court: contemporary reports
say so, but it may be an exaggeration.  Theo pointed out that
a M&ouml;bius strip is not really perfectly suited to a crab canon:

\begin{quote}
   M&ouml;bius strips are cool, and the Crab Canon is cool, but they're
   essentially different.  Notice that in the video, the two players
   are still going around the M&ouml;bius strip in opposite directions, and
   each is keeping to its own side of the strip.  Moreover, in spite
   of visually putting in a twist, the "backwards" player is
   really playing the sound in a mirror, not upside-down.  There's a
   reason Bach calls it "crab": it can be played forward and
   backward.

   Thus, the correct visualization is not a M&ouml;bius strip at all, but 
   the orbifold with boundary formed by reflecting the rectangle in 
   half.  Making this is easy: take a piece of paper with the music 
   written on one side, and fold it so that the music is on the outside. 
   In this way, each side of the orbifold has half the music on it.  
   Now start at the non-mirror end, but play both sides, reflecting 
   through the orbifold boundary and continuing until you're back where 
   you started.
\end{quote}

Someone with the monicker Mixo Lydian sent me an email answering my question 
about why Japan has currents of two different frequencies - 50 and 
60 cycles per second.  As expected, there's some history involved:

\begin{quote}
The 50Hz/60Hz divide in Japan is due to historic reasons. Towards the 
end of the Meiji era, Japan made the switch from DC to AC. Tokyo Dento 
(Japan's first electric power company) adopted 50Hz German AEG generators 
while its rival Osaka Dento decided to adopt 60Hz American GE generators 
to power their respective electric grids.

Neighboring regions built their electric infrastructure adopting either 
Tokyo or Osaka standards which has led to a east-west / Tokyo-Osaka 
divide which continues to the present day, the exact border being the 
Fuji river which runs thru Shizuoka prefecture: east of the river the 
frequency is 50Hz, west of river the frequency is 60Hz.

This has hilarious consequences for the town of Shibakawa-cho, Shizuoka. 
The Fuji river runs directly thru Shibakawa-cho: some parts of town use 
50Hz while others use 60Hz! All you have to do is cross a bridge to 
alternate between (intentional pun)!

Hope this has been helpful.
\end{quote}

For more discussion, visit the <a href = "http://golem.ph.utexas.edu/category/2010/01/this_weeks_finds_in_mathematic_52.html">\emph{n}-Category Caf&eacute;</a>.


\par\noindent\rule{\textwidth}{0.4pt}
<em>Mathematics is not a careful march down a well-cleared highway, but a
journey into a strange wilderness, where the explorers often get lost.
Rigour should be a signal to the historian that the maps have been
made, and the real explorers have gone elsewhere.</em> - W. S. Anglin

\par\noindent\rule{\textwidth}{0.4pt}

% </A>
% </A>
% </A>
