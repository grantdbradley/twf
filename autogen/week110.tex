
% </A>
% </A>
% </A>
\week{October 4, 1997 }

Last time I sketched Wheeler's vision of "spacetime foam", and his
intuition that a good theory of this would require taking spin-1/2
particles very seriously.  Now I want to talk about Penrose's "spin
networks".  These were an attempt to build a purely combinatorial
description of spacetime starting from the mathematics of spin-1/2
particles.  He didn't get too far with this, which is why he moved on to
invent twistor theory.  The problem was that spin networks gave an
interesting theory of \emph{space}, but not of spacetime.  But recent work on
quantum gravity shows that you can get pretty far with spin network
technology.  For example, you can compute the entropy of quantum black
holes.  So spin networks are quite a flourishing business.

Okay.  Building space from spin!  How does it work?

Penrose's original spin networks were purely combinatorial gadgets:
graphs with edges labelled by numbers j = 0, 1/2, 1, 3/2,...  These
numbers stand for total angular momentum or "spin".  He required that
three edges meet at each vertex, with the corresponding spins j_{1}, j_{2}, j_{3}
adding up to an integer and satisfying the triangle inequalities

                  |j_{1} - j_{2}| \le  j_{3} \le  j_{1} + j_{2}

These rules are motivated by the quantum mechanics of angular momentum:
if we combine a system with spin j_{1} and a system with spin j_{2}, the spin
j_{3} of the combined system satisfies exactly these constraints.  

In Penrose's setup, a spin network represents a quantum state of the
geometry of space.  To justify this interpretation he did a lot of
computations using a special rule for computing a number from any spin
network, which is now called the "Penrose evaluation" or "chromatic
evaluation".  In "<A HREF = "week22.html">week22</A>" I said how this works when all the edges have
spin 1, and described how this case is related to the four-color
theorem.  The general case isn't much harder, but it's a real pain to
describe without lots of pictures, so I'll just refer you to the
original papers:

1) Roger Penrose, Angular momentum: an approach to
combinatorial space-time, in Quantum Theory and Beyond,
ed. T. Bastin, Cambridge U. Press, Cambridge, 1971, pp. 151-180.
Also available at <a href = "http://math.ucr.edu/home/baez/penrose/">http://math.ucr.edu/home/baez/penrose/</a>

Roger Penrose, Applications of negative dimensional tensors, in 
Combinatorial Mathematics and its Applications, ed. D. Welsh,
Academic Press, New York, 1971, pp. 221-244.

Roger Penrose, On the nature of quantum geometry, in Magic Without
Magic, ed. J. Klauder, Freeman, San Francisco, 1972, pp. 333-354.

R. Penrose, Combinatorial quantum theory and quantized directions, in
Advances in Twistor Theory, eds. L. Hughston and R. Ward,
Pitman Advanced Publishing Program, San Francisco, 1979, pp. 301-317.



It's easier to explain the \emph{physical meaning} of the Penrose evaluation.
Basically, the idea is this.  In classical general relativity, space is
described by a 3-dimensional manifold with a Riemannian metric: a recipe
for measuring distances and angles.  In the spin network approach to
quantum gravity, the geometry of space is instead described as a
superposition of "spin network states".  In other words, spin networks
form a basis of the Hilbert space of states of quantum gravity, so we
can write any state \Psi  as

                         \Psi  = \sum c_{i} \psi _{i}

where \psi _{i} ranges over all spin networks and the
coefficients c_{i} are complex numbers.  The simplest state is
the one corresponding to good old flat Euclidean space.  In this state,
each coefficient c_{i} is just the Penrose evaluation of the 
corresponding spin network \psi _{i}.

Actually, this interpretation wasn't fully understood until later, when
Rovelli and Smolin showed how spin networks arise naturally in the
so-called "loop representation" of quantum gravity.  They also came up
with a clearer picture of the way a spin network state corresponds to a
possible geometry of space.  The basic picture is that spin network
edges represent flux tubes of area: an edge labelled with spin j
contributes an area proportional to (j(j+1))^{\frac{1}{2} } to any surface it
pierces.  

The cool thing is that Rovelli and Smolin didn't postulate this, they
\emph{derived} it.  Remember, in quantum theory, observables are given by
operators on the Hilbert space of states of the physical system in
question.  You typically get these by "quantizing" the formulas for the
corresponding classical observables.  So Rovelli and Smolin took the
usual formula for the area of a surface in a 3-dimensional manifold with
a Riemannian metric and quantized it.  Applying this operator to a spin
network state, they found the picture I just described: the area of a
surface is a sum of terms proportional to (j(j+1))^{\frac{1}{2} }, 
one for each
spin network edge poking through it.

Of course, I'm oversimplifying both the physics and the history here.
The tale of spin networks and loop quantum gravity is rather long.  I've
discussed it already in "<A HREF = "week55.html">week55</A>" and "<A HREF = "week99.html">week99</A>", but only sketchily.  If
you want more details, try:

2) Carlo Rovelli, Loop quantum gravity, preprint available as
<A HREF = "http://xxx.lanl.gov/abs/gr-qc/9710008">gr-qc/9710008</A>,
also available as a webpage on Living Reviews in Relativity at
<A HREF = "http://www.livingreviews.org/Articles/Volume1/1998-1rovelli/">
http://www.livingreviews.org/Articles/Volume1/1998-1rovelli/</A>
 
The abstract gives a taste of what it's all about:

\begin{quote}The problem of finding the quantum theory of the gravitational 
field,
and thus understanding what is quantum spacetime, is still open.  One of
the most active of the current approaches is loop quantum gravity.  Loop
quantum gravity is a mathematically well-defined, non-perturbative and
background independent quantization of general relativity, with its
conventional matter couplings.  The research in loop quantum gravity
forms today a vast area, ranging from mathematical foundations to
physical applications.  Among the most significant results obtained are:
(i) The computation of the physical spectra of geometrical quantities
such as area and volume; which yields quantitative predictions on
Planck-scale physics. 
(ii) A derivation of the Bekenstein-Hawking black
hole entropy formula. 
(iii) An intriguing physical picture of the
microstructure of quantum physical space, characterized by a
polymer-like Planck scale discreteness.  This discreteness emerges
naturally from the quantum theory and provides a mathematically
well-defined realization of Wheeler's intuition of a spacetime "foam".
Longstanding open problems within the approach (lack of a scalar
product, overcompleteness of the loop basis, implementation of reality
conditions) have been fully solved. The weak part of the approach is the
treatment of the dynamics: at present there exist several proposals,
which are intensely debated.  Here, I provide a general overview of
ideas, techniques, results and open problems of this candidate theory of
quantum gravity, and a guide to the relevant literature.\end{quote}

For a nice picture of Rovelli standing in front of some spin networks,
check out:

3) Carlo Rovelli's homepage, <A HREF =
"http://www.phyast.pitt.edu/~rovelli/
">http://www.phyast.pitt.edu/~rovelli/</A> 

which also has links to many of his papers.

You'll note from this abstract that the biggest problem in loop quantum
gravity is finding an adequate description of \emph{dynamics}.  This is
partially because spin networks are better suited for describing space
than spacetime.  For this reason, Rovelli, Reisenberger and I have been
trying to describe spacetime using "spin foams" - sort of like soap
suds with all the bubbles having faces labelled by spins.  Every slice
of a spin foam is a spin network.

But I'm getting ahead of myself!  I should note that the spin networks
appearing in the loop representation are different from those Penrose
considered, in two important ways.

First, they can have more than 3 edges meeting at a vertex, and the
vertices must be labelled by "intertwining operators", or "intertwiners"
for short.  This is a concept coming from group representation theory;
as described in "<A HREF = "week109.html">week109</A>", what we've been calling "spins" are really
irreducible representations of SU(2).  If we orient the edges of a spin
network, we should label each vertex with an intertwiner from the tensor
product of representations on the "incoming" edges to the 
tensor product
of representations labelling the "outgoing" edges.  When 3 edges
labelled by spins j_{1}, j_{2}, j_{3} meet at a vertex, there is at most one
intertwiner

                        f: j_{1} \otimes  j_{2} \to  j_{3}, 

at least up to a scalar multiple.  The conditions I wrote down - the
triangle inequality and so on - are just the conditions for a nonzero
intertwiner of this sort to exist.  That's why Penrose didn't label his
vertices with intertwiners: he considered the case where there's
essentially just one way to do it!  When more edges meet at a vertex,
there are more intertwiners, and this extra information is physically
very important.  One sees this when one works out the "volume operators"
in quantum gravity.  Just as the spins on edges contribute \emph{area} to
surfaces they pierce, the intertwiners at vertices contribute \emph{volume}
to regions containing them!

Second, in loop quantum gravity the spin networks are \emph{embedded} in some
3-dimensional manifold representing space.  Penrose was being very
radical and considering "abstract" spin networks as a purely
combinatorial replacement for space, but in loop quantum gravity, one
traditionally starts with general relativity on some fixed spacetime and
quantizes that.  Penrose's more radical approach may ultimately be the
right one in this respect.  The approach where we take classical physics
and quantize it is very important, because we understand classical
physics better, and we have to start somewhere.  Ultimately, however,
the world is quantum-mechanical, so it would be nice to have an approach
to space based purely on quantum-mechanical concepts.  Also, treating
spin networks as fundamental seems like a better way to understand the
"quantum fluctuations in topology" which I mentioned in "<A HREF = "week109.html">week109</A>".
However, right now it's probably best to hedge ones bets and work hard
on both approaches.

Lately I've been very excited by a third, hybrid approach:

4) Andrea Barbieri, Quantum tetrahedra and simplicial spin networks, 
preprint available as <A HREF = "http://xxx.lanl.gov/abs/gr-qc/9707010">gr-qc/9707010</A>.  

Barbieri considers "simplicial spin networks": spin networks living in a
fixed 3-dimensional manifold chopped up into tetrahedra.  He only
considers spin networks dual to the triangulation, that is, spin
networks having one vertex in the middle of each tetrahedron and one
edge intersecting each triangular face.

In such a spin network there are 4 edges meeting at each vertex,
and the vertex is labelled with an intertwiner of the form

               f: j_{1} \otimes  j_{2} \to  j_{3} \otimes  j_{4}

where j_{1},...,j_{4} are the spins on these edges.  If you know about the
representation theory of SU(2), you know that j_{1} \otimes  j_{2} is a direct
sum of representations of spin j_{5}, where j_{5} goes from |j_{1} - j_{2}| up to 
j_{1} + j_{2} in integer steps.  So we get a basis of intertwining operators:

               f: j_{1} \otimes  j_{2} \to  j_{3} \otimes  j_{4}

by picking one factoring through each representation j_{5}:

             j_{1} \otimes  j_{2} \to  j_{5} \to  j_{3} \otimes  j_{4}

where:

a) j_{1} + j_{2} + j_{5} is an integer and  |j_{1} - j_{2}| \le  j_{5} \le  j_{1} + j_{2}

b) j_{3} + j_{4} + j_{5} is an integer and  |j_{3} - j_{4}| \le  j_{5} \le  j_{3} + j_{4}.

Using this, we get a basis of simplicial spin networks by labelling all
the edges \emph{and vertices} by spins satisfying the above conditions.
Dually, this amounts to labelling each tetrahedron and each triangle
in our manifold with a spin!  Let's think of it this way.

Now focus on a particular simplicial spin network and a particular
tetrahedron.  What do the spins j_{1},...,j_{5} say about the 
geometry of the
tetrahedron?  By what I said earlier, the spins j_{1},...,j_{4} describe the
areas of the triangular faces: face number 1 has area proportional to
(j_{1}(j_{1}+1))^{\frac{1}{2} }, and so on.  
What about j_{5}?  It also describes an area.
Take the tetrahedron and hold it so that faces 1 and 2 are in front,
while faces 3 and 4 are in back.  Viewed this way, the outline of the
tetrahedron is a figure with four edges.  The midpoints of these four
edges are the corners of a parallelogram, and the area of this
parallelogram is proportional to 
(j_{5}(j_{5}+1))^{\frac{1}{2} }. 
In other words, there
is an area operator corresponding to this parallelogram, and our spin
network state is an eigenvector with eigenvalue proportional to
(j_{5}(j_{5}+1))^{\frac{1}{2} }.  
Finally, there is also a \emph{volume operator}
corresponding to the tetrahedron, whose action on our spin network state
is given by a more complicated formula involving the spins j_{1},...,j_{5}.

Well, that either made sense or it didn't... and I don't think either of
us want to stick around to find out which!  What's the bottom line, you
ask?  First, we're seeing how an ordinary tetrahedron is the classical
limit of a "quantum tetrahedron" whose faces have quantized areas and
whose volume is also quantized.  Second, we're seeing how to put
together a bunch of these quantum tetrahedra to form a 3-dimensional
manifold equipped with a "quantum geometry" - 
which can dually be seen
as a spin network.  Third, all this stuff fits together in a truly
elegant way, which suggests there is something good about it.  The
relationship between spin networks and tetrahedra connects the theory of
spin networks with approaches to quantum gravity where one chops up
space into tetrahedra - like the "Regge calculus" and "dynamical
triangulations" approaches.

Next week I'll say a bit about using spin networks to study quantum
black holes.  Later I'll talk about \emph{dynamics} and spin foams.

Meanwhile, I've been really lagging behind in describing new papers as
they show up... so here are a few interesting ones:

5) C. Nash, Topology and physics - a historical essay, to appear in 
A History of Topology, edited by Ioan James, Elsevier-North Holland, 
preprint available as <A HREF = "http://xxx.lanl.gov/abs/hep-th/9709135">hep-th/9709135</A>.  

6) Luis Alvarez-Gaume and Frederic Zamora, Duality in quantum field
theory (and string theory), available as <A HREF = "http://xxx.lanl.gov/abs/hep-th/9709180">hep-th/9709180</A>.

Quoting the abstract: 

"These lectures give an introduction to duality in Quantum Field Theory. We
discuss the phases of gauge theories and the implications of the
electric-magnetic duality transformation to describe the mechanism of
confinement. We review the exact results of N=1 supersymmetric QCD and the
Seiberg-Witten solution of N=2 super Yang-Mills. Some of its extensions to
String Theory are also briefly discussed."

7) Richard E. Borcherds, What is a vertex algebra?, available as
<A HREF = "http://xxx.lanl.gov/abs/q-alg/9709033">q-alg/9709033</A>.

"These are the notes of an informal talk in Bonn describing how to
define an analogue of vertex algebras in higher dimensions."

8) J. M. F. Labastida and Carlos Lozano, Lectures in topological quantum
field theory, 62 pages in LaTeX with 5 figures in encapsulated
Postscript, available as <A HREF = "http://xxx.lanl.gov/abs/hep-th/9709192">hep-th/9709192</A>.

"In these lectures we present a general introduction to topological
quantum field theories. These theories are discussed in the framework of
the Mathai-Quillen formalism and in the context of twisted N=2
supersymmetric theories. We discuss in detail the recent developments in
Donaldson-Witten theory obtained from the application of results based
on duality for N=2 supersymmetric Yang-Mills theories. This involves a
description of the computation of Donaldson invariants in terms of
Seiberg-Witten invariants.  Generalizations of Donaldson-Witten theory
are reviewed, and the structure of the vacuum expectation values of
their observables is analyzed in the context of duality for the simplest
case."

9) Martin Markl, Simplex, associahedron, and cyclohedron, preprint
available as <A HREF = "http://xxx.lanl.gov/abs/alg-geom/9707009">
alg-geom/9707009</A>.

"The aim of the paper is to give an `elementary' introduction to the
theory of modules over operads and discuss three prominent examples of
these objects - simplex, associahedron (= the Stasheff polyhedron) and
cyclohedron (= the compactification of the space of configurations of
points on the circle)."



\par\noindent\rule{\textwidth}{0.4pt}
% </A>
% </A>
% </A>
