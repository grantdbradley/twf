
% </A>
% </A>
% </A>
\week{March 12, 1995}

Your roving mathematical physics reporter is now in Milan,
where (when not busy eating various kinds of cheese) he
is discussing BF theory with his hosts, Paolo Cotta-Ramusino
and Maurizio Martellini.  This is rather a long way to go to
stumble on the October 1994 issue of the Journal of Mathematical
Physics, but such is life.  This issue is a special issue on 
"Topology and Physics", a nexus dear to my heart, so let me
say a bit about some of the papers in it.

1) Supersymmetric Yang-Mills theory on a four-manifold, by
Edward Witten, Jour. Math. Phys. 35 (1994), 5101-5135.

This paper concerns the relation between supersymmetric Yang-Mills
theory and Donaldson theory, discovered by Seiberg and Witten,
which not too long ago hit the front page of various newspapers.
(See "<A HREF = "week44.html">week44</A>' and "<A HREF = "week45.html">week45</A>" for my own yellow journalism on the
subject.)  I don't have anything new to say about this stuff, of 
which I am quite ignorant.   If you are an expert on N = 2 supersymmetric
Yang-Mills theory, hyper-Kaehler manifolds, cosmic strings and the
renormalization group, the paper should be a piece of cake.  Seriously,
he does seem to be making a serious effort to communicate the
ideas in simple terms to us mere mortals, so it's worth looking at.

2) Louis Crane and Igor Frenkel,
Four-dimensional topological quantum field theory, Hopf
categories, and the canonical bases, Jour. Math. Phys. 35 (1994), 5136-5154.
Also available as
<A HREF = "http://xxx.lanl.gov/abs/hep-th/9405183">hep-th/9405183</A>.   



I discussed this paper a wee bit in "<A HREF = "week38.html">week38</A>".  
Now you can actually see the pictures.   As we begin to understand 
n-categories (see
"<A HREF = "week49.html">week49</A>") our concept of symmetry gets deeper and deeper.  This
isn't surprising.  When all we knew about was 0-categories - that
is, sets! - our concept of symmetry revolved around the notion of
a "group".  This is a set G where you can multiply elements in an
associative way, with an identity element 1 such that 

1g = g1 = g

for all elements g of G, and where every element g has an inverse
g^{-1} with 

gg^{-1} = g^{-1}g = 1.  

For example, the group of rotations in n-dimensional Euclidean space.
When we started understanding 1-categories - that is, categories! -
the real idea behind groups and symmetry could be more clearly
expressed.  Sets have elements, and they are either equal or not - no
two ways about it.  Categories have "objects", and even though two
objects aren't equal, they can still be "isomorphic".  An object can
be isomorphic to itself in lots of different wasy: these are its
symmetries, and the symmetries form a group.  But this is really just
the tip of a still mysterious iceberg.  

For example, in a 2-category, even though two objects aren't equal, or
even isomorphic, they can be "equivalent", or maybe I should say
"2-equivalent".  This is a still more general notion of "sameness".  I
won't try to define it just now, but I'll just note that it arises
from the fact that in a 2-category one can ask whether two morphisms
are isomorphic!  (For people who followed "<A HREF =
"week49.html">week49</A>" and know some category theory, let me note
that the standard notion of equivalence of categories is a good
example of this "equivalence" business.)

As we climb up the n-categorical ladder, this keeps going.  We get
ever more subtle refinements on the notion of "sameness", hence ever
subtler notions of symmetry.  It's all rather mind-boggling at first,
but not really very hard once you get the hang of it, and since
there's lots of evidence that n-dimensional topological quantum field
theories are related to n-categories, I think these subtler notions of
symmetry are going to be quite interesting for physics.

And now, to wax technical for a bit (skip this paragraph if the last
one made you dizzy), it's beginning to seem that the symmetry groups
physicists know and love have glorious reincarnations, or avatars if
one prefers, at these higher n-categorical levels.  Take your favorite
group - mine is SU(2), which describes 3d rotational symmetry hence
angular momentum in quantum mechanics.  It's category of
representations isn't just any old category, its a symmetric monoidal
category!  See the chart in "<A HREF = "week49.html">week49</A>" if
you forget what this is.  Now, there are more general things than
groups whose categories of representations are symmetric monoidal
categories - for example, cocommutative Hopf algebras.  And there are
other kinds of Hopf algebras - "quasitriangular" ones, often known
as "quantum groups" - whose categories of representations are
\emph{braided} monoidal categories.  The cool thing is that SU(2) has an
avatar called "quantum SU(2)" which is one of these quantum groups.
Again, eyeball the chart in "<A HREF = "week49.html">week49</A>".
Symmetric monoidal categories are a special kind of 4-categories,
which is why they show up so much in 4d physics, while braided
monoidal categories are a special kind of 3-categories, which is why
they (and quantum groups) show up in 3d physics.  For example, quantum
SU(2) shows up in the study of 3d quantum gravity (see "<A HREF =
"week16.html">week16</A>").  Now the even cooler thing is that while a
quaistriangular Hopf algebra is a set with a bunch of operations,
there is a souped-up gadget, a "quasitriangular Hopf category", which
is a \emph{category} with an analogous bunch of operations, and these have a
\emph{category} of representations, but not just any old 2-category, but in
fact a \emph{braided monoidal 2-category}.  If you again eyeball the chart,
you'll see this is a special kind of 4-category, so it should be
related to 4d topology and - this is the big hope - 4d physics.
Now the \emph{really} cool thing, which is what Crane and Frenkel show
here, is that SU(2) has yet another avatar which is one of these
quasitriangular Hopf categories.

Regardless of whether it has anything to do with physics, this business
about how symmetry groups have avatars living on all sorts of rungs
of the n-categorical ladder is such beautiful math that I'm sure
it's trying to tell us something.  Right now I'm trying to figure
out just what.

3) Raoul Bott and Clifford Taubes, On the self-linking of knots, 
Jour. Math. Phys. 35 (1994), 5247-5287.

I'd need to look at this a few more times before I could say
anything intelligent about it, but it looks to be a very exciting
way of understanding what the heck is really going on as far
as Vassiliev invariants, Feynman diagrams in Chern-Simons theory,
and so on are concerned - especially if you wanted to generalize
it all to higher dimensions.

Alas, I'm getting worn out, so let me simply \emph{list} a few more
papers, which are every bit as fun as the previous ones... no
disrespect intended... I just have to call it quits soon.  As
always, it should be clear that what I write about and what I 
don't is purely a matter of whim, caprice, chance, and my own
ignorance.  

4) Christopher King and Ambar Sengupta, 
An explicit description of the symplectic struture of moduli
spaces of flat connections, Jour.  Math. Phys. 35 (1994), 5338-5353.

Christopher King and Ambar Sengupta, 
The semiclassical limit of the two-dimensional quantum Yang-Mills
model, Jour. Math. Phys. 35 (1994), 5354-5363.

5) D. J. Thouless, 
Topological interpretations of quantum Hall conductance, 
Jour. Math. Phys. 35 (1994), 5362-5372.

6) J. Bellisard, A. van Elst, and H. Schulz-Baldes, 
The noncommutative geometry of the quantum Hall effect, 
Jour. Math. Phys. 35 (1994),
5373-5451.

7) Steve Carlip and R. Cosgrove, 
Topology change in (2+1)-dimensional gravity, 
Jour. Math. Phys. 35 (1994), 5477-5493.

\par\noindent\rule{\textwidth}{0.4pt}
% </A>
% </A>
% </A>
