
% </A>
% </A>
% </A>
\week{May 25, 2008 }



Today I'd like to talk about the Pythagorean pentagram, Bill Schmitt's
work on Hopf algebras 
in combinatorics, the magnum opus of Aguiar and Mahajan, and 
quaternionic analysis.  But first, the astronomy picture of the week.

I seem to be into moons these days: first Saturn's moon Titan in
"<a href = "week263.html">week263</A>", and then Mars' moon
Phobos in "<a href = "week264.html">week264</A>".  On the
cosmic scale, our Solar System is like our back yard.  It may not be
important in the grand scheme of things, but we should get to know it
and learn to take care of it.  It's got lots of cool moons.  So this
week, let's talk about <a href =
"http://en.wikipedia.org/wiki/Europa_%28moon%29">Europa</a>:

<div align = "center">
<a href = "europa.jpg">
<img width = "500" src = "europa.jpg">
% </a>
</div>

1) Astronomy Picture of the Day, Gibbous Europa, 
<a href = "http://antwrp.gsfc.nasa.gov/apod/ap071202.html">http://antwrp.gsfc.nasa.gov/apod/ap071202.html</a>

Europa is the fourth biggest moon of Jupiter, the smallest of the four
seen by Galileo.  It's 3000 kilometers in diameter, slightly smaller 
than our moon, and it zips around Jupiter once every 3.5 of our days, 
though it's almost twice as far from Jupiter as our moon is from us.

It looks like a cracked ball of ice, and that's what it is - at
least near the surface.  

Indeed, this ancient impact crater looks like a smashed 
windshield, or a frozen lake that's been hit with a sledgehammer:  

<div align = "center">
<a href = "europa_impact_crater.jpg">
<img width = "500" src = "europa_impact_crater.jpg">
% </a>
</div>

2) NASA Photojournal, Ancient impact basin on Europa, 
<a href = "http://photojournal.jpl.nasa.gov/catalog/PIA00702">http://photojournal.jpl.nasa.gov/catalog/PIA00702</a>

But this crater, called Tyre, is huge: about as big as the island
of Hawaii, 145 kilometers across!  (Beware: this picture is a composite of 
three photos taken by the Galileo spacecraft in 1997.  It's in false 
color designed to show off various structures: the original crater, 
the later red cracks, and the blue-green ridges.)
 
The big question is whether there's liquid water beneath the icy
surface... and if so, maybe life?  One model of this moon posits a 
solid ice crust.  Another says there's liquid water too:

<div align = "center">
<a href = "http://photojournal.jpl.nasa.gov/catalog/PIA01669">
<img border = "2" src = "europa_models.gif">
% </a>
</div>

3) NASA Photojournal, Model of Europa's subsurface structure,
<a href = "http://photojournal.jpl.nasa.gov/catalog/PIA01669">http://photojournal.jpl.nasa.gov/catalog/PIA01669</a>

How can we tell?  Europa is the \emph{smoothest} of all solid planets
and moons, with lots of cracks and ridges but few remaining craters.
This suggests either an ocean beneath the surface, or at least ice
warm enough to keep convection going.  The region called Conamara
Chaos looks like pack ice here on Earth, hinting at liquid water
beneath:

<div align = "center">
<a href = "http://photojournal.jpl.nasa.gov/jpeg/PIA01127.jpg">
<img width = "500" src = "europa_chaos.jpg">
% </a>
</div>

4) NASA Photojournal, Europa: ice rafting view,
<a href = "http://photojournal.jpl.nasa.gov/catalog/PIA01127">http://photojournal.jpl.nasa.gov/catalog/PIA01127</a>

The bluish white areas have been blanketed with ice dust ejected from
far away when an impact formed a crater called <a href = "http://en.wikipedia.org/wiki/Pwyll_%28crater%29">Pwyll</a>.  The reddish brown 
regions could contain salts or sulfuric acid - it's hard to find out
using spectroscopy, since there's too much ice.

Another very nice piece of evidence for \emph{salty} liquid water inside 
Europa is that the magnetic field of Jupiter induces electric currents 
in  this moon, which in turn create their own magnetic fields!  These 
fields were detected when the Galileo probe swooped closest to Europa
back in 2000:

5) M. G. Kivelson, K. K. Khurana, C. T. Russell, M. Volwerk, R.J. Walker, 
and C. Zimmer, Galileo magnetometer measurements: a stronger case for a 
subsurface ocean at Europa, Science, 289 (2000), 1340-1343.

At the time, Margaret Kivelson, head of the magnetometer project, said: 

\begin{quote}
  I think these findings tell us that there is indeed a layer of liquid 
  water beneath Europa's surface.  I'm cautious by nature, but this new 
  evidence certainly makes the argument for the presence of an ocean far 
  more persuasive.  Jupiter's magnetic field at Europa's position changes 
  direction every 5-1/2 hours.  This changing magnetic field can drive 
  electrical currents in a conductor, such as an ocean. Those currents 
  produce a field similar to Earth's magnetic field, but with its magnetic 
  north pole - the location toward which a compass on Europa would point - 
  near Europa's equator and constantly moving. In fact, it is actually 
  reversing direction entirely every 5-1/2 hours.
\end{quote}

A couple weeks ago, another nice piece of evidence was announced:

<div align = "center">
<a href = "http://www.lpi.usra.edu/science/schenk/europaCropCircles/">
<img src = "europa_polar_wandering_schenk.jpg"></a><br/>
<font size = "-1"> 
Scale bar is 100 kilometers long.
<br/>
Picture by Paul Schenk, Lunar and Planetary Institute.
</font>
</div>

6) Paul Schenk, Isamu Matsuyama and Francis Nimmo, True polar wander 
on Europa from global-scale small-circle depressions, Nature 453 (2008),
368-371.

Paul Schenk, Scars from Europa's polar wandering betray ocean beneath, 
<a href = "http://www.lpi.usra.edu/science/schenk/europaCropCircles">
http://www.lpi.usra.edu/science/schenk/europaCropCircles/</a>

There are two arc-shaped depressions exactly opposite each other on
Europa, each hundreds of kilometers long and between .3 and 1.5 kilometers
deep.  According to the above paper, these scars have just the right shape 
to be caused the moon's icy shell rotating a quarter turn relative to
the interior!  The authors believe this could happen most easily if 
it were floating on an ocean.

If Europa has an ocean under its ice, other questions immediately arise.  
How thick is the ice and how deep is the ocean?  Some guess 15-30 kilometers
of ice atop 100 kilometers of liquid.  What keeps it warm?  Heating
produced by tidal forces may be the best bet - radioactivity from the core 
contributes just about 100 billion watts, not nearly enough:

7) M. N. Ross and G. Schubert, Tidal heating in an internal ocean model
of Europa, Nature 325 (1987), 133-144.

And then for the really big question: could there be \emph{life} on Europa?
Antarctica has an enormous lake called <a href = "http://en.wikipedia.org/wiki/Lake_Vostok">Lake Vostok</a> buried under 4 kilometers 
of ice, and when people drilled into it they found all sorts of bizarre 
life forms that had never been seen before.  So, especially if Europa had
been warmer once, it's conceivable that life might have formed there and 
survives to this day.  Of course, the surface of Europa makes Antarctica 
look downright balmy: it's -160 Celsius at the equator.  And liquid water 
below could be mixed with sulfuric acid, or lots of nasty salts...

Nonetheless, some dream of sending a satellite to Europa, perhaps 
to impact it at high velocity and see what's inside, or perhaps to land
and melt down through the ice:

8) Leslie Mullen, Hitting Europa hard (interview of Karl Hibbits), 
Astrobiology Magazine, May 1, 2006, 
<a href = "http://www.astrobio.net/news/article1944.html">http://www.astrobio.net/news/article1944.html</a>

But these dreams may not come true anytime soon.  In 2005, NASA
cancelled its ambitious plans for the Jupiter Icy Moons Orbiter:

10) Wikipedia, Jupiter Icy Moons Orbiter, 
<a href = "http://en.wikipedia.org/wiki/Jupiter_Icy_Moons_Orbiter">http://en.wikipedia.org/wiki/Jupiter_Icy_Moons_Orbiter</a>

The U.S. Congress, the National Academy of Sciences, and the 
NASA Advisory Committee have all supported a mission to Europa, 
but NASA has still not funded this project:

11) Leonard David, Europa mission: lost in NASA budget, SPACE.com,
February 7, 2006, <a href = "http://www.space.com/news/060207_europa_budget.html">http://www.space.com/news/060207_europa_budget.html</a>

Unfortunately, NASA still spends most of its money on expensive
manned missions - the Buck Rogers approach to space.  
They think the public wants the "glamor" of manned missions.
So, while they just safely landed the Phoenix spacecraft on 
Mars, they're also busy struggling to fix a toilet in near earth
orbit, on the International Space Station.  

To study the underground ocean of Europa, 
our best hope may lie with the European Space Agency's "Jovian Europa
Orbiter", part of a project called the Jovian Minisat Explorer:

12) ESA Science and Technology, Jovian Minisat Explorer,
<a href = "http://sci.esa.int/science-e/www/object/index.cfm?fobjectid=35982">http://sci.esa.int/science-e/www/object/index.cfm?fobjectid=35982</a>

This hasn't been funded yet, and there's no telling if it ever
will.  But people are already working to make sure Europa
doesn't get contaminated by bacteria from Earth:

13) National Research Council, Preventing the Forward Contamination 
of Europa, The National Academies Press, Washington, DC, 2000.  
Also available at <a href = "http://www.nap.edu/catalog.php?record_id=9895">http://www.nap.edu/catalog.php?record_id=9895</a>

In fact the US and many other countries are obligated to do this, 
since they signed a United Nations treaty that requires it.

The Galileo probe had not been sterilized in a way that would 
kill <a href = "http://en.wikipedia.org/wiki/Extremophile">extremophiles</a> 
- organisms that survive extreme conditions.
So, the National Research Council recommended that NASA crash 
Galileo into Jupiter when its mission was over, to avoid an 
accidental collision with Europa.  So, that's what they did!  
After 14 years of collecting data about Jupiter and its moons,
Galileo crashed into Jupiter and burned up in its atmosphere on 
September 21, 2004.

Maybe I'll talk about other moons of Jupiter next week... the
most interesting ones besides Europa are volcanic, sulfurous 
Io and icy Ganymede, biggest of all.

But now let me turn to the Pythagorean pentagram.

The Pythagoreans - that strange Greek cult of vegetarian 
mathematicians - were apparently fascinated by the pentagram. 
Why?  I don't think there's any textual evidence to help us answer
this question, but luckily there's another way to settle it: 
unsubstantiated wild guesses!

If you take a pentagram and keep on drawing lines through points 
that are already present, you can generate this picture:

<div align = "center">
<a href = "pythagorean_pentagram.jpg">
<img style = "border:none;" width = "500" src = "pythagorean_pentagram.jpg">
% </a>
</div>

14) James Dolan, Pythagorean pentagram,
<a href = "http://math.ucr.edu/home/baez/pythagorean_pentagram.jpg">http://math.ucr.edu/home/baez/pythagorean_pentagram.jpg</a>

This is just the beginning of an infinite picture packed with pentagrams.  
The sizes of these pentagrams are related by various powers of the 
golden ratio:

\Phi  = (1 + \sqrt 5)/2 = 1.6180339...

In particular, if you run up any arm of the big pentagram
you'll see little pentagrams, alternating blue and green
in the above picture, each 1/\Phi  times as big as the one before.

And if you contemplate these, you can see that:

\Phi  = 1 + 1/\Phi 

I could explain how, but I prefer to leave it as a fun little puzzle.
If you get stuck, I'll give you a clue later. 

This might have interested the Pythagoreans, since it quickly implies
that

\Phi  = 1 + 1/\Phi  
   

   
    = 1 + 1/(1 + 1/\Phi )


   
    = 1 + 1/(1 + 1/(1 + 1/\Phi ))


   
    = 1 + 1/(1 + 1/(1 + 1/(1 + 1/\Phi )))

and so on.  This means that the continued fraction expansion of
\Phi  never ends, so it must be irrational!  There's some evidence
that early Greeks were interested in continued fraction expansions...
you can read about that in this marvelous speculative book:

15) David Fowler, The Mathematics Of Plato's Academy:
A New Reconstruction, 2nd edition, Clarendon Press, Oxford, 1999.
Review by Fernando Q. Gouv&ecirc;a for MAA Online available at
<a href = "https://www.maa.org/press/maa-reviews/the-mathematics-of-platos-academy-a-new-reconstruction">https://www.maa.org/press/maa-reviews/the-mathematics-of-platos-academy-a-new-reconstruction</a>

If so, we can imagine that early Greek mathematicians discovered
the irrationality of the golden ratio by contemplating the Pythagorean
pentagram.  

I recently gave a talk about this and other fun aspects of the number
5 at George Washington University and Google.  

I was invited to Google by my student Mike Stay - more about that some
other day, perhaps.  But I'd been invited to George Washington
University by Bill Schmitt.  We went to grad school together.  While I
was studying quantum field theory with Irving Segal, he was studying
combinatorics with Gian-Carlo Rota.  Later he taught me about Joyal's
"especes de structures", also known as "species"
or "structure types".  Later still, these turned out to be
deeply related to the quantum harmonic oscillator and Feynman
diagrams!  For more on that, see "<a href =
"week185.html">week185</A>" and "<a href =
"week202.html">week202</A>".

Bill has always been interested in getting Hopf algebras from structure
types.  The idea is implicit in some work of Rota:

16) Saj-Nicole Joni and Gian-Carlo Rota, Coalgebras and bialgebras in 
combinatorics, Studies in Applied Mathematics 61 (1979), 93-139.

Gian-Carlo Rota, Hopf algebras in combinatorics, in Gian-Carlo
Rota on Combinatorics: Introductory Papers and Commentaries, ed.
J. P. S. Kung, Birkhauser, Boston, 1995.

but my favorite explanation is here:

17) William R. Schmitt, Hopf algebras of combinatorial structures,
Canadian Journal of Mathematics 45 (1993), 412-428.   Also available
at <a href = "http://home.gwu.edu/~wschmitt/papers/hacs.pdf">http://home.gwu.edu/~wschmitt/papers/hacs.pdf</a>

Let me sketch the simplest result in this paper!  For starters, recall 
that a structure type is any sort of structure you can put on finite 
sets.  In other words, it's a functor

F: FinSet_{0} \to  Set

where FinSet_{0} is the groupoid of finite sets and
bijections.  The idea is that for any finite set X, F(X) is the set
all of structures of the given type that we can put on X.  A good
example is F(X) = 2^{X}, the set of 2-colorings of X.

Starting from this, we can form a groupoid of F-structured finite sets
and structure-preserving bijections.  For example, the groupoid of
2-colored finite sets and color-preserving bijections.  The idea
should be obvious, but it's good to make it precise.  For category
hotshots it's just the groupoid of "elements" of F, called
elt(F).  But if you're not a hotshot yet, I should explain this.

An object of elt(F) is a finite set X together with an element a in 
F(X).  A morphism of elt(F), say

f: (X,a) \to  (X',a')

is a bijection 

f: X \to  X'

such that 

F(f)(a) = a'

In other words: f is a bijection that carries the F-structure on X
to the F-structure on X'.

Anyway: given a structure type F, we can form a vector space
B_{F} whose basis consists of isomorphism classes of elt(F).
And in the paper above, Bill describes various ways to make
B_{F} into various kinds of coalgebra or Hopf algebra.

I'll only explain the simplest one.  There are lots of structure types
where you can "restrict" a structure on a big set to a
structure on a smaller set.  For example, a 2-coloring of a set restricts
to a 2-coloring of any subset.  Let's call such a thing a "structure
type with restriction".

Technically, a structure type with restriction is a functor

F: Inj^{op} \to  Set

where Inj is the category of finite sets and injections.  When
we have such a thing, the inclusion 

i: X \to  X'

of a little set X in a bigger set X' gives a map

F(i): F(X') \to  F(X)

that says how to restrict F-structures on X' to F-structures
on X.

In this situation, Bill shows that the vector space B_{F} becomes
a cocommutative coalgebra.  In particular, it gets a comultiplication

\Delta : B_{F} \to  B_{F} \otimes  B_{F}

which satisfies laws just like the commutative and associative laws
for ordinary multiplication, only "backwards".  

The idea is simple: we comultiply a finite set with an F-structure on it
by chopping the set in two parts in all possible ways and using our 
ability to restrict the F-structure to each part.   I could write down 
the formula, but it's better to guess it and then check your guess in 
Bill's paper!  See his Proposition 3.1.

After Bill came up with this stuff, the connection between Hopf algebras
and combinatorics became a big business - largely due to Kreimer's work 
on Hopf algebras and Feynman diagrams.  I talked about this back in 
"<a href = "week122.html">week122</A>" - but here's a more recent review, with a hundred references 
for further study:

18) Kurusch Ebrahimi-Fard and Dirk Kreimer, Hopf algebra approach 
to Feynman diagram calculations, available as <a href = "http://arXiv.org/abs/arXiv:hep-th/0510202">arXiv:hep-th/0510202</A>.

This yields lots of applications of Bill's ideas to quantum physics.
I have no idea how this huge industry is related to my work with James 
Dolan and Jeffrey Morton on structure types, more general "stuff 
types", quantum field theory and Feynman diagrams.  But, maybe you
can figure it out if you read these:

19) John Baez and Derek Wise, Quantization and Categorification. <br/>
Fall 2003 notes: <a href = "http://math.ucr.edu/home/baez/qg-fall2003">http://math.ucr.edu/home/baez/qg-fall2003</a> <br/>
Winter 2004 notes: <a href = "http://math.ucr.edu/home/baez/qg-winter2004/">http://math.ucr.edu/home/baez/qg-winter2004/</a> <br/>
Spring 2004 notes: <a href = "http://math.ucr.edu/home/baez/qg-spring2004/">http://math.ucr.edu/home/baez/qg-spring2004/</a>

20) Jeffrey Morton, Categorified algebra and quantum mechanics, 
Theory and Applications of Categories 16 (2006), 785-854.
Available at <a href = "http://www.emis.de/journals/TAC/volumes/16/29/16-29abs.html">http://www.emis.de/journals/TAC/volumes/16/29/16-29abs.html</a>
and as <a href = "http://arxiv.org/abs/math/0601458">arXiv:math/0601458</a>.


While you're mulling over these ideas, it might pay to ponder
this paper Bill told me about:

21) Marcelo Aguiar and Swapneel Mahajan, Monoidal functors, species
and Hopf algebras, available at <a href = "http://www.math.tamu.edu/~maguiar/a.pdf">http://www.math.tamu.edu/~maguiar/a.pdf</a>

It's 588 pages long!  It's a bunch of very sophisticated combinatorics
touching on ideas dear to my heart: q-deformation, species, Fock space,
and higher categories.  I can't summarize it, but here are some 
immediately gripping portions:

<ul>
<li>
Chapter 5, "Higher monoidal categories".  Here they discuss 
"n-monoidal categories", which are categories equipped with a list 
of tensor products with lax interchange laws relating each tensor
product to all the later ones on the list:

(A \otimes _{i} B) \otimes _{j} (A' \otimes _{i} B') \to 
(A \otimes _{j} A') \otimes _{i} (B \otimes _{j} B')

for i < j.  These gadgets generalize the "iterated monoidal
categories" of Balteanu, Fiedorowicz, Schwaenzel, Vogt and also
Forcey - I gave some references on these back in "<a href =
"week209.html">week209</A>".  The big difference seems to be that
the Fiederowicz gang has all the tensor products share the same unit.
That's great for what they want to do - namely, get a kind of category
whose nerve is an n-fold loop space.  But, Aguiar and Mahajan study a
bunch of examples coming from combinatorics where different products
have different units!  It's really these examples that are interesting
to me, though the abstract concepts are cool too.  
</li>

<li>
Chapter 7, "Hopf monoids in species".  Here they use 
"species" to 
mean what I'd call "linear structure types", that is, functors

F: FinSet_{0} \to  Vect

where Vect is the category of vector spaces.   In Section 7.9 they
take Bill Schmitt's trick for getting cocommutative coalgebras from
structure types with restriction, and use it to get cococommutative
comonoids in the category of linear structure types!  In Section 7.10
they take another trick to get coalgebras from structure types:

22) William R. Schmitt, Incidence Hopf algebras, Journal of Pure and 
Applied Algebra 96 (1994), 299-330.   Also available at
<a href = "http://home.gwu.edu/~wschmitt/papers/iha.pdf">http://home.gwu.edu/~wschmitt/papers/iha.pdf</a>

and do something similar with that.
</li>

<li>
Chapter 9, "From species to graded vector spaces: Fock functors".
This studies what happens when you turn a Hopf monoid in the 
category of linear structure types into a graded Hopf algebra -
a kind of generalized Fock space.
</li>

<li>
Chapter 11, "Hopf monoids from geometry".  Here they get Hopf
monoids from the A_{n} Coxeter complexes, using a lot of ideas related
to Jacques Tits' theory of buildings.  There's a lot of q-deformation
going on here!  All these ideas are close to my heart.
</li>
</ul>

You can get more of a sense of what Aguiar is up to by looking at 
his homepage.  I'll just list a \emph{few} of the cool papers there:

23) Marcelo Aguiar's homepage, <a href = "http://www.math.tamu.edu/~maguiar/">http://www.math.tamu.edu/~maguiar/</a>

Marcelo Aguiar, Internal categories and quantum groups, Ph.D. thesis,
Cornell University, August 1997.  Available at
<a href = "http://www.math.tamu.edu/~maguiar/thesis2.pdf">http://www.math.tamu.edu/~maguiar/thesis2.pdf</a>

Marcelo Aguiar, Braids, q-binomials and quantum groups, Advances in 
Applied Mathematics 20 (1998) 323-365.  Also available at
<a href = "http://www.math.tamu.edu/~maguiar/braids.ps.gz">http://www.math.tamu.edu/~maguiar/braids.ps.gz</a>

Marcelo Aguiar and Swapneel Mahajan, Coxeter groups and Hopf 
algebras, Fields Institute Monographs, Volume 23, AMS, Providence, RI,
2006.  Also available at <a href = "http://www.math.tamu.edu/~maguiar/monograph.pdf">http://www.math.tamu.edu/~maguiar/monograph.pdf</a>

Check out the mysterious table of "generalized binomial
coefficients" in the second of these papers - it suggests
many links between different subjects of mathematics!

I was going to say a bit about quaternionic analysis, but now I'm
worn out.  So, I'll just say that anyone interested in generalizing
complex analysis to the quaternions must read two papers.  The first
I had managed to lose for a long time... but now I've found it again:

24) Anthony Sudbery, Quaternionic analysis, Math. Proc. Camb. \Phi l. 
Soc. 85 (1979),  199-225.  Available at 
<a href = "http://citeseer.ist.psu.edu/10590.html">http://citeseer.ist.psu.edu/10590.html</a>
and (slightly different version)
<a href = "http://theworld.com/~sweetser/quaternions/ps/Quaternionic-analysis.pdf">http://theworld.com/~sweetser/quaternions/ps/Quaternionic-analysis.pdf</a>

The second was brought to my attention by David Corfield:

25) Igor Frenkel and Matvei Libine, Quaternionic analysis, 
representation theory and physics, available as <a href = "http://arxiv.org/abs/0711.2699">arXiv:0711.2699</a>

Since Igor Frenkel is a bigshot, this paper may finally bring this
neglected subject some of the attention it deserves!  Like Corfield, 
I'll just quote the abstract, to make your mouth water:

\begin{quote}
  We develop quaternionic analysis using as a guiding principle 
  representation theory of various real forms of the conformal 
  group.  We first review the Cauchy-Fueter and Poisson formulas 
  and explain their representation theoretic meaning.  The 
  requirement of unitarity of representations leads us to the 
  extensions of these formulas in Minkowski space, which can 
  be viewed as another real form of quaternions.  Representation 
  theory also suggests a quaternionic version of the Cauchy formula 
  for the second order pole.  Remarkably, the derivative appearing 
  in the complex case is replaced by the Maxwell equations in the 
  quaternionic counterpart.  We also uncover the connection between 
  quaternionic analysis and various structures in quantum mechanics 
  and quantum field theory, such as the spectrum of the hydrogen atom, 
  polarization of vacuum, and one-loop Feynman integrals.  We also 
  make some further conjectures.  The main goal of this and our 
  subsequent paper is to revive quaternionic analysis and to show 
  profound relations between quaternionic analysis, representation 
  theory and four-dimensional physics.
\end{quote}

Finally, here's a clue for the Pythagorean pentagram puzzle.  To 
prove that 

\Phi  = 1 + 1/\Phi ,

show the length of the longest red interval here is the sum of the 
lengths of the two shorter ones:

<div align = "center">
<a href = "golden_ratio_pentagram.jpg">
<img style = "border:none;" src = "golden_ratio_pentagram.jpg">
% </a>
</div>

26) James Dolan and John Baez, annotated picture of Pythagorean 
pentagram, <a href = "http://math.ucr.edu/home/baez/golden_ratio_pentagram.jpg">http://math.ucr.edu/home/baez/golden_ratio_pentagram.jpg</a>

For more on the golden ratio, try "<a href = "week203.html">week203</A>".  For more on its 
relation to the dodecahedron, see "<a href = "week241.html">week241</A>".

\par\noindent\rule{\textwidth}{0.4pt}
\textbf{Addenda:}
Here's another stunning picture of the ridges and cracks on Europa:

<div align = "center">
<a href = "http://photojournal.jpl.nasa.gov/catalog/PIA03002">
<img src = "europa_ice_cracks.jpg">
% </a>
</div>

27) NASA Photojournal, Blocks in the Europan crust provide
more evidence of subterranean ocean, <a href = "http://photojournal.jpl.nasa.gov/catalog/PIA03002">http://photojournal.jpl.nasa.gov/catalog/PIA03002</a>

You can see more discussion of this Week's Finds at the <a href =
"http://golem.ph.utexas.edu/category/2008/05/this_weeks_finds_in_mathematic_26.html">\emph{n}-Category
Caf&eacute;</a>.  You can also see a list of questions I'd like your
help with!



\par\noindent\rule{\textwidth}{0.4pt}
<i>There is geometry in the humming of the strings, there is music in
the spacing of the spheres.</i> - Pythagoras

\par\noindent\rule{\textwidth}{0.4pt}

% </A>
% </A>
% </A>
