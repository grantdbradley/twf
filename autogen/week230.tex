
% </A>
% </A>
% </A>
\week{May 4, 2006 }


As we've seen in previous weeks, Mars is a beautiful world, but a
world in a minor key, a world whose glory days - the Hesperian Epoch -
are long gone, whose once grand oceans are now reduced to windy
canyons and icy dunes.  Let's say goodbye to it for now... leaving off
with this Martian sunset, photographed by the rover Spirit in Gusev
Crater on May 19th, 2005:

<DIV ALIGN = CENTER>
<A HREF = "http://marsrovers.nasa.gov/gallery/press/spirit/20050610a.html">
<IMG WIDTH = 400 HEIGHT = 300 SRC =" mars_sunset.jpg">
% </A>
</DIV>

1) A moment frozen in time, NASA Mars Exploration Rover Mission,
<A HREF = "http://marsrovers.nasa.gov/gallery/press/spirit/20050610a.html">
http://marsrovers.nasa.gov/gallery/press/spirit/20050610a.html</A>

This week I'll talk about Dynkin diagrams, quivers and Hall algebras. 
But first, some cool identities!

My student Mike Stay did computer science before he came to UCR.  
When he was applying, he mentioned a result he helped prove, which 
relates Goedel's theorem to the Heisenberg uncertainty principle:

2) C. S. Calude and M. A. Stay, From Heisenberg to Goedel via Chaitin,
International Journal of Theoretical Physics, 44 (2005), 1053-1065.
Also available at <A HREF = "http://math.ucr.edu/~mike/">http://math.ucr.edu/~mike/</A>    

Now, this particular combination of topics is classic crackpot fodder.
People think "Gee, uncertainty sounds like incompleteness, they're
both limitations on knowledge - they must be related!" and go off the 
deep end.  So I got pretty suspicious until I read his paper and 
saw it was CORRECT... at which point I \emph{definitely} wanted him around!  
The connection they establish is not as precise as I'd like, but it's
solid math.

So, now Mike is here at UCR working with me on 
quantum logic and quantum computation using ideas from category
theory.   In his spare time, he 
sometimes fools around with math identities and tries to categorify 
them - see "<A HREF = "week184.html">week184</A>" 
and "<A HREF = "week202.html">week202</A>" if you don't 
know what that means.  Anyway, maybe that's how he stumbled on this: 

3) Jonathan Sondow, A faster product for \pi  and a new integral
for ln(\pi /2), Amer. Math. Monthly 112 (2005), 729-734.  Also 
available as <A HREF = "http://arxiv.org/abs/math.NT/0401406">math.NT/0401406</A>.

In this paper, Sondow gives eerily similar formulas for some of
our favorite math constants.  First, one for e:



$$

       2  1/1      2^{2}   1/2     2^{3} \times  4  1/3       2^{4} \times  4^{4}    1/4
e =  ( - )     ( ----- )     ( ------- )     ( ------------ )     ...
       1         1 \times  3          1 \times  3^{3}          1 \times  3^{6} \times  5
$$
    
Then, one for \pi /2:



$$

\pi      2  1/2      2^{2}    1/4      2^{3} \times  4   1/8        2^{4} \times  4^{4}    1/16
- = ( - )     ( ------ )     ( --------- )     ( ------------- )     ...
2     1         1 \times  3            1 \times  3^{3}            1 \times  3^{6} \times  5
$$
    

Then one for e^{\gamma }, where \gamma  is Euler's constant:

 

$$

 \gamma ^{  }    2  1/2     2^{2}    1/3    2^{3} \times  4   1/4      2^{4} \times  4^{4}    1/5
e  = ( - )     ( ----- )     ( ------- )     ( ----------- )     ...
       1         1 \times  3         1 \times  3^{3}          1 \times  3^{6} \times  5
$$
    

He also points out Wallis' product for \pi /2 and Pippenger's for e:



$$

\pi      2  1/1    2\times 4  1/1    4\times 6\times 6\times 8  1/1
- = ( - )     ( --- )     ( ------- )     ...
2     1         3\times 3         5\times 5\times 7\times 7
$$
    


$$

      2  1/2    2\times 4  1/4    4\times 6\times 6\times 8  1/8
e = ( - )     ( --- )     ( ------- )     ...
      1         3\times 3         5\times 5\times 7\times 7
$$
    

What does it all mean?  I haven't a clue!   Another mystery thrown
down to us by the math gods, like a bone from on high... we can merely
choose to chew on it or not, as we wish.

Julie Bergner gave a great talk on "derived Hall algebras" at the 
Mac Lane memorial conference.  I just want to explain the very
first result she mentioned, due to Ringel - a surprising trick for 
constructing certain quantum groups from simply-laced Dynkin diagrams. 
It's very different from the \emph{usual} method for getting quantum groups 
from Dynkin diagrams, and it's a miracle that it works.  

But, I guess I should start near the beginning!

Way back in 1995, in "<A HREF = "week62.html">week62</A>", "<A HREF = "week63.html">week63</A>", "<A HREF = "week64.html">week64</A>" and "<A HREF = "week65.html">week65</A>", 
I explained how "Dynkin diagrams" - little gizmos like this:


\begin{verbatim}

             o
             |
 o--o--o--o--o--o--o
\end{verbatim}
    
show up all over mathematics.  They have a strange way of tying 
together subjects that superficially seem completely unrelated.
In one sense people understand how they work, but in another sense
they're very puzzling - their power keeps growing in unexpected ways.  

I love mysterious connections, so as soon I understood enough about
Dynkin diagrams to appreciate them, I became fascinated by them, 
and I've been studying them ever since.  I explained their relation
to geometry in "<A HREF = "week178.html">week178</A>", "<A HREF = "week179.html">week179</A>", "<A HREF = "week180.html">week180</A>", "<A HREF = "week181.html">week181</A>" and "<A HREF = "week182.html">week182</A>",
and their relation to quantum deformation and combinatorics in 
"<A HREF = "week186.html">week186</A>" and "<A HREF = "week187.html">week187</A>".  

You might think that would be enough - but you'd be wrong, way wrong!  

I haven't really talked about the most mysterious aspects of Dynkin 
diagrams, like their relation to singularity theory and representations
of quivers.  That's because these aspects were too mysterious!  
I didn't understand them \emph{at all}.  But lately, James Dolan and Todd 
Trimble and I have been making some progress understanding these aspects.  

First, I should remind you how Dynkin diagrams infest so much of
mathematics.  Let's start with a little puzzle mentioned in "<A HREF = "week182.html">week182</A>".

Draw n dots and connect some of them with edges - at most one edge 
between any pair of dots, please:


\begin{verbatim}

                           o
                          / \
                         /   \
            o     o-----o-----o-----o-----o
                       /
                      /
                     o          o-----o

\end{verbatim}
    
Now, try to find a basis of R^{n} 
consisting of one unit vector per dot, 
subject to these rules: if two dots are connected by an edge, the angle 
between their vectors must be 120 degrees, but otherwise their vectors 
must be at right angles. 

This sounds like a silly puzzle that only a mathematician could give a 
hoot about.  It takes a while to see its magnificent depth.   But anyway, 
it turns out you can solve this problem only for certain special diagrams 
called "simply-laced Dynkin diagrams".  The basic kinds are called A_{n}, 
D_{n}, E_{6}, E_{7}, and E_{8}.

The A_{n} Dynkin diagram is a line of n dots connected by edges like this:


\begin{verbatim}

 o---o---o---o---o    
\end{verbatim}
    

The D_{n} diagram has n dots arranged like this:


\begin{verbatim}

                   o
                  /
 o---o---o---o---o
                  \
                   o
\end{verbatim}
    
A line of them but then a little fishtail at the end!  We should
take n to be at least 4, to make the diagram connected and different
from A_{n}.

The E_{6}, E_{7}, and E_{8} diagrams look like this:


\begin{verbatim}

       o                  o                      o
       |                  |                      |
 o--o--o--o--o   o--o--o--o--o--o    o--o--o--o--o--o---o
\end{verbatim}
    

You're also allowed to take disjoint unions of the above diagrams. 

So, a weird problem with a weird answer!   Its depth is revealed
only when we see that many \emph{different} puzzles lead us to the 
\emph{same}
diagrams.  For example:

\textbf{A)} the classification of integral lattices in R^{n} 
having a basis of vectors whose length squared equals 2
\textbf{B)} the classification of simply laced semisimple Lie groups 
\textbf{C)} the classification of finite subgroups of the 3d rotation group
\textbf{D)} the classification of simple singularities
\textbf{E)} the classification of tame quivers

Let me run through these problems and say a bit about how they're
connected:

\textbf{A)}
An "integral lattice" in R^{n} 
is a lattice where the dot product 
of any two vectors in the lattice is an integer.   There are zillions
of these - but if we demand that they have a basis of vectors whose
length squared is 2, we can only get them from simply-laced Dynkin 
diagrams.

It's not very hard to see that finding a lattice like this is equivalent 
to the puzzle I mentioned earlier.  For example, given a solution of 
that puzzle, you can just multiply all your vectors by \sqrt 2 and
form the lattice of their integer linear combinations.  

Here are the lattices you get:

<UL>
<LI>
The diagram A_{n} gives the n-dimensional lattice 
of all (n+1)-tuples of integers (x_{1},...,x_{n+1}) with
x_{1} + ... + x_{n+1} = 0.
For example, A_{2} is a 2-dimensional hexagonal
lattice, the sort you use to pack pennies as densely as
possible.  Similarly, A_{3} gives a standard way of packing 
grapefruit.

<LI>
The diagram D_{n} gives the n-dimensional lattice
of all n-tuples of integers (x_{1},...,x_{n}) where
x_{1} + ... + x_{n} 
is even.
To visualize this, just take an n-dimensional checkerboard,
color the cubes alternately red and black, and take the center of
each red cube.  

<LI>
The diagram E_{8} gives the 8-dimensional lattice of
8-tuples (x_{1},...,x_{8}) such
that the x_{i} are either all integers or all half-integers - a
half-integer being an integer plus 1/2 - and such that
x_{1} + ... + x_{8} 
is even.

<LI>
The diagram E_{7} gives the 7-dimensional lattice consisting
of all vectors in E_{8} that are orthogonal to some vector
that's closest to the origin (and thus has length \sqrt 2).

<LI>
The diagram E_{6} gives the 6-dimensional lattice consisting
of all vectors in E_{7} that are orthogonal to some vector
that's closest to the origin (and thus has length \sqrt 2).
</UL>

For more on these lattices, see "<A HREF = "week65.html">week65</A>".
They show up in the theory of Lie groups....

\textbf{B)} Lie groups are fundamental throughout math and physics: they're
groups of continuous symmetries, like rotations.  The nicest of the
lot are the semisimple Lie groups.  Some familiar examples are the 
group of rotations in n-dimensional space, which is called SO(n), 
and the group of unitary matrices with determinant 1, which is 
called SU(n).  There are more, but people know what they all are.
They're classified by Dynkin diagrams!  

Why?  The key point is that any semisimple Lie group 
has a "root lattice".  
This is an integral lattice spanned by special vectors 
called "roots".  
I won't give the details, since I explained this 
stuff in "<A HREF = "week63.html">week63</A>" 
and "<A HREF = "week64.html">week64</A>", but it turns out 
that root lattices, and thus semisimple 
Lie groups, are classified by Dynkin diagrams.  

Not all these Dynkin diagrams look like the A, D and E diagrams listed
above.  But, it turns out that the length squared of any root must be 
either 1 or 2.  If all the roots have length squared equal to 2, we
say our semisimple Lie group is "simply laced".  In this case, we're
back to problem B), which we already solved!  So then our Lie group 
corresponds to a diagram of type A, D, or E - or a disjoint union of 
such diagrams.

Here's how it goes:

<UL>
<LI>
 The diagram A_{n} gives the compact Lie group SU(n+1), 
  consisting of (n+1) \times  (n+1) unitary matrices with determinant 1.  
  It's the isometry group of complex projective n-space.

<LI>
 The diagram D_{n} gives the compact Lie group SO(2n), 
  consisting of 2n \times  2n orthogonal matrices with determinant 1.
  It's the isometry group of real projective (2n)-space.

<LI>
 The diagram E_{6} gives a 78-dimensional compact Lie group
  that people call E_{6}.  It's the isometry group of the bioctonionic
  projective plane.

<LI>
 The diagram E_{7} gives a 133-dimensional compact Lie group
  that people call E_{7}.  It's the isometry group of the quateroctonionic
  projective plane.

<LI>
 The diagram E_{8} gives a 248-dimensional compact Lie group
  that people call E_{8}.  It's the isometry group of the octooctonionic 
  projective plane.
</UL>

In short, two regular series and three exotic weirdos.  

You may ask where the rotation groups SO(n) with n odd went!  
Well, these correspond to fancier Dynkin diagrams that aren't
simply laced, like this:

 o---o---o---o=>=o

The funny arrow here indicates that the last two vectors aren't 
at a 120-degree angle; they're at a 135-degree angle, and the last
vector is shorter than the rest: it has length one instead of \sqrt 2.

Why are semisimple Lie groups "better" when they're simply laced?
What's the big deal?  I don't really understand this, but for one,
when all the roots have the same length, they're all alike - 
a certain symmetry group called the Weyl group acts transitively on them.  

Anyway, so far our A, D, E Dynkin diagrams have been classifying
things that are clearly related to lattices.  But now things get
downright spooky....

\textbf{C)} 
Take a Platonic solid and look at its group of rotational symmetries.
You get a finite subgroup of the 3d rotation group SO(3).  But in 
general, finite subgroups of SO(3) are classified by ADE Dynkin 
diagrams!  

So, Platonic solids turn out to fit into the game we're playing here!

First I'll say which diagram corresponds to which subgroup of SO(3).
Then I'll explain how the correspondence works:

<UL>
<LI>
 The diagram A_{n} corresponds to the group of obvious rotational symmetries
  of the regular n-gon.  This group is called the "cyclic group" Z/n.

<LI>
 The diagram D_{n} corresponds to the group of rotational symmetries 
  of the regular n-gon where you can turn it and also flip it over.
  By sheer coincidence, this group is called the "dihedral group" D_{n}. 
  A cosmic stroke of good luck!

<LI>
 The diagram E_{6} corresponds to the group of rotational symmetries 
  of the tetrahedron: the "tetrahedral group".  This is also the 
  group of even permutations of 4 elements, the "alternating group"
  A_{4} - not to be
  confused with the A_{n}'s we were just talking about.  A cosmic 
  stroke of bad luck!

<LI>
 The diagram E_{7} corresponds to the group of rotational symmetries
  of the octahedron or cube: the "octahedral group".  This is also 
  the group of all permutations of 4 elements, the "symmetric
  group" S_{4}.  

<LI>
 The diagram E_{8} corresponds to the group of rotational symmetries
  of the icosahedron or dodecahedron: the "icosahedral group".  This 
  is also the group of even permutations of 5 elements, called A_{5}.  
 Darn!
</UL>

So, the exceptional Lie groups E_{6}, E_{7} and E_{8} correspond to Platonic
solids in a sneaky way.  

To understand what's going on here, first we need to switch from SO(3)
to SU(2).  The group SU(2) is used to describe rotations in quantum 
mechanics: it's the double cover of the rotation group SO(3).  

It's really finite subgroups of SU(2) that are classified by ADE Dynkin 
diagrams!   It just so happens that these correspond, in a slightly
slippery way, to finite subgroups of SO(3).  

You'll see how if I list the finite subgroups of SU(2): 

<UL>
<LI>
 The diagram A_{n} corresponds to the cyclic subgroup Z/n of SU(2).
  This double covers a cyclic subgroup of SO(3) when n is even.

<LI>
 The diagram D_{n} corresponds to a subgroup of SU(2) that double 
  covers the dihedral group D_{n}.

<LI>
 The diagram E_{6} corresponds to a subgroup of SU(2) that double
  covers the rotational symmetries of the tetrahedron.  This
  subgroup has 24 elements and it's called the "binary tetrahedral group".

<LI>
 The diagram E_{7} corresponds to a subgroup of SU(2) that double
  covers the rotational symmetries of the octahedron.  This
  subgroup has 48 elements and it's called the "binary octahedral group".

<LI>
 The diagram E_{8} corresponds to a subgroup of SU(2) that double
  covers the rotational symmetries of the icosahedron.  This
  subgroup has 120 elements and it's called the "binary icosahedral group".
</UL>

Now, how does the correspondence work?  For this, I'm afraid I have to 
raise the sophistication level a bit - I've been trying to keep things
simple, but it's getting tough.  

In his book on the icosahedron, Felix Klein noticed it was interesting 
to let the icosahedral group act on the Riemann 
sphere, and look for rational functions invariant under this group.   

It turned out that every such function depends on a single one: 
Klein's icosahedral function!  The explict formula for it is pretty
disgusting, but it's a beautiful thing: you can pick it so that it 
equals 0 at all the vertices of the icosahedron, 1 at the midpoints
of the edges, and infinity at the midpoints of the faces.  Even
better, if you write the function like this:

w = f(z)

then Klein showed that knowing how to solve for z as a function of w 
lets you solve every quintic equation!  The reason is that the Galois
group of the general quintic is a close relative of the icosahedral
group: the former is S_{5}, the latter is A_{5}.

Anyway, when I said that "every such function depends 
on a single one",
what I really meant was this.  Let C(z) be the field of rational functions
of one variable; then the icosahedral group acts on this,
and the invariant functions form a subfield C(w) where w is Klein's
icosahedral function.  The Galois group of the little field in the big
one is the icosahedral group.

The same kind of thing works for the other finite subgroups of SO(3),
except of course for the connection to the quintic equation.

But, it's actually even better to think about finite subgroups of 
SU(2), since SU(2) acts on C^{2}, and when we \emph{projectivize} C^{2} we
get SO(3) acting on the Riemann sphere.  This viewpoint fits more 
squarely into the worldview of algebraic geometry.

If we take the quotient of C^{2} by a finite subgroup G of SU(2), we 
don't get a smooth manifold: the quotient has a singularity at 0.
But we can "resolve" the singularity, finding a smooth complex manifold
with a holomorphic map

p: M \to  C^{2}/G

that has a holomorphic inverse on a dense open set.  There may be 
lots of ways to do this, but in the present case there's just one 
"minimal" resolution, meaning a resolution that every other resolution
factors through.

Then - and here's the magic part! - the inverse image of 0 in M 
turns out to be the union of a bunch of Riemann spheres.  And
if we draw a dot for each sphere, and an edge between these dots 
whenever their spheres intersect, we get a simply laced Dynkin 
diagram on the above list!!!

Well, almost.  We get this diagram with an extra dot thrown in,
connected by some extra edges in a specific way.  This is called 
the "extended" Dynkin diagram.  It also shows up naturally from 
the Lie group viewpoint, when we consider central extensions of 
loop groups.  

That's \emph{one} way the correspondence works.  Another way, discovered
by McKay, is to draw a dot for each irrep of G.  There's always a
2-dimensional representation of G coming from the action of SU(2) 
on C^{2}.  Let's just call this irrep C^{2}.  
Then, draw an edge
from the dot R to the dot S whenever the irrep S shows up 
in the rep R \otimes  C^{2}.
You get the same extended Dynkin diagram as before!
The special extra dot in the Dynkin diagram corresponds to the 
trivial rep of G.

This second way is called the "McKay correspondence".  The first way
is sometimes called the "geometric McKay correspondence", though I think
it was discovered earlier.

Now we're well on the road to the next item...

\textbf{D)} 
Simply-laced Dynkin diagrams also classify the simple critical points
of holomorphic functions 

f: C^{3} \to  C

A "critical point" is just a place where the gradient of f vanishes.
We can try to classify critical points up to a holomorphic change of 
variables.  It's better to classify their "germs", meaning we only 
look at what's going on \emph{right near} the critical point.   But, even 
this is hopelessly complicated unless we somehow limit our quest.

To do this, we can restrict attention to "stable" critical points, 
which are those that don't change type under small perturbations.  
But we can do better: we can classify "simple" critical points, 
namely those that change into only finitely many other types under 
small perturbations.

These correspond to simply-laced Dynkin diagrams!

First I'll say which diagram corresponds to which type of critical
point.  To do this, I'll give a polynomial f(x,y,z) that has a certain 
type of critical point at x = y = z = 0.  Then I'll explain how
the correspondence works:

<UL>
<LI>
The diagram A_{n} corresponds to the critical point of x^{n+1} + y^{2} + z^{2}.

<LI>
The diagram D_{n} corresponds to the critical point of x^{n-1} + xy^{2} + z^{2}.

<LI>
The diagram E_{6} corresponds to the critical point of x^{4} + y^{3} + z^{2}.

<LI>
The diagram E_{7} corresponds to the critical point of x^{3} y + y^{3} + z^{2}.

<LI>
The diagram E_{8} corresponds to the critical point of x^{5} + y^{3} + z^{2}.
</UL>

Here's how the correspondence works.  For each of our Dynkin diagrams
we have a finite subgroup of SU(2), thanks to item C).  This subgroup 
acts on the ring of polynomials on C^{2}, so we can form the subring of 
invariant polynomials.  This turns out to be generated by three polynomials 
that we will arbitrarily call x, y, and z.  But, they satisfy one relation, 
given by the polynomial above!

Conversely, we can start with the polynomial

f: C^{3} \to  C

The zero set

{f = 0}

has an isolated singularity at the origin.  But, we can resolve
this singularity, finding a smooth complex manifold N with a 
holomorphic map

q: N \to  {f = 0}

that has a holomorphic inverse on a dense open set.  There may be 
lots of ways to do this, but in the present case there's just one 
"minimal" resolution, meaning one that every other resolution 
factors through this one.

Then - and here's the magic part! - the inverse image of 0 in N
turns out to be the union of a bunch of Riemann spheres.  And
if we draw a dot for each sphere, and an edge between these dots 
whenever their spheres intersect, we get back our simply laced
Dynkin diagram!!!

This whole section should have given you a feeling of deja vu.
It's a lot like section D).  If I were smarter, I'd probably see 
how it's \emph{exactly} the same stuff, repackaged slightly.  

The last item on our list seems different....

\textbf{E)} A quiver is just a category freely generated by some set of
morphisms.  To specify a quiver we just write down some dots
and arrows.  The dots are the objects of our category; the 
arrows are the generating morphisms.

A representation of a quiver Q is just a functor 

F: Q \to  Vect

So, we get a vector space for each dot and a linear map for
each arrow, with no extra restrictions.  There's an obvious
category of representations Rep(Q) of a quiver Q.

A guy named Gabriel proved a divine result about these categories
Rep(Q).  We say a quiver Q has "finite representation type"
if Rep(Q) has finitely many indecomposable objects - objects that
aren't direct sums of others.  And, it turns out the quivers of
finite representation type are just those coming from simply-laced
Dynkin diagrams!!

Actually, for this to make sense, you need to take your Dynkin diagram
and turn it into a quiver by putting arrows along the edges.  If you
have an ADE Dynkin diagram, you get a quiver of finite representation
type no matter which way you let the arrows point.

There's clearly a lot of mysterious stuff going on here.  In
particular, this last item sounds completely unrelated to the 
rest.  But it's not!  There are cool relationships between quivers
and quantum groups, which tie this item to the rest.

I'll just mention one - the one Julie Bergner started her talk with.

For this, you need to know a bit about abelian categories.

Abelian categories are categories like the category of abelian groups,
or more generally the category of modules of any ring, where you can 
talk about chain complexes, exact sequences and stuff like that.  
You can see the precise definition here:

4) Abelian categories, Wikipedia, 
<A HREF = "http://en.wikipedia.org/wiki/Abelian_category">http://en.wikipedia.org/wiki/Abelian_category</A>

and learn more here:

5) Peter Freyd, Abelian Categories, Harper and Row, New York, 1964.
Also available at <A HREF = "http://www.tac.mta.ca/tac/reprints/articles/3/tr3abs.html">http://www.tac.mta.ca/tac/reprints/articles/3/tr3abs.html</A>

It's really interesting to study the "Grothendieck group" K(A) of an 
abelian category A.  As a set, this consists of formal linear combinations 
of isomorphism classes of objects of A, where we impose the relations

[a] + [b] = [x]

whenever we have a short exact sequence

0 \to  a \to  x \to  b \to  0

It becomes an abelian group in an obvious way.

For example, if A is the category of representations of some group G, 
it's an abelian category and K(A) is called the "representation ring" 
of A - it's a ring because we tensor representations.  Or, if A is the 
category of vector bundles over a space X, it's again abelian, and K(A) 
is called the "K-theory of X".  

The Hall ring H(A) of an abelian category is a vaguely similar idea.
As a set, this consists of formal linear combinations of isomorphism
classes of objects of A.  No extra relations!  It's an abelian group 
with the obvious addition.  But the cool part is, with a little luck, 
we can make it into a \emph{ring} by letting the product [a] [b] be the 
sum of all isomorphism classes of objects [x] weighted by the number 
of isomorphism classes of short exact sequences

0 \to  a \to  x \to  b \to  0

This only works if the number is always finite.  

So far when speaking of "formal linear combinations" I've been 
implicitly using integer coefficients, but people seem to prefer 
complex coefficients in the Hall case, and they get something
called the "Hall algebra" instead of the "Hall ring".

The fun starts when we take the Hall algebra of Rep(Q), where
Q is a quiver.  We could look at representations in vector
spaces over any field, but let's use a finite field - necessarily 
a field with q elements, where q is a prime power.  

Then, Ringel proved an amazing theorem about the Hall algebra 
H(Rep(Q)) when Q comes from a Dynkin diagram of type A, D, or E:

5) C. M. Ringel, Hall algebras and quantum groups, Invent. Math. 101
(1990), 583-592.

He showed this Hall algebra is a quantum group!  More precisely, 
it's isomorphic to the q-deformed universal enveloping algebra 
of a maximal nilpotent subalgebra of the Lie algebra associated 
to the given Dynkin diagram.  

That's a mouthful, but it's cool.  For example, the Lie algebra 
associated to A_{n} is sl(n+1), and the maximal nilpotent subalgebra 
consists of strictly upper triangular matrices.   
We're q-deforming the universal enveloping algebra of this.  
One cool thing
is that the "q" of q-deformation gets interpreted
as a prime power - something we've already seen in "<A HREF = "week185.html">week185</A>" and subsequent weeks.

So, it seems that all the ways simply-laced Dynkin diagrams show
up in math are related.  But, I don't think anyone understands
what's really going on!  It's like black magic.  

And, I've just described \emph{some} of the black magic!  

For example, you'll notice I portrayed the Hall algebra H(A) 
as a kind of evil twin of the more familiar Grothendieck group K(A).  
They have some funny relations.  For example, if you take the minimal 
resolution of C^{2}/G where G is a finite subgroup of SU(2), you get a 
variety whose K-theory (as defined above) is isomorphic to the 
representation ring of G!  This was shown here:

6) G. Gonzalez-Springberg and J. L. Verdier, Construction geometrique
de la correspondance de McKay, Ann. ENS 16 (1983), 409-449.

For further developments, try this paper, which studies the derived 
category of coherent sheaves on this minimal resolution of C^{2}/G:

7) Mikhail Kapranov and Eric Vasserot, Kleinian singularities,
derived categories and Hall algebras, available as <A HREF = 
"http://arxiv.org/abs/math.AG/9812016">math.AG/9812016</A>.

Now let me give a bunch of references for further study.  For a
really quick overview of the whole ADE business, try these:

8) Andrei Gabrielov, Coxeter-Dynkin diagrams and singularities,
in Selected Papers of E. B. Dynkin with Commentary, eds. A. A.
Yushkevich, G. M. Seitz and A. I. Onishchik, AMS, 1999.
Also available at 
<a href = "http://www.math.purdue.edu/~agabriel/dynkin.pdf">
http://www.math.purdue.edu/~agabriel/dynkin.pdf</a>

9) John McKay, A rapid introduction to ADE theory, 
<A HREF = "http://math.ucr.edu/home/baez/ADE.html">http://math.ucr.edu/home/baez/ADE.html</A>

Here's a more detailed but still highly readable introduction:

10) Joris van Hoboken, Platonic solids, binary polyhedral groups, 
Kleinian singularities and Lie algebras of type A,D,E, Master's Thesis, 
University of Amsterdam, 2002, available at 
<A HREF = "http://math.ucr.edu/home/baez/joris_van_hoboken_platonic.pdf">http://math.ucr.edu/home/baez/joris_van_hoboken_platonic.pdf</A>

This classic has recently become available online:

11) M. Hazewinkel, W. Hesselink, D. Siermsa, and F. D. Veldkamp, The 
ubiquity of Coxeter-Dynkin diagrams (an introduction to the ADE problem), 
Niew. Arch. Wisk., 25 (1977), 257-307.
Also available at
<A HREF = "http://repos.project.cwi.nl:8888/cwi_repository/docs/I/10/10039A.pdf
">http://repos.project.cwi.nl:8888/cwi_repository/docs/I/10/10039A.pdf</A>
or 
<A HREF = "http://math.ucr.edu/home/baez/hazewinkel_et_al.pdf">
http://math.ucr.edu/home/baez/hazewinkel_et_al.pdf</A>

Here's a really nice, elementary introduction to Klein's work on
the icosahedron and the quintic:

12) Jerry Shurman, Geometry of the Quintic, Wiley, New York, 1997.
Also available at <a href = "http://people.reed.edu/~jerry/Quintic/quintic.html">http://people.reed.edu/~jerry/Quintic/quintic.html</a>

I haven't seen this book, but I hear it's good:

13) P. Slodowy, Simple Singularities and Algebraic Groups, 
Lecture Notes in Mathematics 815, Springer, Berlin, 1980.

Here's a bibliography with links to online references:

14) Miles Reid, Links to papers on McKay correspondence, 
<A HREF = "http://www.maths.warwick.ac.uk/~miles/McKay/">http://www.maths.warwick.ac.uk/~miles/McKay/</A>

Of those references, I especially like this:

15) Miles Reid, La Correspondence de McKay (in English), 
Seminaire Bourbaki, 52eme annee, November 1999, no. 867, 
to appear in Asterisque 2000.  Also available as <A HREF = 
"http://arxiv.org/abs/math.AG/9911165">math.AG/9911165</A>.

Here you'll also see some material about \emph{generalizations}
of the McKay correspondence.  For example, if we take a finite
subgroup G of SU(3), we get a quotient C^{3}/G, which has 
singularities.  If we take a "crepant" resolution of 

p: M \to  C^{3}/G,

which is the right generalization of a minimal resolution, then
M is a Calabi-Yau manifold.  This gets string theory into the act!
Around 1985, Dixon, Harvey, Vafa and Witten used this to guess 
that the Euler characteristic of M equals the number of irreps of G.
A lot of work has been done on this since then, and Reid's article
summarizes a bunch.

Apparently a "crepant" resolution is one that induces 
an isomorphism of canonical bundles; when this fails to happen
folks say there's a discrepancy, so a crepant resolution is one
with no dis-crepancy.  Get it?  
Since a Calabi-Yau 
manifold is one whose canonical bundle is trivial, it shouldn't
be completely shocking that crepant
resolutions yield Calabi-Yaus.  This all works in the original
2d McKay correspondence, too - the minimal resolutions we saw
there are also crepant.  
 
In fact, string theory also sheds light on the original McKay
correspondence.  The reason is that the minimal resolution of
C^{2}/G is a very nice Riemannian 4-manifold (when viewed as a
\emph{real} manifold).  It's an "asymptotically locally
Euclidean" manifold, or ALE manifold for short.  Doing string
theory on this gives a way of seeing how the extended Dynkin diagrams
sneak into the McKay correspondence: they're the Dynkin diagrams for
central extensions of loop groups, which show up as gauge groups in
string theory!  I don't really understand this, but it makes a kind of
sense.

I guess this is a famous paper about this stuff:

16) Michael R. Douglas and Gregory Moore, D-branes, quivers and ALE
instantons, available as <A HREF = "http://xxx.lanl.gov/abs/hep-th/9603167">hep-th/9603167</A>.

\par\noindent\rule{\textwidth}{0.4pt}
\textbf{Addenda:} 
Thanks go to Jeff Barnes for showing how to get ahold of Hazewinkel
\emph{et al}'s paper online.
I got some nice feedback from Graham Leuschke, David
Rusin, and Leslie Coghlan, and I used Leuschke's to fix a mistake. 

Graham Leuschke wrote:

\begin{quote}
  Hi - 

  A quick correction to your TWF this week.  The ADE diagrams are 
  actually the underlying graphs of quivers of \emph{finite} representation 
  type, not tame.  You gave the right definition, but the wrong name 
  for it.  Tame representation type usually means that there are 
  infinitely many indecomposable representations, but they come in 
  nice one-dimensional families.  (The third option is wild representation 
  type, which usually means that classifying the representations would 
  be at least as hard as classifying all modules over the non-commutative 
  polynomial ring k<x,y>.  It's a theorem of Drozd that one of these must 
  hold.)

  This actually points toward more black magic: the quivers of finite type 
  have ADE diagrams for underlying graphs, while the quivers of tame type 
  have "extended ADE" diagrams underneath them.  These extended graphs 
  are the result of adding one (particular) vertex to each of the ADE 
  graphs, and they often arise as the answer to questions that are just 
  slightly weaker than the questions answered by the ADE graphs.  For 
  example, the ADE graphs are those for which the Tits form is positive 
  definite, while the extended ADEs are those for which it's positive 
  semi-definite.    They correspond to Kac-Moody affine Lie algebras 
  rather than simple Lie algebras, and so on and so on.

  Harm Derksen and Jerzy Weyman had a nice overview of quiver 
  representations and the theorem of Gabriel in the Notices last year:

  17) Harm Derksen and Jerzy Weyman, Quiver representations, 
  AMS Notices 52 (2005), 200-206.  Also available as
  <A HREF = "http://www.ams.org/notices/200502/fea-weyman.pdf">http://www.ams.org/notices/200502/fea-weyman.pdf</A>

  Idun Reiten had a similar one in the Notices back in 1997, but I 
  can't find it online anywhere:

  18) Idun Reiten, Dynkin diagrams and the representation theory of 
  algebras, AMS Notices 44 (1997), 546-556.

  She did a really nice job of explaining the connections with 
  quadratic forms and (sub)additive functions.

  Cheers,<br>
  Graham
\end{quote}

David Rusin wrote:

\begin{quote}
  John Baez wrote:


$$

  The diagram E_{8} corresponds to the critical point of x^{5} + y^{3} + z^{2}.
$$
    

  Milnor has a lovely little book:

  19) John Milnor, Singular points of complex hypersurfaces, 
  Ann. Math. Studies 61, Princeton U. Press, Princeton, 1968.

  which takes the point of view that the RIGHT thing to do at an isolated
  critical point of a complex-analytic projective variety is to intersect
  the variety with a small sphere centered at the critical point.

   Brieskorn did this with the varieties

   x^{4k+1} + y^{3} + z^{2} + w^{2} + u^{2} = 0

   which are 4-dimensional complex varieties with isolated critical 
   points at the origin. So apart from the origin the equation 
   describes an 8-dimensional manifold and the intersection with the 
   sphere in C^{5} = R^{10} is a 7-dimensional manifold.  Brieskorn showed:

<OL>
<LI>
 For every k, these manifolds M_{k} are homeomorphic to the sphere

<LI> For every k, M_{k} is a smooth manifold.

<LI>
   3. M_{j} and M_{k} are diffeomorphic iff  j = k mod 28.
</OL>

   Thus in particular, the whole group of diffeomorphism classes of 
   manifolds which are homeomorphic to the sphere has order 28. 
   Milnor had earlier proved that there are 28 diffeomorphism 
   classes of 7-spheres.  But here they are very explicit!
   dave
\end{quote}

Leslie Coghlan wrote:

\begin{quote}
Please add to Week 230 links to copies of these two papers:

20) H. S. M. Coxeter: The evolution of Coxeter-Dynkin diagrams, in: 
Polytopes: Abstract, Convex and Computational, eds.
T. Bisztriczky, P.  McMullen, R. Schneider and A. Ivic Weiss,  
NATO ASI Series C, Vol. 440, Kluwer, Dordrecht, 1994, pp. 21-42.

21) E. Witt, Spiegelungsgruppen und Aufzahlung halbeinfacher Liescher Ringe,
Abhandl. Math. Sem. Univ. Hamburg. 14 (1941), 289-337.

  Yours,

  Leslie Coghlan
\end{quote}

Here are a couple more online introductions: 


22) William Crawley-Boevey, Notes on quiver representations, 
available at <a href = "http://www.amsta.leeds.ac.uk/~pmtwc/quivlecs.pdf">
http://www.amsta.leeds.ac.uk/~pmtwc/quivlecs.pdf</a>.

23) Alistair Savage, Finite-dimensional algebras and quivers, 
available as <a href = "http://www.arxiv.org/abs/math/0505082">arXiv:math/0505082</a>.


\par\noindent\rule{\textwidth}{0.4pt}

<em>This thesis is an attempt to show an astonishing relation between
basic objects from different fields in mathematics.  Most peculiarly
it turns out that their classification is "the same": the ADE 
classification.  Altogether these objects and the connections between
them form a coherent web.  

The connections are accomplished by direct constructions leading to
bijections between these classes of objects.  These constructions 
however do not always explain or give satisfactory intuition why these
classifications [exist], or to say it better, why they should be
related in this way.  Therefore the deeper reason remains mysterious
and when discovered will have to be of great depth.  This gives a
high motivation to look for new concepts and it shows that simple
and since long understood mathematics can still raise very interesting
questions, show paths for new research and give a glance at the 
mystery of mathematics.  In my opinion to be aware of a certain
truth without having its reason is fundamental to the practise of 
mathematics.</em> - Joris van Hoboken 




\par\noindent\rule{\textwidth}{0.4pt}

% </A>
% </A>
% </A>
