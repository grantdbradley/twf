
% </A>
% </A>
% </A>
\week{August 24, 1995 }

I'd like to return to the theme of octonions, which I began to explore
in "<A HREF = "week59.html">week59</A>".  The recipe I described there, which starts with the real
numbers, and then builds up the complex numbers, quaternions,
octonions, hexadecanions etc. by a recursive process, is called the
"Cayley-Dickson process".  Now let me describe a way to obtain the
octonions using a special property of rotations in 8-dimensional
space, called "triality".  I'll start with a gentle introduction to
the theory of rotation groups; for this, a nice reference is the book
by Fulton and Harris that I mentioned in "<A HREF = "week59.html">week59</A>".  Then I will turn
up the heat a bit and describe triality and how to use it to get the
octonions.  I learned some of this stuff from:

1) Alex J. Feingold, Igor B. Frenkel, and John F. X. Rees, 
Spinor construction of vertex operator algebras, triality, and
E_{8}^{(1)}, Contemp. Math. 121, AMS, Providence Rhode Island.
  

I should emphasize, however, that what I will talk about is older,
while the above book starts with triality and then does far more
sophisticated things.  An older reference for what I'll talk about is

2) Claude Chevalley, The algebraic theory of spinors, Columbia U. Press,
New York, 1954.  

I think the concept of triality goes back to Cartan, but I don't
really know the history.  By the way, I'd really appreciate any
corrections to what I say below.  

Okay, so, how should we start?  Well, probably we should start with
the group of rotations in n-dimensional Euclidean space.  This group
is called SO(n).  It is not simply connected if n > 1, meaning that
there are loops in it which cannot be continuously shrunk to a point.
This is easy to see for SO(2), which is just the circle - or, if you
prefer, the unit complex numbers.  It's a bit trickier to see for
SO(3), but it is easy enough to demonstrate - either mathematically
or via the famous "belt trick" - that the loop consisting of a 360
degree rotation around an axis cannot be continuously shrunk to a
point, while the loop consisting of a 720 degree rotation around an
axis can.

This "doubly connected" property of SO(3) implies that it has an
interesting "double cover", a group G in which all loops \emph{can} be
contracted to a point, together with a two-to-one function F: G \to 
SO(3) with F(gh) = F(g)F(h).  (This sort of function, the nice kind of
function between groups, is called a "homomorphism".)  And this double
cover G is just SU(2), the group of 2x2 complex matrices which are
unitary and have determinant 1.  Better yet - if we are warming up
for the octonions - we can think of SU(2) as the unit quaternions!

Now elements of SO(n) are just nxn real matrices which are orthogonal
and have determinant 1, so given an element g of SO(n) and a vector v
in R^n, we can do matrix multiplication to get a new vector gv in R^n,
which of course is just the result of rotating v by the rotation g.
This makes R^n into a "representation" of SO(n), meaning simply that

(gh)v = g(hv) 

and 

1v = v.

We call R^n the "vector" representation of the rotation group SO(n),
for obvious reasons.  

Now SO(n) has lots of other representations, too.  If we consider
SO(3), for example, there is in addition to the vector representation
(which is 3-dimensional) also the trivial 1-dimensional representation
(where the group element g acts on a complex number v by leaving it
alone!) and also interesting representations of dimensions 5, 7, 9,
etc..  The interesting representation of dimension 2j+1 is called the
"spin-j" representation by physicists.  All representations of SO(3)
can be built up from these representations, and none of these
representations can be broken down into smaller ones - one says they
are irreducible.

But the double cover of SO(3), namely SU(2), has more representations!
Using the two-to-one homomorphism F: SU(2) \to  SO(3) we can convert
any representation of SO(3) into one of SU(2), but not vice versa.
For example, since SU(2) consists of 2x2 complex matrices, it has
a representation on C^2, given by the obvious matrix multiplication.
This is called the "spinor" or "spin-1/2" representation of SU(2).
It doesn't come from a representation of SO(3).

To digress a bit, the reason physicists got so interested in SU(2) is
that to describe what happens when you rotate a particle (in the
framework of quantum theory) it turns out you need, not just the
representations of SO(3), but of its double cover, SU(2).  E.g., an
electron, proton or neutron is described by the spin-1/2
representation.  This implies that when you turn an electron around
360 degrees about an axis, its wavefunction changes sign, but when you
rotate it another 360 degrees, its wavefunction is back to where it
started.  You can't describe this behavior using representations of
SO(3), but you can using SU(2).  In general, for any j = 0, 1/2, 1,
3/2, 2, etc., there is an irreducible representation of SU(2), called
the "spin-j" representation, which is (2j+1)-dimensional.  Only when
the spin is an integer does the representation come from one of SO(3).

Things get more complicated when we consider rotations in higher
dimensional space.  For any n greater than or equal to 3, the
group SO(n) is doubly connected, and has a simply connected double
cover, which in general is called Spin(n).  Folks have figured
out all the representations of Spin(n) and which of these come
from representations of SO(n).  It is more complicated for n > 3
than for n = 3 (in particular, they aren't just classified by
"spin"), but it is still quite comprehensible and charming.
Just to head off any confusions that might occur, let me emphasize
that it's sort of a lucky coincidence that Spin(3) = SU(2).  
In general, the spin groups don't have too much to do with the
groups SU(n) of nxn unitary complex matrices with determinant 1.

There is, however, a doubly lucky coincidence in dimension 4; 
namely, Spin(4) = SU(2) x SU(2).  In other words, an element of
Spin(4) can be thought of as a pair of SU(2) matrices, and
we multiply these pairs like (g,g')(h,h') = (gh,g'h').  This implies
that the irreducible representations of Spin(4) are given by a 
"tensor product" of two irreducible representations of SU(2), so we can 
classify them by pairs of spins, say (j,j').  The dimension of
the (j,j') representation is (2j+1)(2j'+1), since the dimension
of a tensor product is the product of the dimensions.  In particular, we
call the (1/2,0) representation the "left-handed" spinor representation
and the (0,1/2) representation the "right-handed" spinor representation.
The reason is that a reflection transforms one into the other.  
Since spacetime is 4-dimensional, representations of Spin(4) are
important in physics, although really one should keep track of
the fact that time works a bit differently than space, which Spin(4)
fails to do.  In any event, ignoring the subtleties about how
time works differently than space, we can roughly say that the 
existence of left-handed and right-handed spinor representations
of Spin(4) is the reason why massless spin-1/2 particles such as
neutrinos can have a "handedness" or "chirality".

More generally, it turns out that the representation theory
of Spin(n) depends strongly on whether n is even or odd.  When n
is even (and bigger than 2), it turns out that Spin(n) has
left-handed and right-handed spinor representations, each of
dimension 2^{n/2 - 1}.   When n is odd there is just one spinor
representation.  Of course, there is always the representation of
Spin(n) coming from the vector representation of SO(n), which is
n-dimensional.

This leads to something very curious.  If you are an ordinary
4-dimensional physicist you undoubtedly tend to think of spinors as
"smaller" than vectors, since the spinor representations are
2-dimensional, while the vector representation is 3-dimensional.
However, in general, when the dimension n of space (or spacetime) is
even, the dimension of the spinor representations is 2^(n/2 - 1),
while that of the vector representation is n, so after a while
the spinor representation catches up with the vector representation
and becomes bigger!

This is a little bit curious, or at least it may seem so at first, but
what's \emph{really} curious is what happens exactly when the spinor
representation catches up with the vector representation.  That's when
2^(n/2 - 1) = n, or n = 8.  The group Spin(8) has three 8-dimensional
irreducible representations: the vector, left-handed spinor, and
right-handed spinor representation.  While they are not equivalent to
each other, they are darn close; they are related by a symmetry of
Spin(8) called "triality".  And, to top it off, the octonions can be
seen as a kind of spin-off of this triality symmetry...  as one might
have guessed, from all this 8-dimensional stuff.  One can, in fact,
describe the product of octonions in these terms.

So now let's dig in a bit deeper and describe how this triality
business works.  For this, unfortunately, I will need to assume some
vague familiarity with exterior algebras, Clifford algebras, and their 
relation to the spin group.  But we will have a fair amount of fun
getting our hands on a 24-dimensional beast called the Chevalley algebra,
which contains the vector and spinor representations of Spin(8)!

Start with an 8-dimensional \emph{complex} vector space V with a
nondegenerate symmetric bilinear form on it.  We can think of V as the
representation of SO(8), hence Spin(8), where now I've switched
notation and write SO(8) to mean SO(8,C), and Spin(8) to mean
Spin(8,C).  We can split V into two 4-dimensional subspaces V+ and V-
such that <v,w> = 0 whenever v and w are either both in V+, or both in
V-.  Let Cliff be the Clifford algebra over V.  Note that as a vector
space, there is a natural identification of Cliff with

\Lambda  V+ x \Lambda  V-

where \Lambda  means "exterior algebra" and x means "tensor product".
(If you are physicist you may enjoy following Dirac and thinking of
\Lambda  V+ as a Fock space for "holes", and \Lambda  V- as a Fock space
for "particles".  If you don't enjoy this, ignore it!  We will now
to proceed to fill all the holes.)  Pick a nonzero vector v in
\Lambda ^4 V-, the top exterior power of V-.  Let S denote the subspace
of Cliff consisting of all elements of the form uv with u in Cliff.
Note that Cliff and S are representations of Cliff by left
multiplication, and therefore are representations of Spin(8) - 
because Spin(8) sits inside Cliff.  (This is a standard way to get
ones hands on the spin groups.)

Note that \Lambda  V+ and \Lambda  V- both have dimension 2^4 = 16.  We
can think of both of these as subspace of Cliff; for example, we can
think of the vector u in \Lambda  V+ as the vector 1 x u in Cliff.  Note
that uv = 0 when u is in \Lambda  V+.  (For physicists: since the sea of
holes is filled, you can't create another.)  Thus S consists of
vectors of the form uv where u lies in \Lambda  V-, and if you think a
bit, it follows that S is 16-dimensional.

So we have our hands on a 16-dimensional representation of Spin(8),
namely S.  However, we can split it into two 8-dimensional
representations, the left- and right-handed spinor representations, 
as follows.   Let 

\Lambda ^{even} V- 

denote the part of the exterior algebra consisting of stuff with
even degree, and 

\Lambda ^{odd} V- 

the part with odd degree.  Then we can write S = S+ + S-, where
+ means "direct sum" and


$$

S+ = (\Lambda ^{even} V-)v ,  S- = (\Lambda ^{odd} V-)v.
$$
    

Now, since any element of Cliff that's in Spin(8) has even degree
in Cliff, and since even times even is even, while even times odd
is odd, it follows that as a representation of Spin(8), S splits
into S+ and S-, which we call the left-handed and right-handed spinors,
respectively.  (Actually I don't know which one is called which, but
being left-handed myself I think the positive one should obviously be
called the left-handed one.)  

Note, by the way, that everything so far works quite generally for
Spin(n) when n is even, and it's only in the last paragraph that I
used the fact that n was even.  I certainly haven't done anything
weird using the fact that n is 8.  So as a bonus we're learning some
quite general stuff about spinors.

Now let's do something weird using the fact that n is 8.  We've got
these three 8-dimensional representations of Spin(8) on our hands,
namely V, S+, and S-.  How do they relate?  Recall that S+ + S- = S
is a representation of Cliff, and since V sits inside Cliff as the elements
of degree 1, we have for any a in V,

ab is in S- if b is in S+

and

ab is in S+ if a is in S-

If we are in the mood, this might tempt us to lump V, S+, and S- together to
form a 24-dimensional algebra!  Let's call this the Chevalley algebra
and write 

Chev = V + S+ + S-

Of course, we need to figure out how to multiply any two guys in Chev.
We define the product of any two guys in V to be zero, and ditto for
S+ or S-.  But we can find an interesting way to multiply a guy in S+
by a guy in S- to get a guy in V.  I think the vector representation
always sits inside the tensor product of the left- and right-handed
spinor representations when space is even-dimensional, and that
this is what we're looking for.  But explicitly, here's what we do 
in this case.  There is a kind of * operation on Cliff given 
by 

(abc...def)* = fed...cba

where a,b,c,...,d,e,f lie in V.  This lets us define a symmetric
bilinear form on S by

<x,y> v = x* y

Together with the symmetric bilinear form we started with on V,
this gives us a symmetric bilinear form on all of Chev, defining
<a,b> to be 0 if a is in V and b is in S+ or S-.  This bilinear form
on Chev turns out to be nondegenerate, and <a,b> = 0 whenever a and
b lie in different ones of three summands of Chev.  

So now Chev has a nondegenerate symmetric bilinear form it.  This lets
us define a cubic form on Chev!  Say we have (a,b,c) in V + S+ + S- =
Chev.  Then we define our cubic form F by

F(a,b,c) = <ab,c>

using the fact that we already know how to multiply a guy in V with
a guy in S+, and get a guy in S-.  

You probably know - if you've survived this far! - that from a 
quadratic form you can get a symmetric bilinear form by "polarization".  
Well, similarly, we can get a symmetric trilinear form f on Chev by
polarizing F.  Explicitly, for any u1,u2,u3 in Chev, we have


\begin{verbatim}

f(u1,u2,u3) =   F(u1 + u2 + u3) - F(u1 + u2) - F(u2 + u3) - F(u1 + u3)
              + F(u1) + F(u2) + F(u3).
\end{verbatim}
    

Then, since we have a nondegenerate symmetric bilinear form on Chev, we
can turn f into a product on Chev, by setting

<u1 u2, u3> = f(u1,u2,u3).

The assiduous reader can check that this product on Chev agrees
with the product we had partially defined so far; the only new thing
it does is define the product of a guy in S+ and a guy in S-, obtaining
something in V.  This product turns out to be commutative, but not
associative.  

Now, if I were really gung-ho about describing triality, I would
describe how the group of permutations of 3 letters, S_3, acts as
automorphisms of Chev in a way that lets one scramble the summands
V, S+, and S- at will.  In fact, S_3 acts as automorphisms of Spin(8)
in a way that gives rise to this action on Chev.  But right now I'm 
running out of steam, so I think I'll just say how to get the octonions
out of the Chevalley algebra!  

It's simple: pick a vector v in V with <v,v> = 1, and a vector 
s+ in S+ with <s+,s+> = 1.  Then s- = vs+ is a vector in S- with
<s-,s-> = 1.  We now turn V into the octonions as follows.
Given v and w in V, define their octonion product v*w to be

v*w = (v s-) (w s+)

where the product on the right hand side is the product in the Chevalley 
algebra.  In other words: take v and turn it into something in S+
by forming v s-.  Take w and turn it into something in S- by forming
w s+.  The product of these is then something in V.  In short, we
form the octonions from the three 8-dimensional representations of Spin(8)
by a kind of ring-around-the-rosie process using triality!

Note: what we just obtained was a \emph{complex} 8-dimensional algebra, 
which is the complexification of the octonions.  Using the fact
that the vector representation of SO(8,C) on C^8 contains the
vector representation of SO(8,R) on R^8 as a "real part", we should
be able to get the octonions themselves.

One can work out the details following the book of Fulton and Harris, and
the references therein.  I should add that they do a lot more fun stuff
involving the exceptional Lie groups and the 27-dimensional exceptional
Jordan algebra... all of this "exceptional" stuff seems to form a unified
whole!  There is a lot more fun stuff along these lines in 

3) Ian R. Porteous, Topological Geometry, Cambridge U. Press, Cambridge,
1981.

In particular, to correct a widespread misimpression, it's worth noting
that there are a lot of nonassociative division algebras over the reals
besides the octonions; Porteous describes one of dimension 4 in his
book.  However, all division algebras over R are of dimension 1,2,4, or
8.  Also, all normed division algebras over R are the reals, complexes,
quaternions, or octonions, and these are also all the alternative
division algebras over R, as well... where an "alternative" algebra is
one for which any two elements generate an associative algebra.  Nota
bene: here a division algebra is one such that for all nonzero x, the
map y \to  xy is invertible.  In the finite-dimensional case, this implies
that every element has a left and right inverse.  If assume
associativity, the converse is true, but in the nonassociative case it
ain't.  Whew!  Nonassociative algebras are tricky, if you're used to
associative ones, so you're interested, you might try:

4) R. D. Schafer, An Introduction to Non-Associative Algebras,
Dover, New York, 1995.

In addition to the people listed in "<A HREF = "week59.html">week59</A>", I should thank Dan
Asimov, Michael Kinyon, Frank Smith, and Dave Rusin for help with this post.
I also thank Doug Merritt for reminding me about the following nice
book on quaternions, octonions, and all sorts of similar algebras:

5) I. L. Kantor and A. S. Solodovnikov, Hypercomplex Numbers -- an
Elementary Introduction to Algebras, Springer-Verlag, Berlin, 1989; translation of "Giperkompleksnye chisla", Moscow, 1973.
  
Back in the old days when there weren't too many algebras around 
besides the reals, complexes and quaternions, people called algebras
"hypercomplex numbers".
\par\noindent\rule{\textwidth}{0.4pt}

% </A>
% </A>
% </A>
