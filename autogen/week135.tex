
% </A>
% </A>
% </A>
\week{July 31, 1999}


Well, darn it, now I'm too busy running around to conferences to write
This Week's Finds!  First I went to Vancouver, then to Santa Barbara, 
and for almost a month now I've been in Portugal, bouncing between 
Lisbon and Coimbra.  But let me try to catch up.... 

From June 16th to 19th, Steve Savitt and Steve Weinstein of the 
University of British Columbia held a workshop designed to get 
philosophers and physicists talking about the conceptual problems 
of quantum gravity:

1) Toward a New Understanding of Space, Time and Matter, workshop
home page at <A HREF = "http://axion.physics.ubc.ca/Workshop/">http://axion.physics.ubc.ca/Workshop/</A>

After a day of lectures by Chris Isham, John Earman, Lee Smolin and
myself, we spent the rest of the workshop sitting around in a big
room with a beautiful view of Vancouver Bay, discussing various issues 
in a fairly organized way.   For example, Chris Isham led a discussion 
on "What is a quantum theory?" in which he got people to question the 
assumptions underlying quantum physics, and Simon Saunders led one on 
"Quantum gravity: physics, metaphysics or mathematics?" in which we 
pondered the scientific and sociological implications of the fact that 
work on quantum gravity is motivated more by desire for consistency, 
clarity and mathematical elegance than the need to fit new experimental 
data.   

It's pretty clear that understanding quantum gravity will make us rethink 
some fundamental concepts - the question is, which ones?  By the end of 
the conference, almost every basic belief or concept relevant to physics 
had been held up for careful scrutiny and found questionable.  Space, 
time, causality, the real numbers, set theory - you name it!  It was a 
bit unnerving - but it's good to do this sort of thing now and then, to 
prevent hardening of the mental arteries, and it's especially fun to do 
it with a big bunch of physicists and philosophers.  However, I must 
admit that I left wanting nothing more than to do lots of grungy 
calculations in order to bring myself back down to earth - relatively
speaking, of course.

I particularly enjoyed Chris Isham's talk about topos theory because 
it helped me understand one way that topos theory could be applied to 
quantum theory.  I've tended to regard topoi as "too classical" for 
quantum theory, because while the internal logic of a topos is
intuitionistic (the principle of exclude middle may fail), it's
still not very quantum.   For example, in a topos the operation 
"and" still distributes over "or", and vice versa, while failure 
of this sort of distributivity is a hallmark of quantum logic.  If
you don't know what I mean, try these books, in rough order of
increasing difficulty: 

2) David W. Cohen, An Introduction to Hilbert Space and Quantum
Logic, Springer-Verlag, New York, 1989.

3) C. Piron, Foundations of Quantum Physics, W. A. Benjamin, 
Reading, Massachusetts, 1976.

4) C. A. Hooker, editor, The Logico-algebraic Approach to Quantum 
Mechanics, two volumes, D. Reidel, Boston, 1975-1979.  

Perhaps even more importantly, topoi are Cartesian!  What does this
mean?  Well, it means that we can define a "product" of any two 
objects in a topos.  That is, given objects a and b, there's an 
object a x b equipped with morphisms

p: a x b \to  a

and 

q: a x b \to  b

called "projections", satisfying the following property: given 
morphisms from some object c into a and b, say

f: c \to  a

and

g: c \to  b

there's a unique morphism f x g: c \to   a x b such that if we 
follow it by p we get f, and if we follow it by q we get g.  This
is just an abstraction of the properties of the usual Cartesian
product of sets, which is why we call a category "Cartesian" if
any pair of objects has a product.  

Now, it's a fun exercise to check that in a Cartesian
category, every object has a morphism 

\Delta : a \to  a x a

called the "diagonal", which when composed with either of the two 
projections from a x a to a gives the identity.  For example, in 
the topos of sets, the diagonal morphism is given by

\Delta (x) = (x,x)

We can think of the diagonal morphism as allowing "duplication 
of information".   This is not generally possible in quantum 
mechanics:

5) William Wooters and Wocjciech Zurek, A single quantum cannot
be cloned, Nature 299 (1982), 802-803.

The reason is that in the category of Hilbert spaces, the
tensor product is not a product in the above sense!  In 
particular, given a Hilbert space H, there isn't a natural 
diagonal operator 

\Delta : H \to  H tensor H 

and there aren't even natural projection operators from H tensor H 
to H.  As pointed out to me by James Dolan, this non-Cartesianness 
of the tensor product gives quantum theory much of its special flavor.  
Besides making it impossible to "clone a quantum", it's closely 
related to how quantum theory violates Bell's inequality, because 
it means we can't think of an arbitrary state of a two-part quantum 
system as built by tensoring states of both parts.  


 Anyway, this has made me feel for a while that topos theory isn't
sufficiently "quantum" to be useful in understanding the
peculiar special features of quantum physics.  However, after Isham and
I gave our talks, someone pointed out to me that one can think of a
topological quantum field theory as a presheaf of Hilbert spaces over
the category nCob whose morphisms are n-dimensional cobordisms.  Now,
presheaves over any category form a topos, so this means we should be
able to think of a topological quantum field theory as a "Hilbert
space object" in the topos of presheaves over nCob.  From this
point of view, the peculiar "quantumness" of topological
quantum field theory comes from it being a Hilbert space object, while
its peculiar "variability" - i.e., the fact that it assigns a
different Hilbert space to each (n-1)-dimensional manifold representing
space - comes from the fact that it's an object in a topos.  (Topoi are
known for being very good at handling things like "variable
sets".)  I'm not sure how useful this is, but it's worth keeping in
mind.

While I'm talking about quantum logic, let me raise a puzzle 
concerning the Kochen-Specker theorem.  Remember what this says:
if you have a Hilbert space H with dimension but more than 2, 
there's no map F from self-adjoint operators on H to real numbers 
with the following properties:

a) For any self-adjoint operator A, F(A) lies in the spectrum of A,

and

b) For any continuous f: R \to  R, f(F(A)) = F(f(A)).

This means there's no sensible consistent way of thinking of all
observables as simultaneously having values in a quantum system!

Okay, the puzzle is: what happens if the dimension of H equals 2?
I don't actually know the answer, so I'd be glad to hear it if
someone can figure it out!

By the way, I once wanted to do an undergraduate research project
on mathematical physics with Kochen.  He asked me if I knew the 
spectral theorem, I said "no", and he said that in that case there
was no point in me trying to work with him.  I spent the next summer
reading Reed and Simon's book on Functional Analysis and learning 
lots of different versions of the spectral theorem.  I shudder to 
think that perhaps this is why I spent years studying analysis
before eventually drifting towards topology and algebra.  But no: 
now that I think about it, I was already interested in analysis at 
the time, since I'd had a wonderful real analysis class with Robin 
Graham.   

Okay, now let me say a bit about the next conference I went to.  From
June 22nd to 26th there was a conference on "Strong Gravitational 
Fields" at the Institute for Theoretical Physics at U. C. Santa Barbara.
This finished up a wonderful semester-long program by Abhay Ashtekar, 
Gary Horowitz and Jim Isenberg:

6) Classical and Quantum Physics of Strong Gravitational Fields,
program homepage with transparencies and audio files of talks at
<A HREF = "http://www.itp.ucsb.edu/~patrick/gravity99/">http://www.itp.ucsb.edu/~patrick/gravity99/</A>

Like the whole program, the conference covered a wide range of
topics related to gravity: string theory and loop quantum gravity,
observational and computational black hole physics, and \gamma  ray
bursters.  I can't summarize all this stuff; since I usually spend 
a lot of talking about quantum gravity here, let me say a bit about 
other things instead.  

John Friedman gave an interesting talk on gravitational waves from
unstable neutron stars.  When a pulsar is young, like about 5000
years old, it typically rotates about its axis once every 16 
milliseconds or so.  A good example is N157B, a pulsar in the Large
Magellanic Cloud.  Using the current spindown rate one can extrapolate
and guess that pulsars have about a 5-millisecond period at birth.  
It's interesting to think about what makes a newly formed neutron
star slow down.  Theorists have recently come up with a new possible
mechanism: namely, a new sort of gravitational-wave-driven instability
of relativistic stars that could force newly formed slow down to a 
10-millisecond period.  It's very clever: the basic idea is that if
a star is rotating very fast, a rotational mode that rotates slower
than the star will gravitationally radiate \emph{positive} angular momentum,
but such modes carry \emph{negative} angular momentum, since they rotate
slower than the star.  If you think about it carefully, you'll see
this means that gravitational radiation should tend to amplify such 
modes!  I asked for a lowbrow analog of this mechanism and it turns 
out that a similar sort of thing is at work in the formation of water 
waves by the wind - with linear momentum taking the place of angular 
momentum.  Anyway, it's not clear that this process really ever has 
a chance to happen, because it only works when the neutron star is 
not too hot and not too cold, but it's pretty cool.  

Richard Price gave a nice talk on computer simulation of black 
hole collisions.  Quantitatively understanding the gravitational
radiation emitted in black hole and neutron star collisions is a 
big business these days - it's one of the NSF's "grand challenge" 
problems.  The reason is that folks are spending a lot of money 
building gravitational wave detectors like LIGO: 

7) LIGO project home page, <A HREF = "http://www.ligo.caltech.edu/">http://www.ligo.caltech.edu/</A>

8) Other gravitational wave detection projects, 
<A HREF = "http://www.ligo.caltech.edu/LIGO_web/other_gw/gw_projects.html">http://www.ligo.caltech.edu/LIGO_web/other_gw/gw_projects.html</A>

and they need to know exactly what to look for.   Now, head-on
collisions are the easiest to understand, since one can simplify
the calculation using axial symmetry.  Unfortunately, it's not 
very likely that two black holes are going to crash into each 
other head-on.  One really wants to understand what happens when
two black holes spiral into each other.  There are two extreme 
cases: the case of black holes of equal mass, and the case of 
a very light black hole of mass falling into a heavy one.

The latter case is 95% understood, since we can think of the 
light black hole as a "test particle" - ignoring its effect 
on the heavy one.  The light one slowly spirals into the
heavy one until it reaches the innermost stable orbit, and then
falls in.   We can use the theory of a relativistic test particle
falling into a black hole to understand the early stages of this
process, and use black hole perturbation theory to study the
"ringdown" of the resulting single black hole in the late stages
of the process.  (By "ringdown" I mean the process whereby an
oscillating black hole settles down while emitting gravitational 
radiation.)  Even the intermediate stages are manageable, because
the radiation of the small black hole doesn't have much effect on 
the big one.

By contrast, the case of two black holes of equal mass is less
well understood.  We can treat the early stages, where relativistic
effects are small, using a post-Newtonian approximation, and
again we can treat the late stages using black hole perturbation
theory.  But things get complicated in the intermediate stage,
because the radiation of each hole greatly effects the other, 
and there is no real concept of "innermost stable orbit" in this
case.  To make matters worse, the intermediate stage of the process
is exactly the one we really want to understand, because this is
probably when most of the gravitational waves are emitted!

People have spent a lot of work trying to understand black hole
collisions through number-crunching computer calculation, but 
it's not easy: when you get down to brass tacks, general relativity 
consists of some truly scary nonlinear partial differential equations.
Current work is bedeviled by numerical instability and also the 
problem of simulating enough of a region of spacetime to understand
the gravitational radiation being emitted.  Fans of mathematical
physics will also realize that gauge-fixing is a major problem.
There is a lot of interest in simplifying the calculations through
"black hole excision": anything going on inside the event horizon 
can't affect what happens outside, so if one can get the computer
to \emph{find} the horizon, one can forget about simulating what's going
on inside!  But nobody is very good at doing this yet... even using
the simpler concept of "apparent horizon", which can be defined
locally.  So there is some serious work left to be done! 

(For more details on both these talks, go to the conference website
and look at the transparencies.)

I also had some interesting talks with people about black hole entropy, 
some of which concerned a new paper by Steve Carlip.  I'm not really 
able to do justice to the details, but it seems important....

9) Steve Carlip, Entropy from conformal field theory at Killing 
horizons, preprint available at <A HREF = "http://xxx.lanl.gov/abs/gr-qc/9906126">gr-qc/9906126</A>.

Let me just quote the abstract:
 
\begin{quote}
     On a manifold with boundary, the constraint algebra of general 
     relativity may acquire a central extension, which can be computed 
     using covariant phase space techniques.  When the boundary is a 
     (local) Killing horizon, a natural set of boundary conditions 
     leads to a Virasoro subalgebra with a calculable central charge. 
     Conformal field theory methods may then be used to determine the
     density of states at the boundary.  I consider a number of cases -
     black holes, Rindler space, de Sitter space, Taub-NUT and Taub-Bolt 
     spaces, and dilaton gravity - and show that the resulting density 
     of states yields the expected Bekenstein-Hawking entropy.  The 
     statistical mechanics of black hole entropy may thus be fixed 
     by symmetry arguments, independent of details of quantum gravity.
\end{quote}
There was also a lot of talk about "isolated horizons", a concept
that plays a fundamental role in certain treatments of black holes
in loop quantum gravity:

10) Abhay Ashtekar, Christopher Beetle, and Stephen Fairhurst, 
Mechanics of isolated horizons, preprint available as <A HREF = "http://xxx.lanl.gov/abs/gr-qc/9907068">gr-qc/9907068</A>.

11) Jerzy Lewandowski, Spacetimes admitting isolated horizons, 
preprint available as <A HREF = "http://xxx.lanl.gov/abs/gr-qc/9907058">gr-qc/9907058</A>.

For more on isolated horizons try the references in "<A HREF = "week128.html">week128</A>".  
 
Finally, on a completely different note, I've recently seen some
new papers related to the McKay correspondence - see "<A HREF = "week65.html">week65</A>" if
you don't know what \emph{that} is!  I haven't read them yet, but I 
just want to remind myself that I should, so I'll list them here:

12) John McKay, Semi-affine Coxeter-Dynkin graphs and $G \subseteq
SU_2(C)$, preprint available as <A HREF = "http://xxx.lanl.gov/abs/math.QA/9907089">math.QA/9907089</A>.

13) Igor Frenkel, Naihuan Jing and Weiqiang Wang, Vertex 
representations via finite groups and the McKay correspondence,
preprint available as <A HREF = "http://xxx.lanl.gov/abs/math.QA/9907166">math.QA/9907166</A>.

Quantum vertex
representations via finite groups and the McKay correspondence,
preprint available as <A HREF = "http://xxx.lanl.gov/abs/math.QA/9907175">math.QA/9907175</A>.

Next time I want to talk about the big category theory conference
in honor of MacLane's 90th birthday!  Then I'll be pretty much
caught up on the conferences....


\par\noindent\rule{\textwidth}{0.4pt}
Robert Israel's answer to my puzzle about the Kochen-Specker theorem:


\begin{verbatim}

It's not true in dimension 2.  Note that for a self-adjoint 
2x2 matrix A, any f(A) is of the form a A + b I for some 
real scalars a and b (this is easy to see if you diagonalize 
A).  The self-adjoint matrices that are not multiples of I 
split into equivalence classes, where A and B are equivalent 
if B = a A + b I for some scalars a, b (a <> 0).  Pick a 
representative A from each equivalence class, choose F(A) 
as one of the eigenvalues of A, and then F(a A + b I) = 
a F(A) + b.  Of course, F(b I) = b.  Then F satisfies the 
two conditions.

The reason this doesn't work in higher dimensions is that 
in higher dimensions you can have two self-adjoint matrices 
A and B which don't commute,  F(A) = G(B) for some functions 
F and G, and F(A) is not a multiple of I.

Robert Israel                               israel@math.ubc.ca
Department of Mathematics       http://www.math.ubc.ca/~israel 
University of British Columbia            
Vancouver, BC, Canada V6T 1Z2

\end{verbatim}
    



 \par\noindent\rule{\textwidth}{0.4pt}

% </A>
% </A>
% </A>
