
% </A>
% </A>
% </A>
\week{October 26, 1999 }


How can you resist a book with a 
title like "Inconsistent Mathematics"?

1) Chris Mortensen, Inconsistent Mathematics, Kluwer, Dordrecht, 1995.

Ever since Goedel showed that all sufficiently strong systems formulated
using the predicate calculus must either be inconsistent or incomplete,
most people have chosen what they perceive as the lesser of two evils:
accepting incompleteness to save mathematics from inconsistency.  But
what about the other option?  

This book begins with the startling sentence: "The following idea
has recently been gaining support: that the world is or might be
inconsistent." As we reel in shock, Mortensen continues:

\begin{quote}
     Let us consider set theory first.  The most natural set theory to
     adopt is undoubtedly one which has unrestricted set abstraction
     (also known as naive comprehension).  This is the natural principle
     which declares that to every property there is a unique set of
     things having the property. But, as Russell showed, this leads
     rapidly to the contradiction that the the Russell set [the set of
     all sets that do not contain themselves as a member] both is and is
     not a member of itself.  The overwhelming  majority of logicians
     took the view that this contradiction required a weakening of
     unrestricted abstraction in order to ensure a consistent set
     theory, which was in turn seen as necessary to provide a consistent
     foundation for mathematics.  But all ensuing attempts at weakening
     set abstraction proved to be in various ways ad hoc.  Da Costa and
     Routley both suggested instead that the Russell set might be dealt
     with more naturally in an inconsistent but nontrivial set theory
     (where triviality means that every sentence is provable). 
\end{quote}
    
An inconsistent but nontrivial logical system is called 
\emph{paraconsistent}.  
But it's not so easy to create such systems.  To keep an inconsistency
from infecting the whole system and making it trivial, we need to drop
the rule of classical logic which says that "A and not(A) implies B"
for all propositions A and B.  Unfortunately, this rule is built into the 
propositional calculus from the very start!  

So, we need to revise the propositional calculus.  

One way to do it is to abandon "material implication" - the form of
implication where you can calculate the truth value of "P implies Q"
from those of P and Q using the following truth table:
\begin{verbatim}
        P  |  Q  |  P implies Q
      --------------------------
        T  |  T  |     T
        T  |  F  |     F
        F  |  T  |     T
        F  |  F  |     T
\end{verbatim}
    
With material implication, a false statement implies \emph{every} statement, 
so any inconsistency is fatal.  But in real life, if we discover we have
one inconsistent belief, we don't conclude we can fly and go jump off a
building!  Material implication is really just our best attempt to define
implication using truth tables with 2 truth values: true and false.  So
it's not surprising that logicians have investigated other forms of
implication.  

One obvious approach is to use more truth values, like "true",
"false", and "don't know".  There's a long history
of work on such multi-valued logics.

Another approach, initiated by Anderson and Belnap, is called
"relevance logic".  In relevance logic, "P implies
Q" can only be true if there is a conceptual connection between P
and Q.  So if B has nothing to do with A, we don't get "A and
not(A) implies B".

This book describes a logical system called "RQ" - relevance
logic with quantifiers.  It also describes a system called
"R#", which is a version of the Peano axioms of arithmetic
based on RQ instead of the usual predicate calculus.  Following the work
of Robert Meyer, it proves that R# is nontrivial in the sense described
above.  Moreover, this proof can be carried out R# itself!  However, you
can carry out the proof of Goedel's 2nd incompleteness theorem in R#, so
R# cannot prove itself consistent.

To paraphrase Mortensen: "But this is not really a puzzle.  The
explanation is that relevant and other paraconsistent logics turn on
making a distinction between inconsistency and triviality, the former
being weaker than the latter; whereas classical logical cannot make this
distinction.  For what the present author's intuitions are worth, these
do seem to be different concpets.  Thus for R#, consistency cannot be
proved by finitistic means by Goedel's second theorem, whereas
nontriviality can be shown.  Since Peano arithmetic collapses this
distinction, both  kinds of consistency are infected by the same
unprovability."

Mortensen also mentions another approach to get rid of "A and
not(A) implies B" without getting rid of material implication.
This is to get rid of the rule that "A and not(A)" is false!
He calls this "Brazilian logic".  Presumably this is not
because your average Brazilian thinks this way, but because the inventor
of this approach, Da Costa, is Brazilian.

Brazilian logic sounds very bizarre at first, but in fact it's just the
dual of intuitionistic logic, where you drop the rule that "A or
not(A)" is true.  Intuitionistic logic is nicely modeled by open
sets in a topological space: "and" is intersection,
"or" is union, and "not" is the interior of the
complement.  Similarly, Brazilian logic is modeled by closed sets.  In
intuitionistic logic we allow a slight gap between A and not(A); in
Brazilian logic we allow a slight overlap.

In short, this book is full of fascinating stuff.  Lots of passages are
downright amusing at first, like this: 

\begin{quote}
     [...] there have been calls recently for inconsistent calculus,
     appealing to the history of calculus in which inconsistent claims
     abound, especially about infinitesimals (Newton, Leibniz,
     Bernoulli, l'Hospital, even Cauchy).  However, inconsistent
     calculus has resisted development.

\end{quote}
    
But you always have to remember that the author is interested in
theories which, though inconsistent, are still paraconsistent.  And I
think he really makes a good case for his claim that inconsistent
mathematics is worth studying - even if our ultimate goal is to \emph{avoid}
inconsistency! 

Okay, now let me switch gears drastically and say a bit about "exotic
spheres" - smooth manifolds that are homeomorphic but not diffeomorphic
to the n-sphere with its usual smooth structure.  People on
sci.physics.research have been talking about this stuff lately, so it
seems like a good time for a mini-essay on the subject.  Also, my
colleague Fred Wilhelm works on the geometry of exotic spheres, and he
just gave a talk on it here at U. C. Riverside, so I should pass along
some of his wisdom while I still remember it.  

First, recall the "Hopf bundle".  It's easy to describe
starting with the complex numbers.  The unit vectors in C^2 form the
sphere S^{3}.  The unit complex numbers form a group under multiplication.
As a manifold this is just the circle S^{1}, but as a group it's better
known as U(1).  You can multiply a unit vector by a unit complex number
and get a new unit vector, so S^{1} acts on S^{3}.  The quotient space is
the complex projective space CP^{1}, 
which is just the sphere S^{2}.  So
what we've got here is fiber bundle:
S^{1}\to  S^{3} \to  S^{2} = CP^{1}

with fiber S^{1}, total space S^{3} and base space 
S^{2}.  This is the Hopf
bundle.  It's famous because the map from the total space to the base
was the first example of a topologically nontrivial map from a sphere to
a sphere of lower dimension.  In the lingo of homotopy theory, we say
it's the generator of the group \pi _{3}(S^{2}).  

Now in "<A HREF = "week106.html">week106</A>" I talked about how we can mimic this construction by
replacing the complex numbers with any other division algebra.  If we
use the real numbers we get a fiber bundle

S^{0} \to  S^{1} \to  RP^{1} = S^{1}    

where S^{0} is the group of unit real numbers, better known as Z/2.  
This 
bundle looks like the edge of a Moebius strip.  If we use the quaternions 
we get a more interesting fiber bundle:

S^{3} \to  S^{7} \to  HP^{1} = S^{4}

where S^{3} is the group of unit quaternions, better known as SU(2).  
We can even do something like this with the octonions, and we get a fiber
bundle

S^{7}\to  S^{15} \to  OP^{1} = S^{8}

but now S^{7}, the unit octonions, doesn't form a group - because the
octonions aren't associative.  

Anyway, it's the quaternionic version of the Hopf bundle that serves as
the inspiration for Milnor's construction of exotic 7-spheres.  These
exotic 7-spheres are actually total spaces of \emph{other} bundles with fiber
S^{3} and base space S^{4}.  The easiest way to get your 
hands on these
bundles is to take S^{4}, chop it in half along the equator, put a 
trivial
S^{3}-bundle over each hemisphere, and then glue these together.  
To glue
these bundles together we need a way to attach the fibers over each
point x of the equator.  In other words, for each point x in the equator
of S^{4} we need a map 

f_{x}: S^{3} \to  S^{3}

which should vary smoothly with x.  But the equator of S^{4} 
is just S^{3}, and
S^{3} is a group - the unit quaternions - so we can take

f_{x}(y) = x^{n} y x^{m}

for any pair of integers (n,m).  

This gives us a bunch of S^{3}-bundles over S^{4}.
The total space X(n,m) of any one of these bundles is obviously a smooth
7-dimensional manifold.  But when is it homeomorphic to the 7-sphere?
And when is it \emph{diffeomorphic} to the 7-sphere with its usual
smooth structure?

Well, first we use some Morse theory.  You can learn a lot about the
topology of a smooth manifold if you have a "Morse function" on the 
manifold: a smooth real-valued function all of whose critical points 
are nondegenerate.  If you don't believe me, read this book:

2) John Milnor, Morse Theory, Princeton U. Press, Princeton, 1960.

When n + m = 1 there's a Morse function on X(n,m) with only two critical 
points - a maximum and a minimum.  This implies that X(n,m) is 
homeomorphic to a sphere!

Once we know that X(n,m) is homeomorphic to S^{7}, we have to decide 
when it's diffeomorphic to S^{7} with its usual smooth structure.  
This is the hard part.  Notice that X(n,m) is the unit sphere bundle of
a vector bundle over S^{4} whose fiber is the quaternions.  We can
understand a bunch about X(n,m) using the characteristic classes
of this vector bundle.  In particular, we can compute the Euler 
number and the Pontrjagin number of this vector bundle.  Using the
Euler number we can show that X(n,m) is homeomorphic to a sphere
\emph{only} if n + m = 1 - you can't really do this using Morse theory.
But more importantly, using the Pontrjagin number, we can show that 
in this case X(n,m) is diffeomorphic to S^{7} with its usual smooth 
structure if and only if (n - m)^{2} = 1 mod 7.  Otherwise 
it's "exotic".

For the details of the above argument you can try the following book:

3) B. A. Dubrovin, A. T. Fomenko and S. P. Novikov, Modern Geometry -
Methods and Applications, Part III: Introduction to Homology Theory,
Springer-Verlag Graduate Texts, number 125, Springer, New York, 1990.

or the original paper:

4) John Milnor, On manifolds homeomorphic to the 7-sphere, Ann.
Math 64 (1956), 399-405.

Now, with quite a bit more work, you can show that smooth structures on
the n-sphere form an group under connected sum - the operation of chopping 
out a small hole in two spheres and gluing them together - and you can 
show that this group is Z/28 for n = 7.  This means that if we consider 
two smooth structures on the 7-sphere the same when they're related by 
an \emph{orientation-preserving} diffeomorphism, we get exactly 28 kinds.  
Unfortunately we don't get all of them by the above explicit construction.
For more details, see:

5) M. Kervaire and J. Milnor, Groups of homotopy spheres I, Ann. Math.
77 (1963), 504-537.

By the way, part II of the above paper doesn't exist!  Instead, you 
should read this:

6) J. Levine, Lectures on groups of homotopy spheres, in Algebraic and
Geometric Topology, Springer Lecture Notes in Mathematics number
1126, Springer, Berlin, 1985, pp. 62-95.

Anyway, if you're wondering why I'm talking so much about exotic 7-spheres,
instead of lower-dimensional examples that are easier to visualize, check 
out this table of groups of smooth structures on the n-sphere: 

\begin{verbatim}
n      group of smooth structures on the n-sphere

0                  1
1                  1
2                  1
3                  1
4                  ?
5                  1
6                  1
7                  Z/28 
8                  Z/2 
9                  Z/2 x Z/2 x Z/2
10                 Z/6
11                 Z/992
12                 1
13                 Z/3
14                 Z/2
15                 Z/8128 x Z/2
16                 Z/2
17                 Z/2 x Z/2 x Z/2 x Z/2 
18                 Z/8 x Z/2
\end{verbatim}
    
Dimension 7 is the simplest interesting case - except perhaps for
dimension 4, where the answer is unknown!   The "smooth Poincare
conjecture" says that there's only one smooth structure on the
4-sphere, but this remains a conjecture....

As you can see, there are lots of exotic 11-spheres.  In fact, this is
relevant to string theory!  You can get an n-sphere with any possible 
smooth structure by taking two n-dimensional balls and gluing them together 
along their boundary using some orientation-preserving diffeomorphism

f: S^{n-1} \to  S^{n-1}.

Orientation-preserving diffeomorphisms like this form a group called
Diff_{+}(S^{n-1}).  Using the above trick, it turns out
that the group of smooth structures on the n-sphere is isomorphic to the
group of \emph{connected components} of Diff_{+}(S^{n-1}).
So the existence of exotic 11-spheres means that there are lots of
"exotic diffeomorphisms" of the 10-sphere!

Now, string theory lives in 10 dimensions, and one wants certain
quantities to be invariant under orientation-preserving diffeomorphisms
of spacetime - otherwise you say the theory has "gravitational
anomalies".  First you have to check this for "small
diffeomorphisms" of spacetime, that is, those connected to the
identity map by a continuous path.  But then you have to check it for
"large diffeomorphisms" - those living in different connected
components of the diffeomorphism group.  When spacetime is a 10-sphere,
this means you need to check diffeomorphism invariance for all 991
components of Diff_{+}(S^{n-1}) besides the component
containing the identity.  These components correspond to exotic
11-spheres!

Witten did this in the following paper:

7) Edward Witten, Global gravitational anomalies, Commun. Math. Phys.
100 (1985), 197-229.


This may be the first paper about exotic spheres in physics. 

There are other interesting things to do with an exotic sphere.  One is
to put a metric on it and look at its curvature.  The sphere with its
usual "round" metric is very symmetrical and has positive
curvature everywhere.  There are various meanings of "positive
curvature", but the round sphere has positive curvature in all
possible ways!  One kind of curvature is "sectional
curvature".  In general, it's hard to find compact manifolds other
than the sphere with its usual smooth structure that have metrics with
everywhere positive sectional curvature.  Gromoll and Meyer found an
exotic 7-sphere with a metric having \emph{nonnegative} sectional curvature:

8) Detlef Gromoll and Wolfgang Meyer, An exotic sphere with nonnegative 
sectional curvature, Ann. Math. 100 (1974), 401-406. 

The construction isn't terribly hard so let me describe it.  First,
start with the group Sp(2), consisting of 2x2 unitary quaternionic
matrices (see "<A HREF = "week64.html">week64</A>").  As
always with compact Lie groups, this has a metric that's invariant under
right and left translations, and this metric is unique up to a constant
scale factor.  The group of unit quaternions acts as metric-preserving
maps (aka "isometries") of Sp(2) in the following way: let the
quaternion q map
\begin{verbatim}
   (a b)  
   (c d)

\end{verbatim}
    
to 
$$
   (qaq^{-1}  qb)
   (qcq^{-1}  qd)
$$
    
The quotient space is an exotic 7-sphere, and it inherits a metric
with nonnegative sectional curvature.  

Now, since compact manifolds with positive sectional curvature are
tough to find, you might wonder if this exotic 7-sphere can be given 
a metric with \emph{positive} 
sectional curvature.  And the answer is: almost!
It can be given a metric having positive sectional curvature except on a
set of measure zero.  This was recently proved by Wilhelm:


9) Frederick Wilhelm, An exotic sphere with positive curvature 
almost everywhere, preprint, May 12 1999.

It's also an interesting theorem, due to Hitchin, that for any n > 0
there exist exotic spheres of dimensions 8n+1 and 8n+2 having no metric
of positive scalar curvature:

8) Nigel Hitchin, Harmonic spinors, Adv. Math. 14 (1974), 1-55.
 
So some exotic spheres are not so as "round" as you might think!   
In fact, 3 of the exotic spheres in 10 dimensions cannot be given a 
metric such that the connected component of the isometry group is 
bigger than U(1) x U(1), so these are quite "bumpy".  This follows 
from results of Reinhard Schultz, who happens to be the department 
chair here:

9) Reinhard Schultz, Circle actions on homotopy spheres bounding
plumbing manifolds, Proc. A.M.S. 36 (1972), 297-300.

There's a lot more to say about exotic spheres, but let me just
briefly mention two things.  First, there are cool connections 
between exotic spheres and higher-dimensional knot theory.  If 
you want a small taste of this stuff, try:

10) Louis Kauffman, Knots and Physics, World Scientific, Singapore,
1991.

Look in the index under "exotic spheres".   

Second, people have computed the effect of exotic 7-spheres on 
quantum gravity path integrals in 7 dimensions:

11) Kristin Schleich and Donald Witt, Exotic spaces in quantum 
gravity, Class. Quant. Grav. 16 (1999) 2447-2469, preprint available 
as <A HREF = "http://xxx.lanl.gov/abs/gr-qc/9903086">gr-qc/9903086</A>.

I'm not sure exotic spheres are \emph{really} relevant to physics, but
it would be cool, so I'm glad some people are trying to establish
connections.

Okay, that's enough for exotic spheres, at least for now!  I've got 
a few more things here that I just want to mention....

I've been learning a bit about Calabi-Yau manifolds and mirror
symmetry in string theory lately.  The basic idea is that string
theory on different spacetime manifolds can be physically equivalent.
I don't know enough to want to try to explain this stuff yet, but here 
are some place to look in case you're interested:

12) Claire Voisin, Mirror Symmetry, American Mathematical Society, 1999.

13) David A. Cox and Sheldon Katz, Mirror Symmetry and Algebraic Geometry,
American Mathematical Society, Providence, Rhode Island, 1999.

14) Shing-Tung Yau, editor, Mirror Symmetry I, American Mathematical
Society, 1998.

Brian Green and Shing-Tung Yau, editors, Mirror Symmetry II, American
Mathematical Society, 1997.

Duong H. Phong, Luc Vinet and Shing-Tung Yau, editors, Mirror Symmetry III, 
American Mathematical Society, 1999.

So far I'm mainly trying to learn really basic stuff, and for this, 
the following lectures are proving handy:

16) P. Candelas, Lectures on complex manifolds, in Superstrings '87, 
eds. L. Alvarez-Gaume et al, World Scientific, Singapore, 1988, pp. 1-88.

On a different note, the American Mathematical Society has come out
with some good-looking books on surgery theory - the process of making
new manifolds from old by cutting and pasting.  I've got these on
my reading list, so if anyone wants to buy me a Christmas present, 
here's what you should get:

17) Robert E. Gompf and Andras I Stipsicz, 4-Manifolds and Kirby Calculus,
Amderican Mathematical Society, 1999.

18) C. T. C. Wall and A. A. Ranicki, Surgery on Compact Manifolds, 
2nd edition, American Mathematical Society, 1999.

Finally, there's some cool stuff going on with operads that I haven't 
been able to keep up with.  Let me quote the abstracts:

19) Alexander A. Voronov, Homotopy Gerstenhaber algebras, preprint
available as 
<A HREF = "http://xxx.lanl.gov/abs/math.QA/9908040">math.QA/9908040</A>.

The purpose of this paper is to complete Getzler-Jones' proof of Deligne's 
Conjecture, thereby establishing an explicit relationship between the 
geometry of configurations of points in the plane and the Hochschild 
complex of an associative algebra.  More concretely, it is shown that 
the B_infty-operad, which is generated by multilinear operations known to
act on the Hochschild complex, is a quotient of a certain operad associated 
to the compactified configuration spaces.  Different notions of homotopy 
Gerstenhaber algebras are discussed: one of them is a B_infty-algebra, 
another, called a homotopy G-algebra, is a particular case of a 
B_infty-algebra, the others, a G_infty-algebra, an E^1-bar-algebra, and 
a weak G_infty-algebra, arise from the geometry of configuration spaces. 
Corrections to the paper math.QA/9602009 of Kimura, Zuckerman, and the 
author related to the use of a nonextant notion of a homotopy Gerstenhaber 
algebra are made. 

20) Maxim Kontsevich, Operads and motives in deformation quantization,
Lett. Math. Phys. 48 (1999), 35-72, preprint available as
<A HREF = "http://xxx.lanl.gov/abs/math.QA/9904055">math.QA/9904055</A>.

It became clear during last 5-6 years that the algebraic world of 
associative algebras (abelian categories, triangulated categories, etc) 
has many deep connections with the geometric world of two-dimensional 
surfaces.  One of the manifestations of this is Deligne's conjecture 
(1993) which says that on the cohomological Hochschild complex of any 
associative algebra naturally acts the operad of singular chains in 
the little discs operad.  Recently D. Tamarkin discovered that the 
operad of chains of the little discs operad is formal, i.e. it is 
homotopy equivalent to its cohomology.  From this fact and from Deligne's
conjecture follows almost immediately my formality result in deformation
quantization.  I review the situation as it looks now.  Also I conjecture 
that the motivic Galois group acts on deformation quantizations, and 
speculate on possible relations of higher-dimensional algebras and of 
motives to quantum field theories.

21) James E. McClure and Jeffrey H. Smith, A solution of Deligne's 
conjecture, preprint available as 
<A HREF = "http://xxx.lanl.gov/abs/math.QA/9910126">math.QA/9910126</A>

Deligne asked in 1993 whether the Hochschild cochain complex of an
associative ring has a natural action by the singular chains of the little
2-cubes operad. In this paper we give an affirmative answer to this question.
We also show that the topological Hochschild cohomology spectrum of an
associative ring spectrum has an action of an operad equivalent to the little
2-cubes.

\par\noindent\rule{\textwidth}{0.4pt}
My original table of groups of smooth structures on spheres had some
mistakes in it which were corrected by Linus Kramer, Marco Mackaay, Tony
Smith and Frederick Wilhelm.  In fact, the table in the book by
Dubrovin, Fomenko and Novikov differs from the table in Kervaire and
Milnor's paper!  The table above comes from Kervaire and Milnor, taking
advantage of some subsequent work in dimension 3 and also some work of
Brumfield which nailed down the groups in dimensions 9 and 17 - see below
for more information.

The paper by Kervaire and Milnor has a cool formula
for the \emph{order} of the group of smooth structures on the (4n-1)-sphere
for n > 1.  It's:

2^(2n-4) (2^(2n-1) - 1) P(4n-1) B(n) a(n) / n
where:
P(k)    is the order of the kth stable homotopy group of spheres
B(k)      is the kth Bernoulli number, in the sequence
           1/6, 1/30, 1/42, 1/30, 5/66, 691/2730, 7/6, ...
a(k)      is 1 or 2 according to whether k is even or odd.
\par\noindent\rule{\textwidth}{0.4pt}
Here are some remarks by Linus Kramer on exotic spheres in 
dimensions 9 and 17, which he posted to sci.math.research in
response to a question of mine.  Kervaire and Milnor said the group
of exotic spheres in dimension 9 was (Z/2)^{3} or Z/2 x Z/4,
and the group in dimension 17 was (Z/2)^{4} or (Z/2)^{2} 
x Z/4. Linus writes:



% parser failed at source line 652
