
% </A>
% </A>
% </A>
\week{December 14, 2005 }


This week I want to mention a couple of papers lying on the interface of
physics, topology, and higher-dimensional algebra.  But first, some
astronomy pictures... and a bit about the mathematical physicist Hamilton!

I like this photo of a jet emanating from the black hole in the 
center of galaxy M87:

<BR>
<DIV ALIGN = CENTER>
<A HREF = "http://hubblesite.org/newscenter/newsdesk/archive/releases/2005/12/image/o">
<IMG SRC = m87_jet.jpg>
% </A>
</DIV>
<BR>

1) NASA and John Biretta, M87,
<A HREF = "http://hubblesite.org/newscenter/newsdesk/archive/releases/2005/12/image/o">http://hubblesite.org/newscenter/newsdesk/archive/releases/2005/12/image/o</A>

M87 is a giant elliptical galaxy.  It's long been known as a powerful 
radio source, and now we know why: there's a supermassive black hole 
in the center, about 3 billion times the mass of our Sun.  As matter 
spirals into this huge black hole, it forms an "accretion disk", and
some gets so hot that it shoots out in a jet, as envisioned here:

<BR>
<DIV ALIGN = CENTER>
<A HREF = "http://maxim.gsfc.nasa.gov/docs/science/science.html">
<IMG SRC = accretion_disk.jpg>
% </A>
</DIV>
<BR>

2) NASA, MAXIM: Micro-Arcsecond X-ray Imaging Mission,
<A HREF = "http://maxim.gsfc.nasa.gov/docs/science/science.html">http://maxim.gsfc.nasa.gov/docs/science/science.html</A>

Accretion disks and jets are common at many different scales in our 
universe.  They're just nature's way of letting a bunch of matter fall
in under its own gravitation while losing angular momentum and energy.  
We see them when dust clouds collapse to form stars, we see them when 
black holes sucks in mass from companion stars, and they're probably 
also responsible for slow \gamma  ray bursts as huge stars collapse when 
they run out of fuel - see "<A HREF = "week204.html">week204</A>" for that story.  

But, among the biggest accretion disks and jets are those surrounding
supermassive black holes in the middle of galaxies.  These are probably
responsible for all the "active galactic nuclei" or "quasars" that we
see.  In the case of M87 the jet is enormous: 5000 light years long!  
To get a sense of the scale, look at the small white specks away from the 
jet in the next picture.  These are globular clusters: clusters containing 
between ten thousand and a million stars.  

<BR>
<DIV ALIGN = CENTER>
<A HREF = "http://antwrp.gsfc.nasa.gov/apod/ap000706.html">
<IMG SRC = m87jet_hst.jpg>
% </A>
</DIV>
<BR>

3) A jet from galaxy M87, Astronomy Picture of the Day, July 6, 2000,
<A HREF = "http://antwrp.gsfc.nasa.gov/apod/ap000706.html">http://antwrp.gsfc.nasa.gov/apod/ap000706.html</A>

The jet in M87 is so hot that it emits not just radio waves and visible 
light, but even X-rays, as seen by the Chandra X-ray telescope:

<BR>
<DIV ALIGN = CENTER>
<A HREF = "http://chandra.harvard.edu/photo/2001/0134/">
<IMG SRC = "m87_xray_radio_optical.jpg">
% </A>
</DIV>
<BR>

4) M87: Chandra sheds light on the knotty problem of the M87 jet,
<A HREF = "http://chandra.harvard.edu/photo/2001/0134/">
http://chandra.harvard.edu/photo/2001/0134/</A>

It seems the jet consists mainly of electrons moving at relativistic
speeds, focused by the magnetic field of the accretion disk.  They 
come in blobs called "knots".  People can actually see these blobs 
moving out, getting brighter and dimmer.

In fact, many galaxies have super-massive black holes at their centers
with jets like this one.  The special thing about M87 is that it's fairly 
nearby, hence easy to see.  M87 is the biggest galaxy in the Virgo Cluster. 
This is the closest galaxy cluster to us, about 50 million light years away.
That sounds pretty far, but it's only 1000 times the radius of the Milky
Way.  If the Milky Way were a pebble, M87 would be only a stone's throw
away.  So, even amateur astronomers - really good ones, at least - can take 
photos of M87 that show the jet.  But here's a high-quality picture produced
by Robert Lupton using data from the Sloan Digital Sky Survey - you can see 
the jet in light blue:

<BR>
<DIV ALIGN = CENTER>
<A HREF = "http://www.astro.princeton.edu/~rhl/PrettyPictures/">
<IMG HEIGHT = 500 WIDTH = 500 SRC = "m87_core.jpg">
% </A>
</DIV>
<BR>

5) Robert Lupton and the Sloan Digital Sky Survey Consortium, The
central regions of M87, <A HREF =
"http://www.astro.princeton.edu/~rhl/PrettyPictures/">http://www.astro.princeton.edu/~rhl/PrettyPictures/</A>

Backing off a bit further, let's take a look at the Virgo Cluster.  It
contains over a thousand galaxies, but we can tell it's fairly new as
clusters go, since it consists of a bunch of "subclusters"
that haven't merged yet.  Our galaxy, and indeed the whole Local Group
to which it belongs, is being pulled towards the Virgo Cluster and
will eventually join it.  Here's a nice closeup of part of the Virgo
Cluster:

<BR>
<DIV ALIGN = CENTER>
<A HREF = "http://burro.astr.cwru.edu/Schmidt/Virgo/">
<IMG HEIGHT = 300 WIDTH = 500 SRC = "virgo_cluster.jpg">
% </A>
</DIV>
<BR>

6) Chris Mihos, Paul Harding, John Feldmeier and Heather Morrison,
Deep imaging of the Virgo Cluster, <A HREF = "http://burro.astr.cwru.edu/Schmidt/Virgo/">http://burro.astr.cwru.edu/Schmidt/Virgo/</A>

Finally, just for fun, something unrelated - and more mysterious.  It's 
called "Hoag's object":

<BR>
<DIV ALIGN = CENTER>
<A HREF = "http://heritage.stsci.edu/2002/21/">
<IMG HEIGHT = 500 WIDTH = 500 SRC = "hoag.jpg">
% </A>
</DIV>
<BR>

7) The Hubble Heritage Project, Hoag's Object, 
<A HREF = "http://heritage.stsci.edu/2002/21/">http://heritage.stsci.edu/2002/21/</A>

It's a ring-shaped galaxy full of hot young blue stars surrounding a ball 
of yellower stars.  Nobody knows how it formed: perhaps by a collision
of two galaxies?  Such collisions are fairly common, but they don't 
typically create this sort of structure.  

The weirdest part is that
inside the ring, in the upper right, you can see \emph{another} ring galaxy 
in the distance! 
Maybe an advanced civilization over there enjoys this form of art?
Probably not, but if it turns out to be true, you heard it here first.

Anyway... back here on Earth, in the summer of 2004, I visited Dublin for a 
conference on general relativity called GR17.  As recounted in "<A HREF = "week207.html">week207</A>", 
this was where Hawking admitted defeat in his famous bet with John Preskill 
about information loss due to black hole evaporation.  In August of this 
year, Hawking finally came out with a short paper on the subject:

8) Stephen W. Hawking, Information loss in black holes, available as
<A HREF = "http://xxx.lanl.gov/abs/hep-th/0507171">hep-th/0507171</A>.

I spent a lot of time talking to physicists, but I also wandered around 
Dublin a bit.  Besides listening to some great music at a pub called 
Cobblestones - Kevin Rowsome plays a mean uilleann pipe! - and tracking 
down some sites mentioned in James Joyce's novel "Ulysses", I went with 
Tevian Dray on a pilgrimage to Brougham Bridge.  

Tevian Dray is an expert on the octonions, and Brougham Bridge is where 
Hamilton carved his famous formula defining the quaternions!  Now there 
is a plaque commemorating this event, which reads:

<DIV ALIGN = CENTER>
                    Here as he walked by <br>
                 on the 16th of October 1843 <br>
                 Sir William Rowan Hamilton <br>
               in a flash of genius discovered <br>
                the fundamental formula for <br>
                 quaternion multiplication <br>
                i^{2} = j^{2} = k^{2} = ijk = -1 <br>
             \text{\&}  cut it on a stone of this bridge
</DIV>

It does't mention that Hamilton had been racking his brain for the
entire month of October trying to solve this problem: "flashes of
genius" favor the prepared mind.  But it's a nice story and a nice place.
My friend Tevian Dray took some photos, which you can see here:

9) John Baez, Dublin, <A HREF = "http://math.ucr.edu/home/baez/dublin/">http://math.ucr.edu/home/baez/dublin/</A>

It was a bit of a challenge finding Brougham Bridge, since nobody at the main
bus station gave us correct information about which bus went there - except 
the bus driver who finally took us there.  So, to ease your way in case
you want to make your own pilgrimage, the above page includes directions. 
And now, thanks to Dirk Schlimm, it also includes a link to a map showing 
the bridge!  

Speaking of Hamilton, Theron Stanford recently sent me an answer to one of 
life's persistent questions: why is momentum denoted by the letter p?  

Since momentum and position play fundamental roles in Hamiltonian mechanics, 
and they're denoted by p and q, one wonders: could this notation be related 
to Hamilton's alcoholism in later life?  After all, some claim the saying 
\emph{mind your p's and q's} began as a friendly Irish warning not
to imbibe too many pints and quarts!  So, maybe he used these letters
in his work on physics as a secret plea for help.

Umm... probably not.  Just kidding.  But in the absence of hard facts, 
speculation runs rampant.  So, I'm glad Stanford provided some of the former, 
to squelch the latter.

He sent me this email:

\begin{quote}
   While Googling various subjects, I came across 
   <A HREF = "http://math.ucr.edu/home/baez/qg-winter2001/qg12.2.html">the 
   following</A> from 
   your <A HREF = "http://math.ucr.edu/home/baez/QG.html">Quantum Gravity 
   Seminar</A> notes from 2001:

\begin{quote}
       Again Oz was overcome with curiosity, so mimicking Toby's voice, 
       he asked, "Why do we call the momentum p?"

       The Wiz glared at Toby.  "Because m is already taken -- it stands 
       for mass!  Seriously, I don't know why people call position q 
       and momentum p.  All I know is that if you use any other letters, 
       people can tell you're not a physicist.  So I urge you to follow 
       tradition on this point."
\end{quote}
   Well, I have an answer.  Hamilton, the first physicist to actually
   understand the importance of the concept of momentum, chose \pi  to
   stand for momentum (actually, it's not the usual \pi , but what TeX
   calls varpi, a lower-case \omega  with a top, kinda like the top of a
   lower-case \tau ).  Jacobi changed this to p in one of his seminal
   papers on the subject; he also used q in the same paper to stand for
   position.  In the 1800s (I want to say 1850s, though it might have
   been a decade or two later) Cayley presented a paper to the Royal
   Academy in which he says (and I paraphrase), "Well, it seems that p
   and q are pretty well established now, so that's what I'm going to
   use."
\end{quote}

So, now the question is why Hamilton chose the letter "varpi":

<DIV ALIGN = CENTER>
<IMG SRC = varpi.png>
</DIV>

for momentum.  This variant of \pi  was fairly common in the mathematical
literature of the day, so there may be no special explanation.  For 
some further detective work, see:

10) Hamilton: two mysteries solved,
<A HREF = "http://groups.google.com/group/sci.physics/browse_thread/thread/d1b7b4a998682bbb/3a868ae8218a4bca#3a868ae8218a4bca">http://groups.google.com/group/sci.physics/browse_thread/thread/d1b7b4a998682bbb/3a868ae8218a4bca#3a868ae8218a4bca</A>


Also see equation 12 in this paper for one of the first uses of "varpi" 
to mean momentum:

11) William Rowan Hamilton, Second essay on a general method in dynamics,
ed. David R. Wilkins, available at
<A HREF = "http://www.maths.tcd.ie/pub/HistMath/People/Hamilton/Dynamics/SecEssay.pdf">http://www.maths.tcd.ie/pub/HistMath/People/Hamilton/Dynamics/SecEssay.pdf</A>

He doesn't say why he chose this letter - it may have been completely random!

Before I turn to higher-dimensional algebra, maybe this is a good time to
mention a paper related to the octonions:

12) Jakob Palmkvist, A realization of the Lie algebra associated to a 
Kantor triple system, available as <A HREF = "http://arxiv.org/abs/math.RA/0504544">math.RA/0504544</A>.

In "<A HREF = "week193.html">week193</A>" I mentioned how
3-graded Lie algebras come from "Jordan triple systems", and
vaguely hinted that 5-graded Lie algebras come from "Kantor
triple systems".  I explained how the exceptional Lie algebra E8
gets to be 5-graded, but I didn't really say anything about Kantor
triple systems because my understanding of them was so poor.  This
paper by Palmkvist explains them very clearly!  And even better, he
shows how the "magic square" Lie algebras F4, E6, E7, and E8
can be systematically obtained from the octonions, bioctonions,
quateroctonions and octooctonions by means of Kantor triple systems.

Now for some mathematical physics that touches on higher-dimensional
algebra.  If you still don't get why topological field theory and 
n-categories are so cool, read this thesis:

13) Bruce H. Bartlett, Categorical aspects of topological quantum field
theories, M.Sc. Thesis, Utrecht University, 2005.  Available as
<A HREF = "http://arxiv.org/abs/math.QA/0512103">math.QA/0512103</A>.

It's a great explanation of the big picture!  I can't wait to see what 
Bartlett does for his Ph.D..  

If you're a bit deeper into this stuff, you'll enjoy this:

14) Aaron Lauda and Hendryk Pfeiffer, Open-closed strings: two-dimensional
extended TQFTs and Frobenius algebras, available as <A HREF = "http://arxiv.org/abs/math.AT/0510664">math.AT/0510664</A>.

This paper gives a purely algebraic description of the topology of
open and closed strings, making precise and proving some famous results
stated without proof by Moore and Segal, which can be seen here:

15) Greg Moore, Lectures on branes, K-theory and RR charges,
Clay Math Institute Lecture Notes (2002), available at
<A HREF = "http://www.physics.rutgers.edu/~gmoore/clay1/clay1.html">http://www.physics.rutgers.edu/~gmoore/clay1/clay1.html</A>

Lauda and Pfeiffer's paper makes heavy use of Frobenius algebras,
developing more deeply some of the themes I mentioned in "<A HREF
= "week174.html">week174</A>".  In a related piece of work, Lauda
has figured out how to \emph{categorify} the concept of a Frobenius
algebra, and has applied this to 3d topology:

16) Aaron Lauda, Frobenius algebras and ambidextrous adjunctions,
<A HREF = "http://www.tac.mta.ca/tac/volumes/16/4/16-04abs.html">Theory 
and Applications of Categories 16 (2006) 84-122</A>.  Also available as 
<A HREF = "http://arxiv.org/abs/math.CT/0502550">math.CT/0502550</A>.


Aaron Lauda, Frobenius algebras and planar open string topological field 
theories, available as
<A HREF = "http://arxiv.org/abs/math.QA/0508349">math.QA/0508349</A>.

The basic idea behind all this work is a "periodic table" of
categorified Frobenius algebras, which are related to topology in
different dimensions.  For example, in "<A HREF =
"week174.html">week174</A>" I explained how Frobenius algebras
formalize the idea of paint drips on a sheet of rubber.  As you move
your gaze down a sheet of rubber covered with drips of paint, you'll
notice that drips can merge:
 

\begin{verbatim}

                      \ \         / /  
                       \ \       / /
                        \ \     / /
                         \ \   / /
                          \ \_/ /      
                           \   /                                        
                            | |
                            | |
                            | |
                            | |
                            | |   

\end{verbatim}
    
but also split:


\begin{verbatim}

                            | |
                            | |   
                            | |
                            | |
                            | |
                           / _ \                                            
                          / / \ \      
                         / /   \ \
                        / /     \ \
                       / /       \ \
                      / /         \ \ 
\end{verbatim}
    
In addition, drips can start:
                          

\begin{verbatim}

                            _
                           | |
                           | |
                           | |
                           | |
                           | |
                           | |
                           | |
                           | |
                           | |

\end{verbatim}
    
but also end:

                           

\begin{verbatim}

                           | |
                           | |
                           | |
                           | |
                           | |
                           | |
                           | |
                           | |
                           |_|
\end{verbatim}
    

In a Frobenius algebra, these four pictures correspond to four
operations called "multiplication" (merging),
"comultiplication" (splitting), the "unit"
(starting) and the "counit" (ending).  Moreover, these
operations satisfy precisely the relations that you can prove by
warping the piece of rubber and seeing how the pictures change.  For
example, there's the associative law:


\begin{verbatim}

            \ \    / /    / /      \ \    \ \    / /
             \ \  / /    / /        \ \    \ \  / /
              \ \/ /    / /          \ \    \ \/ /
               \  /    / /            \ \    \  /
                \ \   / /              \ \   / /
                 \ \_/ /                \ \_/ /
                  \   /                  \   /
                   | |                    | |
                   | |                    | |    
                   | |          =         | |
                   | |                    | |
                   | |                    | |
                   | |                    | |
                   | |                    | |
                   | |                    | |
\end{verbatim}
    
The idea here is that if you draw the picture on the left-hand side on 
a sheet of rubber, you can warp the rubber until it looks like the 
right-hand side!  There's also the "coassociative law", which is
just an upside-down version of the above picture.   But the most 
interesting laws are the "I = N" equation:


\begin{verbatim}

               \ \     / /                | |        | |
                \ \   / /                 | |        | |
                 \ \_/ /                  | |        | |
                  \   /                   |  \       | |
                   | |                    |   \      | |
                   | |                    | |\ \     | |   
                   | |                    | | \ \    | |
                   | |                    | |  \ \   | |
                   | |          =         | |   \ \  | |
                   | |                    | |    \ \ | |
                   | |                    | |     \ \| |
                   | |                    | |      \   |
                  / _ \                   | |       \  |
                 / / \ \                  | |        | |
                / /   \ \                 | |        | | 
               / /     \ \                | |        | |
\end{verbatim}
    
and its mirror-image version. 

So, the concept of Frobenius algebra captures the topology of regions
in the plane!  Aaron Lauda makes this fact into a precise theorem in
his paper on planar open string field theories, and then generalizes
it to consider "categorified" Frobenius algebras where the
above equations are replaced by isomorphisms, which describe the
\emph{process} of warping the sheet of rubber until the left side looks
like the right.  You should look at his paper even if you don't
understand the math, because it's full of cool pictures.

Lauda and Pfeiffer's paper goes still further, by considering these
paint stripes as "open strings", not living in the plane
anymore, but zipping around in some spacetime of high dimension, where
they might as well be abstract 2-manifolds with corners.  Following
Moore and Segal, they also bring "closed strings" into the
game, which form a Frobenius algebra of their own, where the
multiplication looks like an upside-down pair of pants:


\begin{verbatim}

    O       O
    \ \   / /
     \ \ / /
      \   /
       | |
       | |
       | | 
       | |
       | |
        O
\end{verbatim}
    
These topological closed strings are the subject of Joachim Kock's
book mentioned in "<A HREF = "week202.html">week202</A>";
they correspond to \emph{commutative} Frobenius algebras.  The fun new
stuff comes from letting the open strings and closed strings interact.

You can read more about Lauda and Pfeiffer's work at Urs Schreiber's
blog:

17) Urs Schreiber, Lauda and Pfeiffer on open-closed topological strings,
<A HREF = "http://golem.ph.utexas.edu/string/archives/000680.html">http://golem.ph.utexas.edu/string/archives/000680.html</A>

In fact, I recommend Schreiber's blog quite generally to anyone interested 
in higher categories and/or the math of string theory!

 

\par\noindent\rule{\textwidth}{0.4pt}
\textbf{Addendum:} Bruce Smith, David Rusin and Robert Lupton 
had some comments about the astronomy section; Urs Schreiber 
had more to say about the role of Frobenius algebras in string theory.

Bruce Smith picked up on my comment about accretion disks being
common at many different scales, 
and wondered what the smallest accretion disks are.  We talked about
it and agreed that hurricanes, tornados, dust devils and whirlpools are
\emph{related} phenomena, but not true accretion disks.  

Given this,
the smallest accretion disks I know are those that led to the formation
of planets in our Solar System, and perhaps even some moons.  These 
probably began as eddies in the bigger accretion disk that became our Sun.
The Sun is about 300,00 times heavier than the Earth, and the super-massive
black hole in M87 is about 3 billion times heavier than the Sun, so we're
seeing accretion disks that differ in mass by a factor of a trillion!

David Rusin's reaction to Hoag's object was:

\begin{quote}
 Cool. But what are the chances that there would be not just one but
 TWO fascinating objects which have a significant plane of symmetry, 
 which "just happens" to be perpendicular to our line of sight?
\end{quote}

He asked how many ring galaxies are known!   

I checked and read
there are 100 known "polar-ring galaxies".  Here's a nice one 
called NGC 4650:


<DIV ALIGN = CENTER>
<A HREF = "http://hubblesite.org/newscenter/newsdesk/archive/releases/1999/16/image/a">
<IMG SRC = "NGC4650.jpg">
% </A>
</DIV>

18) Ring around a galaxy, HubbleSite News Archive, May 6, 1999, 
<A HREF = "http://hubblesite.org/newscenter/newsdesk/archive/releases/1999/16/image/a">
http://hubblesite.org/newscenter/newsdesk/archive/releases/1999/16/image/a</A>

I can imagine this thing looking like Hoag's object if we viewed
it head-on.  

Here's another ring galaxy,
called AM 0644-741:

<DIV ALIGN = CENTER>
<A HREF = "http://hubblesite.org/newscenter/newsdesk/archive/releases/2004/15/image/a">
<IMG HEIGHT = 400 WIDTH = 600 SRC = "AM0644-741.jpg">
% </A>
</DIV>

19) The lure of the rings, Hubblesite News Archive, April 22, 2004,
<A HREF = "http://hubblesite.org/newscenter/newsdesk/archive/releases/2004/15/image/a">http://hubblesite.org/newscenter/newsdesk/archive/releases/2004/15/image/a</A>

It's the result of a collision
involving a galaxy that's not in this picture.  So, maybe Hoag's 
object is just a specially pretty case of a galaxy collision!

Robert Lupton referred me to a picture that covers more of the 
Virgo Cluster - but the file is huge, so I won't include it here:

20) Doug Finkbeiner and the Sloan Digital Sky Survey Consortium, 
Some pretty objects as observed by the SDSS: Virgo Cluster, 
<A HREF = "http://www.astro.princeton.edu/~rhl/dfink">
http://www.astro.princeton.edu/~rhl/dfink</A>

See the lower right corner for the picture called "virgobig".

Here's what Urs Schreiber had to say about Frobenius algebras and
string theory:

\begin{quote}
John Baez wrote:

\begin{quote}
  [...]
  Following Moore and Segal, they also bring
  "closed strings" into the game, which form
  a Frobenius algebra of their own, where the
  multiplication looks like an upside-down pair
  of pants:
  [...]
\end{quote}

I would like to make the following general comment
on the meaning of Frobenius algebras in
2-dimensional quantum field theory.

Interestingly, \emph{non}-commutative Frobenius algebras
play a role even for closed strings, and even if the
worldhseet theory is not purely topological.

The archetypical example for this is the class of
2D TFTs invented by Fukuma, Hosono and Kawai. There
one has a non-commutative Frobenius algebra which
describes not the splitting/joining of the entire
worldsheet, but rather the splitting/joining of
edges in any one of its dual triangulations. It is the
\emph{center} of (the Morita class of) the noncommutative
Frobenius algebra decorating dual triangulations
which is the commutative Frobenius algebra describing
the closed 2D TFT.

One might wonder if it has any value to remember
a non-commutative Frobenius algebra when only its center
matters (in the closed case). The point is that the
details of the non-commutative Frobenius algebra acting
in the "interior" of the world sheet affects the
nature of "bulk field insertions" that one can consider
and hence affects the (available notions of) n-point
correlators of the theory, for n > 0.

This aspect, however, is pronounced only when one
switches from 2D topological field theories to
conformal ones.

The fascinating thing is that even 2D \emph{conformal} field
theories are governed by Frobenius algebras. The
difference lies in different categorical internalization.
The Frobenius algebras relevant for CFT don't live
in Vect, but in some other (modular) tensor category,
usually that of representations of some chiral vertex
operator algebra. It is that ambient tensor category
which "knows" if the Frobenius algebra describes a
topological or a conformal field theory (in 2D) -
and which one.

Of course what I am referring to here is the work
by Fjelstad, Froehlich, Fuchs, Runkel, Schweigert and
others. I can recommend their most recent review which
will appear in the Streetfest proceedings. It is
available as 
<A HREF = "http://arxiv.org/abs/math.CT/0512076">math.CT/0512076</A>.



The main result is, roughly, that given any modular tensor
category with certain properties, and given any
(symmetric and special) Frobenius algebra object internal
to that category, one can construct functions on surfaces
that satisfy all the properties that one would demand of
an n-point function of a 2D (conformal) field theory.

If we define a field theory to be something not given by
an ill-defined path integral, but something given by
its set of correlation functions, then this amounts to
constructing a (conformal) field theory.

This result is achieved by first defining a somewhat
involved procedure for generating certain classes of
functions on marked surfaces, and then proving that
the functions generated by this procedure do indeed
satisfy all the required properties.

In broad terms, the prescription is to choose a dual
triangulation of the marked worldsheet whose correlation
function is to be computed, to decorate its edges with
symmetric special Frobenius algebra objects in some
modular tensor category, to decorate its vertices by
product and coproduct morphisms of this algebra, to
embed the whole thing in a certain 3-manifold in a
certain way and for every boundary or bulk field
insertion to add one or two threads labeled by
simple objects of the tensor category which connect
edges of the chosen triangulation with the boundary of
that 3-manifold. Then you are to hit the resulting
extended 3-manifold with the functor of a 3D TFT and
hence obtain a vector in a certain vector space. This
vector, finally, is claimed to encode the correlation
function.

This procedure is deeply rooted in well-known relations
between 3-(!)-dimensional topological field theory,
modular functors and modular tensor categories and
may seem very natural to people who have thought long
enough about it. It is already indicated in Witten's
paper on the Jone's polynomial, that 3D TFT (Chern-Simons
field theory in that case) computes conformal blocks
of conformal field theories on the boundaries of these
3-manifolds. To others, like me in the beginning,
it may seem like a miracle that an involved and
superficially ad hoc procedure like this has anything
to do with correlations functions of conformal
field theory in the end.

In trying to understand the deeper "meaning" of it all
I played around with the idea that this prescription
is really, to some extent at least, the "dual"
incarnation of the application of a certain 2-functor
to the worldsheet. Namely a good part of the rough
structure appearing here automatically drops out
when a 2-functor applied to some 2-category of
surfaces is "locally trivialized". I claim that
any local trivialization of a 2-functor on
some sort of 2-category of surface elements gives
rise to a dual triangulation of the surface whose
edges are labeled by (possibly a generalization of)
a Frobenius algebra object and whose vertices are
labeled by (possibly a generalization of) product
and coproduct operations. There is more data
in a locally trivialized 2-functor, and it seems to
correctly reproduce the main structure of bulk field
insertions as appearing above. But of course there
is a limit to what a \emph{2}-functor can know about a
structure that is inherently 3-dimensional.

I have begun outlining some of the details that I
have in mind here:

<A HREF = "http://golem.ph.utexas.edu/string/archives/000697.html"> 
http://golem.ph.utexas.edu/string/archives/000697.html </A>

This has grown out of a description of gerbes with
connective structure in terms of transport 2-functors.
Note that in what is called a \emph{bundle} gerbe we also
do have a certain product operation playing a
decisive role. Bundle gerbes can be understood as
"pre-trivializations" of 2-functors to Vect:

<A HREF = "http://golem.ph.utexas.edu/string/archives/000686.html">
http://golem.ph.utexas.edu/string/archives/000686.html</A>

and the product appearing is one of the Frobenius
products mentioned above. For a bundle gerbe the
coproduct is simply the inverse of the product,
since this happens to be an isomorphism. The claim
is that 2-functors to Vect more generally give rise
to non-trivial Frobenius algebras when locally
trivialized.

This is work in progress and will need to be refined.
I thought I'd mention it here as a comment to John's
general statements about how Frobenius algebras know
about 2-dimensional physics. I am grateful for all
kinds of comments.
\end{quote}

Here's the paper Urs refers to: 

21) Ingo Runkel, Jens Fjelstad, Jurgen Fuchs and Christoph Schweigert,
Topological and conformal field theory as Frobenius algebras, available
as <A HREF = "http://arxiv.org/abs/math.CT/0512076">math.CT/0512076</A>.

\par\noindent\rule{\textwidth}{0.4pt}
<em>Here's how you do it:<br>
First you're obtuse, <br>
Then you intuit, <br>
Then you deduce! </em> - Garrison Keillor

\par\noindent\rule{\textwidth}{0.4pt}

% </A>
% </A>
% </A>
