
% </A>
% </A>
% </A>
\week{March 5, 1996 }


Before continuing my story about higher-dimensional algebra, let me say
a bit about gravity.  Probably far fewer people study general relativity
than quantum mechanics, which is partially because quantum mechanics is
more practical, but also because general relativity is mathematically
more sophisticated.  This is a pity, because general relativity is so
beautiful!

Recently, I have been spending time on sci.physics leading an informal
(nay, chaotic) "general relativity tutorial".  The goal is to explain
the subject with a minimum of complicated equations, while still getting
to the mathematical heart of the subject.  For example, what does
Einstein's equation REALLY MEAN?  It's been a lot of fun and I've
learned a lot!  Now I've gathered up some of the posts and put them on a
web site:

1) John Baez et al, General relativity tutorial, <A HREF =
"gr/gr.html">http://math.ucr.edu/home/baez/gr/gr.html</A> 

I hope to improve this as time goes by, but it should already be fun to
look at.

Let me also list a couple new papers on the loop representation of quantum
gravity, dealing with ways to make volume and area into observables in
quantum gravity:

2) Abhay Ashtekar and Jerzy Lewandowski, Quantum Theory of Geometry I:
Area Operators, 31 pages in LaTeX format, to appear in Classical and
Quantum Gravity, preprint available as <A HREF = "http://xxx.lanl.gov/abs/gr-qc/9602046">gr-qc/9602046</A>. 

Jerzy Lewandowski, Volume and Quantizations, preprint available as
<A HREF = "http://xxx.lanl.gov/abs/gr-qc/9602035">gr-qc/9602035</A>. 

Roberto De Pietri and Carlo Rovelli, Geometry Eigenvalues and Scalar
Product from Recoupling Theory in Loop Quantum Gravity, 38 pages, 5
Postscript figures, uses RevTeX 3.0 and epsfig.sty, preprint available
as  <A HREF = "http://xxx.lanl.gov/abs/gr-qc/9602023">gr-qc/9602023</A>.

I won't say anything about these now, but see "<A HREF =
"week55.html">week55</A>" for some information on area operators.

% <A NAME = "tale">


 Okay, where were we?  We had started messing around with sets, and
we noted that sets and functions between sets form a category, called
Set.  Then we started messing around with categories, and we noted that
not only are there "functors" between categories, there are
things that ply their trade between functors, called "natural
transformations".  I then said that categories, functors, and
natural transformations form a 2-category.   I didn't really say what a
2-category is, except to say that it has objects, morphisms between
objects, and 2-morphisms between morphisms.  Finally, I said that this
pattern continues: nCat forms an (n+1)-category.

By the way, I said last time that Set was "the primordial
category".  Keith Ramsay reminded me by email that this can be
misleading.  There are other categories that act a whole lot like Set
and can serve equally well as "the primordial category".
These are called topoi.  Poetically speaking, we can think of these as
alternate universes in which to do mathematics.  For more on topoi, see
"<A HREF = "week68.html">week68</A>".  All I meant by saying
that Set was "the primordial category" is that, if we start
from Set and various categories of structures built using sets - groups,
rings, vector spaces, topological spaces, manifolds, and so on - we can
then abstract the notion of "category", and thus obtain Cat.
In the same sense, Cat is the primordial 2-category, and so on.

I mention this because it is part of a very important broad pattern in
higher-dimensional algebra.  For example, we will see that the complex
numbers are the primordial Hilbert space, and that the category of
Hilbert spaces is the primordial "2-Hilbert space", and that
the 2-category of 2-Hilbert spaces is the primordial "3-Hilbert
space", and so on.  This leads to a quantum-theoretic analog of the
hierarchy of n-categories, which plays an important role in mathematical
physics.  But I'm getting ahead of myself!

Let's start by considering a few examples of categories.  I want to pick
some examples that will lead us naturally to the main themes of
higher-dimensional algebra.  Beware: it will take us a while to get
rolling.  For a while - maybe a few issues of This Week's Finds -
everything may seem somewhat dry, pointless and abstract, except for
those of you who are already clued in.  It has the flavor of
"foundations of mathematics," but eventually we'll see these new
foundations reveal topology, representation theory, logic, and quantum
theory to be much more tightly interknit than we might have thought.  So
hang in there.

For starters, let's keep the idea of "symmetry" in mind.  The typical
way to think about symmetry is with the concept of a "group".  But
to get a concept of symmetry that's really up to the demands put on it
by modern mathematics and physics, we need - at the very least - to
work with a \emph{category} of symmetries, rather than a group of symmetries.

To see this, first ask: what is a category with one object?  It is a
"monoid".  The \emph{usual} definition of a monoid is this: a set M
with an associative binary product and a unit element 1 such that a1 =
1a = a for all a in S.  Monoids abound in mathematics; they are in a
sense the most primitive interesting algebraic structures.

To check that a category with one object is "essentially just a
monoid", note that if our category C has one object x, the set
hom(x,x) of all morphisms from x to x is indeed a set with an
associative binary product, namely composition, and a unit element,
namely 1_{x}.  (Actually, in an arbitrary category hom(x,y) could be a
class rather than a set.  But let's not worry about such nuances.)
Conversely, if you hand me a monoid M in the traditional sense, I can
easily cook up a category with one object x and hom(x,x) = M.

How about categories in which every morphism is invertible?  We say a
morphism f: x \to  y in a category has inverse g: y \to  x if fg =
1_{x} and gf = 1_{y}.  Well, a category in which every
morphism is invertible is called a "groupoid".

Finally, a group is a category with one object in which every morphism
is invertible.  It's both a monoid and a groupoid! 

When we use groups in physics to describe symmetry, we think of each
element g of the group G as a "process".  The element 1
corresponds to the "process of doing nothing at all".  We can
compose processes g and h - do h and then g - and get the product gh.
Crucially, every process g can be "undone" using its inverse
g^{-1}.


We tend to think of this ability to "undo" any process as a
key aspect of symmetry.  I.e., if we rotate a beer bottle, we can rotate
it back so it was just as it was before.  We don't tend to think of
SMASHING the beer bottle as a symmetry, because it can't be undone.  But
while processes that can be undone are especially interesting, it's also
nice to consider other ones... so for a full understanding of symmetry
we should really study monoids as well as groups.

But we also should be interested in "partially defined"
processes, processes that can be done only if the initial conditions are
right.  This is where categories come in!  Suppose that we have a bunch
of boxes, and a bunch of processes we can do to a bottle in one box to
turn it into a bottle in another box: for example, "take the bottle
out of box x, rotate it 90 degrees clockwise, and put it in box y".
We can then think of the boxes as objects and the processes as
morphisms: a process that turns a bottle in box x to a bottle in box y
is a morphism f: x \to  y.  We can only do a morphism f: x \to  y to a
bottle in box x, not to a bottle in any other box, so f is a
"partially defined" process.  This implies we can only compose
f: x \to  y and g: u \to  v to get fg: x \to  v if y = u.

So: a monoid is like a group, but the "symmetries" no longer
need be invertible; a category is like a monoid, but the
"symmetries" no longer need to be composable!

Note for physicists: the operation of "evolving initial data from
one spacelike slice to another" is a good example of a
"partially defined" process: it only applies to initial data
on that particular spacelike slice.  So dynamics in special relativity
is most naturally described using groupoids.  Only after pretending that
all the spacelike slices are the same can we pretend we are using a
group.  It is very common to pretend that groupoids are groups, since
groups are more familiar, but often insight is lost in the process.
Also, one can only pretend a groupoid is a group if all its objects are
isomorphic.  Groupoids really are more general.

Physicists wanting to learn more about groupoids might try:

3) Alan Weinstein, Groupoids: unifying internal and external symmetry,
available as <A HREF = "http://math.berkeley.edu/~alanw/Groupoids.ps">
http://math.berkeley.edu/~alanw/Groupoids.ps</A> or
<A HREF = "http://www.ams.org/notices/199607/weinstein.pdf">http://www.ams.org/notices/199607/weinstein.pdf</A>

So: in contrast to a set, which consists of a static collection of
"things", a category consists not only of objects or
"things" but also morphisms which can viewed as
"processes" transforming one thing into another.  Similarly,
in a 2-category, the 2-morphisms can be regarded as "processes
between processes", and so on.  The eventual goal of basing
mathematics upon \omega -categories is thus to allow us the freedom to
think of any process as the sort of thing higher-level processes can go
between.  By the way, it should also be very interesting to consider
"Z-categories" (where Z denotes the integers), having
j-morphisms not only for j = 0,1,2,... but also for negative j.  Then we
may also think of any thing as a kind of process.

How do the above remarks about groups, monoids, groupoids and categories
generalize to the n-categorical context?  Well, all we did was start
with the notion of category and consider two sorts of requirement: that
the category have just one object, or that all morphisms be invertible.

A category with just one object - a monoid - could also be seen as a set
with extra algebraic structure, namely a product and unit.  Suppose we
look at an n-category with just one object?  Well, it's very similar:
then we get a special sort of (n-1)-category, one with a product and
unit!  We call this a "monoidal (n-1)-category".  I will
explain this more thoroughly later, but let me just note that we can
keep playing this game, and consider a monoidal (n-1)-category with just
one object, which is a special sort of (n-2)-category, which we could
call a "doubly monoidal (n-2)-category", and so on.  This game
must seem very abstract and mysterious when one first hears of it.  But
it turns out to yield a remarkable set of concepts, some already very
familiar in mathematics, and it turns out to greatly deepen our notion
of "commutativity".  For now, let me simply display a chart of
"k-tuply monoidal n-categories" for certain low values of n
and k:


\begin{verbatim}

                 k-tuply monoidal n-categories


              n = 0           n = 1             n = 2



k = 0         sets          categories         2-categories
     


k = 1        monoids         monoidal           monoidal
                            categories        2-categories


k = 2       commutative      braided            braided
             monoids         monoidal           monoidal
                            categories        2-categories 


k = 3          `'           symmetric            weakly
                             monoidal          involutory
                            categories          monoidal
                                              2-categories


k = 4          `'              `'               strongly 
                                               involutory
                                                monoidal
                                              2-categories


k = 5          `'              `'                 `'


\end{verbatim}
    
The quotes indicate that each column "stabilizes" past a certain
point.  If you can't wait to read more about this, you might try
"<A HREF = "week49.html">week49</A>" for more, but I will explain it all in more detail in future
issues.   

What if we take an n-category and demand that all j-morphisms (j > 0) be
invertible?  Well, then we get something we could call an "n-groupoid".
However, there are some important subtle issues about the precise sense
in which we might want all j-morphisms to be invertible.  I will have to
explain that, too.  

Let me conclude, though, by mentioning something the experts should
enjoy.  If we define n-groupoids correctly, and then figure out how to
define \omega -groupoids correctly, the homotopy category of
\omega -groupoids turns out to be equivalent to the homotopy category of
topological spaces.  The latter category is something algebraic
topologists have spent decades studying.  This is one of the main ways
n-categories are important in topology.  Using this correspondence
between n-groupoid theory and homotopy theory, the
"stabilization" property described above is then related to a
subject called "stable homotopy theory", and
"Z-groupoids" are a way of talking about "spectra" -
another important tool in homotopy theory.

The above paragraph is overly erudite and obscure, so let me explain the
gist: there is a way to think of a topological space as giving us an
\omega -groupoid, and the \omega -groupoid then captures all the information
about its topology that homotopy theorists find interesting.  (I will
explain in more detail how this works later.)  If this is \emph{all}
n-category theory did, it would simply be an interesting language for
doing topology.  But as we shall see, it does a lot more.  One reason is
that, not only can we use n-categories to think about spaces, we can
also use them to think about symmetries, as described above.  Of course,
physicists are very interested in space and also symmetry.  So from the
viewpoint of a mathematical physicist, one interesting thing about
n-categories is that they \emph{unify} the study of space (or spacetime) with
the study of symmetry.

I will continue along these lines next time and try to fill in some of
the big gaping holes.  

<A HREF = "week75.html#tale">To continue reading the `Tale of
n-Categories', click here.</A>
\par\noindent\rule{\textwidth}{0.4pt}

% </A>
% </A>
% </A>
