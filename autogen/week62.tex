
% </A>
% </A>
% </A>
\week{August 28, 1995 }

Now I'd like to talk about a fascinating subject of importance in both
mathematics and physics, the subject of "ADE
classifications".  Here A, D, and E aren't abbreviations for
anything; they are just names for certain diagrams.  But these
diagrams show up all over the place when you start trying to classify
beautiful and symmetrical things.

Let's start with something nice and simple: the Platonic solids.  
It's not terribly hard to classify all the regular polyhedra in
3-dimensional Euclidean space.  Roughly, it goes like this.  The 
faces could all be equilateral triangles.  Obviously there need to
be at least 3 faces meeting at each vertex to get a polyhedron.
If there are exactly 3, you have a tetrahedron.  If there are 4,
you have an octahedron.  If there are 5, you have an icosahedron.
There can't be 6 or more, since when you have 6 they lie flat in 
the plane, and more is even worse.  The faces could also be squares.
If there are 3 squares meeting at each vertex you have a cube.
There can't be 4 or more, since when you have 4 they lie flat in 
the plane.  The faces could also be regular pentagons.  If there
are 3 pentagons meeting at each vertex you have a dodecahedron.  
There can't be 4 or more, since when you have 4 you already have
more than 360 degree's worth of angles.  

So, there we are: the 5 regular polyhedra are the tetrahedron,
octahedron, icosahedron, cube, and dodecahedron!  Of course, we
haven't shown these solids actually exist.  Sometimes people forget
that you really need to check that all these possibilities are
realized!  But the Greeks did that a while back.  This is perhaps the
first example of an ADE classification.

This had such beauty that in his "Timaeus" dialog, Plato suggested
that the 4 elements were made of these solids, not counting for the
dodecahedron.  Interestingly, Plato considered decomposing the faces
of these solids into "elementary triangles", in order to explain how
one element could turn into another.  This is presumably why he left
out the dodecahedron: one can't chop up a regular pentagon into
30-60-90 triangles.  In a passage that's notoriously hard to
translate, he suggested that the dodecahedron corresponding to some
sort of "quintessence", or perhaps the zodiac.  It's worth pointing
out, also, that Plato explicitly says it's okay if someone comes up
with a better scheme.  He makes it clear that he is just trying to lay
out an \emph{example} of a mathematical scheme for explaining the elements,
to get people interested.

Later, of course, Kepler suggested that the 5 Platonic solids
corresponded to the orbits of the 5 planets:

<div align = center>
<img src = kepler_mysterium_cosmographicum.jpg>
</div>

As it turns out, Plato
and Kepler were in the right ball-park, but not really right.  Both
the solar system and atoms are described pretty well by similar laws -
the inverse-square force laws for gravity and electrostatics. And
solving this problem (in either the classical or quantum case) does
indeed require a deep understanding of rotations in 3-dimensional
space.  It's sort of amusing, however, that the Platonic solids have
as their symmetries finite subgroups of the rotation group in 3
dimensions, while the study of quantum-mechanical atoms instead
involves the theory of "representations" of this group,
which are in some sense dual.  The rotation group in n dimensions, by
the way, is called SO(n).  See "<A HREF =
"week61.html">week61</A>" for a bit more about it.  For
a grand tour of the inverse square law, both classical and quantum,
read:

1) Victor Guillemin and Shlomo Sternberg, Variations on a Theme
by Kepler, American Mathematical Society, Providence, Rhode Island,
1990.

You will see, among other things, that the real reason the inverse
square force law problem is exactly solvable is that it has a hidden
symmetry under SO(4), not just SO(3).  

But I digress!  Recall how I said that "obviously" a regular
polyhedron has to have 3 faces meeting at each vertex?  What would
happen if you relaxed the definition a little bit, and let there be
just 2 faces meeting at a vertex?  Well, then any regular polygon
could qualify as a regular polyhedron, I guess.  Then we would have an
infinite series of regular polyhedron with only two faces, together
with 5 exceptions, the Platonic solids.  That's actually typical of
ADE-type classifications: often, when you are classifying really
symmetrical things, you find some infinite series of
"obvious" or "classical" cases, together with
finitely many weird "exceptional" cases.

Before I get further into ADE classifications, let me note that
the \emph{problem} of why there are so many ADE classifications, and
how they are all related, was explicitly raised by the famous
mathematical physicist V. I. Arnol'd, in 

2) Problems of Present Day Mathematics in Mathematical Developments
Arising from Hilbert's Problems, ed. F. E. Browder, Proc. Symp.
Pure Math. 28, American Mathematical Society, Providence, Rhode
Island, 1976.

This lists a lot of important math problems, following up on
Hilbert's famous turn-of-the-century listing of problems.  Problem 
VIII in this book is the "ubiquity of ADE classifications".  Arnol'd
lists the following examples:

<UL>
<LI> Platonic solids
<LI> Finite groups generated by reflections
<LI> Weyl groups with roots of equal length
<LI> Representations of quivers
<LI> Singularities of algebraic hypersurfaces with definite intersection form
<LI> Critical points of functions having no moduli
</UL>

Don't worry if you don't know what those are except for the first one!
I'll try to explain some of them.  Later I'll also explain two new ones
that came out of string theory:

<UL>
<LI> Minimal models 
<LI> Certain "quantum categories"
</UL>

Perhaps the best single place to start learning about ADE classifications
is:

3) M. Hazewinkel, W. Hesselink, D. Siermsa, and F. D. Veldkamp, The 
ubiquity of Coxeter-Dynkin diagrams (an introduction to the ADE problem), 
Niew. Arch. Wisk., 25 (1977), 257-307.
Also available at
<A HREF =" http://repos.project.cwi.nl:8888/cwi_repository/docs/I/10/10039A.pdf
">http://repos.project.cwi.nl:8888/cwi_repository/docs/I/10/10039A.pdf</A>
or 
<A HREF =" http://math.ucr.edu/home/baez/hazewinkel_et_al.pdf">
http://math.ucr.edu/home/baez/hazewinkel_et_al.pdf</A>

Okay, so what the heck is an ADE classification, after all?  It's
probably good to start by looking at "finite reflection
groups." Say we are in n-dimensional Euclidean space.  Then given
any unit vector v, there is a reflection that takes v to -v, and
doesn't do anything to the vectors orthogonal to v.  Let's call this a
"reflection through v".  A finite reflection group is a
finite group of transformations of Euclidean space such that every
element is a product of reflections.  For example, the group of
symmetries of an equilateral n-gon is a finite reflection group.
(This is a useful exercise if you don't see it right off the bat.)

Note that if we do two reflections, we get a rotation.  In particular,
suppose we have vectors v and w at an angle A from each other, and let
r and s be the reflections through v and w, respectively.  Then rs is
a rotation by the angle 2A.  Draw a picture and check it!  This means
that if A = \pi  / n, then (rs)^{n} is a rotation by the angle
2\pi , which is the same as no rotation at all, so (rs)^{n} =
1.  On the other hand, if A is not a rational number times \pi , we
never have (rs)^{n} = 1, so r and s can not both be in some
\emph{finite} reflection group.

With a little more work, we can convince ourselves that any finite
reflection group is captured by a "Coxeter diagram".  The
idea is that the group is generated by reflections through unit
vectors that are all at angles of \pi /n from each other.  To keep
track of things, we draw a dot for each one of these vectors.  Suppose
two of the vectors are at an angle \pi /n from each other.  If n = 2,
we don't bother drawing a line between the two dots.  Otherwise, we
draw a line between them, and label it with the number n.  Typically,
if n = 3 people don't bother writing the number; they just draw that
line.  That's what I'll do.  (People also sometimes draw n - 2 lines
instead of writing the number n, but I can't do that here.)

Algebraically speaking, if someone hands us a Coxeter diagram like

\begin{verbatim}
      7
o---o---o
\end{verbatim}
    

we get a group having one generator for each dot, and with one
relation r^{2} = 1 for each generator r (since that's what
reflections do), and one relation of the form (rs)^{n} = 1 for
each line connecting dots, or (rs)^{2} = 1 if there is no line
connecting two dots.  It turns out that if a Coxeter diagram yields a
\emph{finite} group this way, it's a finite reflection group.

However, not every diagram we draw yields a finite group!  Here are
all the possible Coxeter diagrams giving finite groups.  They
have names.  First there is A_{n}, which has n dots like this:

\begin{verbatim}
o---o---o---o
\end{verbatim}
    

For example, the group of symmetries of the equilateral triangle is
A_{2}.  The two dots can correspond to the reflections r and s
through two of the altitudes of the triangle, which are at an angle of
\pi /3 from each other.  Thus they satisfy (rs)^{3} = 1.  
More generally,
A_{n} corresponds to the group of symmetries of an
n-dimensional simplex - which is just the group of permutations of
the n+1 vertices.

Then there is B_{n}, which has n dots, where n > 1:

\begin{verbatim}
          4
o---o---o---o
\end{verbatim}
    

It has just one edge labelled with a 4.   B_{n} turns out to be
the group of symmetries of a hypercube or hyperoctahedron in n
dimensions.  

Then there is D_{n}, where n > 3:

\begin{verbatim}
              o
             /
o---o---o---o
             \
              o
\end{verbatim}
    

Then there are E_{6}, E_{7}, and E_{8}:

\begin{verbatim}
      o
      |
o--o--o--o--o

      o
      |
o--o--o--o--o--o

      o
      | 
o--o--o--o--o--o---o
\end{verbatim}
    

Interestingly, this series does \emph{not} go on.  That's what I meant
about "classical" versus "exceptional" structures.  

Then there is F_{4}:

\begin{verbatim}
      4
o---o---o---o
\end{verbatim}
    

Then there's G_{2}:

\begin{verbatim}
  6
o---o
\end{verbatim}
    

and H_{3} and H_{4}:

\begin{verbatim}
  5
o---o---o

  5
o---o---o---o
\end{verbatim}
    


H_{3} is the group of symmetries of the dodecahedron or
icosahedron.  H_{4} is the group of symmetries of a regular
solid in 4 dimensions which I talked about in "<A HREF =
"week20.html">week20</A>".  This regular solid is also
called the "unit icosians" - it has 120 vertices, and is a
close relative of the icosahedron and dodecahedron.  One amazing thing
is that it itself \emph{is} a group in a very natural way.  There are no
"hypericosahedra" or "hyperdodecahedra" in
dimensions greater than 4, which is related to the fact that the H
series quits at this point.

Finally, there is another infinite series, I_{m}:

\begin{verbatim}
  m
o---o
\end{verbatim}
    

This corresponds to the symmetry group of the 2m-gon in the plane,
and people usually require m = 5 or m > 6, so as to not count
twice some Coxeter diagrams that we've already run into.

THAT'S ALL.

So, we have an "ABDEFGHI classification" of finite reflection groups.
(In some future week I had better say what happened to "C".)  Note
that the symmetry groups of the Platonic solids and some of their
higher-dimensional relatives fit in nicely into this classification,
so that's one sense in which the Greeks' discovery of these solids
counts as the first "ADE classification".  But there is at least one
another, deeper, way to fit the Platonic solids themselves into an ADE
classification.  I'll try to say more about this in future weeks.

You may still be wondering what's so special about A, D, and E.  
I'll have to get to that, too.


\par\noindent\rule{\textwidth}{0.4pt}
% </A>
% </A>
% </A>


% parser failed at source line 401
