
% </A>
% </A>
% </A>
\week{ctober 10, 1993}

Louis Kauffman is editing a series of volumes called "Series on Knots
and Everything," published by World Scientific.  The first volume was
his own book, "Knots and Physics."  Right now I'd like to talk about the
second volume, by Carter.  I got to know Carter and Saito when it
started seeming that a deeper understanding of string theory and the
loop representation of quantum gravity might require understanding how
2-dimensional surfaces can be embedded in 4-dimensional spacetime.  The
study of this subject quickly leads into some very fascinating algebra,
such as the "Zamolodchikov tetrahedron equations" (which first appeared in
string theory).  A nice review of this subject and their work on it will
appear in a while: 

1) Knotted surfaces, braid movies, and beyond, by J. Scott Carter and M.
Saito, to appear in Knots and Quantum Gravity, ed. John Baez, Oxford U.
Press.  

but for the non-expert, a great way to get started is:

2) How Surfaces Intersect in Space: An Introduction to Topology, by J.
Scott Carter, World Scientific Press, Singapore 1993.

You can tell this isn't a run-of-the-mill introductory topology book
as soon as you read the little blurb about the author on the back
dustjacket.  Occaisionally there will be tantalizing personal details
in these blurbs that indicate that the author is not just a mathematical
automaton; for example, on the back of Hartshorne's famous text on
algebraic topology it says "He has travelled widely, speakes several
foreign languages, and is an experienced mountain climber.  He is also
an accomplished amateur musician; has played the flute for many years,
and during his last visit to Kyoto, he began studying the shakuhachi."  
This somehow fits with the austere and slightly intimidating quality of
the text itself.  The tone of the blurb on the back of Scott Carter's
book could not be more different: "When he is not drawing pictures,
cooking, or playing with Legos, he is writing songs and playing guitar
for his band The Anteaters who have recorded an eight-song cassette
published by Lobe Current Music."  This is a book that invites the
reader into topology without taking itself too seriously.

I remember first reading about topology as the study of doughnuts,
Moebius strips and the like, and then being in a way disappointed as an
undergrad -- although in another way quite excited --  when it seemed
that what topologists \emph{really} did was a lot of "diagram-chasing," the
algebraic technique widely used in homology and homotopy theory.  Once,
however, as a grad student, I took a course in "geometric topology" by
Tim Cochran, and was immensely pleased to find that \emph{some} topologists
really did draw wild pictures of many-handled doughnuts and the like in
4 dimensions, and prove things by sliding handles around.  The nice
thing about this book is that it is readable by any undergraduate - it
doesn't assume or even mention the definition of a topological space! -
but covers some very nontrivial geometric topology.  It is not a
substitute for the usual introductory course; instead, it concentrates
on the study of surfaces embedded or immersed in 3 and 4 dimensional
space, and shows how much there is to ponder about them.  It is \emph{packed}
with pictures and is lots of fun to read.

The intrinsic topology of surfaces is very simple.  The simplest one is
the sphere (by which, of course, mathematicians mean the \emph{surface} of a
ball, not the ball itself).  The next is the torus, that is, the surface
of the doughnut.   One can also think of the torus as what you get by
taking a square and gluing together the edges as below:



$$

     ----->>----
    |           |n
    |           |  
    v           v
    |           |
    |           |
     ----->>----

$$
    

gluing the two horizontal edges together so the single arrows match up, and
gluing the two verticle edges together so the double arrows match up.
There is also a two-handled torus, and so on.  The number of handles is called
the "genus."   All these surfaces are orientable, that is, one can
define a consistent notion of "right" and "left" on them, so that if one
writes a little word on them and slides the word around it'll never come
back mirror-imaged.  And in fact, all orientable surfaces are just
n-handled tori, so they are classified by their genus.  

A nice example of a nonorientable surface is the projective plane.  One
way to visualize this is to take the surface of the sphere and
"identify" opposite points, that is, decree them "the same" by fiat.
Imagine, for example, a globe in which antipodal points have been
identified.  If one writes a word on the north pole and then slides it
down through the Americas to Ecuador, since the southern hemisphere has
been identified with the northern one, we can think of it popping out
over around India somewhere (sorry, my geography is a little rusty when
it comes to antipodes!), but we will see when we slide it back to the
north pole that it has been reversed, and is now written backwards!  We
see from this not only that the projective plane is nonorientable, but
that it has another description: simply take a disc and identify
opposite points along the boundary.  Since we're doing topology, a
square is just as good as a disc, so we can think of the projective
plane as the result of identifying the points on the boundary of a
square as follows:



$$

     -----<<----
    |           |
    |           |  
    v           ^
    |           |
    |           |
     ----->>----

$$
    

Another famous example of a nonorientable surface is the Klein bottle,
which is given by



$$

     -----<<----
    |           |
    |           |
    v           v
    |           |
    |           |
     ----->>----

$$
    

We can take either the Klein bottle or the projective plane and get more
nonorientable surfaces by adding handles.  Every nonorientable surface
is of one of these forms.  I've included a few more basic facts about
the classification of surfaces as puzzles at the end of this article.

Now, the intrinsic topology of surfaces considers them as abstract
spaces in their own right, but the "extrinsic topology" of them
considers the ways they may be mapped into other spaces - for example,
3- or 4-dimensional Euclidean space.  And here things get much more
interesting and subtle.  For example, while one can embed any orientable
surface in 3d space, one cannot embed any of the nonorientable ones.
Here an embedding is a 1-1 continuous map.  However, one can immerse
the non-orientable ones.  An immersion is a map that is locally an
embedding, but not necessarily globally; e.g., a figure 8 is an
immersion of the circle in the plane.  There's a standard way of
immersing the Klein bottle in 3d space with a circle of "double points,"
that is, places where the immersion is 2-1.  One can easily turn this
immersion into an embedding of the Klein bottle 4d space by representing
the 4th coordinate by how \emph{red} the surface is and having the Klein
bottle blush as it passes through itself.  In fact, one can embed any
surface into 4d space.

While one can't embed the nonorientable surfaces in 3d space, it is
interesting to see how close one can come.  The simplest way an
immersion can fail to be an embedding is by having double points.  
Another simple way is to have triple points.  Carter discusses a
charming immersion of the projective plane in 3d space that only has
curves of double points and a single triple point.  This is known as
"Boy's surface."  A somewhat sneakier way immersions fail to be
embeddings is by having "branch points."  Think, for example, of the
function sqrt(z) on the complex plane.  This is a two-valued function,
so its graph consists of two "sheets" which glom together in a funny
way at z = 0, the branch point.  Carter also talks about another neat
immersion of the projective plane in R^3 that just has double points and
a branch point - the "cross cap."  Another immersion, the "Roman
surface," has both triple points and a branch point.  

The general question, then, is what sort of embeddings and immersions
different surfaces admit in 3 and 4 dimensions, and how to classify
these.  If we are studying embeddings into 4 dimensions, a nice
technique is that of movies.  Calling the 4th coordinate "time," we can
draw slices at different times and get frames of a movie.  Most of the
frames of a movie of an embedded surface will show simply a bunch of
knots.  At a few times, however, a "catastrophe" will occur, e.g.:

Frame 1


\begin{verbatim}

      |    |
      |    |
      |    |
      |    |
      |    |
\end{verbatim}
    

Frame 2 


\begin{verbatim}

      \    /
       |  |
       |  |
       |  |
      /    \
\end{verbatim}
    

Frame 3


\begin{verbatim}

      \    /
       \  /
        ||
       /  \
      /    \
\end{verbatim}
    

Frame 4 (the exciting scene - the catastrophe!)


\begin{verbatim}

      \    /
       \  /
        \/
        /\
       /  \
      /    \
\end{verbatim}
    

Frame 5


\begin{verbatim}

      \     /
       \___/
        ___
       /   \
      /     \
\end{verbatim}
    

Frame 6


\begin{verbatim}

      \_____/
      
      
       ______     
      /      \
\end{verbatim}
    

However, there are always many different movies of essentially the same
embedding.  We can, however, always relate these by a sequence of
transformations called "movie moves."  I wish I could draw these, but it
would take too long, so look at Carter's book!  

And while you're at it, check out the index.  You will enjoy finding the
excuses he has for such entries as "hipster jive," "math jail," "basket
shaped thingy," and "chocolate."  Heck, I can't resist one... on page
81: "Mathematicians use the term "word" to mean any finite seuqnece of
letters or numbers.  This practice can freak out (disturb) people who
are not hip to the lingo (aware of the terminology)."

I should add that the following book also has a lot of interesting
pictures of surfaces in it:

3) A Topological Picturebook, by George Francis, Springer-Verlag, 1987.

Problems:
                   
A. Take a projective plane and cut out a little disc.  Show that what's
left is a Moebius strip.
                   
B. Take two projective planes, cut out a little disc from each one and
attach them along the resulting circles.  This is called taking the
"connected sum" of two projective planes.  Show that the result is a
Klein bottle.  In symbols, P + P = K, or 2P = K.
                  
C. Now take the connected sum of a projective plane and a Klein bottle.
Show that this is the same as a projective plane with a handle attached.
A projective plane with a handle attached is just the connected sum of a
projective plane and a torus, so we have: 3P = P + K = P + T.
                 
D. Show:  4P = K + K = K + T.
                 
E. Show: (2n+1)P = P + nT.
                 
F. Show: (2n+2)P = K + nT.
\par\noindent\rule{\textwidth}{0.4pt}

% </A>
% </A>
% </A>
