
% </A>
% </A>
% </A>
\week{August 16, 2006 }



NASA is trying to built up suspense with this "media advisory":

1) NASA, NASA Announces Dark Matter Discovery, 
<A HREF = "http://www.nasa.gov/home/hqnews/2006/aug/HQ_M06128_dark_matter.html">http://www.nasa.gov/home/hqnews/2006/aug/HQ_M06128_dark_matter.html</A>

which says simply:

\begin{quote}
 Astronomers who used NASA's Chandra X-ray Observatory will host 
 a media teleconference at 1 p.m. EDT Monday, Aug. 21, to announce 
 how dark and normal matter have been forced apart in an extraordinarily 
 energetic collision.
\end{quote}
Hmm!  What's this about?

Someone nicknamed "riptalon" at Slashdot made a good guess.
The media advisory 
lists the "briefing participants" as Maxim Markevitch, Doug
Clowe and Sean Carroll.   Markevitch and Clowe work with the Chandra
X-ray telescope to study galaxy collisions and dark matter.  Last 
November, Markevitch gave a talk on this work, which you can see here:

2) Maxim Markevitch, Scott Randall, Douglas Clowe, and Anthony H. Gonzalez,
Insights on physics of gas and dark matter from cluster mergers, available at 
<A HREF = "http://cxc.harvard.edu/symposium_2005/proceedings/theme_energy.html#abs23">http://cxc.harvard.edu/symposium_2005/proceedings/theme_energy.html#abs23</A>

So, barring any drastic new revelations, we can guess what's up.
Markevitch and company have been studying the "Bullet Cluster", a 
bunch of galaxies that has a small bullet-shaped subcluster zipping 
away from the center at 4,500 kilometers per second.  Here's a picture
of it from the above paper:

<DIV ALIGN = CENTER>
<A HREF = "http://cxc.harvard.edu/symposium_2005/proceedings/theme_energy.html#abs23">
<img src = bullet_cluster.jpg>
% </A>
</DIV>


To help you understand this picture a bit: the official
name of the Bullet Cluster is 1E0657-56.  The "exposure"
for this X-ray photograph taken by Chanda
was apparently 0.5 million seconds - 140 hours!
The distance scale shown, 0.5 megaparsecs, is about 1.6 million light years.
The cluster itself has a redshift z = 0.3, meaning its light
has wavelengths stretched by a factor of 1.3.  Under currently 
popular ideas on cosmology, this means it's
roughly 4 billion light years away.

Anyway, what are we seeing here?

You can see rapidly moving galaxy cluster with a shock wave trailing 
behind it.  It seems to have hit another cluster at high speed.
When this kind of thing happens, the \emph{gas} in 
the clusters is what actually collides - the individual galaxies are
too sparse to hit very often.  And when the gas collides, it gets hot.  
In this case, it heated up to about 160 million degrees and started 
emitting X-rays like mad!  The picture shows these X-rays.  This may 
be hottest known galactic cluster.    

That's fun.  But that's not enough reason to call a press conference.  The 
cool part is not the crashing of gas against gas.  The cool part is
that the dark matter in the clusters was unstopped - it kept right 
on going!

How do people know this?  Simple.  Folks can see the \emph{gravity} of 
the dark matter bending the light from more distant galaxies!
It's called "gravitational lensing".
Here are the mass density contours, as seen by this effect.  
I guess Clowe took this photo using the Hubble Space Telescope:

<DIV ALIGN = CENTER>
<A HREF = "http://cxc.harvard.edu/symposium_2005/proceedings/theme_energy.html#abs23">
<img src = bullet_cluster_lensing.jpg>
% </A>
</DIV>


So: X-rays show the gas in one place, but gravity shows most of the mass 
is somewhere else - two lumps zipping along unstopped.
That's good evidence that dark matter is for real.


For more try these:

3) M. Markevitch, S. Randall, D. Clowe, A. Gonzalez, and M. Bradac,
Dark matter and the Bullet Cluster, available at 
<A HREF = "http://www.cosis.net/abstracts/COSPAR2006/02655/COSPAR2006-A-02655.pdf">http://www.cosis.net/abstracts/COSPAR2006/02655/COSPAR2006-A-02655.pdf</A>

4) M. Markevitch, A. H. Gonzalez, D. Clowe, A. Vikhlinin, L. David,
W. Forman, C. Jones, S. Murray, and W. Tucker, Direct constraints on
the dark matter self-interaction cross-section from the merging galaxy
cluster 1E0657-56, available as <A HREF =
"http://arxiv.org/abs/astro-ph/0309303">arXiv:astro-ph/0309303</A>.

5) Maxim Markevitch, Chandra observation of the most interesting
cluster in the Universe, available as <A HREF =
"http://arxiv.org/abs/astro-ph/0511345">arXiv:astro-ph/0511345</A>.

6) M. Markevitch, A. H. Gonzalez, L. David, A. Vikhlinin, S. Murray,
W. Forman, C. Jones and W. Tucker, A textbook example of a bow shock
in the merging galaxy cluster 1E0657-56, Astrophys. J. 567 (2002),
L27.  Also available as <A HREF =
"http://arxiv.org/abs/astro-ph/0110468">arXiv:astro-ph/0110468</A>.

7) Eric Hayashi and Simon D. M. White, How rare is the Bullet
Cluster?, Mon. Not. Roy. Astron. Soc. Lett. 370 (2006), L38-L41,
available as <A HREF =
"http://arxiv.org/abs/astro-ph/0604443">arXiv:astro-ph/0604443</A>.

The first of these is, alas, only the abstract of a talk.  
But it's worth reading, so I'll quote it in its entirety here:

\begin{quote}
1E0657-56, the "Bullet Cluster", is a merger with a uniquely 
simple geometry. From the long Chandra X-ray observation which revealed 
a classic bow shock in front of a small subcluster, we can derive the 
velocity of the subcluster and its direction of motion. Recent accurate 
weak and strong lensing total mass maps clearly show two merging subclusters, 
including the host of the gas bullet seen in X-rays. This cluster provided 
the first direct, model-independent proof of the dark matter existence 
(as opposed to any modified gravity theory) and a direct constraint on 
the self-interaction cross-section of the dark matter particles. I will 
review these and other related results.
\end{quote}

The Bullet Cluster is not the only direct evidence for dark 
matter.  In fact, last year folks claimed to have found
a "ghost galaxy" made mainly of dark matter
and cold hydrogen, with very few stars:

<DIV ALIGN = CENTER>
<A HREF = "http://www2.naic.edu/~rminchin/press/">
<img width = 400 src = "ghost_galaxy.jpg">
% </A>
</DIV>

8) PPARC, New evidence for a dark matter galaxy, 
<A HREF = "http://www.interactions.org/cms/?pid=1023641">http://www.interactions.org/cms/?pid=1023641</A>

However, Matt Owers informs me that the consensus on this
ghost, VIRGOHI 21, is that it's hydrogen stripped off
from a galaxy by the "wind" it felt as 
it fell into the Virgo Cluster.  This effect is called 
"ram pressure stripping" - the gas of a galaxy can be
stripped off if the galaxy is moving rapidly through a cluster, due 
to interaction with the gas in the cluster.

Nonetheless, dark matter is seeming more and more real.  
It thus becomes ever more interesting to find out what dark matter
actually \emph{is}.  The lightest neutralino?  Axions?  Theoretical
physicists are good at inventing plausible candidates, but finding
them is another thing.

Since I'd like to send this off in time to beat NASA, I won't say a
lot more today... just a bit.  

Dan Christensen and Igor Khavkine have discovered some fascinating 
things by plotting the amplitude of the tetrahedral spin network - 
the basic building block of spacetime in 3d quantum gravity - as 
a function of the cosmological constant.  

They get pictures like this:

<DIV ALIGN = CENTER>
<img width = 400 src = tet4.real.med.jpg>
</DIV>

9) Dan Christensen and Igor Khavkine, Plots of q-deformed tets,
<A HREF = "http://jdc.math.uwo.ca/spinnet/">http://jdc.math.uwo.ca/spinnet/</A>

Here the color indicates the real part of the spin network amplitude,
and it's plotted as a function of q, which is related to the 
cosmological constant by a funky formula I won't bother to write down
here.

You can get some nice books on category theory for free these days:

10) Jiri Adamek, Horst Herrlich and George E. Strecker, 
Abstract and Concrete Categories: the Joy of Cats, available at 
<A HREF = "http://katmat.math.uni-bremen.de/acc/acc.pdf">http://katmat.math.uni-bremen.de/acc/acc.pdf</A>

11) Robert Goldblatt, Topoi: the Categorial Analysis of Logic,
available at 
<A HREF = "http://cdl.library.cornell.edu/cgi-bin/cul.math/docviewer?did=Gold010">http://cdl.library.cornell.edu/cgi-bin/cul.math/docviewer?did=Gold010</A>

12) Michael Barr and Charles Wells, Toposes, Triples and Theories,
available at <A HREF = "http://www.case.edu/artsci/math/wells/pub/ttt.html">http://www.case.edu/artsci/math/wells/pub/ttt.html</A>

The first two are quite elementary - don't be scared of the title
of Goldblatt's book; the only complaints I've ever heard about it
boil down to the claim that it's too easy!

You can also download this classic text on synthetic differential
geometry, which is an approach to differential geometry based on
infinitesimals, formalized using topos theory:

13) Anders Kock, Synthetic Differential Geometry, available at
<A HREF = "http://home.imf.au.dk/kock/">http://home.imf.au.dk/kock/</A>

He asks that we not circulate it in printed form - electrons are 
okay, but not paper.

Next I want to say a \emph{tiny} bit about Koszul duality for Lie
algebras, which plays a big role in the work of Castellani on 
the M-theory Lie 3-algebra, which I discussed in "<A HREF = "week237.html">week237</A>".

Let's start with the Maurer-Cartan form.  This is a gadget that shows 
up in the study of Lie groups.  It works like this.  Suppose you have 
a Lie group G with Lie algebra Lie(G).  Suppose you have a tangent 
vector at any point of the group G.  Then you can translate it to the
identity element of G and get a tangent vector at the identity of G.  
But, this is nothing but an element of Lie(G)!

So, we have a god-given linear map from tangent vectors on G to the
Lie algebra Lie(G).  This is called a "Lie(G)-valued 1-form" on G,
since an ordinary 1-form eats tangent vectors and spits out numbers, 
while this spits out elements of Lie(G).  This particular god-given
Lie(G)-valued 1-form on G is called the "Maurer-Cartan form", and
denoted \omega .  

Now, we can define exterior derivatives of Lie(G)-valued differential
forms just as we can for ordinary differential forms.  So, it's 
interesting to calculate d \omega  and see what it's like.  

The answer is very simple.  It's called the Maurer-Cartan equation:

d\omega  = - \omega  ^ \omega 

On the right here I'm using the wedge product of Lie(G)-valued 
differential forms.  This is defined just like the wedge product of
ordinary differential forms, except instead of multiplication of
numbers we use the bracket in our Lie algebra.  

I won't prove the Maurer-Cartan equation; the proof is so easy you
can even find it on the Wikipedia:

14) Wikipedia, Maurer-Cartan form, 
<A HREF = "http://en.wikipedia.org/wiki/Maurer-Cartan_form">
http://en.wikipedia.org/wiki/Maurer-Cartan_form</A>

An interesting thing about this equation is that it shows 
everything about the Lie algebra Lie(G) is packed into the
Maurer-Cartan form.  The reason is that everything about the
bracket operation is packed into the definition of \omega  ^ \omega .

If you have trouble seeing this, note that we can feed \omega  ^ \omega 
a pair of tangent vectors at any point of G, and it will spit out
an element of Lie(G).  How will it do this?  The two copies of \omega 
will eat the two tangent vectors and spit out elements of Lie(G).
Then we take the bracket of those, and that's the final answer.

Since we can get the bracket of \emph{any} two elements of Lie(G) using
this trick, \omega  ^ \omega  knows everything about the bracket in 
Lie(G).  You could even say it's the bracket viewed as a geometrical
entity - a kind of "field" on the group G!

Now, since 

d\omega  = - \omega  ^ \omega  

and the usual rules for exterior derivatives imply that

d^{2}\omega  = 0

we must have

d(\omega  ^ \omega ) = 0

If we work this concretely what this says, we must get some identity 
involving the bracket in our Lie algebra, since \omega  ^ \omega  is just 
the bracket in disguise.  What identity could this be?

THE JACOBI IDENTITY!

It has to be, since the Jacobi identity says there's a way to take
3 Lie algebra elements, bracket them in a clever way, and get zero:

[u,[v,w]] + [v,[w,u]] + [w,[u,v]] = 0

while d(\omega  ^ \omega ) is a Lie(G)-valued 3-form that happens to vanish,
built using the bracket.

It also has to be since the equation d^{2} = 0 is just another way
of saying the Jacobi identity.  For example, if you write out the
explicit grungy formula for d of a differential form applied to a 
list of vector fields, and then use this to compute d^{2} of that 
differential form, you'll see that to get zero you need the Jacobi
identity for the Lie bracket of vector fields.  Here we're just 
using a special case of that.

The relationship between the Jacobi identity and d^{2} = 0 is actually
very beautiful and deep.  The Jacobi identity says the bracket is
a derivation of itself, which is an infinitesimal way of saying that
the flow generated by a vector field, acting as an operation 
on vector fields, preserves
the Lie bracket!  And this, in turn, follows from the fact that the 
Lie bracket is \emph{preserved by diffeomorphisms} - in other words, it's 
a "canonically defined" operation on vector fields.

Similarly, d^{2} = 0 is related to the fact that d is a natural operation 
on differential forms - in other words, that it commutes with 
diffeomorphisms.   I'll leave this cryptic; I don't feel like trying
to work out the details now.

Instead, let me say how to translate this fact:

<DIV ALIGN = CENTER>
         d^{2}\omega  = 0 IS SECRETLY THE JACOBI IDENTITY 
</DIV>
into pure algebra.  We'll get something called "Kozsul duality".
I always found Koszul duality mysterious, until I realized it's
just a generalzation of the above fact.  

How can we state the above fact purely algebraically, only
using the Lie algebra Lie(G), not the group G?  To get ourselves
in the mood, let's call our Lie algebra simply L.

By the way we constructed it, the Maurer-Cartan form is "left-invariant", 
meaning it doesn't change when you translate it using maps like this:


$$

L_{g}: G \to  G
    x |\to  gx
$$
    

that is, left multiplication by any element g of G.  So, 
how can we describe the left-invariant differential forms on G
in a purely algebraic way?  Let's do this for \emph{ordinary} differential
forms; to get Lie(G)-valued ones we can just tensor with L = Lie(G).

Well, here's how we do it.  The left-invariant vector fields on G
are just 

L

so the left-invariant 1-forms are 

L*

So, the algebra of all left-invariant diferential forms on G
is just the exterior algebra on L*.  And, defining the exterior
derivative of such a form is precisely the same as giving the
bracket in the Lie algebra L!  And, the equation d^{2} = 0 is
just the Jacobi identity in disguise.

To be a bit more formal about this, let's think of L as a graded
vector space where everything is of degree zero.  Then L* is the
same sort of thing, but we should \emph{add one to the degree} to think 
of guys in here as 1-forms.  Let's use S for the operation 
of "suspending"
a graded vector space - that is, adding one to the degree.  Then
the exterior algebra on L* is the "free graded-commutative algebra 
on SL*".

So far, just new jargon.  But this lets us state the observation
of the penultimate paragraph in a very sophisticated-sounding way.
Take a vector space L and think of it as a graded vector space
where everything is of degree zero.  Then:

\begin{quote}
 Making the free graded-commutative algebra on SL* into a \emph{differential}
 graded-commutative algebra is the same as making L into a Lie algebra.  
\end{quote}

This is a basic example of "Koszul duality".  Why do we call it 
"duality"?  Because it's still true if we switch the words 
"commutative" and "Lie" in the above sentence!

\begin{quote}
 Making the free graded Lie algebra on SL* into a \emph{differential}
 graded Lie algebra is the same as making L into a commutative algebra.
\end{quote}

That's sort of mind-blowing.  Now the equation d^{2} = 0 secretly 
encodes the \emph{commutative law}.

So, we say the concepts "Lie algebra" 
and "commutative algebra" are
Koszul dual.  Interestingly, the concept "associative 
algebra" is its own dual:

\begin{quote}
 Making the free graded associative algebra on SL* into a \emph{differential}
 graded associative algebra is the same as making L into an associative
 algebra.
\end{quote}

This is the beginning of a big story, and I'll try to say more later.
If you get impatient, try the book on operads mentioned in "<A HREF = "week191.html">week191</A>",
or else these:

15) Victor Ginzburg and Mikhail Kapranov, Koszul duality for quadratic
operads, Duke Math. J. 76 (1994), 203-272.  Also Erratum, Duke Math. 
J. 80 (1995), 293.

16) Benoit Fresse, Koszul duality of operads and homology of partition 
posets, Homotopy theory and its applications (Evanston, 2002), 
Contemp. Math. 346 (2004), 115-215.  Also available at 
<A HREF = "http://math.univ-lille1.fr/~fresse/PartitionHomology.html">http://math.univ-lille1.fr/~fresse/PartitionHomology.html</A>

The point is that Lie, commutative and associative algebras are all
defined by "quadratic operads", and one can define for any such 
operad O a "dual" operad O* such that:

\begin{quote}
 Making the free graded O-algebra on SL* into a \emph{differential}
 graded O-algebra is the same as making L into an O*-algebra.
\end{quote}

And, we have O** = O, hence the term "duality".

This has always seemed incredibly cool and mysterious to me.
There are other meanings of the term "Koszul duality", and if 
really understood them I might better understand what's going on
here.  But, I'm feeling happy now because I see this special case:

\begin{quote}
 Making the free graded-commutative algebra on SL* into a \emph{differential}
 graded-commutative algebra is the same as making L into a Lie algebra.  
\end{quote}

is really just saying that the exterior derivative of left-invariant
differential forms on a Lie group encodes the bracket in the Lie algebra.
That's something I have a feeling for.  And, it's related to the 
Maurer-Cartan equation... though notice, I never completely spelled out
how.


\par\noindent\rule{\textwidth}{0.4pt}

\textbf{Addenda:} Let me say some more about how d^{2} = 0 is
related to the fact that d is a canonically defined operation on
differential forms.  Being "canonically defined" means that
d commutes with the action of diffeomorphisms.  Saying that d commutes
with "small" diffeomorphisms - those connected by a path to
the identity - is the same as saying

d L_{v} = L_{v} d

where v is any vector field and L_{v} is the corresponding
"Lie derivative" operation on differential forms.  But,
Weil's formula says that

L_{v} = i_{v} d + d i_{v}

where i_{v} is the "interior product with v", which
sends p-forms to (p-1)-forms.  If we plug Weil's formula into the
equation we're pondering, we get

d (i_{v} d + d i_{v}) = (i_{v} d + d i_{v}) d

which simplifies to give

d^{2} i_{v} = i_{v} d^{2}

So, as soon as we know d^{2} = 0, we know d commutes with small 
diffeomorphisms.   Alas, I don't see how to reverse the argument.

Similarly, as soon as we know the Jacobi identity, we know the
Lie bracket operation on vector fields is preserved by small
diffeomorphisms, by the argument outlined in the body of this Week.
This argument is reversable.

So, maybe it's an exaggeration to say that d^{2} = 0 and the Jacobi
identity say that d and the Lie bracket are preserved by 
diffeomorphisms - but at least they \emph{imply} these operations are 
preserved by \emph{small} diffeomorphisms.

\par\noindent\rule{\textwidth}{0.4pt}
% </A>
% </A>
% </A>
