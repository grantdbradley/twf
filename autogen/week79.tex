
% </A>
% </A>
% </A>
\week{April 1, 1996 }


Before I continue my tale of adjoint functors I want to say a little
bit about icosahedra, buckyballs, and last letter Galois wrote before
his famous duel.... all of which is taken from the following marvelous
article:

1) Bertram Kostant, The graph of the truncated icosahedron and the last
letter of Galois, Notices of the AMS 42 (September 1995), 959-968. 
Also available at
<A HREF = "http://www.ams.org/notices/199509/199509-toc.html">http://www.ams.org/notices/199509/199509-toc.html</A>.

When I was a graduate student at MIT I realized that Kostant (who
teaches there) was deeply in love with symmetry, and deeply
knowledgeable about some of its more mysterious byways.  Unfortunately I
didn't dig too deeply into group theory at the time, and now I am
struggling to catch up.

Let's start with the Platonic solids.  Note that the cube and the
octahedron are dual - putting a vertex in the center of each of the
cube's faces gives you an octahedron, and vice versa.  So every
rotational symmetry of the cube can be reinterpreted as a symmetry of
the octahedron, and vice versa.  Similarly, the dodecahedron and the
icosahedron are dual, while the tetrahedron is self-dual.  So while
there are 5 Platonic solids, there are really only 3 different symmetry
groups here.  


These 3 "Platonic groups" are very interesting.  The symmetry
group of the tetrahedron is the group A_{4} of all \emph{even}
permutations of 4 things, since by rotating the tetrahedron we can
achieve any even permutation of its 4 vertices.  The symmetry group of
the cube is S_{4}, the group of \emph{all} permutations of 4
things.  What are the 4 things here?  Well, we can draw 4 line segments
connecting opposite vertices of the cube; these are the 4 things!  The
symmetry group of the icosahedron is A_{5}, the group of even
permutations of 5 things.  What are the 5 things?  It we take all the
line segments connecting opposite vertices we get 6 things, not 5, but
we can't get all even permutations of those by rotating the icosahedron.
To find the \emph{5} things is a bit trickier; I leave it as a puzzle
here.  See

2) John Baez, Some thoughts on the number 6, 
<A HREF =
"http://math.ucr.edu/home/baez/six.html">http://math.ucr.edu/home/baez/six.html</A>

for an answer.  

Once we convince ourselves that the rotational symmetry group of the
icosahedron is A_{5}, it follows that it has 5!/2 = 60 elements.  But
there is another nice way to see this.  Take an icosahedron and chop off
all 12 corners, getting a truncated icosahedron with 12 regular
pentagonal faces and 20 regular hexagonal faces, with all edges the same
length.  It looks just like a soccer ball.  It's called an Archimedean
solid because, while not quite Platonic in its beauty, every face is a
regular polygon and every vertex looks alike: two pentagons abutting one
hexagon.  

<div align = "center">
<a href = "http://en.wikipedia.org/wiki/Truncated_icosahedron">
<img style="border:none;" src = "Truncatedicosahedron.gif">
% </a>
</div>

The truncated icosahedron has 5 \times  12 = 60 vertices.  Every symmetry of
the icosahedron is a symmetry of the truncated icosahedron, so
A_{5} acts to permute these 60 vertices.  Moreover, we can find
an element of A_{5} that moves a given vertex of the truncated
icosahedron to any other one, since "every vertex looks
alike".  Also, there is a \emph{unique} element of A_{5}
that does the job.  So there must be precisely as many elements of
A_{5} as there are vertices of the truncated icosahedron, namely
60.

There is a lot of interest in the truncated icosahedron recently,
because chemists had speculated for some time that carbon might form
C_{60} molecules with the atoms at the vertices of this solid,
and a while ago they found this was true.  In fact, while
C_{60} in this shape took a bit of work to get ahold of at
first, it turns out that lowly soot contains lots of this stuff!

<div align = "center">
<a href = "http://en.wikipedia.org/wiki/Fullerene#Buckminsterfullerene">
<img style="border:none;" src = "160px-C60a.png">
% </a>
</div>


Since Buckminster Fuller was fond of using truncated icosahedra in his
geodesic domes, C_{60} and its relatives are called
fullerenes, and the shape is affectionately called a buckyball.  For
more about this stuff, try:

3) P. W. Fowler and D. E. Manolpoulos, An Atlas of Fullerenes, Oxford
University Press, 1995.

M. S. Dresselhaus, G. Dresselhaus, and P. C. Eklund, Science of
Fullerenes and Carbon Nanotubules, Academic Press, New York, 1994.  

G. Chung, B. Kostant and S. Sternberg, Groups and the buckyball, in Lie
Theory and Geometry, eds. J.-L. Brylinski, R. Brylinski, V. Guillemin
and V. Kac, Birkhauser, 1994.  

In fact, for the person who has everything: you can now buy 99.5% pure
C_{60} at the following site:

4) BuckyUSA homepage, <A HREF =
"http://www.buckyusa.com/Fullerene%20C60.htm">http://www.buckyusa.com/Fullerene%20C60.htm</A>


But I digress.  Coming back to the 3 Platonic groups... there is much
more that's special about them.  Most of it requires a little knowledge
of group theory to understand.  For example, they are the 3 different
finite subgroups of SO(3) having irreducible representations on R^{3}.
And they are nice examples of finite reflection groups.  For more about
them from this viewpoint, try "<A HREF =
"week62.html">week62</A>" and "<A HREF =
"week63.html">week63</A>".  Also, via the McKay correspondence they
correspond to the exceptional Lie groups E_{6}, 
E_{7}, and E_{8} - see "<A
HREF = "week65.html">week65</A>" for an explanation of this!


Yet another interesting fact about these groups is buried in Galois'
last letter, written to the mathematician Chevalier on the night before
Galois' fatal duel.  He was thinking about some groups we'd now call
PSL(2,F).  Here F is a field (for example, the real numbers, the complex
numbers, or Z_{p}, the integers mod p where p is prime).
PSL(2,F) is a "projective special linear group over F." What
does that mean?  Well, first of all, SL(2,F) is the 2\times 2 matrices with
entries in F having determinant equal to 1.  These form a group under
good old matrix multiplication.  The matrices in SL(2,F) that are scalar
multiples of the identity matrix form the "center" Z of
SL(2,F) - the group of guys who commute with everyone else.  We can
form the quotient group SL(2,F)/Z, and get a new group called PSL(2,F).


Now Galois was thinking about PSL(2,Z_{p}) where p is prime.
There's an obvious way to get this group to act as permutations of p+1
things.  Here's how!  For any field F, the group SL(2,F) acts as linear
transformations of the 2-dimensional vector space over F, and it thus
acts on the set of lines through the origin in this vector space...
which is called the "projective line" over F.  But anything
in SL(2,F) that's a scalar multiple of the identity doesn't move lines
around, so we can mod out by the center and think of the quotient group
PSL(2,F) as acting on projective line.  (By the way, this explains the
point of working with PSL instead of plain old SL.)


Now, an element of the projective line is just a line through the
origin in F^{2}.  We can specify such a line by taking any
nonzero vector (x,y) in F^{2} and drawing the line through the
origin and this vector.  However, (x',y') and (x,y) determine the same
line if (x',y') is a scalar multiple of (x,y).  Thus lines are in 1-1
correspondence with vectors of the form (1,y) or (x,1).  When our field
F is Z_{p}, there are just p+1 of these.  So
PSL(2,Z_{p}) acts naturally on a set of p+1 things.

What Galois told Chevalier is that PSL(2,Z_{p}) doesn't act
nontrivially as permutation of any set with fewer than p+1 elements if p
> 11.  This presumably means he knew that PSL(2,Z_{p})
\emph{does} act nontrivially on a set with only p elements if p = 5,7,
or 11.  For example, PSL(2,5) turns out to be isomorphic to
A_{5}, which acts on a set of 5 elements in an obvious way.
PSL(2,7) and PSL(2,11) act on a 7-element set and an 11-element set,
respectively, in sneaky ways which Kostant describes.

These cases, p = 5, 7 and 11, are the the only cases where this happens
and PSL(2,Z_{p}) is simple.  (See "<A HREF =
"week66.html">week66</A>" if you don't know what "simple"
means.)  In each case it is very amusing to look at how
PSL(2,Z_{p}) acts nontrivially on a set with p elements and
consider the subgroup that doesn't move a particular element of this set.  
For example, when p = 5 we have PSL(2,5) = A_{5}, and if we look at
the subgroup of even permutations of 5 things that leaves a particular
thing alone, we get A_{4}.  Kostant explains how if we play this
game with PSL(2,7) we get S_{4}, and if we play this game with
PSL(2,11) we get A_{5}.  These are the 3 Platonic groups again!!


But notice an extra curious coincidence.  A_{5} is both PSL(2,5)
and the subgroup of PSL(2,11) that fixes a point of an 11-element set.  
This gives a lot of relationships between A_{5}, PSL(2,5), 
and PSL(2,11).  What Kostant does is take this and milk it for all 
it's worth!  In particular, it turns out that one can think of A_{5} 
as the
vertices of the buckyball, and describe which vertices are connected by
an edge using the embedding of A_{5} in PSL(2,11).  I won't say
how this goes... read his paper!

This may even have some applications for fullerene spectroscopy, since
one can use symmetry to help understand spectra of compounds.  (Indeed,
this is one way group theory entered chemistry in the first place.)
\par\noindent\rule{\textwidth}{0.4pt}
% <A NAME = "tale">

Now let me return to the tale of adjoint functors!  I have been
stressing the fact that two functors L: C \to  D and R: D \to  C are
adjoint if there is a natural isomorphism between hom(Lc,d) and
hom(c,Rd).  We can say that an "adjunction" is a pair of
functors L: C \to  D and R: D \to  C together with a natural
isomorphism between hom(Lc,d) and hom(c,Rd).  But there is another way
to think about adjunctions which is also good.


In "<A HREF = "week76.html">week76</A>" we talked about an
"equivalence" of categories.  We can summarize it this way: an
"equivalence" of the categories C and D is a pair of functors
F: C \to  D and G: D \to  C together with natural transformations e:
FG => 1_{D} and i: 1_{C} => GF that are themselves
invertible.  (Note that we are now writing products of functors in the
order that ordinary mortals typically do, instead of the backwards way
we introduced in "week73".  Sorry!  It just happens to be
better to write it this way now.)  Now, the concept of
"adjunction" is a cousin of the concept of
"equivalence", and it's nice to have a definition of
adjunction that brings out this relationship.

First, consider what happens in the definition of adjunction if we take
c = Rd.  Then we have a natural isomorphism between hom(LRd,d) and
hom(Rd,Rd).  Now there is a special element of hom(Rd,Rd), namely the
identity 1_{Rd}.  This gives us a special element of hom(LRd,d).  Let's call
this

e_{d}: LRd \to  d

What is this morphism like in an example?  Say L: Set \to  Grp takes each
set to the free group on that set, and R: Grp \to  Set takes each group to
its underlying set.  Then if d is a group, LRd is the free group on the
underlying set of d.  There's an obvious homomorphism from LRd to d,
taking each word of elements in d and their inverses to their product in
d.  That's e_{d}.  It goes from the free thing on the underlying thing of
d to the thing d itself!

In fact, since everything in sight is natural, whenever we have an
adjunction the morphisms e_{d} define a natural transformation

e: LR => 1_{D}

Next, consider what happens in the definition of adjunction if we take
d = Lc.  Then we have a natural isomorphism between hom(c,RLc) and
hom(Lc,Lc).  Now there is a special element in hom(Lc,Lc), namely the
identity 1_{Lc}.  This gives us a special element in hom(c,RLc).  Let's
call this  

i_{c}: c \to  RLc

Again, it's good to consider the example of sets and groups: if c is a
set, RLc is the underlying set of the free group on c.  There is an
obvious way to map c into RLc.  That's i_{c}.  It goes from the thing c
to the underlying thing of the free thing on c.

As before, we get a natural transformation

i: 1_{C} => RL

So, as in an equivalence, when we have an adjunction we have natural
transformations e: LR => 1_{D} and i: 1_{C} => RL.
Unlike in an equivalence, they needn't be natural \emph{isomorphisms},
as the example of sets and groups shows.  But they do have some cool
properties, which are nice to draw using pictures.

First, we draw e as a V-shaped thing:

\begin{verbatim}
                 L      R
                  \    /
                   \  / 
                    \/

\end{verbatim}
    
The idea here is that e goes from LR down to the identity 1_{D}, which we
draw as "nothing".  We can think of L and R as processes, and the
V-shaped thing as the meta-process of L and R "colliding into each other
and cancelling out", like a particle and antiparticle.  (Lest you think
that's just purple prose, wait and see!  Eventually I'll explain what
all this has to do with antiparticles!)  Similarly, we draw i as an
upside-down-V-shaped thing:

\begin{verbatim}
                   /\
                  /  \
                 /    \
                R      L

\end{verbatim}
    
In other words, i goes from the identity 1_{C} to RL.  

We can also use this sort of notation to talk about identity natural
transformations.  For example, if we have any old functor F, there is
an identity natural transformation 1_{F}: F => F, which we can draw as
follows:

\begin{verbatim}
                    F
                    |
                    |
                    |
                    F

\end{verbatim}
    
We draw it as a boring vertical line because "nothing is happening" as
we go from F to F.

Now, I haven't talked much about the ways one can compose natural
transformations like i and e, but remember that they are 2-morphisms,
or morphisms-between-morphisms, in Cat (the 2-category of all
categories).   This means that they are inherently 2-dimensional, and in
particular, one can compose them both "horizontally" and "vertically".
I'll explain this more next time, but for now please take my word for
it!  Using these composition operations, one can make sense of the
following equations involving i and e:
                   
\begin{verbatim}
                   R       R
          /\       |       |
         /  \      |       |
        /    \     |       |
       |      \    /  =    | 
       |       \  /        |
       |        \/         |
       R                   R

\end{verbatim}
    
and

\begin{verbatim}
       L                   L
       |       /\          |
       |      /  \         |
       |     /    \        |
       \    /      |  =    | 
        \  /       |       |
         \/        |       |
                   L       L
                    
\end{verbatim}
    

In the first equation we are asserting that a certain way of sticking
together i and e and some identity natural transformations gives
1_{R}: R => R.  In the second we are asserting that some
other way gives 1_{L}: L => L.


I will explain these more carefully next time, but for now I mainly want
to state that we can also \emph{define} an adjunction to be a pair of
functors L: C \to  D and R: D \to  C together with natural
transformations e: LR => 1_{D} and i: 1_{C} => RL
making the above 2 equations hold!  This is the definition of
"adjunction" that is the most similar to the definition of
"equivalence".

Now, topologically, these 2 equations simply say that if you have a wiggly
curve like

\begin{verbatim}
          /\       |       
         /  \      |       
        /    \     |       
       |      \    /  
       |       \  /        
       |        \/         

\end{verbatim}
    
or

\begin{verbatim}
       |       /\          
       |      /  \         
       |     /    \        
       \    /      |   
        \  /       |       
         \/        |       

\end{verbatim}
    
you can pull it tight to get

\begin{verbatim}
            |
            |
            |
            |
            |
            |

\end{verbatim}
    
Thus, while

\begin{verbatim}
                 \    /
                  \  / 
                   \/

\end{verbatim}
    
and

\begin{verbatim}
                   /\
                  /  \
                 /    \

\end{verbatim}
    
are not exactly "inverses", there is some subtler sense in which they
"cancel out".  This corresponds to the notion that while adjoint functors
are not inverses, not even up to a natural isomorphism, they still are
"like inverses" in a subtler sense.


Now this may seem like a silly game, drawing natural transformations as
"string diagrams" and interpreting adjoint functors as wiggles
in the string.  But in fact this is part of a very big, very important,
and very fun game: the relation between n-category theory and the
topology of submanifolds of R^{n}.  Right now we are dealing
with Cat, which is a 2-category, so we are getting into 2-dimensional
pictures.  But when we get into 3-categories we will get into
3-dimensional pictures, and knot theory... and what got me into this
whole business in the first place: the relation between knots and
physics.  In higher dimensions it gets even fancier.

So I will continue next time and explain the recipes for composing
natural transformations, and the associated string diagrams, more
carefully. 
<A HREF = "week80.html#tale">To continue reading the `Tale of
n-Categories', click here.</A>



\par\noindent\rule{\textwidth}{0.4pt}
% </A>
% </A>
% </A>


% parser failed at source line 530
