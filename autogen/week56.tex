
% </A>
% </A>
% </A>
\week{June 16, 1995 }

I got a copy of the following paper when I showed up in Warsaw:

1) Lee Smolin, Linking topological quantum field theory and 
nonperturbative quantum gravity, available as <A HREF = "http://xxx.lanl.gov/abs/gr-qc/9505028">gr-qc/9505028</A>.

and then I spent a fair amount of time reading it and thinking
about it throughout the conference.  If the big hypothesis formulated in
this paper is correct, I think we are on the verge of having a
really beautiful theory of 4-dimensional quantum gravity, at
least given certain boundary conditions.  Mind you, I just
mean a really beautiful theory, not necessarily a physically
correct theory.  But beautiful theories have a certain tendency
to be right, or at least close, so let me explain this hypothesis. 

First of all, we have to remember that Ashtekar reformulated
Einstein's equation so that the configuration space for general 
relativity on the spacetime R x S, instead of being the space
of \emph{metrics} on a 3-manifold S, is a space of \emph{connections} on S.  
A connection is just what a physicist often calls a vector potential, but
for any old gauge theory, not just electromagnetism.  Different gauge
theories have different gauge groups, so I had better tell you the
gauge group of Ashtekar's version of general relativity: it's SL(2,C), 
the group of 2x2 complex matrices with determinant equal to 1. 
And I should probably tell you which bundle over S we have an SL(2,C)
connection on... but luckily, all SL(2,C) bundles over 3-manifolds
are trivial, so I can cut corners by saying it's the trivial bundle. 
We can think of a connection A
on the trivial SL(2,C) bundle over S as 1-forms taking values in
the Lie algebra sl(2,C), consisting of 2x2 complex matrices with
trace zero.

Okay, so naively you might think a state in the \emph{quantum} version
of general relativity a la Ashtekar is just a wavefunction \psi (A).
That's not too far wrong and I won't bother about certain nitpicky
technicalities here (again, for the full story try 
<A HREF = "net.tex">net.tex</A>).  But there's
one very important catch I can't ignore: general relativity has 
\emph{constraint} equations, meaning that \psi  has to satisfy some 
equations.  The first constraint, the Gauss law, just says that we must have


$$

                      \psi (A) = \psi (A')
$$
    

whenever A' is the result of doing a gauge transformation to A.
Or at the very least, this should hold up to a phase; the point is
that \psi  is only supposed to record physically significant information
about the state of the universe, and two connections are physically
equivalent if they differ by a gauge transformation.  The second
constraint, the diffeomorphism constraint, says we need to have


$$

                     \psi (A) = \psi (A')
$$
    

when A' is the result of applying a diffeomorphism of space, S, to A.
Again, the point is that two solutions of general relativity are physically
the same if they differ only by a coordinate transformation, or -
\emph{roughly} the same thing - a diffeomorphism.   The third constraint
is the real killer.  It's meaning is that \psi (A) doesn't change when we do
a diffeomorphism of spaceTIME to the connection A, but it's usually
formulated `infinitesimally' as the Wheeler-DeWitt equation


$$

                        H \psi  = 0
$$
    

meaning roughly that the time derivative of \psi  is zero.  Think of it
as a screwy quantum gravity version of Schrodinger's equation, where
d\psi /dt had better be zero!

It's hard to find explicit solutions of these equations.  Indeed, it's
hard to know what the heck these equations \emph{mean} in a sufficiently
precise way to recognize a solution if we found one!  However, things 
were even worse back in the old days.  Back in the old days when we 
thought of states as wavefunctions on the space of metrics, we didn't 
know ANY solutions of these equations.  But nowadays we are very
happy, because we know infinitely many times as many solutions!
To be precise, we now know ONE solution.  This is called the Chern-
Simons state, and it was discovered by Kodama:

2) H. Kodama, Holomorphic wavefunction of the universe,
Phys. Rev. D42 (1990), 2548-2565.

Now actually people have proposed other explicit solutions, but
personally I have certain worries about all those other solutions,
and I am not alone in this, whereas everyone seems to agree that, no
matter how you slice it, the Chern-Simons state is a solution.

Now there is a slight catch: the Chern-Simons state is a solution
of quantum gravity \emph{with cosmological constant}.  This is an
extra term that Einstein threw into his equations so that they
wouldn't make the obviously ridiculous prediction that the universe
is expanding.  When Hubble took a look and saw galactic redshifts
all over, Einstein called this extra term the biggest blunder in his
life.  That kind of remark, coming from that kind of person, might
make you a little bit reluctant to get too excited about having found a 
state of quantum gravity with this extra term thrown in!  Luckily
it turns out that you can take the limit as the cosmological constant
goes to zero, and get a state of the theory where the cosmological
constant is zero.  I like to call this the `flat state', because it's zero
except where the connection A is flat.  

(In fact, if the space S is not simply connected, there are lots of 
different flat states, because there is what experts call a moduli 
space of flat connections, i.e., lots of different flat connections 
modulo gauge transformations.  Not many people talk too much 
about these flat states, but I do so in my paper <A HREF = "net.tex">
net.tex</A> and also 
the harder one <A HREF = "knot.tex">knot.tex</A>.)

Now what is this Chern-Simons state?  Well, there is a wonderful
thing you can compute from a connection A on a (compact oriented)
3-manifold S, called the Chern-Simons action:


\begin{verbatim}

            CS(A) = integral_S tr(A ^ dA + (2/3)A ^ A ^ A)
\end{verbatim}
    

which looks weird when you first see it, but gradually starts seeming
very sensible and nice.  The reason why folks like it is that it doesn't
change when you do a small gauge transformation - i.e., one you
can get to following a continuous path from the identity - and it
changes only by an integral multiple of 8\pi ^2 if you do a large
gauge transformation.  Plus, it's diffeomorphism-invariant.  It's
incredibly hard to write down many functions of A with these properties,
so they are precious.  There are deeper reasons why it's so nice, but
let's leave it at that for now.

So, the Chern-Simons state is


$$

                \psi (A) = exp(-6 CS(A)/\Lambda )
$$
    

where \Lambda  is the cosmological constant.  Don't worry
about the factor of 6 too much; depending on how you set up
various things you might get different numbers, and I can
never keep certain factors of 2 straight in this particular
calculation.  Note however that it looks as if things go 
completely haywire as \Lambda  approaches zero, which is why 
my earlier remark about the `flat state' is a bit nontrivial.  

Why does this satisfy the constraints?  Well, I just said above
that the Chern-Simons action was hand-tailored to have the
gauge-invariance and diffeomorphism-invariance we want, so the
only big surprise is that we \emph{also} have a solution of the
Wheeler-DeWitt equation.  Well, we do: a two-line computation
shows it.  

But clearly nature, or at least the goddess of mathematics, is
trying to tell us something if this Chern-Simons state, which
has all sorts of wonderful properties relating to \emph{3-dimensional}
geometry, is also a solution of the Wheeler-DeWitt equation, which
is all about \emph{4-dimensional} geometry, since it expresses
the invariance of \psi  under evolution in TIME.  I have been thinking
about this for several years now and I think I finally really
understand it.  There are probably people out there to whom it's
perfectly obvious, because it's not really all that complicated, but
unfortunately none of these people has ever explained it.  Let me briefly 
say, for those who know about such things, that it all comes down to 
the fact that the Chern-Simons action was \emph{born} as a 3-dimensional 
spinoff of a 4-dimensional thing called the 2nd Chern class.  (If you 
want more details, I explained this as well as I could at the time in 
<A HREF = "knot.tex">knot.tex</A>.)  

What is the physical meaning of the Chern-Simons state?  As far as
I know Kodama's paper hasn't been vastly surpassed in explaining
this.  He shows that in the classical limit this state becomes something
called the anti-deSitter universe, a solution of Einstein's equation
describing a (roughly) exponentially expanding universe.  If you are
wondering what this has to do with Einstein's introduction of the
constant to KEEP the universe from expanding, let me just say this.
In our current big bang theory the universe is expanding, but the
presence of matter, or any sort of positive energy density, tends to pull it
back in, and if there is enough matter it'll collapse again.  Einstein's
stuck in a cosmological constant term to give the vacuum some negative
energy density, exactly enough to counteract the positive energy
density of matter, so things would neither collapse nor expand,
but instead remain in an (unstable, alas) equilibrium.  In the deSitter
universe there's no matter, just a cosmological constant of the opposite 
sign, so that the vacuum has positive energy density.  In the anti-deSitter
universe (invented by deSitter's nemesis anti-deSitter) there's no matter
either, but the cosmological constant has the sign giving the vacuum 
negative energy density, which pushes the universe to keep expanding
faster and faster.  

Now in addition to this physical interpretation, the Chern-Simons
state also has some remarkable relationships to knot theory, which
were discovered by Witten and, since then, studied intensively
by lots of people.  I have written a lot in This Week's Finds about
this!  But briefly, there should be an invariant of knots and links associated
to any state of quantum gravity, and the one associated to the Chern-Simons
state is the Kauffman bracket, a close relative of the Jones polynomial,
which is distinguished by having a very simple, beautiful definition,
and also lots of wonderful relationships to an algebraic structure, 
the quantum group SU_q(2).  I should add that in addition to an invariant
of knots and links, a state of quantum gravity should also give an 
invariant of \emph{spin networks}, and indeed the Kauffman bracket extends
to a wonderful invariant of spin networks.  One can read about this
in many places, but perhaps the most detailed account is Kauffman and
Lins' book referred to in "<A HREF = "week30.html">week30</A>".
  
So the question arises: are all these wonderful features of the 
Chern-Simons state of quantum gravity very special things that 
tell us very little about quantum gravity in general, or are they
important clues that, if we understood them, would reveal a lot
about the nature of \emph{all} states of quantum gravity?

Of course, everyone who has fallen in love with the beauty of
Chern-Simons theory would \emph{like} the answer to be the latter.
Louis Crane, for example, is deeply convinced that Chern-Simons
theory is indeed the key to the whole business.  He has written many
papers on the subject, most of which I've gone over in earlier Finds,
and also one brand new one, which is actually a very readable
introduction to the grand scheme he has in mind:

3) Louis Crane: Clock and category: is quantum gravity algebraic?,
to appear in the November 1995 special issue of Jour. Math. Phys. on
diffeomorphism-invariant physics, preprint available as <A HREF = "http://xxx.lanl.gov/abs/gr-qc/9504038">gr-qc/9504038</A>.

This grand scheme involves a thorough refashioning of quantum gravity 
in terms of category theory, and uses some of the very beautiful mathematics
of n-categories, but neglecting all these subtleties, let us simply say that
he argues that if the universe is IN the Chern-Simons state, there is no
need to understand any other states!  He doesn't really think all there
is in the universe is gravity, of course, so he envisages a souped-up
theory containing other forces and particles, but he argues that a 
generalization of quantum gravity to include all these other forces and
particles will still have a special state, and that that's the state of the 
universe.  

Being a conservative fellow myself, I prefer to remain agnostic
on this issue, but I did write a paper showing how, if you assumed 
that space, the manifold above I called S, is a 3-dimensional sphere -
a so-called S^3 - then if quantum gravity was in the Chern-Simons
state, there would be very nice Hilbert spaces of "relative states" on
each "half" of space.  The relative state notion is often associated
with Everett, who made a big deal out of the fact that, even if
a two-part system was in a single state (a "pure state" in the language
of quantum theory), each part could be regarded as being in a probabilistic
mixture of lots of states (a "mixed state").  Whether or not you like
the "many-worlds interpretation" of quantum theory which Everett's thesis 
spawned, it is true that a single pure state on a two-part system specifies
a whole \emph{space} of states on each half.  So my idea was to take S^3,
arbitrarily split it into two balls (D^3's in math jargon), and work out
the space of states on each half.  If you want to wax rhapsodic of this
you can call one half the "observer" and the other the "observed", though
it's crucial that there is a symmetry interchanging the two - there's
not any way to tell them apart.  

On the more technical side, there is a lot of nice (though well-
understood) knot theory involved.  For example, a special property of
the quantum group SU_q(2) - called the "positivity of the Markov trace", 
and discovered by Jones of Jones polynomial fame - equips each 
space of states with an inner product, even in this situation where 
we have no idea of an inner product to begin with.  This paper is:

4) John Baez, Quantum gravity and the algebra of tangles, Jour. Class.
Quant. Grav. 10 (1993), 673-694, also available (without 
the all-important pictures!) as <A HREF = "tang.tex">tang.tex</A>.

So what has Lee Smolin done?  Actually I have spent so much time
leading up to it that now I'm too tired to do it justice!  So I'll explain
it next time.  But let me just say, in order to tantalize you into tuning
in to the next episode, that he carefully analyzes quantum gravity on
a ball, imposing boundary conditions that let you see why relative
states of Chern-Simons theory give states of quantum gravity.  And
then he makes the hypothesis that I mentioned at the beginning
of this article.  This is that \emph{all} states of quantum gravity
with these boundary conditions come from relative states of
Chern-Simons theory.  He even gives some good evidence for this
hypothesis, coming originally from Hawking's work on the thermal
radiation produced by black holes!  (To be continued.)
\par\noindent\rule{\textwidth}{0.4pt}

% </A>
% </A>
% </A>
