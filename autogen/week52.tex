
% </A>
% </A>
% </A>
\week{May 9, 1995}

So, last "week", I said a bit about how supersymmetry could be
handy for constructing topological quantum field theories, and
this week I want to say a bit more about what that has to do
with getting a purely combinatorial description of Donaldson theory.

But first, I want to lighten things up a bit by mentioning
a good science fiction novel!

1) Permutation City, by Greg Egan, published in Britain
by Millenium (should be available in the U.S. by autumn).

There is a lot of popular interest these days in
the anthropic principle.  Roughly, this claims to explain 
certain features of the universe by noting that if the universe
didn't have those features, there would be no intelligent
life.  So, presumably, the very fact that we are here and
asking certain questions guarantees that the questions will
have certain answers.  

Of course, the anthropic principle is controversial.  Suppose
one could really show that if the universe didn't have property
X, there would be no intelligent life.  Does this really count
as an "explanation" of property X?  People like arguing about
this.  But this question is much too subtle for a simple-minded 
soul such as myself.  I'm still stuck on more basic things!

For example, are there any examples where we \emph{can} really show
that if the universe didn't have property X, there would be
no intelligent life?  It seems that to answer this, we need
to have some idea about what we're counting as "all possible
universes", and what counts as "intelligent life".  So
far we only know ONE example of a universe and ONE example of
intelligent life, so it is difficult to become an expert on these
subjects!  It'd be all to easy for us to unthinkingly assume that
all intelligent life is carbon-based, metabolizes using
oxidation, and eats pizza, just because folks around here do.  

Our unthinking parochialism is probably all the worse as far as 
different universes are concerned!  What counts as a possible universe, 
anyway?  Rather depressingly, we must admit we don't even know
the laws of \emph{this} universe, so we don't really know what it takes
for a universe to be possible, in the strong sense of capable of
actually existing as a universe.  We are a little bit better off
if we consider all \emph{logically possible} universes, but not a whole
lot better.  Certainly every axiom system counts as a logically
possible set of laws of a universe - every set of axioms in every
possible formal system.  But who is to say that universes must have
laws of this form?  We don't even know for sure that \emph{ours} does!

So this whole topic will remain a hopeless quagmire until one takes
a small, carefully limited piece of it and studies that.  People
studying artificial life are addressing one of these bite-sized
pieces, and getting some interesting results.  I hope everyone
has heard about Thomas Ray's program Tierra, for example: he
created an artificial ecosystem - one could call it a "possible
universe" - and found, after seeding it with one self-reproducing
program, a rapid evolution of parasites, etc., following
many of the patterns of ecology here.  But so far, perhaps merely
due to time and memory limitations, no intelligence!  

\emph{One} of the cool things about "Permutation City" is an imagined 
cellular automaton, the "Autoverse", which is complicated enough to 
allow life.  But something much cooler is the main theme of the book.
Egan calls it the "Dust Theory".  It's an absolutely outrageous
theory, but if you think about it carefully, you'll see that it's
rather hard to spot a flaw.  It depends on the tricky
puzzles concealed in the issue of "isomorphism".

Being a mathematician, one thing that always puzzled me about
the notions of "intelligent life" and "all possible universes"
was the question of isomorphisms between universes.  Certainly
we all agree that, say, the Heisenberg "matrix mechanics"
and Schrodinger "wave mechanics" formulations of quantum mechanics 
are isomorphic.  In both of them, the space of states is a Hilbert
space, but in one the states are described as sequences of numbers, while in
the other they are described as wavefunctions.  At first they
look like quite different theories.  But in a while people
realized that there was a unitary operator from Heisenberg's space of 
states to Schrodinger's, and that via this correspondence
all of matrix mechanics is equivalent to wave mechanics.

So does Heisenberg's universe count as the same one as 
Schrodinger's, or a different one?  It seems clear that they're
the same.  But say we had two quantum-mechanical systems
whose Hamiltonians have the same eigenvalues (or spectrum); 
does that mean they are the "same" system, really?  Is that all
there is to a physical system, a list of eigenvalues??? If we are going to go
around talking about "all possible universes", it would probably
pay to think a little about this sort of thing!  

Say we had two candidates for "laws of the universe", written down as 
axioms in different formal systems.   How would we decide if these were 
describing different universes, or were simply different ways of talking 
about the same universe?  Pretty soon it becomes clear that the issue is not 
a black-and-white one of "same" versus "different" universes.  Instead, laws 
of physics, or universes satisfying these laws, can turn out to be isomorphic 
or not depending on how much structure you want the 
isomorphism to preserve.   And even if they are isomorphic, there
may not be a "unique" isomorphism or a "canonical" isomorphism.
(Very roughly speaking, a canonical isomorphism is a "God-given
best one", but one can use some category theory to make this
precise.)  If you think about this carefully you'll see that our
universe could be isomorphic to some very different-seeming ones,
or could have some very different-seeming ones `embedded' in 
it.

Greg Egan takes this issue and runs with it -- in a very interesting
direction.  Everyone interested in cellular automata, artificial life,
virtual reality, or other issues of simulation should read this, as
well as anyone who likes philosophy or just a good story.

Okay, back to business here...

2)  Alberto Cattaneo, Teorie topologiche di tipo BF ed invarianti dei 
nodi, doctoral thesis, department of physics, University of Milan.

Alberto Cattaneo, Paolo Cotta-Ramusino, Juerg Froehlich, 
and Maurizio Martellini, Topological BF theories in 3 and 4
dimensions, preprint available as <A HREF = "http://xxx.lanl.gov/abs/hep-th/9505027">hep-th/9505027</A>.

So, last week I said a wee bit about Blau and Thompson's paper
on supersymmetry and the Casson invariant.  All I said was that
suitably concocted supersymmetric field theories could be used to
compute the Euler characteristics of your favorite spaces, and that
Blau and Thompson talked about one which computed the Casson
invariant, which is (in a rather subtle sense) the Euler characteristic
of the moduli space of flat connections on a trivial SU(2) bundle
over a 3-manifold.  Traditionally one requires that the 3-manifold be
a homology 3-sphere, but Kevin Walker showed
how to do it for rational homology spheres, and Blau and
Thompson mention other work in which the Casson invariant is
generalized still further.  

But I didn't say \emph{which} supersymmetric field theory computes
the Casson invariant for you.  The answer is, N = 2 supersymmetric
BF theory with gauge group SU(2).  So now I should say a little about
BF theory.  Actually I have already mentioned it here and there, especially
in "<A HREF = "week36.html">week36</A>".  But I should say a bit more.  This is going to be pretty 
technical, though, so fasten your seatbelts.

The people I know who are the most excited about BF theory are
the folks I was visiting at Milan, namely Cotta-Ramusino, Martellini
and his student Cattaneo.  They are working on BF theory in 3 and
4 dimensions.  Let me talk about BF theory in 3 dimensions, which is what's
most directly relevant here.  Well, in \emph{any} dimension, say n, the fields
in BF theory are a connection A on a trivial bundle (take your favorite 
gauge group G), whose curvature F we'll think of as a 2-form taking 
values in the Lie algebra of G, and Lie-algebra-valued (n-2)-form B.  
Then the Lagrangian of the theory is

L(B,F) = tr(B ^ F)

where in the "trace" we're using something like the Killing form
to get an honest n-form which we can integrate over spacetime.

But in 3 dimensions, since B is a 1-form, you can add an extra
"cosmological constant" term and take as your Lagrangian

L(B,F,c) = tr(B ^ F + (c^{2}/3) B ^ B ^ B)

where I have put in "c^{2}/3"
as my "cosmological constant" for
insidious reasons to become clear momentarily.  
Now what the article by Cattaneo, Cotta-Ramusino, Froehlich
and Martellini makes really clear is how BF theory is related
to Chern-Simons theory.  This is implicit in Witten's work on 
3d gravity (see "<A HREF = "week16.html">week16</A>"), which is just the special case where G 
is SO(2,1) or SO(3), and where the cosmological constant really 
is the usual cosmological constant.  But I'd never noticed it.  
Recall that the Chern-Simons action is 

L(A) = tr(A ^ dA + (2/3)A ^ A ^ A)

Thus if we have 1-form B around as well, we can set

A' = A + cB,    A'' = A - cB
so we get two different Chern-Simons theories with
actions L(A') and L(A''), respectively. OR, we can form
a theory whose action is the difference of these two, and, lo
and behold:

L(A') - L(A'') =  4cL(B,F,c)

In other words, BF theory with cosmological constant is
just a "difference of two Chern-Simons theories".  Fans of
topological quantum field theory may perhaps be more familiar
with this if I point out that the Turaev-Viro theory is just BF theory
with gauge group SU(2), and the fact that the partition function for 
this theory is the absolute value squared of that for Chern-Simons 
theory is a special case of what I'm talking about.  The nice thing
about all this is that the funny phases coming from framings in 
Chern-Simons theory precisely cancel out when you form this
"difference of two Chern-Simons theories".

Now the Casson invariant is related to BF theory in 3 dimensions 
\emph{without} cosmological constant, i.e., taking c = 0.  We might
get worried by the equation above, which we can't solve
for L(B,F) when c = 0, but as Cattaneo and company point out,


\begin{verbatim}

L(B,F) = lim       L(A') - L(A'')
        c \to  0    -----------------
                         4c
\end{verbatim}
    

so BF theory without cosmological constant is just a limiting
case, actually a kind of \emph{derivative} of Chern-Simons theory.
They use this to make clearer the relation between the vacuum
expectation values of Wilson loops in Chern-Simons theory --
which give you the HOMFLY polynomial for G = SU(N) -- and
the corresponding vacuum expectation values in BF theory without
cosmological constant -- which give you the Alexander polynomial!
Very pretty stuff.

Now back to the Casson invariant and some flagrant speculation
on my part concerning Crane and Frenkel's ideas on Donaldson theory.
(I said last week that this is where I was heading, and now I'm
almost there!)  Okay: we know how to define Chern-Simons theory
in a purely combinatorial way using quantum groups.  I.e., we
can compute the partition function of Chern-Simons theory with
gauge group G using the quantum version of the group G... let
me just call it "quantum G".  If we take c to be imaginary above, 
one can show that BF theory with cosmological constant can be
computed in a very similar way starting with the quantum group
corresponding to the \emph{complexification} of G, i.e. 
"quantum CG".
The point is that A+cB can then be thought of as a connection on 
a bundle with gauge group CG.  So far this is not flagrant
speculation.  Slightly more flagrantly, but not really very much at 
all, the formulas above hint that BF theory without cosmological
constant can be computed in a similar way starting with the
quantum group corresponding to the \emph{tangent bundle} of G, or
"quantum TG".  (The tangent bundle of a Lie group is again
a Lie group, and as we let c \to  0 what we are really doing is
taking a limit in which CG approaches TG; folks call this a
"contraction", and in the SU(2) case many of the details appear
in Witten's paper on 3d quantum gravity; the tangent bundle
of SO(2,1) being just the Poincare group in 3 dimensions.)  If
anyone knows whether folks have worked out the quantization of
these tangent bundle groups, let me know!  I think some examples
have been worked out.

Okay, but Blau and Thompson say that to compute the Casson
invariant you need to use, not BF theory with gauge group SU(2),
but \emph{supersymmetric} BF theory with gauge group SU(2).  Well,
no problemo -- just compute it with "quantum super-T(SU(2))"!  
Here I'm
getting a bit flagrant; there \emph{are} theories of quantum supergroups, 
but I don't know much about them, especially "quantum super-TG"
for G compact semisimple.  Again, if anybody does, please let me know!
(Actually Blau told me to check out a paper by Saleur and somebody
on this, but I never did....)

Okay, but now let's get seriously flagrant.  Recall that the Casson invariant
is really the Euler characteristic of something, just a number, but this
number is just the superdimension of a super-vector-space, namely
the Floer cohomology.  From numbers to vector spaces: this is a typical
sort of "categorification" process that one would expect as one goes
from 3d to 4d TQFTs.  And indeed, folks suspect that the Floer cohomology
is the space of states for a 4d TQFT, or something like a 4d TQFT,
namely Donaldson theory.  ("Something like it" because of many quirky
twists that one wouldn't expect of a full-fledged TQFT satisfying the
Atiyah axioms.)  So, just as the Casson invariant is associated to a certain
Hopf algebra, namely "quantum super-T(SU(2))", we'd expect Donaldson
theory to be associated to a certain Hopf \emph{category}, 
the "categorification
of quantum super-T(SU(2))".  So all we need to do is figure out how
to categorify quantum super-T(SU(2)) and we've got a purely combinatorial
definition of Donaldson theory!

Well, that's not quite so easy, of course.  And I may have made, not
only the inevitable errors involved in painting a simplified sketch of what is
bound to be a rather big task, but also other worse errors.  Still, 
this business should clarify, if only a wee bit, what Crane and Frenkel are up 
to when they are categorifying SU(2).  In fact, it's likely that working with 
SU(2) rather than T(SU(2)) will remove some of the divergences from the state 
sum, since, being compact, SU(2) has a discrete set of representations (and
quantum SU(2) has finitely many interesting ones, at roots of unity).  So
they may get a theory that's allied to but not exactly the same as Donaldson 
theory, yet better-behaved as far as the TQFT axioms go.  

If anyone actually does anything interesting with these ideas I'd
very much appreciate hearing about it.

\par\noindent\rule{\textwidth}{0.4pt}
% </A>
% </A>
% </A>
