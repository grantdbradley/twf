
% </A>
% </A>
% </A>
\week{December 3, 1995 }



This week I will get back to mathematical physics... but before I do, I
can't resist adding that in my talk in that conference I announced that
James Dolan and I had come up with a definition of weak n-categories.
(For what these are supposed to be, and what they have to do with
physics, see "<A HREF = "week38.html">week38</A>", "<A HREF = "week49.html">week49</A>", and "<A HREF = "week53.html">week53</A>".)  John Power was at the
talk, and before long his collaborator Ross Street sent me some email
from Australia asking about the definition.  Gordon,
Power, and Street have done a lot of work on n-categories - see
"<A HREF = "week29.html">week29</A>".  Now, Dolan and I have
been struggling for several months to 
put this definition onto paper in a reasonably elegant and
comprehensible form, so Street's request was a good excuse to get
something done quickly...  sacrificing comprehensibility for terseness.
The result can be found in

1) John Baez and James Dolan, n-Categories, sketch of a definition, 
<A HREF = "http://math.ucr.edu/home/baez/ncat.def.html">http://math.ucr.edu/home/baez/ncat.def.html</A>

A more readable version will appear as a paper fairly soon.  I should
add that this will eventually be part of Dolan's thesis, and he has
done most of the hard work on it; my role has largely been to push him
along and get him to explain everything to me.

On to some physics...

First, it's amusing to note that Maxwell's equations are back in
fashion!  In the following papers you will see how the duality
symmetry of Maxwell's equations (the symmetry between electric and
magnetic fields) plays a new role in modern work on 4-dimensional gauge
theory.  Also, there is some good stuff in Thompson's review article
about the Seiberg-Witten theory, which is basically just a U(1) gauge
theory like Maxwell's equations... but with some important extra twists!

2) Erik Verlinde, Global aspects of electric-magnetic duality, Nuc.
Phys.  B455 (1995), 211-225, available as
<A HREF = "http://arxiv.org/abs/hep-th/9506011">arXiv:hep-th/9506011</A>.

George Thompson, New results in topological field theory and abelian
gauge theory, 64 pages, available as
<A HREF = "http://arxiv.org/abs/hep-th/9511038">arXiv:hep-th/9511038</A>.


Next, it's nice to see that work on the loop representation of quantum
gravity continues apace:

3) Thomas Thiemann, An account of transforms on (A/G)^bar, available
as <A HREF = "http://arxiv.org/abs/gr-qc/9511049">arXiv:gr-qc/9511049</A>.

Thomas Thiemann, Reality conditions inducing transforms for quantum
gauge field theory and quantum gravity, available 
as <A HREF = "http://arxiv.org/abs/gr-qc/9511057">arXiv:gr-qc/9511057</A>.  

Abhay Ashtekar, A generalized Wick transform for gravity, 
available as <A HREF =
"http://arxiv.org/abs/gr-qc/9511083">arXiv:gr-qc/9511083</A>.

Renate Loll, Making quantum gravity calculable, preprint available in
LaTeX form as <A HREF =
"http://arxiv.org/abs/gr-qc/9511080">arXiv:gr-qc/9511080</A>.

Rodolfo Gambini and Jorge Pullin, A rigorous solution of the quantum
Einstein equations, available as <A HREF =
"http://arxiv.org/abs/gr-qc/9511042">arXiv:gr-qc/9511042</A>.

The first three papers here discuss some new work tackling the
"reality conditions" and "Hamiltonian constraint",
two of the toughest issues in the loop representation of quantum
gravity.  First, the Hamiltonian constraint is another name for the
Wheeler-DeWitt equation

                       H \psi  = 0
 
that every physical state of quantum gravity must satisfy (see
"<A HREF = "week11.html">week11</A>" for why).  The
"reality conditions" have to do with the fact that this
constraint looks different depending on whether we are working with
Riemannian or Lorentzian quantum gravity.  The constraint is simpler
when we work with Riemannian quantum gravity.  Classically, in
\emph{Riemannian} gravity the metric on spacetime looks like
 
                dt^{2} + dx^{2} + dy^{2} + dz^{2} 

at any point, if we use suitable local coordinates.  In this Riemannian
world, time is no different from space!  In the real world, the world of
\emph{Lorentzian} gravity, the metric looks like
    
                -dt^{2} + dx^{2} + dy^{2} + dz^{2}

at any point, in suitable coordinates.  Folks often call the
Riemannian world the world of "imaginary time", since in
some vague sense we can get from the Lorentzian world to the
Riemannian world by making the transformation

                         t \to  it,

called a "Wick transform".  It looks simple the way I have
just written it, in local coordinates.  But a priori it's far from
clear that this Wick transform makes any sense globally.  Apparently,
however, there is something we can do along these lines, which
transforms the Hamiltonian for Lorentzian quantum gravity to the
better-understood one of Riemannian quantum gravity!  Alas, I have
been too distracted by n-categories to keep up with the latest work on
this, but I'll catch up in a bit.  Maybe over Christmas I can relax a
bit, lounge in front of a nice warm fire, and read these papers.

The goal of all these machinations, of course, is to find some
equations that make mathematical sense, have a good right to be called
a "quantized version of Einstein's equation", and let one
compute answers to some physics problems.  We don't expect that
quantum gravity is enough to describe what's really going on in
interesting problems...  there are lots of other forces and particles
out there.  Indeed, string theory is founded on the premise that
quantum gravity is completely inseparable from the quantum theories of
everything else.  But here we are taking a different tack, treating
quantum gravity by itself in as simple a way as possible, expecting
that the predictions of theory will be \emph{qualitatively} of great
interest even if they are quantitatively inaccurate.

As described in earlier Finds ("<A HREF = "week55.html">week55</A>", "<A HREF = "week68.html">week68</A>"), Loll has been
working to make quantum gravity "calculable", by working on a
discretized version of the theory called lattice quantum gravity.
If one does it carefully, it's not too bad to treat space as a lattice
in the loop representation of quantum gravity, because even in the
continuum the theory is discrete in a certain sense, since the states
are described by "spin networks", certain graphs embedded in space.
(See "<A HREF = "week43.html">week43</A>" for more on these.)  Her latest paper is an introduction
to some of these issues.  

In a somewhat different vein, Gambini and Pullin have been working on
relating the loop representation to knot theory.  One of their most
intriguing results is that the second coefficient of the
Alexander-Conway knot polynomial is a solution of the Hamiltonian
constraint.  Here they argue for this result using a lattice
regularization of the theory.

Now let me turn to a variety of other matters...

4) Matt Greenwood and Xiao-Song Lin, On Vassiliev knot invariants induced
from finite type, available as
<A HREF = "http://arxiv.org/abs/q-alg/9506001">arXiv:q-alg/9506001</A>.

Lev Rozansky, On finite type invariants of links and rational homology
spheres derived from the Jones polynomial and Witten-
Reshetikhin-Turaev invariant, available as <A HREF =
"http://arxiv.org/abs/q-alg/9511025">arXiv:q-alg/9511025</A>.

Scott Axelrod, Overview and warmup example for perturbation theory with
instantons, available as 
<A HREF = "http://arxiv.org/abs/hep-th/9511196">arXiv:hep-th/9511196</A>.  

These papers all deal with perturbative Chern-Simons theory and its
spinoffs.  The first two consider homology 3-spheres.  In Chern-Simons
theory, this makes the moduli space of flat SU(2) connections trivial,
thus eliminating some subtleties in the perturbation theory.  A homology
3-sphere is a 3-manifold whose homology is the same as that of the
3-sphere... the first one was discovered by Poincare when he was
studying his conjecture that every 3-manifold with the homology of a
3-sphere is a 3-sphere. It turns out that you can get a counterexample
if you just take an ordinary 3-sphere, cut out a solid torus embedded in
the shape of a trefoil knot, and stick it back in with the meridian and
longitude (the short way around, and the long way around) switched ---
making sure they wind up pointing in the correct directions.  This is
called "doing Dehn surgery on the trefoil".  It gives something
with the homology of a 3-sphere that's not a 3-sphere.  So Poincare had
to revise his conjecture to say that every 3-manifold \emph{homotopic} to a
3-sphere is (homeomorphic to) a 3-sphere.  This improved "Poincare
conjecture" remains unsolved... its analog is known to be true in every
dimension other than 3!  Since every possible counterexample to the
Poincare conjecture is a homology 3-sphere, it makes some sense to ponder
these manifolds carefully.

Now, just as perturbative Chern-Simons theory gives certain special
invariants of links, said to be of "finite type", the same is
true for homology 3-spheres.  When we say a link invariants is of finite
type, all we mean is that it satisfies a simple property described in
"<A HREF = "week3.html">week3</A>".  There is a similar but
subtler definition for an invariant of homology 3-spheres to be of
finite type; they need to transform in a nice way under Dehn surgery.
(See "<A HREF = "week60.html">week60</A>" for more references.)

The second paper concentrates precisely on those subtleties due to the
moduli space of flat connections, developing perturbation theory in the
presence of "instantons" (here, nontrivial flat connections).  

5) Alan Carey, Jouko Mickelsson, and Michael Murray, Index theory,
gerbes, and Hamiltonian quantization, available as <A HREF =
"http://arxiv.org/abs/hep-th/9511151">arXiv:hep-th/9511151</A>.

Alan Carey, M. K. Murray and B. L. Wang, Higher bundle gerbes and
cohomology classes in gauge theories, available as
<A HREF = "http://arxiv.org/abs/hep-th/9511169">arXiv:hep-th/9511169</A>


Higher-dimensional algebra is sneaking into physics in yet another way:
gerbs!  What's a gerb?  Roughly speaking, it's a sheaf of groupoids, but
there are some other ways of thinking of them that come in handy in
physics.  See "<A HREF = "week25.html">week25</A>" for a
review of Brylinski's excellent book on gerbs, and also:


6) Jean-Luc Brylinski, Holomorphic gerbes and the Beilinson regulator,
in Proc. Int. Conf. on K-Theory (Strasbourg, 1992), to appear in
Asterisque.  

Jean-Luc Brylinski, The geometry of degree-four characteristic classes
and of line bundles on loop spaces I, Duke Math. Jour. 75 (1994),
603-638.  

Jean-Luc Brylinski, Cech cocyles for characteristic classes, J.-L.
Brylinski and D. A. McLaughlin.

7) Joe Polchinski, Recent results in string duality, available
as <A HREF = "http://arxiv.org/abs/hep-th/9511157">arXiv:hep-th/9511157</A>.

This should help folks keep up with the ongoing burst of work on
dualities relating superficially different string theories. 


8)  Leonard Susskind and John Uglum, String physics and black holes, 
available as <A HREF = "http://arxiv.org/abs/hep-th/9511227">arXiv:hep-th/9511227</A>.

Among other things, this review discusses 
the "holographic hypothesis" 
mentioned in "<A HREF = "week57.html">week57</A>":

9) Boguslaw Broda, A gauge-field approach to 3- and 4-manifold
invariants, available in TeX form as <A HREF = "http://arxiv.org/abs/q-alg/9511010">arXiv:q-alg/9511010</A>.  

This summarizes the Reshetikhin-Turaev construction of 3d topological
quantum field theories from quantum groups, and Broda's own closely
related approach to 4d topological quantum field theories.  

10) John Baez and Martin Neuchl, Higher-dimensional algebra I: braided
monoidal 2-categories, available as <A HREF =
"http://arxiv.org/abs/q-alg/9511013">arXiv:q-alg/9511013</A>.

In this paper, we begin with a brief sketch of what is known and
conjectured concerning 
braided monoidal 2-categories and their applications to 4d topological
quantum field theories and 2-tangles (surfaces embedded in 4-dimensional
space).  Then we give concise definitions of semistrict monoidal
2-categories and braided monoidal 2-categories, and show how these may
be unpacked to give long explicit definitions similar to, but not quite
the same as, those given by Kapranov and Voevodsky.  Finally, we describe
how to construct a semistrict braided monoidal 2-category Z(C) as the
`center' of a semistrict monoidal category C.  This is analogous to the
construction of a braided monoidal category as the center, or `quantum
double', of a monoidal category.  The idea is to develop algebra that
will do for 4-dimensional topology what quantum groups and braided
monoidal categories did for 3d topology.  As a corollary of the center
construction, we prove a strictification theorem for braided monoidal
2-categories.   

\par\noindent\rule{\textwidth}{0.4pt}

% </A>
% </A>
% </A>
