
% </A>
% </A>
% </A>
\week{February 27, 1997}


I feel guilty for slacking off on This Week's Finds, so I should
explain the reason.  Lots of things have been building up that I'm
dying to talk about, but new ones keep coming in at such a rapid rate
that I never feel I have time!

I will have to be ruthless, and face up to the fact that a quick and
imperfect exposition is better than none.

First of all, here at the Center for Gravitational Physics and 
Geometry there are a lot of interesting attempts going on to compute
the entropy of black holes from first principles.   Bekenstein, 
Hawking and many others have used a wide variety of semiclassical arguments 
to argue that black holes satisfy

S = A/4

where S is the entropy and A is the area of the event horizon, both
measured in Planck's units, where G = c = \hbar  = 1.  

For example, using purely classical reasoning (general relativity, but
no quantum theory) one can prove the "2nd law of black hole
thermodynamics", which says that A always increases.  As Bekenstein
noted, this suggests that the area of the event horizon is somehow
analogous to entropy.  However, by itself this does not determine 
the magic number 1/4, which can only be derived using quantum theory
(as one can see by simple dimensional analysis).

By semiclassical reasoning - studying quantum electrodynamics in
the Schwarzschild metric used to describe black holes - Hawking showed
that black holes should radiate as if they had a temperature inversely
proportional to their mass:

T = 1/(8 \pi  M)

This made the analogy between entropy and event horizon area much more
than an analogy, because it meant that one could assign a temperature
to black holes and see if they satisfy the laws of thermodynamics.  It
turns out that if you consider A/4 to be the entropy of a black hole,
you can eliminate seeming violations of the 2nd law that otherwise
arise in thought experiments where you get rid of entropy by throwing
it into a black hole.  In other words, if you throw something with
entropy S into a black hole, calculations seem to show that the area
of the event horizon always increases by at least 4S!

So far nothing I've said is related to full-fledged quantum gravity,
because in the semiclassical arguments one is still working in the
approximation where the gravitational field is treated classically.
An interesting test of any theory of quantum gravity is whether can use
it to derive S = A/4.  In a subject with no real experimental evidence,
this is the closest we have to an "experimental result" that our theory
should predict.

Recently the string theorists have done some calculations claiming to
show that string theory predicts S = A/4.  Personally I feel that
while these calculations are interesting they are far from definitive.
For example, they all involve taking calculations done using
perturbative string theory on \emph{flat} spacetime and extrapolating them
drastically to the regime in which string theory approximates general
relativity.  One typically uses ideas from supersymmetry to justify
such extrapolations; however, these ideas only seem to apply to
"extremal black holes", having the maximum possible charge for a black
hole of a given mass and angular momentum.  Realistic black holes are
far from extremal.  In short, while exciting, these calculations
still need to be taken with a grain of salt.  

Of course, I am biased because I am interested in another approach to
quantum gravity, the loop representation of quantum gravity, which
folks are working on here at the CGPG, among other places.  This is in
many ways a more conservative approach.  The idea is to simply take
Einstein's equation for general relativity and quantize it, rather
than trying to develop a theory of \emph{all} particles and forces as in
string theory.  Of course, for various reasons it is not so easy to
quantize Einstein's equation.  String theorists think it's
\emph{impossible} without dragging in all sorts of other forces and
particles, but folks working on the loop representation are more
optimistic.  This is an ongoing argument, but certainly a good test of
the loop representation would be to try to use it to derive Hawking's
formula S = A/4.  If the loop representation is really any good, this
should be possible, because many different lines of reasoning using
only general relativity and quantum theory lead to this formula.

I've already mentioned a few attempts to do this in "<A HREF = "week56.html">week56</A>","<A HREF = "week57.html">week57</A>", 
and "<A HREF = "week87.html">week87</A>".  These were promising, but they didn't get the magical
number 1/4.   Also, they are rather rough, in that they do computations
on some region with boundary, but don't use anything that ensures
the boundary is an event horizon.  

Recently Kirill Krasnov has made some progress:

1) Kirill Krasnov, On statistical mechanics of Schwarzschild black
hole, preprint available as <A HREF = "http://xxx.lanl.gov/ps/gr-qc/9605047">gr-qc/9605047</A>.

This paper still doesn't get the magic number 1/4, and Krasnov later
realized it has a few mistakes in it, but it does something very cool.
It notes that the boundary conditions holding on the event horizon of
a Schwarzschild black hole are closely related to Chern-Simons theory.
Now is not the time for me to go into Chern-Simons theory, but
basically, it lets you apply a lot of neat mathematics to calculate
everything to your heart's content, very much as Carlip did on his
work on the toy model of a 2+1-dimensional black hole (see "<A HREF = "week41.html">week41</A>").
Also, it sheds new light on the relationship between topological
quantum field theory and quantum gravity, something I am always trying
to understand better.

While I'm at it, I should note the existence of a paper that reworks
Carlip's calculation from a slightly different angle:

2) Maximo Banados and Andres Gomberoff, Black hole entropy in the 
Chern-Simons formulation of 2+1 gravity, preprint available as
<A HREF = "http://xxx.lanl.gov/ps/gr-qc/9611044">gr-qc/9611044</A>.

2+1-dimensional quantum gravity is very simple compared to the
3+1-dimensional quantum gravity we'd really like to understand: in a
sense it's "exactly solvable".  But there are still some puzzling
things about Carlip's computation of the entropy of a black hole in
2+1 dimensions which need figuring out, so every paper on the subject
is worth looking at, if you're interested in black hole entropy.

Speaking of topological quantum field theory and quantum gravity, I just
finished a paper on these topics:

3) John Baez, Degenerate solutions of general relativity from topological
field theory, preprint available as <A HREF = "http://xxx.lanl.gov/ps/gr-qc/9702051">gr-qc/9702051</A> or in Postscript
form at <A HREF = "http://math.ucr.edu/home/baez/deg.ps">http://math.ucr.edu/home/baez/deg.ps</A>.

Let me just summarize the basic idea, resisting the temptation to 
become insanely technical.  

A while ago Rovelli and Smolin introduced Penrose's notion of "spin
network" into the loop representation of quantum gravity.  I described
spin networks pretty carefully in "<A HREF = "week43.html">week43</A>", but here let me just say
that they are graphs embedded in space with edges labelled by spins j
= 0, 1/2, 1, 3/2, and so on, just as in the quantum mechanics of
angular momentum, and with vertices labelled by "intertwining
operators", which are other gadgets that come up in the study of
angular momentum.  In the loop representation these spin networks form
a basis of states.  Geometrical observables like the area of surfaces
and the volumes of regions have been quantized and their matrix
elements computed in the spin network basis, giving us a nice picture
of "quantum 3-geometries", that is, the possible geometries of space
in the context of quantum gravity.  In this picture, the edges of spin
networks play the role of quantized flux tubes of area, much as the
magnetic field comes in quantized flux tubes in a type II
superconductor.  To work out the area of a surface in some spin
network state, you just total up contributions from each edge of the
spin network that pokes through the surface.  An edge labelled with
spin j carries an area equal to sqrt(j(j+1)) times the Planck length
squared.  What's cool is that this is not merely postulated, it's 
derived from fairly standard ideas about how you turn observables into
operators in quantum mechanics.  

However, the dynamics of quantum gravity is more obscure.  Technical
issues aside, the main problem is that while we have a nice picture of
quantum 3-geometries, we don't have a similar picture of the
\emph{4-dimensional}, or \emph{spacetime}, aspects of the theory.  To represent
a physical state of quantum gravity, a spin network state (or linear
combination thereof) has to satisfy something called the
Wheeler-DeWitt equation.  This is sort of the quantum gravity analog
of the Schrodinger equation.  There is a lot of controversy over the
Wheeler-DeWitt equation and what's the right way to write it down in
the loop representation.  The really annoying thing, however, is that
even if you feel you know how to write it down - for example, Thomas
Thiemann has worked out one way (see "<A HREF = "week85.html">week85</A>") - and can find
solutions, you still don't necessarily have a good intuition as to
what the solutions \emph{mean}.  For example, almost everyone seems to
agree that spin networks with no vertices should satisfy the
Wheeler-DeWitt equation.  These are just knots or links with edges are
labelled by spins.  We know these states are supposed to represent
"quantum 4-geometries" satisfying the quantized Einstein equations.
But how should we visualize these states in 4-dimensional terms?

In search of some insight into the 4-dimensional interpretation of
these states, I turn to classical general relativity.  In my paper, I
construct solutions of the equations of general relativity which at a
typical fixed time look like "flux tubes of area" reminiscent of the
loop states of quantum gravity.  These are "degenerate solutions",
meaning that the "3-metric", the tensor you use to measure distances
in 3-dimensional space, is zero in lots of regions of space.  Here I
should warn you that ordinary general relativity doesn't allow
degenerate metrics like this.  The loop representation works with an
extension of general relativity that covers the case of degenerate
metrics; for more on this, see "<A HREF = "week88.html">week88</A>".

More precisely, if you look at these "flux tube" solutions at a
typical time, the 3-metric vanishes outside a collection of solid tori
embedded in space, while inside any of these solid tori the metric is
degenerate in the longitudinal direction, but nondegenerate in the two
transverse directions.

Now since these are classical solutions - no quantum theory in sight! -
there is no problem with understanding what they do as time passes.
We can solve Einstein's equation and get a 4-metric, a metric on
spacetime.  The 4-dimensional picture is as follows: given any surface
\Sigma  embedded in spacetime, I get solutions for which the 4-metric
vanishes outside a neighborhood of \Sigma .  Inside this neighborhood,
the 4-metric is zero in the two directions tangent to \Sigma  but
nondegenerate in the two transverse directions.  In the 4-geometry
defined by one of these solutions, the area of a typical surface
\Sigma ' intersecting \Sigma  in some isolated points is a sum of
contributions from the points where \Sigma  and \Sigma ' intersect.

The solutions I study are inspired by the work of Mike Reisenberger,
who studied a solution for which the metric vanishes outside a
neighborhood of a sphere embedded in R^4.  I consider more general
surfaces embedded in more general 4-manifolds, so I need to worry a
lot more about topological issues.  Also, I allow the possibility of a
nonzero cosmological constant (this being a parameter in Einstein's
equation that determines the energy density of the vacuum).  A lot of
the most interesting stuff happens for nonzero cosmological constant,
and this case actually helps one understand the case of vanishing
cosmological constant as a kind of limiting case.

It turns out that the interesting degrees of freedom of the metric
living on the surface \Sigma  in spacetime are described by fields
living on this surface.  In fact, these fields are solutions of a
2-dimensional topological field theory called BF theory.  To prove
this, I take advantage of the relation between general relativity and
BF theory in 4 dimensions, together with the fact that BF theory
behaves in a simple manner under dimensional reduction.

Another neat thing is that to get a solution of general relativity
this way, we need to pick a "framing" of \Sigma .  Roughly speaking,
this means we need to pick a way of thickening up the surface \Sigma  to
a neighborhood that looks like \Sigma  x D^2, where D^2 is the
2-dimensional disc.  This is precisely the 4-dimensional analog of a
framing of a knot or link in 3-dimensions.  People who know about
topological quantum field theory know that framings are very
important.  In fact, I can show that my solutions of general
relativity are closely related to Chern-Simons theory, a 3-dimensional
topological field theory famous for giving invariants of framed knots
and links.  What's beginning to emerge is a picture that makes the
\emph{spacetime} aspects of framings easier to understand.

Now before I plunge into some even more esoteric stuff, let me
briefly return to reality and answer the question you've all been
secretly dying to ask: how does general relativity impact the world
of big business?  

In plain terms: is all this fancy physics just an excuse to have fun
visualizing evolving spin networks in terms of quantum field theories
on surfaces embedded in 4-dimensional spacetime, etcetera
etcetera... or does it actually contribute to the well-being of the
corporations upon which we depend?

Well, you may be surprised to know that general relativity plays an
significant role in a $200-million business.  Surprised?  Read on!
What follows is taken from the latest issue of "Matters of Gravity",
the newsletter put out by Jorge Pullin.  More precisely, it's from:

4) Neil Ashby, General relativity in the global positioning system,
in Matters of Gravity, ed. Jorge Pullin, no. 9, available at
<A HREF = "http://www.phys.lsu.edu//mog/mog9/node9.html">http://www.phys.lsu.edu//mog/mog9/node9.html</A>.

I will simply quote some excerpts from this fascinating article:


\begin{verbatim}

"The Global Position System (GPS) consists of 24 earth-orbiting
satellites, each carrying accurate, stable atomic clocks.  Four
satellites are in each of six different orbital planes, of inclination
55 degrees with respect to earth's equator.  Orbital periods are 12
hours (sidereal), so that the apparent position of a satellite against
the background of stars repeats in 12 hours.  Clock-driven
transmitters send out synchronous time signals, tagged with the
position and time of the transmission event, so that a receiver near
the earth can determine its position and time by decoding navigation
messages from four satellites to find the transmission event
coordinates, and then solving four simultaneous one-way signal
propagation equations.  Conversely, \gamma -ray detectors on the
satellites could determine the space-time coordinates of a nuclear
event by measuring signal arrival times and solving four one-way
propagation delay equations.

Apart possibly from high-energy accelerators, there are no other
engineering systems in existence today in which both special and
general relativity have so many applications.  The system is based on
the principle of the constancy of c in a local inertial frame: the
Earth-Centered Inertial or ECI frame.  Time dilation of moving clocks
is significant for clocks in the satellites as well as clocks at rest
on earth.  The weak principle of equivalence finds expression in the
presence of several sources of large gravitational frequency
shifts.  Also, because the earth and its satellites are in free fall,
gravitational frequency shifts arising from the tidal potentials of
the moon and sun are only a few parts in 10^16 and can be neglected.

[...]

At the time of launch of the first NTS-2 satellite (June 1977), which
contained the first Cesium clock to be placed in orbit, there were
some who doubted that relativistic effects were real.  A frequency
synthesizer was built into the satellite clock system so that after
launch, if in fact the rate of the clock in its final orbit was that
predicted by GR, then the synthesizer could be turned on bringing the
clock to the coordinate rate necessary for operation.  The atomic
clock was first operated for about 20 days to measure its clock rate
before turning on the synthesizer.  The frequency measured during that
interval was +442.5 parts in 10^12 faster than clocks on the ground;
if left uncorrected this would have resulted in timing errors of about
38,000 nanoseconds per day. The difference between predicted and
measured values of the frequency shift was only 3.97 parts in 10^12,
well within the accuracy capabilities of the orbiting clock.  This then
gave about a 1% validation of the combined motional and gravitational
shifts for a clock at 4.2 earth radii.

[...]

This system was intended primarily for navigation by military users
having access to encrypted satellite transmissions which are not
available to civilian users.  Uncertainty of position determination in
real time by using the Precise Positioning code is now about 2.4
meters.  Averaging over time and over many satellites reduces this
uncertainty to the point where some users are currently interested in
modelling many effects down to the millimeter level.  Even without
this impetus, the GPS provides a rich source of examples for the
applications of the concepts of relativity.

New and surprising applications of position determination and time
transfer based on GPS are continually being invented.  Civilian
applications include for example, tracking elephants in Africa,
studies of crustal plate movements, surveying, mapping, exploration,
salvage in the open ocean, vehicle fleet tracking, search and rescue,
power line fault location, and synchronization of telecommunications
nodes.  About 60 manufacturers now produce over 350 different
commercial GPS products.  Millions of receivers are being made each
year; prices of receivers at local hardware stores start in the
neighborhood of $200."

\end{verbatim}
    

Pretty cool, eh?  

Okay, now for something completely different - homotopy theory!  Well,
everything I write about is actually secretly part of my grand plan to
see how everything interesting is related to everything else, but let
me not delve into how homotopy theory is related to topological
quantum field theory and thus quantum gravity.  Let me simply mention
the existence of this great book:

5) "Handbook of Algebraic Topology", ed. I. M. James, North-Holland,
the Netherlands, 1995, 1324 pages.

Occaisionally you come across a book that you wish you just download
into your brain; for me this is one of those books.  It is probably
not a good idea to read it if you are just wanting to get started
on algebraic topology; it assumes you are pretty familiar with the
basic ideas already, and it goes into a lot of depth, mainly in hardcore
homotopy theory.  A lot of it is too technical for me to appreciate,
but let me list a few chapters that I can understand and like.

Chapter 1, "Homotopy types" by Hans-Joachim Baues, is a great survey
of different models of homotopy types.  Remember, we say two
topological spaces X and Y are homotopy equivalent if there are
continuous functions f: X \to  Y and g: Y \to  X that are inverses "up to
homotopy".  In other words, we don't require that fg and gf are
\emph{equal} to identity functions, but merely that they can both be
\emph{continuously deformed} to identity functions.  So for example the
circle and an annulus are homotopy equivalent, and we say therefore
that they represent the same "homotopy type".

The cool thing is that there turn out to be very elegant algebraic and
combinatorial ways of describing homotopy types that don't mention
topology at all.  Perhaps the most beautiful way of all is a way that
in a sense hasn't been fully worked out yet: namely, thinking of
homotopy types as "\omega -groupoids".  The idea is this.  An
"\omega -category" is something that has

     objects like x
     morphisms between objects like f: x \to  y
     2-morphisms between morphisms like F: f \to  g
     3-morphisms between 2-morphisms like T: F \to  G
     ...

and so on ad infinitum.  There should be some ways of composing these,
and these should satisfy some axioms, and that of course is the tricky
part.  But the basic idea is that if you hand me a topological space
X, I can cook up an \omega -category whose

    objects are points in X
    morphisms are paths between points in X
    2-morphisms are continuous 1-parameter families of paths in X, i.e. 
    "paths of paths" in X
    3-morphisms are "paths of paths of paths" in X
    ...

and so on.  This is better than your garden-variety \omega -category
because all the morphisms and 2-morphisms and 3-morphisms and so on
have inverses, at least "up to homotopy".  We call it an
"\omega -groupoid".  This \omega -groupoid keeps track of the homotopy
type of X in a very nice way.  (If this "\omega " stuff is too
mind-boggling, you may want to start by reading a bit about plain old
categories and groupoids in "<A HREF = "week74.html">week74</A>".)

Conversely, given any \omega -groupoid there should be a nice way to
cook up a homotopy corresponding to it.  This is just the infinite-
dimensional generalization of something I described in "<A HREF = "week75.html">week75</A>".
There, I showed how you could get a groupoid from a "homotopy 1-type"
and vice versa.  Here there 1-morphisms but no interesting
2-morphisms, 3-morphisms, and so on, because the topology of a
"homotopy 1-type" is boring in dimensions greater than 1.  (In case
any experts are reading this, what I mean is that its higher homotopy
groups are trivial; its higher homology and cohomology groups can be
very interesting.)

So we can - and should - think of homotopy theory as, among other
things, the study of \omega -groupoids, and thus a very useful warmup
to the study of \omega -categories.  In my occaisional series on This
Week's Finds called "the tale of n-categories", I have tried to explain
why n-categories, and ultimately \omega -categories, should serve as 
a powerful unifying approach to lots of mathematics and physics.  
In trying to understand this subject, I find time and time again that
homotopy theorists are the ones to listen to.

Chapter 2, "Homotopy theories and model categories", by W. G. Dwyer
and J. Spalinski, is a nice introduction to the formal idea of using
different "models" for homotopy types.  For example, above I was
sketching how one might do homotopy theory using the "model category"
of \omega -groupoids.  Other model categories include gadgets like Kan
complexes, CW complexes, simplicial complexes, and so on.  

Chapter 6, "Modern foundations for stable homotopy theory", by
A. D. Elmendorf, I. Kriz, M. Mandell and J. P. May describes a very
nice approach to spectra.  Loosely speaking, we can think of a
spectrum as a Z-groupoid, where Z denotes the integers.  In other
words, in addition to j-morphisms for all natural numbers j, we also
have j-morphisms for negative j!  This may seem bizarre, but it's a
lot like how in homology theory one is interested in chain complexes
that extend in both the positive and negative directions.  In fact,
we can think of a chain complex as a very special sort of Z-groupoid
or spectrum.  The study of spectra is called stable homotopy theory.

Chapter 13, "Stable homotopy and iterated loop spaces", by G. Carlsson
and R. J. Milgram, is packed with handy information about stable
homotopy theory.

Chapter 21, "Classifying spaces of compact Lie groups and finite loop
spaces", by D. Notbohm, is a good source of heavy-duty information on
classifying spaces of your favorite Lie groups.  To study vector
bundles and the like one really needs to become comfortable with
classifying spaces, and I'm finally doing this, and I hope eventually
I'll be comfortable enough with them to really understand all these
results.

There is a lot more, but I will stop here.

\par\noindent\rule{\textwidth}{0.4pt}

% </A>
% </A>
% </A>
