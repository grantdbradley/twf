
% </A>
% </A>
% </A>
\week{September 12, 1998 }

I haven't been going to the Newton Institute much during my stay 
in Cambridge, even though it's right around the corner from where 
I live.  There's always interesting math and physics going on
at the Newton Institute, and this summer they had some conferences 
on cosmology, but I've been trying to get away from it all for a 
while.  Still, I couldn't couldn't resist the opportunity to go to 
James Hartle's 60th birthday party, which was held there on September 
2nd.  

Hartle is famous for his work on quantum gravity and the foundations 
of quantum mechanics, so some physics bigshots came and gave talks.  
First Chris Isham spoke on applications of topos theory to quantum 
mechanics, particularly in relation to Hartle's work on the so-called 
"decoherent histories" approach to quantum mechanics, which he developed 
with Murray Gell-Mann.  Then Roger Penrose spoke on his ideas of 
gravitationally induced collapse of the wavefunction.  Then Gary Gibbons 
spoke on the geometry of quantum mechanics.  All very nice talks!

Finally, Stephen Hawking spoke on his work with Hartle.  This talk was 
the most personal in nature: Hawking interspersed technical descriptions 
of the papers they wrote together with humorous reminiscences of their 
get-togethers in Santa Barbara and elsewhere, including a long trip in a 
Volkswagen beetle with Hawking's wheelchair crammed into the back seat.  

I think Hawking said he wrote 4 papers with Hartle.  The first really 
important one was this:

1) James Hartle and Stephen Hawking, Path integral derivation of 
black hole radiance, Phys. Rev. D13 (1976), 2188.

As I explained in "<A HREF = "week111.html">week111</A>", Hawking wrote a paper in 1975 establishing 
a remarkable link between black hole physics and thermodynamics.  He 
showed that a black hole emits radiation just as if it had a temperature 
inversely proportional to its mass.  However, this paper was regarded 
with some suspicion at the time, not only because the result was so 
amazing, but because the calculation involved modes of the electromagnetic 
field of extremely short wavelengths near the event horizon - much shorter 
than the Planck length.

For this reason, Hartle and Hawking decided to redo the calculation
using path integrals - a widely accepted technique in particle physics.
Hawking's background was in general relativity, so he wasn't too good 
at path integrals; Hartle had more experience with particle physics 
and knew how to do that kind of stuff.  
 
(By now, of course, Hawking can do path integrals quicker than most
folks can balance their checkbook.  This was a while ago.)

This wasn't straightforward.  In particle physics people usually do 
calculations assuming spacetime is flat, so Hartle and Hawking needed 
to adapt the usual path-integral techniques to the case when spacetime 
contains a black hole.  The usual trick in path integrals is to replace 
the time variable t by an imaginary number, then do the calculation, and 
then analytically continue the answer back to real times.  This isn't so 
easy when there's a black hole around!  

For starters, you have to analytically continue the Schwarzschild solution 
(the usual metric for a nonrotating black hole) to imaginary values of the 
time variable.  When you do this, something curious happens: you find that 
the Schwarzschild solution is periodic in the imaginary time direction.
And the period is proportional to the black hole's mass.

Now, if you are good at physics, you know that doing quantum field
theory calculations where imaginary time is periodic with period 1/T
is the same as doing statistical mechanics calculations where the 
temperature is T.  So right away, you see that a black hole acts like
it has a temperature inversely proportional to its mass!  

(In case you're worried, I'm using units where \hbar , c, G, and k are
equal to 1.)

Anyway, that's how people think about the Hartle-Hawking paper these
days.  I haven't actually read it, so my description may be a bit
anachronistic.  Things usually look simpler and clearer in retrospect.

The other really important paper by Hartle and Hawking is this one:

2) James Hartle and Stephen Hawking, Wavefunction of the universe,
Phys. Rev. D28 (1983), 2960.

In quantum mechanics, we often describe the state of a physical
system by a wavefunction - a complex-valued function on the 
classical configuration space.  If quantum mechanics applies to
the whole universe, this naturally leads to the question: what's
the wavefunction of the universe?  In the above paper, Hartle
and Hawking propose an answer.

Now, it might seem a bit overly ambitious to guess the wavefunction
of the entire universe, since we haven't even seen the whole thing yet.  
And indeed, if someone claims to know the wavefunction of the whole 
universe, you might think they were claiming to know everything
that has happened or will happen.  Which naturally led Gell-Mann to 
ask Hartle: "If you know the wavefunction of the universe, why aren't
you rich yet?"

But the funny thing about quantum theory is that, thanks to the
uncertainty principle, you can know the wavefunction of the universe,
and still be completely clueless as to which horse will win at the
races tomorrow, or even how many planets orbit the sun.  

That will either make sense to you, or it won't, and I'm not sure
anything \emph{short} I might write will help you understand it if you 
don't already.  A full explanation of this business would lead me 
down paths I don't want to tread just now - right into that morass
they call "the interpretation of quantum mechanics".

So instead of worrying too much about what it would \emph{mean} to know the
wavefunction of the universe, let me just explain Hartle and Hawking's
formula for it.  Mind you, this formula may or may not be correct, or 
even rigorously well-defined - there's been a lot of argument about it 
in the physics literature.  However, it's pretty cool, and definitely
worth knowing. 

Here things get a wee bit more technical.  Suppose that space is a 
3-sphere, say X.  The classical configuration space of general relativity 
is the space of metrics on X.  The wavefunction of the universe should 
be some complex-valued function on this classical configuration space.  
And here's Hartle and Hawking's formula for it:


\begin{verbatim}

              \psi (q) = integral exp(-S(g)/\hbar ) dg
                       g|X = q

\end{verbatim}
    
Now you can wow your friends by writing down this formula and
saying "Here's the wavefunction of the universe!"  

But, what does it mean? 

Well, the integral is taken over Riemannian metrics g on a 4-ball
whose boundary is X, but we only integrate over metrics that
restrict to a given metric q on X - that's what I mean by writing
g|X = q.  The quantity S(g) is the Einstein-Hilbert action of the 
metric g - in other words, the integral of the Ricci scalar curvature
of g over the 4-ball.  Finally, of course, \hbar  is Planck's constant.

The idea is that, formally at least, this wavefunction is a solution of
the Wheeler-DeWitt equation, which is the basic equation of quantum
gravity (see "<A HREF = "week43.html">week43</A>").

The measure "dg" is, unfortunately, ill-defined!  In other words, one
needs to use lots of clever tricks to extract physics from this formula, 
as usual for path integrals.  But one can do it, and Hawking and others 
have spent a lot of time ever since 1983 doing exactly this.  This led 
to a subject called "quantum cosmology".

I should add that there are lots of ways to soup up the basic
Hartle-Hawking formula.  If we have other fields around besides
gravity, we just throw them into the action in the action in the
obvious way and integrate over them too.  If our manifold X representing
space is not a 3-sphere, we can pick some other 4-manifold having it
as boundary.  If we can't make up our mind which 4-manifold to use, 
we can try a "sum over topologies", summing over all 4-manifolds
with X as boundary.  We can do this even when X is a 3-sphere, 
actually - but it's a bit controversial whether we should, and 
also whether the sum converges.  

Well, there's a lot more to say, like what the physical interpretation
of the Hartle-Hawking formula is, and what predicts.  It's actually quite 
cool - in a sense, it says that the universe tunnelled into being out of 
nothingness!  But that sounds like a bunch of nonsense - the sort of fluff 
they write on the front of science magazines to sell copies.  To really 
explain it takes quite a bit more work.  And unfortunately, it's just 
about dinner-time, so I want to stop now.  

Anyway, it was an interesting birthday party.  


 \par\noindent\rule{\textwidth}{0.4pt}

% </A>
% </A>
% </A>
