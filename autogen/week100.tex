
% </A>
% </A>
% </A>
\week{March 23, 1997}

% <A NAME = "tale">
Pretty much ever since I started writing "This Week's Finds" I've been
trying to get folks interested in n-categories and other aspects of
higher-dimensional algebra.  There is really an enormous world out
there that only becomes visible when you break out of "linear
thinking" - the mental habits that go along with doing math by writing
strings of symbols in a line.  For example, when we write things in a
line, the sums a+b and b+a look very different.  Then we introduce
a profound and mysterious equation, the "commutative law":


\begin{verbatim}

a + b = b + a
\end{verbatim}
    
which says that actually they are the same.  But in real life,
we prove this equation using higher-dimensional reasoning:
          

\begin{verbatim}

         a           a          a
a + b  =   +    =    +   =    +    =  b + a
             b       b      b
\end{verbatim}
    

If this seems silly, think about explaining to a kid what 9+17 means,
and how you could prove that 9+17 = 17+9.  You might take a pile of 9
rocks and set it to the left of a pile of 17 rocks, and say "this is
9+17 rocks".  Alternatively, you might put the pile of 9 rocks to the
right of the pile of 17 rocks, and say "this is 17+9 rocks".  Thus to
prove that 9+17=17+9, you would simply need to \emph{switch} the two piles
by moving one around the other.  

This is all very simple.  Historically, however, it took people a long
to really understand.  It's one of those things that's too simple to
take seriously until it turns out to have complicated ramifications.
Now it goes by the name of the "Eckmann-Hilton theorem", which says
that "a monoid object in the category of monoids is a commutative
monoid".  You practically need a PhD in math to understand \emph{that}!
However, lest you think that Eckmann and Hilton were merely dressing
up the obvious in fancy jargon, it's important to note that what they
did was to figure out a \emph{framework} that turns the above "picture
proof" that a+b = b+a into an actual rigorous proof!  This is one of
the goals of higher-dimensional algebra.

The above proof that a+b = b+a uses 2-dimensional space, but if you
really think about it also uses a 3rd dimension, namely time: the time
that passes as you move "a" around "b".  If we draw this 3rd dimension
as space rather than time we can visualize the process of moving a
around b as follows:


\begin{verbatim}

             a       b
              \     /
               \   /
                \ /
                 /
                / \
               /   \
              /     \
             b       a

\end{verbatim}
    
This picture is an example of what mathematicians call a "braid".
This particular one is a boring little braid with only two strands and
one place where the two strands cross.  It illustrates another major
idea behind higher-dimensional algebra: equations are best thought of
as summarizing "processes" (or technically, "isomorphisms").  The
equation a+b = b+a is a summary of the process of switching a and b.
There is more information in the process than in the mere equation a+b
= b+a, because in fact there are two \emph{different} ways to switch a and
b: the above way and


\begin{verbatim}

             a       b
              \     /
               \   /
                \ /
                 \
                / \
               /   \
              /     \
             b       a

\end{verbatim}
    
If one has a bunch of objects one can switch them around in a lot
of ways, getting lots of different braids.  

In fact, the mathematics of braids, and related things like knots, is
crucially important for understanding quantum gravity in 3-dimensional
spacetime.  Spacetime is really 4-dimensional, of course, but quantum
gravity in 4-dimensional spacetime is awfully difficult, so in the
late 1980s people got serious about studying 3-dimensional quantum
gravity as a kind of warmup exercise.  It turned out that the math
required was closely related to some mysterious new mathematics
related to knots and "braidings".  At first this must sound bizarre: a
deep relationship between knots and 3-dimensional quantum gravity!
However, after you fight your way through the sophisticated
mathematical physics that's involved, it becomes clear why they are
related: both rely crucially on "3-dimensional algebra", the algebra
describing how you can move things around in 3-dimensional spacetime.


However, there is more to the story, because knot theory also seems
deeply related to \emph{4-dimensional} quantum gravity.  Here the
knots arise as "flux tubes of area" living in 3-dimensional
space at a given time.  Recent work on quantum gravity suggests that as
time passes these knots (or more generally, "spin networks")
move around and change topology as time passes.

To really understand this, we probably need to understand
"4-dimensional algebra".  Unfortunately, not enough is known about
4-dimensional algebra.  The problem is that we don't know much about
4-categories!  To do n-dimensional algebra in a really nice way, you
need to know about n-categories.  Roughly speaking, an n-category is
an algebraic structure that has a bunch of things called "objects", a
bunch of things called "morphisms" that go between objects, and
similarly 2-morphisms going between morphisms, 3-morphisms going
between 2-morphisms, and so on up to the number n.  You can think of
the objects as "things" of whatever sort you like, the morphisms as
processes going from one thing to another, the 2-morphisms as
meta-processes going from one process to another, and so on.
Depending on how you play the n-category game, there are either no
morphisms after level n, or only simple and bland ones playing the
role of "equations".  The idea is that in the world of n-categories,
one keeps track of things, processes, meta-processes, and so on to the
nth level, but after that one calls it quits and uses equations.  

So what is the definition of 4-categories?  Well, Eilenberg and
Mac Lane defined 1-categories, or simply "categories", in a paper
that was published in 1945:

1) S. Eilenberg and S. Mac Lane, General theory of natural
equivalences, Trans. Amer. Math. Soc. 58 (1945), 231-294.

Benabou defined 2-categories - though actually he called them 
"bicategories" - in a 1967 paper:

2) J. Benabou, Introduction to bicategories, Springer Lecture Notes in
Mathematics 47, New York, 1967, pp. 1-77.

Gordon, Power, and Street defined 3-categories - or actually 
"tricategories" - in a paper that came out in 1995:

3) R. Gordon, A. J. Power, and R. Street, Coherence for tricategories,
Memoirs Amer. Math. Soc. 117 (1995) Number 558.

This step took a long time in part because it took a long time for
people to understand deeply where \emph{braidings} fit into the picture.

But what about 4-categories and higher n?  Well, the history is
complicated and I won't get it right, but let me say a bit anyway.
First of all, there are some things called "strict n-categories" that
people have known how to define for arbitrarily high n for quite a
while.  In fact, people know how to go up to infinity and define
"strict \omega -categories"; see for example:

4) S. E. Crans, On combinatorial models for higher dimensional
homotopies, Ph.D. thesis, University of Utrecht, Utrecht, 1991.

Strict n-categories are quite interesting and important, but I'm
mainly mentioning them here to emphasize that they are \emph{not} what I'm
talking about.  People sometimes often call strict n-categories simply
"n-categories", and call the more general n-categories I'm talking
about above "weak n-categories".  However, I think the weak
n-categories will will eventually be called simply "n-categories",
because they are far more interesting and important than the strict
ones.  Anyway, that's what I'm doing here.

Secondly, when you define n-categories you have to make some choice
about the "shapes" of your j-morphisms.  In general they should be
some j-dimensional things, but they could be simplices, or cubes, or
other shapes.  In some ways the simplest shapes are "globes", a
j-dimensional globe being a j-dimensional ball with its boundary
divided into two hemispheres, the "inface" and "outface", which are
themselves (j-1)-dimensional globes.  This corresponds to a picture
where each "process" has one input and one output, which are themselves
processes having the same input and output.   The definitions of
category, bicategory, and tricategory work this way.  In fact, Ross
Street came up with a very nice definition of n-categories for all n
using simplices in 1987:

5) Ross Street, The algebra of oriented simplexes, Jour. Pure
Appl. Alg. 49 (1987), 283-335.

Since then, however, he and his students and collaborators seem to
have been working to translate this definition into the "globular"
formalism... while also making some other important adjustments too
technical to discuss here.  In particular, Dominic Verity and Todd
Trimble have done a lot of work on getting the definition of
n-category worked out, and a while ago I learned that Trimble came up
with a definition of "tetracategory" (or what I'm calling simply
"4-category") in August of 1995.  I don't think this has been
published, however.

James Dolan came to U. C. Riverside in the fall of 1993, and ever
since then, he and have been talking about n-categories and their role
in physics.  Most of the category theory I know, I learned in this
process.  It soon became clear that we needed a nice definition of
n-category for all n in order to turn our hopes and dreams into
theorems.  After a while we started working pretty hard on this.  His
job was to come up with all the bright ideas, and mine was to get him
to explain them, to try to poke holes in them, and to figure out
rigorous proofs of all the things that were so obvious to him that he
couldn't figure out how (or why) to prove them.  We sent a summarized
version of our definition to Ross Street at the end of 1995:

6) J. Baez and J. Dolan, n-Categories - sketch of a definition,
letter to Ross Street, Nov. 29, 1995, available at
<A HREF = "http://math.ucr.edu/home/baez/ncat.def.html">http://math.ucr.edu/home/baez/ncat.def.html</A>

and then for a year I worked on trying to write up a longer, clearer
version, while all the meantime Dolan kept coming up with new ways of
looking at everything.  I finished in February of this year:

7) J. Baez and J. Dolan, Higher-dimensional algebra III: n-Categories
and the algebra of opetopes, to appear in Adv. Math., preprint
available as <A HREF =
"http://xxx.lanl.gov/ps/q-alg/9702014">q-alg/9702014</A> and at <A HREF
=
"http://math.ucr.edu/home/baez/op.ps">http://math.ucr.edu/home/baez/op.ps,</A>
or in compressed form as <A HREF = "http://math.ucr.edu/home/baez/op.ps.Z">http://math.ucr.edu/home/baez/op.ps.Z</A>

The key feature of this definition is that it uses "j-dimensional
opetopes" as the shapes for j-morphisms.  These shapes are very handy
because the (j+1)-dimensional opetopes describe all the legal ways of
sticking together a bunch of j-dimensional opetopes to form another
j-dimensional opetope!  They are related to the theory of "operads",
which is part of the reason for their name.  (By the way, the first
two syllables are pronounced exactly as in "operation".)

In the meantime, Michael Makkai and John Power had begun work using
our definition.  Also, other definitions of "n-category" have appeared
on the scene!  Zouhair Tamsamani came up with one in terms of
"multi-simplicial sets":

8) Z. Tamsamani, Sur des notions de $\infty$-categorie et
$\infty$-groupoide non-strictes via des ensembles multi-simpliciaux,
Ph.D. thesis, Universite Paul Sabatier, Toulouse, France, 1995.

Michael Batanin also has a definition of \omega -categories, of
the "globular" sort:

9) M. A. Batanin, On the definition of weak \omega -category, Macquarie
Mathematics Report number 96/207.

Now the fun will begin!  These different definitions of (weak) n-category
should be equivalent, albeit in a rather subtle sense, so we should check
to see if they really are.  Also, we need to develop many more tools for
working with n-categories.  Then we can really start using them as a tool.

When I started writing this I thought I was going to explain the
definition that Dolan and I came up with.  Now I'm too tired!  It
takes a while to explain, so I think I'll stop here and save that for
some other week or weeks.  Perhaps I'll mix it in with my report on
the Workshop on Higher Category Theory and Physics, which is taking
place next weekend at Northwestern University.



This is the end of the 'Tale n-Categories'.  If you want more, try
<A HREF = "ncat.ps">`An Introduction to n-Categories', available in 
Postscript form by clicking here,</A> or else read the above
papers. 


\par\noindent\rule{\textwidth}{0.4pt}

\par\noindent\rule{\textwidth}{0.4pt}

% </A>
% </A>
% </A>
