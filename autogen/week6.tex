
% </A>
% </A>
% </A>
\week{February 20, 1993 }

1) Alexander Vilenkin,
Quantum cosmology, talk given at Texas/Pascos 1992 at Berkeley by
available as <A HREF = "http://xxx.lanl.gov/abs/gr-qc/9302016">gr-qc/9302016</A> 
This is, as Vilenkin notes, an elementary review of quantum cosmology.
It won't be news to anyone who has kept up on that subject (except
perhaps for a few speculations at the end), but for those who haven't
been following this stuff, like myself, it might be a good way to get
started.   

Let's get warmed up....

Quantizing gravity is mighty hard.  For one thing, there's the "problem
of time" - the lack of a distinguished time parameter in \emph{classical}
general relativity means that the usual recipe for quantizing a
dynamical system - "represent time evolution by the unitary operators
exp(-iHt) on the Hilbert space of states, where t is the time and H, the
Hamiltonian, is a self-adjoint operator" - breaks down!  As Wheeler so
picturesquely put it, in general relativity we have "many-fingered
time"; there are lots of ways of pushing a spacelike surface forwards in
time.  

But if we simplify the heck out of the problem, we might make a little
progress.  (This is a standard method in physics, and whether or not
it's really justified, it's often the only thing one can do!)
For one thing, note that in the big bang cosmology there is a
distinguished "rest frame" (or more precisely, field of timelike
vectors) given by the galaxies, if we discount their small random
motions.  In reality these are maybe not so small, and maybe not so
random - such things as the "Virgo flow" show this - but we're talking
strictly theory here, okay? - so don't bother us with facts!  So, if we
imagine that things go the way the simplest big bang models predict, the
galaxies just sit there like dots on a balloon that is being inflated,
defining a notion of "rest" at each point in spacetime.  This gives
a corresponding notion of time, since one can measure time using
clocks that are at rest relative to the galaxies.  Then, since we are
pretending the universe is completely homogeneous and isotropic - and
let's say it's a closed universe in the shape of a 3-sphere, to be 
specific - the metric is given by

dt^{2} - r(t)^{2}[(d\psi )^{2} + (sin \psi )^{2}{(d\theta )^{2} + (sin \theta )^{2} (d\phi )^{2}}]

What does all this mean?  Here r(t) is the radius of the universe as a
function of time, the following stuff is just the usual metric on the
unit 3-sphere with hyperspherical coordinates \psi , \theta , \phi 
generalizing the standard coordinates on the 2-sphere we all learn in
college:

(d\psi )^{2} + (sin \psi )^{2}{(d\theta )^{2} + (sin \theta )^{2} (d\phi )^{2}}

and the fact that the metric on spacetime is dt^{2} minus a bunch of stuff
reflects the fact that spacetime geometry is "Lorentzian," just as
in flat Minkowski space the metric is

dt^{2} - dx^{2} - dy^{2} - dz^{2}.

The name of the game in this simple sort of big bang cosmology is thus
finding the function r(t)!  To do this, of course, we need to see what
Einstein's equations reduce to in this special case, and since
Einstein's equations tell us how spacetime curves in response to the
stress-energy tensor, this will depend on what sort of matter we have
around.  We are assuming that it's homogeneous and isotropic, whatever
it is, so it turns out that all we need to know is its density \rho  and
pressure P (which are functions of time).  We get the equations


$$

r''/r = -(4\pi /3)(\rho  + 3P)              (r')^{2} = (8\pi /3) \rho  r^{2} - 1 
$$
    

Here primes denote differentiation with respect to t, and I'm using
units in which the gravitational constant and speed of light are equal to 1.

Let's simplify this even more.  Let's assume our matter is "dust," which
is the technical term for zero pressure.  We get two equations:


$$

r''/r = -(4\pi /3)\rho                (r')^{2} = (8\pi /3) \rho  r^{2} - 1.       (1)
$$
    

Now let's take the second one, differentiate with respect to t,

2r'' r' =  (8\pi /3)(\rho ' r^{2} + 2 \rho  r r')

plug in what the first equation said about r'',

-(8\pi /3) \rho  r r' = (8\pi /3)(\rho ' r^{2} + 2 \rho  r r')

clear out the crud, and lo:

3 \rho  r' =  - \rho ' r 

or, more enlighteningly,

d(\rho  r^{3})/dt = 0.

This is just "conservation of dust" - the dust density times the volume
of the universe is staying constant.  This, by the way, is a special
case of the fact that Einstein's equations \emph{automatically imply}
local conservation of energy (i.e., that the stress-energy tensor is
divergence-free).  

Okay, so let's say \rho  r^{3} = D, with D being the total amount of dust.  
Then we can eliminate \rho  from equations (1) and get:


$$

r'' = -4\pi D/3r^{2}          (r')^{2} - (8\pi /3) D/r = - 1                (2)
$$
    

What does this mean?   Well, the first one looks like it's saying
there's a force trying to make the universe collapse, and that the
strength of this force is proportional to 1/r^{2}.  Sound vaguely
familiar?  It's actually misleadingly simple - if we had put in
something besides dust it wouldn't work quite this way - but as long as we
don't take it too seriously, we can just think of this as gravity trying
to get the universe to collapse.  And the second one looks like it's
saying that the "kinetic" energy proportional to (r')^{2}, plus the
"potential" energy proportional to -1/r, is constant!  In other words,
we have a nice analogy between the big bang cosmology and a very
old-fashioned system, a classical particle in one dimension attracted to
the origin by a 1/r^{2} force!   

It's easy enough to solve this equation, and easier still to figure it out
qualitatively.  The key thing is that since the total "energy" in
the second equation of (2) is negative, there won't be enough "energy"
for r to go to infinity, that is, there'll be a big bang and then a big
crunch.  Here's r as a function of t, roughly:



\begin{verbatim}

           |                .     .
           |        .                     .
        r  |    .                            .            Figure 1
           |  .                                .
           | .                                  .
           |------------------------------------
                          t
\end{verbatim}
    

What goes up, must come down!  This curve, which I haven't drawn too
well, is just a cycloid, which is the curve traced out by a point on the
rim of rolling wheel.  So, succumbing to romanticism momentarily we
could call this picture ONE TURN OF THE GREAT WHEEL OF TIME....  But
there is \emph{no} reason to expect further turns, because the
differential equation simply becomes singular when r = 0.  We may either
say it doesn't make sense to speak of "before the big bang" or "after
the big crunch" - or we can look for improved laws that avoid these
singularities.  (I should repeat that we are dealing with unrealistic models
here, since for example there is no evidence that there is enough matter
around to "close the universe" and make this solution qualitatively
valid - it may well be that there's a big bang but no big crunch.  In
this case, there's only one singularity to worry about, not two.)  

People have certainly not been too ashamed to study the \emph{quantum}
theory of this system (and souped-up variants) in an effort to get a
little insight into quantum gravity.  We would expect that quantum
effects wouldn't matter much until the radius of the universe is very
small, but when it \emph{is} very small they would matter a lot, and maybe -
one might hope - they would save the day, preventing the nasty
singularities.  I'm not saying they DO - this is hotly debated - but
certainly some people hope they do.  Of course, serious quantum gravity
should take into account the fact that geometry of spacetime has all
sorts of wiggles in it -it isn't just a symmetrical sphere.  This may make
a vast difference in how things work out.  (For example, the big crunch
would be a lot more exciting if there were lots of black holes around by
then.)   The technical term for the space of all metrics on space is
"superspace" (sigh), and the toy models one gets by ignoring all but
finitely many degrees of freedom are called "minisuperspace" models. 

Let's look at a simple minisuperspace model.  The simplest thing
to try is to take the classical equations of motion (2) and try to
quantize them just like one would a particle in a potential.  This is a
delicate business, by the way, because one can't just take some
classical equations of motion and quantize them in any routine way.  
There are lots of methods of quantization, but all of them require a
certain amount of case-by-case finesse.

The idea of "canonical quantization" of a classical system with one
degree of freedom - like our big bang model above, where the one degree
of freedom is r  - is to turn the "position" (that's r) into a
multiplication operator and the "momentum" (often that's something like
r', but watch out!) into a differentation operator, say -i \hbar  d/dr, so that
we get the "canonical commutation relations" 


$$

			[-i \hbar  d/dr, r] = -i \hbar .
$$
    

We then take the formula for the energy, or Hamiltonian, in terms of
position and momentum, and plug in these operators, so that the
Hamiltonian becomes an operator.  (Here various "operator-ordering"
problems can arise, because the position and momentum commuted in the
original classical system but not anymore!)   To explain what I mean,
why don't I just do it!

So: I said that the formula 


$$

  (r')^{2} - (8\pi /3) D/r = - 1  						(3)
$$
    

looks a lot like a formula of the form "kinetic energy plus potential
energy is constant".   Of course, we could multiply the whole equation
by anything and get a valid equation, so it's not obvious that the 
``right'' Hamiltonian is


$$

  (r')^{2} - (8\pi /3) D/r
$$
    

or (adding 1 doesn't hurt)


$$

  (r')^{2} - (8\pi /3) D/r + 1
$$
    

In fact, note that multiplying the Hamiltonian by some function of r
just amounts to reparametrizing time, which is perfectly fine in general
relativity.  In fact, Vilenkin and other before him have decided it's
better to multiply the Hamiltonian above by r^{2}.   Why?  Well, it has to
do with figuring out what the right notion of "momentum" is
corresponding to the "position" r.   Let's do that.  We use the old
formula


\begin{verbatim}

     p = dL/dq'
\end{verbatim}
    

relating momentum to the Lagrangian, where for us the position, usually
called q, is really r.

The Lagrangian of general relativity is the "Ricci scalar" R - a measure of
curvature of the metric - and in the present problem it turns out to be


$$

   R = 6 (r''/r + (r')^{2}/r^{2})
$$
    

But we are reducing the full field theory problem down to a problem with
one degree of freedom, so our Lagrangian should be the above integrated
over the 3-sphere, which has volume 16 \pi  r^{3}/3, giving us


$$

   32\pi  (r'' r^{2} + (r')^{2} r)
$$
    

However, the a'' is a nuisance, and we only use the integral of the
Lagrangian with respect to time (that's the action, which classically is
extremized to get the equations of motion), so let's do an integration
by parts, or in other words add a total divergence, to get the Lagrangian


$$

   L = -32\pi  (r')^{2} r.
$$
    

Differentiating with respect to r' we get the momentum "conjugate to r",


$$

   p = -64\pi  r'r.
$$
    

Now I notice that Vilenkin uses as the momentum simply -r'r, somehow
sweeping the monstrous 64\pi  under the rug.  I have the feeling that
this amounts to pushing this factor into the definition of \hbar  in the
canonical commutation relations.  Since I was going to set \hbar  to 1 in
a minute anyway, this is okay (honest).  So let's keep life simple and
use 


\begin{verbatim}

   p = -r'r.
\end{verbatim}
    

Okay!  Now here's the point, we want to exploit the analogy with good
old quantum mechanics, which typically has Hamiltonians containing
something like p^{2}.  So let's take our preliminary Hamiltonian


$$

   (r')^{2} - (8\pi /3) D/r + 1
$$
    

and multiply it by r^{2}, getting


$$

   H = p^{2} - (8\pi  D/3)r + r^{2}.
$$
    

Hey, what's this?  A harmonic oscillator!  (Slightly shifted by the 
term proportional to r.)  So the universe is just a harmonic
oscillator... I guess that's why they stressed that so much in all my
classes! 

Actually, despite the fact that we are working with a very simple model
of quantum cosmology, it's not quite \emph{that} simple.  First of all,
recall our original classical equation, (3).  This constrained the
energy to have a certain value.  I.e., we are dealing not with a
Hamiltonian in the ordinary sense, but a "Hamiltonian constraint" -
typical of systems with time reparametrization invariance.  So our
quantized equation says that the "wavefunction of the universe," \psi (r),
must satisfy


$$

  H \psi  = 0.
$$
    

Also, unlike the ordinary harmonic oscillator we have the requirement
that r > 0.  In other word, we're working with a problem that's like a
harmonic oscillator and a "wall" that keeps r > 0.   Think of a particle
in a potential like this:




% </A>
% </A>
% </A>
