
% </A>
% </A>
% </A>
\week{May 27, 2007 }

Today I want to tell you about the electromagnetic snake at the center
of our Galaxy, and continue the Tale of Groupoidification.  

But first, the long-range weather forecast.  There's a 40% chance of
rain on Neptune in 8 billion years!  More precisely, that's the chance
these authors give for the formation of an ocean on Neptune when its
interior cools down enough:

1) Sloane J. Wiktorowica and Andrew P. Ingersoll, Liquid water oceans
in ice giants, available as <A HREF =
"http://xxx.lanl.gov/abs/astro-ph/0609723">astro-ph/0609723</A>.

Right now, even though Neptune is named after the Roman god of seas
and has a nice blue appearance:

<div align = "center">
<a href = "http://en.wikipedia.org/wiki/Neptune">
<img style = "border:none;" width = "300" src = "neptune.jpg">
% </a>
</div>

it's mighty dry - at least on top.  Spectroscopy detects no water at
all in its upper atmosphere!  But that's consistent with the presence
of water down below, since water vapor is a lot heavier than hydrogen
and helium, which make up most of Neptune's upper atmosphere, and the
pull of gravity there is mighty fierce.

In fact, scientists believe Neptune has a core of molten metal and 
rock surrounded by more rock, methane ice, ammonia ice, and water 
ice - all solid due to high pressures.  The overall density of Neptune 
makes sense if there's a lot of water down there.  But, Wiktorowica 
and Ingersoll argue that the planet can't have an ocean of liquid water. 

Surprisingly, this is because Neptune is too \emph{hot} - even though its 
upper atmosphere is a chilly 50 kelvin!  I think their argument goes
roughly like this, though I don't understand it well.  If you fell 
down through the atmosphere of Neptune, you'd find that it gets 
hotter and hotter as you go down, and moister and moister too - but 
always too hot, given the amount of moisture, for liquid water to be 
more stable than water vapor.  

However, they also indulge in some predictions about the far future!

First let's set the stage:

<ul>
<li>
In 1.1 billion years the Sun will become 10% brighter than now, and 
the Earth's atmosphere will dry out.

<li>
In 3 billion years the Andromeda Galaxy will collide with our galaxy.
Many solar systems will be destroyed.  

<li>
In 3.5 billion years the Sun will become 40% brighter than today.  If 
the Earth is still orbiting the sun, its oceans will evaporate.

<li>
In 5.4 billion years from now the Sun's core will run out of hydrogen.
It will enter its first red giant phase, becoming 1.6 times bigger and 
2.2 times brighter than today.

<li>
In 6.5 billion years from now the Sun will become a full-fledged red 
giant, 170 times bigger and 2400 times brighter than today.  The 
Republican Party will finally admit the existence of global warming, 
but point out that it's not human-caused.

<li>
In 6.7 billion years from now the Sun will start fusing helium and 
shrink back down to 10 times bigger and 40 times brighter than today.

<li>
In 6.8 billion years from now the Sun will run out of helium.  Being
too small to start fusing carbon and oxygen, it'll enter a second red 
giant phase, growing 180 times bigger and 3000 times brighter than today. 

</ul>

But then, about 6.9 billion years from now, the Sun will start pulsating, 
ejecting half of its mass in the form of solar wind!  It'll become what
they call a "planetary nebula".  Eventually only its inner core will be 
left.  In "<A HREF = "week223.html">week223</A>" I quoted Bruce Balick's eloquent description: 

\begin{quote}
     The remnant Sun will rise as a dot of intense light, no larger
     than Venus, more brilliant than 100 present Suns, and an
     intensely hot blue-white color hotter than any welder's torch.
     Light from the fiendish blue "pinprick" will braise the
     Earth and tear apart its surface molecules and atoms.  A new but
     very thin "atmosphere" of free electrons will form as
     the Earth's surface turns to dust.  
\end{quote}

This is where Wiktorovika and Ingersoll \emph{begin} their story.  So far, 
Neptune will have warmed up a lot - assuming for the sake of argument 
that it wasn't thrown out of the Solar System when the Milky Way hit
Andromeda.  But when the Sun loses mass, Neptune will either collide 
with Uranus, be ejected from the Solar System, or assume a stable orbit 
about twice its current size.  

In the latter two scenarios, Neptune will chill out, at least when the
remnant Sun cools down and becomes a white dwarf.    When the surface 
temperature of Neptune reaches 30 kelvin, they estimate it has a 41.5% 
plus or minus 4.2% chance of forming oceans. 

So, they say: "Billions of years from now, after the Sun has gone, 
Neptune may therefore become the only object in the Solar system
with liquid water oceans".

This sounds nice - but don't buy your beachfront property there yet.  

First, they don't study how the atmospheric composition of Neptune 
may change when the Sun gets 3000 times brighter than now!  Maybe this 
will \emph{help} Neptune form oceans.  After all, light gases like molecular 
hydrogen and helium, which dominate Neptune's upper atmosphere now, are 
more likely than water vapor to be driven off into outer space when it 
gets hot.  But, they don't even check to see if Neptune will have 
\emph{any} atmosphere left after this era.

Second, the estimate of a "41.5% plus or minus 4.2% chance" seems
strangely precise, given the uncertainties involved.  The error 
bars should probably have big error bars themselves!

Third, it's worth admitting that the atmosphere of Neptune is a bit
mysterious.  For example, nobody knows why it's bright blue.  It's probably 
because of methane - but Uranus also has methane in its atmosphere,
and it's not as blue:

<div align = "center">
<a href = "http://en.wikipedia.org/wiki/Uranus">
<img style = "border:none;" src = "uranus.jpg">
% </a>
</div>

Also, despite being the coldest planet in our 
Solar System, Neptune has the fastest winds: up to 2100 kilometers per 
hour, almost supersonic!  Nobody knows what powers these winds.  

When the Voyager 2 spacecraft flew by Neptune in 1992, it saw a storm 
system the size of Eurasia, which was dubbed "The Great Dark Spot":

<div align = "center">
<a href = "http://en.wikipedia.org/wiki/Great_Dark_Spot">
<img style = "border:none;" src = "neptune_great_dark_spot.jpg">
% </a>
</div>
  
It seemed to resemble the Great Red Spot on Jupiter, which has been 
around for at least 177 years.  But when the Hubble Space Telescope 
took another good look at Neptune in 1994, the Great Dark Spot was 
gone!  Meanwhile, other storms had formed.  

So, the weather on Neptune is dynamic and poorly understood.  Doing 
forecasts for the next 8 billion years seems pretty risky... though fun. 

Of course, liquid water oceans are nice if you're looking for life.
And while there probably isn't life on Neptune, there could be life on 
similar planets in other solar systems.  So far people have found 233
of these "exoplanets", most of them heavier than Jupiter but close 
to their suns - because such planets are the easiest to detect by how 
they pull on their sun.  

Here's a chart showing the masses and orbital radii of known exoplanets 
as of 2004:

<div align = "center">
<a href = "http://en.wikipedia.org/wiki/Extrasolar_planet">
<img style = "border:none;" src = "exoplanets.png">
% </a>
</div>

2) P. R. Lawson, S. C. Unwin, and C. A. Beichman, Precursor Science for
the Terrestrial Planet Finder, JPL Pub 04-014, Oct. 2004, page 21, fig. 5.
Chart at <a href = "http://en.wikipedia.org/wiki/Image:Extrasolar_Planets_2004-08-31.png">http://en.wikipedia.org/wiki/Image:Extrasolar_Planets_2004-08-31.png</a> <br/>
Report at <a href = "http://planetquest.jpl.nasa.gov/documents/RdMp272.pdf">http://planetquest.jpl.nasa.gov/documents/RdMp272.pdf</a>

For comparison, the letters V, E, M, J, S, U, N, stand for planets in 
our Solar System, not counting Mercury or the subsequently dethroned 
Pluto.  

As you'll see, there are lots of "hot Jupiters" - planets as big as 
Jupiter, or even up to 200 times heavier, but closer to their sun than 
we are to ours.  Recently a postdoc at UCLA found evidence for water 
in the atmosphere of one of these planets:

3) Travis Barman, Identification of absorption features in an extrasolar 
planet atmosphere, available as <a href = "http://www.arxiv.org/abs/0704.1114">arxiv:0704.1114</a>.

This planet is only 7 million kilometers away from its yellow-white Sun, 
much closer than Mercury is to ours.  Its year is only 3.5 of our days!  
It's bigger in size than Jupiter, but only 0.7 times as heavy.  Its 
surface temperature must be about 1000 kelvin.  That's one really hot
Jupiter.

People have also been finding "hot Neptunes".  In fact, I
read about a nice one in the newspaper while writing this!  It's
called Gliese 436 b.  It's the size of Neptune and it's orbiting the
red dwarf star Gliese 436, which is 33 light-years from Earth.  This
star is only 1% as bright as our Sun - but the planet is so close that
its year lasts less than three of our days!  So, its surface
temperature is high: higher than the melting point of lead.

However, because this planet is so big, the pressure down below can
still make water into a solid.  In fact, its density suggests that
it's mainly made of ice!

I think this is the paper that triggered the newspaper reports:

4) M. Gillon et al, Detection of transits of the nearby hot Neptune 
GJ 436 b, available as <a href = "http://www.arxiv.org/abs/0705.2219">arxiv:0705.2219</a>.

It seems hot Neptunes like this could have started as hot Jupiters 
and then lost a lot of their atmosphere:  

5) I. Baraffe, G. Chabrier, T. S. Barman, F. Selsis, F. Allard, 
and P. H. Hauschildt, Hot-Jupiters and hot-Neptunes, a common origin?,
available at <A HREF = "http://www.arxiv.org/abs/astro-ph/0505054">astro-ph/0505054</A>.

There could also be cold Neptunes, perhaps with liquid water oceans.
We haven't seen these yet, but they'd be hard to see.  So, while 
Wiktorowica and Ingersoll's paper doesn't convince me about the future 
of \emph{our} Neptune, it raises some interesting possibilities.

Next: the snake at the center of our galaxy! 

Gregory Benford is best known for his science fiction, which spans
the galaxy, but he's also an astrophysicist at U. C. Irvine.  Recently
he spent a week at my school, U. C. Riverside, which has one of the 
world's best SF libraries: the <a href = "http://eaton-collection.ucr.edu/">Eaton Collection</a>.  Since my wife is
involved in the SF program at the comparative literature department here,
he had dinner at our place one night.  The conversation drifted all over 
the place, with a heavy focus on fruit flies that have been bred to live
twice as long as usual.  But when I asked him about his research, he 
said he'd written some papers about enormous glowing filaments near 
the center of the Milky Way. 

I hadn't known about these!  He said the biggest one could be a million 
years old, perhaps formed by some energetic event, maybe a star falling 
into the central black hole.  Here's an expository paper he wrote about
it:

6) Gregory Benford, The electromagnetic snake at the galactic center,
<a href = "http://www.ps.uci.edu/physics/news3/benford.html">http://www.ps.uci.edu/physics/news3/benford.html</a>

I'll just quote a little:

\begin{quote}
   Five years ago radio astronomy revealed the oddest and longest 
   filament yet discovered at our galactic center: a uniquely kinked 
   structure about 150 light years long and two to three light years 
   wide - the Snake.  Its large kinks are its brightest parts.  There 
   is energetic activity at one end and a supernova bubble at the other, 
   which the Snake appears to penetrate unharmed.

   How does nature form stable, long-lived magnetic structures which 
   display considerable polarization (about 60% at 10.55 GHz in the 
   Snake)?  In 1988 I had modeled others of the dozens of filaments 
   seen uniquely at the galactic center in terms of an electrodynamic 
   view, in which currents set up coherent magnetic pinches.  Such 
   self-organizing filaments can exist in laboratory plasmas for long 
   times; the galactic ones could be at least a million years old, as 
   estimated by the time that shear forces would disrupt them.

   The electrodynamic view uses pinch forces of currents to form filaments, 
   driven by the E = v \times  B of conducting molecular clouds moving across a 
   strong milliGauss ambient, ordered field.  A return current must then 
   flow at larger radii, making a closed loop which has a springy 
   flexibility, able to withstand the turbulent velocity fields known 
   near the galactic center.  The picture then anticipates that aberrant 
   molecular clouds, moving contrary to the general galactic rotation, 
   should accompany each filament.  This prediction has held up as more 
   filaments were found. 
\end{quote}

I'd like to learn more about these!  

Back in "<A HREF = "week248.html">week248</A>", I mentioned some of the complex things that 
electromagnetic fields and plasma do in the Sun.  The center of the
Galaxy is another place where electromagnetohydrodynamics runs 
rampant.  There's a black hole there, of course, but also these
filaments, and a fairly strong magnetic field that contains about 
4 \times  10^{47} joules of energy within about 300 light years of the 
galactic center.  By comparison, a supernova emits a mere 10^{44} 
joules.  So, there's a lot of energy around....

The big picture here, created by Farhad Yusef-Zadeh and collaborators, 
shows the galactic center quite nicely, as viewed in radio frequencies:

<div align = "center">
<a href = "http://www.nrao.edu/pr/2004/filaments/">
<img style = "border:none;" src = "galactic_center.jpg">
% </a>
</div>

7) National Radio Astronomy Observatory, Origin of enigmatic galactic-center filaments revealed, <a href = "http://www.nrao.edu/pr/2004/filaments/">http://www.nrao.edu/pr/2004/filaments/</a>

You can see some supernova remnants (SNRs), the region Sgr A which
contains the black hole at the galactic center, various nonthermal 
radio filaments (NRFs), and the Snake.  Some of these filaments 
come from regions where stars are forming... that could be important.

But, this article discusses another piece of the puzzle - the possible 
role of turbulence in winding up the galactic magnetic field: 

8) Stanislav Boldyrev and Farhad Yusef-Zadeh, Turbulent origin of
the galactic-center magnetic field: nonthermal radio filaments,
available as <A HREF = "http://xxx.lanl.gov/abs/astro-ph/0512373">astro-ph/0512373</A>.

It's a complicated stew.  I don't hope to understand it, just admire it.

And finally: the Tale of Groupoidification!  In "<A HREF = "week250.html">week250</A>" we reached
the point of seeing how spans of groupoids over a fixed group G 
subsume the theory of G-sets and invariant relations between 
these - which are traditionally studied using "double cosets".  

There is a lot more we could say about this.  But, our most urgent 
goal is to see how spans of groupoids act like twice categorified 
matrices - matrices whose entries are not just \emph{numbers}, and not 
just \emph{sets}, but \emph{groupoids}!  This will expose the secret 
combinatorial underpinnings of a lot of fancy linear algebra.  Once 
we've categorified linear algebra this way, we'll be in a great 
position to tackle fashionable topics like categorified quantum groups,
invariants of higher-dimensional knots, and the like.

But, we should restrain ourselves from charging ahead too fast!  
Everything will hang together better if we lay the groundwork properly.
For this, it pays to re-examine the history of mathematics a bit.
If we're trying to understand linear algebra using groupoids, it
pays to ask: how did people connect linear algebra and group theory 
in the first place?  

This book is very helpful:

9) Charles W. Curtis, Pioneers of Representation Theory: Frobenius,
Burnside, Schur and Brauer, History of Mathematics vol. 15, AMS,
Providence, Rhode Island, 1999.

Back in 1897, a mathematician named William Burnside wrote the first
book in English on finite groups.  It was called Theory of Groups of 
Finite Order.  

<div align = "center">
<a href = "http://www-groups.dcs.st-and.ac.uk/~history/Biographies/Burnside.html">
<img style = "border:none;" src = "burnside.jpg">
% </a>
</div>

In the preface, Burnside explained why he studied finite 
groups by letting them act as permutations of sets, but not as 
linear transformations of vector spaces:

\begin{quote}
   Cayley's dictum that "a group is defined by means of the laws of
   combination of its symbols" would imply that, in dealing with the
   theory of groups, no more concrete mode of representation should be
   used than is absolutely necessary.  It may then be asked why, in 
   a book that professes to leave all applications to one side, a 
   considerable space is devoted to substitution groups [permutation
   groups], but other particular modes of representation, such as 
   groups of linear transformations, are not even referred to.  My
   answer to this question is that while, in the present state of our
   knowledge, many results in the pure theory are arrived at most 
   readily by dealing with properties of substitution groups, it 
   would be difficult to find a result that could most directly be 
   obtained by the consideration of groups of linear transformations.
\end{quote}

In short, he didn't see the point of representing groups on vector 
spaces - at least as a tool in the "pure" theory of finite
groups, as opposed to their applications.

However, within months after this book was published, he discovered
the work of Georg Frobenius, who used linear algebra very effectively to
study groups! 

<div align = "center">
<a href = "http://www-groups.dcs.st-and.ac.uk/~history/Biographies/Frobenius.html">
<img style = "border:none;" src = "frobenius.jpg">
% </a>
</div>

So, Burnside started using linear algebra in his own work on 
finite groups, and by the time he wrote the second edition of his 
book in 1911, he'd changed his tune completely:

\begin{quote}
   Very considerable advances in the theory of groups of finite 
   order have been made since the appearance of the first edition 
   of this book.  In particular the theory of groups of linear 
   substitutions has been the subject of numerous and important
   investigations by several writers; and the reason given in the
   original preface for omitting any account of it no longer holds
   good.  In fact it is now more true to say that for further 
   advances in the abstract theory one must look largely to the 
   representation of a group as a group of linear transformations.
\end{quote}

It's interesting to see exactly how representing finite groups on vector
spaces lets us understand them better.  By now almost everyone agrees 
this is true.  But how much of the detailed 
machinery of linear algebra is really needed?  How much we could 
do purely combinatorially, using just spans of groupoids?

I don't really know the full answer to this question.  But, it quickly
leads us into the fascinating theory of "Hecke operators", which will
play a big role in the Tale of Groupoidification.  So, let's pursue it 
a bit.

Suppose we have two guys, William and Georg, who are studying a finite
group G.  

William says, "I'm studying how G acts on sets."

Georg replies, "I'm studying how it acts on complex vector spaces, 
as linear transformations.  Mere sets are feeble entities.  I can do 
anything you can do - but I have the tools of linear algebra to help me!"

William says, "But, you're paying a moral price.  You're getting the
complex numbers - a complicated infinite thing - involved in what
should be a completely finitary and combinatorial subject: the study
of a finite group.  Is this really necessary?"

Georg replies, "I don't know, but it's sure nice.  For example,
suppose I have G acting on vector space V.  Then I can always 
break down V into a direct sum of subspaces preserved by G, which 
can't themselves be broken down any further.  In technical terms:
every representation of G is a direct sum of irreducible representations.
And, this decomposition is unique!  It's very nice: it's a lot like
how every natural integer has a unique prime factorization."

William says, "Yes, it's always fun when we can break things down
into 'atoms' that can't be further decomposed.  It's very satisfying
to our reductionist instincts.  But, I can do even better than you!"

Georg raises an eyebrow.  "Oh?"

"Yeah," William says.  "Suppose I have our group G acting on a set S.
Then I can always break down S into a disjoint union of subsets 
preserved by G, which can't themselves be broken down any further.  
In technical terms: every action of G is a disjoint union of transitive
actions.  And, this decomposition is unique!"

Embarrassed, Georg replies, "Oh, right - we heard that back in "<A HREF = "week249.html">week249</A>".
I wasn't paying enough attention.  But how is what you're doing better 
than what I'm doing?  It sounds just the same."

William hesitates.  "Well, first of all, a rather minor point, which I can't 
resist mentioning ... when you said your decomposition of representations 
into irreducibles was unique, you were exaggerating a little.  It's just 
unique up to isomorphism, and not a canonical isomorphism either.  

For example, if you have an irreducible representation of G on V, 
there are lots of \emph{different ways} to slice the direct sum V + V 
into two copies of the representation V.   It's a sort of floppy 
business.  On the other hand, when I have a transitive action of G on S, 
there's exactly \emph{one way} to chop the disjoint union S + S into two 
copies of the G-set S.  I just look at the two orbits."

Georg says, "Hmm.  This is actually a rather delicate point.  
There's not really a \emph{canonical} isomorphism in your case either, 
since S may be isomorphic to itself in lots of ways, even as a G-set.
There's something in what you say, but it's a bit hard to make 
precise, and it's certainly not enough to worry me."

"Okay, but here's my more serious point.  Given that
I can break apart any set on which G acts into 'atoms', my job is
to classify those atoms: the transitive G-sets.  And there's a very
nice classification!  Any subgroup H of G gives a transitive G-set, 
namely G/H, and all transitive G-sets look like this.  More precisely:
isomorphism classes of transitive G-sets correspond to conjugacy 
classes of subgroups of G.  

Even better, this has a nice meaning in terms of Klein geometry.
Any type of figure in Klein geometry gives a transitive G-set, 
with H being the stabilizer of a chosen figure of that type.

You, on the other hand, need to classify irreducible representations 
of G.  And this is not so conceptual.  What do these irreducible
representations \emph{mean} in terms of the group G?"

Georg replies, "Well, there's one irreducible representation for
each conjugacy class in G..."

At this, William pounces.  "You mean the \emph{number} of isomorphism
classes of irreducible representations of G equals the \emph{number}
of conjugacy classes in G!  But as you know full well, there's no 
god-given correspondence.  You can't just take a conjugacy class in 
G and cook up an irreducible representation, or irrep.  So, you've 
just made my point.  You've shown how mysterious these irreps actually 
are!"

Georg replies, "Actually in some cases there \emph{is} a nice way to 
read off irreps from conjugacy classes.  For example, you can
do it for the symmetric groups S_{n}.
But, I admit you can't in general... or at least, I don't know how."

William laughs, "So, I win!"

"Not at all!" replies Georg.  "First, there are lots of theorems
about groups that I can prove using representations, which you can't
prove using actions on sets.  For example, nobody knows how to prove
that <a href = "http://en.wikipedia.org/wiki/Odd_order_theorem">every group
with an odd number of elements is solvable</a> without using the tools of linear algebra."

William nods.  "I admit that linear algebra is very practical.
But just give us time!  I proved back in 1904 that every group of size
p^{a}q^{b} is solvable if p and q are prime.  To do
it, I broke down and used linear algebra.  But then, in 1972, Helmut
Bender found a proof that avoids linear algebra."

Georg said, "Okay, struggle on then.  So far, without using linear
algebra, nobody can even prove my famous theorem on '<a href = "http://en.wikipedia.org/wiki/Frobenius_group">Frobenius groups</a>'.
The details don't matter here: the point is, this is a result on 
group actions, which seems to need linear algebra for its proof.  

But if practicality won't sway you, maybe this conceptual argument 
will.  My atoms are more fine-grained than yours!"

"What do you mean?" asks William.

"You can decompose any action of G into 'atoms', namely transitive
G-sets.  Similarly, I can decompose any representation of G into 
one of my 'atoms', namely irreps.  But, there's an obvious way to 
turn G-sets into representations of G, and if we do this to one of
your atoms, we don't get one of my atoms!  We can usually break it down
further!  So, my atoms are smaller than yours."

"How does this work, exactly?" 

"It's easy," says Georg, getting a bit cocky.  "Say you
have a group G acting on a set S.  Then I can form the vector space
C[S] whose elements are formal linear combinations of elements of S.
In other words, it's the vector space having S as basis.  If we're
feeling sloppy we can think of guys in C[S] as functions on S that
vanish except at finitely many points.  It's better to think of them
as measures on S.  But anyway: since G acts on S, it acts linearly on
C[S]!

So, any G-set gives a representation of G.  But, even when G acts
transitively on S, its representation on C[S] is hardly ever irreducible."

William glowers.  "Oh yeah?"

"Yeah.  Suppose for example that S is finite.  Then the constant functions 
on S form a 1-dimensional subspace of C[S] that's invariant under the 
action of G.  So, at the very least, we can break C[S] into two pieces."

"Well," replies William defensively, "That's pretty obvious.  But it's
also not such a big deal.  So you can break up any my transitive G-sets 
into two of your irreps, one being the 'trivial' irrep.  So what???"

"It wouldn't be a big deal if that's \emph{all} that ever
happened," says Georg.  "In fact, we can break C[S] into
precisely two irreps whenever the action of G on S is 'doubly transitive'
- meaning we can send any \emph{pair} of distinct elements of S to any
other using some element of G.  But, there lots of transitive actions
aren't doubly transitive!  And usually, one your atoms breaks down
into a \emph{bunch} of my atoms.  In fact I'd like to show you how this
works, in detail, for the symmetric groups."

"Maybe next week," says William.  "But, I see your
point.  Your atoms are more atomic than my atoms."

Georg seems to have won the argument.  But, William wouldn't have 
conceded the point quite so fast, if he'd thought about invariant 
relations!

The point is this.  Suppose we have two G-sets, say X and Y.  Any 
G-set map from X to Y gives an intertwining operator from C[X] to C[Y].  
But, even after taking linear combinations, there are typically 
plenty of intertwining operators that don't arise this way.  It's
these extra intertwining operators that let us chop representations 
into smaller atoms.

But where do these extra intertwining operators come from?  They come
from \emph{invariant relations} between X and Y!  

And, what are these extra intertwining operators called?  In some
famous special cases, like in study of modular forms, they're called
"<a href = "http://en.wikipedia.org/wiki/Hecke_operator">Hecke 
operators</a>".  In some other famous special cases,
like in the study of symmetric groups, they form algebras called
"<a href = "http://en.wikipedia.org/wiki/Hecke_algebra">Hecke 
algebras</a>".  

A lot of people don't even know that 
Hecke operators and Hecke algebras are two faces of the same idea:
getting intertwining operators from invariant relations.  
But, we'll see this is true, once we look at some examples.

I think I'll save those for future episodes.  
But if you've followed the Tale so
far, you can probably stand a few extra hints of where we're going.
Recall from "<A HREF = "week250.html">week250</A>" that
invariant relations between G-sets are just spans of groupoids
equipped with some extra stuff.  So, invariant relations between
G-sets are just a warmup for the more general, and simpler, theory of
spans of groupoids.  I said back in "<A HREF =
"week248.html">week248</A>" that spans of groupoids give linear
operators.  What I'm trying to say now is that these linear operators
are a massive generalization - but also a simplification - of what
people call "Hecke operators".

Finally, for students trying to learn a little basic category theory, I'd
like to cast the argument between William and Georg into the language of 
categories, just to help you practice your vocabulary.

A G-set is the same as a functor

A: G \to  Set

where we think of G as a 1-object category.  There's a category of 
G-sets, namely

hom(G,Set)

This has functors A: G \to  Set as objects, and natural transformations
between these as morphisms.  Usually the objects are called "G-sets",
and the morphisms are called "maps of G-sets".  

We can also play this whole game with the category of vector spaces 
replacing the category of sets.  A representation of G is the same as
a functor

A: G \to  Vect

As before, there's a category of such things, namely

hom(G,Vect)

This has functors A: G \to  Vect as objects, and natural transformations
between these as morphisms.  Now the objects are called "representations 
of G" and the morphisms are called "intertwining operators".

We could let any groupoid take the place of the group G.  We could also
let any other category take the place of Set or Vect.  

Part of what William and Georg were debating was: how similar are 
hom(G,Set) and hom(G,Vect)?  How are they related? 

First of all, there's a functor 

F: Set \to  Vect

sending each set S to the vector space C[S] with that set as basis.  So,
given an action of G on a set:

A: G \to  Set

we can compose it with F and get a representation of G:

FA: G \to  Vect

This kind of representation is called a "permutation representation".
And, this trick gives a functor from G-sets to representations of G:


$$

hom(G,Set) \to  hom(G,Vect) 
        A |-> FA
$$
    
If this functor were an equivalence of categories, it would have to
be essentially surjective, full and faithful.
But, not every representation of G is isomorphic to a permutation
representation!  In other words, the functor 

hom(G,Set) \to  hom(G,Vect) 

is not "essentially surjective".  

Moreover, not every intertwining operator between permutation 
representations comes from a map between their underlying G-sets!
In other words, the functor

hom(G,Set) \to  hom(G,Vect) 

is not "full".  

But, given two different maps from one G-set to another, they give 
different intertwining operators.  So, at least our functor is "faithful".  

Maps of G-sets are a special case of invariant relations.  So, to get a 
category that more closely resembles hom(G,Vect), while remaining purely 
combinatorial, we can replace hom(G,Set) by the category with G-sets as 
objects and invariant binary relations as morphisms.  This is the basic 
idea of "Hecke operators".  

Or, even better, we can try a weak 2-category, with 

<ul>
<li>
 groupoids over G as objects
<li>
 spans of groupoids over G as morphisms
<li>
 maps between spans of groupoids over G as 2-morphisms
</ul>

This is where groupoidification comes into its own.

\par\noindent\rule{\textwidth}{0.4pt}
\textbf{Addendum:} 
For more discussion, go to the <a href = "http://golem.ph.utexas.edu/category/2007/05/this_weeks_finds_in_mathematic_13.html">\emph{n}-Category
Caf&eacute;</a>.

\par\noindent\rule{\textwidth}{0.4pt}
<em>The present treatise is intended to introduce to the reader the main
outlines of the theory of groups of finite order apart from any applications.
The subject is one which has hitherto attracted but little attention in 
this country; it will afford me much satisfaction if, by means of this
book, I shall arouse interest among English mathematicians in a branch
of pure mathematics which becomes the more fascinating the more it is 
studied</em> - William Burnside

\par\noindent\rule{\textwidth}{0.4pt}

% </A>
% </A>
% </A>
