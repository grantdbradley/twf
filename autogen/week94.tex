
% </A>
% </A>
% </A>
\week{November 11, 1996}


Today I want to talk a bit about asymptotic freedom.
First of all, remember that in quantum field theory, studying
very small things is the same as studying things at very high
energies.  The reason is that in quantum mechanics you need to
collide two particles at a large relative momentum p to make sure 
the distance x between them gets small, thanks to the uncertainty 
principle.  But in special relativity the energy E and momentum p 
of a particle of mass m are related by
                    E^{2} = p^{2} + m^{2},
in God's units, where the speed of light is 1.  So small x also 
corresponds to large E.   
"Asymptotic freedom" refers to fact that some forces
become very weak at high energies, or equivalently, at very short
distances.   The most interesting example of this is the so-called
"strong force", which holds the quarks together in a hadron, like a
proton or neutron.  True to its name, it is very strong at distances
comparable to the radius of proton, or at energies comparable to the
mass of the proton (where if we don't use God's units, we have to use 
E = mc^2 to convert units of mass to units of energy).  But if we smash 
protons at each other at much higher energies, the constituent quarks act 
almost as free particles, indicating that the strong force gets weak when 
the quarks get really close to each other.   
Now in "<A HREF = "week76.html">week76</A>"
and "<A HREF = "week84.html">week84</A>"
I talked about another phenomenon, called
"confinement".  This simply means that at lower energies, or larger
distance scales, the strong force becomes so strong that it is
\emph{impossible} to pull a quark out of a hadron.  Asymptotic freedom and
confinement are two aspects of the same thing: the dependence of 
the strength of the strong force on the energy scale.  Asymptotic 
freedom is better understood, though, because the weaker a force is, 
the better we can apply the methods of perturbation theory - a widely 
used approach where we try to calculate everything as a Taylor series in 
the "coupling constant" measuring the strength of the force in question.
This is often successful when the coupling constant is small, but not
when it's big.
The interesting thing is that in quantum field theory the coupling 
constants "run".  This is particle physics slang for the fact that 
they depend on the energy scale at which we measure them.  "Asymptotic 
freedom" happens when the coupling constant runs down to zero as we 
move up to higher and higher energy scales.  If you want to impress 
someone about your knowledge of this, just mutter something about 
the "beta function" being negative - this is a fancy way of saying 
the coupling constant decreases as you go to higher energies.  You'll 
sound like a real expert.
Now, Frank Wilczek is one of the original discoverers of asymptotic
freedom.  He \emph{is} a real expert.  He recently won a prize for this work,
and he gave a nice talk which he made into a paper:
1) Frank Wilczek, Asymptotic freedom, preprint available as
<A HREF = "http://xxx.lanl.gov/abs/hep-th/9609099">hep-th/9609099</A>.
Among other things, he gives a nice summary of the work of Nielsen
and Hughes, which gave the first really easy to understand explanation
of asymptotic freedom.  For the original work, try:
2) N. K. Nielsen, Am. J. Phys. 49, 1171 (1981).
3) R. J. Hughes, Nucl. Phys. B186, 376 (1981). 
Why would a force get weak at short distance scales?  Actually it's
easier to imagine why it would get \emph{strong} - and sometimes that is
what happens.  Of course there are lots of forces that decrease with
distance like 1/r^2, but I'm talking about something more drastic: I'm
talking about "screening".
For example, say you have an electron in some water.  It'll make an
electric field, but this will push all the other negatively charged
particles little bit \emph{away} from your electron and pull all the
positively charged ones a little bit \emph{towards} your electron:

\begin{verbatim}

                                   -
                                     +

                         your electron: -        +-
 
                                            +
                                              -
                      
\end{verbatim}
    
In other words, it will "polarize" all the neighboring water molecules.
But this will create a counteracting sort of electric field, since it
means that if you draw any sphere around your electron, there will be a
bit more \emph{positively} charged other stuff in that sphere than negatively
charged other stuff.  The bigger the sphere is, the more this effect
occurs - though there is a limit to how much it occurs.  We say that
the further you go from your electron, the more its electric charge is
"screened", or hidden, behind the effect of the polarization.
This effect is very common in materials that don't conduct electricity,
like water or plastics or glass.  They're called "dielectrics", and the
dielectric constant, \epsilon , measures the strength of this screening
effect.  Unlike in math, this \epsilon  is typically bigger than 1.  If
you apply an electric field to a dielectric material, the electric field
inside the material is only 1/\epsilon  as big as you'd expect if this
polarization wasn't happening.  
What's cool is that according to quantum field theory, screening occurs
even in the vacuum, thanks to "vacuum polarization".  One can visualize
it rather vaguely as due to a constant buzz of virtual particle-antiparticle
pairs getting created and then annihilating - called "vacuum bubbles"
in the charming language of Feynman diagrams, because you can draw them
like this:

\begin{verbatim}


                  /\
               e+/  \e-
                /    \
                \    /
                 \  /
                  \/

\end{verbatim}
    
Here I've drawn a positron-electron pair getting created and then
annihilating as time passes - unfortunately, this bubble is square,
thanks to the wonders of ASCII art.  
There is a lot I should say about virtual particles, and how despite
the fact that they aren't "real" they can produce very real effects
like vacuum polarization.  A strong enough electric field will even
"spark the vacuum" and make the virtual particles \emph{become} 
real!  But 
discussing this would be too big of a digression.  Suffice it to say 
that you have to learn quantum field theory to see how something that 
starts out as a kind of mathematical book-keeping device - a line in a 
Feynman diagram - winds up acting a bit like a real honest particle.  
It's a case of a metaphor gone berserk, but in an exceedingly useful way.
Anyway, so much for screening.   Asymptotic freedom requires something 
opposite, called "anti-screening"!   That's why it's harder to understand.
Nielsen and Hughes realized that anti-screening is easier to understand
using magnetism than electricity.   In analogy to dielectrics, there
are some materials that screen magnetic fields, and these are called
"diamagnetic" - for example, one of the strongest diamagnets is bismuth.
But in addition, there are materials that "anti-screen" magnetic fields -
the magnetic field inside them is stronger than the externally applied
magnetic field - and these are called "paramagnetic".  For example,
aluminum is paramagnetic.  People keep track of paramagnetism using
a constant called the magnetic permeability, \mu .  Just to confuse you,
this works the opposite way from the dielectric constant.  If you
apply a magnetic field to some material, the magnetic field inside it is 
\mu  times as big as you'd expect if there were no magnetic effects going
on.   
The nice thing is that there are lots of examples of paramagnetism
and we can sort of understand it if we think about it.   It turns
out that paramagnetism in ordinary matter is due to the spin of the 
electrons in it.  The electrons are like little magnets - they
have a little "magnetic moment" pointing along the axis of their spin.
Actually, purely by convention it points in the direction opposite
their spin, since for some stupid reason Benjamin Franklin decided
to decree that electrons were \emph{negative}.  But don't worry about this -
it doesn't really matter.  The point is that when you put electrons in a
magnetic field, their spins like to line up in such a way that
their magnetic field points the same way as the externally applied
magnetic field, just like a compass needle does in the Earth's magnetic
field.  So they \emph{add} to the magnetic field.  Ergo, paramagnetism.
Now, spin is a form of angular momentum intrinsic to the electron,
but there is another kind of angular momentum, namely orbital angular
momentum, caused by how the electron (or whatever particle) is moving 
around in space.  It turns out that orbital angular momentum also
has magnetic effects, but only causes diamagnetism.  The idea 
that when you apply a magnetic field to some material, it can also make
the electrons in it tend to move in orbits perpendicular to the
magnetic field, and the resulting current creates a magnetic field.
But this magnetic field must \emph{oppose} the external magnetic field.
Ergo, diamagnetism.  
Why does orbital angular momentum work one way, while spin works
the other way?  I'll say a bit more about that later. Now let
me get back to asymptotic freedom.
I've talked about screening and antiscreening for both electric
and magnetic fields now.  But say the "substance" we're studying
is the \emph{vacuum}.  Unlike most substances, the vacuum doesn't look
different when we look at it from a moving frame of reference.  We
say it's "Lorentz-invariant".  But if we look at an electric field 
in a moving frame of reference, we see a bit of magnetic field
added on, and vice versa.   We say that the electric and magnetic
fields transform into each other... they are two aspects of single
thing, the electromagnetic field.  So the amount of \emph{electric} screening
or antiscreening in the vacuum has to equal the amount of the 
\emph{magnetic} screening or antiscreening.  In other words, thanks to
the silly way we defined \epsilon  differently from \mu , we must have

$$

                       \epsilon  = 1/\mu 
$$
    
in the vacuum.  
Now the cool thing is that the Yang-Mills equations, which describe
the strong force, are very similar to Maxwell's equations.  In 
particular, the strong force, also known as the "color" force, 
consists of two aspects, the "chromoelectric" field and "chromomagnetic"
field.  Moreover, the same argument above applies here: the vacuum
must give the same antiscreening for the chromoelectric field as
it does for the chromomagnetic field, so \epsilon  = 1/\mu  here too.
So to understand asymptotic freedom it is sufficient to see why the
vacuum acts like a paramagnet for the strong force!   This depends
on a big difference between the strong force and electromagnetism.
Just as the electromagnetic field is carried by photons, which are
spin-1 particles, the strong force is carried by "gluons", which
are also spin-1 particles.  But while the photon is electrically 
uncharged, the gluon is charged as far as the strong force goes: we 
say it has "color".  
The vacuum is bustling with virtual gluons.  When we apply a chromomagnetic
field to the vacuum, we get two competing effects: paramagnetism thanks
to the \emph{spin} of the gluons, and diamagnetism due to their <em>orbital
angular momentum</em>.  But - the spin effect is stronger.  The vacuum
acts like a paramagnet for the strong force.  So we get asymptotic
freedom!
That's the basic idea.  Of course, there are some loose ends.
To see why the spin effect is stronger, you have to calculate a bit.  
At least I don't know how to see it without calculating - but Wilczek 
sketches the calculation, and it doesn't look too bad.   It's also true 
in most metals that the spin effect wins, so they are paramagnetic. 
You might also wonder why spin and orbital angular momentum work
oppositely as far as magnetism goes.  Unfortunately I don't have any
really simple slick answer.   One thing is that it seems any answer
must involve quantum mechanics.  [Note: later I realized
some very basic things about this, which I append below.]
In volume II of his magnificent 
series:
4) Richard Feynman, Robert Leighton, and Matthew Sands, "The
Feynman Lectures on Physics", Addison-Wesley, Reading, Mass., 1964. 
Feynman notes: "It is a consequence of classical mechanics that
if you have any kind of system - a gas with electrons, protons, and
whatever - kept in a box so that the whole thing can't turn, there
will be no magnetic effect.  [....]  The theorem then says that if
you turn on a magnetic field and wait for the system to get into
thermal equilibrium, there will be no paramagnetism or diamagnetism -
there will be no induced magnetic moment.  Proof: According to statistical
mechanics, the probability that a system will have any given state
of motion is proportional to exp(-U/kT), where U is the energy of
that motion.  Now what is the energy of motion.  For a particle moving
in a constant magnetic field, the energy is the ordinary potential energy
plus mv^2/2, with nothing additional for the magnetic field.  (You
know that the forces from electromagnetic fields are q(E + v x B),
and that the rate of work F.v is just qE.v, which is not affected by
the magnetic field.)  So the energy of a system, whether it is in
a magnetic field or not, is always given by the kinetic energy plus
the potential energy.  Since the probability of any motion depends only
on the energy - that is, on the velocity and position - it is the same
whether or not there is a magnetic field.  For \emph{thermal} equilibrium,
therefore, the magnetic field has no effect."   
So to understand magnetism we really need to work quantum-mechanically.
Laurence Yaffe has brought to my attention a nice path-integral argument
as to why orbital angular momentum can only yield diamagnetism; this
can be found in his charming book:
5) Barry Simon, "Functional Integration and Quantum Physics", Academic
Press, 1979.
This argument is very simple if you know about path integrals, but
I think there should be some more lowbrow way to see it, too.  I think
it's good to make all this stuff as simple as possible, because
the phenomena of asympotic freedom and confinement are very important
and shouldn't only be accessible to experts.  
I'd like to thank Douglas Singleton, Matt McIrvin, Mike Kelsey, and
Laurence Yaffe for some posts on sci.physics.research that helped
me understand this stuff.

\par\noindent\rule{\textwidth}{0.4pt}

<H3> Addendum </H3>
<H4> November 13, 1996 </H4>

Thanks to email from Yehuda Naveh and Bruce Smith
I'm beginning to understand this stuff at the 13-year-old level it deserves.
If you want to jump to the punchline, skip down to the stuff between double
lines - that's the part I should have known ages ago!
Here's the deal.  Feynman's theorem deals with classical systems made only 
of a bunch of electrically charged point particles.  Remember how it goes:
A magnetic field can never do work on such a system, because it always 
exerts a force perpendicular to the velocity of an electrically charged 
particle.  So the energy of such a system is independent of the externally 
applied magnetic field.  Now, in statistical mechanics the equilibrium state of
a system depends only on the energy of each state, since the probability
of being in a state with energy E is proportional to exp(-E/kT).  So
an external magnetic field doesn't affect the equilibrium state of
this sort of system.  So there can't be anything like paramagnetism
or diamagnetism, where the equilibrium state is affected by an external
magnetic field.
But suppose instead we allowed an extra sort of building block of our
system, in addition to electrically charged particles.  Suppose we
allow little "current loops".  We take these as "primitives", in the
sense that we don't ask how or why the current keeps flowing around
the loop, we just assume it does.  We just \emph{define} one of these 
"current loops" to be a little circle of stuff with a constant 
mass per unit length, with a constant current that flows around it.  
This may or may not be physically reasonable, but we're gonna do it anyway!
Note: If we tried to make a current loop out of classical electrically
charged point particles, the current loop would tend to fall apart!  A
loop is not going to be the equilibrium state of a bunch of charged
particles.  So we are going to get around this by taking current loops
as new primitives - simply \emph{assuming} 
they exist and have the properties
given above.
If we build our system out of current loops and point particles, 
Feynman's theorem no longer applies.   Why?  Well, a constant magnetic
field exerts a force perpendicular to the direction of the current, 
and this applies a \emph{torque} to the current loop - no net force, just
a torque.  But since the current loop is made out of stuff that has
a constant mass per unit length, when the current loop is rotating
it will have kinetic energy.  So by applying a torque to the current loop,
the magnetic field does \emph{work} on the current loop.  Thus Feynman's
reasoning no longer applies to this case.
In particular, what happens is just what we expect.  The torque on
the little current loops makes them want to line up with the external
magnetic field.  In other words, they will have less energy when they 
are lined up like this.  In particular, the energy of the system
\emph{does} depend on the external magnetic field, and the equilibrium
state will tend to have more little current loops lined up with the field
than not.
Now if we keep track of the magnetic field produced by these current
loops, we see it points the same way as the externally applied field.
So we get paramagnetism.
Now, even without doing a detailed quantum-mechanical treatment of
this problem, we see what's special about spin: a particle with
spin is a bit like one of our imaginary "primitive current loops".
This is how spin can give paramagnetism.  
Great.  But what had always been bugging me is this!  If you
put a charged particle in a constant magnetic field, it moves
in a circular or spiral orbit.  For simplicity let's say it moves
in a circle.  You can think of this, if you like, as a kind of current
loop - but a very different sort of current loop than the one we've 
just been considering!  In particular, if you work it out, this particle
circling around will produce a magnetic field that \emph{opposes} the 
external magnetic field.  On the other hand, our primitive current 
loops are in the state of least energy when they're lined up to produce 
a magnetic field that \emph{goes with} the external field.  
What's the deal?  Well, it's just something about how the vector cross
product works; you gotta work it out yourself to believe it.   All
you need to know is that the force on a charged particle is q v x B.
It boils down to this:
\par\noindent\rule{\textwidth}{0.4pt}
A positively charged particle orbiting in a magnetic field pointing
along the z axis will orbit CLOCKWISE in the xy plane.  However,
a primitive current loop in a magnetic field pointing along
the z axis will be in its state of least energy when the current
runs COUNTERCLOCKWISE in the xy plane.
\par\noindent\rule{\textwidth}{0.4pt}
I'm sure this is what was nagging at me.  It's just one of those
basic funny little things.  If I'm still mixed up, someone
had better let me know.  
There are a couple other things perhaps worth saying about this:
<OL>
<LI> In our calculation of the energy of the system, we have
been neglecting the energy due to the electric and magnetic fields
\emph{produced} by our point particles and current loops.  A more
careful analysis would take these into account.  In particular,
the reason ferromagnets prefer to have lots of "domains" than to have
all their little current loops lined up, is to keep the energy
due to the magnetic field produced by these loops from getting too big.
<LI>
A little current loop acts like a magnetic dipole.  We'd
also get interesting effects if we had magnetic monopoles. 
Here I simply assume that, just as an electric field exerts a force on
a electrically charged particle equal to q E, a magnetic field
exerts a force on a magnetically charged particle equal to m B,
where m is the magnetic charge.  A magnetic field would then
be able to do work on a magnetic monopole, and again Feynman's
theorem would not apply.  So it's perhaps not so surprising that
Feynmans' theorem fails when we have magnetic dipoles as primitive
constituents of our system, too (although these dipoles had better
not be points - they need a moment of inertia for a torque on
them to do work).  
</OL>
\par\noindent\rule{\textwidth}{0.4pt}

% </A>
% </A>
% </A>
