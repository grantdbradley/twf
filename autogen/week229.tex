
% </A>
% </A>
% </A>
\week{April 13, 2006}

I'm visiting Chicago now.  I came just in time for a conference
in honor of Saunders Mac Lane, one of the founders of category
theory, who taught at the University of Chicago for many years
and died last year at the age of 95:

1) Category Theory and its Applications: A Conference in Memory
of Saunders Mac Lane, <A HREF = "http://www.math.uchicago.edu/~may/MACLANE/">http://www.math.uchicago.edu/~may/MACLANE/</A>

On Friday there was a memorial service where the friends and family 
of Mac Lane spoke about him, and a kind of reminiscence session 
where everyone could tell their favorite stories involving him.  
Then there were a bunch of math talks, both by people with strong 
connections to Mac Lane - Peter Johnstone, Bill Lawvere, Peter Freyd, 
Ieke Moerdijk, Peter May and Steve Awodey - and by people
working on higher categories and their applications.  

My own connection to Mac Lane is tiny.  Everything 
I do uses his work, but that's true of lots of mathematicians: he
discovered so much.  Apart from watching him celebrate his 90th birthday 
at a category theory conference in Portugal back in 1999, the best moment 
happened when I came to Chicago and gave a talk.  He invited me up to his 
office and we talked a bit.  He told me I should write a book explaining 
n-categories.  I promised I would... I was too shy to say much.

Now I'm trying to write that book, and I just happen to be staying in 
Mac Lane's old office, which makes me feel especially obliged to do it.  
This office is on the third floor of the Ryerson Physical Laboratory.
It has a very high ceiling, and one wall is lined with two stacks of 
metal bookshelves.  You'd need a ladder to reach the top!  When I spoke
to Mac Lane in his office, they were all full of books.  Alas, they're 
empty now.

Next time I'll say a bit about Julie Bergner's talk at the Maclane
memorial conference - she spoke about derived categories of quiver
representations and quantum groups.   But the conference was so intense 
and exhausting that first I need to recover by thinking about something 
completely different.  So, I'll concentrate on last week's puzzle about 
rational tangles.

But first: the astronomy picture of the week!

It'll be more fun after a little background.  The northern part of 
Mars is very different from the rest.   It's much smoother, and the 
altitude is much less:

<DIV ALIGN = CENTER>
<A HREF = "http://photojournal.jpl.nasa.gov/catalog/?IDNumber=PIA02820">
<IMG SRC = "mars_topography.gif">
% </A>
</DIV>

2) Linda M. V. Martel, Ancient floodwaters and seas on Mars,
<A HREF = "http://www.psrd.hawaii.edu/July03/MartianSea.html">
http://www.psrd.hawaii.edu/July03/MartianSea.html</A>

Why is this?

Many scientists believe the north was an ocean during the Hesperian 
Epoch, a period of Martian history that stretches from about 3.5 to 
about 1.8 billion years ago.   In particular, the beautifully named 
"Vastitas Borealis", an enormous plain that covers most of northern 
Mars, has textures that may have been formed by an ocean that froze 
and then slowly sublimated.  (Sublimation is what happens when ice 
turns directly into water vapor without actually melting.)  
Mike Carr and James Head wrote a paper suggesting that around the end of the 
Hesperian, about 30% of the water on Mars evaporated and left the 
atmosphere, drifting off into outer space... part of the danger of 
life on a planet without much gravity:

3) M. H. Carr and J. W. Head, III, Oceans of Mars: An assessment of 
the observational evidence and possible fate, Journal of Geophysical 
Research 108 (2003), 5042.

The rest of the water is now frozen at the poles or lurking underground.

And that brings us to our picture.  Here's some ice in a crater in 
Vastitas Borealis:

<DIV ALIGN = CENTER>
<A HREF = "http://www.esa.int/esa-mmg/mmg.pl?b=b&type=I&mission=Mars%20Express&single=y&start=53&size=b">
<IMG HEIGHT = 500 WIDTH = 666 SRC = "mars_crater_ice.jpg">
% </A>
</DIV>

4) Water ice in crater at Martian north pole, European Space Agency (ESA), <a href = "http://www.esa.int/esaMI/Mars_Express/SEMGKA808BE_3.html">http://www.esa.int/esaMI/Mars_Express/SEMGKA808BE_3.html</a>

Perhaps this is a remnant of a once mighty ocean!

The picture is close to natural color, but the vertical relief is 
exaggerated by a factor of 3.  The crater is 35 kilometers wide 
and 2 kilometers deep.  It's incredible how they can get this kind
of picture from satellite photos and lots of clever image processing.  
I hope they didn't do \emph{too} much stuff just to make it look pretty.

Next: rational tangles.

In "<A HREF = "week228.html">week228</A>", I asked for help understanding the connection between
rational tangles and the group PSL(2,Z).  I got a great reply from
Michael Hutchings, which winds up relating these ideas to the branched
double cover of the sphere by the torus.  And, this gives me an excuse 
to tell you some stuff I learned from James Dolan about elliptic functions 
and a map of the world called "Peirce's quincuncial".   

So, let's dive in!

Did you ever try to wrap a sphere around itself twice?  Mentally,
I mean?  Slit it open, grab it, pull it, stretch it, wrap it around 
itself twice, and glue the seams back together?

It's not hard.  You just take the Riemann sphere - the complex 
numbers together with a point at infinity - and map it to itself 
by the function

f(z) = z^{2}

If you think of the sphere as the surface of the Earth, with zero 
at the south pole and infinity as the north pole, this function doubles 
the longitude.  So, it wraps the sphere around itself twice!   

I hope you're visualizing this.

This function is not quite a "double cover", because it's not quite 
two-to-one everywhere.  Only one point gets mapped to z = 0, namely 
itself, and only one point gets mapped to z = \infty , namely itself.  
Elsewhere f is two-to-one.   

If you walk once around the north pole or south pole, and then apply 
the function f to your path, you get a path that goes around these 
points \emph{twice}.   Summarizing these properties, we call the function 
a "branched double cover" of the sphere by itself, with zero and 
infinity as branch points.  

Now, how about wrapping a torus twice around a sphere?  

This too can be done.  It turns out there's a nice branched double 
cover of the sphere by the torus, which has four branch points.

To visualize this, first take the surface of the Earth and mold it 
into a regular octahedron.  There will be six corners: the north pole, 
the south pole, the east pole, the west pole, the front pole and the 
back pole.  Now take the octahedron and unfold it like this:

<DIV ALIGN = CENTER>

\begin{verbatim}

S-----B-----S
|    /|\    |
|   / | \   |
|  /  |  \  |
| /   |   \ |
|/    |    \|
W-----N-----E
|\    |    /|
| \   |   / |
|  \  |  /  |
|   \ | /   |
|    \|/    |
S-----F-----S
\end{verbatim}
    
</DIV>
We get an interesting map of the world, which was invented in 1876
by the American mathematician and philosopher C. S. Peirce while he
was working at the U. S. Coast and Geodetic Survey.   This map is 
called "Peirce's quincuncial", since when you arrange five dots this 
way:

<DIV ALIGN = CENTER>

\begin{verbatim}

o   o
  o  
o   o
\end{verbatim}
    
</DIV>
it's called a "quincunx".  (Somehow this word
goes back to the name of an ancient Roman coin.  I don't understand 
how this pattern is related to the coin.)

This is how Peirce's quincuncial looks as an actual map:

<DIV ALIGN = CENTER>
<A HREF = 
"http://www.progonos.com/furuti/MapProj/Normal/ProjConf/projConf.html">
<IMG SRC = "quincuncial.jpg">
% </A>
</DIV>

5) Carlos A. Furuti, Conformal projections, 
<A HREF = "http://www.progonos.com/furuti/MapProj/Normal/ProjConf/projConf.html">http://www.progonos.com/furuti/MapProj/Normal/ProjConf/projConf.html</A>

The cool part is that you can tile the plane indefinitely with 
this map:

<DIV ALIGN = CENTER>

\begin{verbatim}

S-----B-----S-----F-----S-----B-----S-----F-----S
|    /|\    |    /|\    |    /|\    |    /|\    |
|   / | \   |   / | \   |   / | \   |   / | \   |
|  /  |  \  |  /  |  \  |  /  |  \  |  /  |  \  |
| /   |   \ | /   |   \ | /   |   \ | /   |   \ |
|/    |    \|/    |    \|/    |    \|/    |    \|
W-----N-----E-----N-----W-----N-----E-----N-----W
|\    |    /|\    |    /|\    |    /|\    |    /|
| \   |   / | \   |   / | \   |   / | \   |   / |
|  \  |  /  |  \  |  /  |  \  |  /  |  \  |  /  |
|   \ | /   |   \ | /   |   \ | /   |   \ | /   |
|    \|/    |    \|/    |    \|/    |    \|/    |
S-----F-----S-----B-----S-----F-----S-----B-----S
|    /|\    |    /|\    |    /|\    |    /|\    |
|   / | \   |   / | \   |   / | \   |   / | \   |
|  /  |  \  |  /  |  \  |  /  |  \  |  /  |  \  |
| /   |   \ | /   |   \ | /   |   \ | /   |   \ |
|/    |    \|/    |    \|/    |    \|/    |    \|
E-----N-----W-----N-----E-----N-----W-----N-----E
\end{verbatim}
    
</DIV>

<br>

<DIV ALIGN = CENTER>
<A HREF = 
"http://www.progonos.com/furuti/MapProj/Normal/ProjConf/projConf.html">
<IMG SRC = "quincuncial_tiled.jpg">
% </A>
</DIV>

This gives a branched cover of the sphere by the plane!  It 
has branch points at the east, west, front and back poles,
since walking once around a point like that on the above map 
corresponds to walking around it twice on the actual Earth.
This is pretty weird, but Peirce cleverly located two of these
branch points in the Pacific Ocean, one in the Atlantic, and one in 
the Indian Ocean.

We can be less extravagant and get a branched cover of the 
sphere by the torus if we take the smallest parallelogram 
whose opposite edges match up:

<DIV ALIGN = CENTER>

\begin{verbatim}

            B            
           /|\           
          / | \          
         /  |  \         
        /   |   \        
       /    |    \       
      E-----N-----W      
     /|\    |    /|\     
    / | \   |   / | \    
   /  |  \  |  /  |  \   
  /   |   \ | /   |   \  
 /    |    \|/    |    \ 
B-----S-----F-----S-----B
 \    |    /|\    |    / 
  \   |   / | \   |   /  
   \  |  /  |  \  |  /   
    \ | /   |   \ | /    
     \|/    |    \|/     
      W-----N-----E      
       \    |    /       
        \   |   /        
         \  |  /         
          \ | /          
           \|/           
            B            
\end{verbatim}
    
</DIV>

This would actually be a square if I could draw it right in ASCII.
We can curl this into a torus by gluing together the opposite edges. 
There's then an obvious function from this torus to the sphere
sending both points labelled "N" to the north pole, both points 
labelled "S" to the south pole, and so on.  

This function is mostly two-to-one, but it's one-to-one at the 
points labelled E, F, W, and B.  After all, there's just \emph{one}
point of each of these sorts in the above picture after we glue 
together the opposite edges.  There are \emph{two} copies of any other 
sort of point.

So, our function is a branched double cover of the sphere by 
the torus, which has four branch points.  In fact, this function 
is quite famous.  It's an example of an "elliptic function"!   

I explained elliptic functions way back in 
"<A HREF = "week13.html">week13</A>".  Briefly,
what we just did starting with this parallelogram:

<DIV ALIGN = CENTER>

\begin{verbatim}

            B            
           /|\           
          / | \          
         /  |  \         
        /   |   \        
       /    |    \       
      E-----N-----W      
     /|\    |    /|\     
    / | \   |   / | \    
   /  |  \  |  /  |  \   
  /   |   \ | /   |   \  
 /    |    \|/    |    \ 
B-----S-----F-----S-----B
 \    |    /|\    |    / 
  \   |   / | \   |   /  
   \  |  /  |  \  |  /   
    \ | /   |   \ | /    
     \|/    |    \|/     
      W-----N-----E      
       \    |    /       
        \   |   /        
         \  |  /         
          \ | /          
           \|/           
            B            
\end{verbatim}
    
</DIV>

actually works for a parallelogram of any shape.  The parallelogram 
curls up to give a torus, and we get a map from this torus to the 
Riemann sphere, called an "elliptic function".  

As before, this is a branched double cover with four branch points.
However, where the branch points sit on the sphere depends on the
shape of the parallelogram.  By picking the parallelogram carefully,
you can put the branch points wherever you want!  Peirce's neat idea
was to put them evenly spaced along the equator - at the east, front, 
west and back poles.  This is nice and symmetrical.

It's also especially nice to put the branch points at the vertices 
of a regular tetrahedron.  I'm not sure, but this may give a map 
developed by the cartographer Laurence P. Lee in 1965.  There's also
a picture of this on Furuti's webpage:

<DIV ALIGN = CENTER>
<IMG SRC = "triangular_map.jpg">
</DIV>

In fact, these two specially nice locations for branch points 
correspond to the two most symmetrical lattices in the plane: 
the square one and the hexagonal one.  I talked about these in
"<A HREF = "week125.html">week125</A>" - they're really important in the theory of elliptic
functions, and even in string theory.

Anyway: for any parallelogram we can make a map of the Earth that 
tiles the plane, with tiles shaped like this parallelogram.   
A cool thing about these maps is that they're all "conformal" - 
they preserve angles except at the branch points.  If you want 
to show off, you express this by saying "elliptic functions are 
complex analytic".

But now I'm digressing a little.  Let's get back on track.  What 
does all this have to do with rational tangles??

Recall my puzzle from last time.  We build rational tangles by 
starting with the trivial one, which we call "zero"


\begin{verbatim}

  |   |
  |   |
  |   |
  |   |
\end{verbatim}
    
and repeatedly doing two operations.  The first is a twisting 
operation that we call "adding one":

\begin{verbatim}

  |   |                |   |
  |   |                |   |
  |   |                |   |
 -------              -------
 |  T  |   |---->     |  T  |        =  "T + 1"
 -------              -------
  |   |                 \ /
  |   |                  / 
  |   |                 / \
\end{verbatim}
    
where the box labelled "T" stands for any tangle we've built 
so far.  The second is a rotation that we call "negative reciprocal":
 

\begin{verbatim}

  |   |             |     |     
  |   |             |     |    ____
  |   |             |     |   /    \
 -------            |    -------    |
 |  T  |   |---->   |    |  T  |    |    =  "-1/T"
 -------            |    -------    |
  |   |              \___/   |      |
  |   |                      |      |
  |   |                      |      |
\end{verbatim}
    
Using these tricks we can try to assign a rational number to
any rational tangle.  The shocking theorem is that this number
is indeed well-defined, and in fact a complete invariant of 
rational tangles.

Every operation built from "adding one" and 
"negative reciprocal" looks like this:


\begin{verbatim}

         az + b
z |->   -------
         cz + d
\end{verbatim}
    
with a,b,c,d integer and ad-bc = 1.  The group of these transformations 
is called PSL(2,Z).   This group acts on the rational numbers together 
with a point at infinity (the "rational projective line") by the formula 
above.  It also acts on rational tangles.  The puzzle is to see why these
actions are isomorphic.  The proofs I listed in "<A HREF = "week228.html">week228</A>" show it's true;
the problem is to understand what's really going on!

Here's the answer given by Michael Hutching on sci.math.research:

\begin{quote}
 There's a simple topological interpretation of the element of the
 rational projective line associated to a rational tangle.  I don't know
 how to use this to prove the theorem, and I don't know a reference for
 it (maybe it is in one of the references you cited).  Anyway, regard a
 rational tangle as a two-component curve C in the 3-ball B^{3} 
  whose four
 boundary points are on the 2-sphere S^{2}.  
 Consider the double branched
 cover of B^{3} along C.  This is a 3-manifold Y whose boundary can be
 identified with the 2-torus T^{2}.  (In fact Y is a solid torus.)  The
 inclusion of T^{2} into Y induces a map from 
 H_{1}(T^{2}) to H_{1}(Y), and the
 kernel of this map is a one-dimensional subspace of H_{1}(T^{2}) = Z^{2}.  If
 I am not mistaken, this is the element in question of the rational
 projective line. 
\end{quote}

In other words, we take a 3-dimensional ball and draw a picture
of a rational tangle in it:

<DIV ALIGN = CENTER>

\begin{verbatim}

    .......   
  .   | |   . 
 .   -----   .
 .  |  T  |  .
 .   -----   .
  .   | |   . 
    .......   
\end{verbatim}
    
</DIV>

The boundary of this ball is a sphere with 4 points marked.  If we 
take a branched double cover of the sphere with these as the branch
points, we get a torus T^{2}.   If we take a branched double cover of 
the whole ball with everything along the vertical lines as branched 
points, we get a solid doughnut Y having T^{2} as its boundary.  

This gets the torus into the game, and also the branched cover I was 
talking about.  And this gets the group PSL(2,Z) into the game!  
SL(2,Z) is the group of 2\times 2 matrices with determinant 1.  When you 
mod out by the matrices +-1, you get PSL(2,Z).  But, topologists know 
that SL(2,Z) is the "mapping class group" of the torus - the 
group of 
orientation-preserving diffeomorphisms modulo those that can be 
smoothly deformed to the identity.

So, something nice is happening.  

Even better, the rational first homology group of the torus is Q^{2}
(pairs of rational numbers), and SL(2,Z) acts in the obvious way, 
by matrix multiplication:
  

\begin{verbatim}

 a  b      x       ax + by
       :      |->     
 c  d      y       cx + dy
\end{verbatim}
    
It therefore acts on the set of 1-dimensional subspaces of Q^{2}.
Any such subspace consists of vectors like this:

\begin{verbatim}

 kx 

 ky 
\end{verbatim}
    
The subspace is determined by the ratio x/y, which however could be 
infinite - so it's just a point in the rational projective line.  So,
we get an action of SL(2,Z) on the rational projective line.  Indeed
we get an action of PSL(2,Z) since +-1 act trivially.  And, you can 
easily check that it's the action we've already seen:


\begin{verbatim}

 a  b             az + b
       : z  |->  --------
 c  d             cz + d
\end{verbatim}
    
In short: "projectivizing" the action of mapping class group of the 
torus on its first homology gives the usual action of PSL(2,Z) on the 
rational projective line.  

What we need next is a natural way to assign to any rational tangle 
a 1-dimensional subspace of the homology of the torus.  And this is 
what Hutchings describes: a rational tangle gives a way of mapping 
the torus T^{2} into the solid torus Y, and this gives a 
map on rational homology

H_{1}(T^{2}) \to  H_{1}(Y)

whose kernel is a 1-dimensional subspace of H_{1}(T^{2}).   

There's more stuff to check....
 
Personally I've been trying to think of the mapping class group 
of the 4-punctured sphere as acting on pictures like this: 

<DIV ALIGN = CENTER>

\begin{verbatim}

    .......   
  .   | |   . 
 .   -----   .
 .  |  T  |  .
 .   -----   .
  .   | |   . 
    .......   
\end{verbatim}
    
</DIV>
and show that the resulting action on rational tangles factors
through a homomorphism from this mapping class group to PSL(2,Z).
The mapping class group should be generated by the twist 


\begin{verbatim}

  |   |                |   |
  |   |                |   |
  |   |                |   |
 -------              -------
 |  T  |   |---->     |  T  |        
 -------              -------
  |   |                 \ /
  |   |                  / 
  |   |                 / \
\end{verbatim}
    
and the 90 degree rotation


\begin{verbatim}

  |   |             |     |     
  |   |             |     |    ____
  |   |             |     |   /    \
 -------            |    -------    |
 |  T  |   |---->   |    |  T  |    |    
 -------            |    -------    |
  |   |              \___/   |      |
  |   |                      |      |
  |   |                      |      |
   
\end{verbatim}
    
and our homomorphism should map these to the famous matrices


\begin{verbatim}

      1   1             
T  =                    "shear"
      0   1
\end{verbatim}
    
and 


\begin{verbatim}

      0  -1          
S =                   "90 degree rotation"
      1   0
\end{verbatim}
    
respectively.   If this works, and I could figure out the kernel of 
this homomorphism and show it acts trivially on rational tangles, 
I think I'd be almost done.  But, I haven't had time!

By the way, if this works, there's a beautiful little sideshow where 
we use as generators of SL(2,Z) not the above matrices but S and 


\begin{verbatim}

      0  -1
ST = 
      1   1
\end{verbatim}
    

I explained why these are so great in "<A HREF = "week125.html">week125</A>".  S is a symmetry of
the square lattice, while ST is a symmetry of the hexagonal lattice.
The square lattice gives Peirce's quincuncial map, while the hexagonal
one presumably gives Laurence Lee's triangular map!

So, there's some intriguing story about elliptic functions and rational
tangles taking shape before our eyes.... and if I weren't so darn busy,
I'd figure out all the details and write a little paper about it. 

Before quitting, there's one more thing I can't resist mentioning.
Any ordered 4-tuple of points (a,b,c,d) in the Riemann sphere gives 
a number called its "cross-ratio":

(a-b)(c-d) / (a-d)(c-b)

It's a famous fact that you can find a conformal transformation of
the Riemann sphere mapping one ordered 4-tuple to another if and 
only if their cross-ratios are equal!  

So, we can play a little trick.   Given a lattice we can get a 
branched double cover of the Riemann sphere as I sketched earlier.
Then we can use the location of the branch points to calculate a cross 
ratio.  

But actually, I'm being a bit sloppy here.  To compute a cross ratio 
from a lattice, we need some extra information to \emph{order} the 4-tuple 
of branch points.  In other words, if one of the points S is the origin
here:

<DIV ALIGN = CENTER>

\begin{verbatim}

S-----B-----S-----F-----S-----B-----S-----F-----S
|    /|\    |    /|\    |    /|\    |    /|\    |
|   / | \   |   / | \   |   / | \   |   / | \   |
|  /  |  \  |  /  |  \  |  /  |  \  |  /  |  \  |
| /   |   \ | /   |   \ | /   |   \ | /   |   \ |
|/    |    \|/    |    \|/    |    \|/    |    \|
W-----N-----E-----N-----W-----N-----E-----N-----W
|\    |    /|\    |    /|\    |    /|\    |    /|
| \   |   / | \   |   / | \   |   / | \   |   / |
|  \  |  /  |  \  |  /  |  \  |  /  |  \  |  /  |
|   \ | /   |   \ | /   |   \ | /   |   \ | /   |
|    \|/    |    \|/    |    \|/    |    \|/    |
S-----F-----S-----B-----S-----F-----S-----B-----S
|    /|\    |    /|\    |    /|\    |    /|\    |
|   / | \   |   / | \   |   / | \   |   / | \   |
|  /  |  \  |  /  |  \  |  /  |  \  |  /  |  \  |
| /   |   \ | /   |   \ | /   |   \ | /   |   \ |
|/    |    \|/    |    \|/    |    \|/    |    \|
E-----N-----W-----N-----E-----N-----W-----N-----E
\end{verbatim}
    
</DIV>
and the lattice is taken just big enough so the pattern 
repeats, we need enough information to \emph{label}
the points E, F, W and B.  This extra information amounts to 
"choosing a basis for the 2-torsion subgroup of the plane modulo 
the lattice".  So, the cross ratio gives a "modular function
of level 2".  

Hmm, this is getting pretty jargonesque!  I don't want to explain the
jargon now, but you can read all about this trick and its consquences 
in Lecture 9 here:

6) Igor V. Dolgachev, Lectures on modular forms, Fall 1997/8,
available at <A HREF = "http://www.math.lsa.umich.edu/~idolga/modular.pdf">http://www.math.lsa.umich.edu/~idolga/modular.pdf</A>

\par\noindent\rule{\textwidth}{0.4pt}
\textbf{Addenda:}  Andrei Sobolevskii points out that the
etymology of the word "quincunx" is explained here:

7) Quincunx, World Wide Words,
<A HREF = "http://www.worldwidewords.org/weirdwords/ww-qui2.htm">
http://www.worldwidewords.org/weirdwords/ww-qui2.htm</A>

Very briefly, "quincunx" was a Latin word for
"five twelfths", from \emph{quinque} and \emph{uncia}.
The latter word is also the root of the word "ounce".
They had a copper coin called the \emph{as} weighing twelve
ounces (!), and the quincunx was apparently not a coin
a symbol for 5/12 of an \emph{as} - or in other words, 5 ounces
of copper.

After reading the above, Peter Dickof clarified and corrected 
the story: 

\begin{quote}
Love "This Week's Finds" (though I seldom follow it all) and can't 
pass up the opportunity to say something.

The \emph{as} was indeed a unit of currency and also a specific bronze coin. 
Early asses (\emph{aes} = bronzes - hence the AE ligature...) were full Roman 
pounds (\emph{librae} -hence the British pound "&#163;" sign) of 288 
scruples with twelve unciae to the pound.  Each uncia was ~27 grams, 
a modern ounce near enough.  Debasement set in around the time of the first 
Punic war and 
accelerated through the second (Hannibal's), by the end of which an as, 
still of bronze, weighed only ~30 grams.

Multiples of the as were minted (misnomer, this was a cast currency): 
the \emph{decussis} (X asses), \emph{quincussis} (V asses), 
\emph{tressis} (III asses), and the \emph{dupondius} (two-pounder). 
The asses were marked with the Roman 
numeral I. Common fractions were the \emph{semis} (half, marked with an S), 
\emph{triens} (third, sometimes called a quatrunx, marked with four 
"pellets" or dots), \emph{quadrans} (quarter, also 
\emph{teruncius}, 3 
pellets), \emph{sextans} (sixth, also biunx, 2 pellets), 
\emph{uncia} and \emph{semuncia} 
(usually unmarked).

There \emph{were} quincunx coins (5 pellets), and also a 
\emph{dextans} (S + 4 
pellets), mostly produced by non-Roman Italians. I have appended photos 
of three quincunxes produced in Luceria (see Thurlow and Vecchi numbers 274, 
281). Note that the "pellets" are sometimes (not always) arranged in a 
quincunx. Luceria (modern Lucera) is 2/3 of the way across the boot from 
Neapolis (Naples).

During the second Punic war, after they captured Syracuse and its 
treasure (and killed Archimedes), the Romans introduced the silver 
\emph{denarius}, 
\emph{quinarius} and 
\emph{sesterius}; they were worth 10, 5, 2 1/2 (IIS) 
asses. The denarius is the origin of the "d" for the British shilling, 
and was about the size of a dime. Later, circa 141 BC, the value of silver 
was re-tariffed so that denarii, quinarii and sestertii became worth 16, 
8, 4 asses; the names did not change but the quinarius and sestertius 
became rare.

Julius Caesar doubled the pay of a legionary to 300 sestertii per 
installment (stipendium), 3 installments per year.

Later yet, during Imperial Rome, the largest bronze/brass coin minted 
(no longer a misnomer) was a sestertius; its weight was less than
30 grams by the time of Claudius and falling, always falling....

Appropriate references are:

<UL>
<LI>
Michael H. Crawford, Roman Republican Coinage, Cambridge University
Press, Cambridge, 1974.   (See picture of Quincunx on plate XVIII.)
<LI>
Bradbury K. Thurlow, 
Italian Cast Coinage, Italian Aes Grave.
Italo G. Vecchi, Italian Aes Rude, Signatum and the Aes Grave of Sicily.
Printed together by Veechi, London, 1979.
<LI>
Herbert A. Grueber,
A Catalogue of the Coins of the Roman Republic
in the British Museum, three volumes, reprinted 1970.
</UL>

\end{quote}

My old pal Squark noted some sloppy language about branched covers.  
Here's my reply to what he wrote.  I've changed what he wrote a tiny
bit, for cosmetic reasons.

\begin{quote}
  Squark wrote:


\begin{verbatim}

  >Hello John and everyone!
\end{verbatim}
    

  Hello!  Long time no see!  How are you doing?  

  I had written:


\begin{verbatim}

  >There's a simple topological interpretation of the element of the
  >rational projective line associated to a rational tangle.  I don't know
  >how to use this to prove the theorem, and I don't know a reference for
  >it (maybe it is in one of the references you cited).  Anyway, 
  >regard a rational tangle as a two-component curve C in the 3-ball 
  >B^{3} whose four boundary points are on the 2-sphere S^{2}.
  >Consider the double branched cover of B^{3} along C.
\end{verbatim}
    
Squark wrote:

\begin{verbatim}

  >What is "_the_ double branched cover"? Is there a way to choose a
  >canonical one, or is there only one in this case, for some reason?
\end{verbatim}
    

  Good point.  I hope there's a specially nice one.  

  To pick a branched cover of B^{3} along C, it's necessary and sufficient
  to pick a homomorphism from the fundamental group of B^{3} - C to Z/2.
  This says whether or not the two sheets switch places as we walk around
  C following some loop in B^{3} - C.


\begin{verbatim}

  >In the case of a sphere with 4 points removed it should be easy to
  >check.
\end{verbatim}
    

  Yes.


\begin{verbatim}

  >The fundamental group has 4 generators - a, b, c, d (loops around 
  >each of the points) and one relation abc = d (since we're on a 
  >sphere). Hence, it is freely generated by a, b, c (say).
\end{verbatim}
    

  Right, the fundamental group of the four-punctured sphere is
  the free group on 3 generators, F_{3}.  I believe the "specially nice" 
  homomorphism 

  f: F_{3} \to  Z/2

  is the one that sends each generator to -1, where I'm thinking 
  multiplicatively:

  Z/2 = {1, -1}

  One reason this homomorphism is especially nice is that it also sends 
  d = abc to -1.  

  So, if you walk around ANY of the four punctures, the two sheets switch!

  This is just what you want for the Riemann surface of an elliptic 
  integral, as someone else pointed out in another post: there are 
  four branch points each like the branch point of \sqrt z.  It's 
  also the most symmetrical, beautiful thing one can imagine.

  Now let's see if and how this branched cover extends to a branched
  cover of the ball B^{3} with C (two arcs) removed.  The fundamental group
  of B^{3} - C is the free group on two generators, say X and Y.  

  The inclusion of the 4-punctured sphere in B^{3} - C gives a homomorphism

  g: F_{3} \to  F_{2}

  as follows:

  a |\to  X  <br>
  b |\to  X^{-1}  <br>
  c |\to  Y      <br>
  d |\to  Y^{-1}   <br>
  So, to extend our branched cover, we need to write our homomorphism

  f: F_{3} \to  Z/2

  as

  f = hg

  for some homomorphism

  h: F_{2} \to  Z/2

  The obvious nice thing to try for h is

  X |\to  -1 <br>
  Y |\to  -1
  It works, and it's unique!
\end{quote}


Lee Rudolph adds:

\begin{quote}

Squark wrote:


\begin{verbatim}

>What is "_the_ double branched cover"? Is there a way to choose a
>canonical one, or is there only one in this case, for some reason?
\end{verbatim}
    

John Baez wrote:


\begin{verbatim}

>Good point.  I hope there's a specially nice one.  
\end{verbatim}
    

In this kind of context, there's always exactly one "double branched 
cover" that actually \emph{does} branch doubly over every component of the 
proposed branch locus.  In particular, in the context of a rational
tangle, of course the pair (B^{3},C) is homeomorphic to (B^{2},X)\times I,
where X is a 2-point set in Int B^{2} and the homeomorphism isn't
required to preserve the tangle structure; so the double branched
cover of B^{3} branched over C is the product of the double branched
cover of B^{2} branched over X with the interval I. Now, because
the branching is \emph{double} at each point of X, and there are \emph{two}
points of X, it follows that the monodromy around the boundary of
B^{2} must be trivial, so that we can sew another B^{2} to that boundary
and extend the branched double covering over the resulting 2-sphere.
But of course the branched double cover of a 2-sphere over 2 points
is another 2-sphere, the model for the situation being z |\to  z^{2}
as a map of the Riemann sphere to itself.  Now remove the interior
of the sewed-on second B^{2} from the downstairs S^{2}, and correspondingly 
the interiors of its \emph{two} preimage B^{2}s from the upstairs S^{2}; you
see that the double cover of B^{2} branched over X is an annulus.
(Once you know that, you can see it directly: take an annulus embedded
in R^{3} as the cylinder where x^{2}+y^{2}=1 and -1 \le  z \le  1; rotate it
by 180 degrees around the x-axis, and convince yourself that the quotient
space is a 2-disk by considering the fundamental domain consisting
of those points of the annulus with non-negative y-coordinate.)
Then the double cover of B^{3} branched over C must be a solid torus.
(Again, now that you know this, you can see it directly: take the
solid torus to be a tubular neighborhood in R^{3} of the circle where
x^{2}+y^{2}=1 and z = 0, and again rotate by 180 degrees around the x-axis
to give yourself the "deck involution".)

Lee Rudolph
\end{quote}


\par\noindent\rule{\textwidth}{0.4pt}
<em>Well, what if we consider our lives to be formed of a 
series of interlocking practices, including the very important 
ones of maintaining a thriving family and community? 
Then we might learn from a practice with the pedigree of mathematics 


% parser failed at source line 1020
