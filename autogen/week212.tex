
% </A>
% </A>
% </A>
\week{March 26, 2005 }


As you may know, theoretical particle physics is highly enamored of 
"supersymmetry" these days.  This is not because there's a shred of 
experimental evidence for it - there's not - but just because it's such
a cool idea from a mathematical point of view.  Mathematicians should
have gotten this idea and run with it first, but physicists did - and
maybe it's turned them into mathematicians.

The unarguable central core of this idea is that everything is made of 
bosons and fermions.  In the Standard Model, most bosons are "force 
carriers", like photons, which carry the electromagnetic force.  Fermions 
are more like what we'd normally call "matter": leptons and quarks, 
for example.  The one big exception is the Higgs boson, which gives elementary 
particles their mass and... umm... hasn't been seen yet! 

But, at a more fundamental level, the really important thing is that
bosons commute:


\begin{verbatim}

xy = yx
\end{verbatim}
    
while fermions anticommute:


\begin{verbatim}

xy = -yx
\end{verbatim}
    
Also, in case you're wondering, bosons commute with fermions.

But already, most mathematicians reading this will be confused and unhappy.
What does it mean for two particles to commute, much less anticommute?  
Does an apple commute with a grape?   Here in the suburbs of Los Angeles 
almost everyone commutes, but that's not what we're talking about. 

The whole idea of particles commuting or anticommuting only occurred to 
people after they invented quantum theory, where the state of any 
system is described by a unit vector in some Hilbert space.  In quantum
theory, if you have a system in some state x, and you check to see if
it's in the state y, your experiment gives you the answer "yes" with
probability 


$$

|<x,y>|^{2}
$$
    
the square of the absolute value of the inner product of x and y.

There!  Now you know quantum theory.

Given this setup, when you have a system consisting of two particles, the 
first in some state x and the second in some state y, it's natural to 
write the state of the whole system as a kind of product xy.  But 
then you have to figure out what rules you want this product to satisfy!

If you require it to be commutative:


\begin{verbatim}

xy = yx
\end{verbatim}
    

you're saying that there's no difference between the \emph{first} particle
being in state x and the \emph{second} particle being in state y, and the
other way around.  In other words, the particles don't have little
name tags on them saying who they are. 

This seems reasonable, and particles satisfying this rule are called
"bosons".  But, there's another popular option, 
called "fermions":


\begin{verbatim}

xy = -yx
\end{verbatim}
    
Here again, the particles don't have name tags, since if we put
the whole system in the state xy and check to see if it's in the
state yx, we get the same answer as when we check to see if it's
in the state xy!  See:


\begin{verbatim}

|<xy,yx>|^{2} = |<xy,-xy>|^{2}

           = |<xy,xy>|^{2}
\end{verbatim}
    
thanks to the absolute value.  This means that the states xy and
yx are indistinguishable.   

Reading what I just said, you'd be forgiven for wondering what's the 
big difference between fermions and bosons!  After all, that absolute 
value in the formula for probabilities just ignores minus signs.

One difference is the "Pauli exclusion principle".  Take a pair of
fermions and check to see if they're both in the state x.  The 
probability is always zero, since


\begin{verbatim}

xx = -xx  
\end{verbatim}
    
so xx = 0.   So, fermions are antisocial: that's why the electrons
in an atom form "shells" with different electrons in different states,
instead of all hanging out at the lowest energy state. 

Bosons, by contrast, are gregarious: when a store clerk uses a laser 
scanner to ring up your purchases, that beam of red light is a bunch of 
photons all in the same state!   A laser is a quintessentially 
quantum-theoretic gadget - we live in a marvelous world, where such things are
taken for granted.

After getting used to these ideas for a while - Bose and Einstein worked 
out the idea of bosons in 1924, Pauli came up with his exclusion principle 
in 1925, and Dirac systematized the whole business in 1926 - physicists
eventually started looking for symmetries that relate bosons and fermions.
\emph{Supersymmetries!}  They're not seen in nature, but physicists 
were looking
to see if they're mathematically possible.  They turn out not only to
be possible, but fascinating.  

Formulating supersymmetries in a slick way requires that we take 
everything we knew about linear algebra and generalize it by letting
all our vector spaces have both an "even" or "bosonic" part and an
"odd" or "fermionic" part.  
Mathematically this just amounts to writing 
our vector space as a direct sum


$$

V = V_{0} + V_{1}
$$
    
where V_{0} is the "even part" and V_{1} is 
the "odd part".  Such a thing
is called a "Z/2-graded vector space", 
or "super vector space".  

So far this is pathetically simple.  But then - and this is the really 
crucial part! - whenever we multiply things, we have to follow this rule:


\begin{verbatim}

       even    odd
     ----------------
even|  even    odd
odd |  odd     even
\end{verbatim}
    
It's a little confusing, since this isn't what happens when you
multiply even and odd numbers - it's what happens when you ADD them.  
But, one quickly adapts.  

Also, when we generalize equations involving multiplication, we must 
remember to stick in an extra minus sign whenever we switch two odd 
vectors.  

So, for example, the usual concept of an algebra gets replaced 
by that of a "superalgebra".  This is a super vector space A
equipped with an associative product and unit such that when we
multiply even and/or odd vectors, the rules in the above table hold. 
We say a superalgebra is "supercommutative" if 


\begin{verbatim}

xy = yx
\end{verbatim}
    
when at least one of x,y lives in the even part A_{0}, while 


\begin{verbatim}

xy = -yx
\end{verbatim}
    
when both x and y live in A_{1}.   

Similarly we can define super Lie algebras, super Lie groups, 
supermanifolds, and so on.... 

People have done a lot of work on this stuff: it would take me days to 
explain it all - even longer if I actually knew something about it.  

But right now, I just want to zoom in the direction of super division 
algebras.  These are not the most important aspect of 
"superalgebra" - 
but they're pretty cool, and Todd Trimble has been explaining them to
me lately.  Everything interesting I'm about to say is due to him.

As you know, I'm inordinately fond of the normed division algebras:
the real numbers, complex numbers, quaternions and octonions.  They're
so beautiful, it's a little sad at times that there are only four!
Could superalgebra allow for more?

YES!  And, they turn out to be related to Bott periodicity.

Nobody seems to have pondered \emph{nonassociative} super division algebras 
yet, but Deligne has a nice article about the associative ones, which
I mentioned in "<A HREF = "week211.html">week211</A>".  I'll 
give more references later.

So, what's the idea?

I've already told you what a superalgebra is.  We say it's a "super 
division algebra" if every nonzero element that's purely even or 
purely odd is invertible.  

That's pretty easy.  What are they like?  

Well, I don't completely understand all the options yet, so I'll
just list the "central" super division algebras over the 
real numbers, 
namely those where the elements that supercommute with everything
form a copy of the real numbers.  There turn out to be 8, and their
beautiful patterns are best displayed in a circular layout:



\begin{verbatim}

                                   R

                                   \textbf{0}
           R[e] mod e^{2} - 1    \textbf{7}          \textbf{1}   R[e] mod e^{2} + 1
 

C[e] mod e^{2} - 1, ei + ie   \textbf{6}               \textbf{2}   C[e] mod e^{2} + 1, ei + ie
 

           H[e] mod e^{2} + 1   \textbf{5}           \textbf{3}   H[e] mod e^{2} - 1
                                   \textbf{4}
 
                                   H

\end{verbatim}
    
What does this notation mean?  Well, as usual R, C, and H stand for 
the reals, complex numbers, and quaternions.  In all but two cases,
we start with one of those algebras and throw in an odd element "e"
satisfying the relations listed: e is either a square root of +1 or
of -1, and in the complex cases it anticommutes with i.  

So, for example, super division algebra number 1:


$$

R[e] mod e^{2} + 1
$$
    
is just the real numbers with an odd element thrown in that satisfies
e^{2} + 1 = 0.  In other words, it's just the complex numbers made into
a superalgebra in such a way that i is \emph{odd}.

The real reason I've arranged these guys in a circle numbered from 
0 to 7 is to remind you of the Clifford algebra 
clock in "<A HREF = "week210.html">week210</A>",
where I discussed the super Brauer group of the real numbers, and
said it was Z/8.  

Indeed, the central super division algebras are a complete set of 
representatives for this super Brauer group!  In particular, the 
Clifford algebra C_{n} is super Morita equivalent to the nth algebra 
on this circle:


\begin{verbatim}

 C_{0} = R           ~ R
 C_{1} = C           ~ R[e] mod e^{2} + 1
 C_{2} = H           ~ C[e] mod e^{2} + 1, ei + ie
 C_{3} = H + H       ~ H[e] mod e^{2} - 1
 C_{4} = H(2)        ~ H
 C_{5} = C(4)        ~ H[e] mod e^{2} + 1
 C_{6} = R(8)        ~ C[e] mod e^{2} - 1, ei + ie
 C_{7} = R(8) + R(8) ~ R[e] mod e^{2} - 1
\end{verbatim}
    
where ~ means "super Morita equivalent", and the notation
for Clifford algebras was explained last week.  

I think this is cool.  I'm not quite sure what to do with it yet,
though.  How much of what people ordinarily do with division algebras
can be done with super division algebras?  For example, can we define
projective spaces over super division algebras?  (See "<A HREF = "week106.html">week106</A>" and
"<A HREF = "week145.html">week145</A>" for why that would be interesting.)

To read more about this, try:

1) Pierre Deligne, Notes on spinors, in Quantum Fields and Strings: 
A Course For Mathematicians, volume 1, American Mathematical Society, 
Providence, 1999.  Also available at 
<A HREF = "http://www.math.ias.edu/QFT/fall/spinors.ps">http://www.math.ias.edu/QFT/fall/spinors.ps</A>

A lot of the ideas go back to here:

2) C. T. C. Wall, Graded Brauer groups, J. Reine Angew. Math. 213
(1963/1964), 187-199.

and here's another good reference:

3) Peter Donovan and Max Karoubi, Graded Brauer groups and K-theory
with local coefficients, Publications Math. IHES 38 (1970), 5-25.
Also available at 
<A HREF = "http://www.math.jussieu.fr/~karoubi/Donavan.K.pdf">http://www.math.jussieu.fr/~karoubi/Donavan.K.pdf</A>

I should admit that I have a yearning to classify \emph{nonassociative}
super division algebras.  Has anyone ever tried this?  It's already 
plain to see that we have two 16-dimensional nonassociative super 
division algebras:


$$

O[e] mod e^{2} + 1
$$
    
and


$$

O[e] mod e^{2} - 1
$$
    

where e is an odd element that commutes with all the octonions.
(I should have mentioned this before, when talking about H[e]:
even though the quaternions are noncommutative, we assume that
e commutes with all of them.)  Maybe one of these algebras 
deserves to be called the \emph{superoctonions}.  I bet these or 
something awfully similar are lurking around in string theory.  

Hmm... next I wanted to write something about the topology of Bott
periodicity and how \emph{that} fits into what I've been discussing,
but I'm running out of energy.  Let me say it briefly, without much
detail, just in case I never get around to a decent explanation.  

Two super algebras are super Morita equivalent precisely when
they have equivalent categories of super representations.  So,
the super Brauer group really consists of 8 different \emph{categories}:
the categories SuperRep(C_{n}), where Bott periodicity says


$$

SuperRep(C_{n+8}) ~ SuperRep(C_{n})
$$
    
Moreover these are symmetric monoidal categories, since direct 
summing lets us "add" objects in these categories in a nice way.

A long time ago, Graeme Segal figured out how to take a symmetric
monoidal category and get an infinite loop space from it.
I explained this construction in "<A HREF = "week199.html">week199</A>", but for a much more
detailed and intense treatment with lots of references to earlier 
work, try:

4) R. W. Thomason, Symmetric categories model all connective 
spectra, Theory and Applications of Categories 1 (1995), 78-118.
Available at <A HREF = "http://www.tac.mta.ca/tac/volumes/1995/n5/1-05abs.html">http://www.tac.mta.ca/tac/volumes/1995/n5/1-05abs.html</A>
  
If we do this to SuperRep(C_{n}), I think we get something like 

\Omega ^{n}(kO)
that is, the n-fold loop space of something called kO, the "connective
K-theory spectrum", which I explained in "<A HREF = "week105.html">week105</A>".  The fact that this 
repeats with period 8:

\Omega ^{n+8}(kO) ~ \Omega ^{n}(kO)

is the topological version of Bott periodicity - see "<A HREF = "week105.html">week105</A>" for more.
So, we get the topological version of Bott periodicity from the algebraic
version by turning symmetric monoidal categories into infinite loop 
spaces!  

But, the interesting puzzle here is: what process can we do to 
SuperRep(C_{n})
to get SuperRep(C_{n+1}), which is the algebraic version of looping?
And I think the answer 
is: "taking super representations of C_{1} in it".
You see, 

C_{1} \otimes  C_{n} = C_{n+1}

where I'm using the super tensor product of superalgebras, and this
means that the category of representations of C_{1} 
in SuperRep(C_{n}) 
is SuperRep(C_{n+1}).  

And, if I were trying to really explain this instead of merely scribbling
notes about it, I would try to explain why this is because C_{1} 
is the complex
numbers, and the unit circle in the complex numbers is related to 
\emph{loops}.

But, sigh, that will have to wait.

One more thing before I quit for today...

I just saw a cool paper by Dror Bar-Natan, Thang Le and Dylan Thurston
about the "Duflo isomorphism".  This is a cousin of the 
Poincare-Birkhoff-Witt 
theorem, which in its best form says that the universal enveloping 
algebra UL of a Lie algebra L is isomorphic \emph{as a coalgebra} to the 
symmetric algebra SL.  You'll often see worse versions of the PBW theorem 
in textbooks, and ugly proofs, but James Dolan showed me the nice version 
and proof a while back.

The kernel of the idea is this: if L is the Lie algebra of a group G, 
UL consists of left-invariant differential operators on G, and there's 
a map UL \to  SL sending any differential operator to its "symbol".  
This is an isomorphism of vector spaces and even of coalgebras, but
not of algebras.  

Anyway, there's something vaguely similar relating the invariant 
subalgebras of UL and SL.  By "invariant" here, I mean that since 
L acts as derivations of UL and SL, we can look at the subalgebra of
either one consisting of guys who are killed by these derivations; 
such guys are called "invariant".   Physicists call invariant 
elements of UL "Casimirs", after the first physicist to think 
about this stuff.  They commute with everything else in 
UL.  Invariant elements of 
SL are like classical Casimirs: there's a Poisson bracket on SL, and
these are the guys whose Poisson bracket with everyone vanishes.

The Duflo map is an \emph{algebra isomorphism} from the invariant
subalgebra of SL to the invariant subalgebra of 
UL.  So, it's
like a very nice way to quantize Casimirs, one that gets along with
multiplication.  It's called the "Duflo map" because it was 
invented by
Harish-Chandra for semisimple Lie algebras and for Kirillov in general.
Kirillov conjectured that it was always an isomorphism; what Duflo
did is prove it: 

5) Michel Duflo, Operateurs differentiels bi-invariants sur un groupe
de Lie, Ann. Sci. Ecole Norm. Sup. 10 (1977), 265-288.

Apparently all known proofs are sort of hard!  According to Bar-Natan,
Le and Thurston:

\begin{quote}
   In the book of Dixmier, the proof is given only in the last chapter
   and it utilizes most of the results developed in the whole book, 
   including many classification results (a situation Godement called
   "scandalous").  As discussed below, there have been several recent
   proofs that do not use classification results, but they all use tools
   from well outside the natural domain of the problem.
\end{quote}
The proof by Bar-Natan, Le and Thurston uses the connection between
knot theory and Lie algebras - namely, the theory of Vassiliev
invariants.  I think there's still something slightly scandalous about 
this, but it's awfully interesting.  Anyway, take a look:

6) Dror Bar-Natan, Thang T. Q. Le and Dylan P. Thurston, Two
applications of elementary knot theory to Lie algebras and Vassiliev
invariants, Geometry and Topology 7 (2003), 1-31.  Available at
<A HREF = "http://www.maths.warwick.ac.uk/gt/GTVol7/paper1.abs.html">http://www.maths.warwick.ac.uk/gt/GTVol7/paper1.abs.html</A> 
and also
as <A HREF = "http://www.arXiv.org/abs/math.QG/0204311">math.QG/0204311</A>.

For more, try Thurston's thesis:

7) Dylan P. Thurston, Wheeling: a diagrammatic analogue of the Duflo
isomorphism, <A HREF = "http://www.arXiv.org/abs/math.QG/0006083">math.QG/0006083</A>.

and, just for fun, Deligne's handwritten letter to Bar-Natan:

8) Pierre Deligne, letter to Dror Bar-Natan about the Duflo map,
available at <A HREF = "http://www.math.toronto.edu/~drorbn/Deligne/">http://www.math.toronto.edu/~drorbn/Deligne/</A>

\par\noindent\rule{\textwidth}{0.4pt}
\textbf{Addendum:} Todd Trimble has kindly allowed me to append
some rough notes in which he outlines proofs of some results
above.

\begin{quote}

\begin{verbatim}

From: Todd Trimble 
Subject: notes on super Brauer 
To: John Baez 
Cc: James Dolan 
Date: Sun, 27 Mar 2005 19:30:00 -0500 

John, 

These are some notes on some of the super Brauer discussion.  

1.  Let V be the category of finite-dimensional super vector 
spaces over R.  By \textbf{super algebra} I mean a monoid in this 
category.  There's a bicategory whose objects are super 
algebras A, whose 1-cells  M: A \to  B  are left A- right B-
modules in V, and whose 2-cells are homomorphisms
between modules.  This is a symmetric monoidal bicategory 
under the usual tensor product on V.  

A and B are \textbf{super Morita equivalent} if they are equivalent 
objects in this bicategory.  Equivalence classes [A] form 
an abelian monoid whose multiplication is given by the 
monoidal product.  The \textbf{super Brauer group} of R is the 
subgroup of invertible elements of this monoid. 

If [B] is inverse to [A] in this monoid, then in particular 
A tensor (-)  can be considered left biadjoint to  B tensor (-). 
On the other hand, in the bicategory above we always have 
a biadjunction 

                    A tensor C \to  D 
                   -------------------
                    C \to  A* tensor D

(essentially because left A-modules are the same as right 
A*-modules, where A* denotes the super algebra opposite 
to A).  Since right biadjoints are unique up to equivalence, 
we see that if an inverse to [A] exists, it must be [A*]. 

This can be sharpened: an inverse to [A] exists iff the 
unit and counit 

         1 \to  A* tensor A       A tensor A* \to  1

are equivalences in the bicategory.  Actually, one is an 
equivalence iff the other is, because both of these 
canonical 1-cells are given by the same A-bimodule,  
namely the one given by A acting on both sides of 
the underlying superspace of A (call it S) by multiplication.  
Either is an equivalence if the bimodule structure map 

                 A* tensor A \to  Hom(S, S),    

which is a map of super algebras, is an isomorphism. 

2.  As an example, let  A = C_{1}  be the Clifford algebra 
generated by the 1-dimensional space R with the usual 
quadratic form Q(x) = |x|^{2},  and Z/2-graded in the usual 
way.  Thus, the homogeneous parts of A are 1-dimensional 
and there is an odd generator i satisfying i^{2} = -1.  The 
opposite A* is similar except that there is an odd generator 
e satisfying  e^{2} = 1.  Under the map 

           A* tensor A \to  Hom(S, S), 

where we write S as a sum of even and odd parts R + Ri, 
this map has a matrix representation 

                               -1   0
             e tensor i  |-->   0   1

                                0  -1 
             1 tensor i  |-->   1   0

                                0   1
             e tensor 1  |-->   1   0

which makes it clear that this map is surjective and thus 
an isomorphism.  Hence [C_{1}]  is invertible. 

One manifestation of Bott periodicity is that [C_{1}] 
has order 8.  We will soon see a very easy proof of this 
fact.  A theorem of C.T.C. Wall is that [C_{1}] in fact 
generates the super Brauer group; I believe this can 
be shown by classifying super division algebras, as 
discussed below. 

3. That [C_{1}] has order 8 is an easy calculation.  Let 
C_{r}  denote the r-fold tensor of C_{1}.  C_{2} for instance 
has two super-commuting odd elements  i, j  satisfying 
i^{2} = j^{2} = -1;  it follows that  k := ij  satisfies  
k^{2} = -1, and we get the usual quaternions, graded so that 
the even part is the span  &lang;1, k&rang;  and the odd part is &lang;i, j&rang;. 

C_{3} has three super-commuting odd elements i, j, l, all 
of which are square roots of  -1.  It follows that 
e = ijl is an odd central involution (here "central" 
is taken in the ungraded sense), and also that  
i' = jl,  j' = li,  k' = ij satisfy the Hamiltonian equations 

         (i')^{2} = (j')^{2} = (k')^{2} = i'j'k' = -1, 

so we have  C_{3} = H[e] mod (e^{2} - 1).   Note this is the 
same as 

                 H tensor (C_{1})*

where the H here is the quaternions viewed as a super 
algebra concentrated in degree 0 (i.e. is \textbf{purely bosonic}). 

Then we see immediately that  C_{4} = C_{3} tensor C_{1} is 
equivalent to purely bosonic H  (since the C_{1} cancels 
(C_{1})*  in the super Brauer group). 

At this point we are done: we know that conjugation on 
(purely bosonic) H gives an isomorphism 

                  H* \to  H 

hence  [H]^{-1} = [H*] = [H],  i.e. [H] = [C_{4}] has order 2!  
Hence  [C_{1}]  has order 8. 

4.  All this generalizes to Clifford algebras: if a real quadratic 
vector space  (V, Q)  has signature (r, s), then the super 
algebra Cliff(V, Q)  is isomorphic to  A^{r} tensor (A*)^{s}, 
where  A^{r}  denotes r-fold tensor product of  A = C_{1}. 
By the above calculation we see that  Cliff(V, Q)  is 
equivalent to  C_{r-s}  where r-s is taken modulo 8. 

For the record, then, here are the hours of the super 
Clifford clock, where e denotes an odd element, and ~ 
denotes equivalence: 

C_{0} ~ R
C_{1} ~ R + Re,  e^{2} = -1
C_{2} ~ C + Ce,  e^{2} = -1, ei = -ie
C_{3} ~ H + He,  e^{2} = 1,  ei = ie, ej = je, ek = ke
C_{4} ~ H
C_{5} ~ H + He,  e^{2} = -1, ei = ie, ej = je, ek = ke
C_{6} ~ C + Ce,  e^{2} = 1,  ei = -ie
C_{7} ~ R + Re,  e^{2} = 1

All the super algebras on the right are in fact <b>super 
division algebras</b>,  i.e. super algebras in which every 
nonzero homogeneous element is invertible. 

To prove Wall's result that [C_{1}] generates the super 
Brauer group, we need a lemma: any element in the super 
Brauer group is the class of a super division algebra. 

[To be continued.  I had wanted to show that every 
element in the super Brauer group must be of the form 
[A] where A is a super division algebra, and then classify 
super (associative) division algebras, showing on a case 
by case basis that those not in the super Clifford clock 
above are seen not to belong to the super Brauer group.] 

Todd
\end{verbatim}
    
\end{quote}

Todd finished off the job later... though by this point we
had formulated the grander goal of classifying 
not-necessarily-associative super division algebras!


\begin{quote}

\begin{verbatim}

From: Todd Trimble
Subject: super division algebras 
To: John Baez
Date: Wed, 27 Apr 2005 22:17:12 EDT 

John, 
 
This is a warm-up to classifying super division algebras 
over R, where I'll consider just the associative case.  
Nothing I say will be deep, but I found it somewhat fun 
and diverting, and there may be echoes of things to come. 
 
I'll take as known that the only associative division algebras 
over R are R, C, H  -- the even part A of an associative super 
division algebra must be one of these cases.  We can 
express the associativity of a super algebra (with even part 
A) by saying that the odd part M is an A-bimodule equipped 
with a A-bimodule map pairing 
 
     < -, - >:  M tensor_A M \to  A
 
such that:

(**)  a<b, c> = <a, b>c  for all a, b, c in M.  
 
If the super algebra is a super division algebra which is not 
wholly concentrated in even degree, then multiplication by a 
nonzero odd element induces an isomorphism 
 
                    A \to  M
 
and so M is 1-dimensional over A; choose a basis element e 
for M. 
 
The key observation is that for any a in A, there exists a 
unique a' in A such that 
 
                   ae = ea'  
 
and that the A-bimodule structure forces  (ab)' = a'b'.  Hence we 
have an automorphism (fixing the real field) 
 
              (--)': A \to  A
 
and we can easily enumerate (up to isomorphism) the possibilities 
for associative super division algebras over R: 
 
(1)  A = R.  Here we can adjust e so that either  e^{2} := <e, e>  is 
-1 or 1.  The corresponding super division algebras occur at 1 o'clock 
and 7 o'clock on the super Brauer clock. 
 
(2)  A = C.  There are two R-automorphisms  C \to  C.  In the case 
where the automorphism is conjugation, condition (**) for super 
associativity gives  <e, e>e = e<e, e>  so that  <e, e>  must be 
\emph{real}.  Again e can be adjusted so that  <e, e> = -1 or 1.  These 
possibilities occur at 2 o'clock and 6 o'clock on the super Brauer 
clock. 
 
For the identity automorphism, we can adjust e so that  <e, e>
is 1.  This gives the super algebra  C[e] mod e^{2} - 1  (where e 
commutes with elements in C).  This does not occur on the 
super Brauer clock over R.  However, it does generate the super 
Brauer group over C (which is of order two). 
 
(3)  A = H.  Here R-automorphisms  H \to  H  are given by 
h |--> xhx^{-1}  for x in H.  In other words 
 
                     he = exhx^{-1}
 
whence  ex  commutes with all elements of H (i.e. we can 
assume wlog that the automorphism is the identity).  The 
properties of the pairing guarantee that  h<e, e> = <e, e>h 
for all h in H, so  <e, e>  is real and again we can adjust 
e so that  <e, e> = 1 or -1.  These cases occur at 3 o'clock 
and 5 o'clock on the super Brauer clock. 
 
This appears to be a complete (even if a bit pedestrian) 
analysis. 
 
Best, 
 
Todd
\end{verbatim}
    
\end{quote}




\par\noindent\rule{\textwidth}{0.4pt}
% </A>
% </A>
% </A>
