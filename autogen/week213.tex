
% </A>
% </A>
% </A>
\week{ April 3, 2005 }


Here's a book I've been reading lately:

1) Kenneth S. Brown, Cohomology of Groups, Graduate Texts in Mathematics 
182, Springer, 1982.

I should have read this book a long time ago - but I probably wouldn't
have enjoyed it as much as I do now.  All sorts of things I struggled 
to learn for years are neatly laid out here.  Best of all, he comes
right out and admits from the start that the cohomology of groups is
really a branch of \emph{topology}, instead of hiding this fact like some
people do.  

This is something every mathematician should know: you can take any 
group and turn it into a space, thus "reducing" group theory 
to topology.
In particular, if you have any trick for telling spaces apart, like
"cohomology theory", you can apply it to groups as well.

Of course topology is \emph{harder} than group theory in many ways - hence
my quotes around "reducing".  Indeed, algebraic topology was invented
as a trick for reducing topology to group theory!  But, the bridge 
turns out to go both ways, and there's a lot of profitable traffic in 
both directions. 

Ultimately, as James Dolan likes to point out, it's all about the unity of 
mathematics.  Topology is about our concept of \emph{space}, 
while group theory 
is about our concept of \emph{symmetry}... 
but the amazing fact is that they turn 
out to be two aspects of the same big thing!  Mathematics is a source of 
endless surprises, but this is one of the biggest jaw-droppers of all.  

The idea goes back at least to Evariste Galois, who noticed that you can 
classify the ways a little thing can sit in a bigger thing by keeping track 
of what we now call its "Galois group": the group of all 
symmetries of the 
big thing that map the little thing to itself.  For example, you can pick 
out a point or line in the plane by keeping track of which symmetries of 
the plane map this point or line to itself.  

However, the idea of using groups to classify how a little thing sits
in a big one was really made explicit in Felix Klein's 
"Erlangen program", 
a plan for reducing \emph{geometry} to group theory.  

You may know Klein for his famous one-sided bottle:

\begin{quote}
            A mathematician named Klein <br>
            Thought the M&ouml;bius strip was divine. <br>
            Said he: "If you glue <br>
            The edges of two <br>
            You'll get a weird bottle like mine!" <br>

\end{quote}
Or maybe you know that the symmetry group of a rectangle, including 
reflections, is called the "Klein 4-group":


\begin{verbatim}

             1 a b c
            ---------
          1| 1 a b c
          a| a 1 c b
          b| b c 1 a
          c| c b a 1
\end{verbatim}
    
He is also known for some other groups called "Kleinian groups", which
act as symmetries of fractal patterns like this:

<A HREF ="http://www.josleys.com/showpic.php?file=INDRA065.jpg&title=Indra20">
<DIV ALIGN = CENTER>
<IMG WIDTH = 400 HEIGHT = 400 SRC = INDRA065.jpg>
</DIV>
% </A>

2) Jos Leys, Kleinian Pages, <A HREF = "http://www.josleys.com/creatures42.htm">http://www.josleys.com/creatures42.htm</A>

If you like cool pictures, check out this website!  I've linked you to 
the page that most closely connects to Kleinian groups, but there are 
lots of other more fanciful pictures.   And if you get interested in 
the math lurking behind these fractals, you've \emph{got} to try this book:

3) David Mumford, Caroline Series, and David Wright, Indra's Pearls: 
The Vision of Felix Klein, Cambridge U. Press, Cambridge, 2002.

Mumford is a world-class mathematician, so this book is completely
different from the superficial descriptions of fractals one often sees
in math popularizations - but it's still readable, and it's packed with 
beautiful pictures.  You can learn a lot about Kleinian groups from this!

The Kleinian groups arose from Klein's studies of complex functions,
which he considered his best work.  But, he was also a mathematical 
physicist.  Among other things, he wrote a four-volume book on tops 
with one of the fathers of quantum mechanics, Arnold Sommerfeld:

4) Felix Klein and Arnold Sommerfeld, &Uuml;ber die Theorie des Kreisels,
4 vols, 1897-1910.  Reprinted by Johnson, New York, 1965.  Also available
at <A HREF = "http://www.hti.umich.edu/cgi/t/text/text-idx?c=umhistmath;idno=ABV7354.0001.001">http://www.hti.umich.edu/cgi/t/text/text-idx?c=umhistmath;idno=ABV7354.0001.001</A>

This came after a book he wrote on his own:

5) Felix Klein, The Mathematical Theory of the Top, Scribner's, 
New York, 1887.

This may seem like a lot of books about a kid's toy!  But, tops 
are profoundly related to the rotation group, and the "exactly 
solvable" tops discovered by Euler, Lagrange/Poisson, and 
Sofia Kowalevskaya are solvable because of their symmetries - 
deeply hidden symmetries, in the case of the Kowalevskaya top.  
So, one can imagine why Klein liked this subject.

Klein also wrote a book on the icosahedron and the quintic equation:

6) Felix Klein, Lectures on the Icosahedron and the Solution of Equations
of the Fifth Degree, 1888.  Reprinted by Dover, New York, 2003.
Also available at <a href = "http://historical.library.cornell.edu/cgi-bin/cul.math/docviewer?did=03070001&seq=7">http://historical.library.cornell.edu/cgi-bin/cul.math/docviewer?did=03070001&seq=7</A>

Galois had already noticed that the number field you get by taking the 
rationals and throwing in the roots of a typical quintic:


$$

ax^{5} + bx^{4} + cx^{3} + dx^{2} + ex + f = 0
$$
    
has as its symmetry group all the permutations of the 5 roots.  Indeed,
he saw that the "unsolvability" of this group, in a technical 
sense, is
what makes it impossible to solve the quintic by radicals.  It must have 
been common knowledge that the symmetry group of the icosahedron is the 
group of all \emph{even} permutations of 5 things.  But, Klein took this much
further!  Alas, I've never really understood what he did.  Perhaps if
I read these and think hard, I'll understand:

7) Jerry Shurman, Geometry of the Quintic, John Wiley and Sons, New York, 
1997.

Peter Doyle and Curt McMullen, Solving the quintic by iteration,
Acta Math. 163 (1989), 151-180.  Available at 
<A HREF = "http://math.dartmouth.edu/~doyle/docs/icos/icos/icos.html">http://math.dartmouth.edu/~doyle/docs/icos/icos/icos.html</A>

Anyway, it should be clear by now that Klein was a lover of symmetry.
He was also a bit of a visionary, and his obituary by Grace Chisholm 
Young shows that this got him in some trouble:

\begin{quote}
     One of Weierstrass' pupils, still alive, told me that at Berlin
     Klein was anathema: it was said that his work was not mathematics
     at all, but mere talk.  This criticism shows a want of appreciation
     of his rare type of mind.  It teemed with ideas and brilliant 
     reflections, but it is true that his work lacks the stern aspects 
     required by mathematical exactitude. It was in personal contact that 
     this was corrected, at least in so far as his students were concerned. 
     His favourite maxim was, "Never be dull". 
\end{quote}

In a talk he wrote in 1872 when he was made professor at Erlangen 
University - a talk he didn't actually give! - Klein outlined what is now 
called his "Erlangen program".  The idea here is that different 
kinds of 
geometry correspond to different symmetry groups.  Taken to the extreme,
this philosophy says that a geometry is just a group!  In a given geometry, 
a "figure" of any kind - like a point or line - can be detected 
by the 
subgroup of symmetries that map that figure to itself.  So, a figure is 
just a subgroup!  

This program eventually led to a grand theory of groups and geometry based 
on "flag manifolds", which I tried to sketch in "<A HREF = "week178.html">week178</A>", "<A HREF = "week180.html">week180</A>", and 
"<A HREF = "week181.html">week181</A>".  

It's important to realize how similar the Erlangen program is to Galois 
theory.  Galois had also used group theory to classify how a little thing 
can sit in a bigger thing, but in situations where the "things" in question 
are commutative algebras - for example, the rational numbers with some roots 
of polynomials thrown in.

Now, commutative algebra is like topology, only backwards.  Any space has 
a commutative algebra consisting of functions on it, and if we're very 
clever we can think of any commutative algebra as functions on some space - 
though this was only achieved long after Galois, by Alexander Grothendieck. 

What do I mean by "backwards"?  Well, suppose you 
have a "covering
space" - a big space sitting over a little one, like a spiral sitting 
over the circle.  In this situation, any function on the little space 
downstairs defines a function on the big one upstairs.  So, the algebra 
of functions on the little space sits inside the algebra of functions on 
the big space.  

Notice how it's backwards.  Classifying how a little commutative
algebra can sit in a big one amounts to classifying how a big space 
can cover a little one!  For more details on this analogy, 
try "<A HREF = "week198.html">week198</A>", 
"<A HREF = "week201.html">week201</A>" and
especially "<A HREF = "week205.html">week205</A>". 
 
I should warn you: the Galois group has a different name when we apply 
it to the classification of covering spaces - we call it the group 
of "deck transformations".  The idea is pretty simple.  Suppose Y is a 
covering space of X, like this:


\begin{verbatim}

                            ----------------   
                            ----------------   Y 
                                               
                                               &darr; p
                                               
                            ----------------   X
\end{verbatim}
    
We've got a function p: Y \to  X, and sitting over each point of X are 
the same number of points of Y, living on different "sheets" that look
locally just like X.   You should imagine the sheets being able to 
twist around from place to place, like the edges of a Moebius strip.

Anyway, a "deck transformation" is just a way of mapping Y to itself 
that permutes the different points sitting over each point of X.  

The theory of this was worked out by Riemann, Poincare, and others.  
Poincare showed you could use this idea to turn any connected space X 
into a group - its "fundamental group".  There are different ways to
define this, but one is to form the most complicated possible 
covering space of X that's still connected - its "universal cover".
Then, take the group of deck transformations of this!  Following 
Galois' philosophy, all the other connected covering spaces of X 
correspond to subgroups of this group.  

The theory of the fundamental group was just the beginning when 
it came to groups and topology.  One of many later big steps, back 
in the late 1940s, was due to Sammy Eilenberg and Saunders Mac Lane.  
They saw how to reverse the "fundamental group" idea and turn any 
group back into a space!   

More precisely: for any group G, there's a space whose fundamental group
is G and whose higher homotopy groups vanish.  It's sometimes called the 
"Eilenberg - Mac Lane space" and denoted K(G,1), but sometimes it's called
the "classifying space" and denoted BG.  It's pretty easy to build; 
I described how back in "<A HREF = "week70.html">week70</A>".  

You start with a point:


\begin{verbatim}

                             o
\end{verbatim}
    
Then you stick on an edge looping from this point to itself for
each element a in G.   Unrolled, it looks like this: 


$$

                         o--a->--o
$$
    
where a is an element of our group.  Then, whenever we have ab = c in 
our group, we stick on a triangle like this:


\begin{verbatim}

                             o
                            / \
                           a   b
                          /     \
                         o---c---o
\end{verbatim}
    
Then, whenever we have abc = d in our group, we stick on a tetrahedron 
like this:


\begin{verbatim}

                             o                      
                            /|\                    
                           / | \                  
                          /  b  \                
                         a   |   bc             
                        /   _o_   \  
                       /   /   \_  \          
                      / _ab      c_ \        
                     /_/           \_\      
                    o--------d--------o
\end{verbatim}
    

And so on, forever!  For each list of n group elements, we get an
n-dimensional simplex in our Eilenberg-Mac Lane space.  The resulting
space knows everything about the group we started with.  In particular, 
the fundamental group of this space will be the group we started with!

Using this idea, we can do some fiendish things.  For example, for each n 
we can form a set C_{n}(G,A) consisting of all functions that eat 
n-dimensional simplices in the Eilenberg-Mac Lane space of G and spit 
out elements of some abelian group A.  There are maps

d: C_{n}(G,A) \to  C_{n+1}(G,A)
</P>
reflecting the fact that each (n+1)-simplex has a bunch of n-simplices
as its faces.  Since the boundary of a boundary is zero, 

d^{2} = 0
Guys who live in the kernel of 

d: C_{n}(G,A) \to  C_{n+1}(G,A)

are called "n-cocycles", and guys who live in the image of

d: C_{n-1}(G,A) \to  C_{n}(G,A)

are called "n-coboundaries".   Since d^{2} 
= 0, every coboundary is
a cocycle, but not always vice versa.  So, we can form the group of
cocycles mod coboundaries.  This is called the "nth cohomology
group" of G with coefficients in A, and it's denoted

H^{n}(G,A)
This sounds unmotivated at first, but the nth cohomology group of a
space is really just a clever way of keeping track of n-dimensional 
holes in that space.  So, what we're doing here is cleverly defining
a way to study "holes" in a \emph{group!}  
There are deeper, more conceptual
ways of understanding group cohomology, but this is not bad for starters.

For example, let's take the simplest group that's not \emph{utterly} dull -
the integers mod 2, or Z/2.  Here we get 

K(Z/2,1) = RP^{\infty }
where RP^{\infty } 
is the space formed by taking an infinite-dimensional
sphere and identifying opposite points.  This space has holes of 
arbitrarily high dimension, so the cohomology groups of Z/2 go on being 
nontrivial for arbitrarily high n.  I sketched a "picture proof" here:

8) John Baez, Fall 2004 Quantum Gravity Seminar, week 10, notes
by Derek Wise, <A HREF = "http://math.ucr.edu/home/baez/qg-fall2004/">http://math.ucr.edu/home/baez/qg-fall2004/</A>

and I showed that, for example


$$

            Z   if n = 0
H^{n}(Z/2,Z) = 0   if n is odd
            Z/2 if n is even and > 0
$$
    
I also explained how this stuff is related to topological quantum field
theory.

Anyway, all this is just the very superficial beginnings of the subject
of group cohomology.  Read Brown's book to dig deeper!

Personally, what I find most exciting about this book now are the
remarks on the "Euler characteristic" of a group.  Let me 
explain this...
though now I'll have to pull out the stops and assume you know some
group cohomology.

We can try to define the "Euler characteristic" of a group G to be the 
Euler characteristic of K(G,1).  This is the alternating sum of the 
dimensions of the rational cohomology groups


$$

H^{n}(G,Q)
$$
    

Of course, this alternating sum only converges if the cohomology
groups vanish for big enough n.  Also, they all need to be 
finite-dimensional.

Unfortunately, not many groups have well-defined Euler characteristic
with this naive definition!  

For example, people have studied groups G whose nth cohomology vanishes
for n > d, regardless of the coefficients.  If we take the smallest
d for which this holds, such a group G is said to have "cohomological 
dimension" d.  Eilenberg and Ganea showed that for d \ge  3, a group has 
cohomological dimension d whenever we can build K(G,1) as a simplicial 
complex (or CW complex) with no cells of dimension more than d.   

This is a nice geometrical interpretation of the cohomological dimension.  
But, one can show that groups with torsion never have finite cohomological
dimension!  We've seen an example already: Z/2, whose Eilenberg-Mac Lane
space is infinite-dimensional.

However, it turns out that there's a generalization of the Euler 
characteristic that makes sense for any group G that has a torsion-free 
subgroup H whose Euler characteristic is well-defined in the naive way, 
as long as H has finite index in G.  We just define the Euler characteristic
of G to be the Euler characteristic of H divided by the index of H in G.
The answer doesn't depend on the choice of H!

Take my favorite example, SL(2,Z).  This has torsion, so its cohomological
dimension is infinite and its naive Euler characteristic is undefined!  
Indeed, I wrote a whole issue of This Week's Finds about some elements of 
orders 4 and 6 sitting inside SL(2,Z), related to the symmetries of square 
and hexagonal lattices - see "<A HREF = "week125.html">week125</A>".   

But, SL(2,Z) has a torsion-free subgroup of index 12, namely its
commutator subgroup - the group you need to quotient by to make SL(2,Z)
be abelian.  This subgroup has finite cohomological dimension and its
Euler characteristic is -1.  I'm not sure why this is true, but Brown says
so!  This means the Euler characteristic of SL(2,Z) works out to be -1/12.  

If you've read my stuff about Euler characteristics in "<A HREF = "week147.html">week147</A>", you'll
see why this gets me so excited - I can add this stuff to my list of
weird ways of calculating the Euler characteristic.  Plus, it's related 
to the magical role of the number "24" in string theory, and also the 
Riemann zeta function!

Indeed, the Riemann zeta function gives a way to make rigorous Euler's
zany observation that


\begin{verbatim}

1 + 2 + 3 + .... = -1/12,
\end{verbatim}
    
as I explained here:

9) John Baez, Euler's Proof that 1+2+3+ ... = -1/12, Bernoulli
Numbers and the Riemann Zeta Function, Winter 2004 Quantum Gravity 
Seminar, homework for weeks 5,6,7, available at 
<A HREF = "http://math.ucr.edu/home/baez/qg-winter2004/">http://math.ucr.edu/home/baez/qg-winter2004/</A>

This suggests that there should be a version of the Eilenberg-Mac Lane
space for SL(2,Z) which has 1 cell of dimension 0, 2 cells of dimension
2, 3 cells of dimension 4, and so on.  Does anyone know if this is true?

More generally, G. Harder computed the (generalized) Euler characteristic
for a large class of arithmetic groups:

10) G. Harder, A Gauss-Bonnet formula for discrete arithmetically defined
groups, Ann. Sci. Ecole Norm. Sup. 4 (1971), 409-455.

For example, he looked at the symplectic group defined over the
integers, Sp(n,Z), and showed that its Euler characteristic is
this product of values of the Riemann zeta function:

&zeta;(-1) &zeta;(-3) &sdot;&sdot;&sdot; &zeta;(1-2n)

In the case n = 1 we get back SL(2,Z) and &zeta;(-1) = -1/12.

In fact, every Chevalley group over the integers has a well-defined
Euler characteristic, and Harder was able to compute it
in terms of Bernoulli numbers.  A Chevalley group is
sort of like a simple Lie group, but defined algebraically.  
Sp(n,Z) is one example.   SL(n,Z) is another, but it's Euler characteristic
turns out to vanish for n > 2, so it's not too interesting.  

Harder worked them all out.  For example, he showed
the Euler characteristic of the integral form of the exceptional group 
E_{7}
is some wacky number like

      - 691 \times  43867 &nbsp; / &nbsp; 2^{21} \times  3^{9} \times  5^{2} \times  7^{3} \times  11 \times  13 \times  19
Serre went even further, computing Euler characteristics of Chevalley
groups defined over algebraic number fields.  He also noticed that when
you write the Euler characteristic of a group as a fraction, the primes
in the denominator are precisely the primes p for which the group has
p-torsion.  He was thus able to conclude, for example, that 
E_{7} defined
over the integers has p-torsion for p = 2, 3, 5, 7, 11, 13 and 19.

For more, see:

11) Jean-Pierre Serre, Cohomologie des groups discretes, Ann. Math.
Studies 70 (1971), 77-169.

This only takes us up to 1971.  I shudder to think what bizarre 
results along these lines are known by now!  Probably they'd seem not
bizarre but beautiful if I understood this stuff better: I don't really
have a clue how the Riemann zeta function gets into this game, so 
everything after that seems like black magic to me - bewitching but
bewildering.

But, it's clear that the study of groups and symmetry has not lost its
ability to turn up surprises.

\par\noindent\rule{\textwidth}{0.4pt}
Addenda: I had often wondered how Klein's name got attached to the
pathetic little "4-group" mentioned above, which is just Z/2 x Z/2.
John McKay proffered an explanation:

\begin{quote}
 There is a group called the Klein group.
 It is denoted V_{4} = The Vierer-Gruppe  (The fours group).

 Klein worked with the simple group of order 168 and found
 the "Klein quadric" which has it for symmetry group.

 The suggestion is that friends decided to call the non-cyclic 
 abelian group of order 4 the "Klein group" = the "little group"
 as a joke.

 I have a question you may like to posit to your readers:

 Is V_{4} the abstract group or a permutation group?
 
 There are other points ... I presume you know that your
 -1/12 is &zeta;(-1). There is a paper by Lepowsky on the occurrence
 of such &zeta;(-n) involving vertex algebras.

 I dearly wish I understood cohomology!
 I am busy tethering moonshine!
 Best,
 John
\end{quote}

This group of order 168 has made an appearance here before, in "<A HREF
= "week194.html">week194</A>": it's PSL(3,Z/2) - the group of
symmetries of the projective plane over Z/2, or "Fano plane", whose
points can also be thought of as imaginary unit octonions.  It's also
PSL(2,Z/7).  I've long been mystified by its relation to Klein's
quartic, mainly because I've never spent time trying to understand it!
- it's just one of those things that's been gnawing at the edges of my
consciousness, especially ever since I saw this book come out:

12) Silvio Levy, The Eightfold Way: the Beauty of Klein's Quartic Curve,
MSRI Research Publications 35, Cambridge U. Press, Cambridge 1999.
Available at <A HREF = "http://www.msri.org/publications/books/Book35/">http://www.msri.org/publications/books/Book35/</A>

It has a translation of Klein's original paper on this subject.
Someday I'll break down and study this.

Anyway....

James Dolan mentioned some other folklore saying that the "Kleinian 
groups" were \emph{also} named after Klein as a joke:
\begin{quote}
 by the way, i enjoyed the latest twf a lot (although i don't know why
 we seem to never get a chance to talk about all this stuff ourselves
 that much), but i noticed that you (apparently non-ironically)
 mentioned kleinian groups as a famous thing named after klein, without
 telling the story that i always hear about how poincare gave kleinian
 groups the name "kleinian groups" after klein complained to poincare
 about poincare's use of the terminology "fuchsian groups" for
 something that fuchs apparently didn't event.
 
 i guess that the versions of the story that i'd heard seemed to
 suggest that klein was complaining because he thought that fuchs
 hadn't significantly contributed to the study of fuchsian groups, and
 that poincare may have been naively trying to placate klein and/or
 not-so-naively twitting him by then giving the name "kleinian groups"
 to something that klein hadn't significantly contributed to the study
 of.

 however i did just look for the story on the web, and the tellings
 that i found there i guess don't really suggest that klein didn't
 "significantly contribute to the study of" kleinian groups (or at
 least not by my standards).  it's still not clear though what sort of
 reaction poincare may have been trying to provoke in klein, and
 whether he succeeded in provoking it.  it's claimed that poincare did
 come up with the name "kleinian function" later in the same day after
 klein complained about the name "fuchsian function", and also that
 klein was subsequently just as vociferous in complaining to poincare
 about the name "kleinian function" as he was in complaining about the
 name "fuchsian function".  but apparently klein's complaints were
 based on _very_ exacting concerns about absolute priority, so that the
 names "fuchsian function" and "kleinian function" 
 can be seen as
 inappropriate only by the standards of someone with similarly
 ridiculous concerns about absolute priority, rather than by a
 reasonable person such as myself.

 i'd also heard that klein's nervous breakdown was provoked by the
 stress of trying to keep up with a genius like poincare, but maybe it
 was actually provoked by poincare's apparently casual attitude towards
 priority disputes and/or concept-naming.

 i'd thought of asking you about this issue of whether klein really did
 have much to do with kleinian groups right after i read the advance
 copy of twf that you sent me, but i guess that i didn't notice that it
 was an advance copy.  i guess that it doesn't matter though, since
 apparently there _is_ a case to be made that klein had lots to do with
 developing the theory of kleinian groups; just not by his own
 apparently ridiculous standards.
\end{quote}

Noam Elkies suggested that the commutator subgroup of SL(2,Z)
has Euler characteristic -1 because it's a a free group on 2
generators, so its classifying space is a figure 8, with Euler
characteristic 1 - 2 = -1 since it has one vertex and two edges.

This sounds right.  In particular, I already mentioned how Brown
claims the commutator subgroup of SL(2,Z) is torsion-free.  Further,
Kevin Buzzard shows below that any torsion-free subgroup of SL(2,Z)
is a free group.  So, we just need to check that the commutator
subgroup of SL(2,Z) can be generated by two elements but not by just one.

Laurent Bartholdi just made this job easier;
he sent me an email saying these are free generators
for the commutator subgroup of SL(2,Z):


\begin{verbatim}

( 2 -1)
(-1  1)
\end{verbatim}
    

and


\begin{verbatim}

( 1 -1)
(-1  2)
\end{verbatim}
    

In fact, Kevin Buzzard's email was packed with wisdom.  He wrote:

\begin{quote}
 I know one elementary argument which you don't appear to, so I thought
 I'd fill you in. The argument below is waffly but rather easy really.

 John Baez wrote:


\begin{verbatim}

 > But, SL(2,Z) has a torsion-free subgroup of index 12, namely its
 > commutator subgroup - the group you need to quotient by to make SL(2,Z)
 > be abelian.  This subgroup has finite cohomological dimension and its
 > Euler characteristic is -1.  I'm not sure why this is true, but Brown says
 > so!  This means the Euler characteristic of SL(2,Z) works out to be -1/12.
\end{verbatim}
    

 One doesn't have to use such a "strange" subgroup as the commutator
 subgroup of SL(2,Z). People who do modular forms, like me, far
 prefer "congruence subgroups", as these are the ones that show
 up when you study automorphic forms for SL(2,Z). So here's an easy
 way to compute the Euler characteristic of SL(2,Z): take your
 favourite congruence subgroup which has no torsion, work out
 its Euler characteristic (this is easy, I'll show you how to do it
 in a second) and then deduce what the Euler characteristic of SL(2,Z) is.

 Here are some examples of congruence subgroups: for any integer N\ge 1,
 consider the subgroup \Gamma _{1}(N) of SL(2,Z), defined as the 
matrices

\begin{verbatim}

 (a b)
 (c d)
\end{verbatim}
    
in SL(2,Z) such that c=0 mod N and a=d=1 mod N. It's just
 the pre-image in SL(2,Z) of the upper triangular unipotent matrices
 in SL(2,Z/NZ) so it's a subgroup of SL(2,Z). Here's a neat fact
 that makes life easy:

\textbf{Lemma:} if N\ge 5 then \Gamma _{1}(N) has no torsion.

 \textbf{Proof:} say g in SL(2,Z) has finite order d\ge 1. Then its min poly
 divides X^{d}-1 so over the complexes it has distinct linear factors
 so it's diagonalisable with roots of unity z and w on the diagonal.
 Now |z|=|w|=1 so |trace(g)| \le  2. But it's an integer, so it's -2,-1,0,1,2.
 And for N>=5 the only one of these congruent to 2 mod N is 2. So z=w=1
 and so g is the identity.

 Deeper, but also completely standard (and not logically necessary
 for what follows)---any torsion-free subgroup
 of SL(2,Z) is free! This is because SL(2,Z) acts very naturally
 on a certain tree in the upper half plane. This is a neat piece
 of mathematics. SL(2,Z) acts on the upper half plane {z=x+iy:y>0}
 via the rule:

\begin{verbatim}

(a b)
(c d)
\end{verbatim}
    
 sends z to (az+b)/(cz+d). Now draw dots
 at the points i=\sqrt -1 and \rho =exp(2\pi i/6), the primitive 6th
 root of unity in the upper half plane, and draw the obvious arc
 between them (the one that lies on the circle |z|=1), this is
 our first edge, and now look at the image of what you have
 under the SL(2,Z) action. It's a rather pretty tree, with two kinds
 of vertices---those in the i orbit have valency 2 and stabiliser
 of order 4, and those in the \rho  orbit have valency 3 and stabiliser
 of order 6. Now a group is free iff it acts freely on a tree, 
 and anything torsion-free in SL(2,Z) must be acting freely
 on this tree because the stabiliser of each vertex and edge under
 the SL(2,Z) action is finite.

 So \Gamma _{1}(5) is, by this general theorem, free. In fact I don't
 really need this general nonsense, one can give a hands-on proof
 of this fact, which I'll do now. We've seen that SL(2,Z), and hence
 \Gamma _{1}(5), acts on the upper half plane. There is no torsion
 in \Gamma _{1}(5) so the action is very nice, one checks easily that
 the action is free in fact by a similar sort of argument to the lemma
 above, it's the sort of thing you can find in the first few pages
 of any book on modular forms. So we can quotient out the upper half
 plane by \Gamma _{1}(5) and get a quotient Riemann surface. The point
 is that this computation is very manageable and can be done "in practice".
 There is a standard argument which shows how to quotient out the
 upper half plane by SL(2,Z)---the answer is a Riemann surface
 isomorphic to the complex plane (although you have to take care
 at the points where the action isn't free---this is exactly the vertices
 of the tree above), and the isomorphism can even
 be given "explicitly" via the j-function coming from the theory
 of elliptic curves---there is a standard fundamental domain even,
 the one with corners \rho , \rho ^{2} and +i\infty . 
I'm sure you'll have
 come across this sort of thing many times before. Now SL(2,Z)
 surjects onto SL(2,Z/5Z) so the index of \Gamma _{1}(5) in SL(2,Z)
 is just the index of 

\begin{verbatim}

(1 *)
(0 1)
\end{verbatim}
    
 in SL(2,Z/5Z) and by counting
 orders this comes out to be 24. Now it's not hard to find explicitly
 24 translates of the standard fundamental domain and then glue
 them together to work out the quotient of the upper half plane
 by \Gamma _{1}(5)---it turns out that it is isomorphic to the
 Riemann Sphere minus 4 points.

 In fact there is no need to do this sort of computation---the modular
 forms people have automated it long ago. The quotient of the upper
 half plane by \Gamma _{1}(N) is a Riemann surface called
 Y_{1}(N) and I can just ask my computer to compute the genus
 of its natural compactification (this exists and is called
 X_{1}(N)) and also to compute how many cusps were added to
 compactify it.  So in practice you just have to find a friendly
 modular forms person and then say "hey, what's the genus of
 X_{1}(5) and how many cusps does it have?" and then you have
 a complete description of \Gamma _{1}(5) because it's
 \pi _{1} of the answer.
 
 OK, the upper half plane modulo \Gamma _{1}(5) is the sphere
 minus 4 points, so \Gamma _{1}(5) is \pi _{1} of this,
 i.e. it's free on three generators.  That makes the Euler
 characteristic of \Gamma _{1}(5) equal to 1-3=-2.  And we
 already checked that the index was 24, so the Euler Characteristic of
 SL(2,Z) works out to be -1/12.
 
 Grothendieck wouldn't have chosen \Gamma _{1}(5); he would have chosen
 something called \Gamma (2), the subgroup of SL(2,Z) consisting
 of the matrices which are the identity mod 2. There is another
 classical modular function \lambda  inducing an isomorphism
 of Y(2), the quotient of the upper half plane by \Gamma (2),
 with the sphere minus three points---this is what gives
 the one-line proof of the fact that any analytic function C\to C
 that misses two points must be constant, because it then lifts
 to a function from C to the upper half plane which is the same
 as the unit disc, so we're done by Liouville. There is a subtlety
 here though: 

\begin{verbatim}

(-1 0)
(0 -1)
\end{verbatim}
    
 is in \Gamma (2). So you have to work
 with PSL(2,Z)=SL(2,Z)/{&plusmn;1} instead. Let P\Gamma (2) denote the image
 of \Gamma (2) in PSL(2,Z). Note that -1 is kind of a pain in the
 theory of modular forms sometimes because it acts trivially on
 everything but isn't the identity. Grothendieck was very interested
 in the sphere minus three points but it's much older than this
 that P\Gamma (2) is its fundamental group, so P\Gamma (2) has
 Euler characteristic 2-3=-1 and index 6 in PSL(2,Z), so PSL(2,Z)
 has Euler characteristic -1/6, so SL(2,Z) has Euler characteristic
 -1/12 because that's how they work. :-)

 John Baez wrote:

\begin{verbatim}

 > This only takes us up to 1971.  I shudder to think what bizarre
 > results along these lines are known by now!  Probably they'd seem not
 > bizarre but beautiful if I understood this stuff better: I don't really
 > have a clue how the Riemann zeta function gets into this game, so
 > everything after that seems like black magic to me - bewitching but
 > bewildering.
\end{verbatim}
    
 Nowadays almost any analytic function that is involved in number
 theory, when evaluated at certain "natural" points, gives
 an answer which has a conjectural interpretation in terms
 of relations between cohomology theories---this is the
 subject of many conjectures (Deligne, Beilinson, Bloch-Kato,...).
 It is still absolutely black magic! Actually I'm being unfair,
 the relation between special values of &zeta; and Euler characteristics
 is somehow less profound than this stuff. I wish I knew more about it!
 It can actually be used to compute certain values of L-functions
 (things more general than the zeta function but along the same lines)...
 Kevin
\end{quote}

I replied:

\begin{quote}
 Hi -

 Thanks VERY much for this email.  I was actually wondering why
 Brown used the commutator subgroup of SL(2,Z) as a kind of "warmup"
 for computing the Euler characteristic of SL(2,Z) instead of one of
 the congruence subgroups.  It seems this subgroup is not any of the
 beloved congruence subgroups....

 In fact, I've finally managed to turn up the thing I was looking
 for.  How does this relate to the stuff you're saying?  It involves
 \Gamma (3) rather than the \Gamma _{1}(N) groups:

 In "<A HREF = "week97.html">week 97</A>", I wrote:


$$

  Where does the extra 24 come from?  I don't know, but Stephan Stolz
  said it has something to do with the fact that while PSL(2,Z) doesn't
  act freely on the upper half-plane - hence these elliptic curves with
  extra symmetries - the subgroup "\Gamma (3)" does.  This subgroup consists
  of integer matrices

  (a b)
  (c d)

  with determinant 1 such that each entry is congruent to the corresponding
  entry of

  (1 0)
  (0 1)

  modulo 3.

  So, if we form

  H/\Gamma (3)

  we get a nice space without any "points of greater symmetry".
  To get the moduli space of elliptic curves from this, we just
  need to mod out by the group

  SL(2,Z)/\Gamma (3) = SL(2,Z/3)

  But this group has 24 elements!

  In fact, I think this is just another way of explaining the
  period-24 pattern in the theory of modular forms, but I like
  it.
$$
    


 Kevin wrote:

\begin{verbatim}

 >It's a rather pretty tree,
\end{verbatim}
    
 Yes, there's a picture of it in Brown's book, drawn on top of
 an old picture by Klein of a triangulation of the hyperbolic
 plane.

 What Brown seems to be doing there is showing that this tree
 is a deformation retract of that triangulation (with its simplicial
 topology, where the points on the boundary of the hyperbolic plane
 form a discrete set), and thus proving that the cohomological dimension
 of SL(2,Z) is just 1.

 Anyway, this is all great stuff.  Do you mind if I attach a copy of
 your email to the copy of "week213" on my website?  I think people
 will find it helpful, especially because of its friendly
 straight-to-the-point style, which books rarely seem to manage....
   
 Best,
 jb
\end{quote}

Kevin replied:

\begin{quote}

 John Baez wrote:

\begin{verbatim}

 > I was actually wondering why
 > Brown used the commutator subgroup of SL(2,Z) as a kind of "warmup"
 > for computing the Euler characteristic of SL(2,Z) instead of one of
 > the congruence subgroups.  It seems this subgroup is not any of the
 > beloved congruence subgroups....
\end{verbatim}
    
 You're right, I don't think it is. For N\ge 1 define \Gamma (N) to
 be the kernel of the obvious map SL(2,Z)\to SL(2,Z/NZ); a congruence
 subgroup is any subgroup of SL(2,Z) that contains one of these
 \Gamma (N)'s. Clearly such things have finite index in SL(2,Z). But
 unfortunately there exist subgroups of finite index in SL(2,Z) that
 are not congruence subgroups. This is a "low-dimensional"
 phenomenon---the moment you have a bit more freedom, e.g. you're
 working with SL(3,Z) or indeed SL(n,Z) for any n\ge 3, or even
 SL(2,Z[1/p]) for some prime p, then any subgroup of finite index
 is a congruence subgroup---these groups satisfy the "congruence
 subgroup property". But I've never understood the commutator
 of SL(2,Z) precisely for the reason that it's not a congruence
 subgroup (this is essentially because the commutator subgroup
 of SL(2,Z/NZ) never has index 12 in SL(2,Z/NZ)! The index is always
 smaller than 12 because SL(2,Z/pZ) is essentially a simple group.)

John Baez wrote:

$$

 > In fact, I've finally managed to turn up the thing I was looking
 > for.  How does this relate to the stuff you're saying?  It involves
 > \Gamma (3) rather than the \Gamma _{1}(N) groups.
$$
    
 Anything will do. If you know about \Gamma (3) then great. The same
 key observation is true---\Gamma (3) contains no elements of finite
 order, because any finite order element 

\begin{verbatim}

(a b)
(c d)
\end{verbatim}
    
 of \Gamma (3) which isn't
 the identity must have trace in {-2,-1,0,1} congruent to 2 mod 3,
 so the trace must be -1, so d=-1-a, so the det is a(-1-a) mod 9,
 which is never 1 mod 9. Now the index of \Gamma (3) in SL(2,Z) is 24,
 and the modular curve X(3) has genus 0 (everyone knows this because Wiles
 needed it to prove Fermat's Last Theorem!) and four cusps (zero, 1, 1/2
 and \infty ) and hence the Euler Characteristic of \Gamma (3) is 2-4=-2, so
 we recover the result that the Euler Characteristic of SL(2,Z) is -1/12 again.
 John Baez wrote:

$$

  > Where does the extra 24 come from?  I don't know, but Stephan Stolz
  > said it has something to do with the fact that while PSL(2,Z) doesn't
  > act freely on the upper half-plane - hence these elliptic curves with
  > extra symmetries - the subgroup "\Gamma (3)" does.
$$
    
 One can see that any subgroup of SL(2,Z) which has finite index and
 is free, must have index a multiple of 12 (and hence at least 12). Because
 if it has index d and is free on g generators, when we know (1-g)/d=-1/12,
 so 12 divides the denominator of (1-g)/d in lowest terms. Geometrically
 what is going on is that perhaps the "correct" quotient of the upper
 half plane by SL(2,Z) is not just the complex numbers, it's something that
 looks a bit like the complex numbers except there is a little bit of
 extra magic going on at i and \rho , corresponding to the fact that one
 shouldn't really have attempted to quotient out there, one should
 just remember that really the quotient is kind of "crumpled up"
 near there. So for example the fundamental group of the quotient
 shouldn't be the trivial group---if you take a small loop around i then
 this should not be regarded as contractible---you have to go around i
 twice before you can hope to contract the loop. Similarly you have to
 go around \rho  three times. Even worse---if you do this carefully enough
 then even going around i twice or \rho  three times isn't enough to
 contract the loop---because the resulting loop somehow corresponds
 to the element -1 in SL(2,Z), which acts trivially but which isn't
 the identity! So you have to do everything again before you
 get to the element 1. Mumford thought hard about how to make all this
 sort of thing rigorous, and managed in the late 60s to prove that the
 Picard group of the quotient of the upper half plane by SL(2,Z) was in
 fact Z/12Z.
 John Baez wrote:

\begin{verbatim}

  > Anyway, this is all great stuff.  Do you mind if I attach a copy of
  > your email to the copy of "week213" on my website?
\end{verbatim}
    
 Go ahead!
 Kevin
\end{quote}


\par\noindent\rule{\textwidth}{0.4pt}
<em>
Regarding the fundamental investigations of mathematics, there is 
no final ending ... no first beginning.</em> - Felix Klein

<em>In point of fact, it has traditionally been the 
"continuous" aspect of 
things which has been the central focus of Geometry, while those properties 
associated with "discreteness", notably computational and 
combinatorial 
properties, have been passed over in silence or treated as an afterthought. 
It was therefore all the more astonishing to me when I made the discovery, 
about a dozen years ago, of the combinatorial theory of the icosahedron, 
even though this theory is barely scratched (and probably not even understood) 
in the classic treatise of Felix Klein on the icosahedron.  I see in this 
another significant indicator of this indifference (of over 2000 years) 
of geometers vis-a-vis those discrete structures which present themselves 
naturally in Geometry: observe that the concept of the group (notably of 
symmetries) appeared only in the last century (introduced by Evariste 
Galois), in a context that was considered to have nothing to do with 
Geometry.  Even in our own time it is true that there are lots of 
algebraists who still haven't understood that Galois theory is primarily, 
in essence, a geometrical vision, which was able to renew our understanding 
of so-called "arithmetical" phenomena.</em> - Alexander Grothendieck

\par\noindent\rule{\textwidth}{0.4pt}

% </A>
% </A>
% </A>
