
% </A>
% </A>
% </A>
\week{October 27, 1996 }




Lately I've been trying to learn more about string theory.  I've always
had grave doubts about string theory, but it seems worth knowing about.
As usual, when I'm trying to learn something I find it helpful to write
about it - it helps me remember stuff, and it points out gaps in my
understanding.  So I'll start trying to explain some string theory in
this and forthcoming Week's Finds.

However: watch out!  This isn't going to be a systematic introduction to
the subject.  First of all, I don't know enough to do that.  Secondly,
it will be very quirky and idiosyncratic, because the aspects of string
theory I'm interested in now aren't necessarily the ones most string
theorists would consider central.  I've been taking as my theme of
departure, "What's so great about 10 and 26 dimensions?"  When one
reads about string theory, one often hears that it only works in 10
or 26 dimensions - and the obvious question is, why?

This question leads one down strange roads, and one runs into lots of
surprising coincidences, and spooky things that sound like coindences
but might NOT be coincidences if we understood them better.   

For example, when we have a string in 26 dimensions we can think of it
as wiggling around in the 24 directions perpendicular to the
2-dimensional surface the string traces out in spacetime (the "string
worldsheet").  So the number 24 plays an especially important role in
26-dimensional string theory.  It turns out that


$$

  1^{2} + 2^{2} + 3^{2} + ... + 24^{2} = 70^{2}.  
$$
    
In fact, 24 is the \emph{only} integer n > 1 such that the sum of squares 
from 1^{2} to n^{2} is itself a perfect square.  Is this a 
coincidence?
Probably not, as I'll eventually explain!  This is just one of
many eerie facts one meets when trying to understand this stuff.


For starters I just want to explain why dimensions of the form 8k + 2
are special.  Notice that if we take k = 0 here we get 2, the dimension
of the string worldsheet.  For k = 1 we get 10, the dimension of
spacetime in "supersymmetric string theory".  For k = 3 we get
26, the dimension of spacetime in "purely bosonic string
theory".  So these dimensions are important.  What about k = 2 and
the dimension 18, I hear you ask?  Well, I don't know what happens there
yet... maybe someone can tell me!  All I want to do now is to explain
what's good about 8k + 2.

But I need to start by saying a bit about fermions.  

Remember that in the Standard Model of particle physics - the model
that all fancier theories are trying to outdo - elementary particles
come in 3 basic kinds.  There are the basic fermions.  In general a
"fermion" is a particle whose angular momentum comes in units of
Planck's constant \hbar  times 1/2, 3/2, 5/2, and so on.  Fermions satisfy
the Pauli exclusion principle - you can't put two identical fermions
in the same state.  That's why we have chemistry: the electrons stack up
in "shells" at different energy levels, instead of all going to the
lowest-energy state, because they are fermions and satisfy the exclusion
principle.  In the Standard Model the fermions go like this:


\begin{verbatim}

        LEPTONS                                     QUARKS

electron        electron neutrino         down quark        up quark
muon            muon neutrino             strange quark     charm quark
tauon           tauon neutrino            bottom quark      top quark

\end{verbatim}
    
There are three "generations" here, all rather similar to each other.  

There are also particles in the Standard Model called "bosons" having
angular momentum in units of \hbar  times 0,1,2, and so on.  Identical
bosons, far from satisfying the exclusion principle, sort of like to all
get into the same state: one sees this in phenomena such as lasers,
where lots of photons occupy the same few states.  Most of the bosons
in the Standard Model are called "gauge bosons".  These carry the 
different forces in the standard model, by which the particles interact:


\begin{verbatim}

    ELECTROMAGNETIC FORCE          WEAK FORCE         STRONG FORCE
        
    photon                          W+                  8 gluons 
                                    W-
                                    Z  

\end{verbatim}
    
Finally, there is also a bizarre particle in the Standard Model called the 
"Higgs boson".  This was first introduced as a rather ad hoc hypothesis:
it's supposed to interact with the forces in such a way as to break the
symmetry that would otherwise be present between the electromagnetic
force and the weak force.  It has not yet been observed; finding it would
would represent a great triumph for the Standard Model, while \emph{not}
finding it might point the way to better theories.  

Indeed, while the Standard Model has passed many stringent experimental
tests, and successfully predicted the existence of many particles which
were later observed (like the W, the Z, and the charm and top quarks),
it is a most puzzling sort of hodgepodge.  Could nature really be this
baroque at its most fundamental level?  Few people seem to think so;
most hope for some deeper, simpler theory.

It's easy to want a "deeper, simpler theory", but how to get it?  What
are the clues?  What can we do?  Experimentalists certainly have their
work cut out for them.  They can try to find or rule out the Higgs.
They can also try to see if neutrinos, assumed to be massless in the
Standard Model, actually have a small mass - for while the Standard
Model could easily be patched if this were the case, it would shed
interesting light on one of the biggest mysteries in physics, namely why
the fermions in nature seem not to be symmetric under reflection, or
"parity".  Right now, we believe that neutrinos only exist in a
left-handed form, rotating one way but not the other around the axis
they move along.  This is intimately related to their apparent
masslessness.  In fact, for reasons that would take a while to explain,
the lack of parity symmetry in the Standard Model forces us to assume
all the observed fermions acquire their mass only through interaction with the Higgs
particle!  For more on the neutrino mass puzzle, try:

1) Paul Langacker, Implications of Neutrino Mass,
<A HREF = "http://dept.physics.upenn.edu/neutrino/jhu/jhu.html">http://dept.physics.upenn.edu/neutrino/jhu/jhu.html</A>

And, of course, experimentalists can continue to do what they always do
best: discover the utterly unexpected.  

Theorists, on the other hand, have been spending the last couple of
decades poring over the standard model and trying to understand what
it's telling us.  It's so full of suggestive patterns and partial 
symmetries!  First, why are there 3 forces here?  Each force goes along
with a group of symmetries called a "gauge group", and electromagnetism
corresponds to U(1), while the weak force corresponds to SU(2) and the
strong force corresponds to SU(3).  (Here U(n) is the group of n \times  n
unitary complex matrices, while SU(n) is the subgroup consisting of
those with determinant equal to 1.)  Well, actually the Standard Model
partially unifies the electromagnetic and weak force into the
"electroweak force", and then resorts to the Higgs to explain why these
forces are so different in practice.  Various "grand unified theories"
or "GUTs" try to unify the forces further by sticking the group 
SU(3) \times  SU(2) \times  U(1) into a bigger group - but then resort to 
still more Higgses to break the symmetry between them!

Then, there is the curious parallel between the leptons and quarks in
each generation.  Each generation has a lepton with mass, a massless or
almost massless neutrino, and two quarks.  The massive lepton has charge
-1, the neutrino has charge 0 as its name suggests, the "down" type
quark has charge -1/3, and the "up" type quark has charge 2/3.  Funny
pattern, eh?  The Standard Model does not really explain this, although
it would be ruined by "anomalies" - certain nightmarish problems that
can beset a quantum field theory - if one idly tried to mess with the
generations by leaving out a quark or the like.  It's natural to try to "unify" the quarks and leptons, and
indeed, in grand unified theories like the SU(5) theory proposed in 1974
of Georgi and Glashow, the quarks and leptons are treated in a unified
way.

Another interesting pattern is the repetition of generations itself.  
Why is there more than one?  Why are there three, almost the same,
but with the masses increasing dramatically as we go up?   The Standard
Model makes no attempt to explain this, although it does suggest that
there had better not be more than 17 quarks - more, and the strong force
would not be "asymptotically free" (weak at high energies), which would
cause lots of problems for the theory.  In fact, experiments strongly
suggest that there are no more than 3 generations.  Why?  

Finally, there is the grand distinction between bosons and fermions.  
What does this mean?  Here we understand quite a bit from basic
principles.  For example, the "spin-statistics theorem" explains why
particles with half-integer spin should satisfy the Pauli exclusion
principle, while those with integer spin should like to hang out
together.  This is a very beautiful result with a deep connection to
topology, which I try to explain in 

2) John Baez, Spin, statistics, CPT and all that jazz, 
<A HREF = "http://math.ucr.edu/home/baez/spin.stat.html">http://math.ucr.edu/home/baez/spin.stat.html</A>

But many people have tried to bridge the chasm between bosons and
fermions, unifying them by a principle called "supersymmetry".  As in
the other cases mentioned above, when they do this, they then need to
pull tricks to "break" the symmetry to get a theory that fits the
experimental fact that bosons and fermions are very different.
Personally, I'm suspicious of all these symmetries postulated only to be
cleverly broken; this approach was so successful in dealing with the
electroweak force - modulo the missing Higgs! - that it seems to
have been accepted as a universal method of having ones cake and eating
it too.  

Now, string theory comes in two basic flavors.  Purely bosonic
string theory lives in 26 dimensions and doesn't have any fermions in
it.  Supersymmetric string theories live in 10 dimensions and have both
bosons and fermions, unified via supersymmetry.  To deal with the 
fermions in nature, most work in physics has focused on the
supersymmetric case.  Just for completeness, I should point out that
there are 5 different supersymmetric string theories: type I, type
IIA, type IIB, E8 \times  E8 heterotic and SO(32) heterotic.  For more on
these, see "<A HREF = "week72.html">week72</A>".  We won't be getting into them here.  Instead,
I just want to explain how fermions work in different dimensions, and
why nice things happen in dimensions of the form 8k + 2.  Most of
what I say is in Section 3 of

3) John H. Schwarz, Introduction to supersymmetry, in Superstrings
and Supergravity, Proc. of the 28th Scottish Universities Summer
School in Physics, ed. A. T. Davies and D. G. Sutherland, University
Printing House, Oxford, 1985.

but mathematicians may also want to supplement this with material
from the book "Spin Geometry" by Lawson and Michelson, cited in
"<A HREF = "week82.html">week82</A>".  

To understand fermions in different dimensions we need to understand
Clifford algebras.  As far as I know, when Clifford originally invented 
these algebras in the late 1800s, he was trying to generalize Hamilton's
quaternion algebra by considering algebras that had lots of different
anticommuting square roots of -1.  In other words, he considered
an associative algebra generated by a bunch of guys e_{1},...,
e_{n}, satisfying

e_{i}^{2} = -1

for all i, and 

e_{i} e_{j} = - e_{j} e_{i}

whenever i is not equal to j.  I discussed these algebras in "<A HREF = "week82.html">week82</A>"
and I said what they all were - they all have nice descriptions in terms
of the reals, the complexes, and the quaternions.  

These original Clifford algebras are great for studying rotations in 
n-dimensional Euclidean space - please take my word for this for now.
However, here we want to study rotations and Lorentz transformations
in n-dimensional Minkowski spacetime, so we need to work with a slightly 
Different kind of Clifford algebra, which was probably invented by Dirac.  
In n-dimensional Euclidean space the metric (used for measuring distances) 
is

                        dx_{1}^{2} + dx_{2}^{2} + ... + dx_{n}^{2}  

while in n-dimensional Minkowski spacetime it is

                        dx_{1}^{2} + dx_{2}^{2} 
+ ... - dx_{n}^{2}   
 
or if you prefer (it's just a matter of convention), you can
take it to be

                       -dx_{1}^{2} - dx_{2}^{2} 
- ... + dx_{n}^{2}   


So it turns out that we need to switch some signs in the definition 
of the Clifford algebra when working in Minkowski spacetime.  

In general, we can define the Clifford algebra C_{p,q} to be the 
algebra generated by a bunch of elements e_{i}, with p of them 
being square roots 
of -1 and q of them being square roots of 1.  As before, we require that 
they anticommute:

e_{i} e_{j} = - e_{j} e_{i}

when i and j are different.  Physicists usually call these guys "gamma
matrices".  For n-dimensional Minkowski space we can work either
with C_{n-1,1} or C_{1,n-1}, depending on our preference.  
As Cecile DeWitt has pointed out, it \emph{does} 
make a difference which one we 
use.  

With some work, one can check that these algebras go like this:


\begin{verbatim}

C_{0,1}   R + R               C_{1,0}   C
C_{1,1}   R(2)                C_{1,1}   R(2)
C_{2,1}   C(2)                C_{1,2}   R(2) + R(2)
C_{3,1}   H(2)                C_{1,3}   R(4)
C_{4,1}   H(2) + H(2)         C_{1,4}   C(4)
C_{5,1}   H(4)                C_{1,5}   H(4)
C_{6,1}   C(8)                C_{1,6}   H(4) + H(4) 
C_{7,1}   R(16)               C_{1,7}   H(8)

\end{verbatim}
    
I've only listed these up to 8-dimensional Minkowski spacetime, and
the cool thing is that after that they sort of repeat - more precisely,
C_{n+8,1} is just the same as 16 \times  16 matrices with entries in 
C_{n,1},
and C_{1,n+8} is just 16 \times  16 matrices with entries in 
C_{1,n}!  
This "period-8" phenomenon, sometimes called Bott periodicity, has 
implications for all sorts of branches of math and physics.  This is
why fermions in 2 dimensions are a bit like fermions in 10 dimensions
and 18 dimensions and 26 dimensions....

In physics, we describe fermions using "spinors", but there are
different kinds of spinors: Dirac spinors, Weyl spinors, Majorana
spinors, and even Majorana-Weyl spinors.  This is a bit technical but
I want to dig into it here, since it explains what's special about
8k + 2 dimensions and especially 10 dimensions.  

Before I get technical, though, let me just summarize the point for 
those of you who don't want all the gory details.   "Dirac spinors"
are what you use to describe spin-1/2 particles that come in both
left-handed and right-handed forms and aren't their own antiparticle 
- like the electron.  Weyl spinors have half as many components,
and describe spin-1/2 particles with an intrinsic handedness that 
aren't their own antiparticle - like the neutrino.   "Weyl spinors"
are only possible in even dimensions!

Both these sorts of spinors are "complex" - they have complex-valued 
components.  But there are also real spinors.  These are used for describing 
particles that are their own antiparticle, because the operation of 
turning a particle into an antiparticle is described mathematically
by complex conjugation.  "Majorana spinors" describe spin-1/2 particles 
that come in both left-handed and right-handed forms and are their 
own antiparticle.  Finally, "Majorana-Weyl spinors" are used to describe 
spin-1/2 particles with an intrinsic handedness that are their own
antiparticle.  

As far as we can tell, none of the particles we've seen are Majorana
or Majorana-Weyl spinors, although if the neutrino has a mass it
might be a Majorana spinor.  Majorana and Majorana-Weyl spinors
only exist in certain dimensions.  In particular, Majorana-Weyl spinors
are very finicky: they only work in dimensions of the form 8k + 2.  
This is part of what makes supersymmetric string theory work in 10 
dimensions!

Now let me describe the technical details.  I'm doing this mainly
for my own benefit; if I write this up, I'll be able to refer to
it whenever I forget it.  For those of you who stick with me, there
will be a little reward: we'll see that a certain kind of supersymmetric 
gauge theory, in which there's a symmetry between gauge bosons and 
fermions, only works in dimensions 3, 4, 6, and 10.  Perhaps 
coincidentally - I don't understand this stuff well enough to know -
these are also the dimensions when supersymmetric string theory works
classically.  (It's the quantum version that only works in dimension 10.)

So: part of the point of these Clifford algebras is that they give 
representations of the double cover of the Lorentz group in different
dimensions.  In "<A HREF = "week61.html">week61</A>" I explained this double cover business,
and how the group SO(n) of rotations of n-dimensional Euclidean space 
has a double cover called Spin(n).  Similarly, the Lorentz group
of n-dimensional Minkowski space, written SO(n-1,1), has a double cover 
we could call Spin(n-1,1).  The spinors we'll discuss are all 
representations of this group.  

The way Clifford algebras help is that there is a nice way to
embed Spin(n-1,1) in either C_{n-1,1} or C_{1,n-1}, so any 
representation of these Clifford algebras gives a representation
of Spin(n-1,1).   We have a choice of dealing with real representations or 
complex representations.  Any complex representation of one of
these Clifford algebras is also a representation of the \emph{complexified} 
Clifford algebra.   What I mean is this: above I implicitly wanted
C_{p,q} to consist of all \emph{real} linear combinations of products of 
the e_{i}, but we could have worked with \emph{complex} linear combinations 
instead.  Then we would have "complexified" C_{p,q}.  
Since the
complex numbers include a square root of minus 1, the complexification
of C_{p,q} only depends on the dimension p + q, not on how many minus 
signs we have. 

Now, it is easy and fun and important to check that if you complexify R 
you get C, and if you complexify C you get C + C, and if you complexify 
H you get C(2).  Thus from the above table we get this table: 


\begin{verbatim}

dimension n        complexified Clifford algebra

    1                  C + C
    2                  C(2)
    3                  C(2) + C(2)
    4                  C(4)
    5                  C(4) + C(4)
    6                  C(8)
    7                  C(8) + C(8)
    8                  C(16)

\end{verbatim}
    
Notice this table is a lot simpler - complex Clifford algebras
are "period-2" instead of period-8.  

Now the smallest complex representation of the complexified Clifford
algebra in dimension n is what we call a "Dirac spinor".  We can figure
out what this is using the above table, since the smallest complex 
representation of C(n) or C(n) + C(n) is on the n-dimensional complex
vector space C^{n}, given by matrix multiplication.  Of course, for 
C(n) + C(n) there are \emph{two} representations depending on which copy 
of C(n) we use, but these give equivalent representations of Spin(n-1,1), 
which is what we're really interested in, so we still speak of "the" 
Dirac spinors.

So we get:


\begin{verbatim}

dimension n       Dirac spinors 

     1                 C
     2                 C^{2}
     3                 C^{2}
     4                 C^{4}
     5                 C^{4}
     6                 C^{8}
     7                 C^{8}
     8                 C^{16}

\end{verbatim}
    
The dimension of the Dirac spinors doubles as we go to each new
even dimension.

We can also look for the smallest real representation of C_{n-1,1}
or C_{1,n-1}.  This is easy to work out from our tables using
the fact that the algebra R has its smallest real representation 
on R, while for C it's on R^{2} and for H it's on R^{4}.  

Sometimes this smallest real representation is secretly just the 
Dirac spinors \emph{viewed as a real representation} - we can view C^{n}
as the real vector space R^{2n}.   But sometimes the Dirac spinors 
are the \emph{complexification} of the smallest real representation -
for example, C^{n} is the complexification of R^{n}.   In this
case folks call the smallest real representation "Majorana spinors". 

When we are looking for the smallest real representations, we get 
different answers for C_{n-1,1} and C_{1,n-1}.  
Here is what we get:


\begin{verbatim}

n   C_{n-1,1}      smallest            C_{1,n-1}       smallest 
              real rep                        real rep 

1    R + R        R    Majorana     C             R^{2}    
2    R(2)         R^{2}   Majorana     R(2)          R^{2}   Majorana
3    C(2)         R^{4}                R(2) + R(2)   R^{2}   Majorana
4    H(2)         R^{8}                R(4)          R^{4}   Majorana  
5    H(2) + H(2)  R^{8}                C(4)          R^{8} 
6    H(4)         R^{16}               H(4)          R^{16}  
7    C(8)         R^{16}               H(4) + H(4)   R^{16}  
8    R(16)        R^{16}  Majorana     H(8)          R^{32}

\end{verbatim}
    
I've noted when the representations are Majorana spinors.  Everything
repeats with period 8 after this, in an obvious way.

Finally, sometimes there are "Weyl spinors" or
"Majorana-Weyl" spinors.  The point is that sometimes the
Dirac spinors, or Majorana spinors, are a \emph{reducible}
representation of Spin(1,n-1).  For Dirac spinors this happens in every
even dimension, because the Clifford algebra element 
 \Gamma  =
e_{1} ... e_{n} 
 commutes with everything in
Spin(1,n-1) and \Gamma ^{2} is 1 or -1, so we can break the space
of Dirac spinors into the two eigenspaces of \Gamma , which will be
smaller reps of Spin(1,n-1) - the "Weyl spinors".  Physicists
usually call this \Gamma  thing "\gamma _{5}", and it's an
operator that represents parity transformations.  We get
"Majorana-Weyl" spinors only when we have Majorana spinors, n
is even, and \Gamma ^{2} = 1, since we are then working with real
numbers and -1 doesn't have a square root.  You can work out
\Gamma ^{2} for either C_{n-1,1} or C_{1,n-1},
and see that we'll only get Majorana-Weyl spinors when n = 8k + 2.

Whew!  Let me summarize some of our results:


\begin{verbatim}

n    Dirac     Majorana       Weyl    Majorana-Weyl

1     C           R             
2     C^{2}          R^{2}           C           R
3     C^{2}          R^{2} 
4     C^{4}          R^{4}           C^{2}
5     C^{4}        
6     C^{8}                       C^{4}
7     C^{8}
8     C^{16}         R^{16}          C^{8}

\end{verbatim}
    
When there are blanks here, the relevant sort of spinor doesn't
exist.  Here I'm not distinguishing Majorana spinors that come from
C_{n-1,1} and those that come from C_{1,n-1}; you can do that with
the previous table.  Again, things continue for larger n in an obvious
way.  

Now, let's imagine a theory that has a supersymmetry between a gauge
bosons and a fermion.  We'll assume there are as many physical degrees of 
freedom for the gauge boson as there are for the fermion.   Gauge
bosons have n - 2 physical degrees of freedom in n dimensions: for
example, in dimension 4 the photon has 2 degrees of freedom, the left
and right polarized states.  So we want to find a kind of spinor that
has n - 2 physical degrees of freedom.  But the number of physical
degrees of freedom of a spinor field is half the number of (real) 
components of the spinor, since the Dirac equation relates the
components.  So we are looking for a kind of spinor that has 2(n - 2)
real components.  This occurs in only 4 cases:

n = 3:    2(n-2) = 2, and Majorana spinors have 2 real components
n = 4:  2(n-2) = 4, and Majorana or Weyl spinors have 4 real components
n = 6:  2(n-2) = 8, and Weyl spinors have 8 real components
n = 10:  2(n-2) = 16, and Majorana-Weyl spinors have 16 real components
Note we count complex components as two real components.  And note how
dimension 10 works: the dimension of the spinors grows pretty fast as
n increases, but the Majorana-Weyl condition reduces the dimension by
a factor of 4, so dimension 10 just squeaks by!

Here John Schwarz explains how nice things happen in the same dimensions
for superstring theory: 

4) John H. Schwarz, Introduction to superstrings, in Superstrings
and Supergravity, Proc. of the 28th Scottish Universities Summer
School in Physics, ed. A. T. Davies and D. G. Sutherland, University
Printing House, Oxford, 1985.

He also makes a tantalizing remark: perhaps these 4 cases correspond
somehow to the reals, complexes, quaternions and octonions.  Note:
3 = 1 + 2, 4 = 2 + 2, 6 = 4 + 2 and 10 = 8 + 2.  You can never tell
with this stuff... everything is related.

I thank Joshua Burton for helping me overcome my fear of Majorana
spinors, and for correcting a number of embarassing errors in the
first version of this article.

\par\noindent\rule{\textwidth}{0.4pt}
\textbf{Addendum:} In July 2001, long after the above article was written,
Lubos Motl explained where the number 18 shows up in string theory:


\begin{quote}
Today we know that the two heterotic string theories are related by
various dualities. For example, in 17+1 dimension, the lattices 
\Gamma _{16} and \Gamma _{8}+\Gamma _{8}, 
with an added Lorentzian \Gamma _{1,1}, become
isometric. There is a single even self-dual lattice in 17+1 dimensions,
\Gamma _{17,1}. This is the reason why two heterotic string theories 
are T-dual to each other. The compactification on a circle adds two extra
U(1)s (from Kaluza-Klein graviphoton and the B-field), and with
appropriate Wilson lines, a compactification of one heterotic string
theory on radius R is equivalent to the other on radius 1/R, using 
correct units.
\end{quote}


Also, in <a href = "week104.html">week104</a>, and especially in the 
Addendum written by Robert Helling, we'll see that it's \emph{not}
a coincidence that super-Yang-Mills theory works nicely in dimensions 
that are 2 more than the dimensions of the reals, complex numbers,
quaternions and octonions.  

\par\noindent\rule{\textwidth}{0.4pt}
<em>Since the mathematicians have grabbed ahold of the theory of relativity,
I no longer understand it.</em> - Albert Einstein


\par\noindent\rule{\textwidth}{0.4pt}

% </A>
% </A>
% </A>
