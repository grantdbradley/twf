
% </A>
% </A>
% </A>
\week{January 11, 2003 }



Now I'm in Sydney, Australia, trying to learn a bit of category theory
from the experts here.  It's quite a change.  Hong Kong was louder, 
faster, more densely packed and more commercial than the USA.
Australia seems quieter, slower, sparser and less commercialized.
Odd to think that all three were British colonies!  They seem like
different worlds.

Anyway, on to business.  People are starting to get more interested
in the role played by octonions and exceptional groups in superstring
theory and supergravity.  There are a lot of pretty patterns here that
may boil down to pure algebra... in which case I might be able to
understand them and maybe even come up with something cool!

Here are some of the papers I'm struggling to read about this.  First, 
a nice introduction to how supergravity works in different dimensions:

1) Antoine Van Proeyen, Structure of supergravity theories, available as
<A HREF = "http://www.arXiv.org/abs/hep-th/0301005">hep-th/0301005</A>. 


 "We give an elementary introduction to the structure of
supergravity theories.  This leads to a table with an overview of
supergravity and supersymmetry theories in dimensions 4 to 11.  The
basic steps in constructing supergravity theories are considered:
determination of the underlying algebra, the multiplets, the actions,
and solutions.  Finally, an overview is given of the geometries that
result from the scalars of supergravity theories."

Second, a interesting study of how you get supergravities in 
different dimensions by "oxidizing" 4-dimensional theories. 
This is a pun on "reduction", the process whereby we go down from
high dimensions to lower ones by curling up the extra dimensions.
It turns out that oxidation is deeply related to Dynkin diagrams:

2) Arjan Keurentjes, The group theory of oxidation, available as
<A HREF = "http://www.arXiv.org/abs/hep-th/0210178">hep-th/0210178</A>.


"Dimensional reduction of (super-)gravity theories to 3 dimensions
results in \sigma  models on coset spaces G/H, such as the E8/SO(16) coset
in the bosonic sector of 3 dimensional maximal supergravity.  The
reverse process, oxidation, is the reconstruction of a higher
dimensional gravity theory from a coset \sigma  model. Using the group G
as starting point, the higher dimensional models follow essentially from
decomposition into subgroups.  All equations of motion and Bianchi
identities can be directly reconstructed from the group lattice;
Kaluza-Klein modifications and Chern-Simons terms are encoded in the
group structure.  Manipulations of extended Dynkin diagrams encode
matter content and (string) dualities.  The reflection symmetry of the
"magic triangle" for E_{n} gravities, and approximate
reflection symmetry of the older "magic triangle" of
supergravities in 4 dimensions, are easily understood in this
framework."

Next, a tour of places where the octonions show up in string 
theory:

3) Luis J. Boya, Octonions and M-theory, available as <A HREF =
"http://www.arXiv.org/abs/hep-th/0301037">hep-th/0301037</A>.

"We explain how structures related to octonions are ubiquitous in 
M-theory.  All the exceptional Lie groups, and the projective Cayley 
line and plane, appear in M-theory.  Exceptional G2-holonomy manifolds 
show up as compactifying spaces, and are related to the M2 Brane and 
3-form.  We review this evidence, which comes from the initial 11-dim 
structures.  Relations between these objects are stressed, when extant 
and understood.  We argue for the necessity of a better understanding of 
the role of the octonions themselves (in particular non-associativity) 
in M-theory." 

And here's an article about where the exceptional groups show up,
from a true expert on the subject:

4) Pierre Ramond, Exceptional groups and physics, available as
<A HREF = "http://www.arXiv.org/abs/hep-th/0301050">hep-th/0301050</A>. 

"Quarks and leptons charges and interactions are derived from gauge 
theories associated with symmetries.  Their space-time labels come 
from representations of the non-compact algebra of Special Relativity. 
Common to these descriptions are the Lie groups stemming from their 
invariances.  Does Nature use Exceptional Groups, the most distinctive 
among them?  We examine the case for and against their use.  They do 
indeed appear in charge space, as the Standard Model fits naturally 
inside the exceptional group E6.  Further, the advent of the E8 x E8 
Heterotic Superstring theory adds credibility to this venue.  On
the other hand, their use as space-time labels has not been as evident 
as they link spinors and tensors under space rotations, which flies 
in the face of the spin-statistics connection.  We discuss a way to 
circumvent this difficulty in trying to generalize eleven-dimensional 
supergravity."

I haven't read this, but indeed, it's really annoying how structures
like triality mix integer and half-integer spin objects in a way that
doesn't seem to make physical sense.  Does he really have a way to get
around it?

Oh well.  Let me talk about something I \emph{do} understand.


Last week I said a bunch about "structure types", also called
"species".  A structure type is any sort of structure you can
put on finite sets, but the cool part is that structure types act like
power series.  This fact has various spinoffs.  Last week I sketched how
people use it to solve problems in combinatorics.  In "<A HREF =
"week185.html">week185</A>" I explained how it lets us categorify
the harmonic oscillator!  And now I want to explain how it gives a nice
way of understanding operads.

But first I need to say what operads \emph{are}.  The slick way to define 
them uses structure types - but this is a bit devious, so it might
fool you into thinking that operads are hard to understand.  They're
actually not, so I'll start out with an elementary introduction to operads, 
then give you some references for further study... and then pull out all
the stops and explain how they're related to structure types.

So: what's an operad?  An operad O consists of a set O_{n} of abstract 
`n-ary operations' for each natural number n, together with rules for 
composing these operations.  We can think of an n-ary operation as a 
little black box with n wires coming in and one wire coming out:  


\begin{verbatim}

                  \    |    /
                   \   |   / 
                    \  |  /
                     -----
                    |     | 
                     -----
                       |
                       |
\end{verbatim}
    
We're allowed to compose these operations like this:


\begin{verbatim}

         \    /     \  |  /       | 
          \  /       \ | /        |
          -----      -----      -----
         |     |    |     |    |     |
          -----      -----      -----
              \        |        /  
               \       |       /
                \      |      /
                 \     |     /
                  \    |    / 
                   \   |   /  
                    \  |  /
                     -----
                    |     | 
                     -----
                       |
                       |
\end{verbatim}
    
feeding the outputs of n operations g_{1},..,g_{n} into
the inputs of an n-ary operation f, obtaining a new operation which we
call f o (g_{1},...,g_{n}).  We demand that there be a
unary operation serving as the identity for composition, and we impose
an "associative law" that makes a composite of composites like
this well-defined:



\begin{verbatim}

              \    /   |  \  |  /   \     / 
               \  /    |   \ | /     \   / 
                ---   ---   ---       ---
               |   | |   | |   |     |   |
                ---   ---   ---       ---
                   \   |   /          /
                    \  |  /          / 
                     \ | /          / 
          -----      -----      -----
         |     |    |     |    |     |
          -----      -----      -----
              \        |        /  
               \       |       /   
                \      |      /
                 \     |     / 
                  \    |    / 
                   \   |   /   
                    \  |  /
                     -----
                    |     | 
                     -----
                       |
                       |
\end{verbatim}
    
(This picture has a 0-ary operation in it, just to emphasize
that this is allowed.)  We can permute the inputs of an n-ary 
operation and get a new operation:
                      

\begin{verbatim}

                      \ /   /
                       /   /
                      / \ /    
                     /   /   
                    /   / \
                    \  |  /
                     -----
                    |     | 
                     -----
                       |
                       |
\end{verbatim}
    
We demand that this give an action of each permutation group S_{n} 
on each set O_{n}.  Finally, we demand that these actions
be compatible with composition, in a way that's supposed to be
obvious from the pictures.   For example:



\begin{verbatim}

       \  |  /   |   \   /               \\\ /   / /
        \ | /    |    \ /                 \\/   / /
         ---    ---   ---                  /\\ / /
        | a |  | b | | c |                / \\/ /
         ---    ---   ---                /   / /
           \     /   /                  /   / /\\
            \   /   /                  /   | | \\\  
             \ /   /                  /    | |  \\\
              /   /                  ---   ---   ---
             / \ /           =      | b | | c | | a |
            /   /                    ---   ---   ---
           /   / \                      \   |   /
           \  |  /                       \  |  /
            -----                         -----
           |  d  |                       |  d  | 
            -----                         -----
              |                             |
              |                             |
\end{verbatim}
    
That's all there is to it!

With this answered, your next question is probably: "why should
I \emph{care} about operads?" This gets a little more technical.
For a detailed answer, the best place to look is this book:

5)  Martin Markl, Steve Schnider and Jim Stasheff, Operads in Algebra, 
Topology and Physics, AMS, Providence, Rhode Island, 2002. 

But if you just want a taste, try Stasheff's infamous "operadchik" paper -
get it? - which for some reason isn't on the arXiv:

6) James Stasheff, Hartford/Luminy talks on operads, available
at <A HREF = "http://www.math.unc.edu/Faculty/jds/operadchik.ps">  
http://www.math.unc.edu/Faculty/jds/operadchik.ps</A>.  

Another good introduction is this paper by Sasha Voronov:

7) Alexander Voronov, Notes on universal algebra, available as
<A HREF = "http://www.arXiv.org/abs/math.QA/0111009">math.QA/0111009</A>.

Tom Leinster is writing a book on the applications of operads to 
higher-dimensional algebra, but you'll have to wait a while for that.

Anyway, there are many reasons why you should care about operads.
Historically, the first come from topology.  In homotopy theory, the
main way to probe a space X is by looking at maps from the k-sphere to
X.  We define the "kth loop space" of X, \Omega ^{k}(X),
to be the space of all such maps sending the north pole to a chosen
point in X, called the "basepoint".  The set of connected
components of \Omega ^{k}(X) is called the "kth homotopy
group" of X; this is a group for k > 0 and an abelian group for
k > 1.

Most homotopy theorists would gladly sell their souls for the ability to
compute the homotopy groups of an arbitrary space.  However, there is
extra information lurking in the space \Omega ^{k} X that gets lost when
we consider only its connected components.  Starting in the late 1950s,
a large number of excellent topologists including Adams and MacLane,
Stasheff, Boardman and Vogt, and May struggled to understand \emph{all} the 
structure possessed by an k-fold loop space.  


For example, \Omega ^{1}(X) is something like a topological group,
thanks to our ability to "compose" loops.  (For details, see
"<A HREF = "week119.html">week119</A>".)  However, the usual
group laws such as associativity hold only up to homotopy.  To make
matters even trickier, these homotopies satisfy certain laws of their
own, but only up to homotopy - and so on ad infinitum.  Similarly,
\Omega ^{k}(X) is something like an abelian topological group for
k > 1, but again only up to homotopies that themselves satisfy
certain laws up to homotopy, and so on - and in a manner that gets ever
more complicated for higher k!

After more than decade of hard work, it became clear that operads are
the easiest way to organize all these higher homotopies.  Just as a
group can act on a set, so can an operad O, each abstract operation 
in O_{n} being realized as actual n-ary operation on the set in a
manner preserving composition, the identity, and the permutation group
actions.  A set equipped with an action of the operad O is usually
called an "algebra over O", though personally I'd prefer to call
it an action of O on the set.  It turns out that the structure of a
k-fold loop space is completely captured by saying that it is an
algebra over a certain operad!  

Even better, if we choose this operad O to be "cofibrant" - whatever
that means - any space equipped with a homotopy equivalence to a k-fold 
loop space will also become an algebra over O.  This is the simplest 
example of how operads are used to describe "homotopy invariant algebraic 
structures", in which all laws hold up to an infinite sequence of higher 
homotopies.  


For an operad to do this job, it must really have a <em>topological
space</em> of operations O_{n} for each n, since the fact that
various laws hold up to homotopy is expressed by the existence of
certain continuous paths in these spaces.  Similarly, composition and
the permutation group actions should be \emph{continuous maps}.  Finally, we
should only consider algebras that are topological spaces on which the
operad acts \emph{continuously}.

In short, topology really requires operads and their algebras in the
category of topological spaces rather than sets.  The ability to
transplant the theory of operads to various different contexts is an
important aspect of their power.  So, it's good that Markl, Schnider
and Stasheff treat operads in an arbitrary symmetric monoidal category.  
They also prove the worth of this level of generality by discussing many 
examples in detail.  For example, they describe how operads in the category 
of chain complexes have been used to study deformation quantization - and 
also string theory, where the operations of gluing together Riemann 
surfaces are important.  Indeed, these physics applications have led to 
a kind of renaissance in the theory of operads!

Okay.  The last paragraph was packed with buzzwords, so now all
the scaredy-cats are gone.  Let me explain the relation between
operads and structure types.

I said that a structure type is "any sort of structure you can
put on finite sets", but let me make that more precise.  A structure
type is really a functor

F: FinSet_{0} \to  Set

where FinSet_{0} is the groupoid of finite sets and bijections,
and Set is the category of sets and functions.  FinSet_{0} is 
equivalent to the category that has one object, "the n-element
set", for each n, with the morphisms from this object to itself
forming the permutation group S_{n}.  So, we can also think of a
structure type as consisting of a set F(n) for each n, together 
with an action of S_{n} on this set F(n).  This latter viewpoint is 
good for calculation, while the original viewpoint is better for
conceptual work.

We also have morphisms between structure types, which are just
natural transformations between functors of the above sort.  
So, the category of structure types is the functor category 

hom(FinSet_{0}, Set)

To understand why this category acts like the ring of formal
power series in one variable, it's crucial to understand the 
analogy between ordinary set-based algebra and categorified 
algebra.  The quickest way to get a feel for this may be a big 
chart, which starts like this:


\begin{verbatim}

sets                            categories

monoids                         monoidal categories

commutative monoids             symmetric monoidal categories

commutative rigs                symmetric 2-rigs

the free commutative rig on     the free symmetric 2-rig on
no generators: N                no generators: Set

the free commutative rig on     the free symmetric 2-rig on
one generator: N[x]             on generator: Set[[x]] = hom(FinSet_{0}, Set) 
\end{verbatim}
    
I'll assume you understand the first three lines of the chart,
e.g. that just as a monoid is a set equipped with an associative
multiplication and identity element, a "monoidal category" is
a category equipped with the same sort of structure, but where 
all the laws hold only up to isomorphism, and these isomorphisms
in turn satisfy some coherence laws.  Similarly, a symmetric 
monoidal category is like a commutative monoid.  

We can then throw in an extra operation, "addition".  Recall that 
a "rig" is a set with two monoid structures + and x, where + is 
commutative and x distributes over +.  Most algebraists prefer
rings, where you can also subtract, but the natural numbers N are
just a rig, and working over N instead the integers is important in 
combinatorics.
The reason, ultimately, is that N is the free commutative rig on no 
generators!  

\emph{No} generators?  Yes - since you get the numbers 0 and 1 
for free in the definition of a rig, without needing to throw in any 
generators, and then the rig operations give you 1+1, 1+1+1, and so on.


Now, a 2-rig should be a categorified analogue of a rig.  The classic
example is the category of sets, where "addition" is disjoint
union and "multiplication" is Cartesian product.  It would be
nice if this were the free 2-rig on no generators, to emphasize the
analogy between natural numbers and sets.


There are various different ways to accomplish this, but one nice way is
to define a "2-rig" as a monoidal category with colimits,
where the monoidal structure preserves colimits in each argument.  The
colimits act like addition and the monoidal structure acts like
multiplication.  Given this, it's easy to check that the free 2-rig on
no generators is the category Set.

(If we prefer an analogy between natural numbers and \emph{finite} sets, we
should say "finite colimits" instead of colimits in the definition of
2-rig: then FinSet will be the free 2-rig on no generators.)

Now, what's the free commutative rig on \emph{one} generator?

It's N[x], the algebra of polynomials in one variable, with natural
number coefficients.

If we complete this a bit, we get N[[x]], the algebra of formal
power series with natural number coefficients.  But let's categorify
it, instead...

What's the free symmetric 2-rig on one generator?

It's the category of STRUCTURE TYPES!
I'll leave the proof of this as a puzzle for the budding category
theorists out there.  This is supposed to explain very precisely the
sense in which structure types are a categorified version of formal
power series.


(You might argue that structure types are the categorified version of
polynomials, not formal power series, since the free commutative rig on
one generator is an algebra of \emph{polynomials}.  But unlike in a
rig, we have no trouble doing "infinite sums" in a 2-rig,
since we've got arbitrary colimits.  So, the difference between
polynomials and formal power is not so big.  Indeed, there's nothing
"formal" about infinite sums in the categorified situation,
since divergent sums aren't a problem: a sum will always converges to
some set, though possibly an infinite set.  This is one of the great
reasons to categorify.  Of course, the price you pay is that nobody
is sure how to handle negative numbers in categorified mathematics.)

Now, formal power series can be multiplied in two ways: ordinary
multiplication:

(FG)(x) = F(x) G(x)

which is commutative, and composition: 

(FoG)(x) = F(G(x))

which is not.  I talked about both of these and their combinatorial 
meaning for generating functions last time.  Ordinary multiplication
makes power series into a commutative rig; composition is
noncommutative, and it doesn't give us a rig, since it only distributes 
over addition on one side:
(F+G) o H = FoH + GoH
Even worse, the composite F o G can diverge!
Similarly, structure types can be multiplied in two ways:
ordinary multiplication and composition.  I described how both
of these work last time.  Ordinary multiplication makes power 
series into a symmetric 2-rig; composition is not symmetric, and 
it doesn't give us a 2-rig, since it only distributes over
colimits on one side.  However, we
don't have to worry about; composition really does put
a well-defined monoidal structure on the category of structure types.

The "ordinary" multiplication is what makes structure types into
the free symmetric 2-rig on one generator, but composition is also
cool.  It's related to operads.  And here's how.

Recall from "<A HREF = "week89.html">week89</A>" that we can define a "monoid object" in any
monoidal category.  This leads to another puzzle:

What's a monoid object in the category of structure types 
with composition as the monoidal structure?  

And the answer is: AN OPERAD!

Now, this took me quite a while to deeply understand, but when I did it 
was great.   So, if you have enough category theory under your belt to 
have any chance at seeing why what I said is true, please work on it for 
a while and try to understand it.  Just follow through all the definitions, 
until you see that indeed, what I'm claiming is true.  It will strengthen 
your brain... you will literally grow new neurons.  

\par\noindent\rule{\textwidth}{0.4pt}
Addendum: after an informally summarized list of axioms for the
definition of an operad, I wrote above:

\begin{verbatim}

That's all there is to it!
\end{verbatim}
    
Alas, this isn't quite true.  Peter May has subsequently pointed out to me 
that the book by Stasheff \emph{et al}
omits a crucial clause in the definition of operad,
namely that operations like this are well-defined:


\begin{verbatim}

  \    /   |  |    \ /  
   \  /    |  |     / 
     /     |  |    / \ 
    / \    |  |   /   \
    \  |  /   |   \   /    
     \ | /    |    \ /    
      ---    ---   ---   
     | a |  | b | | c | 
      ---    ---   --- 
         \    |    /
          \   |   /  
           \  |  / 
            ----- 
           |  d  |  
            -----  
              |   
              |  
\end{verbatim}
    
Here we can either compose the operations a,b,c with d and then
apply a permutation to the arguments of the result, or apply 
permutations to the arguments of a,b, and c and then compose
the resulting operation with d - we get the same answer either way!

I hope in some future edition they'll be able to correct this
mistake.

\par\noindent\rule{\textwidth}{0.4pt}
% </A>
% </A>
% </A>
