
% </A>
% </A>
% </A>
\week{June 18, 2000}

First I'd like to say some stuff about Lagrange points.  Then I'll
continue talking about complex oriented cohomology theories.

Euler and Lagrange won the Paris Academy Prize in 1722 for their work 
on the orbit of the moon.  The essay that Lagrange submitted for this
prize, "Essai sur le probleme des trois corps", contained some
interesting results on the 3-body problem.  Among other things, he
studied the case where a lighter body was revolving about a heavier one
in a circular orbit, and found all the places where a third much lighter
body could sit "motionless" with respect to the other two.  
There are 5
such places, and they're now called the "Lagrange points" L1 - L5.  

For example: imagine the moon going around the earth in a circular
orbit.  Then there are 5 Lagrange points where we can put a satellite.
3 of these are unstable equilibria.  They lie on the line through the 
earth and moon.  L1 is between the earth and moon, L2 is in the same 
direction as the moon but a bit further out, and L3 is opposite the moon.  
The other 2 Lagrange points are stable equilibria.  L4 lies in the moon's
orbit 60 degrees ahead of the moon, while L5 lies 60 degrees behind the 
moon.  

Here's a nice picture of the Lagrange points: 

1) Lagrange points, <A HREF = "http://map.gsfc.nasa.gov/m_mm/ob_techorbit1.html">http://map.gsfc.nasa.gov/m_mm/ob_techorbit1.html</A>

In general, the points L4 and L5 will be stable equilibria as long as
the heavy body (e.g. the earth in the above example) is more than 
25 times as massive as the intermediate-sized one (e.g. the moon).  
And in case you're wondering, this magic number is just an approximation 
to the exact figure, which is 

(25/2) (1 + sqrt(1 - 4/625))   =   24.95993579437711227887....

Here's a proof that the Lagrange points work as advertised, including
a derivation of the above number:

2) Neil J. Cornish, The Lagrange points, available at
<A HREF = "http://www.astro.princeton.edu/~njc/lagrange.ps.gz">
http://www.astro.princeton.edu/~njc/lagrange.ps.gz</A>

Now, Lagrange did his calculation as a mathematical exercise and didn't 
believe it was relevant to the actual solar system.  But he was wrong
about that!  The stable Lagrange points L4 and L5 are quite interesting,
because stuff tends to accumulate there.
 
For example, people have found over six hundred asteroids called 
"Trojans" 
at the stable Lagrange points of Jupiter's orbit around the sun.  The 
first to be discovered was 588 Achilles, back in 1906 - the number here 
meaning that it was the 588th asteroid found.  In general, the Trojans 
at L4 are named after Greek soldiers in the Trojan war, while those at 
L5 are named after actual Trojans - soldiers from the city of Troy!  You
can see the Trojans quite clearly in this picture of the asteroid belt:

3) Asteroid belt,
<A HREF = "http://www-groups.dcs.st-and.ac.uk/~history//Diagrams/Asteroids.gif">http://www-groups.dcs.st-and.ac.uk/~history//Diagrams/Asteroids.gif</A>

The asteroid 5261 Eureka is a "Martian Trojan", occupying the
L5 point of Mars' orbit around the sun.  A second Martian Trojan was 
discovered at L5 in 1998, but it doesn't have a name yet.  For more 
information, try:

4) Minor Planet Center, Trojan minor planets, 
<A HREF = "http://cfa-www.harvard.edu/cfa/ps/lists/Trojans.html">
http://cfa-www.harvard.edu/cfa/ps/lists/Trojans.html</A>

There may also be a few small asteroids at the Lagrange points of 
Venus and Earth's orbits around the sun.  Does anyone know more 
about this?   My brief search revealed only some information about 
the curious asteroid 3753 Cruithne, which is a companion asteroid 
of the Earth.  But 3753 Cruithne is not at a Lagrange point!  Instead, 
it moves in a very complicated spiralling horse-shoe shaped orbit 
relative to the earth.  For a beautiful explanation with pictures 
by the discoverers of this phenomenon, see:

5) Paul Wiegert, Kimmo Innanen and Seppo Mikkola, Near-earth asteroid
3753 Cruithne - Earth's curious companion, 
<A HREF = "http://www.astro.queensu.ca/~wiegert/">http://www.astro.queensu.ca/~wiegert/</A>

For over 150 years, astronomers have been searching for other satellites
of the Earth besides the big one I see out my window right now.  There
have been a lot of false alarms, with people even giving names to the
satellites they thought they discovered, like "Kleinchen" and "Lilith".
For the fascinating story of these "second moons" and other hypothetical
planets, see:

6) Paul Schlyter, Hypothetical planets,
<A HREF = "http://seds.lpl.arizona.edu/nineplanets/nineplanets/hypo.html">http://seds.lpl.arizona.edu/nineplanets/nineplanets/hypo.html</A>

Unfortunately, none of these second moons were real.  But in the 1960s,
people discovered dust clouds at the stable Lagrange points of the Moon's 
orbit around the Earth!   They are about 4 times as big as the Moon, but
they are not very substantial.    

What else lurks at Lagrange points? 

Well, Saturn has a moon called Dione, and 60 degrees ahead of Dione, 
right at the Lagrange point L4, there is a tiny moon called Helene.  
Here's a picture of Helene:

7) Astronomy picture of the day: Dione's Lagrange moon Helene, 
<A HREF = "http://antwrp.gsfc.nasa.gov/apod/ap951010.html">http://antwrp.gsfc.nasa.gov/apod/ap951010.html</A>

Isn't she cute?  There's also a small moon called Telesto at the L4 
point of Saturn's moon Tethys, and one called Calypso at the L5 point 
of Tethys:

8) Bill Arnett, Introduction to the nine planets: Tethys, Telesto and Calypso,
<A HREF = "http://seds.lpl.arizona.edu/nineplanets/nineplanets/tethys.html">http://seds.lpl.arizona.edu/nineplanets/nineplanets/tethys.html</A>

Does anyone know other natural occupants of Lagrange points?  

For a long time crackpots and science fiction writers have talked about
a "Counter-Earth", complete with its own civilization, located 
at the L3
point of the Earth's orbit around the Sun - exactly where we can never
see it!  I think satellite explorations have ruled out this possibility
by now, but since L3 is an unstable Lagrange point, it was never very
likely to begin with.   

On the other hand, fans of space exploration have long dreamt of 
setting up a colony at the stable Lagrange points of the Moon's 
orbit around the earth - right in those dust clouds, I guess.   
But now people are putting artificial satellites at the Lagrange 
points of the Earth's orbit around the Sun!  For example, L1 is 
the home of "SOHO": the Solar and Heliospheric Observatory.  Sitting 
between the Earth and Sun gives SOHO a nice clear view of sunspots, 
solar flares, and the solar wind.  It's not stable, but it can exert 
a bit of thrust now and then to stay put.  For more information and 
some pretty pictures, try:

9) SOHO website, <A HREF = "http://sohowww.nascom.nasa.gov/">http://sohowww.nascom.nasa.gov/</A>

In April 2001, NASA plans to put a satellite called "MAP" at the
Lagrange point L2.  MAP is the Microwave Anisotropy Probe, which will 
study anisotropies in the cosmic microwave background.  These are
starting to be a really interesting window into the early history of
the universe.  For more, see:

10) MAP website, <A HREF = "http://map.gsfc.nasa.gov/">http://map.gsfc.nasa.gov/</A>

Finally, for quantum analogues of the Lagrange points, see:

11) T. Uzer, Ernestine A. Lee, David Farrelly, and Andrea F. Brunello,
Synthesis of a classical atom: wavepacket analogues of the Trojan asteroids,
Contemp. Phys. 41 (2000), 1-14.   Abstract available at
<A HREF = 
"http://www.catchword.co.uk/titles/tandf/00107514/v41n1/contp1-1.htm">
http://www.catchword.co.uk/titles/tandf/00107514/v41n1/contp1-1.htm
% </A>

Okay, enough about Lagrange points!  Now I want to talk a bit more about
complex oriented cohomology theories.  Last time I left off at the 
definition.  So let me start with a little review, and then plunge ahead.

A generalized cohomology theory assigns to each space \times  a bunch of
groups h^{n}(X), one for each integer n.  We impose some axioms that make
them work very much like ordinary cohomology.   However, when \times  is a
point, we no longer require that h^{n}(X) is trivial for nonzero n.  It
may not seem like much, but it turns out to make a big difference!   
There are all sorts of very interesting examples.  

Like what?

Well, there's K-theory, which lets us study a space by looking at vector
bundles over that space and its iterated suspensions.  We get various
flavors of K-theory from various kinds of vector bundle: real K-theory,
complex K-theory, quaternionic K-theory, and so on.  There's even a sort
of K-theory invented by Atiyah that's based on Clifford algebras, called
"KR theory".  For more about all these, try:

12) Dale Husemoller, Fibre Bundles, Springer-Verlag, New York, 1975.

13) H. Blaine Lawson, Jr. and Marie-Louise Michelsohn, Spin Geometry, 
Princeton University Press, Princeton, 1989.
     
Then there's cobordism theory, which lets us study a space by looking at
manifolds mapped into that space and its iterated suspensions.  To be
honest, this is actually how \emph{bordism} theory works - this being a
generalized \emph{homology} theory.  But every generalized homology theory
goes hand-in-hand with a generalized cohomology theory, and if you
understand one you understand the other, at least in principle.... 
Anyway, there are various flavors of cobordism theory corresponding to
various kinds of extra structure you can put on a manifold: 
piecewise-linear cobordism theory, smooth cobordism theory, oriented 
cobordism theory, spin cobordism theory, complex cobordism theory, symplectic
cobordism theory, stable homotopy theory, and so on.  For more about
these, try:

14) Robert E. Stong, Notes on Cobordism Theory, Princeton University
Press, Princeton, 1968.

Finally, there are generalized cohomology theories inspired more by
algebra than by geometry, which only hardcore homotopy theorists seem to
understand, like Morava K-theory and Brown-Peterson theory.

To round off this little tour, I guess I should add that there are lots
of maps going between different generalized cohomology theories!  As I
explained in "<A HREF = "week149.html">week149</A>", each generalized cohomology h^{n} corresponds to a
"spectrum": a list of spaces, each being the loop space of the next. 
Spectra form a category, and given a map between spectra we get a map
between their generalized cohomology theories.  So we shouldn't study
these one of at a time: it's better to play around with lots at once! 
For some important examples of the stuff you can do this way, try:

15) P. E. Conner and E. E. Floyd, The Relation of Cobordism to K-theories,
Lecture Notes in Mathematics 28, Springer-Verlag, New York, 1966.
     
16) Douglas C. Ravenel, Complex Cobordism and Stable Homotopy Groups 
of Spheres, Academic Press, 1986.

and also the books on generalized cohomology listed in "<A HREF = "week149.html">week149</A>".        

Anyway, to bring order to this zoo, it's nice to pick out some of the
special features of \emph{ordinary} cohomology theory and study the 
generalized cohomology theories that share these features.  

For example, in ordinary cohomology, the cohomology groups of a space
fit together to form a graded ring.  If a generalized cohomology theory
is like this, we call it "multiplicative".   Whenever this is the case,
we get a graded ring R called the "coefficient ring" of our theory,
which is simply the cohomology ring of the one-point space. 
For ordinary cohomology theory the coefficient ring is just
Z, but for other theories it can be very interesting and complicated.
By easy
abstract nonsense,  the cohomology ring of any space is an algebra over
the coefficient ring.  

Another nice feature of ordinary cohomology is the first Chern class.
Whenever you have a complex line bundle over a space X, you get an
invariant called its "first Chern class" which lives in
H^{2}(X), and this invariant is sufficiently powerful to
completely classify such line bundles.  Last week I described the
\emph{universal} line bundle over infinite-dimensional complex
projective space:

L \to  CP^{\infty }

and showed how the first Chern class of \emph{any} line bundle comes from the
first Chern class of this one, which I called c.   

If a generalized cohomology theory is multiplicative and there's
an element c of h^{2}(CP^{\infty }) that acts like the first Chern class
of the universal line bundle, we call the theory "complex oriented".   
Of course, to make this precise we need to isolate the key features
of the first Chern class and abstract them.  I did this in "<A HREF = "week149.html">week149</A>",
so I won't do it again here.   Instead, I'll just say a bit about
what we can \emph{do} with a complex oriented cohomology theory.  

For starters, we can use the element c to get an invariant of complex
line bundles - a kind of generalized version of the first Chern class. 
To do this, just remember from "<A HREF = "week149.html">week149</A>" that CP^{\infty } is the
classifying space for complex line bundles.  In other words, \emph{any} line
bundle over \emph{any} space X is isomorphic to a pullback of the universal
line bundle by some map

f: X \to  CP^{\infty }.

Thus, given a line bundle we can find such a map f and use it to pull
back the element c to get an element of h^{2}(X).  This is exactly like
the usual first Chern class of our line bundle, except now we're using 
a generalized cohomology theory instead of ordinary cohomology.  

Can we get any \emph{other} invariants of complex line bundles from
\emph{other} elements of the cohomology of CP^{\infty }?  Not
really: in any complex oriented cohomology theory, the cohomology ring
of CP^{\infty } is just the algebra of formal power series in
the element c:

h^{*}(CP^{\infty }) = R[[c]]

where R is the coefficient ring.   

However, we can get other invariants of complex \emph{vector} bundles, which
are analogous to the higher Chern classes.  In fact, we can mimic a huge
amount of the usual theory of characteristic classes in the context of a
complex oriented cohomology theory.  I'd love to talk about this, but
it would be a digression from my main goal, which is to make elliptic
cohomology at least halfway comprehensible to the amateur.

So: last time I mentioned that CP^{\infty } is an abelian topological
group, with a multiplication map 

m: CP^{\infty } \times  CP^{\infty } \to  CP^{\infty } 

and an inverse map

i: CP^{\infty } \to  CP^{\infty }.

And I explained how these represent the operations of \emph{tensoring} two
line bundles and taking the \emph{dual} of a line bundle, respectively.   Now
let's see what we can do with these maps when we have a complex oriented
cohomology theory.  First of all, since cohomology is contravariant we
get homomorphisms

m^{*}: h^{*}(CP^{\infty }) \to  h^{*}(CP^{\infty } \times  CP^{\infty })

and

i^{*}: h^{*}(CP^{\infty }) \to  h^{*}(CP^{\infty })

But as I already said, we have

h^{*}(CP^{\infty }) = R[[c]]

and similarly we have

h^{*}(CP^{\infty } \times  CP^{\infty }) = 
R[[c]] \otimes  R[[c]]

where the product "\otimes " on the right side is a slightly fattened-up
version of the usual tensor product of algebras over R.  So we really
have homomorphisms

m^{*}: R[[c]] \to  R[[c]] \otimes  R[[c]]

i^{*}: R[[c]] \to  R[[c]]

which satisfy all the usual axioms for the product and inverse in an
abelian group - but turned around backwards.

Folks who like Hopf algebras will immediately note that R[[c]] is like a
Hopf algebra.  A nice way to form a Hopf algebra is to take the algebra
of functions on a group and use the product and inverse in the group to
give this algebra extra operations called the "coproduct" and 
"antipode".  
We're doing the same thing here, except that we're using formal power
series in one variable c instead of functions of one variable.  So folks
call R[[c]] a "formal group law".  

In short: complex oriented cohomology theories give formal group laws!

Lest this seem overly abstract and unmotivated, remember that it's just
a way of talking about what happens to the generalized "first Chern
class" when we tensor line bundles.  In a vague but useful way, we can
visualize guys in R[[c]] as formal power series on the line, where the
line has been equipped with some abelian group structure, at least 
right near the origin.  This group structure is what yields the coproduct

m^{*}: R[[c]] \to  R[[c]] \otimes  R[[c]]

and antipode

i^{*}: R[[c]] \to  R[[c]]

But the real point is that this group structure tells us how to compute 
the generalized "first Chern class" of a tensor product of line bundles
starting from both of their generalized first Chern classes.  

Some examples may help.  In ordinary cohomology theory, when we tensor
two line bundles, we just \emph{add} their first Chern classes.  So
in this case, we've got the line made into a group using addition, and
R[[c]] becomes a formal group law called the "additive formal
group law".

Another famous example is complex K-theory.  In this theory, when we
tensor two line bundles, we basically just \emph{multiply} their
generalized first Chern classes.  That's because cohomology classes in
K-theory are just equivalence classes of vector bundles, and
multiplying them just \emph{means} tensoring them.  So in this case,
R[[c]] becomes the "multiplicative" formal group law.  Of
course, some fiddling around is required, because we don't usually
think of the multiplicative unit 1 as the origin of the line... but we
can if we want.

What are some other examples?  Well, complex cobordism theory is one.
This corresponds to the "universal" formal group law: a
formal group law that has a unique homomorphism to any other one!  In
this case R is quite big: it's called the "Lazard ring".
And this universal aspect of complex cobordism theory makes it the
king of all complex oriented generalized cohomology theories.  I
really should spend about 10 pages explaining it to you in detail, but
I won't....

... because I want to finally say a word about elliptic cohomology!

I've talked a lot about elliptic curves in "<A HREF =
"week13.html">week13</A>", "<A HREF =
"week125.html">week125</A>" and "<A HREF =
"week126.html">week126</A>", so I get to assume you know about
them: an elliptic curve is a 1-dimensional compact complex manifold
which is also an abelian group.  As such, any elliptic curve gives a
formal group law.  And thus we can wonder if this formal group law
comes from a complex oriented cohomology theory... and it does!  And
this is elliptic cohomology!

Now this is just the beginning of a long story: there's much more to 
say, and I don't have the energy to say it here... but I'll just
tantalize you with some of the high points.  Since elliptic curves
can be thought of as the worldsheets of strings, there are a bunch
of interesting relationships between string theory and elliptic 
cohomology.  In addition to the references I've gave you in 
"<A HREF = "week149.html">week149</A>",
see for example:

17) Hirotaka Tamanoi, Elliptic Genera and Vertex Operator Super-Algebras,
Springer Lecture Notes in Mathematics 1704, Springer, Berlin, 1999.

We also get lots of nice relationships to the theory of modular forms.
But the one thing we apparently don't have \emph{yet} is a very neat 
\emph{geometrical} description of elliptic cohomology!  There's got to be 
one, since so much nice geometry is involved... but what is it?


(For an answer to this question, 
see "<A HREF = "week197.html">week197</A>".)


 \par\noindent\rule{\textwidth}{0.4pt}
\textbf{Addendum:} For the proof that the complex cobordism theory
corresponds to the universal formal group law, read Quillen's original
paper on the subject:

18) Daniel Quillen, On the formal group laws of unoriented and complex 
cobordism theory, Bull. Amer. Math. Soc. 75 (1969), 1293-1298. 
Also available as <a href = "http://projecteuclid.org/euclid.bams/1183530915">http://projecteuclid.org/euclid.bams/1183530915</a>



 \par\noindent\rule{\textwidth}{0.4pt}
\emph{In any field, find the strangest thing and then explore it.} - 
John Wheeler


 \par\noindent\rule{\textwidth}{0.4pt}

% </A>
% </A>
% </A>
