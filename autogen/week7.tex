
% </A>
% </A>
% </A>
\week{arch 1, 1993}

1)   Mathematical problems of non-perturbative quantum general
relativity (lectures delivered at the 1992 Les Houches summer school on
Gravitation and Quantization), by Abhay Ashtekar, 87 pp, Plain TeX,
available as <A HREF = "http://xxx.lanl.gov/abs/gr-qc/9302024">gr-qc/9302024</A>.

I described this paper in "<A HREF = "week3.html">week3</A>", but now it's available from gr-qc.
It's a good quick introduction to the loop representation of quantum gravity.

2)  Lectures on Non-perturbative Canonical Gravity, by Abhay Ashtekar,
World Scientific Press, 1991. 

This book, which I finally obtained, is \emph{the} introduction to the loop
representation of quantum gravity.  What's the loop representation?
Well, this is a long story, so you really should read the book.  But
just to get you going, let me describe Ashtekar's "new variables," which
form the basis for Rovelli and Smolin's construction of the loop
representation.  

First, recall that general relativity is usually thought of as a theory
about a metric on spacetime - more precisely, a Lorentzian metric.  Here
spacetime is a 4-dimensional manifold, and a Lorentzian metric allows
you to calculate the "dot product" of any two tangent vectors at a
point.  This is in quotes because, while a normal dot product might look
like  

(v_0,v_1,v_2,v_3).(w_0,w_1,w_2,w_3) = v_0w_0 + v_1w_1 + v_2w_2 + v_3w_3

relative to some basis, for a Lorentzian metric we can always find a
basis of the tangent space such that 

(v_0,v_1,v_2,v_3).(w_0,w_1,w_2,w_3) = v_0w_0 - v_1w_1 - v_2w_2 - v_3w_3

Now the metric in general relativity defines a "connection," which
tells you a tangent vector might "twist around" as you parallel
translate it, that is, move it along while trying to keep it from
rotating unnecessarily.  Here "twist around" is in quotes because, since
you are parallel translating the vector, it's not really "twisting
around" in the usual sense, but it might seem that way relative to some
coordinate system.  For example, if you used polar coordinates to
describe parallel translation on the plane, it might seem that the unit
vector in the r direction "twisted around" towards the \theta  direction
as you dragged it along.  But in another coordinate system - say the
usual x-y system - it would not appear to be "twisting around".  This
fact is expressed by saying "the connection is not a tensor".  

But from the connection we can cook up a big fat tensor, the
"Riemann tensor" R^i_{jkl}, which says how much the vector in the lth
direction (here the indices range from 0 to 3) twists towards the ith
direction when you move it around a teeny little square in the j-k
plane.  The Lagrangian in ordinary GR is just the integral of the 
"Ricci scalar curvature," R, which is gotten from the Riemann tensor by
"contraction", i.e. summing over the indices in a certain way:

R = R^i_{ji}^j

where we are raising indices using the metric in a manner beloved by
physicists and feared by many mathematicians.  If you integrate the
Lagrangian over a region of spacetime you get the 
"action", and in classical general relativity (in a vacuum, for
simplicity) one can formulate the laws of motion simply by saying:
any teeny change in the metric that vanishes on the boundary of the
region should leave the action constant to first order.  In other words,
the solutions of the equations of general relativity are the *stationary
points* of the action.  If you know how to do variational calculus you
can derive Einstein's equations from this variational principle, as it's
called.  Mathematicians will be pleased to know that Hilbert beat
Einstein to the punch here, so the integral of R is called the
"Einstein-Hilbert" action for general relativity.  

But there's another formulation of general relativity in terms of an
action principle.  This is called the "Palatini" action  - and actually I'm
going to describe a slight variation on it, that is conceptually
simpler, and apparently appears for the first time in Ashtekar's book.
The Palatini approach turns out to be more elegant and is a nice
stepping-stone to the Ashtekar approach.  In the Palatini approach one
thinks of general relativity not as being a theory of a metric, but of a
"tetrad" and an "so(3,1) connection".  To explain what these are, I will
cut corners and assume all the fiber bundles lurking around are trivial;
the experts will easily be able to figure out the general case.  So: an
(orthonormal) tetrad, or "vierbein," is a just a kind of field on
spacetime which at each point consists of an (ordered) orthonormal basis
of the tangent space.  If we express the metric in terms of a tetrad, it
looks just like the formula for the standard "inner product" 

(v_0,v_1,v_2,v_3).(w_0,w_1,w_2,w_3) = v_0w_0 - v_1w_1 - v_2w_2 - v_3w_3

This allows us to identify the group of linear transformations of the
tangent space that preserve the metric with the group of linear
transfomations preserving the standard "inner product," which is called
SO(1,3) since there's one plus sign and three minuses.  And from the
connection mentioned above one gets an SO(1,3) connection, or, what's
more or less the same thing, an so(1,3)-valued 1-form, that is, a kind
of field that can eat a tangent vector at any point and spits out 
element of the Lie algebra so(1,3).  

What's so(1,3)?  Well, elements of so(1,3) include
"infinitesimal" rotations and Lorentz transformations, since
SO(1,3) is generated by rotations and Lorentz transformations.  More
precisely, so(1,3) is a 6-dimensional Lie algebra having as a basis the
three infinitesimal rotations J_1, J_2, and J_3 around the three axes,
and the three infinitesimal Lorentz transformations or "boosts" K_1,
K_2, K_3.  The bracket in this most important Lie algebra is given by


\begin{verbatim}

[J_i,J_j] = J_k
[K_i,K_j] = -J_k
[J_i,K_j] = K_k
\end{verbatim}
    

where (i,j,k) is a cyclic permutation of (1,2,3).  (I hope I haven't
screwed up the signs.)  Note that the J's by themselves form a Lie
subalgebra called so(3), the Lie algebra of the rotation group SO(3).
Note that so(3) is isomorphic to the the cute little Lie algebra su(2) I
described in my post "<A HREF = "week5.html">week5</A>"; J_1, J_2, and J_3 correspond to the guys
I, J, and K divided by two.  

The so(1,3) connection has a curvature, and using the tetrads again we
can identify this with the Riemann curvature tensor.  So the Palatini
trick is to rewrite the Einstein-Hilbert action in terms of the
curvature of the so(1,3) connection and the tetrad field.  This is
called the Palatini action.  Charmingly, even though the tetrad field is
utterly unphysical, we can treat it and the so(1,3) connection as
independent fields and, doing calculus of variations to find stationary
points of the action, we get equations equivalent to Einstein's
equations.

Ashtekar's "new variables" - from this point of view - rely on a curious
and profound fact about so(1,3).  Note that so(1,3) is a Lie algebra
over the real numbers.  But if we allow ourselves to form \emph{complex}
linear combinations of the J's and K's, thus:


\begin{verbatim}

M_i = (J_i + iK_i)/2
N_i = (J_i - iK_i)/2
\end{verbatim}
    

(please don't mix up the subscript i = 1,2,3 with the other i, the
square root of minus one) we get the following brackets:


\begin{verbatim}

[M_i,M_j] = M_k
[N_i,N_j] = N_k
[M_i,N_j] = 0
\end{verbatim}
    

I think the signs all work but I wouldn't trust me if I were you.  
The wonderful thing here is that the M's and N's commute with each
other, and each set has commutation relations just like the J's!  The
J's, recall, are infinitesimal rotations, and the Lie algebra they span
is so(3).  So in a sense the Lie algebra of the Lorentz group can
be "split" into "left-handed" and "right-handed" copies of so(3), also
known as "self-dual" and "anti-self-dual" copies.  This is, in fact,
what lies behind the handedness of neutrinos, and many other wonderful
things.  

But let me phrase this result more precisely.  Since we allowed
ourselves complex linear combinations of the J's and K's, we are now
working in the "complexification" of the Lie algebra so(3,1), and we
have shown that this Lie algebra over the complex numbers splits into
two copies of so(3,C), the complexification of so(3).  

Ashtekar came up with some "new variables" for general relativity in the
context of the Hamiltonian approach.  Here we are working in the
Lagrangian approach, where things are simpler because they are
"generally covariant," not requiring a split of spacetime into space and
time.  The Lagrangian approach to the new variables is due to Samuel,
Jacobson and Smolin, and in this approach all they amount to is this:
so(1,3) connection of the Palatini approach, think of the so(1,3) as
sitting inside the complexification thereof, and consider only the
"right-handed" part!  Thus, from an so(1,3) connection, we get a so(3,C)
connection.  The "new variables" are just the tetrad field and this
so(3,C) connection.  

I have tried to keep down the indices but I think I will write down the
Palatini Lagrangian and then the "new variables" Lagrangian, without
explaining exactly what they mean, just to show how amazingly
similar-looking they are.  In the Palatini approach we have a tetrad
field, which now we write in its full glory as e_I^i, and the curvature
of the so(1,3) connection, which now we write as \Omega _{ij}^{IJ}.  The
Lagrangian is then


$$

       e_I^i e_J^j \Omega _{ij}^{IJ}
$$
    

(which we integrate against the usual volume form to get the action).  
In the new variables approach we have a tetrad field again, and we write
the curvature of the so(3,C) connection as F_{ij}^{IJ}.  (This turns out
to be just the "right-handed" part of \Omega _{ij}^{IJ}.)  The Lagrangian
is


\begin{verbatim}

	e_I^i e_J^j F_{ij}^{IJ} !
\end{verbatim}
    

Miraculously, this also gives Einstein's equations.  

What's utterly unclear from what I've said so far is why this helps so
much in trying to quantize gravity.  I may eventually get around to
writing about that, but for now, read the book!

3)  We are not stuck with gluing, by David Yetter and Louis Crane, preprint
available as <A HREF = "http://xxx.lanl.gov/abs/hep-th/9302118">hep-th/9302118</A> in latex form, 2 pages.

Well, in "<A HREF = "week2.html">week2</A>" I mentioned Crane and Yetter's marvelous construction of
a 4d topological quantum field theory using the representations of
the quantum group SU_q(2) - and in "<A HREF = "week5.html">week5</A>" I mentioned Ocneanu's "proof"
that the resulting 4-manifold invariants were utterly trivial (equal to
1 for all 4-manifolds).  Now Crane and Yetter have replied, saying that
their 4-manifold invariants are not trivial and that Ocneanu interpreted
their paper incorrectly.  I look forward to the conference on quantum
topology in Kansas at the end of May, where the full story will
doubtless come out.

4)  The initial value problem in light of Ashtekar's variables, by 
R. Capovilla, J. Dell and T. Jacobson, preprint available as
<A HREF = "http://xxx.lanl.gov/abs/gr-qc/9302020">gr-qc/9302020</A>, 15 pages.

The advantage of Ashtekar's new variables is that the simplify the form
of the constraint equations one gets in the initial-value problem for
general relativity.  This is true both of the classical and quantum
theories.  Rovelli and Smolin used this to find, for the first time,
lots of states of quantum gravity defined by link invariants.  Here the
above authors are trying to apply the new variables to the \emph{classical}
theory.  

5)  Combinatorial expression for universal Vassiliev link invariant, by
Sergey Piunikhin, preprint available as <A HREF = "http://xxx.lanl.gov/abs/hep-th/9302084">hep-th/9302084</A>

Somebody ought to teach those Russians how to use the word "the" now
that the cold war is over.  Anyway, this paper defines a kind of
universal object for Vassiliev invariants, which is sort of similar to what I
was trying to do in 

Link invariants of finite type and perturbation theory, by John Baez,
Lett. Math. Phys. 26 (1992) 43-51

but more concrete, and (supposedly) simpler than Kontsevich's approach.
My parenthesis simply indicates that I haven't had time to figure out
what's going on here.
\par\noindent\rule{\textwidth}{0.4pt}

% </A>
% </A>
% </A>
