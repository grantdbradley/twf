
% </A>
% </A>
% </A>
\week{August 16, 2000 }


It's a hot summer day here in Riverside, so I just want to have fun. 
Break out the Klein bottles and Platonic solids!  

I still remember the day as a kid when I first made a M&ouml;bius strip, and
saw how it didn't fall apart when cut in half.  I could see it, but I
couldn't quite grok it.  I was fascinated - and more than a little
annoyed when it turned out my dad already knew about it.  

I don't remember exactly when I first saw a Klein bottle, but I loved it
at first sight:


\begin{verbatim}

            A mathematician named Klein
            Thought the M&ouml;bius strip was divine.
            Said he: "If you glue 
            The edges of two
            You'll get a weird bottle like mine!"
\end{verbatim}
    
Recently, when I was trying to explain some stuff about Klein bottles to
my friend Oz on sci.physics.research, I bumped into the website of a
company that sells the things - Acme Klein Bottles.  I couldn't resist
mentioning to the world at large that I'd dearly like one.  And lo and
behold, a regular reader of This Week's Finds took me up on this:
Timothy J. Kordas.  After a few weeks, a handcrafted glass Klein bottle
arrived via United Parcel Service.  It's great! - it sits on my desk
now, gleaming contentedly.  I think everybody should have one.  You can
even buy them sliced in half, exhibiting the M&ouml;bius strip quite clearly:

1) Acme Klein bottles sliced in half, 
<A HREF = "http://www.kleinbottle.com/sliced_klein_bottles.htm">http://www.kleinbottle.com/sliced_klein_bottles.htm</A>

Meanwhile, I've been thinking about the Platonic solids lately, and also
their generalizations to higher dimensions - the so-called "regular
polytopes".  To really learn about regular polytopes, you have to go to
the source: the king of geometry, Harold Scott Macdonald Coxeter.
But for some reason I didn't get around to reading his
books until just recently:

2) H. S. M. Coxeter, Regular Polytopes, 3rd edition, Dover, New York, 1973.

Regular Complex Polytopes, 2nd edition, Cambridge U. Press, Cambridge, 1991.

Now my head is full of neat facts about regular polytopes, so I want
to rattle some off before I forget!   

Let's start in 3 dimensions.  I assume you're friends with the
tetrahedron, cube, octahedron, dodecahedron and icosahedron.  But you
might not know all the nice relationships between them!  

For example, there's a nice way to fit a tetrahedron snugly into a cube:
if you take every other vertex of the cube, you get the vertices of a
tetrahedron.  And of course I mean a \emph{regular} tetrahedron - I'm not
interested in any other kind, here.  There are two ways to do this, and
if you put both these tetrahedra inside the cube, they combine to form a
star-shaped solid called the "stella octangula".  This was
discovered and given its name by Kepler, who was really fond of this
sort of thing.

Here's a picture:

3) Eric Weisstein, stella octangula, 
<A HREF = "http://mathworld.wolfram.com/StellaOctangula.html">http://mathworld.wolfram.com/StellaOctangula.html</A>

You can rotate it by grabbing it with your mouse!

Similarly, there is a nice way to fit a cube in a dodecahedron.  The
dodecahedron has 20 vertices, and we can use 8 of these as the vertices
of a cube.  This becomes obvious once we realize that these points are
the vertices of a dodecahedron:


\begin{verbatim}

                (+-1/G, +-G, 0),
                (+-G, 0, +-1/G),
                (0, +-1/G, +-G),
                (+-1, +-1, +-1),
\end{verbatim}
    
where G = (sqrt(5) + 1)/2 is the golden ratio and we get to pick each of
the plus or minus signs independently.  The points (+-1, +-1, +-1) form
the vertices of a cube.  

By rotating the whole picture, we get some other ways of putting a cube
in a dodecahedron: 5 in all.  Any rotation of the dodecahedron permutes
these 5 cubes, and we get all even permutations of the cubes this way:
this is one nice way to prove that the rotational symmetry group of the
dodecahedron is A_{5} (the group of even permutations of 5 things).

If we put all 5 cubes inside the dodecahedron, we get a fancy shape
that would make a marvelous Christmas tree decoration - I don't know 
what it's called, but you can see a picture of it in Coxeter's "Regular
Polytopes", and also here:

4) Eric Weisstein, cube 5-compound, 
<A HREF = "http://mathworld.wolfram.com/Cube5-Compound.html">http://mathworld.wolfram.com/Cube5-Compound.html</A>

Now let's combine these two tricks.  If we put a tetrahedron in a cube,
and then put the cube in a dodecahedron, we get a way of fitting the
tetrahedron snugly into the dodecahedron!  If we choose one way of doing
this and then rotate the picture to get other ways, we get 5 tetrahedra
in the dodecahedron.  Putting these all together gives a scary-looking
shape:

5) Eric Weisstein, tetrahedron 5-compound, 
<A HREF = "http://mathworld.wolfram.com/Tetrahedron5-Compound.html">http://mathworld.wolfram.com/Tetrahedron5-Compound.html</A>

but the coolest thing about this shape is that it has an inherent
handedness - like a sugar molecule, it comes in "levo" and
"dextro" forms!  If we reflect it, we get 5 \emph{other} ways to
put a tetrahedron into a dodecahedron, for a total of 10.  All of these
tetrahedra taken together form a mirror-symmetric shape:

6) Eric Weisstein, tetrahedron 10-compound, 
<A HREF = "http://mathworld.wolfram.com/Tetrahedron10-Compound.html">http://mathworld.wolfram.com/Tetrahedron10-Compound.html</A>

Okay.  So far we've related the tetrahedron, the cube and the dodecahedron.  
What about the other two Platonic solids: the octahedron and icosahedron?  
Well, from the point of view of \emph{symmetry groups} these guys are
redundant.  The octahedron is dual to the cube, so it has the same
rotational symmmetry group. Similarly, the icosahedron is dual to the
dodecahedron and has the same symmetry group.

From the group-theoretic viewpoint, here's what's really going on.  Our
trick for fitting the tetrahedron in the cube lets us turn any symmetry
of the tetrahedron into a symmetry of the cube.  The rotational symmetry
group of the tetrahedron is A_{4} - that is, all even
permutations of the 4 vertices.  The symmetry group of the cube is
S_{4} - that is, all permutations of the 4 lines connecting
opposite vertices.  So what we've got is a trick for making
A_{4} into a subgroup of S_{4}. 

(This immediately leads to a little puzzle.  There's an \emph{obvious} way
to find A_{4} as a subgroup of S_{4}, since even 
permutations are a special
case of permutations.   So: does the above trick give this obvious way,
or some other way?)

Anyway, it's also true that any way of fitting the tetrahedron
in the dodecahedron lets us turn any symmetry
of the tetrahedron into a symmetry of the dodecahedron.  
So we've also got a trick for making A_{4} into a subgroup of 
A_{5}.

(You might also think that our trick for fitting the cube in
the dodecahedron gives a way to turn any symmetry of the
cube into a symmetry of the dodecahedron.   I thought this for a while,
but it's not true!  For starters, if it \emph{were} true, we'd get
a trick for making S_{4} into a subgroup of A_{5} -
which is impossible, since the order of the group S_{4} 
doesn't divide that of A_{5}.  And the problem turns out
to be this: a 90 degree rotation of the cube does not correspond
to a symmetry of the dodecahedron.)



% parser failed at source line 205
