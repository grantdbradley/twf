
% </A>
% </A>
% </A>
\week{September 19, 1999 }


Last time I described some of the talks at James Hartle's 60th birthday 
celebration at the Newton Institute.  But I also met some people at that 
party that I'd been wanting to talk to.  There's a long story behind this, 
so if you don't mind, I'll start at the beginning....

A while ago, Phillip Helbig, one of the two moderators of 
sci.physics.research who do astrophysics, drew my attention to an 
interesting paper: 

1) Vipul Periwal, Cosmological and astrophysical tests of 
quantum gravity, preprint available at <A HREF = "http://xxx.lanl.gov/abs/astro-ph/9906253">astro-ph/9906253</A>

The basic idea behind this is that quantum gravity effects could cause 
deviations from Newton's inverse square law at large distance scales, and 
that these deviations might explain various puzzles in astrophysics, like 
the "missing mass problem" and the possibly accelerating expansion of the 
universe.  

This would be great, because it might not only help us understand these 
astrophysics puzzles, but also help solve the big problem with quantum 
gravity, namely the shortage of relevant experimental data.

But of course one needs to read the fine print before getting too excited 
about ideas like this!

Following the argument in Periwal's paper requires some familiarity 
with the renormalization group, since that's what people use to study 
how "constants" like the charge of the electron or Newton's gravitational 
constant depend on the distance scale at which you measure them - due 
to quantum effects.  Reading the paper, I immediately became frustrated 
with my poor understanding of the renormalization group.  It's really 
important, so I decided to read more about it and explain it in the 
simplest possible terms on sci.physics.research - since to understand 
stuff, I like to try to explain it.  

In the process, I found this book very helpful:

1) Michael E. Peskin and Daniel V. Schroeder, An Introduction 
to Quantum Field Theory, Addison-Wesley, Reading, Massachusetts 
1995. 

The books I'd originally learned quantum field theory from didn't 
incorporate the modern attitude towards renormalization, due to 
Kenneth Wilson - the idea that quantum field theory may not 
ultimately be true at very short distance scales, but that's
okay, because if we assume it's a good approximation at pretty 
short distance scales, it becomes a \emph{better} approximation at 
\emph{larger} distance scales.  This is especially important when you're 
thinking about quantum gravity, where godawful strange stuff may be 
happening at the Planck length.  Peskin and Schroeder explain this 
idea quite well.  For my own sketchy summary, try this:

2) John Baez, Renormalization made easy, <A HREF = "renormalization.html">
http://math.ucr.edu/home/baez/renormalization.html</A>

I deliberately left out as much math as possible, to concentrate 
on the basic intuition.

Thus fortified, I returned to Periwal's paper, and it made a bit 
more sense.  Let me describe the main idea: how we might expect 
Newton's gravitational constant to change with distance.  

So, suppose we have any old quantum field theory with a coupling constant
G in it.  In fact, G will depend on the length scale at which we
measure it.  But using Planck's constant and the speed of light we
can translate length into 1/momentum.  This allows us to think of G
as a function of momentum.   Roughly speaking, when you shoot particles 
at each other at higher momenta, they come closer together before 
bouncing off, so measuring a coupling constant at a higher momentum 
amounts to measuring at a shorter distance scale.  

The equation describing how G depends on the momentum p is called 
the "Callan-Symanzik equation".  In general it looks like this:


\begin{verbatim}

   dG
 -------  = \beta (G)
 d(ln p)
\end{verbatim}
    
But all the fun starts when we use our quantum field theory to calculate 
the right hand side, which is called - surprise! - the "beta function" 
of our theory.   Typically we get something like this:


\begin{verbatim}

   dG 
 -------  = (n - d)G + aG^{2} + bG^{3} + ....
 d(ln p)
\end{verbatim}
    
Here n is the dimension of spacetime and d is a number called the
"upper critical dimension".  You see, it's fun when possible to think
of our quantum field theory as defined in a spacetime of arbitrary
dimension, and then specialize to the case at hand.  I'll show you 
how work out d in a minute.  It's harder to work out the numbers 
a, b, and so on - for this, you need to do some computations using the 
quantum field theory in question.

What does the Callan-Symanzik equation really mean?  Well, for starters 
let's neglect the higher-order terms and suppose that


\begin{verbatim}

  dG(p) 
 -------  = (n - d)G
 d(ln p)
\end{verbatim}
    
This says G is proportional to p^{n-d}.  There are 3 cases:


A) When n < d, our coupling constant gets \emph{smaller} at higher
momentum scales, and we say our theory is
"superrenormalizable".  Roughly, this means that at larger and
larger momentum scales, our theory looks more and more like a "free
field theory" - one where particles don't interact at all.  This
makes superrenormalizable theories easy to study by treating them as a
free field theory plus a small perturbation.


 B) When n > d, our coupling constant gets \emph{larger} at
higher momentum scales, and we say our theory is
"nonrenormalizable".  Such theories are hard to study using
perturbative calculations in free field theory.

C)  When n = d, we are right on the brink between the two cases above.
We say our theory is "renormalizable", but we really have to work out
the next term in the beta function to see if the coupling constant
grows or shrinks with increasing momentum.

Consider the example of general relativity.  We can figure out
the upper critical dimension using a bit of dimensional analysis
and handwaving.  Let's work in units where Planck's constant and the 
speed of light are 1.  The Lagrangian is the Ricci scalar curvature 
divided by 8 \pi  G, where G is Newton's gravitational constant.  We 
need to get something dimensionless when we integrate the Lagrangian 
over spacetime to get the action, since we exponentiate the action 
when doing path integrals in quantum field theory.  Curvature has 
dimensions of 1/length^{2}, so when spacetime has dimension n, G must 
have dimensions of length^{n-2}.  

This means that if you are a tiny little person with a ruler X 
times smaller than mine, Newton's constant will seem X^{n-2} times 
bigger to you.  But measuring Newton's constant at a length scale 
that's X times smaller is the same as measuring it at a momentum scale 
that's X times bigger.  We already solved the Callan-Symanzik equation 
and saw that when we measure G at the momentum scale p, we get an
answer proportional to p^{n-d}.  We thus conclude that d = 2.  

(If you're a physicist, you might enjoy finding the holes in the
above argument, and then plugging them.)  

This means that quantum gravity is nonrenormalizable in 4 dimensions.
Apparently gravity just keeps looking stronger and stronger at 
shorter and shorter distance scales.  That's why quantum gravity has 
traditionally been regarded as hard - verging on hopeless.  

However, there is a subtlety.  We've been ignoring the higher-order
terms in the beta function, and we really shouldn't!

This is obvious for renormalizable theories, since when n = d, the 
beta function looks like


\begin{verbatim}

   dG 
 -------  = aG^{2} + bG^{3} + ....
 d(ln p)
\end{verbatim}
    
so if we ignore the higher-order terms, we are ignoring the whole 
right-hand side!  To see the effect of these higher-order terms let's 
just consider the simple case where


\begin{verbatim}

   dG 
 -------  = aG^{2}
 d(ln p)
\end{verbatim}
    
If you solve this you get


\begin{verbatim}

             c 
   G = -------------
        1 - ac ln p 
\end{verbatim}
    
where c is a positive constant.   What does this mean?  Well, if a < 0, 
it's obvious even before solving the equation that G slowly \emph{decreases} 
with increasing momentum.  In this case we say our theory is 
"asymptotically free".   For example, this is true for the strong 
force in the Standard Model, so in collisions at high momentum quarks 
and gluons act a lot like free particles.  (For more on this, try "<A HREF = "week94.html">week94</A>".)

On the other hand, if a > 0, the coupling constant G
\emph{increases} with increasing momentum.  To make matters worse, it
becomes INFINITE at a sufficiently high momentum!  In this case we say
our theory has a "Landau pole", and we cluck our tongues
disapprovingly, because it's not a good thing.  For example, this is
what happens in quantum electrodynamics when we don't include the weak
force.  Of course, one should really consider the effect of even
higher-order terms in the beta function before jumping to conclusions.
However, particle physicists generally feel that among renormalizable
field theories, the ones with a < 0 are good, and the ones with a
> 0 are bad.

Okay, now for the really fun part.  Perturbative quantum gravity
in 2 dimensions is not only renormalizable (because this is the 
upper critical dimension), it's also asympotically free!  Thus
in n dimensions, we have


\begin{verbatim}

   dG 
 -------  = (n - 2)G + aG^{2} + ....
 d(ln p)
\end{verbatim}
    
where a < 0.  If we ignore the higher-order terms which I have
written as "....", this implies something very interesting for
quantum gravity in 4 dimensions.  Suppose that at low momenta
G is small.  Then the right-hand side is dominated by the first
term, which is positive.  This means that as we crank up the 
momentum scale, G keeps getting bigger.  This is what we already 
saw about nonrenormalizable theories.  But after a while, when G 
gets big, the second term starts mattering more - and it's negative.
So the growth of G starts slowing!

In fact, it's easy to see that as we keep cranking up the momentum,
G will approach the value for which 


\begin{verbatim}

   dG 
 -------  = 0
 d(ln p)
\end{verbatim}
    
We call this value an "ultraviolet stable 
fixed point" for the gravitational
constant.  Mathematically, what we've got is a flow in the space 
of coupling constants, and an ultraviolet stable fixed point is one
that attracts nearby points as we flow in the direction of higher 
momenta.  This particular kind of ultraviolet stable fixed point - coming 
from an asymptotically free theory in dimensions above its upper 
critical dimension - is called a "Wilson-Fisher fixed point".  

So: perhaps quantum gravity is saved from an ever-growing Newton's
constant at small distance scales by a Wilson-Fisher fixed point!
But before we break out the champagne, note that we neglected the
higher-order terms in the beta function in our last bit of reasoning.
They can still screw things up.  For example, if


\begin{verbatim}

   dG 
 -------  = (n - 2)G + aG^{2} + bG^{3} 
 d(ln p)
\end{verbatim}
    
and b is positive, there will not be a Wilson-Fisher fixed point
when the dimension n gets too far above 2.  Is 4 too far above 2?
Nobody knows for sure.  We can't really work out the beta function
exactly.  So, as usual in quantum gravity, things are a bit iffy.

However, Periwal cites the following paper as giving numerical
evidence for a Wilson-Fisher fixed point in quantum gravity:

3) Herbert W. Hamber and Ruth M. Williams, Newtonian potential in 
quantum Regge gravity, Nucl. Phys. B435 (1995), 361-397.  

And he draws some startling conclusions from the existence of
this fixed point.  He says it should have consequences for the
missing mass problem and the value of the cosmological constant!
However, I found it hard to follow his reasoning, so I decided
to track down some of the references - starting with the above
paper.  

Now, Ruth Williams works at Cambridge University, so I was not
surprised to find her at Hartle's party.  She was busy talking
to John Barrett, who also does quantum gravity, up at Nottingham
University.  I arranged to stop by her office, get a copy of her 
paper, and have her explain it to me.  I also arranged to visit
John in Nottingham and have him explain his work with Louis Crane 
on Lorentzian spin foam models - but more about that next week!

Anyway, here's how the Hamber-Williams paper goes, very roughly.  
They simulate quantum gravity by chopping up a 4-dimensional torus 
into 16 x 16 x 16 x 16 hypercubes, chopping each hypercube into 24 
4-simplices in the obvious way, and then doing a Monte Carlo calculation 
of the path integral using the Regge calculus, which is a discretized 
version of general relativity suited to triangulated manifolds (see 
"<A HREF = "week119.html">week119</A>" for details).  Their goal 
was to work out how Newton's 
constant varies with distance.  They did it by calculating correlations
between Wilson loops that wrap around the torus.  They explain how
you can deduce Newton's constant from this information, but I don't
have the energy to describe that here.  Anyway, they claim that Newton's
constant varies with distance as one would expect if there was a 
Wilson-Fisher fixed point!

(It's actually more complicated that this because besides Newton's
constant, there is also another coupling constant in their theory:
the cosmological constant.  And of course this is very important
for potential applications to astrophysics.)

Unfortunately, I'm still mystified about a large number of things.  
Let me just mention two.  First, Hamber and Williams consider values
of G which are \emph{greater} than the Wilson-Fisher fixed point.  Since
this is an ultraviolet stable fixed point, such values of G flow \emph{down}
to the fixed point as we crank up the momentum scale.  Or in other 
words, in this regime Newton's constant gets \emph{bigger} with increasing
distances.  At least to my naive brain, this sounds nice for explaining
the missing mass problem.  But the funny thing is, this regime is 
utterly different from the regime where G is close to zero - namely,
\emph{less} than the Wilson-Fisher fixed point.  I thought all the usual
perturbative quantum gravity calculations were based on the assumption
that at macroscopic distance scales G is small, and flows </em>up</em> to 
the fixed point as we crank up the momentum scale!  Are these folks 
claiming this picture is completely wrong?  I'm confused.  

Another puzzle is that Periwal thinks Newton's constant will start
to grow at distance scales roughly comparable to the radius of the
universe (or more precisely, the Hubble length).  But it looks like
Hamber and Williams say their formula for G as a function of momentum
holds at \emph{short} distance scales.  

I guess I need to read more stuff, starting perhaps with Weinberg's 
old paper on quantum gravity and the renormalization group:

4) Steven Weinberg, Ultraviolet divergences in quantum theories of
gravitation, in General Relativity: an Einstein Centenary Survey, 
eds. Stephen Hawking and Werner Israel, Cambridge U. Press, Cambridge
(1979).

and then perhaps turning to his paper on the cosmological constant:

5) Steven Weinberg, The cosmological constant problem, Rev. Mod. Phys.
61 (1989), 1.  

and some books on the renormalization group and quantum gravity:

6) Claude Itzykson and Jean-Michel Drouffe, Statistical Field Theory,
2 volumes, Cambridge U. Press, 1989.

7) Jean Zinn-Justin, Quantum Field Theory and Critical Phenomena,
Oxford U. Press, Oxford, 1993.

8) Jan Ambjorn, Bergfinnur Durhuus, and Thordur Jonsson, Quantum 
Geometry: A Statistical Field Theory Approach, Cambridge U. Press, 
Cambridge, 1997.

I should also think more about this recent paper, which claims to
find a phase transition in a toy model of quantum gravity where
one does the path integral over a special class of metrics - namely 
those with 2 Killing vector fields.

9) Viqar Husain and Sebastian Jaimungal, Phase transition in 
quantum gravity, preprint available as <A HREF = "http://xxx.lanl.gov/abs/gr-qc/9908056">gr-qc/9908056</A>.

But if anyone can help me clear up these issues, please let me know!

Okay, enough of that.  Another person I met at the party was Roger
Penrose!  Later I visited him in Oxford.  Though recently retired,
he still holds monthly meetings at his house in the country, attended
by a bunch of young mathematicians and physicists.  At the one I went
to, the discussion centered around Penrose's forthcoming book.  The goal 
of this book is to explain modern physics to people who know only a 
little math, but are willing to learn more.  A nice thing about it is
that it treats various modern physics fads without the uncritical 
adulation that mars many popularizations.  In particular, when I 
visited, he was busy writing a chapter on inflationary cosmology, 
so he talked about a bunch of problems with that theory, and cosmology
in general.  

I've never been sold on inflation, since it relies on fairly speculative
aspects of grand unified theories (or GUTs), so most of these problems merely 
amused me.  Theorists take a certain wicked glee in seeing someone else's 
theory in trouble.  However, one of these problems concerned the Standard 
Model, and this hit closer to home.  Penrose made the standard observation 
that the most distant visible galaxies in opposite directions have not had 
time to exchange information - at least not since the time of recombination, 
when the initial hot fireball cooled down enough to become transparent.  
But if the symmetry between the electromagnetic and weak forces is 
spontaneously broken only when the Higgs field cools down enough to line 
up, as the Standard Model suggests, this raises the danger that the Higgs 
field could wind up pointing in different directions in different patches 
of the visible universe! - since these different "domains" would not yet 
have had time to expand to the point where a single one fills the whole 
visible universe.  But we don't see such domains - or more precisely, we 
don't see the "domain walls" one would expect at their boundaries.  

Of course, inflation is an attempt to deal with similar problems, but 
inflation is posited to happen at GUT scale energies, too soon (it seems)
to solve \emph{this} problem, which happens when things cool down to the 
point where the electroweak symmetry breaks.  

Again, if anyone knows anything about this, I'd love to hear about it.


 \par\noindent\rule{\textwidth}{0.4pt}

% </A>
% </A>
% </A>
