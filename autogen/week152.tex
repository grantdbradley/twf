
% </A>
% </A>
% </A>
\week{July 5, 2000}


I've been reading about the mathematical physicist William Rowan 
Hamilton lately, because I'm writing a review article about the 
octonions - that famous nonassociative 8-dimensional division 
algebra.  

You see, the day after Hamilton discovered the quaternions and carved 
the crucial formula


$$

                i^{2} = j^{2} = k^{2} = ijk = -1
$$
    
on the Brougham bridge, he mailed a letter explaining his discovery
to his friend John Graves.  And about two months later, Graves 
discovered the octonions!  In December 1843, he sent a letter about 
them to Hamilton.   

Graves called them "octaves" at first, but later introduced
the term "octonions".  He showed they were a normed division
algebra and used this to prove the 8 squares theorem, which says that
the product of two sums of 8 perfect squares is again a sum of 8 perfect
squares.  The complex numbers and quaternions allow one to prove similar
theorems for 2 and 4 squares.  In January 1844, Graves considered the
idea of a general theory of "2^{m}-ions".  He tried 
to construct
a 16-dimensional normed division algebra and use it to prove a 16
squares theorem, but he "met with an unexpected hitch" and came to
doubt that this was possible.  

(If you read "<A HREF = "week59.html">week59</A>" you'll see why.)

Hamilton was the one who noticed that the octonions were nonassociative - in 
fact, he invented the word "associative" right about this time.  He 
offered to write a paper publicizing Graves' work, and Graves accepted 
the offer, but Hamilton kept putting it off.  He was probably busy 
working on the quaternions!  

Meanwhile, Arthur Cayley had heard about the quaternions right when 
Hamilton announced his discovery, and he eventually discovered the 
octonions on his own.  He published a description of them in the March 
1845 issue of the Philosophical Magazine.  Graves was upset, so he added 
a postscript about the octonions to a paper of his that was due to 
appear in the following issue of the same journal, asserting that 
he'd known about them since Christmas 1843.  Also, Hamilton eventually 
got his act together and published a short note about Graves' discovery 
in the June 1847 issue of the Proceedings of the Royal Irish Academy.  
But by then it was too late - everyone was calling the octonions "Cayley 
numbers".

Of course it wasn't \emph{really} too late, since everybody who cares 
can now tell that Graves was the first to discover the octonions.  
And anyway, it doesn't really make a difference who discovered them 
first, except as a matter of historical interest.  

But just for the heck of it, I'm trying to find out everything I can 
about the early history of the octonions.  Hamilton is very famous, and 
much has been written about him, but Graves is mainly famous for being 
Hamilton's friend - so to learn stuff about Graves, I have to read books 
on Hamilton.   In the process, I've learned some interesting things that 
aren't really relevant to my review article.  And I want to tell you about 
some of them before I forget! 

Hamilton was a strangely dreamy sort of guy.  He spent most of his
life as the head of a small observatory near Dublin, but quickly lost
interest in actually staying up nights to make observations.  Instead,
he preferred writing poetry.  He was friends with Coleridge, who 
introduced him to the philosophy of Kant, which influenced him greatly.
He was also friends with Wordsworth - who told him to not to write poetry.  
He fell deeply in love with a woman named Catherine Disney, who was forced 
by her parents to marry a wealthy man 15 years older than her.  Hamilton
remained hopelessly in love with her the rest of his life, though he 
eventually married someone else.  He became an alcoholic, then foreswore 
drink, then relapsed.  Eventually, many years later, Catherine began a 
secret correspondence with him - she still loved him!  Her husband became 
suspicious, she attempted suicide by taking laudanum... and then, five 
years later, she became ill.   Hamilton visited her and gave her a copy of 
his "Lectures on Quaternions" - they kissed at long last - and 
then she died two weeks later.  He carried her picture with him ever 
afterwards and talked about her to anyone who would listen.   A very 
sad and very Victorian tale.  

He was a bit too far ahead of his time to have maximum impact during his 
own life.  The Hamiltonian approach to mechanics and the Hamilton-Jacobi 
equation relating waves and particles became really important only when 
quantum mechanics came along.  Luckily Klein liked this stuff, and told 
Schroedinger about it.  But it's a pity that Hamilton's unification of 
particle and wave mechanics came along right when the advocates of the
wave theory of light seemed to have definitively won the battle against 
the particle theory - the need for a compromise became clear only later.  

Quaternions, too, might have had more impact if they'd come along later,
when people were trying to understand spin-1/2 particles.  After all, 
the unit quaternions form the group SU(2), which is perfect for studying
spin-1/2 particles.   But the way things actually went, quaternions were 
not very popular by the time people dreamt of spin-1/2 particles - so 
Pauli just used 2 x 2 complex matrices to describe the generators of SU(2).   

I like what Hamilton wrote about quaternions, space, and time:

\begin{quote}
    The quaternion was born, as a curious offspring of a quaternion
    of parents, say of geometry, algebra, metaphysics, and poetry... 
    I have never been able to give a clearer statement of their nature
    and their aim than I have done in two lines of a sonnet addressed
    to Sir John Herschel: 
\begin{quote}

$$

           "And how the One of Time, of Space the Three
            Might in the Chain of Symbols girdled be."
$$
    
\end{quote}
\end{quote}

It's also amusing how Hamilton responded when de Morgan told him about
the four-color conjecture: "I am not likely to attempt your
`quaternion of colours' very soon." The pun is ironic, given the
relations people have recently discovered between what is now the
four-color theorem, the vector cross product, and the group SU(2).  (See
"<A HREF = "week8.html">week8</A>" and "<A HREF =
"week92.html">week92</A>" for more.)

Of course quaternions were very influential for a while - they were taught 
in many mathematics departments in America in the late 1800s, and were 
even a mandatory topic of study at Dublin!  But then they were driven 
out by the vector notation of Gibbs and Heaviside.  If you don't know 
this story, you've got to read this book - it's fascinating:

1) Michael J. Crowe, A History of Vector Analysis, University of Notre 
Dame Press, Notre Dame, 1967.

Check out the graphs showing how many books were written on quaternions:
the big boom in the 1860s, and then the bust!

I hadn't even known about what many people at the time considered
Hamilton's greatest achievement: the prediction of "conical
refraction" by a biaxial crystal like aragonite.  Folks compared
this to the discovery of Neptune by Adams and Leverrier - another
triumph of prediction - and Hamilton won a knighthood for it.

Does anyone understand how this phenomenon works?  I don't.

Personally, I think one of Hamilton's greatest triumphs was his
treatment of complex numbers as pairs of real numbers - this finally
exorcised the long-standing fears about whether imaginary numbers 
"really exist", and helped opened up the way for other algebras.
Interestingly, the person who got him interested in this problem was
John Graves.  Graves was the one who introduced Hamilton to John
Warren's book "A Treatise on the Geometrical Representation of the
Square Root of Negative Quantities", which explained the concept of
the complex plane.  Hamilton turned this from geometry into algebra.

One of Hamilton's last inventions was the icosian calculus.  Faithful
fans of This Week's Finds will remember the icosians from "<A HREF
= "week20.html">week20</A>".  These were invented by Conway and
Sloane; Hamilton's original icosian calculus was a bit different.  In
August 1856, Hamilton went to the British Association Meeting at
Cheltenham and stayed at the house of his pal John Graves.  He enjoyed
talking to Graves and reading his books: "Conceive me shut up and
revelling for a fortnight in John Graves' Paradise of Books! of which he
has really an astonishingly extensive collection, especially in the
curious and mathematical kinds.  Such new works from the Continent he
has picked up! and such rare old ones too!" Graves posed some
puzzles to Hamilton, and either Graves or his books got Hamilton to
thinking about regular polyhedra.  When Hamilton returned to Dublin he
thought about the symmetry group of the icosahedron, and used it to
invent an algebra he called the "icosians".  He then sent a
letter to Graves explaining the icosians.

He basically said: assume we've got three symbols I, K, and L 
satisfying these relations:


$$

        I^{2} = 1,   K^{3} = 1,   L^{5} = 1,   L = IK 
$$
    
together with the associative law but not the commutative law.  
You can think of L as corresponding to rotating an icosahedron
1/5 of a turn around a vertex.  K corresponds to rotating it
1/3 of a turn around a face, and I corresponds to rotating it
1/2 of a turn around an edge.  The relations above all follow
from this idea.   

These days, we would call the icosians the "group algebra of 
A_{5}".
In modern lingo, the symmetry group of the icosahedron is called
A_{5}, since it's the group of all even permutations of 5 things.  
If you don't know why this is true, check this out:

2) John Baez, Some thoughts on the number six, 
<A HREF = "http://math.ucr.edu/home/baez/six.html">http://math.ucr.edu/home/baez/six.html</A> 

We form the "group algebra" of a group by taking all formal linear 
combinations of group elements with real coefficients, and defining 
a product of such combinations using the product in the group.  The 
dimension of a group algebra is just the number of elements in the 
group.  Since A_{5} has 60 elements, the icosians are a 60-dimensional 
algebra.  These days this stuff is no big deal.  But back then, I bet
a 60-dimensional noncommutative algebra was really mindblowing!

In a way that I don't fully understand, Hamilton connected the icosian
calculus to the problem of travelling along the edges of a dodecahedron,
hitting each vertex just once, and coming back to where you started.  In
graph theory, this sort of thing is now called a "Hamiltonian
circuit".  Hamilton even invented a puzzle where the first player
takes the first five steps any way they want, and the other player has
to complete the Hamiltonian circuit.  He called this the "icosian
game".

It was John Graves' idea to actually design a game board with the 
dodecahedron graph drawn on it and holes at the vertices that you
could put small cylindrical markers into.   In 1859, a friend of Graves 
manufactured a version where the game board had legs like a small table,
and sent a copy to Hamilton.  Naturally Hamilton was delighted!  Graves
put Hamilton in contact with a London toymaker named John Jacques, and
Hamilton sold Jacques the rights to the game for 25 pounds and 6 copies.
Jacques marketed two versions, one for the parlor, which was played on
a flat board, and another for the "traveler", which was played on an
actual dodecahedron - there was a nail at each vertex, and the players
wound string about these nails as they traced out their Hamiltonian 
circuit.  

With charming naivete, Hamilton had hopes that the game would sell wildly.
Alas, it did not.  Jacques never even recouped his investment.  The problem 
was that the icosian game was too easy, even for children!  Amusingly, 
Hamilton had more trouble with it than most people - perhaps because he 
was using the icosian calculus to figure out his moves, instead of just 
trying different paths.  

By the way, if anyone knows any good source of information about 
Graves or the invention of the octonions, I'd love to hear about it.  
So far I've gotten most of my stuff from the following sources.  
First of all, there's this nice biography of Hamilton:

3) Thomas L. Hankins, Sir William Rowan Hamilton, John Hopkins 
University Press, Baltimore, 1980.

Check out the picture of the icosian game on page 342!

Then there's this much longer biography, which includes lots
of correspondence:

4) Robert Perceval Graves, Life of Sir William Rowan Hamilton, 
3 volumes, Arno Press, New York 1975.

Robert Perceval Graves was the brother of John Graves!  He idolized 
Hamilton, so this is not the most balanced account of his work.
    
Then there is this very helpful summary of the Hamilton-Graves 
correspondence on octonions:

5) W. R. Hamilton, Four and eight square theorems, Appendix 3 of 
vol. III of The Mathematical Papers of William Rowan Hamilton, eds. 
H. Halberstam and R. E. Ingram, Cambridge University Press, Cambridge,
1967.

Unfortunately this does not include Graves' first letter to 
Hamilton about the octonions.  Is it lost?

Finally, there's this history of later work on the octonions and the
eight square theorem:

6) L. E. Dickson, On quaternions and their generalization and the history 
of the eight square theorem, Ann. Math. 20 (1919), 155-171.

It turns out the eight square theorem was proved in 1822, before
Graves.  Also, there's some good material in here:

7) Heinz-Dieter Ebbinghaus et al, Numbers, Springer, New York, 1990.

This book is a lot of fun for anyone interested in all sorts of
"numbers". 

Finally, for an excellent \emph{online} source of information on
the history of quaternions, octonions, and other "hypercomplex
number systems", this is the place to go:

8) Jeff Biggus, A history of hypercomplex numbers, 
<A HREF = "http://history.hyperjeff.net/hypercomplex.html">http://history.hyperjeff.net/hypercomplex.html</A>


\par\noindent\rule{\textwidth}{0.4pt}
<B>Addendum:</B>


On April 14th, 2005 I received the following email from 
Geoff Corbishley in response to my
plea for more information about John Graves:

\begin{quote}
John

Your page asks for information on John Graves. I am reading <em>Goodbye to 


% parser failed at source line 364
