
% </A>
% </A>
% </A>
\week{November 28, 2008 }


Today I want to talk about the Enceladus plumes, the Io flux tube,
special properties of the number 6 - and the wonders of standard Borel
spaces, commutative von Neumann algebras and Polish groups.

Let's start by leaving the fire and brimstone of Io.  Let's sail
out to the icy splendor of Saturn's moon Enceladus, and gaze at its
geysers:

<div align = "center">
<img border = "2" src = "enceladus_geysers.jpg">
</div>

1) NASA Photojournal, Jet blue,
<a href = "http://photojournal.jpl.nasa.gov/catalog/PIA08386">http://photojournal.jpl.nasa.gov/catalog/PIA08386</a>

We've seen Enceladus once before, in "<a href =
"week231.html">week231</A>".  I showed you a photo of the icy
geysers near its south pole, taken by the Cassini probe back in 2005.
Above is a prettier version of the same picture, released this August
to celebrate Cassini's return to Enceladus.  You can see the geysers
shooting plumes of ice into outer space!

You see, this year NASA has been using the Cassini probe to take a
better look at large cracks on the icy surface of Enceladus. 
They're called "tiger stripes", and this where the geysers are.  
On August 11th, the Cassini probe shot past Enceladus at a distance of
just 50 kilometers.  It took some wonderful photos as it approached:

<div align = "center">
<img border = "2" src = "enceladus_tiger_stripes.jpg">
</div>

2) NASA Photojournal, Great southern land, 
<a href = "http://photojournal.jpl.nasa.gov/catalog/PIA11112">http://photojournal.jpl.nasa.gov/catalog/PIA11112</a>

Here you see the four main tiger stripes on Enceladus.  They're the
big greenish cracks near the south pole, running from left to right.
For some reason these tiger stripes are named after cities mentioned
in that wonderful book of tales called One Thousand and One Nights.
From top to bottom: Damascus Sulcus, Baghdad Sulcus, Cairo Sulcus and
Alexandria Sulcus.  "Sulcus" is Latin for a depression or
fissure.  It's often used as a medical term, especially for features
of the brain - but on Enceladus it just means "tiger stripe".

These tiger stripes are about 2 kilometers wide and 500 meters deep.
They're flanked by ridges that are about 2-4 kilometers wide and 
100 meters tall.  The Damascus and Baghdad Sulci are the most active
ones, as far as geysers go:

<div align = "center">
<img border = "2" src = "enceladus_baghdad_damascus.jpg">
</div>

Alexandria Sulcus is the least active.  

The white stuff in this false-color image is fine-grained ice.  The
green stuff is bigger chunks of ice - even house-sized boulders.
These bigger chunks are concentrated along valley floors and walls, 
and along the upraised flanks of the tiger stripes.

At its closest approach, Cassini took much higher resolution
photographs of the tiger stripes, like this closeup of the big fork in
Damascus Sulcus:

<div align = "center">
<img width = "500" border = "2" src = "enceladus_damascus_sulcus.jpg">
</div>

3) NASA Photojournal, Damascus Sulcus on Enceladus,
<a href = "http://photojournal.jpl.nasa.gov/catalog/PIA11113">http://photojournal.jpl.nasa.gov/catalog/PIA11113</a>

The hard part was compensating for the fact that the spacecraft was
whizzing by the moon.  The NASA team used a clever maneuver they
called a "skeet shoot" to tackle this problem:

\begin{quote}
     On Earth, skeet shooting is an outdoor shotgun sport that           
     simulates shooting game birds in flight.  A small Frisbee-like 
     ceramic disk, called a clay pigeon, is launched through the air, 
     usually diagonally across in front of a shooter armed with a 
     shotgun.  The skill in successfully hitting the moving clay      
     target with the birdshot is for the shooter to point a little     
     ahead of the clay pigeon, and match its angular velocity when 
     the trigger is pulled.  The clay pigeon then passes into the 
     bird shot at exactly when the shot arrives at its destination 
     in the path of the moving target.  So far so good.  At and just
     after closest-approach on the Enceladus 4 flyby, relative to
     Cassini and Optical Remote Sensing boresight directions,
     Enceladus was streaking too quickly across the sky for the      
     spacecraft to be able to stably target and track any geological
     feature on the surface.  Borrowing from the firearms sport, the
     trick was to turn the spacecraft as fast as possible in the same
     direction as Enceladus' path across the sky.  The plan was to be
     leading Enceladus and match its angular velocity at the exact
     times when our targets of interest passed into our camera's 
     field of view.
\end{quote}

It worked amazingly well.  They got some photos that could see 
features as small as 10 meters!  It's great fun to read blogs of
the scientists involved, and watch a video of the maneuver:

4) NASA Blogs, Enceladus - Aug08,
<a href = "http://blogs.nasa.gov/cm/blog/cassini-aug08/">http://blogs.nasa.gov/cm/blog/cassini-aug08/</a>

During this flyby, an infrared spectrometer on the Cassini probe 
saw that the tiger stripes are significantly warmer than their 
surroundings.  Here's a temperature map with approximate locations
of active geysers.  From left to right you can see Damascus Sulcus, 
Baghdad Sulcus, Cairo Sulcus and Alexandria Sulcus:

<div align = "center">
<img width = "400" border = "2" src = "enceladus_jet_spots.jpg">
</div>

5) NASA Photojournal, Jet spots in tiger stripes,
<a href = "http://photojournal.jpl.nasa.gov/catalog/PIA10361">http://photojournal.jpl.nasa.gov/catalog/PIA10361</a>

Other instruments detected a plume of water vapor, ice,
methane, carbon dioxide, nitrogen, and more complicated
organic compounds.    All this is good evidence for
"cryovolcanism": a version of volcanic activity that can happen on 
really cold worlds, with water playing the role of lava.  
Cryovolcanism has also been seen on other moons of outer planets: 
Triton, Europa, Titan, Ganymede and Miranda.

Cassini made two more flybys in October, and another in November:

6) NASA Blogs, Enceladus,
<a href = "http://blogs.nasa.gov/cm/blog/enceladus">http://blogs.nasa.gov/cm/blog/enceladus</a>

And just this week, Candice Hansen and her coauthors came out with 
a new paper about the geysers of Enceladus:

7) C. J. Hansen, L. W. Esposito, A. I. F. Stewart, B. Meinke, B. 
Wallis, J. E. Colwell, A. R. Hendrix, K. Larsen, W. Pryor and F. Tian, 
Water vapour jets inside the plume of gas leaving Enceladus, Nature 
456 (27 November 2008), 477-479.

8) NASA, Enceladus Jets: Are They Wet or Just Wild?,
<a href = "http://www.nasa.gov/mission_pages/cassini/whycassini/cassinif-20081126.html">http://www.nasa.gov/mission_pages/cassini/whycassini/cassinif-20081126.html</a>

They analyzed data from the ultraviolet spectrometer on Cassini and 
saw four separate jets of water blocking out the light of distant
stars.  With further clever reasoning they estimated the velocity of 
these jets.  They seem to be moving at supersonic speeds - up to 2000
kilometers per hour.  

Then, they argue that such high speeds are only possible if there's
lots of underground liquid water near the surface on this part of Enceladus!

This conclusion is still controversial.  Susan Kieffer of the University
of Illinois at Urbana-Champaign has a competing theory, which says the
water is in the form of gas clathrates.  I showed you pictures of these
back in "<a href = "week269.html">week269</A>": they're cage-shaped ice crystal structures that trap
gas molecules inside.  There are about 6 trillion tons of gas clathrates
on the ocean floors here on Earth.  Maybe they're also on Enceladus:

9) Susan W. Kieffer, Xinli Lu, Craig M. Bethke, John R. Spencer, Stephen Marshak, Alexandra Navrotsky, A clathrate reservoir hypothesis for 
Enceladus' south polar plume, Science 314 (15 December 2006), 1764-1766.

Only time will tell.

What's not controversial is that the geysers of Enceladus shoot out
plumes of stuff - mainly water vapor and ice, but also other gases and
dust - at speeds exceeding escape velocity.  Enceladus is small: just
500 kilometers across.  So its gravity is weak; its escape velocity is
low.  So it's easy for geysers to blast stuff straight out into space.

And then what?  Then it spreads out and form a huge ring around
Saturn: the "E ring".  The E ring is much more diffuse than
the really famous rings of Saturn.  It's much further out, and usually
almost invisible.  But here's a great photo of the E ring, again taken by
the hero of this story - the Cassini probe:

<div align = "center">
<img width = "500" border = "2" src = "enceladus_e_ring.jpg">
</div>

10) NASA Photojournal, Ghostly fingers of Enceladus,
<a href = "http://photojournal.jpl.nasa.gov/catalog/PIA08321">http://photojournal.jpl.nasa.gov/catalog/PIA08321</a>

The trick was to take the shot with the sun almost directly behind the
camera.  Unexpected wispy patterns may hint at interactions between
the ring and Saturn's magnetic field.  Enceladus is the white blob in
the middle of the ring.  If you look carefully, you can see its
geysers... and you can even see it carving a path through the E ring!

By the way, the volcanic activity on Io also produces a kind of ring:
a of ionized sulfur, oxygen, sodium, and chlorine.  And since the
magnetic field of Jupiter is very strong, and Io is close by, this
causes dramatic effects.  For example, charged particles from this
plasma torus are funnelled down something called the "Io flux
tube" to Jupiter's surface, where they create light shows 1000
times brighter than the aurora borealis here on earth!  You can see
pictures here:

11) Astronomy Picture of the Day, Jovian aurora,
<a href = "http://apod.nasa.gov/apod/ap980123.html">http://apod.nasa.gov/apod/ap980123.html</a>

Next: some exceptional properties of the number 6.

Every set has a group of symmetries, but so does every group.  The
symmetries of a group are called "automorphisms" of the
group.  Every symmetry of a set gives a symmetry of its group of its
symmetries, called an "inner automorphism".  But of all
finite sets, only the six-element set has a symmetry group with an
EXTRA symmetry - a symmetry that doesn't come from a symmetry of that
set!  This is called an "outer automorphism".

In more jargonesque terms: of all the permutation groups
S_{n}, only S_{6} has an outer automorphism.  It has
just one, and I described it here:

12) John Baez, Some thoughts on the number six, <a href = "http://math.ucr.edu/home/baez/six.html">http://math.ucr.edu/home/baez/six.html</a>

Recently my friend Bruce Westbury told me some more cool facts about 
the number 6.  One of these is another nice description of the outer
automorphism of S_{6}.  He heard this from the group theorist Brian Bowditch, who heard it from Chris Penrose (Roger's son).  I haven't
checked it!  I'll just pass it on, as gossip, and let you see if it's true.

It goes like this.  Start with the icosahedron, drawn as a graph on
the sphere, and identify oposite points.  This gives a graph on the 
projective plane.  This graph is called K_{6}: the "complete graph on 6 vertices".  In other words, it's the graph with 6 vertices, each one
connected to every other one by a single edge.   The faces of this 
graph are triangles.  Apparently there are twelve embeddings of K_{6} 
in the projective plane and they come in six pairs.  Each permutation 
of the six vertices of K_{6} induces a permutation of these six pairs. 
This process gives an automorphism of S_{6}.  And apparently this is the
outer automorphism!

Westbury later pointed out this related paper:

13) Ben Howard, John Millson, Andrew Snowden, Ravi Vakil, A 
description of the outer automorphism of S_{6}, and the invariants 
of six points in projective space, available as <a href = "http://arxiv.org/abs/0710.5916">arXiv:0710.5916</a>.

I haven't checked to see if the results here imply the claim above.

Another group theorist, Derek Holt, told Westbury about an intriguing
conjecture which he believes but has not had the energy to check.

Take a finite simple group and choose two elements at random.  What 
is the probability that they generate the group?  

One cool fact is that this probability is never zero.  Every finite
simple group can be generated by two elements!  So, they're all
quotients of the free group on two generators.

Another cool fact is that this probability tends to 1 as the order of
the group tends to infinity:

14) M. Liebeck and A. Shalev, The probability of generating a finite 
simple group, Geometrica Dedicata 56 (1995), 103-111.

This implies that this probability attains a \emph{mininum value} for
some finite simple group.  For A_{6} - the group of even
permutations of a 6-element set - the probability is 53/90.  Holt's
conjecture is that this is the minimum among all finite simple groups.

Here are the probabilities for a few finite simple groups, listed
in order of increasing size:

A_{2}                 
                          
                                      1     
                      
                                   = 1.000

A_{3}                
                            
            
8/9   
                     
       
           ~ .889

PSL(2,3) \cong  A_{4}              
              
              
  2/3    
   
                             
    
   ~ .667

PSL(2,4) \cong  PSL(2,5) \cong  A_{5}      
19/30
                                  
~ .633

PSL(2,7) \cong  PSL(3,2)                          
19/28
                                        
              
~ .679

PSL(2,8)                  
                     
                                   71/84
                  
            
 ~ .845

PSL(2,9) \cong  A_{6}      
                           
             
53/90
                            
~ .589

PSL(2,11)      
                     
                                
 127/165
                                
      ~ .770

A_{7} 
                          
               
               
                           229/315   
                    
    ~ .727 

PSL(4,2) \cong  A_{8}          
                          
   
                   133/180   
                          
      
  ~ .739

A_{9}
               
        
        
        
        
     
                15403/18144 
                        
~ .849

Here A_{n} is the group of even permutations of an n-element set,
while PSL(n,q) is the group of n \times  n matrices with determinant 1
having entries in the field with q elements, mod multiples of the identity
matrix.  Of the groups listed above, only A_{4} is not simple.  I
included this one just because I wanted to list \emph{all} the unexpected
isomorphisms between "alternating" groups (A_{n}'s) and "projective
special linear" groups (PSL's).

(Each of these isomorphisms makes a wonderful story in 
itself - but not for today!  If you want to know these stories, 
try the first of Conway's "Three Lectures on Exceptional Groups" from his 
book with Sloane on Sphere Packings, Lattices and Groups.  Also try 
"<a href = "week79.html">week79</a>", to see how these stories
are related to Galois' fatal duel and the buckyball.) 

So, you see the number 6 has several special properties, all related to
permutations of a 6-element set.  What do these special properties
really mean?  I don't know!  They're just little clues waiting for a
big mystery story to come along.

Lately I've been finishing up a big paper on infinite-dimensional
representations of 2-groups, which I'm writing along with my former
student Derek Wise, the physicist Laurent Freidel, and his student
Aristide Baratin.  Derek and Aristide are doing all the really hard
work, but I used to do functional analysis when I was a youngster, 
and there's a lot of measure theory in our paper, so sometimes it's 
my job to dig up papers on this subject and see if they help.  And 
this sort of work can actually be fun, once I get into the mood.

I don't want to try to explain this paper now.  I just feel like
explaining some tidbits of measure theory that I've picked up while
writing it.

A measure space is basically a space that you can do integrals over.
But some measure spaces are better than others.  Some of them are
immensely huge and nasty.  But "standard Borel spaces" include all 
the examples I care about, and they're incredibly nice.

So, what's a standard Borel space?  

It's a kind of "measurable space", meaning a space equipped a collection
of subsets that's closed under countable intersections, countable
unions and complement.  Such a collection is called a "sigma-algebra",
and we call the sets in here "measurable".  A "measure" on a measurable 
space assigns a number between 0 and +\infty  to each measurable set,
in such a way that for any countable disjoint union of measurable sets,
the measure of their union is the sum of their measures.  

A nice way to build a measurable space is to start with a topological
space.  Then you take its open sets and keep taking countable
intersections, countable unions and complements until you get a
sigma-algebra.  This may take a long time, but if you believe in
transfinite induction you're bound to eventually succeed.  The sets in
this sigma-algebra are called "Borel sets".

A basic result in real analysis is that if you put the usual
topology on the real line, and use this to cook up a sigma-algebra
as just described, there's a unique measure on the resulting measurable
space that assigns to each interval its usual length.  This is called
"Lebesgue measure".

Some topological spaces are big and nasty.  But 
<a href = "http://en.wikipedia.org/wiki/Separable_space">separable</a>
<a href = "http://en.wikipedia.org/wiki/Complete_metric_space">complete
metric spaces</a> are not so bad.  

We don't care about the metric in this game.  So, we use the term
"Polish space" for a topological space that's \emph{homeomorphic} to a
complete separable metric space.

And often we don't even care about the topology.  So, we use the
term "standard Borel space" for a measure space whose measurable
sets are the Borel sets for some topology making it into a Polish 
space. 

In short: every complete separable metric space has a Polish space as
its underlying topological space, and every Polish space has a 
standard Borel space as its underlying measurable space.

Now, it's hopeless to classify complete separable metric spaces.
It's even hopeless to classify Polish spaces.  But it's not
hopeless to classify standard Borel spaces!  The reason is that metric
spaces are like diamonds: you can't bend or stretch them at all without
breaking them entirely.  But topological spaces are like rubber... and
measurable spaces are like dust.  So, it's very hard for two metric
spaces to be isomorphic, but it's easier for their underlying
topological spaces - and even easier for their underlying measurable
spaces.  

For example, the line and plane are isomorphic, if we use their 
usual sigma-algebras of Borel sets to make them into measurable
spaces!   And the plane is isomorphic to R^{n} for every n, and all
these are isomorphic to a separable Hilbert space!  As measurable
spaces, that is.

In fact, every standard Borel space is isomorphic to one of these:

<UL>
<LI>
a countable set with its sigma-algebra of all subsets,
<LI>
the real line with its sigma-algebra of Borel subsets.
</UL>

That's pretty amazing.   It means that standard Borel spaces are
classified by just their \emph{cardinality}, which can only be finite,
countably infinite, or the cardinality of the continuum.  The "continuum
hypothesis" says there's no cardinality between the countably infinite
one and the cardinality of the continuum - but we don't need the
continuum hypothesis to prove this result.

This amazing result has a nice relative in the world of von Neumann
algebras.

Starting from a measure space X, we can form the Hilbert space
L^{2}(X).  We can also form L^{\infty }(X), the space
of equivalence classes of bounded measurable complex functions on X,
where we identify functions that agree except on a set of measure
zero.

L^{\infty }(X) is an algebra where we multiply functions
pointwise.  We can think of it as an algebra of multiplication
operators on L^{2}(X).  In fact, it forms a "von Neumann
algebra" - a kind of algebra that's great for quantum theory.  I
defined von Neumann algebras back in "<a href =
"week175.html">week175</a>", so I won't do it again here.

L^{\infty }(X) is a very special sort of von Neumann algebra,
namely a \emph{commutative} one.  There's a nice theorem: every
commutative von Neumann algebra is isomorphic (as an abstract C*-algebra)
to one of this form.
Even better, every commutative von Neumann algebra of operators on a
\emph{separable} Hilbert space is isomorphic 
to L^{\infty }(X) for one of these choices of X:

<UL>
<LI>
a countable set (equipped with counting measure),
<LI>
the real line (with Lebesgue measure),
<LI>
the disjoint union of a countable set and the real line 
(with the obvious measure).
</UL>

We're really talking about measure theory now, even when it seems
I'm talking about commutative von Neumann algebras.  But sometimes
topology sneaks into the game.  This especially true if we're studying 
groups.  We can talk about "standard Borel groups": standard Borel
spaces made into groups where all the group operations are measurable.
But nobody seems to know much about these!  What people know about are
"Polish groups": Polish spaces made into groups where all the group
operations are continuous.  

If you like keeping your categories straight, it may seem annoying to
put a topology on your group when you're trying to do measure theory.
But apparently some topology is a good thing... either that, or people
haven't found the best framework for this subject.

It's a bit slippery, you see.  Not every standard Borel group comes 
from a Polish group.  But the counterexamples seem to be horrid.
And more importantly, every measurable homomorphism between Polish
groups is automatically continuous!  So there must be some way to 
define Polish groups as standard Borel groups with an extra \emph{property} -
not extra structure.

There are other similar results with the same flavor.  For example,
suppose we have a Polish group G acting as measurable transformations 
on a standard Borel space X.  Then we can make X into a Polish space so
that the action of G on X becomes continuous!  

(Of course we also require that the new Borel sets of X are the 
same as its original measurable sets.)

There are also interesting results saying that Polish spaces and
Polish groups "aren't too big".  Some of these theorems mention the
"Hilbert cube" - that is, a countable product of copies of [0,1], 
with its product topology.  This space sounds big, but thanks to 
Tychonoff's theorem it's compact!  It's also metrizable: that is,
its topology comes from a metric.  And it's separable: it has a
countable dense subset.  In fact, a topological space is separable
and metrizable iff it's homeomorphic to a subset of the Hilbert cube.  

Here's how the Hilbert cube relates to Polish spaces: a topological
space is a Polish space iff it's homeomorphic to a countable 
intersection of open subsets of the Hilbert cube!

And here's how the Hilbert cube relates to Polish groups: a 
topological group is a Polish group iff it's isomorphic to a
subgroup of the group of homeomorphisms of the Hilbert cube!

I'll wrap up with a nice fact about measures on standard Borel 
spaces.  A "probability measure" on a measure space X is one that
assigns the number 1 to the whole set X.  Then the measure of a
measurable subset S of X can be interpreted as the probability
of a point being in X.  This idea of using measures to study
probabilities lies at the foundation of modern probability theory.

But here's a lesser-known fact: the set of probability measures on X,
say M(X), is itself a measurable space!

How does this work?  First we give M(X) its "weak topology".
This is the topology where a bunch of measures \mu _{i} converge to
\mu  if for every bounded continuous function f on X, 

&int; f d\mu _{i} \to  &int; f d\mu 

Starting with this topology and taking the Borel sets, M(X)
becomes a measurable space.

It then turns out that M(X) is a standard Borel space iff X is!

In fact, M is what category theorists call a "monad" on the category
of standard Borel spaces.  I said what a monad is back in "<a href =
"week89.html#tale">week89</a>", so I won't do it again, but here it
involves the presence of god-given measurable maps

M(M(X)) \to  M(X)  

and 

X \to  M(X)

The first map is the most interesting: a probability measure
on the space of probability measures of X gives a probability
measure on X!  In layman's terms: if you're uncertain about
how uncertain you are, you might as well just say you're uncertain.

The second map just sends any point in X to the Dirac delta measure at
that point.  In layman's terms: if you're certain, you might as well
say you're uncertain (just not very much).  That is, if you're
completely certain about something, you can still describe your state
of knowledge by a probability distribution.

You can see some applications of this monad here:

15) David Corfield, Category theoretic probability theory II,
<a href = "http://golem.ph.utexas.edu/category/2007/02/category_theoretic_probability_1.html">http://golem.ph.utexas.edu/category/2007/02/category_theoretic_probability_1.html</a>

16) Ernst-Erich Doberkat, Characterizing the Eilenberg-Moore 
algebras for a monad of stochastic relations, Universitat 
Dortmund, Fachbereit Informatik, Lehrstule fuer Software-Technologie,
Memo 147.  Also available at 
<a href = "https://eldorado.uni-dortmund.de/bitstream/2003/2717/1/147.pdf">https://eldorado.uni-dortmund.de/bitstream/2003/2717/1/147.pdf</a>

For more on standard Borel spaces in probability theory, try this:

17) K. R. Parthasrathy, Probability Measures on Metric Spaces, 
Academic Press, San Diego, 1967.

For standard Borel spaces and von Neumann algebras, try this:

18) W. Arveson, An Invitation to C*-Algebra, Springer, Berlin, 1976.

And for really hard-core results on standard Borel spaces and 
Polish groups, try this:

19) H. Becker and A. S. Kechris, Descriptive Set Theory of Polish 
Group Actions, Cambridge U. Press, Cambridge, 1996.

I can't imagine any of the normal readers of This Week's Finds 
will enjoy this book, but I could be wrong: there could be some
really scary people who read this column.

\par\noindent\rule{\textwidth}{0.4pt}
\textbf{Addenda:} I thank Greg Egan, Benoit Jubin,
Ted Nitz and Ultrawaffle for corrections 
and improvements.   Originally I had a less complete table of 
probabilities for finite simple groups, and I begged my readers for 
help.  Ted Nitz and Greg Egan stepped into save me.

Ted Nitz calculated the odds that a pair of elements in 
A_{4} generates the group and got 2/3:

\begin{quote}
$$
I decided to take your challenge. For A_{4} the odds 
are 96/144 = 2/3 that a pair generates. For reference, 
the sage code that gave the probability is:

A4 = AlternatingGroup(4)
order = A4.order()
i = 0
j = 0
for g in A4:
    for h in A4:
        i += 1
        if A4.subgroup([g,h]).order() == order:
            j += 1
print (j,i)
j/i

It was blindingly fast for A_{4}, and if it ever 
finishes for A_{9}, I'll let you know the probability 
there.

-Ted
$$
    
\end{quote}

He later realized that the A_{9} calculation would take at least
a century.  Greg Egan wrote a Mathematica program based on a more
efficient algorithm and got the answer in less than a day.

For more discussion, including the idea behind Egan's algorithm,
visit the <a href = "http://golem.ph.utexas.edu/category/2008/11/this_weeks_finds_in_mathematic_33.html">n-Category Caf&eacute;</a>.

\par\noindent\rule{\textwidth}{0.4pt}
<i>
The splitting into something discrete and something continuous
seems to me to be a basic issue in all morphology.</i> - Hermann Weyl

\par\noindent\rule{\textwidth}{0.4pt}

% </A>
% </A>
% </A>
