
% </A>
% </A>
% </A>
\week{May 10, 2004 }


I just got back from Marseille, where Carlo Rovelli, Laurent Freidel
and Phillipe Roche held the first really big conference on loop quantum 
gravity and spin foams since the 2nd Warsaw workshop run by Jerzy 
Lewandowski back in 1997:

1) Non Perturbative Quantum Gravity: Loops and Spin Foams, 
3-7 May 2004, CIRM, Luminy, Marseille, France,
<A HREF = "http://w3.lpm.univ-montp2.fr/~philippe/quantumgravitywebsite/">
http://w3.lpm.univ-montp2.fr/~philippe/quantumgravitywebsite/</A>

It was good to see old friends and talk about quantum gravity near 
the "<A HREF = "marseille/">Calanques</A>" - 
the rugged limestone cliffs lining the Mediterranean 
coastline.   It was good to meet lots of young people who have recently 
entered this difficult field: about 100 people attended, considerably 
more than at any previous meeting.  But most of all, it was good to 
see some progress on the tough problem of understanding dynamics in 
nonperturbative quantum gravity.  

Can we get the 4-dimensional spacetime we know and love, whose geometry
is described by general relativity, to emerge from some theory that takes
quantum physics into account?  And can we do it \emph{nonperturbatively?}
 
In other words, can we do quantum physics without choosing some fixed 
spacetime geometry from the start, a "background" on which small 
perturbations move like tiny quantum ripples on a calm pre-established 
lake?  A background geometry is convenient: it lets us keep track of 
times and distances.  It's like having a fixed stage on which the actors - 
gravitons, strings, branes, or whatever - cavort and dance.  But, the 
main lesson of general relativity is that spacetime is \emph{not} a fixed 
stage: it's a lively, dynamical entity!   There's no good way to separate 
the ripples from the lake.  This distinction is no more than a convenient
approximation - and a dangerous one at that.

So, we should learn to make do without a background when studying quantum
gravity.  But it's tough!  There are knotty conceptual issues like the
"problem of time": how do we describe time evolution without 
using a fixed 
background to measure the passage of time?  There are also practical 
problems: in most attempts to describe spacetime from the ground up in 
a quantum way, all hell breaks loose!  

We can easily get spacetimes that crumple up into a tiny blob... or 
spacetimes that form endlessly branching fractal "polymers" of 
Hausdorff 
dimension 2... but it seems hard to get reasonably smooth spacetimes of 
dimension 4.  It's even hard to get spacetimes of dimension 10 or 11...
or \emph{anything} remotely interesting!

It almost seems as if we need a solid background as a bed frame to keep 
the mattress of spacetime from rolling up, getting all lumpy, or 
otherwise misbehaving.  
Unfortunately, even \emph{with} a background there are serious problems: we 
can use perturbation theory to write the answers to physics questions as 
power series, but these series diverge and nobody knows how to resum them.

String theorists are pragmatic in a certain sense: they don't mind using 
a background, and they don't mind doing what physicists always do: 
approximating a divergent series by the sum of the first couple of terms.
But this attitude doesn't solve everything, because right now in string 
theory there is an enormous "landscape" of different backgrounds, with no 
firm principle for choosing one.  Some estimates guess there are over 
10^{100}.  Leonard Susskind guesses there are 10^{500}, 
and argues that 
we'll need the anthropic principle to choose the one describing our
world:

2) Leonard Susskind, The Landscape, article and interview on John
Brockman's "EDGE" website, 
<A HREF = "http://www.edge.org/3rd_culture/susskind03/susskind_index.html">
http://www.edge.org/3rd_culture/susskind03/susskind_index.html</A>

This position is highly controversial, but my point here shouldn't be: 
developing a background-free theory of quantum gravity is tough, but
working \emph{with} a background has its own difficulties.  And let's face
it: we haven't spent nearly as much time thinking about background-free
or nonperturbative physics as we've spent on background-dependent 
or perturbative physics.   So, it's quite possible that our failures
with the former are just a matter of inexperience.

Given all this, I'm delighted to see some real progress on getting 4d
spacetime to emerge from nonperturbative quantum gravity:

3) Jan Ambjorn, Jerzy Jurkiewicz and Renate Loll, Emergence of a 4d world 
from causal quantum gravity, available as 
<A HREF = "http://xxx.lanl.gov/abs/hep-th/0404156">hep-th/0404156</A>.

This trio of researchers have revitalized an approach called "dynamical 
triangulations" where we calculate path integrals in quantum gravity by 
summing over different ways of building spacetime out of little 4-simplices.
They showed that if we restrict this sum to spacetimes with a well-behaved
concept of causality, we get good results.  This is a bit startling,
because after decades of work, most researchers had despaired of getting 
general relativity to emerge at large distances starting from the dynamical 
triangulations approach.  But, these people hadn't noticed a certain flaw 
in the approach... a flaw which Loll and collaborators noticed and fixed!  

If you don't know what a path integral is, don't worry: it's pretty 
simple.  Basically, in quantum physics we can calculate the expected value 
of any physical quantity by doing an average over all possible histories 
of the system in question, with each history weighted by a complex number 
called its "amplitude".  For a particle, a history is just a path in 
space; to average over all histories is to integrate over all paths - 
hence the term "path integral".  But in quantum gravity, a history is 
nothing other than a SPACETIME.

Mathematically, a "spacetime" is something like a 4-dimensional 
manifold 
equipped with a Lorentzian metric.  But it's hard to integrate over all 
of these - there are just too darn many.  So, sometimes people instead 
treat spacetime as made of little discrete building blocks, turning
the path integral into a sum.  You can either take this seriously or treat
it as a kind of approximation.  Luckily, the calculations work the same 
either way!  

If you're looking to build spacetime out of some sort of discrete building
block, a handy candidate is the "4-simplex": the 4-dimensional 
analogue 
of a tetrahedron.   This shape is rigid once you fix the lengths of its 10
edges, which correspond to the 10 components of the metric tensor in 
general relativity.  

There are lots of approaches to the path integrals in quantum gravity
that start by chopping spacetime into 4-simplices.  The weird special
thing about dynamical triangulations is that here we usually assume 
every 4-simplex in spacetime has the same shape.  The different spacetimes
arise solely from different ways of sticking the 4-simplices together.

Why such a drastic simplifying assumption?   To make calculations quick 
and easy!  The goal is get models where you can simulate quantum geometry 
on your laptop - or at least a supercomputer.  The hope is that simplifying 
assumptions about physics at the Planck scale will wash out and not make 
much difference on large length scales.  

Computations using the so-called "renormalization group flow" suggest 
that this hope is true \emph{if} the path integral is dominated by spacetimes 
that look, when viewed from afar, almost like 4d manifolds with smooth 
metrics.  Given this, it seems we're bound to get general relativity at
large distance scales - perhaps with a nonzero cosmological constant, and
perhaps including various forms of matter.  

Unfortunately, in all previous dynamical triangulation models, the path
integral was \emph{not} 
dominated by spacetimes that look like nice 4d manifolds
from afar!  Depending on the details, one either got a "crumpled 
phase"
dominated by spacetimes where almost all the 4-simplices touch each other,
or a "branched polymer phase" dominated by spacetimes where 
the 4-simplices
form treelike structures.  There's a transition between these two phases, 
but unfortunately it seems to be a 1st-order phase transition - not the 
sort we can get anything useful out of.  For a nice review of these 
calculations, see:

4) Renate Loll, Discrete approaches to quantum gravity in four dimensions,
available as 
<A HREF = "http://xxx.lanl.gov/abs/gr-qc/9805049">gr-qc/9805049</A> 
or as a website at Living Reviews in Relativity, 
<A HREF = "http://www.livingreviews.org/Articles/Volume1/1998-13loll/">
http://www.livingreviews.org/Articles/Volume1/1998-13loll/</A>

Luckily, all these calculations shared a common flaw!

Computer calculations of path integrals become a lot easier if instead of 
assigning a complex "amplitude" to each history, we assign it a 
positive
real number: a "relative probability".   The basic reason is 
that unlike 
positive real numbers, complex numbers can cancel out when you sum them! 
 
When we have relative probabilities, it's the \emph{highly probable} 
histories 
that contribute most to the expected value of any physical quantity.  We
can use something called the "Metropolis algorithm" to spot 
these highly
probable histories and spend most of our time focusing on them.  

This doesn't work when we have complex amplitudes, since even a history 
with a big amplitude can be canceled out by a nearby history with the 
opposite big amplitude!  Indeed, this happens all the time.  So, instead 
of histories with big amplitudes, it's the <em>bunches of histories that 
happen not to completely cancel out</em> that really matter.  Nobody knows an 
efficient general-purpose algorithm to deal with this!
 
For this reason, physicists often use a trick called "Wick rotation"
that converts amplitudes to relative probabilities.  To do this trick, we
just replace time by imaginary time!  In other words, wherever we see the 
variable "t" for time in any formula, we replace it 
by "it".  Magically,
this often does the job: our amplitudes turn into relative probabilities!
We then go ahead and calculate stuff.  Then we take this stuff and go 
back and replace "it" everywhere by "t" to get our final 
answers.

While the deep inner meaning of this trick is mysterious, it can be 
justified in a wide variety of contexts using the "Osterwalder-Schrader
theorem".  Here's a pretty general version of this theorem, suitable
for quantum gravity:

5) Abhay Ashtekar, Donald Marolf, Jose Mourao and Thomas Thiemann,
Constructing Hamiltonian quantum theories from path integrals in a 
diffeomorphism invariant context, 
Class. Quant. Grav. 17 (2000) 4919-4940.  Also
available as <A HREF = "http://www.arxiv.org/abs/quant-ph/9904094">
quant-ph/9904094</A>.

People use Wick rotation in all work on dynamical triangulations. 
Unfortunately, this is \emph{not} a context where you can justify this trick
by appealing to the Osterwalder-Schrader theorem.  The problem is that
there's no good notion of a time coordinate "t" on your typical 
spacetime built by sticking together a bunch of 4-simplices!  

The new work by Ambjorn, Jurkiewiecz and Loll deals with this by 
restricting to spacetimes that \emph{do} have a time coordinate.  More 
precisely, they fix a 3-dimensional manifold and consider all possible 
triangulations of this manifold by regular tetrahedra.  These are the 
allowed "slices" of spacetime - they represent different possible 
geometries of space at a given time.  They then consider spacetimes 
having slices of this form joined together by 4-simplices in a few 
simple ways.  

The slicing gives a preferred time parameter "t".  On the one hand 
this
goes against our desire in general relativity to avoid a preferred time 
coordinate - but on the other hand, it allows Wick rotation.  So, they 
can use the Metropolis algorithm to compute things to their hearts'
content and then replace "it" by "t" at the end.  

When they do this, they get convincing good evidence that the spacetimes 
which dominate the path integral look approximately like nice smooth 
4-dimensional manifolds at large distances!  Take a look at their graphs 
and pictures - a picture is worth a thousand words.

Naturally, what \emph{I'd} like to do is use their work to develop some spin
foam models with better physical behavior than the ones we have so far.  
If you look at my talk you can see some of the problems we've encountered:

6) John Baez, Spin foam models, talk at Non Perturbative Quantum Gravity: 
Loops and Spin Foams, May 4, 2004, transparencies available at
<A HREF = "http://math.ucr.edu/home/baez/spin_foam_models/">
http://math.ucr.edu/home/baez/spin_foam_models/</A>

Now that Loll and her collaborators have gotten something that works, 
we can try to fiddle around and make it more elegant while making sure it
still works.  In particular, I'm hoping we can get well-behaved models 
that don't introduce a preferred time coordinate as long as they rule out 
"topology change" - that is, slicings where the topology of space 
changes.
After all, the Osterwalder-Schrader theorem doesn't require a \emph{preferred}
time coordinate, just \emph{any} time coordinate together with good behavior 
under change of time coordinate.  For this we mainly need to rule out 
topology change.  Moreover, Loll and her collaborators have argued in 2d 
toy models that topology change is one thing that makes models go bad: the 
path integral can get dominated by spacetimes where "baby universes" 
keep branching off the main one:

7) Jan Ambjorn, Jerzy Jurkiewicz and Renate Loll, Non-perturbative 
Lorentzian quantum gravity, causality and topology change, Nucl. Phys. 
B536 (1998) 407-434.  Also available as <A HREF = "http://xxx.lanl.gov/abs/hep-th/9805108">hep-th/9805108</A>.

Renate Loll and W. Westra, Space-time foam in 2d and the sum over 
topologies, Acta Phys. Polon. B34 (2003) 4997-5008.  Also available as
<A HREF = "http://xxx.lanl.gov/abs/hep-th/0309012">hep-th/0309012</A>.

By the way, it's also reading about their 3d model:

8) Jan Ambjorn, Jerzy Jurkiewicz and Renate Loll, Non-perturbative 3d 
Lorentzian quantum gravity, Phys.Rev. D64 (2001) 044011.  Also available
as <A HREF = "http://xxx.lanl.gov/abs/hep-th/0011276">hep-th/0011276</A>.

and for a general review, try this:
 
9) Renate Loll, A discrete history of the Lorentzian path integral,
Lecture Notes in Physics 631, Springer, Berlin, 2003, pp. 137-171.
Also available as 
<A HREF = "http://xxx.lanl.gov/abs/hep-th/0212340">hep-th/0212340</A>.

All this is great, but don't get me wrong - there were a lot of \emph{other}
cool talks at the conference besides Loll's.  I'll just mention a few.

Laurent Freidel spoke on his work on spin foam models.  Especially
exciting is how David Louapre and he have managed to "sum over 
topologies" in 3d Riemannian quantum gravity with vanishing cosmological
constant - otherwise known as the Ponzano-Regge model  He has to subtract 
out a counterterm that would otherwise lead to a bubble divergence, but 
then he gets a beautiful theory where the sum over spin foams is Borel 
summable:

10) Laurent Freidel and David Louapre, Non-perturbative summation over 
3D discrete topologies, Phys. Rev. D68 (2003) 104004.  Also available as
<A HREF = "http://xxx.lanl.gov/abs/hep-th/0211026">hep-th/0211026</A>.

Their work on gauge-fixing and the inclusion of spinning point particles
in the Ponzano-Regge model is also very impressive, especially given how
long this model has been studied.  It shows we have lots left to learn! 

11) Laurent Freidel and David Louapre, Ponzano-Regge model revisited I: 
Gauge fixing, observables and interacting spinning particles, available
as <A HREF = "http://xxx.lanl.gov/abs/hep-th/0401076">hep-th/0401076</A>.

The title suggests we're in for more treats to come.

Kirill Krasnov gave a talk entitled simple "ln(3)" - it was all about
the appearance of this constant in the work of Hod, Dreyer, Motl and 
Neitzke on black hole entropy and the ringing of black holes.  I've
discussed all this at length 
in "<A HREF = "week198.html">week198</A>", but Krasnov has given 
an elegant
new proof of Hod's conjecture using Riemann surface theory.  One can 
even think of this as a "stringy" explanation of the quasinormal 
modes of black holes - but much remains mysterious here:

12) Kirill Krasnov, Black hole thermodynamics and Riemann surfaces,
Class. Quant. Grav. 20 (2003) 2235-2250.  Also available 
as <A HREF = "http://xxx.lanl.gov/abs/gr-qc/0302073">gr-qc/0302073</A>.

Kirill Krasnov and Sergey N. Solodukhin, Effective stringy description 
of Schwarzschild black holes, available 
as <A HREF = "http://xxx.lanl.gov/abs/hep-th/0403046">hep-th/0403046</A>.

While I'm at it, I can't resist mentioning Krasnov's work on including
point particles in 3d Lorentzian quantum gravity with negative
cosmological constant, since it has close connections with that of 
Freidel and Louapre, though the context is a bit different:

13) Kirill Krasnov, \Lambda <0 quantum gravity in 2+1 dimensions I:
quantum states and stringy S-matrix, Class. Quant. Grav. 19 (2002) 
3977-3998.  Also available 
as <A HREF = "http://xxx.lanl.gov/abs/hep-th/0112164">hep-th/0112164</A>.

Kirill Krasnov, \Lambda <0 quantum gravity in 2+1 dimensions II: 
black hole creation by point particles, Class. Quant. Grav. 19 (2002) 
3999-4028.  Also available 
as <A HREF = "http://xxx.lanl.gov/abs/hep-th/0202117">hep-th/0202117</A>.

If I could duplicate myself, I'd have one copy write a book on 3d quantum 
gravity that would synthesize all these wonderful results in a nice big 
picture.  It's not realistic physics; it's just a toy model.  But the 
math is \emph{so} nice, and so enlightening for real-world physics in some 
ways, that it's hard to resist pondering it!  TQFTs, Riemann surfaces,
hyperbolic geometry, spinning point particles colliding and creating
black holes - a wonderful stew!  Alas, I don't have time to savor it.

There were a lot of other interesting talks - but I don't have time to go 
through and describe all of them, either.  So, I'll wrap up with something
very different!  

Lee Smolin told me some neat stuff about MOND - that's "Modified 
Newtonian Dynamics", which is Mordehai Milgrom's way of trying to explain 
the strange behavior of galaxies without invoking dark matter.  The basic 
problem with galaxies is that the outer parts rotate faster than they 
should given how much mass we actually see.  

If you have a planet in a circular orbit about the Sun, Newton's laws 
say its acceleration is proportional to 1/r^{2}, 
where r is its distance to 
the Sun.  Similarly, if almost all the mass in a galaxy were concentrated
right at the center, a star orbiting in a circle at distance r from the 
center would have acceleration proportional to 1/r^{2}.  
Of course, not all
the mass is right at the center!  So, the acceleration should drop off 
more slowly than 1/r^{2} 
as you go further out.  And it does.  But, the 
observed acceleration drops off a lot more slowly than the acceleration 
people calculate from the mass they see.  It's not a small effect: it's a 
\emph{huge} effect!

One solution is to say there's a lot of mass we don't 
see: "dark matter"
of some sort.  If you take this route, which most astronomers do, you're
forced to say that \emph{most} 
of the mass of galaxies is in the form of dark 
matter.

Milgrom's solution is to say that Newton's laws are messed up.

Of course this is a drastic, dangerous step: the last guy who tried this
was named Einstein, and we all know what happened to him.  Milgrom's theory 
isn't even based on deep reasoning and beautiful math like Einstein's!  
Instead, it's just a blatant attempt to fit the experimental data.  
And it's not even elegant.  In fact, it's downright ugly.

Here's what it says: the usual Newtonian formula for the acceleration 
due to gravity is correct as long as the acceleration is bigger than


$$

a = 2 x 10^{-10} m/sec^{2}   
$$
    
But, for accelerations less than this, you take the geometric mean
of the acceleration Newton would predict and this constant a.

In other words, there's a certain value of acceleration such that above
this value, the Newtonian law of gravity works as usual, while below this
value the law suddenly changes.

Any physicist worth his salt who hears this modification of Newton's law
should be overcome with a feeling of revulsion!  There just \emph{aren't} 
laws
of physics that split a situation in two cases and say "if this is bigger
than that, then do X, but if it's smaller, then do Y."  Not in fundamental
physics, anyway!  Sure, water is solid below 0 centigrade and fluid above
this, but that's not a fundamental law - it presumably follows from other 
stuff.  Not that anyone has derived the melting point of ice from first
principles, mind you.  But we think we could if we were better at big
messy calculations.

Furthermore, you can't easily invent a Lagrangian for gravity that makes 
it fall off more \emph{slowly} than 1/r^{2}.  
It's easy to get it to fall off 
\emph{faster} - just give the graviton a mass, for example!  But not more 
slowly.  It turns out you can do it - Bekenstein and Milgrom have a way - 
but it's incredibly ugly. 

So, MOND should instantly make any decent physicist cringe.  Esthetics 
alone would be enough to rule it out, except for one slight problem: it 
seems to fit the data!  In some cases it matches the observed rotation of 
galaxies in an appallingly accurate way, fitting every wiggle in the graph
of stellar rotation velocity as a function of distance from the center.

So, even if MOND is wrong, there may need to be some reason why it 
\emph{acts}
like it's right!  Apparently even some proponents of dark matter agree 
with this.  

But: take everything I'm saying here with a grain of salt.  I'm no expert 
on this stuff, so if you know any astrophysics you should read the 
literature and make up your own mind.  

Here are two reviews that Smolin especially recommended:

14) Robert H. Sanders and Stacy S. McGaugh, Modified Newtonian Dynamics 
as an Alternative to Dark Matter, available as <A HREF = "http://xxx.lanl.gov/abs/astro-ph/0204521">astro-ph/0204521</A>.

15) Anthony Aguirre, Alternatives to dark matter (?), available as
<A HREF = "http://xxx.lanl.gov/abs/astro-ph/0310572">astro-ph/0310572</A>.

Here's McGaugh's website with links to many papers on MOND, including
Milgrom's original papers:

16) The MOND pages, 
<A HREF = "http://www.astro.umd.edu/~ssm/mond/litsub.html">
http://www.astro.umd.edu/~ssm/mond/litsub.html</A>

McGaugh is a strong proponent of MOND - though he didn't start out that
way - so the selection may be biased.  Does anyone know an intelligent 
detailed critique of MOND?  If so, I want to see it!  We can't throw out 
Newton's law of gravity (or more precisely, general relativity, which has 
Newtonian gravity as a limiting case for low densities and low velocities)
unless we have \emph{very} good reasons!  So we have to think about things
carefully, and weigh the evidence on both sides.

If I could duplicate myself, I'd have one copy try to get to the bottom
of this dark matter / MOND puzzle.  But I can't...

... so if you're an expert who knows a lot about this, let me 
know what you think - or better yet, post an article about this to 
sci.physics.research!

By the way, you can see lots of photos of the Marseille conference
here:

17) John Baez, Marseille, <A HREF = "http://math.ucr.edu/home/baez/marseille/">http://math.ucr.edu/home/baez/marseille/</A>

Almost everyone working on loop quantum gravity and spin foams can
be seen here!


\par\noindent\rule{\textwidth}{0.4pt}
<B>Addendum:</B>
A few people took me up on my request.  

Steve Carlip wrote:

\begin{quote}
John Baez wrote:


\begin{verbatim}

> So, even if MOND is wrong, there may need to be some reason why it <em>acts</em>
> like it's right!  Apparently even some proponents of dark matter agree 
> with this.  
\end{verbatim}
    


Try this:

M. Kaplinghat and M. S. Turner, "How Cold Dark Matter Theory Explains 
Milgrom's Law," 
<A HREF = "http://www.arXiv.org/abs/astro-ph/0107284">astro-ph/0107284</A>, 
Astrophys. J. 569 (2002) L19.  Note
that this analysis also explains why the "critical acceleration"
in MOND does \emph{not} apply at cluster scales.  There is some debate over
these results, but the paper is certainly worth reading.

Steve Carlip
\end{quote}

Rein Halbersma wrote:

\begin{quote}
Dear John Baez,

Your writings in Week 206 brought back some vivid memories from the good old 
days in graduate school with all-night philosophical discussions! In your 
Finds in Week 206 you discuss the MOND-framework of Milgrom and asked for 
detailed critique of it. A few years ago the authors Scott, White, Cohn and 
Pierpaoli (<A HREF = "http://www.arXiv.org/astro-ph/0104435">astro-ph/0104435</A>) published precisely such an account. 
Hopefully it is of use to you.

As an aside, my connection with the whole MOND story is this: I have a PhD in 
high-energy physics from Groningen University (my advisor was Eric 
Bergshoeff, one of the inventors of the supermembrane). While I was working 
as a graduate student in string theory & conformal supergravity, a 
roommate of mine, Roland Eppinga, was an undergraduate student for Robert 
Sanders, who is an astronomy professor in Groningen. My friend was assigned a 
project involving cosmological simulations within the MOND-framework. 
Needless to say, we had many discussions on MOND in which my esthetical 
views of general relativity were put to the test by the need to fit a 
damn rotational curve. 
My personal view is that MOND is indeed too ugly to be true. Or as Einstein 
would have said, if Nature is not described by relativity, then God 
designed it badly! 
Best wishes,
Rein Halbersma
\end{quote} 

Christine Dantas wrote:

\begin{quote}
Hello all,
Concerning MOND x GR, the recent paper by Bekenstein seems to be 
a relevant contribution to this issue (see below).
Regards,<BR>
Christine Dantas<BR>
INPE/Brazil<BR>

<A HREF = "http://www.arXiv.org/abs/astro-ph/0403694">astro-ph/0403694</A><BR>  
Relativistic gravitation theory for the MOND paradigm<BR>
Jacob D. Bekenstein<BR>

The modified newtonian dynamics (MOND) paradigm of Milgrom can boast
of a number of successful predictions regarding galactic dynamics;
these are made without the assumption that dark matter plays a
significant role. MOND requires gravitation to depart from Newtonian
theory in the extragalactic regime where dynamical accelerations are
small. So far relativistic gravitation theories proposed to underpin
MOND have either clashed with the post-Newtonian tests of general
relativity, or failed to provide significant gravitational lensing, or
violated hallowed principles by exhibiting superluminal scalar waves
or an a priori vector field. We develop a relativistic MOND inspired
theory which resolves these problems. In it gravitation is mediated by
metric, a scalar field and a 4-vector field, all three dynamical. For
a simple choice of its free function, the theory has a Newtonian limit
for nonrelativistic dynamics with significant acceleration, but a MOND
limit when accelerations are small. We calculate the &beta; and \gamma 
PPN coefficients showing them to agree with solar system
measurements. The gravitational light deflection by nonrelativistic
systems is governed by the same potential responsible for dynamics of
particles. Consequently, the new theory predicts gravitational lensing
by extragalactic structures that cannot be distinguished from that
predicted within the dark matter paradigm by general
relativity. Cosmological models based on the theory are quite similar
to those based on general relativity; they predict slow evolution of
the scalar field. For a range of initial conditions, this last result
makes it easy to rule out superluminal propagation of metric, scalar
and vector waves.
\end{quote}

Ethan Vishniac wrote:
\begin{quote}
Hi,

    I don't have a reference for a skeptical review of MOND.  As you 
might
expect, this is considered a fringe hypothesis by most.  However, there 
is
an interesting paper you should see: 
<A HREF = "http://www.arXiv.org/abs/astro-ph/0312273">astro-ph/0312273</A>.
    Briefly, they examine a galaxy cluster with a strong sub-cluster, 
which has
just passed through the main cluster for the first time (probably).  
Most of the
baryonic mass is in the hot gas (by a factor of ten) so the initial 
pass has
stripped the gas out of the sub-cluster.  In fact, in the X-rays the 
subcluster is
not evident.  If stars and gas are all there is then there is no 
significant
mass concentration associated with the sub-cluster.

     However, the sub-cluster is quite easy to see in the gravitational 
lensing
map.  Evidently, the mass of the sub-cluster has not been significantly 
reduced
by losing all of the gas.   (That is, the mass to light ratio for the 
sub-cluster is what
one would expect for an isolated system.)

    This looks like a simple demonstration that most of the mass in 
galaxy clusters
is non-luminous and dissipationless.

    There have also been attempts to disprove MOND by comparing time 
delays in
strong lensing systems with MOND based models.  Unfortunately, the real 
problem
here is that there is no clear set of predictions for MOND.
         Ethan Vishniac
\end{quote}

He also wrote:

\begin{quote}
    BTW, one way to address MOND on its own terms is to try to follow 
galactic
rotation curves out to very very great distances.  If the dark matter 
model is
correct, they will eventually turn over and fall as r^{-1/2}.  
This is hard, perhaps
impossible, using gas.  There is some work using the velocity 
dispersion of
satellite galaxies around otherwise isolated bright galaxies (Prada et 
al.,
ApJ 598, 260-,2003).  (The Sloan Digital Sky Survey makes it possible to
get good statistics for very weak signals.)  They claim to have detected
a drop in the velocity dispersion by a factor of 2 between 20 and 350
kpc.  This is roughly in line with expectations from cosmological 
simulations,
and stands in contradiction to what one would expect from MOND.
   Finally, one can try to measure the shape of galaxy halos using weak 
lensing.
The line of reasoning is a bit indirect, but the point is that an 
elliptical or disk-like
distribution of mass at small radii gives rise to nearly spherical 
equipotential
surfaces at large radii.  On the other hand, dark matter halos are 
generally
triaxial, and will appear elliptical on the sky.   Hoekstra et al. 
(Astrophysical Journal,
606, 67-77, 2004) have done this and claim a strong elliptical signal 
in the
weak lensing data.
\end{quote}

Finally, Renate Loll corrected an oversimplification in my
account of her model:
\begin{quote}
[...] I never claimed the geometries
we find are nice \emph{and smooth}, I think they almost certainly
will be fairly wild individually, even at relatively large
scales. Like the particle paths in the quantum mechanical
path integral, the individual histories should not be taken
too literally, the physics'll all be in suitable expectation
values of course.
\end{quote}

\par\noindent\rule{\textwidth}{0.4pt}
<em>When I am working on a problem, I never think about
beauty. I think only about how to solve the problem.
But when I have finished, if the solution is not
beautiful, I know it is wrong.</em> - Buckminister Fuller
\par\noindent\rule{\textwidth}{0.4pt}

% </A>
% </A>
% </A>
