
% </A>
% </A>
% </A>
\week{May 17, 1996}


I will continue to take a break from the tale of n-categories.  As the
academic year winds to an end, an enormous pile of articles and books is
building up on my desk.  I can kill two birds with one stone if I list
some of them while filing them.  Here is a sampling:

1) Advances in Applied Clifford Algebras, ed. Jaime Keller.
(Subscriptions are available from Mrs. Irma Aragon, F. Q., UNAM,
Apartado 70-528, 04510 Mexico, D.F., MEXICO, for US $10 per year.)

This is a homegrown journal for fans of Clifford algebras.  What are
Clifford algebras?  Well, let's start at the beginning, with the
quaternions.... 

As J. Lambek has pointed out, not many mathematicians can claim to have
introduced a new kind of number.  One of them was the Sir William Rowan
Hamilton.  He knew about the real numbers R, of course, and also the
complex numbers C, which are the reals with a square root of -1, usually
called i, thrown in.  Why not try putting in another square root of -1?
This might give a 3-dimensional algebra that'd help with 
3-dimensional space as much as the complex numbers help with
2 dimensions.  He tried this but couldn't get division to work out well.
He struggled this for a long time.  On the 16th of October, 1843, he was
walking along the Royal Canal with his wife to a meeting of the Royal
Irish Academy when he had a good idea: "...there dawned on me the notion
that we must admit, in some sense, a fourth dimension of space for the
purpose of calculating with triples ... An electric circuit seemed to
close, and a spark flashed forth."  He carved the decisive relations

i^{2} = j^{2} = k^{2} = ijk = -1

in the stone of Brougham Bridge as he passed it.  This was bold: a
\emph{noncommutative} algebra, since ij = -ji, jk = -kj, and ik = -ki follow
from the above equations.  These are the quaternions, which now we call H
after Hamilton.

Hamilton wound up spending much of his time on quaternions.  The
lawyer and mathematician Arthur Cayley heard Hamilton lecture on
quaternions and - I imagine - was influenced by this to invent his
"octonions", an 8-dimensional nonassociative algebra in which division
still works nicely.  For more on quaternions, octonions, and the general
subject of division algebras, try "<A HREF = "week59.html">week59</A>" and "<A HREF = "week61.html">week61</A>".  

In 1845, two years after the birth of the quaternions, the visionary
William Clifford was born in Exeter, England.  He only lived to
the age of 37: despite suffering from lung disease, he worked with
incredible intensity, and his closest friend wrote that "He could
not be induced, or only with the utmost difficulty, to pay even
moderate attention to the cautions and observances which are commonly
and aptly described as 'taking care of one's self'".  But
in his short life, he pushed quite far into the mathematics that
would become the physics of the 20th century.  He studied the geometry
of Riemann and prophetically envisioned general relativity in 1876, in
the following famous remarks:

\begin{quote}
"Riemann has shown that as there are different kinds of lines and
surfaces, so there are different kinds of space of three dimensions; and
that we can only find out by experience to which of these kinds the
space in which we live belongs.  I hold in fact

(1) That small portions of space \emph{are} in fact of a nature analogous
to little hills on a surface which is on the average flat; namely, that
the ordinary laws of geometry are not valid for them.

(2)  That this property of being curved or distorted is continually
being passed on from one portion of space to another after the manner of
a wave.

(3)  That this variation of the curvature of space is what really
happens in that phenomenon which we call the \emph{motion of matter},
whether ponderable or etherial.

(4)  That in the physical world nothing else takes place but this
variation, subject (possibly) to the law of continuity. 

\end{quote}
He also substantially generalized Hamilton's quaternions, dropping the
condition that one have a division algebra, and focusing on the aspects
crucial to n-dimensional geometry.  He obtained what we call the
Clifford algebras.  

What's a Clifford algebra?  Well, there are various flavors.  But one of
the nicest - let's call it C_{n} - is just the associative algebra
over the real numbers generated by n anticommuting square roots of -1.
That is, we start with n fellows called

e_{1}, ... , e_{n}

and form all formal products of them, including the empty product, which
we call 1.  Then we form all real linear combinations of these products,
and then we impose the relations

e_{i}^{2} = -1

e_{i} e_{j} = - e_{j} e_{i}.

What are these algebras like?  Well, C_{0} 
is just the real numbers, since
none of these e_{i}'s are thrown into the stew.  C_{1} 
has one square root of minus 1, so it is just the complex numbers.  
C_{2} has two square roots of minus 1, e_{1} and e_{2},
with 

e_{1} e_{2} = - e_{2} e_{1}.
Thus C_{2} is just the quaternions, with 
e_{1}, e_{2}, and e_{1} e_{2} corresponding
to Hamilton's i, j, and k.  

How about the C_{n} for larger values of n?  
Well, here is a little table up to n = 8:

\begin{verbatim}

C_{0}   R
C_{1}   C
C_{2}   H
C_{3}   H + H
C_{4}   H(2) 
C_{5}   C(4)
C_{6}   R(8)
C_{7}   R(8) + R(8)
C_{8}   R(16)
\end{verbatim}
    
What do these entries mean?  Well, R(n) means the n x n matrices with
real entries.  Similarly, C(n) means the n x n complex matrices, and
H(n) means the n x n quaternionic matrices.  All these become algebras
with the usual matrix addition and matrix multiplication.  Finally, if A
is an algebra, A + A means the algebra consisting of pairs of guys in A,
with the obvious rules for addition and multiplication:

(a, a') + (b, b') = (a + b, a' + b')

(a, a') (b, b') = (ab, a'b')

You might enjoy checking some of these descriptions of the Clifford
algebras C_{n} for n up to 8.  You have to find the 
"isomorphism" - the
correspondence between the Clifford algebra and the algebra I claim is
really the same.  This gets pretty tricky when n gets big.  

How about n larger than 8?  Well, here a remarkable fact comes into
play.  Clifford algebras display a certain sort of "period 8"
phenomenon.  Namely, C_{n+8} consists of 16 x 16 matrices with entries
in C_{n} !   For a proof you might try 

2) H. Blaine Lawson, Jr. and Marie-Louise Michelson, "Spin Geometry",
Princeton U. Press, Princeton, 1989.

or

3) Dale Husemoller, "Fibre Bundles", Springer-Verlag, Berlin, 1994.

These books also describe some of the amazing consequences of this
periodicity phenomenon.  The topology of n-dimensional manifolds is very
similar to the topology of (n+8)-dimensional manifolds in some subtle
but important ways!  I should describe this "Bott periodicity" sometime,
but for now let me leave it as a tantalizing mystery.  

I will also have to take a rain check when it comes to describing the
importance of Clifford algebras in physics... let me simply note that
they are crucial for understanding spin-1/2 particles.  I talked a bit
about this in "<A HREF = "week61.html">week61</A>".  

The "Spin Geometry" book goes into a lot of detail on Clifford algebras,
spinors, the Dirac equation and more.  The "Fibre Bundles" book
concentrates on the branch of topology called K-theory, and uses this
together with Clifford algebras to tackle various subtle questions, such
as how many linearly independent vector fields you can find on a sphere.

4) Ralph L. Cohen, John D. S. Jones, and Graeme B. Segal, Morse theory
and classifying spaces, preprint as of Sept. 13, 1991.

This is a nice way to think about what's really going on with Morse
theory.  In Morse theory we study the topology of a compact Riemannian
manifold by putting a "Morse function" on it: a real-valued smooth
function with only nondegenerate critical points.  The gradient of this
function defines a vector field and we use the way points flow along
this vector field to chop the manifold up into convenient pieces or
"cells".  A while back, Witten discovered, or rediscovered, a very cute
way to compute a topological invariant called the "homology" of the
invariant using Morse theory.  (I've heard that this was previously
known and then largely forgotten.)  

Here the authors refine this construction.  They cook up a category C
from the Morse function: the objects of C are critical points of the
Morse function, and the morphisms are piecewise gradient flow lines.
This is a topological category, meaning that for any pair of objects x
and y the morphisms in hom(x,y) form a topological space, and
composition is a continuous map.  There is a standard recipe to
construct the "classifying space" of any topological category, invented
by Segal in the following paper:

5) Graeme B. Segal, Classifying spaces and spectral sequences, Pub. IHES 34
(1968), 105-112.  

I described classifying spaces for discrete groups in "<A HREF = "week70.html">week70</A>", and the
more general case of discrete groupoids in "<A HREF = "week75.html">week75</A>".  The construction
for topological categories is similar: we make a big space by sticking
in one point for each object, one edge for each morphism, one triangle
for each composable pair of morphisms:


\begin{verbatim}

      y
     / \                       f: x \to  y
    f   g                      g: y \to  z
   /     \                    fg: x \to  z
  x--fg---z

\end{verbatim}
    
and so on.  The only new trick is to make sure this space gets a
topology in the right way using the topologies on the spaces hom(x,y).  

Anyway, if we form this classifying space from the topological
category C coming from the Morse function on our manifold M, we get a
space homotopic to M!  In other words, for many topological purposes the
category C is just as good as the manifold M itself.

6) Ross Street, Descent theory, preprint of talks given at Oberwolfach,
Sept. 17-23, 1995.  

Ross Street, Fusion operators and cocycloids in monoidal categories, 
preprints.

Street is one of the gurus of n-category theory, which he notes "might
be called post-modern algebra (or even `post-modern mathematics' since
geometry and algebra are handled equally well by higher categories)."
His paper on "Descent theory" serves as a rapid introduction to
n-categories.  But the real point of the paper is to explain the role
n-categories play in cohomology theory: in particular, how to do
cohomology with coefficients in an \omega -category!

7) Viqar Husain, Intersecting-loop solutions of the hamiltonian constraint
of quantum general relativity, Nucl. Phys. B313 (1989), 711-724.

Viqar Husain and Karel V. Kuchar, General covariance, new variables,
and dynamics without dynamics, Phys. Rev. D 42 (1990), 4070-4077.

Viqar Husain, Einstein's equations and the chiral model, to appear in
Phys. Rev. D, preprint available as <A HREF = "http://xxx.lanl.gov/abs/gr-qc/9602050">gr-qc/9602050</A>.

Viqar is one of the excellent younger folks at the Center for
Gravitational Physics and Geometry at Penn State; I only had a bit of
time to speak with him during my last visit there, but I got some of his
papers.  The first paper is from the good old days when folks were just
beginning to find explicit solutions of the constraints of quantum
gravity using the loop representation - it's still worth reading!  The
second introduced a field theory now called the Husain-Kuchar model,
which has the curious property of resembling gravity without the
dynamics.  The third studies 4-dimensional general relativity assuming
there are two commuting spacelike Killing vector fields.  Here he
reduces the theory to a 2-dimensional theory which appears to be completely
integrable - though it has not been proved to be so in the sense of
admitting a complete set of Poisson-commuting conserved quantities.  

8) The Interface of Knots and Physics, ed. Louis H. Kauffman, Proc.
Symp. Appl. Math. 51, American Mathematical Society, Providence, Rhode
Island, 1996.  

This slim volume contains the proceedings of an AMS "short course" on
knots and physics held in San Francisco in January 1995, namely:

Louis H. Kauffman, Knots and statistical mechanics
Ruth J. Lawrence, An introduction to topological field theory
Dror Bar-Natan, Vassiliev and quantum invariants of braids
Samuel J. Lomonaco, The modern legacies of Thomson's atomic vortex
                    theory in classical electrodynamics
John C. Baez, Spin networks in nonperturbative quantum gravity

\par\noindent\rule{\textwidth}{0.4pt}
<p align = center><em>
William Kingon Clifford<br>
Born May 4th, 1845<br>
Died March 3rd, 1879<br>
</p>
<p align = center>
I was not, and was conceived<br>
I loved, and did a little work<br>
I am not, and grieve not.<br>
</p>
<p align = center>
And<br>
</p>
<p align = center>
Lucy, his wife<br>
Died April 21st, 1929<br>
<p align = center>
Oh, two such silver currents when they join<br>
Do glorify the banks that bound them in.<br>
</em></p>


\par\noindent\rule{\textwidth}{0.4pt}

% </A>
% </A>
% </A>
