
% </A>
% </A>
% </A>
\week{August 6, 2008 }


This Week will be all about Frobenius algebras and modular tensor
categories.  But first, here's a beautiful photo of Io, the volcanic
moon of Jupiter that I introduced back in "<a href =
"week266.html">week266</A>":

<div align = "center">
<a href = "io_and_jupiter.jpg">
<img width = "600" src = "io_and_jupiter.jpg">
% </a>
</div>

1) NASA Photojournal, A new year for Jupiter and Io,
<a href = "http://photojournal.jpl.nasa.gov/catalog/PIA02879">http://photojournal.jpl.nasa.gov/catalog/PIA02879</a>

Io looks awfully close to Jupiter here!  It's actually 2.5 Jupiter 
diameters away... but that's close enough to cause the intense 
tidal heating that leads to sulfur volcanoes.

I told you about Frobenius algebras in "<a href =
"week174.html">week174</A>" and "<a href =
"week224.html">week224</A>", but I think it's time to talk about
them again!  In the last few weeks, I've run into them - and their
generalizations - in a surprising variety of ways.

First of all, Jamie Vicary visited me here in Paris and explained 
how certain Frobenius algebras can be viewed as classical objects 
living in a quantum world - governed by quantum logic.  

Mathematicians in particular are used to thinking of the quantum 
world as a mathematical structure resting on foundations of classical 
logic: first comes set theory, then Hilbert spaces on top of that.  
But what if it's really the other way around?  What if classical 
mathematics is somehow sitting inside quantum theory?   The world
is quantum, after all.

There are a couple of papers so far that discuss this provocative 
idea:

2) Bob Coecke and Dusko Pavlovic, Quantum measurements without
sums, in The Mathematics of Quantum Computation and 
Technology, eds. Chen, Kauffman and Lomonaco,  Chapman and Hall/CRC, 
New York, pp. 559-596.  Also available as
<a href = "http://arxiv.org/abs/quant-ph/0608035">arXiv:quant-ph/0608035</A>.

3) Jamie Vicary, Categorical formulation of quantum algebras,
available as <a href = "http://arxiv.org/abs/0805.0432">arXiv:0805.0432</a>.


Second, Paul-Andr&eacute; Melli&egrave;s, the computer scientist and logician
who's my host here, has been telling me how logic can be nicely
formulated in certain categories - "*-autonomous categories"
- which can be seen as \emph{categorified} Frobenius algebras.  Here
the idea goes back to Ross Street:

4) Ross Street, Frobenius monads and pseudomonoids, J. Math. Physics 
45 (2004) 3930-3948.  Available as 
<a href = "http://www.math.mq.edu.au/~street/Frob.pdf">http://www.math.mq.edu.au/~street/Frob.pdf</a>

Paul-Andr&eacute; is teaching a course on this and related topics;
you can see the slides for his course here:

5) Paul-Andr&eacute; Melli&egrave;s, Groupoides quantiques et logiques
tensorielles: une introduction, course notes at <a href =
"http://www.pps.jussieu.fr/~mellies/teaching.html">http://www.pps.jussieu.fr/~mellies/teaching.html</a>

See especially the fourth class.  

But to get you ready for this material, I should give a quick
introduction to the basics!

If you're a normal mathematician, the easiest definition of
"Frobenius algebra" is something like this.  For starters, it's
an "algebra": a vector space with an associative product that's 
linear in each argument, and an identity element 1.  But what 
makes it "Frobenius" is that it's got a nondegenerate bilinear 
form g satisfying this axiom:

g(ab,c) = g(a,bc)

I'm calling it "g" to remind geometers of how nondegenerate
bilinear forms are used as "metrics", like the metric tensor
at a point of a Riemannian or Lorentzian manifold.  But beware: we'll
often work with complex instead of real vector spaces.  And, we won't
demand that g(a,b) = g(b,a), though this holds in many examples.

Let's see some examples!  For starters, we could take the algebra 
of n \times  n matrices and define

g(a,b) = tr(ab)

where "tr" is the usual trace.  Or, we could perversely stick
any nonzero number in this formula, like

g(a,b) = -37 tr(ab)

Or, we could take a bunch of examples like this and take their direct
sum.  This gives us the most general "semisimple" Frobenius
algebra.

So, semisimple Frobenius algebras are pathetically easy to classify.  
There's also a vast wilderness of non-semisimple ones, which will 
never be classified.  But for a nice step in this direction, 
try Prop. 2 in this paper:

6) Steve Sawin, Direct sum decompositions and indecomposable
TQFTs, J. Math. Phys. 36 (1995) 6673-6680.  Also available
as <a href = "http://arxiv.org/abs/q-alg/9505026">q-alg/9505026</a>.

This classifies all commutative Frobenius algebras that are
"indecomposable" - not a direct sum of others.

Note the mention of topological quantum field theories, or TQFTs.
Here's why.  Suppose you have an n-dimensional TQFT.  This gives
vector spaces for (n-1)-dimensional manifolds describing possible
choices of "space", and operators for n-dimensional
manifolds going between these, which describe possible choices of
"spacetime".

So, it gives you some vector space for the (n-1)-sphere, say A.  
And, this vector space is a commutative Frobenius algebra!  

Let me sketch the proof.  I'll use lots of hand-wavy reasoning,
which is easy to make rigorous using the precise definition of
a TQFT.  

For starters, there's the spacetime where two spherical universes 
collide and fuse into one.  Here's what it looks like for n = 2:

<div align = "center">
<img src = "multiplication.jpg">
</div>

This gives the vector space A a multiplication: 

$$
m: A \otimes  A  \to  A
   a \otimes  b |\to  ab
$$
    
Next there's the spacetime where a spherical universe appears 
from nothing - a "big bang":

<div align = "center">
<img src = "unit.jpg">
</div>

This gives A an identity element, which we call 1:

$$
i: C  \to  A
   1 |\to  1
$$
    
Here C stands for the complex numbers, but mathematicians could
use any field. 

Now we can use topology to show that A is an algebra - namely, 
that it satisfies the associative law:

(ab)c = a(bc) 

and the left and right unit laws: 

1a = a = 1

<div align = "center">
<img src = "monoid_laws.jpg">
</div>

But why is it a Frobenius algebra?  To see this, let's switch the 
future and past in our previous argument!  The spacetime where 
a spherical universe splits in two gives A a "comultiplication":

\Delta : A \to  A \otimes  A

<div align = "center">
<img src = "comultiplication.jpg">
</div>

The spacetime where a spherical universe disappears into nothing -
a "big crunch" - gives A a trace, or more precisely a "counit":

e: A \to  C

<div align = "center">
<img src = "counit.jpg">
</div>

And, a wee bit of topology shows that these make A into a
"coalgebra", satisfying the "coassociative law"
and the left and right "counit laws":

<div align = "center">
<img src = "comonoid_laws.jpg">
</div>

Everything has just been turned upside down!

It's easy to see that the multiplication on A is commutative, 
at least for n > 1:

<div align = "center">
<img src = "commutative_law.jpg">
</div>

Similarly, the comultiplication is "cocommutative" - just
turn the above proof upside down!

But why is A a Frobenius algebra?  The point is that the algebra
and coalgebra structures interact in a nice way.  We can use the 
product and counit to define a bilinear form:

g(a,b) = e(ab)

This is just what we did in our matrix algebra example, where e
was a multiple of the trace.                  

We can also think of g as a linear operator 

g: A \otimes  A \to  C

But now we see this operator comes from a spacetime where two 
universes collide and then disappear into nothing:

<div align = "center">
<img src = "pairing.jpg">
</div>
               
           
To check the Frobenius axiom, we just use associativity:

g(ab,c) = e((ab)c) = e(a(bc)) = g(a,bc)

<div align = "center">
<img src = "frobenius_pairing.jpg">
</div>

But why is g nondegenerate?  I'll just give you a hint.  
The bilinear form g gives a map from A to the dual vector space A*:

a |\to  g(a,-)

Physicists would call this map "lowering indices with the metric
g".  To show that g is nondegenerate, it's enough to find an
inverse for this map, which physicists would call "raising
indices".  This should be a map going back from A* to A.  To
build a map going back like this, it's enough to get a map

h: C \to  A \otimes  A

and for this we use the linear operator coming from this spacetime:

<div align = "center">
<img src = "copairing.jpg">
</div>

The fact that "raising indices" is the inverse of "lowering
indices" then follows from the fact that you can take a zig-zag
in a piece of pipe and straighten it out!

<div align = "center">
<img src = "zigzag.jpg">
</div>


So, any n-dimensional TQFT gives a Frobenius algebra, and in
fact a commutative Frobenius algebra for n > 1.  

In general there's more to the TQFT than this Frobenius algebra, 
since there are spacetimes that aren't made of the building 
blocks I've drawn.  But in 2 dimensions, every spacetime can be 
built from these building blocks: the multiplication and unit, 
comultiplication and counit.  So, with some work, one can show that

<div align = "center">
                  A 2D TQFT IS THE SAME AS <br/>
              A COMMUTATIVE FROBENIUS ALGEBRA.
</div>
 
This idea goes back to Dijkgraaf:

7) Robbert H. Dijkgraaf, A Geometric Approach To Two-Dimensional 
Conformal Field Theory, PhD thesis, University of Utrecht, 1989.

and a formal proof was given by Abrams:

8) Lowell Abrams, Two-dimensional topological quantum field theories
and Frobenius algebra, Jour. Knot. Theory and its Ramifications 5
(1996), 569-587.

This book is probably the best place to learn the details:

9) Joachim Kock, Frobenius Algebras and 2d Topological Quantum 
Field Theories, Cambridge U. Press, Cambridge, 2004.

but for a goofier explanation, try this:

10) John Baez, Winter 2001 Quantum Gravity Seminar, Track 1,
weeks 11-17, <a href = "http://math.ucr.edu/home/baez/qg-winter2001/">http://math.ucr.edu/home/baez/qg-winter2001/</a>

To prove the equivalence of 2d TQFTs and commutative Frobenius 
algebras, it's handy to use a different definition of Frobenius 
algebra, equivalent to the one I gave.  I said a Frobenius algebra 
was an algebra with a nondegenerate bilinear form satisfying 

g(ab,c) = g(a,bc).  

But this is equivalent to having an algebra that's also a coalgebra,
with multiplication and comultiplication linked by the "Frobenius 
equations":

(\Delta  \otimes  1_{A}) (1_{A} \otimes  m) = \Delta  m =
(m \otimes  1_{A}) (1_{A} \otimes   \Delta ) 

These equations are a lot more charismatic in pictures!  

<div align = "center">
<img src = "frobenius_laws.jpg">
</div>

We can also 
interpret them conceptually, as follows.  If you have an algebra
A, it becomes an (A,A)-bimodule in an obvious way... well, obvious 
if you know what this jargon means, at least.   A \otimes  A also 
becomes an (A,A)-bimodule, like this:

a (b \otimes  c) d = ab \otimes  cd

Then, a Frobenius algebra is an algebra that's also a coalgebra,
where the comultiplication is an (A,A)-bimodule homomorphism!
This scary sentence has the Frobenius equations hidden inside it.

The Frobenius equations have a fascinating history, going back to 
Lawvere, Carboni and Walters, Joyal, and others.  Joachim Kock's 
website includes some nice information about this.  Read what Joyal 
said about Frobenius algebras that made Eilenberg ostentatiously 
rise and leave the room!

11) Joachim Kock, Remarks on the history of the Frobenius equation, 
<a href = "http://mat.uab.es/~kock/TQFT.html#history">http://mat.uab.es/~kock/TQFT.html#history</a>

The people I just mentioned are famous category theorists.  They
realized that Frobenius algebra can be generalized from the category
of vector spaces to any "monoidal category" - that is, any
category with tensor products.  And if this monoidal category is
"symmetric", it has an isomorphism between X \otimes  Y and Y
\otimes  X for any objects X and Y, which lets us generalize the notion
of a \emph{commutative} Frobenius object.

For a nice intro to these ideas, try the slides of this talk:

12) Ross Street, Frobenius algebras and monoidal category, talk at
the annual meeting of the Australian Mathematical Society, September
2004, available at <a href = "http://www.maths.mq.edu.au/~street/FAMC.pdf">http://www.maths.mq.edu.au/~street/FAMC.pdf</a>

These ideas allow for a very slick statement of the slogan I 
mentioned:

<div align = "center">
                    A 2D TQFT IS THE SAME AS <br/>
                A COMMUTATIVE FROBENIUS ALGEBRA.
</div>

For any n, there's a symmetric monoidal category nCob, with: 

<ul>
<li>
 compact oriented (n-1)-manifolds as objects; 
</li>
<li>
 compact oriented n-dimensional cobordisms as morphisms. 
</li>
</ul>

The objects are choices of "space", and the morphisms are
choices of "spacetime".

The sphere is a very nice object in nCob; let's call it A.  Then all 
the pictures above show that A is a Frobenius algebra in nCob!  It's 
commutative when n > 1.  And when n = 2, that's all there is to say!  
More precisely:

<div align = "center">
                          2Cob IS THE <br/>
                FREE SYMMETRIC MONOIDAL CATEGORY </br>
              ON A COMMUTATIVE FROBENIUS ALGEBRA.
</div>

So, to define a 2d TQFT, we just need to pick a commutative Frobenius
algebra in Vect (the category of vector spaces).  By
"freeness", this determines a symmetric monoidal functor

Z: 2Cob \to  Vect

and that's precisely what a 2d TQFT is!  

If you don't know what a symmetric monoidal functor is, don't worry - 
that's just what I'd secretly been using to translate from pictures 
of spacetimes to linear operators in my story so far.  You can get a
precise definition from those seminar notes of mine, or many other 
places. 

Now let's talk about some variations on the slogan above.

We can think of the 2d spacetimes we've been drawing as the
worldsheets of "closed strings" - but ignoring the geometry
these worldsheets usually have, and keeping only the topology.  So,
some people call them "topological closed strings".

We can also think about topological \emph{open} strings, where we
replace all our circles by intervals.  Just as the circle gave a
commutative Frobenius algebra, an interval gives a Frobenius algebra
where the multiplication comes from two open strings joining
end-to-end to form a single one:

<div align = "center">
<img src = "frobenius_algebra.jpg">
</div>

This open string Frobenius algebra
is typically noncommutative - draw the picture and see!  But, it's
still "symmetric", meaning:

g(a,b) = g(b,a)

<div align = "center">
<img src = "symmetric_pairing_2.jpg">
</div>

This is very nice.  But physically, open strings like to join together
and form closed strings, so it's better to consider closed and open
strings together in one big happy family... or category.  

The idea of doing this for topological strings was developed by
Moore and Segal:

13) Greg Moore, Lectures on branes, K-theory and RR charges, Clay 
Math Institute Lecture Notes (2002), available at 
<a href = "http://www.physics.rutgers.edu/~gmoore/clay1/clay1.html">http://www.physics.rutgers.edu/~gmoore/clay1/clay1.html</a>

Lauda and Pfeiffer developed this idea and proved that this category
has a nice description in terms of Frobenius algebras:

14) Aaron Lauda and Hendryk Pfeiffer, Open-closed strings:
two-dimensional extended TQFTs and Frobenius algebras, Topology
Appl. 155 (2008) 623-666.  Also available as <a href =
"http://arxiv.org/abs/math.AT/0510664">arxiv:math.AT/0510664</a>.

Here's what they prove, encoded as a mysterious slogan:

<div align = "center">
        THE CATEGORY OF OPEN-CLOSED TOPOLOGICAL STRINGS IS THE <br/>
                  FREE SYMMETRIC MONOIDAL CATEGORY <br/>
               ON A "KNOWLEDGEABLE" FROBENIUS ALGEBRA.<br/>
</div>

If you like the pictures I've been drawing so far, you'll \emph{love}
this paper - since that's where I got most of these pictures!
And, it's just the beginning of a longer story where 
Lauda and Pfeiffer build 2d TQFTs using state sum models:

15) Aaron Lauda and Hendryk Pfeiffer, State sum construction of
two-dimensional open-closed topological quantum field theories,
J. Knot Theory and its Ramifications 16 (2007), 1121-1163.  Also
available as <a href =
"http://arXiv.org/abs/math/0602047">arXiv:math/0602047</A>.

This generalizes a construction due to Fukuma, Hosono and Kawai,
explained way back in "<a href = "week16.html">week16</A>"
and also in my seminar notes mentioned above.  Then Lauda and Pfeiffer
use this machinery to study knot theory!

16) Aaron Lauda and Hendryk Pfeiffer, Open-closed TQFTs extend 
Khovanov homology from links to tangles, available as 
<a href = "http://arXiv.org/abs/math/0606331">math/0606331</A>.

Alas, explaining this would be a vast digression.  I want to keep
talking about basic Frobenius stuff.

I guess I should say a bit more about semisimple versus 
non-semisimple Frobenius algebras.  

Way back at the beginning of this story, I said you can get a 
Frobenius algebra by taking the algebra of n \times  n matrices and 
defining

g(a,b) = k tr(ab)

for any nonzero constant k.  Direct sums of these give all the
semisimple Frobenius algebras. 

But any algebra acts on itself by left multiplication:

L_{a}: b |\to  ab

so for any algebra we can try to define 

g(a,b) = tr(L_{a} L_{b})

This bilinear form is nondegenerate precisely when our algebra is
"strongly separable":

17) Marcelo Aguiar, A note on strongly separable algebras, available
at <a href = "http://www.math.tamu.edu/~maguiar/strongly.ps.gz">http://www.math.tamu.edu/~maguiar/strongly.ps.gz</a>

Over the complex numbers, or any field of characteristic zero, an 
algebra is strongly separable iff it's finite-dimensional and 
semisimple.  The story is trickier over other fields - see that 
last paper of Lauda and Pfeiffer if you're interested.

Now, for n \times  n matrices, 

g(a,b) = tr(L_{a} L_{b})

is n times the usual tr(ab).  But it's better, in a way.  The reason
is that for any strongly separable algebra, 

g(a,b) = tr(L_{a} L_{b})

gives a Frobenius algebra with a cute extra property: if we 
comultiply and then multiply, we get back where we started!  

<div align = "center">
<img src = "separability.jpg">
</div>

This is 
easy to see if you write the above formula for g using diagrams.  
Frobenius algebras with this cute extra property are sometimes 
called "special".   


If we use a commutative special Frobenius algebra to get a 2d TQFT, 
it fails to detect handles!  That seems sad.  But these papers:

18) Stephen Lack, Composing PROPs, Theory and Applications of
Categories 13(2004), 147-163.  Available at <a href = "http://www.tac.mta.ca/tac/volumes/13/9/13-09abs.html">http://www.tac.mta.ca/tac/volumes/13/9/13-09abs.html</a>

19) R. Rosebrugh, N. Sabadini and R.F.C. Walters, Generic commutative 
separable algebras and cospans of graphs, Theory and Applications of 
Categories 15 (Proceedings of CT2004), 164-177.  Available at
<a href = "http://www.tac.mta.ca/tac/volumes/15/6/15-06abs.html">http://www.tac.mta.ca/tac/volumes/15/6/15-06abs.html</a>

makes that sad fact seem good!  Namely:

<div align = "center">
      Cospan(FinSet) IS THE <br/> 
      FREE SYMMETRIC MONOIDAL CATEGORY <br/>
           ON A COMMUTATIVE SPECIAL FROBENIUS ALGEBRA.
</div>

Here Cospan(FinSet) is the category of "cospans" of finite sets.
The objects are finite sets, and a morphism from X to Y looks like
this:

\begin{verbatim}
                  X         Y
                   \       /
                   F\     /G
                     \   /
                      v v
                       S
\end{verbatim}
    

If you remember the "Tale of Groupoidication" starting in
"<a href = "week247.html">week247</A>", you'll know about
spans and how to compose spans using pullback.  This is just the same
only backwards: we compose cospans using pushout.

But here's the point.  A 2d cobordism is itself a kind of cospan:

\begin{verbatim}
                  X         Y
                   \       /
                   F\     /G
                     \   /
                      v v
                       S
\end{verbatim}
    
with two collections of circles included in the 2d manifold S.  If we 
take connected components, we get a cospan of finite sets.  Now we've
lost all information about handles!  And the circle - which was a 
commutative Frobenius algebra - becomes a mere one-point set - which 
is a \emph{special} commutative Frobenius algebra.  

Now for a few examples of \emph{non}-semisimple Frobenius algebras.

First, take the exterior algebra \Lambda V over an n-dimensional
vector space V, and pick any nonzero element of degree n - what
geometers would call a "volume form".  There's a unique
linear map

e: \Lambda V \to  C

which sends the volume form to 1 and kills all elements of degree <
n.  This is a lot like "integration" - and so is taking a
trace.  So, you should want to make \Lambda V into a Frobenius
algebra using this formula:

g(a,b) = e(a ^ b)

where ^ is the product in the exterior algebra.  It's easy to see
this is nondegenerate and satisfies the Frobenius axiom:

g(ab,c) = e(a ^ b ^ c) = g(a,bc)

So, it works!  But, this algebra is far from semisimple.  

If you know about cohomology, you should want to copy this trick
replacing the exterior algebra by the deRham cohomology of a compact
oriented manifold, and replacing e by "integration".  It
still works.  So, every compact manifold gives us a Frobenius algebra!

If you know about algebraic varieties, you might want to copy \emph{this}
trick replacing the compact manifold by a complex projective variety.  
I'm no expert on this, but people seem to say that it only works for
Calabi-Yau varieties.  Then you can do lots of cool stuff:

20) Kevin Costello, Topological conformal field theories and 
Calabi-Yau categories, available as <a href = "http://arxiv.org/abs/math/0412149">arxiv:math/0412149</a>.

Here a "Calabi-Yau category" is just the "many-object" version of
a Frobenius algebra - a Calabi-Yau category with one object is a 
Frobenius algebra.  There's much more to say about this wonderful
paper, but I'm afraid for now you'll have to read it... I'm getting
worn out, and I want to get to the new stuff I just learned!

But before I do, I can't resist rounding off one corner I cut.  I said
that Frobenius algebras show up naturally by taking string theory and
watering it down: ignoring the geometrical structure on our string
worldsheets and remembering only their topology.  A bit more
precisely, 2d TQFTs assign linear operators to 2d cobordisms, but
\emph{conformal} field theories assign operators to 2d cobordisms
\emph{equipped with conformal structures}.  Can we describe
conformal field theories using Frobenius algebras?

Yes!

21) Ingo Runkel, Jens Fjelstad, Jurgen Fuchs, Christoph Schweigert,
Topological and conformal field theory as Frobenius algebras,
available as <a href =
"http://arXiv.org/abs/math/0512076">arXiv:math/0512076</A>.

But, you need to use Frobenius algebras inside a modular tensor
category!  

I wish I had more time to study modular tensor categories, and tell
you all about them.  They are very nice braided monoidal categories
that are \emph{not} symmetric.  You can use them to build 3d topological 
quantum field theories, and they're also connected to other branches
of math.  

For example, you can modular tensor categories consisting of nice
representations of quantum groups.  You can also can get them from 
rational conformal field theories - which is what the above paper by 
Runkel, Fjelstad, Fuchs and Schweigert is cleverly turning around.  
You can also get them from von Neumann algebras!

If you want to learn the basics, this book is great - there's a 
slightly unpolished version free online:

22) B. Bakalov and A. Kirillov, Jr., Lectures on Tensor Categories 
and Modular Functors, American Mathematical Society, Providence, 
Rhode Island, 2001.  Preliminary version available at
<a href = "http://www.math.sunysb.edu/~kirillov/tensor/tensor.html">http://www.math.sunysb.edu/~kirillov/tensor/tensor.html</a>

But if a book is too much for you, here's a nice quick intro.  It 
doesn't say much about topological or conformal field theory, but it 
gives a great overview of recent work on the algebraic aspects of 
tensor categories:

23) Michael M&uuml;ger, Tensor categories: a selective guided tour,
available as <a href =
"http://arxiv.org/abs/0804.3587">arXiv:0804.3587</a>.

Here's a quite different introduction to recent developments, at 
least up to 2004:

24) Damien Calaque and Pavel Etingof, Lectures on tensor categories,
available as <a href = "http://arXiv.org/abs/arXiv:math/0401246">arXiv:math/0401246</A>.

Still more recently, Hendryk Pfeiffer has written what promises to 
be a fundamental paper describing how to think of any modular tensor 
category as the category of representations of an algebraic gadget - 
a "weak Hopf algebra":

25) Hendryk Pfeiffer, Tannaka-Krein reconstruction and a 
characterization of modular tensor categories, available as
<a href = "http://arxiv.org/abs/0711.1402">arXiv:0711.1402</a>.

And here's a paper that illustrates the wealth of examples:

26)  Seung-moon Hong, Eric Rowell, Zhenghan Wang, On exotic modular
tensor categories, available as <a href = "http://arxiv.org/abs/0710.5761">arXiv:07108.5761</a>.

The abstract of this makes me realize that people have bigger hopes of
understanding all modular tensor categories than I'd imagined:

\begin{quote}
  It has been conjectured that every (2+1)-dimensional TQFT is a
  Chern-Simons-Witten (CSW) theory labelled by a pair (G,k), where G
  is a compact Lie group, and k in H<sup>4</sup>(BG,Z) is a cohomology
  class.  We study two TQFTs constructed from Jones' subfactor theory
  which are believed to be counterexamples to this conjecture: one is
  the quantum double of the even sectors of the E<sub>6</sub>
  subfactor, and the other is the quantum double of the even sectors
  of the Haagerup subfactor. We cannot prove mathematically that the
  two TQFTs are indeed counterexamples because CSW TQFTs, while
  physically defined, are not yet mathematically constructed for every
  pair (G,k).  The cases that are constructed mathematically include:

<ul>
<li>
  G is a finite group - the Dijkgraaf-Witten TQFTs; 
</li>
<li>
  G is a torus T<sup>n</sup>;
</li>
<li>
  G is a connected semisimple Lie group - the Reshetikhin-Turaev TQFTs.
</li>
</ul>

  We prove that the two TQFTs are not among those mathematically
  constructed TQFTs or their direct products.  Both TQFTs are of the
  Turaev-Viro type: quantum doubles of spherical tensor categories.
  We further prove that neither TQFT is a quantum double of a braided
  fusion category, and give evidence that neither is an orbifold or
  coset of TQFTs above. Moreover, the representation of the braid
  groups from the half E<sub>6</sub> TQFT can be used to build
  universal topological quantum computers, and the same is expected
  for the Haagerup case.
\end{quote}
    

Anyway, now let me say what Vicary and Melli&egrave;s have been explaining 
to me.  I'll give it in a highly simplified form... and all mistakes
are my own.

First, from what I've said already, every commutative special
Frobenius algebra over the complex numbers looks like

C \oplus  C \oplus  ... C \oplus  C

It's a direct sum of finitely many copies of C, equipped with its
god-given bilinear form

g(a,b) = tr(L_{a} L_{b})

So, this sort of Frobenius algebra is just an algebra of complex
functions on a \emph{finite set}.   A map between finite sets gives an
algebra homomorphism going back the other way.  And the algebra 
homomorphisms between two Frobenius algebras of this sort \emph{all}
come from maps between finite sets.

So, the category with:

<ul>
<li>
 commutative special complex Frobenius algebras as objects;
</li>
<li>
 algebra homomorphisms as morphisms
</li>
</ul>

is equivalent to FinSet^{op}.  This means we can find the category
of finite sets - or at least its opposite, which is just as good -
lurking inside the world of Frobenius algebras!  

Coecke, Pavlovic and Vicary explore the ramifications of this result
for quantum mechanics, using Frobenius algebras that are Hilbert
spaces instead of mere vector spaces.  This lets them define 
a "&dagger;-Frobenius algebra" to be one where the
comultiplication and counit are adjoint to the the multiplication
and unit.  They show that making a finite-dimensional Hilbert space
into a commutative special &dagger;-Frobenius algebra is the same
as <i>equipping it with an orthonormal basis</i>.


There's no general
way to duplicate quantum states - "you can't clone a
quantum" - but if you only want to duplicate states lying in a
chosen orthonormal basis you can do it.  So, you can think of
commutative special &dagger;-Frobenius algebras as "classical data
types", which let you duplicate information.  That's what
the comultiplication does: duplicate!

<div align = "center">
<img src = "comultiplication.jpg">
</div>

Any commutative special &dagger;-Frobenius algebra has a finite
set attached to it: namely, the set of basis elements.
So, we now see how to describe finite sets starting from
Hilbert spaces and introducing a notion of "classical data
type" formulated purely in terms of quantum concepts. 

The papers by Coecke, Pavlovic and Vicary go a lot further
than my summary here.  Jamie Vicary even studies how to 
\emph{categorify} everything I've just mentioned!

A subtlety: it's a fun puzzle to show that in any monoidal category,
morphisms between Frobenius algebras that preserve \emph{all} the
Frobenius structure are automatically \emph{isomorphisms}.  See the
slides of Street's talk if you get stuck: he shows how to construct
the inverse, but you still get the fun of proving it works.

So, the category with:

<ul>
<li>
 commutative special complex Frobenius algebras as objects;
</li>
<li>
 Frobenius homomorphisms as morphisms
</li>
</ul>

is equivalent to the \emph{groupoid} of finite sets.  We get
FinSet^{op} if we take algebra homomorphisms, and I guess we
get FinSet if we take coalgebra homomorphisms.

Finally, a bit about categorified Frobenius algebras and logic!

I'm getting a bit tired, so I hope you believe that the concept of
Frobenius algebra can be categorified.  As I already mentioned,
Frobenius algebras make sense in any monoidal category - and then
they're sometimes called "Frobenius monoids".  Similarly,
categorified Frobenius algebras make sense in any monoidal bicategory,
and then they're sometimes called "Frobenius pseudomonoids".
These were introduced in Street's paper "Frobenius monads and
pseudomonoids", cited above - but if you like pictures, you may
also enjoy learning about them here:

27) Aaron Lauda, Frobenius algebras and ambidextrous adjunctions,
Theory and Applications of Categories 16 (2006), 84-122, available at
<a href =
"http://tac.mta.ca/tac/volumes/16/4/16-04abs.html">http://tac.mta.ca/tac/volumes/16/4/16-04abs.html</a>
<br/> Also available as <a href =
"http://arXiv.org/abs/math/0502550">arXiv:math/0502550</A>.

I explained some of the basics behind this paper in "<a href =
"week174.html">week174</A>".

But now, I want to give a definition of *-autonomous categories,
which simultaneously makes it clear that they're natural structures
in logic, and that they're categorified Frobenius algebras!

Suppose A is any category.  We'll call its objects
"propositions" and its morphisms "proofs".
So, a morphism

f: a \to  b


is a proof that a implies b. 

Next, suppose A is a symmetric monoidal category and call the tensor
product "or".  So, for example, given proofs

f: a \to  b,  f': a' \to  b'

we get a proof

f or f': a or a' \to  b or b'


Next, suppose we make the opposite category A^{op} into a
symmetric monoidal category, but with a completely different tensor
product, that we'll call "and".  And suppose we have a
monoidal functor:

not: A \to  A^{op}

So, for example, we have

not(a or b) = not(a) and not(b)

or at least they're isomorphic, so there are proofs going both ways.  

Now we can apply "op" and get another functor I'll also call
"not":

not: A^{op} \to  A

Using the same name for this new functor could be confusing, but it 
shouldn't be.  It does the same thing to objects and morphisms; we're
just thinking about the morphisms as going backwards.
 
Next, let's demand that this new functor be monoidal!  This too is 
quite reasonable; for example it implies that

not(a and b) = not(a) or not(b)

or at least they're isomorphic.

Next, let's demand that this pair of functors:

$$
      not
   --------->
A             A^{op}
  <----------
      not

$$
    
be a monoidal adjoint equivalence.  So, for example, there's a
one-to-one correspondence between proofs

not(a) \to  b

and proofs

not(b) \to  a

Now for the really fun part.
Let's define a kind of "bilinear form":

g: A \times  A \to  Set

where g(a,b) is the set of proofs 

not(a) \to  b

And let's demand that g satisfy the Frobenius
axiom!  In other words, let's suppose there's a natural
isomorphism:

g(a or b, c) &cong; g(a, b or c)

Then A is a "*-autonomous category"!  And this is 
a sensible notion, since it amounts to requiring a natural
one-to-one correspondence between proofs

not(a or b) \to  c

and proofs

not(a) \to  b or c

So, categorified Frobenius algebras are a nice framework for
propositional logic!

In case it slipped by too fast, let me repeat the definition of
*-autonomous category I just gave.  It's a symmetric monoidal
category A with a monoidal adjoint
equivalence called "not" from A (with one tensor product,
called "or") to A^{op} (with another, called 
"and"), such that the functor 

$$
g: A \times  A  \to  Set
   (a,b) |\to  hom(not(a),b)
$$
    

is equipped with a natural isomorphism 

g(a or b, c) &cong; g(a, b or c)

I hope I didn't screw up.  I want this definition to
be equivalent to the usual one,
which was invented by Michael Barr quite a while ago:

28) Michael Barr, *-Autonomous Categories, Lecture Notes in 
Mathematics 752, Springer, Berlin, 1979.

By now *-autonomous categories become quite popular among those
working at the interface of category theory and logic.  And, there
are many ways to define them.  Brady and Trimble found a nice one:

29) Gerry Brady and Todd Trimble, A categorical interpretation 
of C. S. Peirce's System Alpha, Jour. Pure Appl. Alg. 149
(2000), 213-239.  

Namely, they show a *-autonomous category is the same as a symmetric
monoidal category A equipped with a <i>contravariant</i> 
adjoint equivalence

not: A \to  A 

which is equipped with a "strength", and where the unit and counit of 
the adjunction respect this strength.

Later, in his paper "Frobenius monads and pseudomonoids", Street 
showed that *-autonomous categories really do give Frobenius 
pseudomonoids in a certain monoidal bicategory with:

<ul>
<li>
categories as objects;
</li>
<li>
profunctors (also known as distributors) as morphisms;
</li>
<li>
natural transformations as 2-morphisms.
</li>
</ul>

Alas, I'm too tired to explain this now!  It's a slicker way of saying
what I already said.  But the cool part is that this bicategory is
like a categorified version of Vect, with the category of finite sets
replacing the complex numbers.  That's why in logic, the
"nondegenerate bilinear form" looks like

g: A \times  A \to  Set

So: Frobenius algebras are lurking all over in physics, logic
and quantum logic, in many tightly interconnected ways.  There 
should be some unified explanation of what's going on!  Do you have 
any ideas?

Finally, here are two books on math and music that I should read
someday.  The first seems more elementary, the second more advanced:

30) Trudi Hammel Garland and Charity Vaughan Kahn, Math and Music -
Harmonious Connections, Dale Seymour Publications, 1995.  Review 
by Elodie Lauten on her blog Music Underground, 
<a href = "http://www.sequenza21.com/2007/04/microtonal-math-heads.html">http://www.sequenza21.com/2007/04/microtonal-math-heads.html</a>

31) Serge Donval, Histoire de l'Acoustique Musicale (History of 
Musical Acoustics), Editions Fuzaeau, Bressuire, France, 2006.
Review at Music Theory Online, 
<a href = "http://mto.societymusictheory.org/mto-books.html?id=11">http://mto.societymusictheory.org/mto-books.html?id=11</a>

<HR>
 
\textbf{Addenda}: I thank Bob Coecke, Robin Houston, Steve Lack,
Paul-Andr&eacute; Melli&egrave;s,
Todd Trimble, Jamie Vicary, and a mysterious fellow named Stuart for
some very helpful corrections.

You can't really appreciate the pictorial approach to Frobenius
algebras until you use it to prove some things.  Try proving that 
every homomorphism of Frobenius algebras is an isomorphism!  Or
for something easier, but still fun, start by assuming
that a Frobenius algebra is an algebra and coalgebra satisfying
the Frobenius equations

<div align = "center">
<img src = "frobenius_laws.jpg">
</div>

and use this to prove the following facts:

<div align = "center">
<img src = "comultiplication_facts.jpg">
</div>

For more discussion, visit the 
<a href = "http://golem.ph.utexas.edu/category/2008/08/this_weeks_finds_in_mathematic_29.html">\emph{n}-Category 
Caf&eacute;</a>.  In particular, you'll see there's a real
morass of conflicting terminology concerning what I'm calling
"special" Frobenius algebras and "strongly 
separable" algebras.  But if we define them as I do above,
they're very nicely related.

More precisely: an algebra is strongly separable iff it can be given a
comultiplication and counit making it into a special Frobenius
algebra.  If we can do this, we can do it in a unique way.
Conversely, the underlying algebra of a special Frobenius algebra is
strongly separable.

For more details, see:

32) nLab, Frobenius algebra,
<a href = "http://ncatlab.org/nlab/show/Frobenius+algebra">http://ncatlab.org/nlab/show/Frobenius+algebra</a>

and: 

33) nLab, Separable algebra, <a href = "http://ncatlab.org/nlab/show/separable+algebra">http://ncatlab.org/nlab/show/separable+algebra</a>

<HR>
 
<em>'Interesting Truths' referred to a kind of theorem which
captured subtle unifying insights between broad classes of
mathematical structures. In between strict isomorphism - where the
same structure recurred exactly in different guises - and the loosest
of poetic analogies, Interesting Truths gathered together a panoply of
apparently disparate systems by showing them all to be reflections of
each other, albeit in a suitably warped mirror.</em> - Greg Egan,
<i>Incandescence</i>

<HR>

% </A>
% </A>
% </A>


% parser failed at source line 1392
