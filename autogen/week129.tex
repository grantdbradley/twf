
% </A>
% </A>
% </A>
\week{February 15, 1999}

For the last 38 years the Austrians have been having winter workshops on
nuclear and particle physics in a little Alpine ski resort town called
Schladming.  This year it was organized by Helmut Gausterer and Hermann
Grosse, and the theme was "Geometry and Quantum Physics":
 
1) Geometry and Quantum Physics lectures, 38th Internationale 
Universitaetswochen fuer Kern- und Teilchenphysik,
<A HREF = "http://physik.kfunigraz.ac.at/utp/iukt/iukt_99/iukt99-lect.html">http://physik.kfunigraz.ac.at/utp/iukt/iukt_99/iukt99-lect.html</A>

I was invited to give some talks about spin foam models, and the other
talks looked interesting, so I decided to leave my warm and sunny home
for the chilly north.  I flew out to Salzburg in early January and took
a train to Schladming from there.  Jet-lagged and exhausted, I almost
slept through my train stop, but I made it and soon collapsed into my
hotel bed.  

The next day I alternately slept and prepared my talks.  The workshop
began that evening with a speech by Helmut Grosse, a speech by the town
mayor, and a reception featuring music by a brass band.  The last two
struck me as a bit unusual - there's something peculiarly Austrian about
drinking beer and discussing quantum gravity over loud oompah music!
This was also the first conference I've been to that featured skiing and
bowling competitions.

Anyway, there were a number of 4-hour minicourses on different subjects,
which should eventually appear as articles in this book:

2) Geometry and Quantum Physics, proceedings of the 38th Int.
Universitaetswochen fuer Kern- und Teilchenphysik, Schladming, Austria,
Jan. 9-16, 1999, eds. H. Gausterer, H. Grosse and L. Pittner, to appear
in Lecture Notes in Physics, Springer-Verlag, Berlin.

Right now they exist in the form of lecture notes:
<UL>
<LI>
Anton Alekseev: Symplectic and noncommutative geometry of systems with symmetry
<LI>
John Baez: Spin foam models of quantum gravity
<LI>
Cesar Gomez: Duality and D-branes
<LI>
Daniel Kastler: Noncommutative geometry and fundamental physical interactions
<LI>
John Madore: An introduction to noncommutative geometry
<LI>
Rudi Seiler: Geometric properties of transport in quantum Hall systems
<LI>
Julius Wess:   Physics on noncommutative spacetime structures
</UL>
All these talks were about different ways of combining quantum theory
and geometry.  Quantum theory is so strange that ever since its
invention there has been a huge struggle to come to terms with it at all
levels.  It took a while for it to make its full impact in pure
mathematics, but now you can see it happening all over: there are lots
of papers on quantum topology, quantum geometry, quantum cohomology,
quantum groups, quantum logic... even quantum set theory!  There are
even some fascinating attempts to apply quantum mechanics to unsolved
problems in number theory like the Riemann hypothesis... will they bear
fruit?  And if so, what does this mean about the world?  Nobody really
knows yet; we're in a period of experimentation - a bit of a muddle.

I don't have the energy to summarize all these talks so I'll concentrate
on part of Alekseev's - just a tiny smidgen of it, actually!  But first,
let me just quickly say a word about each speaker's topic.

Alekseev talked about some ideas related to the stationary phase
approximation.  This is one of the main tools linking classical
mechanics to quantum mechanics.  It's a trick for approximately
computing the integral of a function of the form exp(iS(x)) knowing only
S(x) and its 2nd derivative at points where its first derivative
vanishes.  In physics, people use it to compute path integrals in the
semiclassical limit where what matters most is paths near the classical
trajectories.   Alekseev discussed problems where the stationary phase
approximation gives the exact answer.  There's a wonderful thing called
the Duistermaat-Heckman formula which says that this happens in certain
situations with circular symmetry.  There are also generalizations to
more complicated symmetry groups.  These are related to `equivariant
cohomology' - more on that later.

I talked about the spin foam approach to quantum gravity.  I've already
discussed this in "<A HREF = "week113.html">week113</A>", "<A HREF = "week114.html">week114</A>", "<A HREF = "week120.html">week120</A>", and "<A HREF = "week128.html">week128</A>", so
there's no need to say more here.

Cesar Gomez gave a wonderful introduction to string theory, starting
from scratch and rapidly working up to T-duality and D-branes.  The idea
behind T-duality is very simple and pretty.  Basically, if you have
closed strings living in a space with one dimension curled up into a
circle of radius R, there is a symmetry that involves replacing R by 1/R
and switching two degrees of freedom of the string, namely the number of
times it winds around the curled-up direction and its momentum in the
curled-up direction.  Both these numbers are integers.   D-branes are
something that shows up when you consider the consequences of this symmetry 
for \emph{open} strings.   

String theory is rather conservative in that, at least until recently, 
it usually treated spacetime as a manifold with a fixed geometry and
only applied quantum mechanics to the description of the strings
wiggling around \emph{in} spacetime.  In spin foam models, by contrast,
spacetime itself is modelled quantum-mechanically as a kind of
higher-dimensional version of a Feynman diagram.   There are also  other
ideas about how to treat spacetime quantum-mechanically.  One of them is
to treat the coordinates on spacetime as noncommuting variables.  In
this approach, called noncommutative geometry, the uncertainty
principle limits our ability to simultaneously know all the coordinates
of a particle's position, giving spacetime a kind of quantum "fuzziness".
Personally I don't find noncommutative geometry convincing as a theory 
of physical spacetime, because there are no clues that spacetime actually
has this sort of fuzziness.  But I find it quite interesting as mathematics.

Daniel Kastler talked about Alain Connes' theories of physics based on
noncommutative geometry.  He discussed both the original Connes-Lott
version of the Standard Model and newer theories that include gravity. 
Kastler is a real character!  As usual, his talks lauded Connes to the
heavens and digressed all over the map in a frustrating but entertaining
manner.  Throughout the conference, he kept us well-fed with anecdotes,
bringing back the aura of heroic bygone days.  A random example: Pauli
liked to work long into the night - so when a student asked "Could I
meet you at your office at 9 a.m.?" he replied "No, I can't possibly
stay that late".   

One nice idea mentioned by Kastler came from this paper:

3) Alain Connes, Noncommutative geometry and reality, J. Math. Phys.
36 (1995), 6194.

The idea is to equip spacetime with extra curled-up dimensions shaped
like the quantum group SU_{q}(2) where q is a 3rd root of unity.  A
quantum group is actually a kind of noncommutative algebra, but using
Connes' ideas you can think of it as a kind of "space".  If you mod out
this particular algebra by its nilradical, you get the algebra M_{1}(C) +
M_{2}(C) + M_{3}(C), where M_{n}(C) is the algebra of n x n complex matrices. 
This has a tantalizing relation to the gauge group of the Standard
Model, namely U(1) x SU(2) x SU(3).  


John Madore also spoke about noncommutative geometry, but more on the
general theory and less on the applications to physics.  He concentrated
on the notion of a "differential calculus" - a structure you
can equip an algebra with in order to do differential geometry thinking
of it as a kind of "space".

Julius Wess also spoke on noncommutative geometry, focussing on a
q-deformed version of quantum mechanics.   The process of
"q-deformation" is something you can do not only to groups like SU(2)
but also other spaces.  You get noncommutative algebras, and these often
have nice differential calculi that let you go ahead and do
noncommutative geometry.  Wess had a nice humorous way of defusing tense
situations.  When one questioner pointedly asked him whether the
material he was presenting was useful in physics or merely a pleasant
game, he replied "That's a very good question.  I will try to answer
that later.  For now you're just like students in calculus: you don't
know why you're learning all this stuff...."  And when Kastler and other
mathematicians kept hassling him over whether an operator was
self-adjoint or merely hermitian, he begged for mercy by saying "I would
like to be a physicist.  That was my dream from the beginning."   

Anyway, I hope that from these vague descriptions you get some
sense of the ferment going on in mathematical physics these days.
Everyone agrees that quantum theory should change our ideas about
geometry.  Nobody agrees on how.  

Now let me turn to Alekseev's talk.   In addition to describing his own
work, he explained many things I'd already heard about.  But he  did it
so well that I finally understood them!  Let me talk about one of these
things: equivariant deRham cohomology.  For this, I'll assume you know
about deRham cohomology, principal bundles, connections and curvature. 
So I assume you know that given a manifold M, we can learn a lot about
its topology by looking at differential forms on M and figuring out the
space of closed p-forms modulo exact ones - the so-called pth deRham
cohomology of M.  But now suppose that some Lie group G acts on M in a
smooth way.  What can differential forms tell us about the topology of
this group action?

All sorts of things!  First suppose that G acts freely on M - meaning
that gx is different from x for any point x of M and any element g of G
other than the identity.  Then the quotient space M/G is a manifold.  
Even better, the map M \to  M/G gives us a principal G-bundle with total
space M and base space M/G.

Can we figure out the deRham cohomology of M/G?  Of course if we were
smart enough we could do it by working out M/G and then computing its
cohomology.  But there's a sneakier way to do it using the differential
forms on M.  The map M \to  M/G lets us pull back any form on M/G to
get a form on M.  This lets us think of forms on M/G as forms on M
satisfying certain equations - people call them "basic"
differential forms because they come from the base space M/G.

What are these equations?  Well, note that each element v of the Lie
algebra of G gives a vector field on M, which I'll also call v.  This
give two operations on the differential forms on M: the Lie derivative 
L_{v}
and the interior product i_{v}.  
It's easy to see that any basic differential
form is annihilated by these operations for all v.  The converse is true
too!  So we have some nice equations describing the basic forms.  

If we now take the space of closed basic p-forms modulo the exact basic
p-forms, we get the deRham cohomology of M/G!  This lets us study the
topology of M/G using differential forms on M.  It's very convenient.

If the action of G on M isn't free, the quotient space M/G might not be
a manifold.  This doesn't stop us from defining "basic" differential
forms on M just as before.  We can also define some cohomology groups by
taking the closed basic p-forms modulo the exact ones.  But topologists
know from long experience that another approach is often more useful. 
Group actions that aren't free are touchy, sensitive creatures - a real
nuisance to work with.  Luckily, when you have an action that's not
free, you can tweak it slightly to make it free.  This involves "puffing
up" the space that the group acts on - replacing it by a bigger space
that the group acts on freely.

For example, suppose you have a group G acting on a one-point space. 
Unless G is trivial, this action isn't free.  In fact, it's about as far
from free as you can get!  But we can "puff it up" and get a space
called EG.  Like the one-point space, EG is contractible, but G acts
freely on it.  Actually there are various spaces with these two
properties, and it doesn't much matter which one we use - people call
them all EG.  People call the quotient space EG/G the "classifying
space" of G, and they denote it by BG.  

More generally, suppose we have \emph{any} action of G on a manifold M.  How
can we puff up M to get a space on which G acts freely?  Simple: just
take its product with EG.  Since G acts on M and EG, it acts on the
product M x EG in an obvious way.  Since G acts freely on EG, its action
on M x EG is free.  And since EG is contractible, the space M x EG is a
lot like M, at least as far as topology goes.  More precisely, it has
the same homotopy type!    

Actually the last 2 paragraphs can be massively generalized at no extra
cost.  There's no need for G to be a Lie group or for M to be a manifold.  
G can be any topological group and M can be any topological space!  But
since I want to talk about \emph{deRham} cohomology, I don't need this extra
generality here.

Anyway, now we know the right substitute for the quotient space M/G when 
the action of G on M isn't free: it's the quotient space (M x EG)/G.


So now let's figure out how to compute the pth deRham cohomology of (M x
EG)/G.  Since G acts freely on M x EG, this should be just the closed
basic p-forms on M x EG modulo the exact ones, where "basic"
is defined as before.  In fact this is true.  We call the resulting
space the pth "equivariant deRham cohomology" of the space M.
It's a kind of well-behaved substitute for the deRham cohomology of M/G in the 
case when M/G isn't a manifold. 

There's only one slight problem: the space EG is very big, so it's not
easy to deal with differential forms on M x EG!  

You'll note that I didn't say much about what EG looks like.  All I said
is that it's some contractible space on which G acts freely.   I didn't
even say it was a manifold, so it's not even obvious that "differential
forms on EG" makes sense!  If you are smart you can choose your space EG
so that it's a manifold.  However, you'll usually need it to be 
infinite-dimensional.

Differential forms make perfect sense on infinite-dimensional
manifolds, but they can be a bit tiresome when we're trying to do
explicit calculations.  Luckily there is a small subalgebra of the
differential forms on EG that's sufficient for the purpose of computing
equivariant cohomology!  This is called the "Weil algebra", WG.  

To guess what this algebra is, let's just list all the obvious
differential forms on EG that we can think of.  Well, I guess none of
them are obvious unless we know a few more facts!  First of all, since
the action of G on EG is free, the quotient map EG \to  BG gives us a
principal G-bundle with total space EG and base space BG.  This bundle
is very interesting.  It's called the "universal" principal 
G-bundle.  The reason is that any other principal G-bundle is a pullback 
of this one.  

(I guess I'm upping the sophistication level again here: I'm assuming
you know how to pull back bundles!)  

Even better, if we choose our space EG so that it's a manifold, then
there is a god-given connection on the bundle EG \to  BG, and any other
principal G-bundle \emph{with connection} is a pullback of this one.  

(And now I'm assuming you know how to pull back connections!  However,
this pullback stuff is not necessary in what follows, so just ignore it
if you like.)

Okay, so how can we get a bunch of differential forms on EG just using
the fact that it's the total space of a G-bundle equipped with a connection? 

Well, whenever we have a G-bundle E \to  B, we can think of a connection on
it as a 1-form on E taking values in the Lie algebra of G.  Let's see what
differential forms on E this gives us!   Let's call the connection A.  If
we pick a basis of the Lie algebra, we can take the components of A in
this basis, and we get a bunch of 1-forms A_{i} on E.   
We also get a bunch of 2-forms dA_{i}.  
We also get a bunch of 2-forms A_{i} ^ A_{j}.  And so on.


In general, we can form all possible linear combinations of wedge products 
of the A_{i}'s and the dA_{i}'s.  
We get a big fat algebra.  In the case when
our bundle is EG \to  BG, equipped with its god-given connection, we define
this algebra to be the Weil algebra, WG!  

Great.  But let's try to define WG in a purely algebraic way, so we can
do computations with it more easily.  We're starting out with the
1-forms A_{i} and taking all linear 
combinations of wedge products of them
and their exterior derivatives.  There are in fact no relations except
the obvious ones, so WG is just "the supercommutative differential
graded algebra freely generated by the variables A_{i}".    
Note: all the
mumbo-jumbo about supercommutative differential graded algebras is a way
of mentioning the \emph{obvious} relations.

Warning: people don't usually describe the Weil algebra quite this way.  
They usually seem describe it in terms of the connection 1-forms and
curvature 2-forms.  However, the curvature is related to the connection by
the formula F = dA + A ^ A, and if you use this you can go from the usual 
description of the Weil algebra to mine - I think.

(Actually, people often describe the Weil algebra as an algebra generated
by a bunch of things of degree 1 and a bunch of things of degree 2, without
telling you that the things of degree 1 are secretly components of a
connection 1-form and the things of degree 2 are secretly components of
a curvature 2-form!  That's why I'm telling you all this stuff - so that
if you ever study this stuff you'll have a better chance of seeing what's
going on behind all the murk.)

Okay, so here is the upshot.  Say we want to compute the equivariant
deRham cohomology of some manifold M on which G acts.  In other words,
we want to compute the deRham cohomology of (M x EG)/G.  On the one
hand, we can start with the differential forms on M x EG, figure out the
"basic" p-forms, and take the space of closed basic p-forms
modulo exact ones.  But remember: up to details of analysis, the algebra
of differential forms on M x EG is just the tensor product of the
algebra of forms on M and the algebra of forms on EG.  And we have this
nice small "substitute" for the algebra of forms on EG, namely
the Weil algebra WG.  So let's take the algebra of differential forms on
M and just tensor it with WG.  We get a differential graded algebra with
Lie derivative operations L_{v} and interior product operations 
i_{v}
defined on it.  We then proceed as before: we take the space of closed
basic elements of degree p modulo exact ones.  Voila!  This is something
one can actually compute, with sufficient persistence.  And it gives the
same answer, at least when G is connected and simply connected.

There are all sorts of other things to say.  For example, if we take the
simplest posssible case, namely when M is a single point, this gives a
nice trick for computing the deRham cohomology of EG/G = BG.  Guys in 
this cohomology ring are called "characteristic classes", and they're
really important in physics.  Since any principal G-bundle is a pullback
of EG \to  BG, and cohomology classes pull back, these characteristic
classes give us cohomology classes in the base space of any principal
G-bundle - thus helping us classify G-bundles.  But if I started explaining
this now, we'd be here all night.

Also sometime I should say more about how to construct EG.  



 \par\noindent\rule{\textwidth}{0.4pt}

% </A>
% </A>
% </A>
