
% </A>
% </A>
% </A>
\week{March 30, 2002 }


I've just been visiting my friend Minhyong Kim at the Korea Institute
for Advanced Studies (KIAS), and before I take off on my next jaunt I'd 
like to mention a couple of cool papers he showed me. 

1) Alain Connes and Dirk Kreimer, Renormalization in quantum field 
theory and the Riemann-Hilbert problem I: the Hopf algebra structure 
of graphs and main theorem, Comm. Math. Phys. 210 (2000), 249-273.
Also available as <A HREF = "http://xxx.lanl.gov/abs/hep-th/9912092">hep-th/9912092</A>.

I've already mentioned Kreimer's work in "<A HREF = "week122.html">week122</A>" 
and "<A HREF = "week123.html">week123</A>", 
and since then I've been to a bunch of talks on it, but I've never 
fully absorbed it.  Minhyong shamed me into trying harder to 
understand what Kreimer is up to.  It's really important, because
he's managed to take the nitty-gritty details of renormalization
and point people to the elegant math lurking inside.  Something like 
this is probably a prerequisite for cracking one of the biggest 
problems in mathematical physics: finding a rigorous approach to 
quantum field theory. 
 
As you may know, renormalization is the process for sweeping
infinities under the rug in quantum field theory.  There are 
lots of approaches, which all give equivalent answers.   My
favorite is the approach pioneered by Epstein and Glaser and 
explained here:

2) G. Scharf, Finite Quantum Electrodynamics, Springer, Berlin, 1995. 

since in this approach, the infinities never show up in the first 
place.  However, the work involved in this approach is comparable 
to that in other approaches, and you wind up getting more or less
the same thing: a multiparameter family of recipes for computing 
complex numbers from Feynman diagrams.  

Hmm... I guess I need to give a quick bare-bones explanation of
that last phrase!  Feynman diagrams are graphs that describe 
processes where particles interact, like this:

\begin{verbatim}
                 \       /               Two particles come in,
                  \     / 
                   -----                 they exchange a virtual particle,
                  /     \
                 /       \               and two particles go out.
\end{verbatim}
    
and the number we compute from a Feynman diagram gives the amplitude 
for the process to occur.  The Feynman diagrams in a given theory 
are built from certain basic building blocks, and we get one parameter 
for each building block.  

For example, in the quantum field theory called "\phi ^3 theory", the 
diagrams are trivalent graphs - graphs with three edges meeting at 
each vertex.  As you can see from the above example, these graphs 
are allowed to have "external edges" - that is, loose ends representing 
particles that come in or go out.   Each external edge is labelled by a 
vector in R^4 describing the energy-momentum of the corresponding particle.  

The basic building blocks of Feynman diagrams in this theory are the
edge:


\begin{verbatim}
                 ---------

\end{verbatim}
    
and the vertex: 
                       
\begin{verbatim}
                   \   / 
                    \ /
                     |
                     |
\end{verbatim}
    
We can draw these in any rotated way that we like.  The parameter 
corresponding to the edge is called the "mass" of the particle in 
this theory, because in quantum theory, a particle's mass affects
what it does when it's just zipping along minding its own business.  
The parameter corresponding to the vertex is called the "coupling 
constant", because it affects how likely two particles are to couple 
and give birth to a third.  

Fancier theories will have more basic building blocks for their Feynman
diagrams: various kinds of edges corresponding to different kinds
of particles, and also various kinds of vertices, corresponding 
to different kinds of interactions.  This means these theories 
have more parameters (masses and coupling constants).  In every 
case, the basic building blocks can be thought of as Feynman 
diagrams in their own right... that'll be important in a minute.

Okay. Here's what Connes and Kreimer do in the above paper.  To
say this in a finite amount of time I'm afraid I'm gonna need to
assume you know some stuff about Hopf algebras....

First, they fix a renormalizable quantum field theory.  They use the
\phi ^3 theory in 6d spacetime, but it doesn't matter too much which
one; quantum electrodynamics or the Standard Model should work as well. 

They show that there's a Hopf algebra having "one-particle irreducible"
Feynman diagrams as a basis - these are the Feynman diagrams that don't
fall apart into more connected components when you remove one edge.
In this Hopf algebra, the product of two Feynman diagrams is just their 
disjoint union, but their coproduct is a sneakier thing which encodes a 
lot of the crucial aspects of renormalization.  Oversimplifying a bit, 
the coproduct of a diagram x is

x tensor 1 + 1 tensor x + sum_{i} x_{i} tensor y_{i}

where x_{i} ranges over all subdiagrams of x whose external edges
match those of one of the elementary building blocks, and y_{i} is
obtained from x by collapsing the subdiagram x_{i} to the corresponding
elementary building block.  Look at their paper for some pictures of
how this works, and also a more precise statement. 

Next, by a general theorem on commutative Hopf algebras, we can think
of H as consisting of functions on some group G, with pointwise 
multiplication as the product in H.  Since elements of H are linear 
combinations of Feynman diagrams, this means that any \emph{point} of G 
gives a way to evaluate Feynman diagrams and get numbers.   The group
G is an interesting sort of infinite-dimensional Lie group which they
study further in another paper:

3) Alain Connes and Dirk Kreimer, Renormalization in quantum field 
theory and the Riemann-Hilbert problem I: the beta-function, 
diffeomorphisms and the renormalization group, Comm. Math. Phys. 
216 (2001), 215-241.  Also available as <A HREF = "http://xxx.lanl.gov/abs/hep-th/0003188">hep-th/0003188</A>.

It may even deserve to be called the "renormalization group", which
is a piece of physics jargon that's been waiting for an interesting group to
come along... but let's not worry about that now!  All that matters now is that
each point in G gives a way to evaluate Feynman diagrams.  

Now, for any choice of values for all the parameters in our theory, 
there's a simple recipe for evaluating Feynman diagrams.  I won't
explain this recipe; it's one of those things you learn in any intro 
course on quantum field theory.  You could hope this recipe defines a 
point of G, but there's a catch: this recipe typically gives infinite
answers!  

Luckily, using a trick called "dimensional regularization", one can get 
finite answers if one analytically continues the dimension of spacetime to 
any complex number z \emph{near} the actual dimension d.  The infinities show up 
as a pole at z = d.  Connes and Kreimer use this trick to get a map from a 
little circle around the point z = d to the group G.  Let's call this map

g: S^{1} \to  G

where S^{1} is the circle.  Using some old ideas from complex analysis 
(buzzword: the "Riemann-Hilbert problem") they write g as the product
of two maps 

g+, g-: S^{1} \to  G

where g+ is well-defined and analytic \emph{inside} the circle, and g- is 
well-defined and analytic \emph{outside}.  The punchline is that evaluating 
g+ at the point z = d we get a point in G which gives the actual 
renormalized value of any Feynman diagram in our theory!  

For a bigger tour of Kreimer's ideas, try his book:

4) Dirk Kreimer, Knots and Feynman Diagrams, Cambridge University Press,
Cambridge, 2000.

Part of why Minhyong wanted to understand this stuff is that he
also invited Graeme Segal to the KIAS.  Segal is one of the
mathematical gurus behind string theory, and he did some very
important work on "loop groups" - maps from a circle into a group, 
made into a group by pointwise multiplication:

5) Andrew Pressley and Graeme Segal, Loop Groups, Oxford University
Press, Oxford, 1986. 
 
The factorization of a map g: S^{1} \to  G into parts that are analytic
inside and outside the unit disk plays a big role in string theory: 
it corresponds to taking certain 2d field theories called Wess-Zumino-Witten 
models and splitting the solutions into left-moving and right-moving modes.  
So, it's intriguing to find it also showing up in renormalization theory.

Segal gave some talks on D-branes which I wish I had time to summarize.
One main point was that just as topological quantum field theories 
are certain nice functors taking 2d cobordisms to linear operators, 
topological quantum field theories "with D-branes" are certain nice 
2-functors that know how to handle 2d cobordisms with corners.  I can 
only assume something similar is true of D-branes in conformal field 
theory, where the cobordisms are equipped with a complex structure.
He's apparently writing a paper on this sort of thing with Gregory Moore, 
which won't mention 2-functors... but us n-category theorists know a
2-functor when we see one!

Speaking of strings, my spies say everyone is raving about this
new paper:

6) David Berenstein, Juan Maldacena and Horatiu Nastase, Strings
in flat space and pp waves from N = 4 Super Yang Mills, available as
<A HREF = "http://xxx.lanl.gov/abs/hep-th/0202021">hep-th/0202021</A>.

However, apart from this piece of gossip, I have very little to
report!  Ask your local string theorist what it's all about.

Here's another cool paper Minhyong mentioned:

7) Yuri Manin and Matilde Marcolli, Holography principle and arithmetic 
of algebraic curves, available as <A HREF = "http://xxx.lanl.gov/abs/hep-th/0201036">hep-th/0201036</A>.

It talks about Kirill Krasnov's extensive dictionary relating 
everything about Riemann surfaces and 3d hyperbolic geometry
to stuff about black holes in 3d quantum gravity - this is worth 
a Week in itself - but what really got my attention is that it
develops a far-out analogy between "spacelike infinity" in 3d 
quantum gravity and "the prime at infinity" in algebra.  Zounds!

Alas, I have to hit the sack now and catch some sleep before my
morning flight, or I would tell you more about this....


<p> <hr>
<em>Not till we are lost ... do we begin to find ourselves
and realize where we are and the infinite extent of our
relations.</em> - Henry David Thoreau
<HR>

% </A>
% </A>
% </A>


% parser failed at source line 309
