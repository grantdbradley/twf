
% </A>
% </A>
% </A>
\week{June 26, 2000}

Recently I've been talking a bit about elliptic cohomology, but
I've really just been nibbling around the edges so far.  Sometime
I want to dig deeper, but not just now.  Right now, I instead want
to say a bit more about the physics lurking in the space K(Z,2).

But first, here's a cool article on violins:

1) Colin Gough, Science and the Stradivarius, Physics World, vol. 13
no. 4, April 2000, 27-33.

Before reading this, I never knew how a string on a violin vibrates!
Lots of well-known European physicists have studied the violin, and in
the 19th century, Helmholtz showed that the bow excites a mode of the
violin string that is quite unlike the sine waves we all know and love.
In this "Helmholtz waveform", the string consists of two straight-line
segments separated by a kink:



\begin{verbatim}


                         .   
                    .       .
               .               .
          .                       .

\end{verbatim}
    
As time passes, the kink travels back and forth along the string, being
reflected at the ends.  The beauty of this becomes apparent as we watch
the string right at the point where the bow is rubbing over it, near the
bottom end of the string.  When the kink is between the bow and the top
end of the string:



\begin{verbatim}


                            bow
                             |
                             ^
                         .   |
   TOP              .       .|          BOTTOM
               .             | .
          .                  |    .
                             |
                             |


\end{verbatim}
    
this point in the string moves at the same speed and in the same direction 
as the bow.  This is called the "sticking regime", because the static 
friction of the rosin-coated bow is enough to pull the string along with it.  
But when the kink moves past the bow:


\begin{verbatim}

                            |
                            ^
                            |  .
                         .  | 
   TOP              .       |   .     BOTTOM
               .            | 
          .                 |    .
                            |
\end{verbatim}
    
the string slips off the bow and starts moving in the opposite direction
to it.   This is called the "sliding regime".  Since the coefficient of
sliding friction is less than the coefficient of static friction, the
string can slide against the motion of the bow in this regime.

The really nice thing is that the string is vibrating almost freely: the
violinist just needs to apply the right amount of pressure to keep this
vibrational mode excited - too much pressure will ruin it!  Being able
to delicately control the Helmholtz waveform is part of what distinguishes 
the virtuoso from the blood-curdling amateur.

The full physics of the violin is infinitely more complicated than this,
of course.  The vibrating string excites the bridge which excites the
sound box, and \emph{that} produces most of the sound we hear.  For more
information try these:

2) A. H. Benade, Fundamentals of Musical Acoustics, Oxford University
Press, Oxford, 1976.

L. Cremer, The Physics of the Violin, MIT Press, Cambridge, Massachusetts,
1984.

N. H. Fletcher and T. D. Rossing, The Physics of Musical Instruments, 
2nd edition, Springer, New York, 1998.

C. Hutchins and V. Benade, editors, Research Papers on Violin Acoustics
1975-1993, 2 volumes, Acoustical Society of America, New York, 1997.

Okay, now on to K(Z,2)!  I explained a bit about this space in "<A HREF = "week149.html">week149</A>",
but I've been pondering it a lot lately, so I'd like to say a bit more.  

First let me review and elaborate on some basic stuff I said already.   
If G is any topological group, there is a topological 
space BG with a basepoint such that the space of loops in BG starting 
and ending at this point is homotopy equivalent to G.  This space BG 
is unique up to homotopy equivalence.  [1]

BG is important because it's the "classifying space for G-bundles".  
What this means is that there's a principal G-bundle over BG called
the "universal G-bundle", with the marvelous property that \emph{any}
principal G-bundle over \emph{any} space X is a pullback of this one by 
some map

f: X \to  BG.

(I explained in "<A HREF = "week149.html">week149</A>" how to pull back complex line bundles, and
pulling back principal G-bundles works the same way.)   Even better, 
two G-bundles that we get this way are isomorphic if and only if the maps
they come from are homotopic!  So there is a one-to-one correspondence
between:

A) isomorphism classes of principal G-bundles over X 

and

B) homotopy classes of maps from X to BG.

Now, suppose G is an \emph{abelian} topological group.  Then BG is better
than a topological space with basepoint.  It's an abelian topological
group!  

This means that we can \emph{iterate} this trick.   Starting with an abelian
topological group G we can form BG, and BBG, and BBBG, and so on.   This 
is called "delooping", because the loop space of each of these spaces is 
the previous one.   

It's always fun to iterate any process whenever you can - Freud called
this "repetition compulsion" - but there's more going on here than just
that.  In "<A HREF = "week149.html">week149</A>" I said that when we have a list of spaces, each being
the loop space of the previous one, it's called a "spectrum".  And I
said that we can use a spectrum to get a generalized cohomology theory. 
So we now have a trick for getting a generalized cohomology theory from
a topological abelian group!

In particular, suppose we start with a plain old abelian group A.  
We can think of it as a topological group with the discrete topology - 
let's call this K(A,0).  Then we can define

K(A,1) = B(K(A,0))
K(A,2) = B(K(A,1))
K(A,3) = B(K(A,2)) 
    
... and so on.   We get a spectrum K(A,n) called an "Eilenberg-MacLane
spectrum".  The corresponding generalized cohomology theory is just
ordinary cohomology with coeffients in the abelian group A!  This means
that

H^{n}(X,A) = [X, K(A,n)]

where the right-hand side is the set of homotopy classes of maps from X
to K(A,n).  In short, K(A,n) knows everything there is to know about the
nth cohomology with coefficients in A.  

We've seen this trick a couple of times lately, and it's actually a big
theme in homotopy theory: whenever we have some interesting invariant of
spaces, we try to cook up a space that "represents" this invariant.  I
could say a LOT more about THIS idea, but that would propel us into
further heights of abstraction, when what I really want is to come down 
to earth a bit.  Just a little bit.... 

So: let's take A to be the integers, Z.  As I said in "<A HREF = "week149.html">week149</A>",
we then get

K(Z,0) = Z,

K(Z,1) = U(1),

where U(1) is the group of "phases" or unit complex numbers, and

K(Z,2) = CP^{\infty }

where CP^{\infty } is infinite-dimensional complex projective space.
There are a couple of slightly different versions of this.  Topologists
like to start with the direct limit of the spaces C^{n}, which they call
C^{\infty }.  Then they take the space of all 1-dimensional 
subspaces and
call that CP^{\infty }.  Mathematical physicists prefer to start with a
Hilbert space of countable dimension.  Then they take the space of unit
vectors modulo phase.  Both these versions are equally good models of
K(Z,2).  The first one is a lean, stripped-down version of the second.

Now U(1) is very important in quantum theory, and so are unit vectors
modulo phase in a Hilbert space - physicists call these "pure states".
So something cool is going on here.  For some mysterious reason, it
looks like K(Z,n)'s are important quantum physics!  This is especially
interesting because the abstract definition of the K(Z,n)'s has nothing
to do with the complex numbers - just the integers.  The complex numbers
show up on their own accord.  So maybe this hints at some explanation of
why the complex numbers are important in quantum mechanics.

Why are K(Z,n)'s connected to quantum theory?  I don't really know. 
But we can get some clues by asking some more specific questions.  

First of all, why is K(Z,2) the same as CP^{\infty }?  In "<A HREF = "week149.html">week149</A>" I
just asserted this without proof.  That's one of the fun things I'm
allowed to do in this column.  But let me sketch why it's true.  

First I need to remind you of some more basic facts about topology.  
Suppose G is any topological group, and let P \to  X be any principal 
G-bundle.   This gives us a long exact sequence of homotopy groups:

 ... \to  \pi _{n+1}(X) \to  \pi _{n}(G) \to  \pi _{n}(P) \to  \pi _{n}(X) \to  \pi _{n-1}(G) \to  ...
Two-thirds of the arrows in this sequence come from the maps

                       G \to  P \to  X 
while the less obvious remaining one-third come from the map

           LX \to  G
sending each loop in the base space to the holonomy of some connection
on our bundle.   Here LX means the space of based loops in X, and we're 
using the fact that

                   \pi _{n}(LX) = \pi _{n+1}(X)
which is obvious from the definition of the homotopy groups.

But now suppose P is contractible!  Then all its homotopy groups vanish,
so the above long exact sequence breaks up into lots of puny exact
sequences like this:

  0 \to  \pi _{n+1}(X) \to  \pi _{n}(G) \to  0
or in other words:

  0 \to  \pi _{n}(LX) \to  \pi _{n}(G) \to  0
This says that the map from LX to G induces isomorphisms on all homotopy
groups.   By the Whitehead theorem, this implies that this map is a
homotopy equivalence!  So LX is really just G!!  So X is just BG!!!

In short: if we have a space X with a principal G-bundle P over it, and
P is contractible, X must be BG.        [2]

Now let's use this fact to show that CP^{\infty } is K(Z,2).  Remember
that by our recursive definition, 

K(Z,2) = B(K(Z,1)) = B(U(1))

so to show that CP^{\infty } is K(Z,2), we just need to find a principal
U(1)-bundle over it with a contractible total space.  

In "<A HREF = "week149.html">week149</A>" we discussed a complex line bundle over CP^{\infty } called
the "universal complex line bundle".  If you take the space of unit
vectors in a complex line bundle you get a principal U(1)-bundle.  So
let's do this to the universal complex line bundle.  What do we get?
We get a principal U(1)-bundle like this:

S^{\infty } \to  CP^{\infty }

Being a mathematical physicist, I'm using S^{\infty } here to stand for
the unit sphere in some countable-dimensional Hilbert space, and the map
sends each unit vector to the corresponding pure state, or unit vector
mod phase.  Since there's a circle of unit vectors for each pure state,
this is indeed a principal U(1)-bundle.  But now for the cool part:  the
unit sphere in an infinite-dimensional Hilbert space is contractible! 
So we've got a principal U(1)-bundle with a contractible total space 
sitting over CP^{\infty }, proving that CP^{\infty } is K(Z,2).  Even
better, the bundle 

S^{\infty } \to  CP^{\infty }

is the universal principal U(1)-bundle.

I can't resist explaining why the unit sphere in an infinite-dimensional
Hilbert space is contractible.  It seems very odd that a sphere could be
contractible, but this is one of those funny things about infinite
dimensions.  Take our Hilbert space to be L^{2}[0,1] and consider any 
function f in the unit sphere of this Hilbert space:

&int; |f(x)|^{2} dx = 1

For t between 0 and 1, let f_{t}(x) 
be a function that equals 1 for x < t,
and a sped-up version of f for x greater than or equal to t.  If you do
this right f_{t} 
will still lie in the unit sphere, and you'll have a way
of contracting the whole unit sphere down to a single point, namely the
constant function 1.  

Cute, huh?

Next question: how does CP^{\infty } become an abelian topological group?
There's a very pretty answer.  Consider the space of rational functions
of a single complex variable.  This is a infinite-dimensional complex
vector space, and there's a natural way to give it the topology of
C^{\infty }.   This gives us a nice way to think of CP^{\infty }: it's
just the \emph{nonzero} rational functions modulo multiplication by constants.  

But nonzero rational functions form an abelian group under multiplication!
And this is still true when we mod out by constant factors!  So CP^{\infty } 
becomes an abelian group - and in fact an abelian topological group.

We can visualize CP^{\infty } quite easily this way.  A
rational function of a single complex variable has a bunch of zeros and
poles - think of them as points on the Riemann sphere.  We should really 
stick an integer at each of these points: a positive integer at each zero,
and a negative integer at each pole, to tell us the order of that zero
or pole.  This gives enough information to completely specify the
rational function up to a constant factor.  So a point in
CP^{\infty } is the same as a finite set of points on the
sphere labelled by integers - which must add up to zero.

Of course, we have to get the right topology on CP^{\infty }.  
As we move our point in CP^{\infty } around in a continuous way, 
the corresponding points on the sphere all move around continuously, 
like a swarm of flies... but when points collide, their numbers add!
For example, when a point labelled by the number 7 collides with a 
point labelled by the number -3, it turns into a point labelled by 
the number 7 - 3 = 4.   

In the lingo of physics, we've got a picture of points in
CP^{\infty } as "collections of particles and
antiparticles on the sphere".  The integer at any point on the
sphere tells us the number of particles sitting there - but if it's
negative, it means we've got \emph{antiparticles} there.
Particle-antiparticle pairs can be created out of nothing, and they 
annihilate when they collide... it's very nice!  

By the way, there's something called the Thom-Dold theorem that lets 
us generalize the heck out of this.  We just showed that if you take
the 2-sphere and consider the space of particle-antiparticle swarms in 
it, you get K(Z,2).  But suppose instead we started with the n-sphere
and considered the space of particle-antiparticle swarms in \emph{that}.  
Then we'd get K(Z,n)!  

More generally, suppose we didn't use integers to say how many particles
were at each point in the n-sphere - suppose we used elements of some
abelian group A.  Then we'd get K(A,n)!

For more tricks like this, try this paper:

3) Dusa McDuff, Configuration spaces of positive and negative particles, 
Topology 14 (1975), 91-107. 

Now let me mention a different picture of K(Z,2), that's also nice, 
and also related to quantum theory.  Take any countable-dimensional 
Hilbert space H and let U(H) be the group of unitary operators on H.  
Just like the unit sphere in this Hilbert space is contractible, it 
turns out that U(H) is contractible if we give it the norm topology 
or the strong topology.   

Anyway, now let PU(H) be the "projective unitary group" of H, meaning 
the group of unitary operators modulo phase.  There's an obvious map

U(H) \to  PU(H)

sending a circle's worth of points to each point in PU(H).  It's
easy to check that this is a principal U(1)-bundle.  Since the total
space U(H) is contractible, it follows that PU(H) is K(Z,2)!  

This give a \emph{nonabelian} group structure on K(Z,2), which may seem
kind of weird, given that we just made it into an \emph{abelian} group a
minute ago.   But I guess this other product is "abelian up to homotopy" 
in a very strong sense, so it's just as good as abelian for the purposes
of homotopy theory.

Anyway, some people in Australia have figured out an extra trick you
can do with this PU(H) group:

4) Alan L. Carey, Diarmuid Crowley and Michael K. Murray, Principal 
bundles and the Dixmier-Douady class, Comm. Math. Physics 193 (1998)
171-196, preprint available as <A HREF = "http://xxx.lanl.gov/abs/hep-th/9702147">hep-th/9702147</A>. 

Here's how it goes, at least in part.  We say a linear operator 

A: H \to  H

is "Hilbert-Schmidt" if the trace of AA* is finite.  The space 
of Hilbert-Schmidt operators is a Hilbert space in its own right, 
with this inner product:

<A,B> = tr(AB*)

Let's call this Hilbert space X.  U(H) acts on X by conjugation, and 
this gives an action of PU(H) on X, because phases commute with everything.  
This in turn gives an action of PU(H) on U(X)!  Is your brain melting yet?  
Anyway, it turns out that this makes U(X) into the total space of a 
principal PU(H)-bundle:

PU(H) \to  U(X) \to  U(X)/PU(H)

But X is a countable-dimensional Hilbert space, so U(X) is contractible, 
so this is the \emph{universal} principal PU(H)-bundle.  And as we've seen, 
this means that 

U(X)/PU(H) = B(PU(H)) 

but we just saw that

PU(H) = K(Z,2)

so 

U(X)/PU(H) = B(PU(H)) = B(K(Z,2)) = K(Z,3) !

In "<A HREF = "week149.html">week149</A>", I said I'd like K(Z,3) to be some sort of infinite-
dimensional manifold closely related to quantum physics.   I'm 
happier now, because here we are getting just that - technically,
we're getting it to be a "Banach manifold".   Of course, I could 
still complain that this description doesn't make the <em>abelian 
group structure</em> on K(Z,3) obvious.  But it's definitely a big 
step towards understanding what K(Z,n)'s have to do with quantum 
theory.  

While I'm at it, I should report some other things people have told me
via email.  If you ponder what I've said, you can see that
CP^{\infty } has 2nd homology equal to Z, and that the
generator of this homology group - the "universal cycle" -
is given geometrically by the obvious way of sticking the sphere
CP^{1} inside CP^{\infty }.  This is nice because
CP^{1} is actually a submanifold of the manifold
CP^{\infty }.  But according to email from Mark Goresky, Rene
Thom has shown that for n > 6, we cannot make K(Z,n) into a
manifold in such a way that the universal cycle is represented by a
submanifold!

On the other hand, Michael Murray reports that Pawel Gajer has managed
to make K(Z,n) into something called a "differential space",
which is not quite a manifold, but good enough to do geometry on.  I'm
not sure how this relates to Thom's work... but anyway, I should read
this stuff:

5) Pawel Gajer, Geometry of Deligne cohomology, Invent. Math. 127
(1997), 155-207, also available as 
<A HREF = "http://xxx.lanl.gov/abs/alg-geom/9601025">alg-geom/9601025</A>.

Pawel Gajer, Higher holonomies, geometric loop groups and smooth Deligne 
cohomology, Advances in Geometry, Birkhauser, Boston, 1999, pp. 195-235.

Now, so far I've been restraining myself from talking about
"gerbes", but if you've gotten this far you must be pretty
comfortable with abstract nonsense, so you'll probably like gerbes.
Very roughly speaking, a gerbe is a categorified version of a
principal bundle!  Actually it's a categorified version of a sheaf,
but sometimes we can think of it as analogous to the sheaf of sections
of a bundle.  And just as K(Z,2) is the classifying space for U(1)
bundles, K(Z,3) is the classifying space for a certain sort of gerbe!

I sort of explained how this works in "<A HREF =
"week25.html">week25</A>", but you can read the details here:

6) Jean-Luc Brylinski, Loop Spaces, Characteristic Classes and Geometric 
Quantization, Birkhauser, Boston, 1993.

What this means is that as we explore the meaning of these K(Z,n)'s
for quantum theory, we are really \emph{categorifying} familiar
ideas from quantum theory.  In particular, this story should keep
going on forever: K(Z,4) should be the classifying space for a certain
sort of 2-gerbe, and so on.  But I don't think people have worked out
the details beyond the case of 2-gerbes.  If you want to learn about
2-gerbes, you have to read this:

7) Lawrence Breen, On the Classification of 2-Gerbes and 2-Stacks, 
Asterisque 225, 1994. 

Finally, for more applications to physics, try these papers:

8) Alan L. Carey and Michael K. Murray, Faddeev's anomaly and bundle 
gerbes, Lett. Math. Phys. 37 (1996), 29-36.

Jouko Mickelsson, Gerbes and Hamiltonian quantization of chiral fermions,
Lie Theory and Its Applications in Physics, World Scientific, Singapore,
1996, pp. 216-225.

Michael K. Murray, Bundle gerbes, J. London Math. Soc. 54 (1996), 403-416.

Alan L. Carey, Jouko Mickelsson and Michael K. Murray, Index theory,
gerbes, and Hamiltonian quantization, Comm. Math. Phys. 183 (1997), 707-722,
preprint available as <A HREF = "http://xxx.lanl.gov/abs/hep-th/9511151">
hep-th/9511151</A>.

Alan L. Carey, Michael K. Murray and B. L. Wang, Higher bundle gerbes and 
cohomology classes in gauge theories, J. Geom. Phys. 21 (1997) 183-197,
preprint available as <A HREF = "http://xxx.lanl.gov/abs/hep-th/9511169">
hep-th/9511169</A>.

Alan L. Carey, Jouko Mickelsson and Michael K. Murray, Bundle gerbes
applied to quantum field theory, Rev. Math. Phys. 12 (2000), 65-90,
preprint available as <A HREF = "http://xxx.lanl.gov/abs/hep-th/9711133">
hep-th/9711133</A>.

I thank N. Christopher Phillips of the University of Oregon, 
Michael K. Murray and Diarmuid Crowley of the University of 
Adelaide, and Mark Goresky of IHES for educating me about these
matters... all remaining errors are mine!

Footnotes: 

[1] 
I'm being sloppy here.  Throughout this discussion, when
I say "homotopy equivalent",
I really mean "weakly homotopy equivalent" - a technical
nuance that you can read about in any good book on homotopy theory.

[2] Moreover, P must be the universal principal G-bundle. Conversely,
for any topological group G the total space of the universal principal
G-bundle is contractible.    Everything fits together very neatly!  But
I don't need all this stuff now.




<br>
\par\noindent\rule{\textwidth}{0.4pt}

% </A>
% </A>
% </A>
