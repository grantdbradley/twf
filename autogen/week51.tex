
% </A>
% </A>
% </A>
\week{April 23, 1995}

For people in theoretical physics, Trieste is a kind of mecca.  
It's an Italian town on the Adriatic quite near the border with 
Slovenia, and it's quite charming, especially the castle of 
Maximilian near the coast, built when parts of northern Italy were 
under Hapsburg rule.  Maximilian later took his architect with him 
to Mexico when he became Emperor there, who built another castle for 
him in Mexico City.  (The Mexicans, apparently unimpressed, overthrew
and killed Maximilian.)   These days, physicists visit Trieste
partially for the charm of the area, but mainly to go to 
the ICTP and SISSA, two physics institutes, the latter of which 
has grad students, the former of which is purely for research.
There are lots of conferences and workshops at Trieste, and
I was lucky enough to be invited to Trieste while one I found 
interesting was going on.

As I described to some extent in "<A HREF = "week44.html">week44</A>" and "<A HREF = "week45.html">week45</A>", Seiberg
and Witten have recently shaken up the subject of Donaldson theory
by using some physical reasoning to radically simplify the
computations involved.  Donaldson theory has always had a lot
to do with physics, since it uses the special features of 
of gauge theory in 4 dimensions to obtain invariants of 4-dimensional
manifolds.  So perhaps it is not surprising that physicists
have had a lot to say about Donaldson theory all along, even 
before the recent Seiberg-Witten revolution.  And indeed, at
Trieste lots of mathematicians and physicists were busy talking
to each other about Donaldson theory, trying to catch up with 
the latest stuff and trying to see what to do next.

Now I don't know much about Donaldson theory, but I have
a vague interest in it for various reasons.  First, it's
\emph{supposed} to be a 4-dimensional topological quantum field
theory, or TQFT.  Indeed, the very first paper on TQFTs was
about Donaldson theory in 4 dimensions:

1) Topological quantum field theory, by Edward Witten, Comm.
Math. Phys. 117 (1988) 353.

Only later did Witten turn to the comparatively easier
case of Chern-Simons theory, which is a 3-dimensional TQFT:

2) Quantum field theory and the Jones polynomial, by Edward
Witten, Comm. Math. Phys. 121 (1989) 351.

However, when \emph{mathematicians} talk about TQFTs they usually
mean something satisfying Atiyah's axioms for a TQFT (which
are nicely presented in his book - see "<A HREF = "week39.html">week39</A>").  Now it
turns out that Chern-Simons theory can be rigorously constructed
as a TQFT satisfying these axioms most efficiently using
braided monoidal categories, which play a big role in 3d topology.
So it makes quite a bit of sense in a \emph{general} sort of way that
Crane and Frenkel are trying to construct Donaldson theory using 
braided monoidal 2-categories, which seem to play a comparable
role in 4d topology.  In the paper which I cite in "<A HREF = "week50.html">week50</A>", they begin 
to construct a braided monoidal 2-category related to the group SU(2), 
which they conjecture gives a TQFT related to Donaldson theory.  That
also makes some \emph{general} sense, because Donaldson theory, at
least "old" Donaldson theory, is closely related to gauge theory
with gauge group SU(2).  Still, I've always wanted to see a
more \emph{specific} reason why Donaldson theory should be related to 
the Crane-Frenkel ideas, not necessarily a proof, but at least
a good heuristic argument.  

Luckily George Thompson, who invited me to Trieste, knows a bunch
about TQFTs.  Unluckily he was sick and I never really got to talk
to him very much!  But luckily his collaborator Matthias Blau was
also there, so I took the opportunity to pester him with questions.
I learned a bit, most of which is in their paper:

3) N = 2 topological gauge theory, the Euler characteristic
of moduli spaces, and the Casson invariant, by Matthias Blau
and George Thompson, Comm. Math. Phys. 152 (1993), 41-71.

This paper helped me a lot in understanding Crane and Frenkel's 
ideas.  But so that this "week" doesn't get too long, I'll just 
focus on one basic aspect of the paper, which is the importance of 
supersymmetric quantum theory for TQFTs.  Then next week I'll say a
bit more about the Donaldson theory business.

If you look at Witten's paper on Donaldson theory above, you'll
see he writes down the Lagrangian for a "supersymmetric" field
theory, which is supposed to be a TQFT, namely, Donaldson theory.
Supersymmetric field theories treat bosons and fermions in an
even-handed way.  But why does supersymmetry show up here?
The connection with TQFTs is actually pretty simple
and beautiful, at least in essence.  

Suppose we are doing quantum field theory, and "space" (as opposed
to "spacetime") is some manifold M.  Then we have some Hilbert space
of states Z(M) and some Hamiltonian H, which is a self-adjoint operator 
on Z(M).  To evolve a state (a vector in Z(M)) in time, we hit it with
the unitary operator exp(-itH), where t is the amount of time we want 
to evolve by, and the minus sign is just a convention designed to confuse 
you.
  
We can think of this geometrically as follows.  We are taking spacetime to
be [0,t] x M.  You can visualize spacetime as a kind of pipe, if you 
want, and then imagine sticking in the state \psi  at one end and 
having exp(-itH)\psi  pop out at the other end.

Now say we bend the pipe around and connect the input end to 
the output end!  Then we get the spacetime S^{1} x M, where 
S^{1}
is the circle of circumference t, formed by gluing the two ends of the
interval [0,t] together.  For this kind of "closed" spacetime,
or compact manifold, a quantum field theory should give us not
an operator like exp(-itH), but a number, the "partition function",
which in this case is just the \emph{trace} tr(exp(-itH)).  

The deep reason for this is that taking the trace of an operator - 
remember, that means taking the sum of the diagonal entries, when 
you think of it as a matrix - is really very much like as taking 
a pipe and bending it around, connecting the input end to the output 
end, forming a closed loop.  This may seem bizarre, but observe 
that taking the sum of the diagonal entries really is just a 
quantitative measure of how much the "output constructively interferes
with the input".  (And a very nice one, since it winds up not
depending on the basis in which we write the matrix!)  This sort of
idea is basic in the Bohm-Aharonov effect, where we take a particle
in an electomagnetic field around a loop and see how much it interferes 
with itself, and it is also the basic idea of a "Wilson loop", where
we do the same thing for a particle in a gauge field.  In other words,
the trace measures the amount of "positive feedback".  If this still
seems bizarre, or just vague, take a look at:

4) Knots and Physics, by Louis Kauffman, World Scientific
Press, Singapore, 1991.

You will see that the same idea shows up in knot theory, where 
taking a trace corresponds to taking something (like a braid or tangle) 
and folding it over to connect the input and output.  In a later "week" I'll 
talk a bit about a new paper by Joyal, Street and Verity that
studies the notion of "trace", "feedback" and "folding over" in a 
really general context, the context of category theory.

Anyway, the partition function tr(exp(-itH)) typically depends
on t, or in other words, it depends on the circumference of
our circle S^{1}, not just on the topology of the manifold
S^{1} x M.  In a TQFT, the partition function is only supposed to 
depend on the topology of spacetime!  So, how can we get tr(exp(-itH))
to be independent of t?

There is a banal answer and a clever answer.  The banal
answer is to take H = 0!  Then tr(exp(-itH)) = tr(1) is
just the \emph{dimension} of the Hilbert space:

                     tr(exp(-itH)) = dim(Z(M)).
Actually this isn't quite as banal as it may sound; indeed, the basic
equation of quantum gravity is the Wheeler-DeWitt equation,

                      H \psi  = 0,
which must hold for all physical states.  In other words,
in quantum gravity there is a big space of "kinematical states"
on which H is an operator, but the really meaningful "physical
states" are just those in the subspace 
Z(M) = {\psi : H \psi  = 0}.  
Read "<A HREF = "week11.html">week11</A>" for more on this.

But there is a clever answer involving supersymmetry!
You might hope that there were some more interesting 
self-adjoint operators H such that tr(exp(-itH)) is 
time-independent, but there aren't.  So we seem stuck.
This reminds me of a course I took from Raoul Bott.  He
said "so, we think about the problem... and still we are 
stuck, so what should we do?  SUPERTHINK!"  

Recall that the Hamiltonian of a free particle in quantum mechanics 
is - up to boring constants - just minus the Laplacian on 
configuration space which is some Riemannian manifold that the
particle roams around on.  For this Hamiltonian, tr(exp(-itH)) doesn't
quite make sense, since the Hilbert space is infinite-dimensional
and the sum of the diagonal matrix entries diverges.  But 
tr(exp(-tH)) often \emph{does} converge.  This is why folks
often replace t by -it in formulas, which is called "going to 
imaginary time" or a "Wick transform"; it really amounts to
replacing Schrodinger's equation by the heat equation: i.e.,
instead of a quantum particle, we have a particle undergoing
Brownian motion!  In any event, tr(exp(-tH)) certainly depends 
on t in these situations, but there is something very similar that 
does NOT.

Namely, let's replace the Laplacian on \emph{functions} by the
Laplacian on \emph{differential forms}.  I won't try to remind
you what these are; I'll simply note that functions are 0-forms,
but there are also 1-forms, 2-forms, and so on - tensor fields
of various sorts - and the Laplacian of a j-form is another j-form.   
So for each j we get a kind of Hamiltonian H_{j}, 
which is just minus the 
Laplacian on j-forms.  We can also consider the space of \emph{all} forms,
never mind the j, and on this space there is a Hamiltonian H, which is 
just minus the Laplacian on \emph{all} forms.  Now, we could try to take 
the trace of exp(-tH), but it's more interesting to take the 
"supertrace":

str(exp(-tH)) = tr(exp(-tH_{0})) - tr(exp(-tH_{1})) + 
tr(exp(-tH_{2})) - ...

in other words, the trace of exp(-tH) acting on even forms,
\emph{minus} the trace on odd forms.  

Why??  Well, odd forms are sort of "fermionic" in nature, while
even forms are sort of "bosonic".  The idea of supersymmetry
is to throw in minus signs when you've got "odd things", because
they are like fermions, and physicists know that lots formulas
for fermions are just like formulas for bosons, which are "even
things", except for these signs.  That's the rough idea.  It's all
related to how, when you interchange two identical bosons, their
wavefunction remains unchanged, while for fermions it picks up
a phase of -1.

Now the amazing cool thing is that str(exp(-tH)) is independent
of t.  This follows from some stuff called Hodge theory, or, if
you want to really show off, index theory.   Basically it works
like this.  If you have an operator A with eigenvalues \lambda _i,
then 

                 tr(exp(-tA)) = \sum_{i} exp(-t \lambda _{i})
if the sum makes sense.  We can use this formula to write out
str(exp(-tH)) in terms of eigenvalues of the Laplacians H_{j},
and it turns out that all the terms coming from nonzero eigenvalues
exactly cancel!  So all that's left is the part coming from the
zero eigenvalues, which is independent of t.  If you believe 
this for a second, it means we can compute the supertrace by taking 
the limit as t \to  \infty .  The eigenvalues are all nonnegative,
so all the quantities exp(-t \lambda _{i}) go to zero except for the
zero eigenvalues, and we're left with str(exp(-tH)) being equal to
the alternating sum of the dimensions of the spaces

                  {\psi  : H_{j} \psi  = 0}
Now in fact, Hodge theory tells us that these spaces are really just 
the "cohomology groups" of our configuration space, so the answer we
get for str(exp(-tH)) is what folks call the "Euler characteristic" of our 
configuration space... an important topological invariant.   

So, generalizing the heck out of this idea, we can hope to get
TQFTs from supersymmetric quantum field theories as follows.
Start with some recipe for associating to each choice of "space"
M a "configuration space" C(M)... some space of fields on M,
typically.  Let Z(M) be the space of all forms on C(M), and
let H be the minus the Laplacian, as an operator on Z(M).  Then we expect
that the partition function  str(exp(-tH))  will be independent of
t.  This is just what one wants in a TQFT.  Moreover, the partition 
function will be the Euler characteristic of the configuration space C(M).  

But what if we want to get a TQFT out of this trick, and avoid 
reference to the Laplacian?  Then we can just do the following equivalent
thing (at least it's morally equivalent: there will usually be things to
check).  Let Z_{+}(M) 
be the direct sum of all the even cohomology groups of
C(M), and let Z_{-}(M) be the direct sum of all the odd ones.  Then

str(exp(-tH)) = dim(Z_{+}(M)) - dim(Z_{-}(M))

so what we expect is, not quite a TQFT in the Atiyah sense, but
a "superTQFT" whose space of states has an "even" part equal
to Z_{+}(M) and an "odd" part equal to Z_{-}(M); 
the right hand side is then the "superdimension" of the space of states this
"superTQFT" assigns to M.

Now actually in real life things get tricky because the configuration
space C(M) might be infinite-dimensional, or a singular variety.
If C(M) is too weird, it gets hard to say what its Euler characteristic
should be!  But as Blau and Thompson's paper and the references in it
point out, one can often still make it make sense, with enough work.
In particular, when we are dealing with Donaldson theory, C(M) is
just the moduli space of flat SU(2) connections on M.  This means
that the partition function of S^{1} x M should be the Euler 
characteristic 
of moduli space, better known as the Casson invariant.  And what is
the vector space our superTQFT assigns to M?  Well, it's called
Floer homology.  Now actually there are a lot of subtleties here
I'm deliberately sloughing over.  Read Blau and Thompson's paper
for some of them -- and read the references for more!


\par\noindent\rule{\textwidth}{0.4pt}
% </A>
% </A>
% </A>
