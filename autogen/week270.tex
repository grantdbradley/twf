
% </A>
% </A>
% </A>
\week{October 11, 2008 }

Greg Egan has a new novel out, called "Incandescence" - so I want to 
talk about that.  Then I'll talk about three of my favorite numbers: 
5, 8, and 24.  I'll show you how each regular polytope with 5-fold 
rotational symmetry has a secret link to a lattice living in twice 
as many dimensions.  For example, the pentagon is a 2d projection of 
a beautiful shape that lives in 4 dimensions.  Finally, I'll wrap up 
with a simple but surprising property of the number 12. 

But first: another picture of Jupiter's moon Io!  Now we'll zoom in
much closer.  This was taken
in 2000 by the Galileo probe:


<div align = "center">
<a href = "http://apod.nasa.gov/apod/ap000606.html">
<img src = "io_lava.jpg">
% </a>
</div>

1) A continuous eruption on Jupiter's moon Io, Astronomy Picture 
of the Day, <a href = "http://apod.nasa.gov/apod/ap000606.html">http://apod.nasa.gov/apod/ap000606.html</a>

Here we see a vast plain of sulfur and silicate rock, 250 kilometers
across - and on the left, glowing hot lava!  The white dots are 
spots so hot that their infrared radiation oversaturated the detection
equipment.  This was the first photo of an active lava flow on another
world.  

If you like pictures like this, maybe you like science fiction.  And
if you like hard science fiction - "diamond-scratching
hard", as one reviewer put it - Greg Egan is your man.  His
latest novel is one of the most realistic evocations of the distant
future I've ever read.  Check out the website:

2) Greg Egan, Incandescence, Night Shade Books, 2008.
Website at <a href = "http://www.gregegan.net/INCANDESCENCE/Incandescence.html">http://www.gregegan.net/INCANDESCENCE/Incandescence.html</a>

The story features two parallel plots.  One is about a galaxy-spanning 
civilization called the Amalgam, and two of its members who go on a 
quest to our Galaxy's core, which is home to enigmatic beings that 
may be still more advanced: the Aloof.  The other is about the 
inhabitants of a small world orbiting a black hole.  This is where 
the serious physics comes in.  

I might as well quote Egan himself:

\begin{quote}
  "Incandescence" grew out of the notion that the theory of general 
  relativity - widely regarded as one of the pinnacles of human 
  intellectual achievement - could be discovered by a pre-industrial 
  civilization with no steam engines, no electric lights, no radio 
  transmitters, and absolutely no tradition of astronomy.

  At first glance, this premise might strike you as a little hard 
  to believe.  We humans came to a detailed understanding of gravity 
  after centuries of painstaking astronomical observations, most 
  crucially of the motions of the planets across the sky.  Johannes 
  Kepler found that these observations could be explained if the 
  planets moved around the sun along elliptical orbits, with the 
  square of the orbital period proportional to the cube of the 
  length of the longest axis of the ellipse.  Newton showed 
  that just such a motion would arise from a universal attraction 
  between bodies that was inversely proportional to the square of 
  the distance between them.  That hypothesis was a close enough 
  approximation to the truth to survive for more than three centuries.

  When Newton was finally overthrown by Einstein, the birth of the 
  new theory owed much less to the astronomical facts it could explain -
  such as a puzzling drift in the point where Mercury made its closest 
  approach to the sun - than to an elegant theory of electromagnetism 
  that had arisen more or less independently of ideas about gravity. 
  Electrostatic and magnetic effects had been unified by James Clerk 
  Maxwell, but Maxwell's equations only offered one value for the speed 
  of light, however you happened to be moving when you measured it. 
  Making sense of this fact led Einstein first to special relativity, 
  in which the geometry of space-time had the unvarying speed of light 
  built into it, then general relativity, in which the curvature of the 
  same geometry accounted for the motion of objects free-falling 
  through space.

  So for us, astronomy was crucial even to reach as far as Newton, and 
  postulating Einstein's theory - let alone validating it to high 
  precision, with atomic clocks on satellites and observations of 
  pulsar orbits - depended on a wealth of other ideas and technologies.
 
  How, then, could my alien civilization possibly reach the same 
  conceptual heights, when they were armed with none of these apparent 
  prerequisites?   The short answer is that they would need to be 
  living in just the right environment: the accretion disk of a large 
  black hole.

  When SF readers think of the experience of being close to a black 
  hole, the phenomena that most easily come to mind are those that are 
  most exotic from our own perspective: time dilation, gravitational 
  blue-shifts, and massive distortions of the view of the sky.  But 
  those are all a matter of making astronomical observations, or at 
  least arranging some kind of comparison between the near-black-hole 
  experience and the experience of other beings who have kept their 
  distance.  My aliens would probably need to be sheltering deep inside
  some rocky structure to protect them from the radiation of the 
  accretion disk - and the glow of the disk itself would also render 
  astronomy immensely difficult.

  Blind to the heavens, how could they come to learn anything at all 
  about gravity, let alone the subtleties of general relativity?  After
  all, didn't Einstein tell us that if we're free-falling, weightless, 
  in a windowless elevator, gravity itself becomes impossible to detect?

  Not quite!  To render its passenger completely oblivious to gravity, 
  not only does the elevator need to be small, but the passenger's 
  observations need to be curtailed in time just as surely as they're 
  limited in space.  Given time, gravity makes its mark.  Forget about 
  black holes for a moment: even inside a windowless space station 
  orbiting the Earth, you could easily prove that you were not just 
  drifting through interstellar space, light-years from the nearest 
  planet.  How?  Put on your space suit, and pump out all the station's 
  air.  Then fill the station with small objects - paper clips, pens, 
  whatever - being careful to place them initially at rest with respect
  to the walls.

  Wait, and see what happens.

  Most objects will eventually hit the walls; the exact proportion will
  depend on the station's spin. But however the station is or isn't
  spinning, some objects will undergo a cyclic motion, moving back and
  forth, all with the same period.

  That period is the orbital period of the space station around the 
  Earth.  The paper clips and pens that are moving back and forth 
  inside the station are following orbits that are inclined at a very 
  small angle to the orbit of the station's center of mass.  Twice in 
  every orbit, the two paths cross, and the paper clip passes through 
  the center of the space station.  Then it moves away, reaches the 
  point of greatest separation of the orbits, then turns around and 
  comes back.

  This minuscule difference in orbits is enough to reveal the fact that
  you're not drifting in interstellar space.  A sufficiently delicate 
  spring balance could reveal the tiny "tidal gravitational force" 
  that is another way of thinking about exactly the same thing, but 
  unless the orbital period was very long, you could stick with the 
  technology-free approach and just watch and wait.

  A range of simple experiments like this - none of them much harder 
  than those conducted by Galileo and his contemporaries - were the 
  solution to my aliens' need to catch up with Newton.  But catching 
  up with Einstein?  Surely that was beyond hope?

  I thought it might be, until I sat down and did some detailed 
  calculations.  It turned out that, close to a black hole, the 
  differences between Newton's and Einstein's predictions would easily 
  be big enough for anyone to spot without sophisticated instrumentation.

  What about sophisticated mathematics?  The geometry of general 
  relativity isn't trivial, but much of its difficulty, for us, 
  revolves around the need to dispose of our preconceptions.  By 
  putting my aliens in a world of curved and twisted tunnels, rather 
  than the flat, almost Euclidean landscape of a patch of planetary 
  surface, they came better prepared for the need to cope with a 
  space-time geometry that also twisted and curved.

  The result was an alternative, low-tech path into some of the most 
  beautiful truths we've yet discovered about the universe.  To add 
  to the drama, though, there needed to be a sense of urgency; the 
  intellectual progress of the aliens had to be a matter of life and 
  death.  But having already put them beside a black hole, danger was 
  never going to be far behind.
\end{quote}

As you can tell, this is a novel of ideas.  You have to be willing to
work through these ideas to enjoy it.  It's also not what I'd call
a feel-good novel.  As with "Diaspora" and 
"Schild's Ladder", the main characters seem to become more and 
more isolated and focused on their work as they delve deeper into the 
mysteries they are pursuing.  By the time the mysteries are unraveled,
there's almost nobody to talk to.  It's a problem many mathematicians 
will recognize.  Indeed, near the end of "Diaspora" we read: 
"In the end, there was only mathematics".  
 
So, this novel is not for everyone!  But then, neither is 
This Week's Finds.  

In fact, I was carrying "Incandescence" with me when in
mid-September I left the scorched and smoggy sprawl of southern
California for the cool, wet, beautiful old city of Glasgow.  I spent
a lovely week there talking math with Tom Leinster, Eugenia Cheng,
Danny Stevenson, Bruce Bartlett and Simon Willerton.  I'd been invited to the
University of Glasgow to give a series of talks called the 2008 Rankin
Lectures.  I spoke about my three favorite numbers, and you can see
the slides here:

3) John Baez, My favorite numbers, available at
<a href = "http://math.ucr.edu/home/baez/numbers/">http://math.ucr.edu/home/baez/numbers/</a>

I wanted to explain how different numbers have different personalities
that radiate like force fields through diverse areas of mathematics and 
interact with each other in surprising ways.  I've been exploring 
this theme for many years here.  So, it was nice to polish some things
I've written and present them in a more organized way.  These lectures 
were sponsored by the trust that runs the Glasgow Mathematical Journal, 
so I'll eventually publish them there.   I plan to add a lot of detail 
that didn't fit in the talks.

I began with the number 5, since the golden ratio and the five-fold 
symmetry of the dodecahedron lead quickly to a wealth of easily enjoyed 
phenomena: from Penrose tilings and quasicrystals, to Hurwitz's theorem on 
approximating numbers by fractions, to the 120-cell and the Poincare
homology sphere.  

After giving the first talk I discovered the head of the math department, 
Peter Kropholler, is a big fan of Rubik's cubes.  I'd never been attracted 
to them myself.  But his enthusiasm was contagious, especially when he 
started pulling out the unusual variants that he collects, eagerly 
explaining their subtleties.  My favorite was the Rubik's dodecahedron, 
or "Megaminx":

4) Wikipedia, Megaminx, <a href = "http://en.wikipedia.org/wiki/Megaminx">http://en.wikipedia.org/wiki/Megaminx</a>

Then I got to thinking: it would be even better to have a Rubik's 
icosahedron, since its symmetries would then include M_{12}, the smallest 
Mathieu group.  And it turns out that such a gadget exists!  It's 
called "Dogic":

5) Wikipedia, Dogic, <a href = "http://en.wikipedia.org/wiki/Dogic">http://en.wikipedia.org/wiki/Dogic</a>

The Mathieu group M_{12} is the smallest of the sporadic finite simple
groups.  Someday I'd like to understand the Monster, which is the 
biggest of the lot.  But if the Monster is the Mount Everest of finite 
group theory, M_{12} is like a small foothill.  A good place to start.

Way back in "<a href = "week20.html">week20</A>", I gave a
cute description of M_{12} lifted from Conway and Sloane's
classic book.  If you get 12 equal-sized balls to touch a central one
of the same size, and arrange them to lie at the corners of a regular
icosahedron, they don't touch their neighbors.  There's even room to
roll them around in interesting ways!  For example, you can twist 5 of
them around clockwise so that this arrangement:


\begin{verbatim}

                           1
 
                      5         2
                           6
                        
                        4     3
\end{verbatim}
    
becomes this: 


\begin{verbatim}

                           5
 
                      4         1
                           6
                        
                        3     2
\end{verbatim}
    
We can generate lots of permutations of the 12 outer balls using 
twists of this sort - in fact, all even permutations.  But suppose
we only use moves where we first twist 5 balls around clockwise and 
then twist 5 others counterclockwise.  These generate a smaller group: 
the Mathieu group M_{12}.

Since we can do twists like this in the Dogic puzzle, I believe
M_{12} sits inside the symmetry group of this puzzle!  In a
way it's not surprising: the Dogic puzzle has a vast group of
symmetries, while M_{12} has a measly

8 \times  9 \times  10 \times  11 \times  12 = 95040

elements.  But it'd still be cool to have a toy where you can explore 
the Mathieu group M_{12} with your own hands!

The math department lounge at the University of Glasgow has some old 
books in the shelves waiting for someone to pick them up and read them
and love them.  They're sort of like dogs at the pound, sadly waiting 
for somebody to take them home.  I took one that explains how Mathieu 
groups arise as symmetries of "<a href = "http://en.wikipedia.org/wiki/Steiner_system">Steiner systems</a>":

6) Thomas Beth, Dieter Jungnickel, and Hanfried Lenz, Design Theory,
Cambridge U. Press, Cambridge, 1986.

Here's how they get M_{12}.  Take a 12-point set and think of it as
the "projective line over F11" - in other words, the integers mod 
11 together with a point called infinity.  Among the integers mod 11, 
six are perfect squares:

{0,1,3,4,5,9}

Call this set a "block".  From this, get a bunch more blocks
by applying fractional linear transformations:

z |\to  (az + b)/(cz + d)

where the matrix 

(a b)<br/>
(c d)<br/>

has determinant 1.
These blocks then form a "(5,6,12) Steiner system".  In other words: 
there are 12 points, 6 points in each block, and any set of 5 points 
lies in a unique block.  

The group M_{12} is then the group of all transformations of the 
projective line that map points to points and blocks to blocks!

If I make more progress on understanding this stuff I'll let you
know.  It would be fun to find deep mathematics lurking in mutant
Rubik's cubes.

Anyway, in my second talk I turned to the number 8.  This gave me a 
great excuse to tell the story of how Graves discovered the octonions, 
and then talk about sphere packings and the marvelous E_{8} lattice, 
whose points can also be seen as "integer octonions".   I also sketched
the basic ideas behind Bott periodicity, triality, and the role of 
division algebras in superstring theory.  

If you look at my slides you'll also see an appendix that describes two
ways to get the E_{8} lattice starting from the dodecahedron.   This is a 
nice interaction between the magic powers of the number 5 and those of 
the number 8.  After my talk, Christian Korff from the University of 
Glasgow showed me a paper that fits this relation into a bigger pattern:

7) Andreas Fring and Christian Korff, Non-crystallographic reduction
of Calogero-Moser models, Jour. Phys. A 39 (2006), 1115-1131.   Also 
available as <a href = "http://arxiv.org/abs/hep-th/0509152">arXiv:hep-th/0509152</a>.

They set up a nice correspondence between some non-crystallographic
Coxeter groups and some crystallographic ones:

the H_{2} Coxeter group and the A_{4} Coxeter group,<br/>
the H_{3} Coxeter group and the D_{6} Coxeter group,<br/>
the H_{4} Coxeter group and the E_{8} Coxeter group.<br/>

A Coxeter group is a finite group of linear transformations of 
R^{n} that's generated by reflections.  We say such a group is
"non-crystallographic" if it's not the symmetries of any lattice. 
The ones listed above are closely tied to the number 5:

H_{2} is the symmetry group of a regular 
<a href = "http://en.wikipedia.org/wiki/Pentagon">pentagon</a>.<br/>
H_{3} is the symmetry group of a regular <a href = "http://en.wikipedia.org/wiki/Dodecahedron">dodecahedron</a>.<br/>
H_{4} is the symmetry group of a regular <a href = "http://en.wikipedia.org/wiki/120-cell">120-cell</a>.<br/>

Note these live in 2d, 3d and 4d space.  Only in these dimensions 
are there regular polytopes with 5-fold rotational symmetry!  Their 
symmetry groups are non-crystallographic, because no lattice can 
have 5-fold rotational symmetry.

A Coxeter group is "crystallographic", or a "Weyl
group", if it <i>is</i> symmetries of a lattice.  In particular:

A_{4} is the symmetry group of a 4-dimensional lattice also
called A_{4}.<br/> 
D_{6} is the symmetry group of a 6-dimensional lattice also 
called D_{6}.<br/> 
E_{8} is the symmetry group of an 8-dimensional lattice also 
called
E_{8}.<br/>

You can see precise descriptions of these lattices in "<a href = "week65.html">week65</A>" -
they're pretty simple.

Both crystallographic and noncrystallographic Coxeter groups are 
described by Coxeter diagrams, as explained back in "<a href = "week62.html">week62</A>".  The 
H_{2}, H_{3} and H_{4} Coxeter diagrams look like this:


\begin{verbatim}

  5
o---o

  5
o---o---o

  5
o---o---o---o
\end{verbatim}
    

The A_{4}, D_{6} and E_{8} Coxeter diagrams (usually
called Dynkin diagrams) have twice as many dots as their smaller
partners H_{2}, H_{3} and H_{4}:
 

\begin{verbatim}

        o---o---o---o

                o
                |
    o---o---o---o---o

                o
                |
                o
                | 
o---o---o---o---o---o
\end{verbatim}
    

I've drawn these in a slightly unorthodox way to show how they
"grow".

In every case, each dot in the diagram corresponds to one of the
reflections that generates the Coxeter group.  The edges in the
diagram describe relations - you can read how in "<a href =
"week62.html">week62</A>".

All this is well-known stuff.  But Fring and Korff investigate
something more esoteric.  Each dot in the big diagram corresponds to 
2 dots in its smaller partner:


$$

  5
o---o                       o---o---o---o
A   B                       B'  A"  B"  A'



                                    o C"
  5                                 |
o---o---o               o---o---o---o---o
A   B   C               C'  B'  A"  B"  A'


                                    o D"
                                    |
                                    o C"
  5                                 |
o---o---o---o       o---o---o---o---o---o
A   B   C   D       D'  C'  B'  A"  B"  A'
$$
    

If we map each generator of the smaller group (say, the generator D in
H_{4}) to the product of the two corresponding generators in
the bigger one (say, D'D" in E_{8}), we get a group
homomorphism.

In fact, we get an <i>inclusion</i> of the smaller group in the bigger
one! 

This is just the starting point of Fring and Korff's work.  Their 
real goal is to show how certain exactly solvable physics problems
associated to crystallographic Coxeter groups can be generalized to
these three noncrystallographic ones.  For this, they must develop
more detailed connections than those I've described.  But I'm already
happy just pondering this small piece of their paper.

For example, what does the inclusion of H_{2} in A_{4}
really look like?

It's actually quite beautiful.  H_{2} is the symmetry group of a
regular pentagon, including rotations and reflections.  A_{4} happens
to be the symmetry group of a 4-simplex.  If you draw a 4-simplex
in the plane, it looks like a pentagram:

<div align = "center">
<img src = "240px-Complete_graph_K5.svg.png">
</div>

So, any symmetry of the 
pentagon gives a symmetry of the 4-simplex.  So, we get an inclusion
of H_{2} in A_{4}. 

People often say that Penrose tilings arise from lattices in 4d space.  
Maybe I'm finally starting to understand how!  The A_{4} lattice has a 
bunch of 4-simplices in it - but when we project these onto the plane 
correctly, they give pentagrams.   I'd be very happy if this were
the key.

What about the inclusion of H_{3} in D_{6}?

Here James Dolan helped me make a guess.  H_{3} is the
symmetry group of a regular dodecahedron, including rotations and
reflections.  D_{6} consists of all linear transformations of
R^{6} generated by permuting the 6 coordinate axes and
switching the signs of an even number of coordinates.  But a
dodecahedron has 6 "axes" going between opposite pentagons!
If we arbitrarily orient all these axes, I believe any rotation or
reflection of the dodecahedron gives an element of D_{6}.  So,
we get an inclusion of H_{3} in D_{6}.

And finally, what about the inclusion of H_{4} in E_{8}?  

H_{4} is the symmetry group of the 120-cell, including
rotations and reflections.  In 8 dimensions, you can get 240
equal-sized balls to touch a central ball of the same size.
E_{8} acts as symmetries of this arrangement.  There's a
clever trick for grouping the 240 balls into 120 ordered pairs, which
is explained by Fring and Korff and also by Conway's "icosian"
construction of E_{8} described at the end of my talk on the
number 8.  Each element of H_{4} gives a permutation of the
120 faces of the 120-cell - and thanks to that clever trick, this
gives a permutation of the 240 balls.  This permutation actually comes
from an element of E_{8}.  So, we get an inclusion of
H_{4} in E_{8}.

My last talk was on the number 24.  Here I explained Euler's crazy
"proof" that

1 + 2 + 3 + ... = -1/12

and how this makes bosonic strings happy when they have 24 transverse
directions to wiggle around in.  I also touched on the 24-dimensional
Leech lattice and how this gives a version of bosonic string theory
whose symmetry group is the Monster: the largest sporadic finite simple group.

A lot of the special properties of the number 24 are really properties
of the number 12 - and most of these come from the period-12 behavior
of modular forms.  I explained this back in "<a href =
"week125.html">week125</A>".  I recently ran into these papers
describing yet another curious property of the number 12, also related
to modular forms, but very easy to state:

8) Bjorn Poonen and Fernando Rodriguez-Villegas, Lattice polygons
and the number 12.  Available at 
<a href = "http://citeseerx.ist.psu.edu/viewdoc/summary?doi=10.1.1.43.2555">http://citeseerx.ist.psu.edu/viewdoc/summary?doi=10.1.1.43.2555</a>

9) John M. Burns and David O'Keeffe, Lattice polygons in the plane
and the number 12, Irish Math. Soc. Bulletin 57 (2006), 65-68.
Also available at <a href = "http://www.maths.tcd.ie/pub/ims/bull57/M5700.pdf">http://www.maths.tcd.ie/pub/ims/bull57/M5700.pdf</a>

Consider the lattice in the plane consisting of points with integer
coordinates.  Draw a convex polygon whose vertices lie on this lattice.   
Obviously, the <i>differences</i> of successive vertices also lie on the 
lattice.  We can create a new convex polygon with these differences as 
vertices.  This is called the "dual" polygon.  

Say our original polygon is so small that the only lattice point in its
interior is (0,0).  Then the same is true of its dual!  Furthermore, 
the dual of the dual is the original polygon!

But now for the cool part.  Take a polygon of this sort, and add up the 
number of lattice points on its boundary and the number of lattice 
points on the boundary of its dual.  The total is 12.

You can see an example in Figure 1 of the paper by Poonen and
Rodriguez-Villegas:

<div align = "center">
<img style = "border:none;" src = "lattice_polygons_and_the_number_12.jpg">
</div>

Note that p_{2} - p_{1} = q_{1} and so on.
The first polygon has lattice 5 points on its boundary; the second,
its dual, has 7.  The total is 12.   

I like how Poonen and Rodriguez-Villegas' paper uses this theorem as a
springboard for discussing a big question: what does it mean to
"explain" the appearance of the number 12 here?
They write:

\begin{quote}
   Our reason for selecting this particular statement, besides the 
   intriguing appearance of the number 12, is that its proofs display 
   a surprisingly rich variety of methods, and at least some of them 
   are symptomatic of connections between branches of mathematics that 
   on the surface appear to have little to do with one another.  The 
   theorem (implicitly) and proofs 2 and 3 sketched below appear in 
   Fulton's book on toric varieties.  We will give our new proof 4, 
   which uses modular forms instead, in full.
\end{quote}

\par\noindent\rule{\textwidth}{0.4pt}
\textbf{Addenda:} I thank Adam Glesser and David Speyer for catching 
mistakes.

The only noncrystallographic Coxeter groups are the
symmetry groups of the 120-cell (H_{4}), the dodecahedron
(H_{3}), and the regular n-gons where n = 5,7,8,9,...  The last
list of groups is usually called I_{n} - or better,
I_{2}(n), so that the subscript denotes the number of dots in
the Dynkin diagram, as usual.  But Fring and Korff use "H_{2}"
as a special name for I_{2}(5), and that's nice if you're
focused on 5-fold symmetry, because then H_{2} forms a little
series together with H_{3} and H_{4}.

If you examine Poonen and Rodriguez-Villegas' picture
carefully, you'll see a subtlety
concerning the claim that the dual of the dual is the original
polygon.  Apparently you need to count every boundary point as a vertex!
Read the papers for more precise details.

For more discussion visit the <a href = "http://golem.ph.utexas.edu/category/2008/10/this_weeks_finds_in_mathematic_31.html">\emph{n}-Category
Caf&eacute;</a>.

\par\noindent\rule{\textwidth}{0.4pt}
<em>When the blind beetle crawls over the surface of a globe, he doesn't
realize that the track he has covered is curved.  I was lucky enough
to have spotted it.</em> - Albert Einstein

\par\noindent\rule{\textwidth}{0.4pt}

% </A>
% </A>
% </A>
