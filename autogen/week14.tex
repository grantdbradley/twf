
% </A>
% </A>
% </A>
\week{May 8, 1993 }

Things are moving very fast in the quantum gravity/4d topology game, so
I feel I should break my vow not to continue this series until after
next weekend's conference on Knots and Quantum Gravity.  

Maybe I should recall where things were when I left off.  The physics
problem motivating a lot of work in theoretical physics today is
reconciling general relativity and quantum theory.   The key feature of
general relativity is that time and space do not appear as a "background
structure," but rather are dynamical variables.  In mathematical terms,
this just means that there is not a fixed metric; instead gravity \emph{is}
the metric, and the metric evolves with time like any other physical
field, satisfying some field equations called the Einstein equations.  

But it is worth stepping back from the mathematics and trying to put
into simple words why this makes general relativity so special.  Of
course, it's very hard to put this sort of thing into words.  But
roughly, we can say this: in Newtonian mechanics, there is a universal
notion of time, the "t" coordinate that appears in all the equations of
physics, and we assume that anyone with a decent watch will be able to
keep in synch with everyone else, so there is no confusion about what
this "t" is (apart from choosing when to call t = 0, which is a small
sort of arbitrariness one has to live with).   In special relativity
this is no longer true; watches moving relative to each other will no
longer stay in synch, so we need to pick an "inertial frame," a notion
of rest, in order to have a "t" coordinate to play with.  Once we pick
this inertial frame, we can write the laws of physics as equations
involving "t".  This is not too bad, because there is only a
finite-parameter family of inertial frames, and simple recipes to
translate between them, and also because nothing going on will screw up
the functioning of our (idealized) clocks: that is, the "t" coordinate
doesn't give a damn about the \emph{state} of the universe.  That's what is
meant by saying a "background structure" - it's some aspect of the
universe that is unaffected by everything else that's going on.

In general relativity, things get much more interesting: there is no
such thing as an inertial frame that defines coordinates on spacetime,
because there is no way you can get a lot of things at different places
to remain at rest with each other - this is what is meant by saying that
spacetime is curved.  You can measure time with your watch, so-called
"proper time," but this applies only near you.  More interestingly
still, to compare what your watch is doing to what someone else's is
doing, you actually need to know a lot about the state of the universe,
e.g., whether there are any heavy masses around that are curving
spacetime.  The "metric," whereby one measures distances and proper
time, depends on the state of the universe - or more properly, it is
part of the state of the universe.  

Trying to do \emph{quantum} theory in this context has always been too hard
for people.  Part of the reason why is that built into the heart of
traditional quantum theory is the "Hamiltonian," which describes the
evolution of the state of the system relative to a God-given
"background" notion of "t".  Anyone who has taken quantum mechanics will
know that the star of the show is the Schrodinger equation:


$$

                      i dPsi/dt = H \Psi 
$$
    

saying how the wavefunction \Psi  changes with time in a way depending on
the Hamiltonian H.  No "t," no "H" - this is one basic problem with
trying to reconcile quantum theory with general relativity.

Actually, it turns out that the analog to Schrodinger's equation for
quantum gravity is the Wheeler-DeWitt equation.  The Hamiltonian is
replaced by an operator called the "Hamiltonian constraint" and we have


$$

                         H \Psi  = 0.
$$
    

Note how this cleverly avoids mentioning "t"!  The problem is, people
still aren't quite sure what to do with the solutions to this equation -
we're so used to working with Schrodinger's equation.

Now in 1988 Witten wrote a paper in which he coined the term
"topological quantum field theory," or TQFT, for short.  This was meant
to capture in a rigorous way what field theories like quantum gravity
should be like.  Actually, Witten was working on a different theory
called Donaldson theory, which also has the property of having no
background structures.  Shortly thereafter the mathematician Atiyah came
up with a formal definition of a TQFT.   To get an idea of this
definition, try my notes on <A HREF = "symmetries.html">symmetries</A> and 
(if you don't know what categories are) <A HREF = "categories.html">categories</A>.
For a serious tour of TQFTs and the like, try his book:

The Geometry and Physics of Knots, by Michael Atiyah, Cambridge U.
Press, 1990.

One can think of a TQFT as a framework in which a Wheeler-DeWitt-like
equation governs the dynamics of a quantum field theory.  Experts may
snicker here, but it is true, if not as enlightening as other things one
can say.  

I won't bother to define TQFTs here, but I think Smolin put it very well
when he said the idea of TQFTs really helped us break out of our
traditional idea of fields as being something defined at every point of
spacetime, wiggling around, and allowed us to see field theory from many
new angles.  For example, TQFTs let us wiggle out of the old conundrum
of whether spacetime is continuous or discrete, because many TQFTs can
be \emph{equivalently} described in either of two ways: via a continuum model
of spacetime, or via a discrete one in which spacetime is given a
"simplicial structure," like a big tetrahedral tinkertoy lattice kind of
thing.  The latter idea appears to be due to Turaev and Viro, although
certainly physicists have had similar ideas for years, going back to
Ponzano and Regge, who worked on simplicial quantum gravity.

Now the odd thing is that while interesting 3d TQFTs have been found,
the most notable being Chern-Simons theory, nobody has quite been able
to make 4d TQFTs rigorous.  Witten's original work on Donaldson theory
has led to many interesting things, but not yet a full-fledged TQFT in
the rigorous sense of Atiyah.  And quantum gravity still resists being
formulated as a TQFT.  

A while back I noted that Crane and Yetter had invented a 4d TQFT using
the simplicial approach.  There has been a lot of argument over whether
this TQFT is interesting or "trivial."  Of course, trivial is not a
precise concept.  For a while Ocneanu claimed that the partition
function of every compact 4-manifold equalled 1 in this TQFT, which
counts as very trivial.  But this appears not to be the case.  Broda
invented another 4d TQFT and here on "This Week's Finds" Ruberman showed
it was trivial in the sense that the partition function of any compact
4-manifold was a function of the "signature" of the 4-manifold.  This is
trivial because the signature is a well-understood invariant and if we
are trying to do something new and interesting that just isn't good enough.

In the following paper:

1) Skein theory and Turaev-Viro invariants, by Justin Roberts, Pembroke
College preprint, April 14, 1993 (Roberts is at J.D.Roberts@pmms.cam.ac.uk)

Roberts \emph{almost} claims to show that the Crane-Yetter invariant is
trivial in the same sense, namely that the partition function of any
compact 4-manifold is an exponential of the signature.  Now if Crane and
Yetter's own computations are correct, this cannot be the case, but it
\emph{could} be an exponential of a linear combination of the signature and
the Euler characteristic, as far as I know.  The catch is that Roberts
does not normalize his version of the Crane-Yetter invariant in the same
way that Crane and Yetter do, so it is hard to compare results.  But
Roberts says: "The normalisations here do not agree with those in Crane
and Yetter, and I have not checked the relationship.  However, when
dealing with the [3d TQFT] invariants, different normalisations of the
initial data change the invariants by factors depending on standard
topological invariants (for example Betti numbers), so there is every
reason to belive that these [4d TQFT] invariants are trivial (that is,
they differ from 1 only by standard invariant factors) in all
normalisations."  

This is a bit of a disappointment, because Crane at least had hoped that
their TQFT might actually turn out to \emph{be} quantum gravity.  This was
not idle dreaming; it was because the Crane-Yetter construction was a
rigorous analog of some work by Ooguri on simplicial quantum gravity.  


Then, about a week ago, Rovelli put a paper onto the net:

2) The basis of the Ponzano-Regge-Turaev-Viro-Ooguri model is the loop
representation basis, 16 pages in LaTeX, Friday April 30, available as
<A HREF = "http://xxx.lanl.gov/abs/hep-th/9304164">hep-th/9304164</A>. 

This is a remarkable paper that I have not been able to absorb yet.
First it goes over 3d quantum gravity - which \emph{has} been made into a
rigorous TQFT.  It works with the simplicial formulation of the theory.
That is, we consider our (3-dimensional) spacetime as being chopped up
into tetrahedra, and assign to each edge a length, which is required to
be 0,1/2,1,3/2,....  This idea of quantized edge-lengths goes back to
4d work of Ponzano and Regge, but recently Ooguri showed that in 3d
this assumption gives the same answers as Witten's continuum approach to
3d quantum gravity.  The "half-integers" 0,1/2,1,3/2, etc. should remind
physicists of spin, which is quantized in the same way, and
mathematically this is exactly what is going on: we are really labelling
edges with representations of the group SU(2), that is, spins.  What
Rovelli shows is that if one starts with the loop representation of 3d
quantum gravity (yet another approach), one can prove it equivalent to
Ooguri's approach, and what's more, using the loop representation one
can \emph{calculate} the lengths of edges of triangles in a given state of
space (space here is a 2-dimensional triangulated surface) and \emph{show}
that lengths are quantized in units of the Planck length over 2.  (Here
the Planck length L is the fundamental length scale in quantum gravity,
about 1.6 times 10^{-33} meters.)  

And, most tantalizing of all, he sketches a generalization of the above
to 4d.  In 4d it is known that in the loop representation of quantum
gravity it is areas of surfaces that are quantized in units of L^2/2,
rather than lengths.  Rovelli considers an approach where one chops
4-dimensional spacetime up into simplices and assigns to each
2-dimensional face a half-integer area.  He uses this to write down a
formula for the inner product in the Hilbert space of quantum gravity -
thus, at least formally, answering the long-standing "inner product
problem" in quantum gravity.  The problem is that, unlike in 3d quantum
gravity, here one must sum over ways of dividing spacetime into
simplices, so the formula for the inner product involves a sum that does
not obviously converge.  This is however sort of what one might expect,
since in 4d quantum gravity, unlike 3d, there are "local excitations" -
local wigglings of the metric, if you will - and this makes the Hilbert
space be infinite-dimensional, whereas the 3d TQFTs have
finite-dimensional Hilbert spaces.  

I think I'll quote him here.  It's a bit technical in patches, but worth
it... 


\begin{quote}
We conclude with a consideration on the formal structure of 4-d quantum
gravity, which is important to understand the above construction.  Standard
quantum field theories, as QED and QCD, as well as their generalizations like
quantum field theories on curved spaces and perturbative string theory, are
defined on metric spaces.  Witten's introduction of the topological quantum
field theories has shown that one can construct quantum field theories
defined on a manifold which has only its differential structure, and no fixed
metric structure. The theories introduced by Witten and axiomatized by
Atiyah have the following peculiar feature: they have a finite number of
degrees of freedom, or, equivalently, their quantum mechanical Hilbert
spaces are finite dimensional; classically this follows from the fact that
the number of fields is equal to the number of gauge transformations. However,
not any diff-invariant field theory on a manifold has a finite number of
degrees of freedom.  Witten's gravity in 3-d is given by the action 
                S[A,E] = integral(F ^ E)
which has finite number of degrees of freedom. Consider the action
              S[A,E] =  integral(F ^ e ^ e)
in 3+1 dimensions, for a (self dual) SO(3,1) connection A and a (real)
one form e with values in the vector representation of
SO(3,1). This theory has a strong resemblance with its 2+1 dimensional
analog: it is still defined on a differential manifold without any fixed
metric structure, and is diffeomorphism invariant.  We expect that a
consistent quantization of such a theory should be found along lines
which are more similar to the quantization of the integral(F ^ E),
theory than to the quantization of theories on flat space, based on the
Wightman axioms namely on n-points functions and related objects.
Still, the theory integral(F ^ e ^ e) has genuine field
degrees of freedom: its physical phase space is infinite dimensional, and we
expect that its Hilbert state space will also be infinite dimensional.  There
is a popular belief that a theory defined on a differential manifold
without metric and diffeomorphism invariant has necessarily a finite
number of degrees of freedom ("because thanks to general covariance
we can gauge away any local excitation"). This belief is of course wrong.  A
theory as the one defined by the action integral(F ^ e ^ e) 
is a theory that shares many features with the topological theories, in
particular, no quantity defined "in a specific point" is gauge
invariant; but at the same time it has genuinely infinite degrees of
freedom.  Indeed, this theory is of course nothing but (Ashtekar's
form of) standard general relativity.

The fact that "local" quantities like the n-point functions are not
appropriate to describe quantum gravity non-perturbatively has been
repeatedly noted in the literature.  As a consequence, the issue of
what are the quantities in terms of which a quantum theory of gravity can be
constructed is a much debated issue. The above discussion indicates
a way to face the problem: The topological quantum field theories studied by
Witten and Atiyah provide a framework in terms of which quantum gravity
itself may be framed, in spite of the infinite degrees of freedom.  In
particular, Atiyah's axiomatization of the topological field theories
provides us with a clean way of formulating the problem.  Of course, we
have to relax the requirement that the theory has a finite number of
degrees of freedom.  These considerations leads us to propose that the
correct general axiomatic scheme for a physical quantum theory of
gravity is simply Atiyah's set of axioms up to finite dimensionality
of the Hilbert state space. We denote a structure that satisfies all
Atiyah's axioms, except the finite dimensionality of the state space,  as
a \textbf{generalized topological theory}. 

The theory we have sketched is an example of such a generalized topological
theory. We associate to the connected
components of the boundary of M the infinite
dimensional state space of the Loop Representation of quantum
gravity.  Eq. 5 [the magic formula I alluded to - jb], then, provides a map,
in Atiyah's sense, between the state spaces constructed on two of these
boundary components. Equivalently, it provides the definition of the Hilbert
product in the state space.

One could argue that the framework we have described cannot be
consistent, because it cannot allow us to recover the "broken phase
of gravity" in which we have a nondegenerate background metric: in
the proposed framework one has only non-local observables on the
boundaries, while in the broken phase a \emph{local} field in M has
non-vanishing vacuum expectation value. We think that this argument is
weak because it disregards the diffeomorphism invariance of the theory:
in classical general relativity no experiment can distinguish a
Minkowskian spacetime metric from a non-Minkowkian flat metric.  The two
are physically equivalent, as two gauge-related Maxwell potentials.  For
the same reason, no experiment could detect the absolute \emph{position} of,
say, a gravitational wave, (while of course the position of an e.m. wave
is observable in Maxwell theory).
Physical locality in general relativity is only defined as coincidence
of some physical variable with some other physical variable, while in
non general relativistic physics locality is defined with respect to a
fixed metric structure.  In classical general relativity, there is no
gauge-invariant obervable which is local in the coordinates. Thus,
any observation can be described by means of the value of the fields
on arbitrary boundaries of spacetime. This is the correct consequence
of the fact that "thanks to general covariance we can gauge away any
local excitation", and this is the reason for which one can have the ADM
"frozen time" formalism.  The spacetime manifold of general relativity
is, in a sense, a much weaker physical object than the spacetime metric
manifold of ordinary theories.  All the general relativistic physics can
be read from the boundaries of this manifold. At the same time, however,
these boundaries still carry an infinite dimensional number of degrees
of freedom. 
\end{quote}

Next, let me take the liberty of describing some work of my own:

3)  Diffeomorphism-invariant generalized measures on the space of
connections modulo gauge transformations, by John Baez, to appear in the
proceedings of the Conference on Quantum Topology, Manhattan, Kansas, 
May 8, 1993, available as <A HREF = "state.tex">state.tex</A>.

This is an extremely interesting paper by a very good mathematician.
Whoops!   Let's be objective here.   In the loop representation of
quantum gravity, states of quantum gravity are given naively by certain
"measures" on a space A/G of connections modulo gauge transformations.  
The Chern-Simons path integral is also such a "measure".  In both cases,
the "measure" in question is invariant under diffeomorphisms of space.  
And in both cases, the loop transform allows one to think of these
measures as instead being functions of multiloops (collections of loops
in space).  This is the origin of the relationship to knot theory.  

The problem, as always in quantum field theory, is that the notion of
"measure" must be taken with a big grain of salt - it's not the sort of
measure they taught you about in real analysis.  Instead, these measures
are a kind of "generalized measure" that allows you to integrate not all
continuous functions on A/G but only those lying in an algebra called
the "holonomy algebra," defined by Ashtekar, Isham and Lewandowski.
To be precise and technical, this is the closure in the L^infty norm of
the algebra of functions on A/G generated by "Wilson loops," or traced
holonomies around loops.   So what we are really interested in is not
diffeomorphism-invariant measures on A/G, but diffeomorphism invariant
elements of the dual of the holonomy algebra.  I begin with a review of
generalized measures, introduce the holonomy algebra, and then do a
bunch of new work in which I show how to rigorously construct lots 
of diffeomorphism-invariant elements of the dual of the holonomy algebra
by doing lattice gauge theory on graphs embedded in space.  Again, as
with the work discussed above, we see that the discrete and continuum
approaches to space go hand-in-hand!  And we see that there are some
interesting connections between singularity theory and group
representation theory showing up when we try to understand "measures" on
the space A/G.  


The following is a part of a paper discussed in "<A HREF = "week5.html">week5</A>", now available
from gr-qc:

4)  Completeness of Wilson loop functionals on the moduli space of
SL(2,C) and SU(1,1)-connections, Abhay Ashtekar and Jerzy Lewandowski,
Plain TeX, 7 pages, available as <A HREF = "http://xxx.lanl.gov/abs/gr-qc/9304044">gr-qc/9304044</A>.  

I didn't discuss this aspect of the paper, so let me quote the abstract:

\begin{quote}
The structure of the moduli spaces M := A/G of (all, not just
flat) SL(2,C) and SU(1,1) connections on a n-manifold is analysed.
For any topology on the corresponding spaces A of all connections
which satisfies the weak requirement of compatibility with the affine
structure of A, the moduli space M is shown to be non-Hausdorff.
It is then shown that the Wilson loop functionals - i.e., the traces
of holonomies of connections around closed loops - are complete in the
sense that they suffice to separate all separable points of M. The
methods are general enough to allow the underlying n-manifold to be
topologically non-trivial and for connections to be defined on
non-trivial bundles. The results have implications for canonical
quantum general relativity in 4 and 3 dimensions.
\end{quote}

By the way, someone should extend this result to more general noncompact
semisimple Lie groups, and also show that for all compact semisimple Lie
groups the Wilson loop functionals in any faithful representation \emph{do}
separate points (this is known for the fundamental representation of
SU(n)).  If I had a bunch of grad students I would get one to do so.


The following was discussed in an earlier edition of this series,
"<A HREF = "week11.html">week11</A>," but is now available from gr-qc:

5)  An algebraic approach to the quantization of constrained systems: finite
dimensional examples, by Ranjeet S. Tate, (Ph.D. Dissertation, Syracuse
University), 124 pages, LaTeX (run thrice before printing), available as
<A HREF = "http://xxx.lanl.gov/abs/gr-qc/9304043">gr-qc/9304043</A>.  


I haven't read the following one but it seems like an interesting
application of loop variables to more down-to-earth physics; Gambini was
one of the originators of the loop representation, and intended it for
use in QCD:

6) SU(2) QCD in the path representation, by Rodolfo Gambini and Leonardo
Setaro, LaTeX 37 pages (7 figures included), available as
<A HREF = "http://xxx.lanl.gov/abs/hep-lat/9305001">hep-lat/9305001</A>.  ("hep-lat" is the computational and lattice
physics preprint list, at hep-lat@ftp.scri.fsu.edu.)

Let me quote the abstract:

\begin{quote}
We introduce a path-dependent hamiltonian representation (the path
representation) for SU(2) with fermions in 3 + 1 dimensions. The
gauge-invariant operators and hamiltonian are realized in a Hilbert
space of open path and loop functionals. We obtain a new type of
relation, analogous to the Mandelstam identity of second kind, that
connects open path operators with loop operators. Also, we describe the
cluster approximation that permits to accomplish explicit calculations
of the vacuum energy density and the mass gap.  
\end{quote}
\par\noindent\rule{\textwidth}{0.4pt}

% </A>
% </A>
% </A>
