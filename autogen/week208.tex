
% </A>
% </A>
% </A>
\week{November 6, 2004 }

Last weekend I went to a conference at the Perimeter Institute:

1) Workshop on Quantum Gravity in the Americas, 
<A HREF = "http://www.perimeterinstitute.ca/activities/scientific/PI-WORK-2/">
http://www.perimeterinstitute.ca/activities/scientific/PI-WORK-2/</A>

It was great to see the new building.  I'd visited this institute
before in its temporary location, which was a funky old hotel 
building complete with pool tables and a bar.    The new building is 
very different: four stories of intensely modern architecture
overlooking a lake, consisting mainly of an enormous atrium lined with 
walkways and glass-walled offices.  There's also a big lecture 
theater, a couple of smaller seminar rooms, a library, a restaurant 
whose walls are all blackboards, a reflecting pool, and lots of 
little places to sit and talk, complete with espresso machines.  

In short, a theoretical physicist's idea of heaven!  

But perhaps the design of heaven shouldn't be left to theoretical
physicists.  Some aspects of the setup don't seem very comfortable.
Like most modern architecture, the place is short on coziness - 
there's too much glass, metal and concrete for my taste.  You 
also find yourself spending a lot of time climbing up and down 
uncomfortably narrow staircases. 

The last, at least, is no accident: they made the stairs skinny on 
purpose, so you have to say hello to anyone you meet going the other 
way.  It'll be interesting to see how many collaborative papers come 
out of this.

Abhay Ashtekar was supposed to give the first talk, but he got lost
walking to the new building, so suddenly I had to give the first talk.
Yikes!    Jet-lagged and not fully awake, I sketched the problem of 
dynamics in quantum gravity: the problem of describing motion in a 
world where the geometry of spacetime is quantum-mechanical and 
interacts with matter.  I gave a generally downbeat assessment of the 
progress so far in all known approaches:

2) John Baez, The problem of dynamics in quantum gravity,
<A HREF = "http://math.ucr.edu/home/baez/dynamics/">
http://math.ucr.edu/home/baez/dynamics/</A>

Even though the last few Weeks have been on quantum gravity conferences,
I've been mainly working on n-categories lately, because I've been sort
of fed up with quantum gravity.  I did, however, sketch some avenues 
for progress - and later in this workshop I saw some work that really
cheered me up!

For example, I've always been fascinated by John Wheeler's old dream of
"matter without matter".  In its original version, the idea was to 
describe elementary particles as the ends of wormholes: if there's
an electric field going in one end and out the other, the ends will 
look like particles of equal and opposite charge!  So, the formation
of a wormhole could mimic the creation of a particle-antiparticle
pair.  But there were big problems with this idea: for example, 
getting the wormhole ends to act like spin-1/2 particles.

More recently this idea was reincarnated in the spin network formalism 
by Lee Smolin, with spin network edges replacing wormholes:

3) Lee Smolin, Fermions and topology, available as 
<A HREF = "http://xxx.lanl.gov/abs/gr-qc/9404010">gr-qc/9404010</A>.

A spin network is a gadget with vertices and edges, where the edges
represent "field lines" - lines of the electric field or the analogous
thing for other forces, including gravity.  If a spin network edge goes 
between vertices that would otherwise be far apart, it acts a bit 
like a wormhole.  It will be hidden from observers in the rest of 
spacetime, and its ends will look like particles of equal and opposite 
charge.  Even better, it seems easy to get spin-1/2 particles this way:
they don't call them "spin networks" for nothing!    

A variant on this idea is to have spin networks with "loose ends":
edges that just fizzle out.  This is more ad hoc, but easier to study
in some ways.   A decade ago, Kirill Krasnov and I showed how to describe 
the kinematics of charged spin-1/2 particles this way:

4) John Baez and Kirill Krasnov, Quantization of diffeomorphism-
invariant theories with fermions, <A HREF = "http://xxx.lanl.gov/abs/hep-th/9703112">hep-th/9703112</A>.

However, the hard problem in quantum gravity is always dynamics. 
 
Does the dynamics of spin networks with loose ends actually mimic that
of particles?  Recently Krasnov and other people have begun tackling 
this question in a toy model, 3-dimensional Lorentzian gravity:

5) Kirill Krasnov, \Lambda <0 Quantum Gravity in 2+1 Dimensions I: 
Quantum States and Stringy S-Matrix, Class. Quant. Grav. 19 (2002) 
3977-3998, also available as <A HREF = "http://xxx.lanl.gov/abs/hep-th/0112164">hep-th/0112164</A>.

Kirill Krasnov, \Lambda <0 Quantum Gravity in 2+1 Dimensions II: 
Black Hole Creation by Point Particles, Class. Quant. Grav. 19 (2002) 
3999-4028, also available as <A HREF = "http://xxx.lanl.gov/abs/hep-th/0202117">hep-th/0202117</A>.

He saw that in this theory you can indeed think of particles as spin 
network ends - though you don't need to emphasize that viewpoint, 
since there are also other nice ways to think about what's going on,
using hyperbolic geometry and complex analysis.  It all fits together 
in a beautiful picture.  In principle you can even calculate 
the amplitude for particles to form black holes when they collide!

In this conference, Laurent Freidel explained how this idea works in
3-dimensional Riemannian gravity - a less physical but mathematically 
more tractable spin foam model.  Some but not all of his work can
be found here:

6) Laurent Freidel and David Louapre, Ponzano-Regge model revisited I: 
Gauge fixing, observables and interacting spinning particles, available
as <A HREF = "http://xxx.lanl.gov/abs/hep-th/0401076">hep-th/0401076</A>.

Laurent Freidel and David Louapre, Ponzano-Regge model revisited II: 
Equivalence with Chern-Simons, available as <A HREF = "http://xxx.lanl.gov/abs/gr-qc/0410141">gr-qc/0410141</A>.

Freidel showed that if you take this theory and allow spin networks with
loose ends, they'll act like particles.  The spin of these particles is 
automatically quantized.  More surprisingly, so is their mass - and 
there's an upper bound on the mass!  That's because when we quantize 
this theory, its gauge group automatically gets replaced by a "quantum 
group".   Physically, this means that spacetime becomes quantum-mechanical 
in such a way that it no longer makes sense to talk about times shorter 
than about the Planck time.  Since the energy of a particle is proportional 
to the rate at which its wavefunction oscillates, this puts an upper 
bound on the energy of a particle.  And since E = mc^2, this means 
there's an upper bound on the mass a particle can have.  

Mathematically, part of the point is that we can describe 3d Riemannian 
gravity as a gauge theory where the gauge group is the double cover of
the 3d Euclidean group - the analogue of the Poincare group in this 
context.  But when we quantize the theory, this gets replaced by a 
quantum group: the "quantum double" of SU(2).  As with the 3d Euclidean 
group, unitary representations of this quantum group are classified by 
mass and spin... but now both mass and spin are discrete, and both are 
bounded above.

Anyway, what's great about Freidel and Louapre's work is that it gives
a simplified but mathematically rigorous testbed in which loose ends 
of spin networks act like particles.  We can also think about spin 
networks with "hidden edges" in this setup.  So, we should be able to 
do calculations and see if a spin network with a hidden edge acts like 
a spin network with a pair of loose ends - and thus, a particle-antiparticle 
pair.

Unfortunately, all this work on gravity in 3d spacetime doesn't easily
generalize to 4d spacetime.  The reason is that gravitational waves are 
only possible when spacetime has dimension 4 or more... so 3d gravity 
theories have no local degrees of freedom until we include particles: 
all the fun comes from global topology, like wormholes.  That's why 3d 
theories are easy to calculate with - we can use ideas from topological
quantum field theory.  The danger, though, is that these calculations 
are misleading it comes to real-world physics.  Indeed, that's precisely
the sort of thing I was worrying about in my talk.  

So, it really cheered me up when a young guy named Artem Starodubtsev
spoke about a promising new spin foam model of quantum gravity in 4 
dimensions!  He's working on it now with Laurent Freidel.  He has a 
couple of papers out that \emph{hint} at the main ideas, but you'll have to 
wait to see what they're up to now:

7) Artem Starodubtsev, Topological excitations around the vacuum of 
quantum gravity I: The symmetries of the vacuum, available as 
<A HREF = "http://xxx.lanl.gov/abs/hep-th/0306135">hep-th/0306135</A>.

Artem Starodubtsev and Lee Smolin, General relativity with a topological 
phase: an action principle, available as <A HREF = "http://xxx.lanl.gov/abs/hep-th/0311163">hep-th/0311163</A>.

The basic idea is to treat 4d general relativity with positive 
cosmological constant as a perturbation of a topological quantum 
field theory.  The topological theory has a single state, which 
corresponds to a quantum version of "deSitter space": an exponentially
expanding universe similar to the one we see today, but with no
matter.  To calculate in full-fledged gravity, we then use perturbation
theory, getting answers as power series in a coupling constant.  
But the cool part is that unlike ordinary perturbative quantum gravity
this perturbation theory is manifestly diffeomorphism invariant 
term by term.  And each term is a sum over spin foams!

Even better, the coupling constant in this theory is the cosmological 
constant in Planck units!  That's an incredibly small dimensionless 
number: about 10^{-123}.  Physicists like perturbation theory when the 
coupling constant is small, since then the first few terms tend to give 
reasonably accurate answers - even if the whole series diverges.  
For example, quantum electrodynamics gives high-precision answers
because the fine structure constant is about 1/137, which is pretty
small.  But 10^{-123} is \emph{really} small.

I'd seen Starodubtsev talk about this in Marseille (see "<A HREF = "week206.html">week206</A>")
but now he and Freidel have done calculations recovering Newton's
law of gravity in an appropriate approximation from this theory.  
That may not seem like a big deal, but it's actually very cool to see 
Newton's law reemerge from a manifestly diffeomorphism-invariant 
theory of quantum gravity: no model had ever managed this feat before.  

For those of you hungering for technical details, I'll just say that
the topological theory in question is BF theory with the symmetry group
of deSitter spacetime, namely SO(4,1), as the gauge group.  General 
relativity can be regarded as a perturbation of this BF theory by 
borrowing some ideas from the "MacDowell-Mansouri" formulation of 
general relativity.  If you haven't heard of that, well, neither had I.  
It's a sort of old idea:

8) S. W. MacDowell and F. Mansouri, Unified geometric theory of gravity
and supergravity, Phys. Rev. Lett. 38 (1977), 739-742. 

... but here we aren't using anything anything about supergravity,
just the fact that ordinary general relativity can be treated as a
theory with gauge group SO(4,1) and a Lagrangian that breaks this 
symmetry down to the Lorentz group SO(3,1).  The paper by Smolin and
Starodubtsev listed above describes the details, but in the case of
going from SO(5) down to SO(4).  

When we quantize BF theory in 4 dimensions we get a spin foam model
called the Crane-Yetter model, where the spin foams are defined using
the representation theory of a quantum group: a "q-deformed version" of 
the original gauge group.  So, the real engine behind Freidel and 
Starodubtsev's calculations are spin foams involving the q-deformed 
version of SO(4,1), called SO_{q}(4,1).  This is technically tricky 
because SO(4,1) is noncompact, and noncompact quantum groups are just 
beginning to be understood.  So, there's probably still tons of 
mathematical work left to be done.  But, the upshot is that Freidel 
and Starodubtsev calculate stuff as power series in the cosmological 
constant where each term is computed using SO_{q}(4,1) spin foams.  
It's sort of like a souped-up Feynman diagram expansion, but with spin foams 
replacing Feynman diagrams.

Now that I've thrown around enough buzzwords to scare off the kids, 
I can tell you about Lee Smolin's talk, which was definitely X-rated:
for adults only, people who can listen to speculations with just the
right mixture of disbelief and open-mindedness.  It was the last talk
in the conference.  And it was about the possible physical effects of
spin networks with "hidden edges"!  

First of all, he reminded us how these can mimic particles, and
even some of the usual particle interactions.  But then he went 
ahead and suggested that hidden edges can cause nonlocal effects in
physics, like the force of gravity decaying more slowly than 1/r^{2} - 
just as it does in MOND, the wacky but strangely accurate explanation 
of galactic rotation curves that uses a modification of Newtonian 
gravity instead of positing dark matter!  (See "<A HREF = "week206.html">week206</A>" for more on 
MOND.)  It's hard to make up sensible theories of forces that decay 
more slowly than 1/r^{2}, 
but nonlocal interactions would be one way to 
do it... and hidden spin network edges might cause those.  

There are a million things that could go wrong with this idea, but
I like it, because it suggests a way quantum gravity might try to 
explain one of the big mysteries of physics - dark matter.  And until
we get our theories to make contact with experiment, it'll be very hard
for us to tell if they're on the right track.

Anyway, Smolin hasn't come out with a paper on this stuff yet, so we'll
have to wait for more details.

By the way:

In what I've written this week, I've had to seriously downplay the 
cool math involved, to give (I hope) some inkling of the cool physics.
Krasnov work on 2+1-dimensional Lorentzian gravity with positive
cosmological constant uses the fact that the phase space of this
theory is closely related to "Teichmueller space" - the space of 
complex structures mod diffeomorphisms that are connected to the 
identity.  I talked about this space in "<A HREF = "week28.html">week28</A>", but I forgot to say 
that we can think of it as a space of flat SO(2,1) connections mod 
gauge transformations.  Here SO(2,1) is just the Lorentz group in 3
dimensions.  So, if we quantize 2+1 Lorentzian gravity with positive 
cosmological constant, we get a theory where states are described by 
SO_{q}(2,1) spin networks... but this is also a theory of "quantum 
Teichmueller space".  Again this is tricky because SO(2,1) is noncompact, 
but people have made a lot of progress lately, thanks in part to a line 
of work started by Kashaev:

9) R. M. Kashaev, Quantization of Teichmueller spaces and the quantum 
dilogarithm, available as <A HREF = "http://xxx.lanl.gov/abs/q-alg/9705021">q-alg/9705021</A>.

10) L. Chekhov and V. V. Fock, Quantum Teichmueller space, 
Theor. Math. Phys. 120 (1999) 1245-1259, also available as 
<A HREF = "http://www.arXiv.org/abs/math.QA/9908165">
math.QA/9908165</A>.

You can get a sense of who's working on this stuff and what they're
doing by looking at the references for this recent conference on 3d 
quantum gravity in Edinburgh, which unfortunately took place when I 
was in Hong Kong:

11) Workshop on physics and geometry of 3-dimensional quantum gravity,
<A HREF = "http://www.ma.hw.ac.uk/~bernd/references.html">
http://www.ma.hw.ac.uk/~bernd/references.html</A>

I should also add that people don't usually don't talk about the 3d 
Lorentz group SO(2,1) here; they talk about its double cover SL(2,R).

Anyway, I'll quit here.  The next conference on loops and spin foams
will probably happen in Berlin at the Albert Einstein Institute in 
2005, which happens to be the hundredth birthday of special relativity.
I hope we can make a lot of progress before then and make Al proud.

\par\noindent\rule{\textwidth}{0.4pt}
\emph{The best way to have a good idea is to have a lot of ideas}. -
Linus Pauling
(Not necessarily true, but worth keeping in mind.)

\par\noindent\rule{\textwidth}{0.4pt}

% </A>
% </A>
% </A>
