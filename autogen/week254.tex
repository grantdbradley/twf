
% </A>
% </A>
% </A>


This week I'd like to talk about exceptional Lie algebras and the
Standard Model, Witten's new paper on the Monster group and black
holes in 3d gravity, and Connes and Marcolli's new book!  Then 
I want to continue the Tale of Groupoidification.

However, I don't have the energy to do this all now.  And even 
if I did, you wouldn't have the energy to read it.  

So, I'll just point you towards Connes and Marcolli's new book, 
which you can download for free:

1) Alain Connes and Mathilde Marcolli, Noncommutative Geometry,
Quantum Fields and Motives, available at 
<a href = "http://www.alainconnes.org/downloads.html">http://www.alainconnes.org/downloads.html</a>

I hope to discuss it sometime, especially since it tackles a 
question I've been mulling lately: is there a good "explanation" 
for the Standard Model of particle physics?   

For now, I'll start by discussing Witten's latest paper:

2) Edward Witten, Three-dimensional gravity revisited, available
as <a href = "http://arxiv.org/abs/0706.3359">arXiv:0706.3359</a>.

This is a bold piece of work, which seeks to relate the entropy of
black holes in 3d quantum gravity to representations of the 
Monster group - the largest sporadic finite simple group, with 
about 10^{54} elements.  

If the main idea is right, this gives a whole new view of
"Monstrous Moonshine" - the bizarre connection between the
Monster and fundamental concepts in complex analysis like the
j-function.  (See "<A HREF = "week66.html">week66</A>" for a
quick intro to Monstrous Moonshine.)

As the title hints, Witten had already tackled quantum gravity in 
3 spacetime dimensions.  In this earlier work, he argued it was an 
exactly soluble problem: a topological field theory called 
Chern-Simons theory.  However, this theory is really an 
\emph{extension} of gravity to the case of "degenerate" metrics: 
roughly speaking, geometries of spacetime where certain regions 
get squashed down to zero size.  Degenerate metrics are weird.  
So, what happens if we try to quantize 3d gravity while insisting 
that the metric be nondegenerate?

It's hard to say.  So, Witten takes a few clues and cleverly fits them
together to make a surprising guess.  He considers 3d general
relativity with negative cosmological constant.  This has 3d
anti-DeSitter space as a solution.  Anti-DeSitter space has a
"boundary at infinity": a 2d cylinder with a conformal
structure.  The "AdS-CFT" idea, also known as
"holography", suggests that in this sort of situation, 3d
quantum gravity should be completely described by a field theory
living on this boundary at infinity - a field theory theory with all
conformal transformations as symmetries.

Which conformal field theory should correspond to 3d quantum 
gravity with negative cosmological constant?  It depends on the 
value of the cosmological constant!  Some topological arguments 
suggest that the Chern-Simons description of 3d quantum gravity 
is only gauge-invariant when the cosmological constant \Lambda  
takes certain special values, namely 

\Lambda  = -1/(16 k)^{2}

where k is an integer, known as the "level" in Chern-Simons theory.

By the way: I'm working in Planck units here, and I'm assuming 
our Chern-Simons theory is left-right symmetric, just to keep 
things simple.  I may also be making some small numerical errors.

This quantization of the cosmological constant must seem strange 
if you've never seen it before, but it's not really so weird.  
What's weird is that Witten is using Chern-Simons theory to 
determine the allowed values of the cosmological constant even 
though he wants to study what happens if gravity is \emph{not}
described by Chern-Simons theory!   

Witten knows this is weird: later he says "we used the gauge 
theory approach to get some hints about the right values of the 
cosmological constant simply because it was the only tool 
available."  

Indeed, the whole paper seems designed to refute the notion that 
mathematicians get less daring as they get older.  He writes: "We 
make at each stage the most optimistic possible assumption."  
Perhaps he has some evidence for his guesses that he's not 
revealing yet.  Or perhaps he's decided it takes courage verging 
on recklessness to track the Monster to its lair.

Anyway: next Witten relates the level k to something called the 
"central charge" of the conformal field theory living at the 
boundary at infinity.  

What's the "central charge"?  This is a standard concept in 
conformal field theory.  Perhaps the simplest explanation is that 
in a conformal field theory, the total energy of the vacuum state 
is -c/24, where c is the central charge.  So, naively you'd expect 
c = 0, but quantum effects make nonzero values of the vacuum energy
possible, and even typical.  A closely related cool fact is that 
the partition function of a conformal field theory is only a 
well-defined number up to multiples of 

exp(2 \pi  i c / 24)

This means the partition function is a well-defined number when 
c is a multiple of 24.  This happens in certain especially nice 
conformal field theories which are said to have "holomorphic 
factorization".

The appearance of the magic number 24 here is the first taste of
Monstrous Moonshine!  For more on the importance of this number in
string theory, see "<A HREF = "week124.html">week124</A>",
"<A HREF = "week125.html">week125</A>" and "<A HREF =
"week126.html">week126</A>".

As you can see, there are lots of subtleties here, which I really 
don't want to get into, but feel guilty about glossing over.

Here's another.  There are really \emph{two} conformal field
theories in this game: one that describes ripples of the gravitational
field moving clockwise around the boundary at infinity, and another
for ripples moving counterclockwise.  Our simplifying assumption
about left-right symmetry lets us describe these
"right-movers" and "left-movers" with the same
theory.  So, both have the same central charge.

In this case, the relation between central charge and level 
is simple:

c = 24 k

Next, Witten considers the situation where k takes its smallest 
interesting value: k = 1, so c = 24.  It just so happens that 
c = 24 conformal field theories with holomorphic factorization 
have been classified, at least modulo a certain conjecture: 

2) A. N. Schellekens, Meromorphic c = 24 conformal field theories, 
Comm. Math. Phys. 153 (1993) 159-196.  Also available as 
<A HREF = "http://xxx.lanl.gov/abs/hep-th/9205072">hep-th/9205072</A>.

It's believed there are 71 of them.  Which one could describe 
3d quantum gravity?  

Of these 71, \emph{all but one} have gauge symmetries!  Now, Witten 
is assuming 3d quantum gravity is \emph{not} described by Chern-Simons 
theory, which is a gauge theory.  So, he guesses that the one 
exceptional theory is the right one!

And this is a very famous conformal field theory.  It's a theory 
of a bosonic string wiggling around in a 26-dimensional spacetime 
curled up in clever way with the help of a 24-dimensional lattice 
called the Leech lattice.  This theory is famous because its 
symmetry group is the Monster group!  It is, in fact, the simplest 
thing we know that has the Monster group as symmetries.  

For more details, try these - in rough order of increasing 
thoroughness:

3) Terry Gannon, Postcards from the edge, or Snapshots 
of the theory of generalised Moonshine, available as 
<a href = "http://arxiv.org/abs/math/0109067">arXiv:math/0109067</a>.

Terry Gannon, Monstrous Moonshine: the first twenty-five years,
available as <a href =
"http://arxiv.org/abs/math/0402345">arXiv:math/0402345</A>.

4) Richard Borcherds, online papers, available at <a href =
"http://math.berkeley.edu/~reb/papers/">http://math.berkeley.edu/~reb/papers/</a>

5) Igor Frenkel, James Lepowsky, Arne Meurman, Vertex operator 
algebras and the Monster, Academic Press, New York, 1988.

Now, if this monstrous conformal field theory turned out to be 
3d quantum gravity in disguise - viewed from infinity, so to 
speak - it might someday give a much better understanding of 
Monstrous Moonshine.  However, Witten gives no explanation as to 
\emph{why} this theory should be 3d gravity, except for the indirect 
argument I just sketched.  The precise relation between 3d 
quantum gravity and the bosonic string wiggling around in 
26 dimensions remains obscure.

However, while Witten leaves this mysterious, he does offer a 
tantalizing extra tidbit of evidence that the relation is real!

The partition function of the monstrous conformal field theory 
I just mentioned is the j-function, or more precisely:

J(q) = q^{-1} + 196884 q + 21493760 q^{2} + ...

As I mentioned, this function shows up naturally in complex analysis.
More precisely, it parametrizes the moduli space of elliptic curves
(see "<A HREF = "week125.html">week125</A>").  But, its
bizarre coefficients turn out to be dimensions of interesting
representations of the Monster group.  For example, the smallest
nontrivial representation of the Monster has dimension 196883; adding
the trivial representation, we get 196884.  This was one of several
strange clues leading to the discovery of Monstrous Moonshine.

What Witten does is assume that the monstrous conformal field 
theory describes 3d quantum gravity for k = 1, and then use
properties of the j-function to compute the entropy of black 
holes!

I won't attempt to explain the calculation.  Suffice it to say 
that the lightest possible black hole turns out to have 
196883 quantum states - its space of states is the smallest
nontrivial representation of the Monster group.  So, its entropy
is:

ln(196883) &asymp; 12.19

On the other hand, Hawking's semiclassical calculation gives

4 \pi  &asymp; 12.57

The match is not perfect - but it doesn't need to be, since we 
expect quantum corrections to Hawking's formula for small black 
holes.

What's more impressive is that Witten can guess the entropy of
the lightest possible black hole for other values of k - meaning, 
other values of the cosmological constant.  The space of states
of these black holes are always representations of the Monster
group, so we get logarithms of weird-looking integers.  For 
example, for k = 4 the entropy is

ln(81026609426) &asymp; 25.12

while Hawking's formula gives

8 \pi  &asymp; 25.13

Much better!  And, using a formula of Petersson and Rademacher 
for asymptotics of coefficients of the j-function, together with 
some facts about Hecke operators, he shows that as k \to  \infty , 
the agreement becomes perfect!

In short, there are some fascinating hints of a relation between
the Monster group and black hole entropies in 3d gravity, but the
details of Witten's hoped-for "AdS-CFT correspondence" between 3d
gravity and the monstrous conformal field theory remain obscure.
Indeed, there are lots of problems with Witten's proposal:

6) Jacques Distler, Witten on 2+1 gravity,
<a href = "http://golem.ph.utexas.edu/~distler/blog/archives/001335.html">http://golem.ph.utexas.edu/~distler/blog/archives/001335.html</a>

But, time will tell.  In fact, if history is any guide, we can 
expect to see armies of string theorists marching into this 
territory any day now.  So, I'll just pose one question.  

There's a well-known route from 2d rational conformal field 
theories (or "RCFTs") to 3d topological quantum field theories
(or "TQFTs"), which passes through modular tensor categories. 
For example, an RCFT called the Wess-Zumino-Witten model gives 
the TQFT called Chern-Simons theory.  

But, now Witten is saying "3d quantum gravity isn't Chern-Simons theory;
instead, it's something related to the monstrous CFT".  

So: is the monstrous conformal field theory known to be an RCFT? 
If so, what 3d TQFT does it give?  Could this TQFT be the 3d 
quantum gravity theory Witten is seeking?  

Even though Witten is now claiming 3d quantum gravity \emph{can't be}
a TQFT, I think this is an interesting question.  At the very 
least, I'd like to know more about this "Monster TQFT" - if it 
exists.

Now let's move from 3d quantum gravity to real-world particle
physics...

Last week I described some mathematical relations between the 
Standard Model of particle physics, the most famous grand unified 
theories, and some "exceptional" structures in mathematics: the 
exceptional Lie group E_{6}, the complexified octonionic projective 
plane, and the exceptional Jordan algebra.  

This week I want to go a bit further, and talk about the work 
of Kac, Larsson and others on the exceptional Lie superalgebras 
E(3|6), E(3|8) and E(5|10).  

As before, my goal is to point out some curious relations between 
the messy pack of particles we see in nature and the "exceptional" 
structures we find in mathematics.  By this, I mean structures 
that show up when you classify algebraic gadgets, but don't fit 
into nice systematic infinite families.  Right now the Monster
is the king of all exceptional structures, the biggest of the 
26 sporadic finite simple groups.  But, there are lots of other
such structures, and they all seem to be related.

As mentioned back in "<A HREF = "week66.html">week66</A>",
Edward Witten once suggested that the correct theory of our universe
could be an exceptional structure of some sort.  There's even a fun
hand-wavy argument for this idea.  It goes like this: the theory of
our universe \emph{must} be incredibly special, since out of all the
theories we can write down, only one describes the universe that
actually exists!

In particular, lots of very simple theories do \emph{not} describe our 
universe.  So there must be some principle besides simplicity that
picks out the theory of our universe.

Unfortunately, when we try to think about these issues seriously, 
we're quickly led into very deep waters.  In practice, people 
quickly muddy these waters and create a quagmire.  It's very 
hard to discuss this stuff without uttering nonsense.  If you want
to see my try, look at "<A HREF = "week146.html">week146</A>".  

But right now, I prefer to act like a sober, serious mathematical 
physicist.  So, I'll tell you a bit about exceptional Lie 
superalgebras and how they could be related to the Standard Model.

First, some history.  In 1887, Wilhelm Killing sent a letter to 
Friedrich Engel saying he'd classified the simple Lie algebras.  
Besides the "classical" ones - namely the infinite series sl(n,C), 
so(n,C) and sp(n,C) - he found 6 exceptions: a 14-dimensional one, 
two 52-dimensional ones, a 78-dimensional one, a 133-dimensional 
one and a 248-dimensional one.  

In 1894, Eli Cartan finished a doctoral thesis in which he cleaned 
up Engel's work.   In the process, he noticed that Engel's two 
52-dimensional Lie algebras were actually the same.  Whoops!  

So, we now have just 5 "exceptional" simple Lie algebras.  In 
order of increasing size, they're called g_{2}, 
f_{4}, e_{6}, e_{7} and e_{8}.

In 1914, Cartan realized that the smallest exceptional Lie algebra
g_{2}, comes from the symmetry group of the octonions!  Later
it was realized that all 5 are connected to the octonions.  I've
written a lot about this in previous Weeks, but most of that material
can be found here:

6) John Baez, Exceptional Lie algebras, 
<a href = "http://math.ucr.edu/home/baez/octonions/node13.html">http://math.ucr.edu/home/baez/octonions/node13.html</a>

Now, whenever mathematicians do something fun, they want to keep 
doing it, which means \emph{generalizing} it.  

One way to generalize Cartan's work is to study "symmetric
spaces", which I defined last week.  Briefly, a symmetric space
is a manifold equipped with a geometrical structure that's very
symmetrical: so much so that every point is just like every other, and
the view in any direction is the same as the view in the opposite
direction.

In fact, it was Cartan himself who invented the concept of 
symmetric space, and after he classified the simple Lie algebras 
he went ahead and classified these.   

More precisely, I think he classified the "compact Riemannian" 
symmetric spaces.  Every simple Lie algebra gives one of these, 
namely a compact simple group.  But, there are others too.  So, 
compact Riemannian symmetric spaces are a nice generalization 
of simple Lie algebras - and I believe Cartan succeeded in 
classifying them all.  

Again, there are some infinite series, but also some exceptions 
coming from the octonions.  I talked about one of these last week, 
namely EIII, the complexified octonionic projective plane.  You 
can see a list here:

7) Wikipedia, Riemannian symmetric space,
<a href = "http://en.wikipedia.org/wiki/Riemannian_symmetric_space">http://en.wikipedia.org/wiki/Riemannian_symmetric_space</a>

For a quick intro to the classification of simple Lie algebras and 
compact Riemannian symmetric spaces, try this great book:

8) Daniel Bump, Lie Groups, Springer, Berlin, 2004.

For a slower, more thorough introduction, try the book by Helgason
mentioned last Week.

A second way to generalize Cartan's work is to consider simple 
Lie \emph{superalgebras}.  

Lie superalgebras are just like Lie algebras, except they're split
into an "even" or bosonic and "odd" or fermionic
part.  The idea is that we stick minus signs in the usual Lie algebra
formulas whenever we switch two "odd" elements.

This is very natural from a physics viewpoint, since whenever 
you switch two identical fermions, the wavefunction of the 
universe gets multiplied by -1.  (Take my word for it - I've seen 
it happen!)

It's also very natural from a math viewpoint, since "super vector
spaces" form a symmetric monoidal category with almost all the
nice properties of plain old vector spaces.  This lets crazed
mathematicians and physicists systematically generalize pretty much
all of linear algebra to the "super" world.  So, why not Lie
algebras?

The simple Lie superalgebras were classified by Victor Kac in 
1977:

9) Victor Kac, Lie superalgebras, Adv. Math. 26 (1977), 8-96.

Not counting the ordinary simple Lie algebras, there are 8 
series of simple Lie superalgebras and a few exceptional ones.  
At least some of these exceptions come from the octonions:

10)  Anthony Sudbery, Octonionic description of exceptional 
Lie superalgebras, Jour. Math. Phys. 24 (1983), 1986-1988.

Do they all?  I don't know!  Someone please tell me!

A third way to generalize Cartan's work is to classify 
\emph{infinite-dimensional} simple Lie algebras - or for that matter,
Lie superalgebras.

So far I've implicitly assumed all our algebraic gadgets are 
finite-dimensional, but we can lift that restriction.   If you 
try to classify infinite-dimensional gadgets without \emph{any} 
restrictions, it can get really hairy.  It turns out the nice 
thing is to classify "linearly compact" infinite-dimensional 
simple Lie algebras.  I won't define the quoted phrase, since 
it's technical and it's explained near the beginning 
of this paper:

11) Victor Kac, Classification of infinite-dimensional simple 
linearly compact Lie superalgebras, Erwin Schr&ouml;dinger Institut 
preprint, 1998.
Available at <a href = "http://www.esi.ac.at/Preprint-shadows/esi605.html">http://www.esi.ac.at/Preprint-shadows/esi605.html</a>

Anyway, back in 1880 Lie himself made a guess about 
infinite-dimensional Lie algebras that would solve the problem 
I'm talking about now, though he didn't phrase it in the modern
way.  And, Cartan proved Lie's guess in 1909!  Actually, there 
was a hole in Cartan's proof, which was only noticed much later.  
It was filled by Guillemin, Quillen and Sternberg in 1966.

So, here's the answer: there are 4 families of linearly compact 
infinite-dimensional simple Lie algebras, and no exceptions.  
Ignoring an important nuance I'll explain later, these are:

<ul>
<li>
The Lie algebra of all complex vector fields on C^{n}.
</li>
<li>
The Lie algebra of all complex vector fields v on C^{n} 
that are "divergence-free":

    div v = 0
</li>
<li>
The Lie algebra of all complex vector fields v on C^{2n}
 that are "symplectic":

    L_{v} \omega  = 0

    where \omega  is the usual symplectic structure on C^{2n}, 
    and L means "Lie derivative"
</li>
<li>
The Lie algebra of all complex vector fields v on C^{2n+1} 
    that are "contact":

    L_{v} &alpha; = f &alpha;

    for some function f depending on v, where &alpha; is the usual
    contact structure on C^{2n+1}.
</li>
</ul>

If you don't know about symplectic structures or contact 
structures, don't worry - we won't need them now.  The main point 
is that they're differential forms that show up throughout 
classical mechanics.  So, this classification theorem is 
surprisingly nice.  

Notice: no exceptions!  That's a kind of exception in its own 
right. 

In 1998, Victor Kac proved the "super" version of this result.  
In other words, he classified linearly compact infinite-dimensional 
Lie superalgebras!  This result is Theorem 6.3 of his paper above.  
There turn out to be 10 families and 6 exceptions, which are 
called E(1|6), E(2|2), E(3|6), E(3|8), E(4|4) and E(5|10).

Many of the families are straightforward "super" generalizations 
of the 4 families I just showed you.  Some are stranger.  Most 
important for us today are the exceptions discovered by Irina 
Shchepochkina in 1983:

12) Irena Shchepochkina, New exceptional simple Lie superalgebras, 
C. R. Bul. Sci. 36 (1983), 313-314.

The easiest to explain is E(5|10).  And, you'll soon see that the 
number 5 here is related to the math of the SU(5) grand unified 
theory, which I explained last Week!

The even part of E(5|10) is the Lie algebra of divergence-free complex 
vector fields on C^{5}.  

The odd part of E(5|10) consists of closed complex 2-forms on C^{5}.

The bracket of two even guys is the usual Lie bracket of vector 
fields. 

The bracket of an even guy and an odd guy is the usual "Lie 
derivative" of a differential form with respect to a vector field.   

The only tricky bit is the bracket of two odd guys!  So, suppose \mu 
and &nu; are closed complex 2-forms on C^{5}.  Their wedge
product is a 4-form \mu  ^ &nu;.  But, we can identify this with a vector
field v by demanding:

i_{v} vol = \mu  ^ &nu;

Here vol is the volume form:

vol = dx_{1} ^ dx_{2} ^ dx_{3} ^ dx_{4} ^ dx_{5}

and i_{v} vol is the "interior product", which feeds
v into vol and leaves us with a 4-form.  You can check that this
vector field v is divergence-free.  So, we define the bracket of \mu 
and &nu; to be v.

sl(5,C) sits inside the even part of E(5|10) in a nice way, as 
the divergence-free vector fields whose coefficients are \emph{linear} 
functions on C^{5}.   So, since su(5) sits inside sl(5,C), we get a 
tempting relation to SU(5).

(Now I'll come clean now and explain the "important nuance" I 
ignored earlier.  For the classification theorems I mentioned
earlier, we must use vector fields and differential forms with 
\emph{formal power series} as coefficients.    But for the purposes
of mathematical physics, we should keep a more flexible attitude.)

Next, what about E(3|6)?  This is contained in E(5|10).  To define 
it, we give E(5|10) a clever grading where x_{1}, x_{2}, 
x_{3} are 
treated differently from the other two variables.  Then we take 
the subalgebra of degree-zero guys.  The details are explained in 
the above papers - or more simply, here:

13) Victor Kac, Classification of infinite-dimensional simple 
groups of supersymmetries and quantum field theory, available as 
<A HREF = "http://xxx.lanl.gov/abs/math.QA/9912235">math.QA/9912235</A>.

All this is reminiscent of how SU(5) contains the gauge group of 
the Standard Model, namely S(U(3) \times  U(2)).  In particular, the 
even part of E(3|6) contains the Lie algebra

sl(3,C) \oplus  sl(2,C) \oplus  gl(1,C)

in a canonical way.  So, any representation of E(3|6) automatically 
gives a representation of the Standard Model Lie algebra

su(3) \oplus  su(2) \oplus  u(1)

And in the above paper Kac goes even further!  He defines a fairly
natural class of representations of E(3|6), and proves something 
remarkable: these restrict to representations of 

su(3) \oplus  su(2) \oplus  u(1) 

that correspond precisely to the gluon, the photon and the W and 
Z bosons, and the quarks and leptons in one generation...

... together with one other particle, which is \emph{not} the Higgs 
boson, but instead acts like a gluon with electric charge &plusmn;1.  

Darn.  

One nice thing is how these Lie superalgebras get both bosons and 
fermions into the game in a natural way without forcing the 
existence of a bunch of unseen "superpartners".  One unfortunate 
thing is that the above result gives no hint as to why there 
should be three generations of quarks and leptons.  However, Kac 
and Rudakov develop some mathematics to address that question 
here:

14) Victor Kac and Alexi Rudakov, Representations of the 
exceptional Lie superalgebra E(3,6): I. Degeneracy conditions.  
Available as <a href = "http://arxiv.org/abs/math-ph/0012049">math-ph/0012049</a>.

Representations of the exceptional Lie superalgebra E(3,6): II. 
Four series of degenerate modules.  Available as <a href = "http://arxiv.org/abs/math-ph/0012050">math-ph/0012050</a>.

Representations of the exceptional Lie superalgebra E(3,6) 
III: Classification of singular vectors.  Available as 
<a href = "http://arxiv.org/abs/math-ph/0310045">math-ph/0310045</a>.

Their results are summarize at the end of this review article:

15) Victor Kac, Classification of supersymmetries, Proceedings 
of the ICM, Beijing, 2002, vol. 1, 319-344.  Available as 
<a href = "http://arxiv.org/abs/math-ph/0302016">math-ph/0302016</a>

Here Kac writes that "three generations of leptons occur in 
the stable region [whatever that means], but the situation with 
quarks is more complicated: this model predicts a complete fourth 
generation of quarks and an incomplete fifth generation (with 
missing down type triplets)."  

So, while I don't understand "this model", it seems tantalizingly
close to capturing the algebraic patterns in the Standard Model... 
without quite doing so.

Some more nice explanations and references can be found here:

16) Irina Shchepochkina, The five exceptional simple Lie superalgebras
of vector fields.  Available as <A HREF =
"http://arxiv.org/abs/hep-th/9702121">hep-th/9702121</A>.

17) Pavel Grozman, Dimitry Leites and Irina Shchepochkina, 
Defining relations for the exceptional Lie superalgebras of 
vector fields pertaining to The Standard Model, available as 
<a href = "http://arxiv.org/abs/math-ph/0202025">math-ph/0202025</a>.

18) Pavel Grozman, Dimitry Leites and Irina Shchepochkina, 
Invariant operators on supermanifolds and Standard Models, 
available as <a href = "http://arxiv.org/abs/math.RT/0202193">math.RT/0202193</a>.

Thomas Larsson has been working on similar ideas, mainly 
using E(3|8) instead.  This also contains 

sl(3,C) \oplus  sl(2,C) \oplus  gl(1,C) 

in a canonical way.  

19) Thomas A. Larsson, Symmetries of everything, available as 
<a href = "http://arxiv.org/abs/math-ph/0103013">math.RT/0202193</a>.

Exceptional Lie superalgebras, invariant morphisms, and a 
second-gauged Standard Model, available as <a href = "http://arxiv.org/abs/math-ph/020202">math-ph/020202</a>.

Thomas A. Larsson, Maximal depth implies su(3)+su(2)+u(1), 
available as <A HREF = "http://xxx.lanl.gov/abs/hep-th/0208185">hep-th/0208185</A>.  

Alas, E(3|8) gets the hypercharges of some fermions wrong.  
Larsson seems to say this problem also occurs for E(3|6), 
which would appear to contradict what Kac claims - but I 
could be misunderstanding.

I'll end with few questions.  First, is there any relation 
between the exceptional Lie superalgebras E(5|10), E(3|6) or 
E(5|10) and the exceptional Lie algebra e6?  Last week I 
explained some relations between e6 and the Standard Model; 
are those secretly connected to what I'm discussing this week?

Second, has anyone tried to unify all three generalizations of
Cartan's classification of simple Lie algebras?  Starting from 
simple Lie algebras, we've seen three ways to generalize:

<ul>
<li>
go to symmetric spaces,
<li>
go the "super" version,
<li>
go to the infinite-dimensional case.
</ul>

So: has anyone tried to classify infinite-dimensional super 
versions of symmetric spaces?  Or even finite-dimensional ones?  

(Maybe the super version of a symmetric space should be 
called a "supersymmetric space", just for the sake of a nice pun.)

% <a name = "tale">
Next, the Tale of Groupoidification!  I'll keep this week's
episode short, since you're probably exhausted already.

I want to work my way to the concept of "Hecke operator" 
through a series of examples.  The examples I'll use are a bit
trickier than the concept I'm really interested in, since the 
examples involve integrals, where the Hecke operators I ultimately
want to discuss involve sums.  But, the examples are nice if
you like to visualize stuff...

In these examples we'll always have a relation between two sets 
X and Y.  We'll use this to get an operator that turns functions 
on X into functions on Y - a "Hecke operator".  

<ul>
<li>
\textbf{The Radon transform in 2 dimensions}
Suppose you're trying to do a CAT scan.  You want to obtain 
a 3d image of someone's innards.  Unfortunately, all you do is 
take lots of 2d X-ray photos of them.  How can you assemble all 
this information into the picture you want?

Who better to help you out than a guy named after a 
radioactive gas: Radon!  

In 1917, the Viennese mathematician Johann Radon tackled a related 
problem one dimension down.  You could call it a "CAT scan for 
flatlanders".  

Suppose you want to obtain a complete image of the insides of a 
2-dimensional person, but all you can do is shine beams of X-rays 
through them and see how much each beam is attenuated.  

So, mathematically: you have a real-valued function on the plane -
roughly speaking, the density of your flatlander.  You're trying 
to recover this function from its integrals along all possible 
lines.  Someone hands you this function on the space of \emph{lines}, 
and you're trying to figure out the original function on the space
of \emph{points}. 

(Points lying on lines!   If you've been following the Tale of 
Groupoidification, you'll know this "incidence relation" is 
connected to Klein's approach to geometry, and ultimately to 
spans of groupoids.  But pretend you don't notice, yet.)

Now, it's premature to worry about this tricky "inverse problem"
before we ponder what it's the inverse of: the "Radon transform".  
This takes our original function on the space of \emph{points} and 
gives a function on the space of \emph{lines}.

Let's call the Radon transform T.  It takes a function f on the 
space of points and gives a function Tf on the space of lines, 
as follows.  Given a line y, (Tf)(y) is the integral of f(x) over 
the set of all points x lying on y.

What Radon did is figure out a nice formula for the inverse of 
this transform.  But that's not what I'm mainly interested in now.  
It's the Radon transform itself that's a kind of Hecke operator!

Next, look at another example.

</li>
<li>
<b>
The X-ray transform in n dimensions
</b>

This is an obvious generalization to higher dimensions of what 
I just described.  Before we had a space

X = {points in the plane}

and a space

Y = {lines in the plane}

and an incidence relation

S = {(x,y): x is a point lying on the line y}

If we go to n dimensions, we can replace all this with

X = {points in R^{n}}

Y = {lines in R^{n}}

S = {(x,y): x is a point lying on the line y}

Again, the X-ray transform takes a function f on the space of 
points and gives a function Tf on the space of lines.  Given a 
line y, (Tf)(y) is the integral of f(x) over the set of all x
with (x,y) in S.

Next, yet another example!

</li>
<li>
<b>
The Radon transform in n dimensions
</b>

Radon actually considered a different generalization of the
2d Radon transform, using hyperplanes instead of lines.  Using 
hyperplanes is nicer, because it gives a very simple relationship
between the Radon transform and the Fourier transform.  But never 
mind - that's not the point here!  The point is how similar
everything is.  Now we take:

X = {points in R^{n}}

Y = {hyperplanes in R^{n}}

S = {(x,y): x is a point lying on the hyperplane y}

And again, the Radon transform takes a function f on X 
and gives a function Tf on Y.  Given y in Y, (Tf)(y) is the 
integral of f(x) over the set of all x with (x,y) in S.
</ul>

We're always doing the same thing here.  Now I'll describe 
the general pattern a bit more abstractly.

We've always got three spaces, and maps that look like this:


\begin{verbatim}

                     S
                    / \
                   /   \
                 P/     \Q
                 /       \
                v         v 
               X           Y
\end{verbatim}
    
In our examples so far these maps are given by

P(x,y) = x
Q(x,y) = y

But, they don't need to be.

Now, how do we get a linear operator in this situation?  

Easy!  We start with a real-valued function on our space X:

f: X \to  R

Then we take f and "pull it back along P" to get a function on S.  
"Pulling back along P" is just impressive jargon for composing 
with P:

f o P: S \to  R

Next, we take f o P and "push it forwards along Q" to get a 
function on Y.  The result is our final answer, some function

Tf: Y \to  R

"Pushing forwards along Q" is just impressive jargon for 
integrating: Tf(y) is the integral over all s in S with Q(s) = y.
For this we need a suitable measure, and we need the integral
to converge.

This is the basic idea: we define an operator T by pulling
back and then pushing forward functions along a "span", meaning
a diagram shaped like a bridge:


\begin{verbatim}

                     S
                    / \
                   /   \
                 P/     \Q
                 /       \
                v         v 
               X           Y
\end{verbatim}
    
But, the reason this operator counts as a "Hecke operator" 
is that it gets along with a symmetry group G that's acting 
on everything in sight.  In the examples so far, this is 
the group of Euclidean symmetries of R^{n}.  But, it could be
anything.  

This group G acts on all 3 spaces: X, Y, and S.  This makes the 
space of functions on X into a representation of G!  And, ditto 
for the space of function on Y and S.

Furthermore, the maps P and Q are "equivariant", meaning

P(gs) = gP(s)

and

Q(gs) = gQ(s)

This makes "pulling back along P" into an intertwining operator
between representations of G.  "Pushing forwards along Q" will
also be an intertwining operator if the measure we use is 
G-invariant in a suitable sense.  In this case, our transform T 
becomes an intertwining operator between group representations!  
Let's call an intertwining operator constructed this way a "Hecke 
operator".

If you're a nitpicky person, e.g. a mathematician, you may wonder what
I mean by "a suitable sense".  Well, each "fiber"
Q^{-1}(y) of the map

Q: S \to  Y

needs a measure on it, so we can take a function on S and 
integrate it over these fibers to get a function on Y.   We need 
this family of measures to be invariant under the action of G, 
for pushing forwards along Q be an intertwining operator.

This stuff about invariant families of measures is mildly 
annoying, and so is the analysis involved in making precise 
\emph{which} class of functions on X we can pull back to S and then 
push forward to Y - we need to make sure the integrals converge, 
and so on.  When I really get rolling on this Hecke operator 
business, I'll often focus on cases where X, Y, and S are 
\emph{finite} sets... and then these issues go away!

Hmm.  I'm getting tired, but I can't quit until I say one more
thing.  If you try to read about Hecke operators, you \emph{won't}
see anything about the examples I just mentioned!  You're most
likely to see examples where X and Y are spaces of lattices in
the complex plane.  This is the classic example, which we're
trying to generalize.  But, this example is more sophisticated
than the ones I've mentioned, in that the "functions" on X and
Y become "sections of vector bundles" over X and Y.  The same
sort of twist happens when we go from the Radon transform to the
more general "Penrose transform".

Anyway, next time I'll talk about some really easy examples,
where X, Y, and S are finite sets and G is a finite group.
These give certain algebras of Hecke operators, called "Hecke
algebras".

In the meantime, see if you can find \emph{any} reference in the 
literature which admits that "Hecke algebras" are related
to "Hecke operators".  It ain't easy!  

It's a great example of a mathematical cover-up - one we're gonna
bust wide open.

\par\noindent\rule{\textwidth}{0.4pt}
\textbf{Addendum:} David Corfield notes that Helgason has a good
textbook on the Radon transform which is <i>free online</i>.
Snap it up while you can!

20) Sigurdur Helgason, Radon Transform, second edition, Birkh&auml;user,
New York, 1999.
Also available at <a href = "http://www-math.mit.edu/~helgason/Radonbook.pdf">http://www-math.mit.edu/~helgason/Radonbook.pdf</a>.


For more discussion, go to the 
<a href = "http://golem.ph.utexas.edu/category/2007/07/this_weeks_finds_in_mathematic_15.html">\emph{n}-Category Caf&eacute;</a>.

\par\noindent\rule{\textwidth}{0.4pt}
<em>"The Big Crunch" had always been a slightly mocking,
irreverent term, but now she was struck anew by how little justice it
did to the real trend that had fascinated the Niah.  It was not a matter
of everything in mathematics collapsing in on itself, with one branch 
turning out to have been merely a recapitulation of another under a
different guise.  Rather, the principle was that
every sufficiently beautiful mathematical system was rich enough


% parser failed at source line 1179
