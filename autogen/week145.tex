
% </A>
% </A>
% </A>
\week{February 9, 2000 }


I know I promised to talk about homotopy theory and n-categories, but 
I've gotten sidetracked into thinking about projective planes, so I'll 
talk about that this Week and go back to the other stuff later.  Sorry, 
but if I don't talk about what intrigues me at the instant I'm writing
this stuff, I can't get up the energy to write it.  

So:

There are many kinds of geometry.  After Euclidean geometry, one of the
first to become popular was projective geometry.   Projective geometry
is the geometry of perspective.  If you draw a picture on a piece of
paper and view it from a slant, distances and angles in the picture will
get messed up - but lines will still look like lines.  This kind of
transformation is called a "projective transformation".  Projective
geometry is the study of those aspects of geometry that are preserved
by projective transformations.

Interestingly, 2-dimensional projective geometry has some curious
features that don't show up in higher dimensions.  To explain this,  
I need to tell you about projective planes.

I talked a bit about projective planes in "<A HREF = "week106.html">week106</A>".  The basic idea is
to take the ordinary plane and add some points at infinity so that every
pair of distinct lines intersects in exactly one point.  Lines that were
parallel in the ordinary plane will intersect at one of the points at
infinity.  This simplifies the axioms of projective geometry.

But what exactly do I mean by "the ordinary plane"?  Well, ever since
Descartes, most people think of the plane as R^{2}, which consists of
ordered pairs of real numbers.  But algebraists also like to use C^{2},
consisting of ordered pairs of complex numbers.  For that matter, you
could take any field F - like the rational numbers, or the integers 
modulo a prime - and use F^{2}.  Algebraic geometers call this sort of
thing an "affine plane".  

A projective plane is a bit bigger than an affine plane.  For this,
start with the 3-dimensional vector space F^{3}.  Then define the
projective plane over F, denoted FP^{2}, to be the space of lines 
through the origin in F^{3}.  
You can show the projective plane is the same as
the affine plane together with extra points, which play the role of 
"points at infinity".

In fact, you can generalize this a bit - you can make sense of the
projective plane over F whenever F is a division ring!  A division ring
is a like a field, but where multiplication isn't necessarily
commutative.  The best example is the quaternions.  In "<A HREF = "week106.html">week106</A>" 
I talked about the real, complex and quaternionic projective planes, 
their symmetry groups, and their relation to quantum mechanics.  Here's 
a good book about this stuff, emphasizing the physics applications:

1) V. S. Varadarajan, Geometry of Quantum Mechanics, Springer-Verlag,
Berlin, 2nd ed., 1985.  
 
So far, so good.  But there's another approach to projective planes
that's even more general.  This approach goes back to Euclidean
geometry: it's based on a list of axioms.  In this approach, a
projective plane consists of a set of "points", a set of "lines", and 
a relation which tells us whether or not a given point "lies on" a 
given line.  I'm putting quotes around all these words, because in this
approach they are undefined terms.  All we get to work with are the
following axioms:

A) Given two distinct points, there exists a unique line that both
points lie on.

B) Given two distinct lines, there exists a unique point that lies on
both lines.

C) There exist four points, no three of which lie on the same line.

D) There exist four lines, no three of which have the same point lying
on them.  

Actually we can leave out either axiom C) or axiom D) - the rest of the
axioms will imply the one we leave out.  It's a nice little exercise to
convince yourself of this.  I put in both axioms just to make it obvious
that this definition of projective plane is "self-dual".  In other
words, if we switch the words "point" and "line" and switch who lies on
who, the definition stays the same!  

Duality is one of the great charms of the theory of projective planes:
whenever you prove any theorem, you get another one free of charge with
the roles of points and lines switched, thanks to duality.  There are
lots of different kinds of "duality" in mathematics, but this is
probably the grand-daddy of them all.

Now, it's easy to prove that starting from any division ring F, we get a
projective plane FP^{2} satisfying the above axioms.  The fun part 
is to try to go the other way!  Starting from a structure satisfying the
above axioms, can you cook up a division ring that it comes from?  

Well, starting from a projective plane, you can try to recover a
division ring as follows.  Pick a line and throw out one point - and
call that point "the point at infinity".  What's left is an 
"affine
line" - let's call it L.  Let's try to make L into a division ring.  To
do this, we first need to pick two different points in L, which we call
0 and 1.  Then we need to cook up rules for adding and multiplying
points on L.  

For this, we use some tricks invented by the ancient Greeks!   

This should not be surprising.  After all, those dudes thought about
arithmetic in very geometrical ways.   How can you add points on a line
using the geometry of the plane?  Just ask any ancient Greek, and here's
what they'll say:

First pick a line L' that's parallel to L - meaning that L and L'
intersect only at the point at infinity.  Then pick a line M that
intersects L at the point 0 and L' at some point which we call 0'.  
We get a picture like this:

\begin{verbatim}
             |             
   ----------0'------------------------------------------ L'
             |             
             |             
            M|             
             |             
   ----------0------------------------------------------  L
             |             
\end{verbatim}
    
Then, to add two points x and y on L, draw this picture:


\begin{verbatim}
             |                 |
   ----------0'----------------|------------------------- L'
             | \               | \
             |  \              |  \
            M|   \N          M'|   \N'
             |    \            |    \
   ----------0-----x-----------y-----z------------------  L
             |      \          |      \
\end{verbatim}
    
In other words, draw a line M' parallel to M through the point y,
draw a line N through x and 0', and draw a line N' parallel to N and
going through the point where M' and L' intersect.  L and N' intersect
at the point called z... and we define this point to be x + y!   

This is obviously the right thing, because the two triangles in the
picture are congruent.

What about multiplication?  Well, first draw a line L' that intersects
our line L only at the point 0.  Then draw a line M from the point 1 to
some point 1' that's on L' but not on L:

\begin{verbatim}
                  L'/
                   / 
                  /
                 1'
                /|
               / |
              /  |
   ----------0---1------------------------------------  L
                 |
                 |M
                 |
\end{verbatim}
    
Then, to multiply x and y, draw this picture:
\begin{verbatim}
                        /
                       /|\   
                      / | \
                     /  |  \
                  L'/   |   \
                   /    |    \
                  /     |     \
                 1'     |      \
                /|\     |       \
               / | \N   |        \N'
              /  |  \   |         \
   ----------0---1---x--y----------z------------------  L
                 |    \ |           \
                 |M     |M'
                 |      |
\end{verbatim}
    
In other words, draw a line N though 1' and x, draw a line M' parallel
to M through the point y, and draw a line N' parallel to N through the
point where L' and M' intersect.  L and N' intersect at the point called
z... and we define this point to be xy!

This is obviously the right thing, because the triangle containing
the points 1 and x is similar to the triangle containing y and z.  

So, now that we've cleverly figured out how to define addition and
multiplication starting from a projective plane, we can ask: do we
get a division ring?  

And the answer is: not necessarily.   It's only true if our projective
plane is "Desarguesian".  This is a special property named after an 
old theorem about the real projective plane, proved by Desargues.
A projective plane is Desarguesian if Desargues' theorem holds for
this plane.

But wait - there's an even more basic question we forgot to ask! 
Namely: was our ancient Greek method of defining addition and
multiplication independent of the choices we made?  We needed to pick
some points and lines to get things going.  If you think about it hard,
these choices boil down to picking four points, no three of which lie on
a line - exactly what axiom C) guarantees we can do.   

Alas, it turns out that in general our recipe for addition and 
multiplication really depends on \emph{how} we chose these four points.
But if our projective plane is Desarguesian, it does not!   

In fact, if we stick to Desarguesian projective planes, everything works
very smoothly.   For any division ring F the projective plane FP^2 is
Desarguesian.  Conversely, starting with a Desarguesian projective
plane, we can use the ancient Greek method to cook up a division ring F.
Best of all, these two constructions are inverse to each other - at
least up to isomorphism.

At this point you should be pounding your desk and yelling "Great - but
what does `Desarguesian' mean?  I want the nitty-gritty details!"

Okay....

Given a projective plane, define a "triangle" to be three
points that don't lie on the same line.  Now suppose you have two
triangles xyz and x'y'z'.  The sides of each triangle determine three
lines, say LMN and L'M'N'.  If we're really lucky, the line through x
and x', the line through y and y', and the line through z and z' will
all intersect at the same point.  We say that our projective plane is
"Desarguesian" if whenever this happens, something else
happens: the intersection of L and L', the intersection of M and M', and
the intersection of N and N' all lie on the same line.

If you have trouble visualizing what I just said, take a look at this 
webpage, which also gives a proof of Desargues' theorem for the real
projective plane:

2) Roger Mohr and Bill Trigs, Desargues' Theorem, 
<A HREF = "http://spigot.anu.edu.au/people/samer/Research/Doc/ECV_Tut_Proj_Geom/node25.html">http://spigot.anu.edu.au/people/samer/Research/Doc/ECV_Tut_Proj_Geom/node25.html</A>

Desargues' theorem is a bit complicated, but one cool thing is that 
its converse is its dual.  This is easy to see if you stare at it: 
"three lines intersecting at the same point" is dual to "three points 
lying on the the same line".   Even cooler, Desargues' theorem implies 
its own converse!  Thus the property of being Desarguesian is self-dual.

Another nice fact about Desarguesian planes concerns collineations.  A
"collineation" is a map from a projective plane to itself that 
preserves
all lines.   Collineations form a group, and this group acts on the set
of all "quadrangles" - a quadrangle being a list of four points, no
three of which lie on a line.   Axiom C) says that every projective
plane has at least one quadrangle.  It turns out that if a projective 
plane is Desarguesian, the group of collineations acts transitively on 
the set of quadrangles: given any two quadrangles, there's a collineation
carrying one to the other.  This is the reason why the ancient Greek trick 
for adding and multiplying doesn't depend on the choice of quadrangle when 
our projective plane is Desarguesian!

An even more beautiful fact about Desarguesian planes concerns their
relation to higher dimensions.  Just as we defined projective planes
through a list of axioms, we can also define projective  spaces of any
dimension n = 1,2,3,....  The simplest example is FP^{n} - the space of
lines through the origin in F^{n+1}, where F is some division
ring.  The neat part is that when n > 2,
this is the \emph{only} example.  Moreover, any projective plane sittting
inside one of these higher-dimensional projective spaces is automatically 
Desarguesian!  So the non-Desarguesian projective planes are really
freaks of dimension 2.  

All this is very nice.   But there are some obvious further questions, 
namely: what's special about projective planes that actually come from
\emph{fields}, and what can we say about non-Desarguesian projective planes?

The key to the first question is an old theorem proved by the last of
the great Greek geometers, Pappus, in the 3rd century CE.  It turns out
that in any projective plane coming from a field, the Pappus Theorem
holds.  Conversely, any projective plane satisfying the Pappus Theorem 
comes from a unique field.  We call such projective planes "Pappian".

The Pappus theorem will be too scary if I explain it using only words,
so I'll tell you to look at a picture instead.  The fun thing about this 
picture is that you can move the red and green points around with your 
mouse and see how things change:

3) Pappus' theorem (a JavaSketchPad demo by MathsNet),
<A HREF = "http://www.anglia.co.uk/education/mathsnet/dynamic/pappus.html">http://www.anglia.co.uk/education/mathsnet/dynamic/pappus.html</A>

Now, what about the non-Desarguesian projective planes?  If we try to
get a division ring from an \emph{arbitrary} projective plane, we fail
miserably.  However, we can still define addition and multiplication
using the tricks described above.  These operations depend crucially our
choice of a quadrangle.  But if we list all the axioms these operations
satisfy, we get the definition of an algebraic gadget called a "ternary
ring".   

They're called "ternary rings" because they're usually described in
terms of a ternary operation that generalizes xy + z.   But the precise
definition is too depressing for me give here.  It's a classic example
of what James Dolan calls "centipede mathematics", where you take a
mathematical concept and see how many legs you can pull off before it
can no longer walk.   A ternary ring is like a division ring that can
just barely limp along on its last legs. 

I'm not a big fan of centipede mathematics, but there is one really
nice example of a ternary ring that isn't a division ring.  Namely, 
the octonions!  These are almost a division ring, but their multiplication 
isn't associative.  

I already talked about the octonions in "<A HREF = "week59.html">week59</A>", "<A HREF = "week61.html">week61</A>", "<A HREF = "week104.html">week104</A>"
and "<A HREF = "week105.html">week105</A>".  In "<A HREF = "week106.html">week106</A>", I explained how you can define OP^{2}, the
projective plane over the octonions.  This is the best example of a
non-Desarguesian projective plane.  One reason it's so great is that
that its group of collineations is E6.  E6 is one of the five
"exceptional simple Lie groups" - mysterious and exciting things that
deserve all the study they can get!   

Next I want to talk about the relation between projective geometry
and the \emph{other} exceptional Lie groups, but first let me give you
some references.  To start, here's a great book on projective planes 
and all the curious centipede mathematics they inspire: 

4) Frederick W. Stevenson, Projective Planes, W. H. Freeman and Company,
San Francisco, 1972.

You'll learn all about nearfields, quasifields, Moufang loops, Cartesian 
groups, and so on.  

Much of the same material is covered in these lectures by Hall,
which are unfortunately a bit hard to find:

5) Marshall Hall, Projective Planes and Other Topics, California
Institute of Technology, Pasadena, 1954.   

For a more distilled introduction to the same stuff, 
try the last chapter of Hall's book on group theory:

6) Marshall Hall, The Theory of Groups, Macmillan, New York, 1959.

If you're only interested in Desarguesian projective planes, try
this:

7) Robin Hartshorne, Foundations of Projective Geometry, Benjamin, 
New York, 1967. 

In particular, this book gives a nice account of the collineation group 
in the Desarguesian case.  The punchline is simple to state, so I'll 
tell you.  Suppose F is a division ring.  Then the collineation 
group of FP^{2} is generated by two obvious subgroups: PGL(3,F) and 
the automorphism group of F.  The intersection of these two subgroups 
is the group of inner automorphisms of F.  

If the above references are too intense, try this leisurely, literate
introduction to the subject first:

8) Daniel Pedoe, An Introduction to Projective Geometry, Macmillan,
New York, 1963.

And you're really interested in the \emph{finite} projective planes, you
can try this reference, which assumes very little knowledge of algebra:

9) A. Adrian Albert and Reuben Sandler, An Introduction to Finite
Projective Planes, Holt, Rinehart and Winston, New York, 1968.

For a nice online introduction to projective geometry over the real
numbers and its applications to image analysis, try this:

10) Roger Mohr and Bill Triggs, Projective geometry for image analysis,
<A HREF = "http://spigot.anu.edu.au/people/samer/Research/Doc/ECV_Tut_Proj_Geom/node1.html">http://spigot.anu.edu.au/people/samer/Research/Doc/ECV_Tut_Proj_Geom/node1.html</A>

Finally, for interesting relations between projective geometry
and exceptional Lie groups, try this:

11) J. M. Landsberg and L. Manivel: The projective geometry of
Freudenthal's magic square, preprint available as 
<A HREF = "http://xxx.lanl.gov/abs/math.AG/9908039">math.AG/9908039</A>.

The Freudenthal-Tits magic square is a strange way of describing
most of the exceptional Lie groups in terms of the real numbers,
complex numbers, quaternions and octonions.  In the usual way of 
describing it, you start with two of these division algebras, say
F and F'.  Then let J(F) be the space of 3x3 self-adjoint matrices 
with coefficients in F.  This is a Jordan algebra with the product
xy + yx.  As mentioned in "<A HREF = "week106.html">week106</A>", Jordan algebras have a lot to
do with projective planes.  In particular, the nontrivial projections 
in J(F) correspond to the 1- and 2-dimensional subspaces of F^{3},
and thus to the points and lines in the projective plane PF^{2}.   

Next, let J_0(F) be the subspace of J(F) consisting of the \emph{traceless} 
self-adjoint matrices.  Also, let Im(F') be the space of pure imaginary 
element of K'.  Finally, let the magic Lie algebra M(F,F') be given by

M(K,K') = Der(J(K)) + J_0(K) x Im(K') + Der(K')

Here + stands for direct sum, x stands for tensor product, and Der 
stands for the space of derivations of the algebra in question.  
It's actually sort of tricky to describe how to make M(F,F') into
a Lie algebra, and I'm sort of tired, so I'll wimp out and tell
you to read this stuff:

12) Hans Freudenthal, Lie groups in the foundations of geometry,
Adv. Math. 1 (1964) 143.

13) Jacques Tits, Algebres alternatives, algebres de Jordan et 
algebres de Lie exceptionelles, Proc. Colloq. Utrecht, vol. 135, 1962. 

14) R. D. Schafer, Introduction to Non-associative Algebras, Academic
Press, 1966.

By the way, the paper by Freudenthal is a really mind-bending mix of 
Lie theory and axiomatic projective geometry, definitely worth looking
at.  Anyway, if you do things right you get the following square of Lie 
algebras M(F,F'):

\begin{verbatim}
             F = R     F = C     F = H     F = O

F' = R        A1        A2        C3        F4
F' = C        A2       A2+A2      A5        E6
F' = H        C3        A5        B6        E7
F' = O        F4        E6        E7        E8

\end{verbatim}
    

Here R, C, H and O stand for the reals, complexes, quaternions and
octonions.  If you don't know what all the Lie algebras in the square
are, check out "<A HREF = "week64.html">week64</A>".  (I
should admit that the above square is not very precise, because I don't
say which real forms of the Lie algebras in question are showing up.)

The first fun thing about this square is that F4, E6, E7 and E8 are 
four of the five exceptional simple Lie algebras - and the fifth 
one, G2, is just Der(O).  So all the exceptional Lie algebras are
related to the octonions!  And the second fun thing about this square
is that it's symmetrical along the diagonal, even though this is
not at all obvious from the definition.  This is what makes the square
truly "magic".  

I don't really understand the magic square, but it's on my to-do
list.  That's one reason I'm glad there's a new paper out that 
describes the magic square in a way that makes its symmetry manifest:

15) C. H. Barton and A. Sudbery, Magic squares of Lie algebras, 
preprint available as  <A HREF = "http://xxx.lanl.gov/abs/math.RA/0001083">math.RA/0001083</A>.

It also generalizes the magic square in a number of directions.  But
what I really want is for the connection between projective planes, 
division algebras, exceptional Lie groups and the magic square to 
becomes truly \emph{obvious} to me.  I'm a long way from that point.


\par\noindent\rule{\textwidth}{0.4pt}

Here's an interesting email from David Broadhurst 
about the failure of the Pappus theorem in quaternionic projective
space:


\begin{verbatim}
Date: Fri, 3 Mar 2000 20:50:03 GMT
From: D.Broadhurst@open.ac.uk (David Broadhurst)
Subject: Paul Dirac and projective geometry

John:

Shortly before his death I spent a charming afternoon with Paul Dirac.
Contrary to his reputation, he was most forthcoming.

Among many things, I recall this: Dirac explained that while trained
as an engineer and known as a physicist, his aesthetics were mathematical.
He said (as I can best recall, nearly 20 years on): At a young age,
I fell in love with projective geometry.  I always wanted to use to use 
it in physics, but never found a place for it.

Then someone told him that the difference between complex and quaternionic
QM had been characterized as the failure of theorem in classical projective 
geometry.

Dirac's face beamed a lovely smile: "Ah," he said, "it was just such a thing
that I hoped to do".

I was reminded of this when backtracking to your "<A HREF = "week106.html">week106</A>", today.

Best,

David
\end{verbatim}
    


 <HR>
<em>The reader should not attempt to form a mental picture of a closed
straight line.</em> - Frank Ayres, Jr., Projective Geometry



<p> <hr>

% </A>
% </A>
% </A>


% parser failed at source line 621
