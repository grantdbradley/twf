
% </A>
% </A>
% </A>
\week{June 21, 1997}


There are some spooky facts in mathematics that you'd never guess in a
million years... only when someone carefully works them out do they
become clear.  One of them is called "Bott periodicity".  

A 0-dimensional manifold is pretty dull: just a bunch of points.
1-dimensional manifolds are not much more varied: the only
possibilities are the circle and the line, and things you get by
taking a union of a bunch of circles and lines.  2-dimensional
manifolds are more interesting, but still pretty tame: you've got your
n-holed tori, your projective plane, your Klein bottle, variations on
these with extra handles, and some more related things if you allow
your manifold to go on forever, like the plane, or the plane with a
bunch of handles added (possibly infinitely many!), and so on....  You
can classify all these things.  3-dimensional manifolds are a lot
more complicated: nobody knows how to classify them.  4-dimensional
manifolds are a \emph{lot} more complicated: you can \emph{prove}
that it's \emph{impossible} to classify them - that's called Markov's Theorem.  

Now, you probably wouldn't have guessed that a lot of things start
getting simpler when you get up around dimension 5.  Not everything,
just some things.  You still can't classify manifolds in these high
dimensions, but if you make a bunch of simplifying assumptions you
sort of can, in ways that don't work in lower dimensions.  Weird, huh?
But that's another story.  Bott periodicity is different.  It says
that when you get up to 8 dimensions, a bunch of things are a whole
lot like in 0 dimensions!  And when you get up to dimension 9, a bunch
of things are a lot like they were in dimension 1.  And so on - a
bunch of stuff keeps repeating with period 8 as you climb the ladder
of dimensions.

(Actually, I have this kooky theory that perhaps part of the reason
topology reaches a certain peak of complexity in dimension 4 is that
the number 4 is halfway between 0 and 8, topology being simplest in
dimension 0.  Maybe this is even why physics likes to be in 4
dimensions!  But this is a whole other crazy digression and I will
restrain myself here.)

Bott periodicity takes many guises, and I already described one in
"<A HREF = "week104.html">week104</A>".  Let's start with the real numbers, and then throw in n
square roots of -1, say e_{1},...,e_{n}.   Let's make them "anticommute",
so

e_{i} e_{j} = - e_{j} e_{i}

when i is different from j.  What we get is called the "Clifford
algebra" C_{n}.  For example, when n = 1 we get the complex numbers,
which we call C.  When n = 2 we get the quaternions, which we call H,
for Hamilton.  When n = 3 we get... the octonions??  No, not the
octonions, since we always demand that multiplication be associative!
We get the algebra consisting of \emph{pairs} of quaternions!  We call that
H + H.  When n = 4 we get the algebra consisting of 2x2 \emph{matrices} of
quaternions!  We call that H(2).  And it goes on, like this:

C_{0} = R
C_{1} =  C
C_{2} = H
C_{3} = H + H
C_{4} =  H(2)
C_{5} = C(4)
C_{6} = R(8)
C_{7} = R(8) + R(8)
C_{8} = R(16)

Note that by the time we get to n = 8 we just have 16x16 matrices of
real numbers.  And that's how it keeps going: C_{n+8} is just 16x16
matrices of guys in C_{n}!  That's Bott periodicity in its simplest form.

Actually right now I'm in Vienna, at the Schroedinger Institute, and
one of the other people visiting is Andrzej Trautman, who gave a talk
the other day on "Complex Structures in Physics", where he mentioned a
nice way to remember the above table.  Imagine the day is only 8 hours
long, and draw a clock with 8 hours.  Then label it like this:

                        



% parser failed at source line 168
