
% </A>
% </A>
% </A>
\week{February 14, 1998}


A true physicist loves matter in all its states.  The phases we all
learned about in school - solid, liquid, and gas - are just the
beginning of the story!  There lots of others: liquid crystal, plasma,
superfluid, and neutronium, for example.  Today I want to say a little
about two more phases that people are trying to create: quark-gluon
plasma and strange quark matter.  The first almost certainly exists;
the second is a matter of much discussion.

1) The E864 Collaboration, Search for charged strange quark matter
produced in 11.5 A GeV/c Au + Pb collisions, Phys. Rev. Lett. 79 (1997)
3612-3616, preprint available as 
<A HREF = "http://xxx.lanl.gov/abs/nucl-ex/9706004">nucl-ex/9706004</A>.
Last week I went to a talk on the search for strange quark matter by
one of these collaborators, Kenneth Barish.  This talk was based on
Barish's work at the E864 experiment at the "AGS", the alternating
gradient synchrotron at Brookhaven National Laboratory in Long Island,
New York.

What's "strange quark matter"?  Well, first remember from "<A HREF = "week93.html">week93</A>" that
in the Standard Model there are bosonic particles that carry forces:


\begin{verbatim}

    ELECTROMAGNETIC FORCE          WEAK FORCE         STRONG FORCE
        
    photon                          W+                  8 gluons 
                                    W-
                                    Z  
\end{verbatim}
    
and fermionic particles that constitute matter:
      

\begin{verbatim}

        LEPTONS                                     QUARKS

electron        electron neutrino         down quark        up quark
muon            muon neutrino             strange quark     charm quark
tauon           tauon neutrino            bottom quark      top quark
\end{verbatim}
    
(There is also the mysterious Higgs boson, which has not yet been seen.)

The quarks and leptons come in 3 generations each.  The only quarks in
ordinary matter are the lightest two, those from the first generation:
the up and down.  These are the constituents of protons and neutrons,
which are the only stable particles made of quarks.  A proton consists
of two ups and a down held together by the strong force, while a neutron
consists of two downs and a up.  The up has electric charge +2/3, while
the down has electric charge -1/3.  They also interact via the strong
and weak forces.

The other quarks are more massive and decay via the weak interaction
into up and down quarks.  Apart from that, however, they are quite
similar.  There are lots of short-lived particles made of various
combinations of quarks.  All the combinations we've seen so far are of
two basic sorts.  There are "baryons", which consist of 3 quarks, and
"mesons", which consist of a quark and an antiquark.  Both of these
should be visualized roughly as a sort of bag with the quarks and a
bunch of gluons confined inside.

Why are they confined?  Well, I sketched an explanation in "<A HREF = "week94.html">week94</A>", so
you should read that for more details.  For now let's just say the
strong force likes to "stick together", so that energy is minimized if
it stays concentrated in small regions, rather than spreading all over
the place, like the electromagnetic field does.  Indeed, the strong
force may even do something like this in the absence of quarks, forming
short-lived "glueballs" consisting solely of gluons and virtual
quark-antiquark pairs.  (For more on glueballs, see "<A HREF = "week68.html">week68</A>".)

For reasons I don't really understand, the protons and neutrons in the
nucleus do not coalesce into one big bag of quarks.  Even in a neutron
star, the quarks stay confined in their individual little bags.  But
calculations suggest that at sufficiently high temperatures or
pressures, "deconfinement" should occur.  Basically, under these
conditions the baryons and mesons either smash into each other so hard,
or get so severely squashed, that they burst open.  The result should be
a soup of free quarks and gluons: a "quark-gluon plasma".

To get deconfinement to happen is not easy - at low pressures, it's 
expected to occur at a temperature of 2 trillion Kelvin!  According
to the conventional wisdom in cosmology, the last time deconfinement 
was prevalent was about 1 microsecond after the big bang!  In the E864 
experiment, they are accelerating gold nuclei to energies of 11.5 GeV 
per nucleon and colliding them with a fixed target made of lead, which 
is apparently \emph{not} enough energy to fully achieve deconfinement - they
believe they are reaching temperatures of about 1 trillion Kelvin.  
At CERN they are accelerating lead nuclei to 160 GeV per nuclei 
and colliding them with a lead target.  They may be getting signs of 
deconfinement, but as Jim Carr explained in a recent post to sci.physics, 
they're being very cautious about coming out and saying so.  By mid-1999, 
the folks at Brookhaven hope to get higher energies with the Relativistic 
Heavy Ion Collider, which will collide two beams of gold nuclei head-on 
at 100 GeV per nucleon... see "<A HREF = "week76.html">week76</A>" for more on this.

One of the hoped-for signs of deconfinement is "strangeness enhancement".
The lightest quark besides the up and down is the strange quark, and in
the high energies present in a quark gluon plasma, strange quarks should
be formed.  Moreover, since Pauli exclusion principle prevents two identical
fermions from being in the same state, it can be energetically favorable 
to have strange quarks around, since they can occupy lower-energy states 
which are already packed with ups and downs.  They seem to be seeing
strangeness enhancement at CERN:

2) Juergen Eschke, NA35 Collaboration, Strangeness enhancement in sulphur-
nucleus collisions at 200 GeV/N, preprint available as
<A HREF = "http://xxx.lanl.gov/abs/hep-ph/9609242">hep-ph/9609242</A>.

As far as I can tell, people are just about as sure that deconfinement 
occurs at high temperatures as they would be that tungsten boils at
high temperatures, even if they've never actually seen it happen.  A 
more speculative possibility is that as quark-gluon plasma cools down 
it forms "strange quark matter" in the form of 
"strangelets": big bags of 
up, down, and strange quarks.  This is what they're looking for at E864.  
Their experiment would only detect strangelets that live long enough 
to get to the detector.  When their experiment is running they get 10^6 
collisions per second.  So far they've set an upper bound of 10^{-7} 
charged strangelets per collision, neutral strangelets being harder to 
detect and rule out.  For more on strangelets, try this:

3) E. P. Gilson and R. L. Jaffe, Very small strangelets, Phys. Rev. Lett. 
71 (1993) 332-335, preprint available as
<A HREF = "http://xxx.lanl.gov/abs/hep-ph/9302270">hep-ph/9302270</A>.

Strange quark matter is also of interest in astrophysics.  In 1984 
Witten wrote a paper proposing that in the limit of large quark number, 
strange quark matter could be more stable than ordinary nuclear matter!

4) Edward Witten, Cosmic separation of phases, Phys. Rev. D30 (1984) 
272-285.

More recently, a calculation of Farhi and Jaffe estimates that in
the limit of large quark number, the energy of strange quark matter is
301 MeV per quark, as compared with 310 Mev/quark for iron-56, which
is the most stable nucleus.   This raises the possibility that under 
suitable conditions, a neutron star could collapse to become a "quark
star" or "strange star".  Let me quote the abstract of the following
paper:

5) Dany Page, Strange stars: Which is the ground state of QCD at finite 
baryon number?, `High Energy Phenomenology' eds. M. A. Perez \text{\&}  R. Huerta 
(World Scientific), 1992, pp. 347 - 356, preprint available as 
<A HREF = "http://xxx.lanl.gov/abs/astro-ph/9602043">astro-ph/9602043</A>.


\begin{quote}
  Witten's conjecture about strange quark matter (`Strange Matter') being 
  the ground state of QCD at finite baryon number is presented and stars 
  made of strange matter (`Strange Stars') are compared to neutron stars. 
  The only observable way in which a strange star differs from a neutron 
  star is in its early thermal history and a detailed study of strange star 
  cooling is reported and compared to neutron star cooling. One concludes 
  that future detection of thermal radiation from the compact object 
  produced in the core collapse of SN 1987A could present the first 
  evidence for strange matter. 

\end{quote}
Here are a couple of books on the subject, which unfortunately I've
not been able to get ahold of:

6) Strange Quark Matter in Physics and Astrophysics: Proceedings of the
International Workshop on Strange Quark Matter in Physics and Astrophysics,
ed. Jes Madsen, North-Holland, Amsterdam, 1991.

7) International Symposium on Strangeness and Quark Matter, eds. Georges 
Vassiliadis et al, World Scientific, Singapore, 1995.

If anyone out there knows more about the latest theories of strange quark
matter, and can explain them in simple terms, I'd love to hear about it.

Okay, enough of that.

Now, on with my tour of homotopy theory!

So far I've mainly been talking about simplicial sets.  I described a
functor called "geometric realization" that turns a simplicial set into
a topological space, and another functor that turns a space into a
simplicial set called its "singular simplicial set".  I also showed how
to turn a simplicial set into a simplicial abelian group, and how
to turn one of \emph{those} into a chain complex... or vice versa.

As you can see, the key is to have lots of functors at your disposal,
so you can take a problem in any given context - or more precisely,
any given category! - and move it to other contexts where it may be
easier to solve.  Eventually I want to talk about what all these
categories we're using have in common: they are all "model
categories".  Once we understand that, we'll be able to see more
deeply what's going on in all the games we've been playing.

But first I want to describe a few more important tricks for turning
this into that.  Recall from "<A HREF =
"week115.html">week115</A>" that there's a category \Delta  whose
objects 0,1,2,... are the simplices, with n corresponding to the
simplex with n vertices - the simplex with 0 vertices being the
"empty simplex".  We can also define \Delta  in a purely
algebraic way as the category of finite totally ordered sets, with n
corresponding to the totally ordered set {0,1,....,n-1}.  The
morphisms in \Delta  are then the order-preserving maps.  Using this
algebraic definition we can do some cool stuff:

\par\noindent\rule{\textwidth}{0.4pt}
<STRONG>J.</STRONG>  The Nerve of a Category.  This is a trick to turn a category into a
simplicial set.  Given a category C, we cook up the simplicial set
Nerve(C) as follows.  The 0-dimensional simplices of Nerve(C) are just
the objects of C, which look like this:


\begin{verbatim}

                            x
\end{verbatim}
    
The 1-simplices of Nerve(C) are just the morphisms, which look like
this:


$$

                        x---f-->y
$$
    
The 2-simplices of Nerve(C) are just the commutative diagrams that
look like this:


\begin{verbatim}

                            y
                           / \                 
                          f   g
                         /     \
                        x---h-->z 
\end{verbatim}
    
where f: x \to  y, g: y \to  z, and h: x \to  z.  And so on.  In general,
the n-simplices of Nerve(C) are just the commutative diagrams in 
C that look like n-simplices!  

When I first heard of this idea I cracked up.  It seemed like an insane
sort of joke.  Turning a category into a kind of geometrical object
built of simplices?  What nerve!  What use could this possibly be?  

Well, for an application of this idea to computer science, see "<A HREF = "week70.html">week70</A>".
We'll soon see lots of applications within topology.  But first, let me
give a slick abstract description of this "nerve" process that turns
categories into simplicial sets.  It's really a functor

Nerve: Cat \to  SimpSet

going from the category of categories to the category of simplicial
sets.  

First, a remark on Cat.  This has categories as objects and functors
as morphisms.  Since the "category of all categories" is a bit
creepy, we really want the objects of Cat to be all the "small" 
categories, i.e., those having a mere \emph{set} of objects.  This 
prevents Russell's paradox from raising its ugly head and disturbing
our fun and games.  

Next, note that any partially ordered set can be thought of as a
category whose objects are just the elements of our set, and where we
say there's a single morphism from x to y if x <= y.  Composition of
morphisms works out automatically, thanks to the transitivity of "less
than or equal to".  We thus obtain a functor

i: \Delta  \to  Cat

taking each finite totally ordered set to its corresponding category,
and each order-preserving map to its corresponding functor.

Now we can copy the trick we played in section F of "<A HREF = "week116.html">week116</A>".  For any
category C we define the simplicial set Nerve(C) by

                 Nerve(C)(-) = hom(i(-),C) 

Think about it!  If you put the simplex n in the blank slot, we
get hom(i(n),C), which is the set of all functors from that simplex,
\emph{regarded as a category}, to the category C.  This is just the
set of all diagrams in C shaped like the simplex n, as desired!

We can say all this even more slickly as follows: take


$$

                               i x 1                  hom
                 \Delta ^{op} x Cat ------> Cat^{op} x Cat -----> Set
$$
    
and dualize it to obtain


$$

                 Nerve: Cat \to  SimpSet.
$$
    
I should also point out that topologists usually do this stuff with
the topologist's version of \Delta , which does not include the "empty
simplex".  

\par\noindent\rule{\textwidth}{0.4pt}
<STRONG>K.</STRONG>  The Classifying Space of Category.  
If compose our new functor

                Nerve: Cat \to  SimpSet

with the "geometric realization" functor 

                | |: SimpSet \to  Top

defined in section E, we get a way to turn a category into a space, 
called its "classifying space".  This trick was first used by Graeme 
Segal, the homotopy theorist who later became the guru of conformal 
field theory.  He invented this trick in the following paper:

8) Graeme B. Segal, Classifying spaces and spectral sequences,
Publ. Math. Inst. des Haut. Etudes Scient. 34 (1968), 105-112.

As it turns out, every reasonable space is the classifying space of some
category!  More precisely, every space that's the geometric realization
of some simplicial set is homeomorphic to the classifying space of some
category.  To see this, suppose the space X is the geometric realization
of the simplicial set S.  Take the set of all simplices in S and
partially order them by saying x <= y if x is a face of y.  Here by
"face" I don't mean just mean a face of one dimension less than that of
y; I'm allowing faces of any dimension less than or equal to that of y.
We obtain a partially ordered set.  Now think of this as a category, C.
Then Nerve(C) is the "barycentric subdivision" of S.  In other words,
it's a new simplicial set formed by chopping up the simplices of S into
smaller pieces by putting a new vertex in the center of each one.  It
follows that the geometric realization of Nerve(C) is homeomorphic to
that of S.

There are lots of interesting special sorts of categories, like
groupoids, or monoids, or groups (see "<A HREF = "week74.html">week74</A>").  These give special
cases of the "classifying space" construction, some of which were
actually discovered before the general case.  I'll talk about some of
these more next week, since they are very important in topology.  

Also sometimes people take categories that they happen to be interested
in, which may have no obvious relation to topology, and study them by
studying their classifying spaces.  This gives surprising ways to apply
topology to all sorts of subjects.  A good example is "algebraic
K-theory", where we start with some sort of category of modules over a
ring.

\par\noindent\rule{\textwidth}{0.4pt}
<STRONG>L.</STRONG>  \Delta  as the Free Monoidal Category on a Monoid Object.  Recall that
a "monoid" is a set with a product and a unit element satisfying
associativity and the left and right unit laws.  Categorifying this
notion, we obtain the concept of a "monoidal category": a category C
with a product and a unit object satisfying the same laws.  A nice
example of a monoidal category is the category Set with its usual
cartesian product, or the category Vect with its usual tensor product.
We usually call the product in a monoidal category the "tensor product".

Now, the "microcosm principle" says that algebraic gadgets often like to
live inside categorified versions of themselves.  It's a bit like the
"homunculus theory", where I have a little copy of myself sitting in my
head who looks out through my eyes and thinks all my thoughts for me.
But unlike that theory, it's true!

For example, we can define a "monoid object" in any monoidal category.
Given a monoidal category A with tensor product x and unit object 1, we
define a monoid object a in A to be an object equipped with a "product"

m: a x a \to  a

and a "unit"

i: 1 \to  a

which satisfy associativity and the left and right unit laws (written
out as commutative diagrams).  A monoid object in Set is just a monoid,
but a monoid object in Vect is an algebra, and I gave some very
different examples of monoid objects in "<A HREF = "week89.html">week89</A>".

Now let's consider the "free monoidal category on a monoid object".  In
other words, consider a monoidal category A with a monoid object a in
it, and assume that A has no objects and no morphisms, and satisfies no
equations, other than those required by the definitions of "monoidal
category" and "monoid object".

Thus the only objects of A are the unit object together with a and its
tensor powers.  Similarly, all the morphism of A are built up by
composing and tensoring the morphisms m and i.  So A looks like this:


\begin{verbatim}

                          1x1xi
                        -------->
            1xi           1xix1
           ----->       -------->
    i       ix1           ix1x1
1 -----> a -----> a x a --------> a x a x a    ...
             m             mx1
           <-----       <--------
                           1xm
                        <--------
\end{verbatim}
    
Here I haven't drawn all the morphisms, just enough so that every
morphism in A is a composite of morphisms of this sort.  

What is this category?  It's just \Delta !  The nth tensor power of a
corresponds to the simplex with n vertices.  The morphisms going to
the right describe the ways the simplex with n vertices can be a face
of the simplex with n+1 vertices.  The morphisms going to the left
correspond to "degeneracies" - ways of squashing a simplex
with n+1 vertices down into one with n vertices.

So: in addition to its other descriptions, we can define \Delta  as the
free monoidal category on a monoid object!  Next time we'll see how
this is fundamental to homological algebra.






 \par\noindent\rule{\textwidth}{0.4pt}

% </A>
% </A>
% </A>
