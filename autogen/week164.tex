
% </A>
% </A>
% </A>
\week{January 13, 2001 }


What are the top ten questions for physics in this millennium?  
The participants of the conference Strings 2000 chose these:

<OL>
<LI>
Are all the (measurable) dimensionless parameters that characterize 
the physical universe calculable in principle or are some merely
determined by historical or quantum mechanical accident and uncalculable? 

<LI>How can quantum gravity help explain the origin of the universe? 

<LI>
What is the lifetime of the proton and how do we understand it? 

<LI> Is Nature supersymmetric, and if so, how is supersymmetry broken? 

<LI> Why does the universe appear to have one time and three space dimensions? 

<LI> Why does the cosmological constant have the value that it has, is it 
zero and is it really constant? 

<LI> What are the fundamental degrees of freedom of M-theory (the theory 
whose low-energy limit is eleven-dimensional supergravity and
which subsumes the five consistent superstring theories) and does 
the theory describe Nature? 

<LI> What is the resolution of the black hole information paradox? 

<LI> What physics explains the enormous disparity between the gravitational 
scale and the typical mass scale of the elementary particles? 

<LI> Can we quantitatively understand quark and gluon confinement in 
Quantum Chromodynamics and the existence of a mass gap? 

</OL>
For details see:

1) Physics problems for the next millennium, 
<A HREF = "http://feynman.physics.lsa.umich.edu/strings2000/millennium.html">
http://feynman.physics.lsa.umich.edu/strings2000/millennium.html</A>

I think most of these questions are pretty good if one limits physics 
to mean the search for new fundamental laws, rather than interesting 
applications of the laws we know.  I would leave out question 7, since
it's too concerned with a particular theory, rather than the physical
world itself.  I'd instead prefer to ask: "What physics underlies the
Standard Model gauge group SU(3) x SU(2) x U(1)?"

Of course, this business of limiting "physics" to mean
"the search for fundamental laws" annoys condensed matter
physicists like Philipp Anderson, since it excludes everything they work
on.  He writes:

\begin{quote}
     My colleagues in the fashionable fields of string theory and 
     quantum gravity advertise themselves as searching desperately 
     for the 'Theory of Everything", while their experimental colleagues 
     are gravid with the "God Particle", the marvelous Higgson which 
     is the somewhat misattributed source of all mass.  (They are also 
     after an understanding of the earliest few microseconds of the Big 
     Bang.) As Bill Clinton might remark, it depends on what the meaning 
     of "everything" is.  To these savants, "everything" means a list of 
     some two dozen numbers which are the parameters of the Standard Model. 
     This is a set of equations which already exists and does describe very 
     well what you and I would be willing to settle for as "everything". 
     This is why, following Bob Laughlin, I make the distinction between 
     "everything" and "every thing".  Every thing that you and I have 
     encountered in our real lives, or are likely to interact with in 
     the future, is no longer outside of the realm of a physics which is 
     transparent to us: relativity, special and general; electromagnetism; 
     the quantum theory of ordinary, usually condensed, matter; and, for
     a few remote phenomena, hopefully rare here on earth, our almost 
     equally cut-and-dried understanding of nuclear physics.  [Two 
     parenthetic remarks: 1) I don't mention statistical mechanics 
     only because it is a powerful technique, not a body of facts; 
     2) our colleagues have done only a sloppy job so far of deriving 
     nuclear physics from the Standard Model, but no one really doubts 
     that they can.]

     I am not arguing that the search for the meaning of those two 
     dozen parameters isn't exciting, interesting, and worthwhile: 
     yes, it's not boring to wonder why the electron is so much 
     lighter than the proton, or why the proton is stable at least 
     for another 35 powers of ten years, or whether quintessence exists. 
     But learning why can have no real effect on our lives, spiritually 
     inspiring as it would indeed be, even to a hardened old atheist 
     like myself.
\end{quote}
For the rest of his remarks, see:

2) What questions have disappeared?, The World Question Center,
<A HREF = "http://www.edge.org/documents/questions/q2001.html">http://www.edge.org/documents/questions/q2001.html</A>

Personally, I would be wary of asserting that a piece of knowledge
"can have no real effect on our lives" unless we are limiting
the discussion to short-term effects - not the next millennium.  But I
don't think physics should be construed to mean only the search for
"fundamental laws".  That neglects too much fun stuff!  It
would be nice to see the condensed matter theorists' list of problems
for the next millennium, for example.

On to something a bit more mathematical....

Careful readers of This Week's Finds will remember Diarmuid Crowley from
"<A HREF = "week151.html">week151</A>".  This week he visited
U. C. Riverside and talked about the topology of 7- and 15-dimensional
manifolds.  He also told me the following cool things.

You may recall from "<A HREF = "week163.html">week163</A>"
that the Poincare homology 3-sphere is a compact 3-manifold that has the
same homology groups as the ordinary 3-sphere, but is not homeomorphic
to the 3-sphere.  I explained how this marvelous space can be obtained
as the quotient of SU(2) = S^{3} by a 120-element subgroup - the double
cover of the symmetry group of the dodecahedron.  Even better, the
points in S^{3} which lie in this subgroup are the centers of the faces a
4d regular polytope with 120 dodecahedral faces.  That's pretty cool.
But here's another cool way to get the Poincare homology sphere:

E8 is the biggest of the exceptional Lie groups.  As I explained in
"<A HREF = "week64.html">week64</A>", the Dynkin diagram of
this group looks like this:



\begin{verbatim}

   o----o---o----o----o----o----o
                      |
                      |
                      o
\end{verbatim}
    
Now, make a model of this diagram by linking together 8 rings:


\begin{verbatim}

   /\   /\   /\   /\   /\   /\   /\ 
  /  \ /  \ /  \ /  \ /  \ /  \ /  \
 /    \    \    \    \    \    \    \
/    / \  / \  / \  / \  / \  / \    \ 
\    \ /  \ /  \ /  \ /  \ /  \ /    / 
 \    \    \    \    \ /\ \    \    /
  \  / \  / \  / \  / \  \ \  / \  /
   \/   \/   \/   \/ / \/ \ \/   \/
                    /      \ 
                    \      /
                     \    /
                      \  /
                       \/ 
\end{verbatim}
    
Imagine this model as living in S^{3}.  Next, hollow out all
these rings: actually delete the portion of space that lies inside them!
We now have a 3-manifold M whose boundary dM consists of 8 connected
components, each a torus.  Of course, a solid torus also has a torus as
its boundary.  So attach solid tori to each of these 8 components of dM,
but do it via this attaching map:

(x,y) \to  (y,-x+2y)

where x and y are the obvious coordinates on the torus, numbers between 
0 and 2\pi , and we do the arithmetic mod 2\pi .  We now have a new 3-manifold 
without boundary... and this is the Poincare homology sphere.

We see here a strange and indirect connection between E8 and the
dodecahedron.  This is not the only such connection!  There's also the
"McKay correspondence" (see "<A HREF =
"week65.html">week65</A>") and a way of getting the E8 root lattice
from the "icosians" (see "<A HREF =
"week20.html">week20</A>").

Are these three superficially different connections secretly just 
different views of the same grand picture?   I'm not sure.  I think
I'd know the answer to part of this puzzle if I better understood
the relation between ADE theory and singularities.  

But Diarmuid Crowley told me much more.   The Poincare homology sphere
is actually the boundary of a 4-manifold, and it's not hard to say what
this 4-manifold is.  I just gave you a recipe for cutting out 8 solid
tori from the 3-sphere and gluing them back in with a twist.  Suppose 
we think of 3-sphere as the boundary of the 4-disk D^{4}, and think of 
each solid torus as part of the boundary of a copy of D^{2} x D^{2}, using
the fact that

d(D^{2} x D^{2}) = S^{1} x D^{2} + D^{2} x S^{1}.

Then the same recipe can be seen as instructions for gluing 8 copies 
of D^{2} x D^{2} to the 4-ball along part of their boundary, getting a new 
4-manifold with boundary.  If you ponder it, you'll see that the 
boundary of this 4-manifold is the Poincare homology 3-sphere.

Now, this is actually no big deal, at least for folks who know some
3-dimensional topology.  But Crowley likes higher-dimensional topology,
and what he told me is this: the whole story generalizes to higher 
dimensions!  Instead of starting with this picture of linked 1-spheres 
in the 3-sphere:


\begin{verbatim}

   /\   /\   /\   /\   /\   /\   /\ 
  /  \ /  \ /  \ /  \ /  \ /  \ /  \
 /    \    \    \    \    \    \    \
/    / \  / \  / \  / \  / \  / \    \ 
\    \ /  \ /  \ /  \ /  \ /  \ /    / 
 \    \    \    \    \ /\ \    \    /
  \  / \  / \  / \  / \  \ \  / \  /
   \/   \/   \/   \/ / \/ \ \/   \/
                    /      \ 
                    \      /
                     \    /
                      \  /
                       \/ 
\end{verbatim}
    
start with an analogous pattern of 8 n-spheres linked in the (2n+1)-sphere.
Do all the same stuff, boosting the dimensions appropriately... and you'll
get an interesting (2n+1)-dimensional manifold dM which is the boundary of 
a (2n+2)-dimensional manifold M.  

When n is \emph{odd} and greater than 1, this manifold dM is actually an
"exotic sphere".  In other words, it's homeomorphic but not
diffeomorphic to the usual sphere of dimension 2n+1.

Now, exotic spheres of a given dimension form an abelian group G under
connected sum (see "<A HREF = "week141.html">week141</A>").
This group consists of two parts: the easy part and the hard part.  The
easy part is a normal subgroup N consisting of the exotic spheres that
bound parallelizable smooth manifolds.  The size of this subgroup can be
computed in terms of Bernoulli numbers and stuff like that.  The hard
part is the quotient group G/N.  This is usually the cokernel of a
famous gadget called the "J-homomorphism".  I say
"usually" because this is known to be true in most dimensions,
but in certain dimensions it remains an open question.

Anyway: the easy part N is always a finite cyclic group, and this is
\emph{generated} by the exotic sphere dM that I just described!   

For example: 

In dimension 7 we have G = N = Z/28, so there are 28 exotic spheres in
this dimension (up to orientation-preserving diffeomorphism), and they 
are all connected sums of the exotic 7-sphere dM formed by the above 
construction.   

In dimension 11 we have G = N = Z/992, so there are 992 exotic spheres,
and they are all connected sums of the exotic 11-sphere dM formed by the 
above construction.

In dimension 15 we no longer have G = N.  Instead we have N = Z/8128
and G = Z/8128 + Z/2.  There are thus 16256 exotic spheres in this
dimension, only half of which are connected sums of the exotic 15-sphere 
dM formed by the above construction.

And so on.  

While we're on the subject of exotic 15-spheres, I can't resist
mentioning this.  I explained in "<A HREF =
"week141.html">week141</A>" how to construct a bunch of exotic
7-spheres (24 of them, actually) using the quaternions.  Once you
understand this trick, it's natural to wonder if you can construct
exotic 15-spheres the same way, but using octonions instead of
quaternions.  Well, you can:

3) Nobuo Shimada, Differentiable structures on the 15-sphere and 
Pontrjagin classes of certain manifolds, Nagoya Math. Jour. (12) 1957, 59-69. 

I should also explain what I really like about the above stuff.  In
topological quantum field theory, people like to get 3-manifolds by
"surgery on framed links".  The idea is to start with a framed
link in the 3-sphere, use the framing to thicken each component to an
embedded solid torus, cut out these solid tori, and reattach them
"the other way", using the fact that S^{1} x
S^{1} is the boundary of both S^{1} x D^{2} and
D^{2} x S^{1}.  We can get any compact oriented
3-manifold this way.

The above construction of the Poincare homology sphere was just an
example of this, where the link was


\begin{verbatim}

   /\   /\   /\   /\   /\   /\   /\ 
  /  \ /  \ /  \ /  \ /  \ /  \ /  \
 /    \    \    \    \    \    \    \
/    / \  / \  / \  / \  / \  / \    \ 
\    \ /  \ /  \ /  \ /  \ /  \ /    / 
 \    \    \    \    \ /\ \    \    /
  \  / \  / \  / \  / \  \ \  / \  /
   \/   \/   \/   \/ / \/ \ \/   \/
                    /      \ 
                    \      /
                     \    /
                      \  /
                       \/ 
\end{verbatim}
    
and each component had two twists in the framing as we go around,
as compared to the standard "blackboard" framing.  This is why there 
was that mysterious number "2" in my formula for the attaching map.

Whenever we describe a 3-manifold using "surgery on framed
links" this way, there's an important matrix where the entry in the
ith row and jth column is the linking number of the ith component and
the jth component of our framed link, with the diagonal entries standing
for the "self-linking" numbers of the components, that is, the
number of twists their framings have.  This matrix is important because
it also describes the "intersection form" on the 2nd homology
group of a simply-connected 4-manifold M whose boundary dM is the
3-manifold we're describing.

For example, in the case of the Poincare homology sphere, this matrix 
is called the E8 Cartan matrix:


\begin{verbatim}

 2 -1  0  0  0  0  0  0
-1  2 -1  0  0  0  0  0 
 0 -1  2 -1  0  0  0  0 
 0  0 -1  2 -1  0  0  0 
 0  0  0 -1  2 -1  0 -1
 0  0  0  0 -1  2 -1  0
 0  0  0  0  0 -1  2  0 
 0  0  0  0 -1  0  0  2
\end{verbatim}
    
The Dynkin diagram simply summarizes this matrix in pictorial form.  
I already described the 4-manifold M whose boundary is the Poincare
homology sphere; now you know its intersection form. 

Anyway, what I find exciting is that all this stuff generalizes to
higher dimensions if we restrict attention to manifolds that have
trivial homotopy groups up to a certain point!  For example, it works
for compact oriented smooth 7-manifolds that have trivial \pi _{1}
and \pi _{2}.  Any such manifold can be obtained by doing surgery
on some framed 3-spheres embedded in S^{7}.  Just as 1-spheres
can link in 3d space since 1+1 is one less than 3, 3-spheres can link in
7d space since 3+3 is one less than 7.  We again get a matrix of linking
numbers.  As before, this matrix is also an intersection form: namely,
the intersection form on the 4th homology group of an 8-manifold M whose
boundary dM is the 7-manifold we're describing.  Moreover, this matrix
is symmetric in both the 3-manifold example and the 7-manifold example,
since it describes an intersection pairing on an
\emph{even-dimensional} homology group.

Even better, all the same stuff happens in manifolds with enough
trivial homotopy groups in dimension 11, and dimension 15... and 
all dimensions of the form 4n-1.  And what's \emph{really} neat is that 
these higher-dimensional generalizations are in some ways simpler 
than the 3d story.  The reason is that a 1-sphere can be knotted 
in 3-space in really complicated ways, but the higher-dimensional 
generalizations do not involve such complicated knotting.  The 
framing aspects can be more complicated, since there's more to 
framing an embedded sphere than just an integer, but it's not all
\emph{that} complicated. 

So maybe I can learn some more 3d topology by first warming up with
the simpler 7d case....

Finally, I'd like to list a few articles that I've been meaning to
read, but haven't gotten around to.  I hope to read them sometime
\emph{this} millennium!  I'll quote the abstracts and make a few comments.

4) Jack Morava, Cobordism of symplectic manifolds and asymptotic expansions,
a talk at the conference in honor of S.P. Novikov's 60th birthday, 
available as 
<A HREF = "http://xxx.lanl.gov/abs/math.SG/9908070">math.SG/9908070</A>.  

\begin{quote}
    The cobordism ring of symplectic manifolds defined by V.L. Ginzburg 
    is shown to be isomorphic to the Pontrjagin ring of complex-oriented 
    manifolds with free circle actions. This suggests an interpretation 
    of the formal group law of complex cobordism, in terms of a 
    composition-law on semiclassical expansions.  An appendix discusses 
    related questions about cobordism of toric varieties.
\end{quote}

I started trying to explain the relation between formal group laws and
complex oriented cohomology theories in "<A HREF =
"week150.html">week150</A>", because I'm quite puzzled about the
deep inner meaning of this relation.  This paper might be the key to
this mystery!

5) Detlev Buchholz, Current trends in axiomatic quantum field theory,
available as <A HREF =
"http://xxx.lanl.gov/abs/hep-th/9811233">hep-th/9811233</A>.

\begin{quote}
     In this article a non-technical survey is given of the present
     status of Axiomatic Quantum Field Theory and interesting future
     directions of this approach are outlined. The topics covered are
     the universal structure of the local algebras of observables, their
     relation to the underlying fields and the significance of their
     relative positions. Moreover, the physical interpretation of the
     theory is discussed with emphasis on problems appearing in gauge
     theories, such as the revision of the particle concept, the
     determination of symmetries and statistics from the superselection
     structure, the analysis of the short distance properties and the
     specific features of relativistic thermal states. Some problems
     appearing in quantum field theory on curved spacetimes are also
     briefly mentioned.

\end{quote}
I've been falling behind on new developments in axiomatic quantum field
theory.  Lots of cool stuff is happening, I hear.  This might help me
catch up.

6) Matt Visser, The reliability horizon, available as <A HREF = "http://xxx.lanl.gov/abs/gr-qc/9710020">gr-qc/9710020</A>.

\begin{quote}
     The "reliability horizon" for semi-classical quantum gravity
     quantifies the extent to which we should trust semi-classical
     quantum gravity, and gives a handle on just where the "Planck
     regime" resides. The key obstruction to pushing semi-classical
     quantum gravity into the Planck regime is often the existence of
     large metric fluctuations, rather than a large back-reaction.

\end{quote}
This seems like a very sensible enterprise: determining just where
semiclassical calculations are likely to break down, and quantum gravity
effects to become important.  Why haven't I read this?  It's obviously
worthwhile!

7) Bianca Letizia Cerchiai and Julius Wess, q-Deformed Minkowski Space
based on a q-Lorentz Algebra, available as
<A HREF = "http://xxx.lanl.gov/abs/math.QA/9801104">math.QA/9801104</A>.

\begin{quote}
     The Hilbert space representations of a non-commutative q-deformed
     Minkowski space, its momenta and its Lorentz boosts are
     constructed. The spectrum of the diagonalizable space elements
     shows a lattice-like structure with accumulation points on the
     light-cone.
\end{quote}
The q-deformed Lorentz algebra plays a role in quantum gravity with
nonzero cosmological constant, but it also shows up in noncommutative
geometry.  Are the two roles related?  I don't know! This is on my list
of puzzles to ponder.


The people applying the q-deformed Lorentz algebra to noncommutative
geometry want to develop the theory of q-deformed Minkowski space, see
if it makes the infinities in quantum field theory go away, and see what
physical predictions it makes.  It makes spacetime discrete in a very
pretty way; that I know from Julius Wess' talk in Schladming a few years
back (see "<a href = "week129.html">week129</a>").  But I should learn more about this, and not just
because Bianca Letizia Cerchiai is a very nice person who invited my
girlfriend and I to lunch at her parents' apartment in Milan.... oh, now
I'm feeling \emph{terribly} guilty for not reading her paper!  How
nasty of me!  I'd better print it out and read it as soon as I go into
the office!

In fact, now that I think of it, I've had at least \emph{some} dealings with
\emph{all} the authors of these papers.  And now I'm publicly admitting I
haven't read some of their most interesting papers!  Ugh!  At least
this admission may shame me into reading them now...

Bye.

\par\noindent\rule{\textwidth}{0.4pt}
On sci.physics.research, Aaron Bergman clarified something about
these millennial physics problems:


\begin{verbatim}

>John Baez wrote:

>> Aaron Bergman (abergman@Princeton.EDU) wrote:
 
>>> John Baez wrote:

>>>> Of course, this business of limiting "physics" to mean "the 
>>>> search for fundamental laws" annoys condensed matter 
>>>> physicists like Philipp Anderson, since it excludes everything 
>>>> they work on.  
 
>>> One should note that Gross explicitly says -- there's a Realaudio 
>>> of the talk online -- that this is a very narrowminded list that 
>>> excludes fundamental questions in other fields. It's not really 
>>> intended to be a universal list.
 
>> Good!   It's too bad the text of the webpage doesn't make that
>> clearer.   I'm appending your comment to the version of "week164" 
>> on my website, assuming you don't object.

>Sure. Or you can just refer them to the transparencies and the 
>talk. For those who don't want to bother listening to the whole 
>thing, start listening at about 7:30 mins into the RealAudio 
>stream:
>
><A HREF = "http://feynman.physics.lsa.umich.edu/cgi-bin/s2ktalk.cgi?questions">http://feynman.physics.lsa.umich.edu/cgi-bin/s2ktalk.cgi?questions</A>
>
>It's on transparency 4 which is why I mentioned,

>>> And Witten is coming back.
 
>> You mean he's not staying in LA?  Can't take the winters out here?

>I won't speculate on the reasons, but his grad students have said 
>that he's coming back to the Institute. 
>
>Aaron
>-- 
>Aaron Bergman
><http://www.princeton.edu/~abergman/>

\end{verbatim}
    




 \par\noindent\rule{\textwidth}{0.4pt}

% </A>
% </A>
% </A>
