<HTML>

% </A>
% </A>
% </A>
\week{April 20, 1996}


There are a number of interesting books I want to mention.

Huw Price's book on the arrow of time is finally out!  It's good to see
a philosopher of science who not only understands what modern physicists
are up to, but can occaisionally beat them at their own game.  

Why is the future different from the past?  This has been vexing people
for a long time, and the stakes went up considerably when Boltzmann
proved his "H-theorem", which seems at first to show that the entropy of
a gas always increases, despite the time-reversibility of the laws of
classical mechanics.  However, to prove the H-theorem he needed an
assumption, the "assumption of molecular chaos".  It says roughly that
the positions and velocities of the molecules in a gas are uncorrelated
before they collide.  This seems so plausible that one can easily 
overlook that it has a time-asymmetry built into it --- visible in the
word "before".  In fact, we aren't getting something for nothing in the
H-theorem; we are making a time-asymmetric assumption in order to
conclude that entropy increases with time!

The "independence of incoming causes" is very intuitive: if we do an
experiment on an electron, we almost always assume our choice of how to
set the dials is not correlated to the state of the electron.  If we
drop this time-asymmetric assumption, the world looks rather
different... but I'll let Price explain that to you.

Anyway, Price is an expert at spotting covertly time-asymmetric
assumptions.  You may remember from "<A HREF = "week26.html">week26</A>" that he even got into a
nice argument with Stephen Hawking about the arrow of time, thanks to
this habit of his.  You can read more about it in:

1) Huw Price, Time's Arrow and Archimedes' Point: New Directions for a
Physics of Time, Oxford University Press, 1996.  

Also, there is a new book out by Hawking and Roger Penrose on quantum
gravity.  First they each present their own ideas, and then they duke it
out in a debate in the final chapter.  This book is an excellent place
to get an overview of some of the main ideas in quantum gravity.  It
helps if you have a little familiarity with general relativity, or
differential geometry, or are willing to fake it.  

There is even some stuff here about the arrow of time!  Hawking has a
theory of how it arose, starting from his marvelous "no-boundary
boundary conditions", which say that the wavefunction of the universe is
full of quantum fluctuations corresponding to big bangs which erupt and
then recollapse in big crunches.  The wavefunction itself has no obvious
"time-asymmetry", indeed, time as we know it only makes sense
\emph{within} any one of the quantum fluctuations, one of which is presumably
the world we 
know!  But Hawking thinks that each of these quantum fluctuations, or at
least most of them, should have an arrow of time.  This is what Price
raised some objections to.  Hawking seems to argue that each quantum
fluctuation should "start out" rather smooth near its big bang and
develop more inhomogeneities as time passes, "winding up"
quite wrinkly near its big crunch.  But it's not at all clear what this
"starting out" 
and "winding up" means.  Possibly he is simply speaking
vaguely, and all 
or most of the quantum fluctuations can be shown to have one smooth end
and wrinkly at the other.  That would be an adequate resolution to the
arrow of time problem.  But it's not clear, at least not to me, that
Hawking really showed this.

Penrose, on the other hand, has some closely related ideas.  His "Weyl
curvature hypothesis" says that the Weyl curvature of spacetime goes to
zero at initial singularities (e.g. the big bang) and infinity at final
ones (e.g. black holes).  The Weyl curvature can be regarded as a
measure of the presence of inhomogeneity --- the "wrinkliness"
I alluded to above.  The Weyl curvature hypothesis can be regarded as a
time-asymmetric law built into physics from the very start.

To see them argue it out, read

2) Stephen Hawking and Roger Penrose, The Nature of Space and Time,
Princeton University Press, 1996.


There are also a couple of more technical books on general relativity
that I'd been meaning to get ahold of for a long time.  They both
feature authors of that famous book,

3) Charles Misner, Kip Thorne and John Wheeler, Gravitation, Freeman
Press, 1973,

which was actually the book that made me decide to work on quantum
gravity, back at the end of my undergraduate days.  They are:

4) Ignazio Ciufolini and John Archibald Wheeler, Gravitation and
Inertia, Princeton University Press, 1995.

and 

5) Kip Thorne, Richard Price and Douglas Macdonald, eds., Black Holes:
The Membrane Paradigm, 1986.

The book by Ciufolini and Wheeler is full of interesting stuff, but it
concentrates on "gravitomagnetism": the tendency, predicted by general
relativity, for a massive spinning body to apply a torque to nearby
objects.  This is related to Mach's old idea that just as spinning a
bucket pulls the water in it up to the edges, thanks to the centrifugal
force, the same thing should happen if instead we make lots of stars
rotate around the bucket!  Einstein's theory of general relativity was
inspired by Mach, but there has been a long-running debate over whether
general relativity is "truly Machian" --- in part because nobody knows
what "truly Machian" means.  In any event, Ciufolini and Wheeler argue
that gravitomagnetism exhibits the Machian nature of general relativity,
and they give a very nice tour of gravitomagnetic effects.

That is fine in theory.  However, the gravitomagnetic effect has never
yet been observed!   It was supposed to be tested by Gravity Probe B, a
satellite flying at an altitude of about 650 kilometers, containing a
superconducting gyroscope that should precess at a rate of 42
milliarcseconds per year thanks to gravitomagnetism.  I don't know what
ever happened with this, though: the following web page says "Gravity
Probe B is expected to fly in 1995", but now it's 1996, right?  Maybe
someone can clue me in to the latest news.... I seem to remember some
arguments about funding the program.

6) Gravity Probe B, http://stugyro.stanford.edu/RELATIVITY/GPB/
(Note added in 2002: now this webpage is gone; see
<A HREF = "http://einstein.stanford.edu/">http://einstein.stanford.edu/</A>
for the latest story.) 

Kip Thorne's name comes up a lot in conjuction with black holes and the
LIGO --- or Laser-Interferometer Gravitational-Wave Observatory --- project.
As pairs of black holes or neutron stars spiral emit gravitational
radiation, they should spiral in towards each other.  In their final
moments, as they merge, they should emit a "chirp" of gravitational
radiation, increasing in frequency and amplitude until their ecstatic
union is complete.  The LIGO project aims to observe these chirps, and
any other sufficiently strong gravitational radiation that happens to be
passing by our way.  LIGO aims to do this by using laser interferometry
to measure the distance between two points about 4 kilometers apart to
an accuracy of about 10^{-18} meters, thus detecting tiny ripples in the
spaceteim metric.  For more on LIGO, try

7) LIGO project home page, <A HREF = "http://www.ligo.caltech.edu/">http://www.ligo.caltech.edu/</A>

Thorne helped develop a nice way to think of black holes by envisioning
their event horizon as a kind of "membrane" with well-defined
mechanical, electrical and magnetic properties.  This is called the
membrane paradigm, and is useful for calculations and understanding what
black holes are really like.  The book "Black Holes: The Membrane
Paradigm" is a good place to learn about this.


% <A NAME = "tale">
Now let me return to the tale of 2-categories.  So far I've said only
that a 2-category is some sort of structure with objects, morphisms
between objects, and 2-morphisms between morphisms.  But I have been
attempting to develop your intuition for Cat, the primordial example of
a 2-category.  Remember, Cat is the 2-category of all categories!  Its
objects are categories, its morphisms are functors, and its 2-morphisms
are natural transformations --- these being defined in "<A HREF = "week73.html">week73</A>" and
again in "<A HREF = "week75.html">week75</A>".  

How can you learn more about 2-categories?  Well, a really good place is
the following article by Ross Street, who is one of the great gurus of
n-category theory.  For example, he was the one who invented
\omega -categories!

8) Ross Street, Categorical structures, in Handbook of Algebra, vol. 1,
ed. M. Hazewinkel, Elsevier, 1996.  

Physicists should note his explanation of the Yang-Baxter and
Zamolodchikov equations in terms of category theory.  If you have
trouble finding this, you might try

9) G. Maxwell Kelly and Ross Street, Review of the elements of 2-categories,
Springer Lecture Notes in Mathematics 420, Berlin, 1974, pp. 75-103.

I can't really compete with these for thoroughness, but at least let me
give the definition of a 2-category.  I'll give a pretty nuts-and-bolts
definition; later I'll give a more elegant and abstract one.  Readers
who are familiar with Cat should keep this example in mind at all times!

This definition is sort of long, so if you get tired of it, concentrate
on the pictures!  They convey the basic idea.  Also, keep in mind is
that this is going to be sort of like the definition of a category, but
with an extra level on top, the 2-morphisms.

So: first of all, a 2-category consists of a collection of "objects" and a
collection of "morphisms".  Every morphism f has a "source" object and a
"target" object.  If the source of f is X and its target is Y, we write
f: X \to  Y.  In addition, we have:

1)  Given a morphism f: X \to  Y and a morphism g: Y \to  Z, there
is a morphism fg: X \to  Z, which we call the "composite" of f and g.

2)  Composition is associative:  (fg)h = f(gh).

3)  For each object X there is a morphism 1_{X}: X \to  X, called the
"identity" of X.  For any f: X \to  Y we have 1_{X} f = f 1_{Y} = f.

You should visualize the composite of f: X \to  Y and g: Y \to  Z as
follows:


\begin{verbatim}

                     f           g
               X ---->---- Y ---->---- Z

\end{verbatim}
    
So far this is exactly the definition of a category!  But a 2-category
ALSO consists of a collection of "2-morphisms".  Every 2-morphism T has
a "source" morphism f and a target morphism g.  If the source of T is f
and its target is g, we write T: f => g.  If T: f => g, we require that
f and g have the same source and the same target; for example, f: x \to  y
and g: x \to  y.  You should visualize T as follows:


\begin{verbatim}

                      f
                  ---->---
                 /         \
                x     T     y
                 \         /
                  ---->----
                      g

\end{verbatim}
    
People usually draw a double arrow like => going down next to the T, but
I can't do that here.  

In addition, we have:

1') Given a 2-morphism S: f => g and a 2-morphism T: g => h, there is a
2-morphism ST: f => h, which we call the "vertical composite" of S and
T.

2')  Vertical composition is associative:  (ST)U = S(TU).


3') For each morphism f there is a 2-morphism 1_{f}: f => f,
called the "identity" of f.  For any T: f => g we have
1_{f} T = T 1_{g} = T.

Note that these are just like the previous 3 rules.  We draw the vertical
composite of S: f => g and T: g => h like this:


\begin{verbatim}

                          f
                      ---->----
                     /  S      \
                    /     g     \
                   x ----->----- y
                    \   T       /
                     \         /
                      ---->----
                          h

\end{verbatim}
    
Now for a twist.  We also require that we can "horizontally" compose
2-morphisms as follows:

\begin{verbatim}

 
                      f           f'
                  ---->---    ---->---
                 /         \ /         \
                x     S     y     T     z
                 \         / \         /
                  ---->----   ---->----
                      g           g'

\end{verbatim}
    
So we also demand:
1'') Given morphisms f,g: x \to  y and f',g': y \to  z, and 2-morphisms
S: f => g and T: f' => g', there is a 2-morphism S.T: ff' => gg', which we
call the "horizontal composite" of S and T.
2'') Horizontal composition is associative:  (S.T).U = S.(T.U).
3'') The identities for vertical composition are also the identities for
horizontal composition.  That is, given f,g: x \to  y and T: f => g we


% parser failed at source line 342
