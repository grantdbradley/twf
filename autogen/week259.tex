
% </A>
% </A>
% </A>
\week{December 9, 2007 }

This week I'll talk about the "field with one element" - even though 
it doesn't exist.  It's a mathematical phantom.

But first: the Egg Nebula.

In "<a href = "week257.html">week257</A>" and "<a href
= "week258.html">week258</A>" I talked about interstellar dust.
As I mentioned, lots of it comes from "asymptotic giant
branch" stars - stars like our Sun, but later in their life, when
they're big, red, pulsing, and puffing out elements like hydrogen,
helium, carbon, nitrogen, and oxygen.

The pulsations grow wilder and wilder until the star blows off its 
entire outer atmosphere, forming a big cloud of gas and dust 
misleadingly called a "planetary nebula".  It leaves behind 
its dense inner core as a hot white dwarf.  Intense radiation from
this core eventually heats the gas and dust until they glow.

Back in "<a href = "week223.html">week223</A>" I showed my favorite 
example of a planetary nebula: the Cat's Eye.   

<div align = "center">
<a href = "http://heritage.stsci.edu/2004/27/index.html">
<img src = "cats-eye_nebula.jpg">
% </a>
</div>

And I quoted the astronomer Bruce Balick 
on what will happen here when our Sun becomes a planetary nebula 
6.9 billion years from now:

\begin{quote}
     Here on Earth, we'll feel the wind of the ejected gases 
     sweeping past, slowly at first (a mere 5 miles per second!), 
     and then picking up speed as the spasms continue - eventually 
     to reach 1000 miles per second!!  The remnant Sun will rise as 
     a dot of intense light, no larger than Venus, more brilliant 
     than 100 present Suns, and an intensely hot blue-white color 
     hotter than any welder's torch.  Light from the fiendish blue 
     "pinprick" will braise the Earth and tear apart its surface 
     molecules and atoms. A new but very thin "atmosphere" of free 
     electrons will form as the Earth's surface turns to dust.
\end{quote}
Eerie!  

Here's a "protoplanetary nebula" - that is, a planetary nebula that's
just getting started: 

<div align = "center">
<img width = "600" src = "egg_nebula.jpg">
</div>

1) Rainbow image of a dusty star, NASA, 
<a href = "http://hubblesite.org/newscenter/archive/releases/nebula/planetary/2003/09/">http://hubblesite.org/newscenter/archive/releases/nebula/planetary/2003/09/</a>

It's called the "Egg Nebula".  You can see layers of dust coming out
puff after puff, shooting outwards at about 20 kilometers per second,
stretching out for about a third of a light year.  The colors - red,
green and blue - aren't anything you'd actually see.  They're just  
an easy way to depict three different polarizations of light.  I 
don't know why the light is polarized that way.

You can see a dark disk of thicker dust running around 
the star.  It could be an "accretion disk" spiralling into the star.  The 
"beams" shining out left and right are still poorly understood.  
Maybe they're jets of matter ejected from the north and south 
poles of the disk?  This idea seem more plausible when you look at 
this photo taken by NICMOS, which is Hubble's "Near Infrared Camera 
and Multi-Object Spectrometer":

<div align = "center">
<img width = "500" src = "egg_nebula_infrared.jpg">
</div>


2) Raghvendra Sahai, Egg Nebula in polarized light, Hubble Heritage
Project, <a href = "http://heritage.stsci.edu/2003/09/supplemental.html">http://heritage.stsci.edu/2003/09/supplemental.html</a>

This near-infrared image also shows a bright spot called "Peak
A" about 500 AU from the central star.  An "AU", or
astronomical unit, is the distance from the Sun to the Earth.

Nobody knows what this bright spot is.  Some argue that it's just a 
clump of dust reflecting light from the main star.  Others advocate a 
more exciting theory: it's a white dwarf orbiting the main star, which 
exploded in a "thermonuclear burst" after accreting a bunch of dust.

3) Joel H. Kastner and Noam Soker, The Egg Nebula (AFGL 2688): deepening 
enigma, to appear in Asymmetrical Planetary Nebulae III, eds. M. Meixner, 
J. Kastner, N. Soker, and B. Balick, ASP Conference Series.  Available 
as <a href = "http://arXiv.org/abs/astro-ph/0309677">arXiv:astro-ph/0309677</A>. 

I hope to say more about planetary nebulae in future Weeks, mainly because
they're so beautiful.

But now: the field with one element!

A <a href = "http://en.wikipedia.org/wiki/Field_%28mathematics%29">field</a> 
is a mathematical structure where you can add, multiply,
subtract and divide in ways that satisfy these familiar rules:


\begin{verbatim}

 x + y = y + x        (x + y) + z = x + (y + z)   x + 0 = x

 xy = yx              (xy)z = x(yz)               1x = x

 x(y + z) = xy + xz 

 every element x has an element -x with x + (-x) = 0
 
 every element x that's not 0 has an element 1/x with x (1/x) = 1
\end{verbatim}
    

You'll note that the last clause is the odd man out.  Addition,
subtraction and multiplication can all be described as everywhere
defined operations.  Division cannot, since we can't divide by 0.
This is the funny thing about fields, which is what causes the 
problem we'll run into.

Everyone who has studied math knows three examples of fields: 
the rational numbers Q, the real numbers R and the complex numbers C.  
There are a lot more, too - for example, function fields, number
fields, and finite fields.  

Let me say a tiny bit about these three kinds of fields.

The simplest sort of "<a href =
"http://en.wikipedia.org/wiki/Function_field_of_an_algebraic_variety">function
field</a>" consists of rational functions of one variable - that
is, ratios of polynomials, like this:

(2z^{3} + z + 1)/(z^{2} - 7)

Here the coefficients of your polynomials should lie in some field F
you already know about.  The resulting field is called F(z).  If F is
the complex numbers, we can think of F(z) = C(z) as consisting of
functions on the Riemann sphere.  In "<a href =
"week201.html">week201</A>", I explained how the symmetries of
this field form a group important in special relativity: the Lorentz
group!

It's also very interesting to study the field of functions on a
surface fancier than the sphere, but still defined by algebraic 
equations, like the surface of a doughnut or n-holed doughnut.  
Number theorists and  algebraic geometers spend a lot of time 
thinking about these fields, which are called "function fields 
of complex curves".  

For example, different ways of describing the surface of a doughnut by
algebraic equations give different "<a href =
"http://en.wikipedia.org/wiki/Elliptic_curve">elliptic
curves</a>".  This is terminology is bound
to puzzle beginners!  They're called "curves" even though
it's 2-dimensional, because it takes one \emph{complex} number to
say where you are on a little patch of a surface, just as it takes one
\emph{real} number to say where you are on an ordinary curve like a
circle.  That's the origin of the term "complex curve".
And, they're called "elliptic" because they first showed up
when people were studying elliptic integrals, which are
generalizations of trig functions from circles to ellipses.

I explained more about elliptic curves in "<a href = "week13.html">week13</A>" and "<a href = "week125.html">week125</A>".  
Lurking behind this, there's a lot of fascinating stuff about 
function fields of elliptic curves.

The simplest sort of "<a href =
"http://en.wikipedia.org/wiki/Algebraic_number_field">number
field</a>" comes from taking the rational numbers and throwing in
the solutions of a polynomial equation.  For example, in
"<a href = "week20.html">week20</A>" I talked about the
"golden field", which consists of all numbers of the form

a + b \sqrt  5

where a and b are rational.  

One of the most beautiful ideas in math is the analogy between 
number fields and function fields - the idea that numbers are like
functions on some sort of "space".  I began explaining this in
"<a href = "week205.html">week205</A>", "<a href = "week216.html">week216</A>" and "<a href = "week218.html">week218</A>", but there's much more to say
about what's known... and also many things that remain mysterious.  

In particular, it's pretty well understood how number fields resemble
function fields of complex curves, and how this relates number theory
to \emph{2-dimensional} topology.  But, there are also many
analogies between number theory and \emph{3-dimensional} topology,
which I began listing in "<a href =
"week257.html">week257</A>".  It seems these analogies are doomed
to remain mysterious until we get a handle on the field with one
element.  But more on that later.

The simplest sort of "<a href =
"http://en.wikipedia.org/wiki/Finite_field">finite field</a>"
comes from choosing a prime number p and taking the integers modulo p.
The result is sometimes called Z/p, especially when you're just
concerned with addition.  But when you think of it as a field, it's
better to call it F_{p}.

The reason is that there's a finite field of size q whenever q is a
\emph{power} of a prime, and this field is unique - so it's called
F_{q}.  You build F_{q} sort of like how you build the
complex numbers starting from the real numbers, or number fields
starting from the rational numbers.  Namely, to construct
F_{p^{n}}, you take F_{p} and throw in the
roots of a well-chosen polynomial of degree n: one that doesn't have
any roots in F_{p}, but "wants" to have n different
roots.

Okay: that was a tiny bit about function fields, number fields and finite
fields.  But now I need to point out some slight lies I told!  

I said there was a finite field with q elements whenever q was a prime
power.  You might think this should include q = 1, since 1 is the
\emph{zeroth} power of \emph{any} prime.  

So, is there a field with one element?  

If so, it must have 1 = 0.  That doesn't violate the definition of
a field that I gave you... does it?  The definition said any element 
that's not 0 has a reciprocal.  In this particular example, 0 also has
a reciprocal, since we can set 1/0 = 1 and not get into any contradictions.  
But that's not a problem: in usual math practice, saying "we can divide 
by anything that's not zero" doesn't deny the possibility that we can
divide by 0.

Unfortunately, allowing a field with 1 = 0 causes nothing but grief.
For example, we can define vector spaces using any field (people say
"over" any field), and there's a nice theorem saying two
vector spaces are isomorphic if and only if they have the same
dimension.  And normally, there's one vector space of each dimension.
But the last part isn't true for a field with 1 = 0.  In a vector
space over such a field, every vector v has

v = 1 v = 0 v = 0

So, every vector space is 0-dimensional!  

To prevent such problems, people add one extra clause to the definition
of a field:


\begin{verbatim}

 1 is not equal to 0
\end{verbatim}
    
This clause looks even more tacked-on and silly than the clause
saying everything \emph{nonzero} has a reciprocal... but it works fairly
well.

However, the field with one element still wants to exist!  Not the
silly field with 1 = 0, but something else, something more mysterious...
something that Gavin Wraith calls a "mathematical phantom":
 
4) Gavin Wraith, Mathematical phantoms,
<a href = "http://www.wra1th.plus.com/gcw/rants/math/MathPhant.html">http://www.wra1th.plus.com/gcw/rants/math/MathPhant.html</a>

What's a mathematical phantom?  According to Wraith, it's an object 
that doesn't exist within a given mathematical framework, but 
nonetheless "obtrudes its effects so convincingly that one is forced 
to concede a broader notion of existence".

Like a genie that talks its way out of a bottle, a sufficiently powerful 
mathematical phantom can talk us into letting it exist by promising to
work wonders for us.  Great examples include the number zero, irrational
numbers, negative numbers, imaginary numbers, and quaternions.  At one 
point all these were considered highly dubious entities.  Now they're
widely accepted.  They "exist".  Someday the field with one element 
will exist too!

Why?

I gave a lot of reasons in "<a href = "week183.html">week183</A>", "<a href = "week184.html">week184</A>", "<a href = "week185.html">week185</A>", "<a href = "week186.html">week186</A>" 
and "<a href = "week187.html">week187</A>", but let me rapidly summarize.  

It's all about "q-deformation".  In physics, people talk
about q-deformation when they're taking groups and turning them into
"quantum groups".  But it has a closely related aspect
that's in some ways more fundamental.  When we count things involving
n-dimensional vector spaces over the finite field F_{q}, we
often get answers that are polynomials in q.  If we then set q = 1,
the resulting formulas count analogous things involving n-element
sets!

So, finite sets want to be finite-dimensional vector spaces over the 
(nonexistent) field with one element... or something like that.  We can
be more precise after looking at some examples.

Here's the simplest example.  Say we count lines through the origin 
in an n-dimensional vector space over F_{q}.   We get the "q-integer"


$$

 q^{n} - 1
------- = 1 + q + q^{2} + ... + q^{n-1}
 q - 1
$$
    

which I'll write as [n] for short.  

Setting q = 1, we get n.  This is the number of points in an n-element
set.  Sure, that sounds silly.  But, I'm trying to make a point here!
At q = 1, stuff about n-dimensional vector spaces over F_{q}
reduces to stuff about n-element sets, and the q-integer [n] reduces
to the ordinary integer n.

This may not be the best way to understand the pattern, though.
Lines through the origin in an n-dimensional vector space are the 
same as points in an (n-1)-dimensional projective space.  So, the
real analogy may be between "points in a projective space" and 
"points in a set".

Here's a more impressive example.   Pick any uncombed Young diagram D 
with n boxes.  Here's one with 8 boxes:


\begin{verbatim}

X                     1 box in the first row
X X                   3 boxes in the first two rows
X X X                 6 boxes in the first three rows
X X                   8 boxes in the first four rows
\end{verbatim}
    

Then, count the "D-flags on an n-dimensional vector space over
F_{q}".  In our example, such a D-flag is:


$$

    a 1-dimensional subspace
 of a 3-dimensional subspace
 of a 6-dimensional subspace
 of a 8-dimensional vector space over F_{q}
$$
    
If you actually count these D-flags you'll get some formula, which is 
a polynomial in q.  And when you set q = 1, you'll get the number of 
"D-flags on an n-element set".  In our example, such a D-flag is:


\begin{verbatim}

   a 1-element subset 
of a 3-element subset
of a 6-element subset
of a 8-element set
\end{verbatim}
    
For details, and a proof that this really works, try:

5) John Baez, Lecture 4 in the Geometric Representation Theory
Seminar, October 9, 2007.  Available at 
<a href = "http://math.ucr.edu/home/baez/qg-fall2007/qg-fall2007.html#f07_4">http://math.ucr.edu/home/baez/qg-fall2007/qg-fall2007.html#f07_4</a>

These examples can be generalized.  In "<a href =
"week187.html">week187</A>" I showed how to get one example for
each subset of the dots in any Dynkin diagram!  This idea goes back to
Jacques Tits, who was the first to suggest that there should be a
field with one element.  Dynkin diagrams give algebraic groups over
F_{q}... but he noticed that these groups reduce to
"Coxeter groups" as q \to  1.  And, if you mark some dots
on a Dynkin diagram you get a "flag variety" on which your
algebraic group acts... but as q \to  1, this reduces to a finite set
on which your Coxeter group acts.

If you don't understand the previous paragraph, don't worry - it's
over now.  It's great stuff, but my main point is that there seems to 
be an analogy like this:


$$

 q = 1                      q = a power of a prime number    

 n-element set              (n-1)-dimensional projective space over F_{q}
 integer n                  q-integer [n]
 permutation groups S_{n}      projective special linear group PSL(n,F_{q})
 factorial n!               q-factorial [n]!
$$
    

This opens up lots of questions.  For example, if projective spaces over
F_{1} are just finite sets, what should \emph{vector spaces} 
over F_{1} be?  

People have thought about this, and the answer seems to be
"pointed sets" - sets with a distinguished point, which you
can think of as the "origin".  A pointed set with n+1
elements seems to act like an n-dimensional vector space over
F_{q}.

For more clues, and an attempt to do algebraic geometry using the 
field with one element, try this:

6) Christophe Soul&eacute;, On the field with one element, Talk given at the 
Arbeitstagung, Bonn, June 1999, IHES preprint available at
<a href = " http://www.ihes.fr/~soule/f1-soule.pdf">http://www.ihes.fr/~soule/f1-soule.pdf</a>

Soul&eacute; tries to define "algebraic varieties" over
F_{1}, namely curves and their higher-dimensional
generalization.  And, he talks a lot about zeta functions for such
varieties.  He goes into more detail here:

7) Christophe Soul&eacute;, Les varietes sur le corps a un element, Moscow
Math. Jour. 4 (2004), 217-244, 312.

The theme of zeta functions - see "<a href =
"week216.html">week216</A>" - is deeply involved in this
business.  For more, try these papers:

8) N. Kurokawa, Zeta functions over F_{1}, Proc. Japan Acad. Ser. A
Math. Sci. 81 (2006), 180-184.

9) Anton Deitmar, Remarks on zeta functions and K-theory over F_{1},
available as <a href =
"http://arXiv.org/abs/arXiv:math/0605429">arXiv:math/0605429</A>.
 
But instead of talking about zeta functions, I'd like to talk about
two approaches to giving a formal definition of the field with one
element.  Both of them involve taking the concept of field and 
modifying it so it doesn't necessary involve the operation of addition.
The first one, due to Deitmar, simply throws out addition!  The second,
due to Nikolai Durov, allows for a wide choice of operations - and thus
a wide supply of "exotic fields".

For Deitmar's approach, try these:

10) Anton Deitmar, Schemes over F_{1}, available as <a href = "http://arXiv.org/abs/arXiv:math/0404185">arXiv:math/0404185</A>.

F_{1}-schemes and toric varieties, available as <a href = "http://arXiv.org/abs/arXiv:math/0608179">arXiv:math/0608179</A>.

The usual approach to fields treats fields as specially nice
commutative rings.  A "<a href =
"http://en.wikipedia.org/wiki/Commutative_ring">commutative
ring</a>" is a gadget where you can add and multiply, and these
rules hold:


\begin{verbatim}

 x + y = y + x        (x + y) + z = x + (y + z)   x + 0 = x

 xy = yx              (xy)z = x(yz)               1x = x

 x(y + z) = xy + xz 

 every element x has an element -x with x + (-x) = 0
\end{verbatim}
    

Deitmar throws out addition and treats fields as specially nice
commutative monoids.  A commutative "<a href =
"http://en.wikipedia.org/wiki/Commutative_ring">monoid</a>" is a gadget where you can multiply, and these rules
hold:


\begin{verbatim}

 xy = yx              (xy)z = x(yz)               1x = x
\end{verbatim}
    

For Deitmar, the field with one element, F_{1}, is just the
commutative monoid with one element, namely 1.  A "vector space
over F_{1}" is just a set on which this monoid acts via
multiplication... but that amounts to just a plain old set.  The
"dimension" of such a "vector space" is just its
cardinality.

All this so far is quite trivial, but Deitmar makes a nice attempt at 
redoing algebraic geometry to include this field with one element.  
One reason to do this is to understand the mysterious 3-dimensional
aspect of number theory.  

To explain this, I need to say a bit about "<a href =
"http://en.wikipedia.org/wiki/Scheme_%28mathematics%29">schemes</a>".
In ordinary algebraic geometry, we turn commutative rings into spaces
to think about them geometrically.  I explained this back in "<a
href = "week199.html">week199</A>" and "<a href =
"week205.html">week205</A>", but let me review quickly, and go
further:

We can think of elements of a commutative ring R as functions on
certain space called the "<a href =
"http://en.wikipedia.org/wiki/Spectrum_of_a_ring">spectrum</a>"
of R, Spec(R).  This space has a topology, so we can also talk about
functions that are defined, not on all of Spec(R), but just
\emph{part} of Spec(R) - namely some open set.  Indeed, for each
open set U in Spec(R), there's a commutative ring O(U) consisting of
those functions defined on U.  These commutative rings are related in
nice ways:

<OL>
<LI>
If the open set V is smaller than U, we can restrict functions from 
U to V, getting a ring homomorphism O(U) \to  U(V)
</LI>
<LI>If U is covered by a bunch of open sets U_{i}, and we have
a function f_{i} in each O(U_{i}), such that
f_{i} and f_{j} agree when restricted to the set
U_{i} \cap  U_{j}, then there's a unique function f in
O(U) that restricts to each of these functions f_{i}.  </LI>
</OL>

Something satisfying condition 1 is called a "<a href =
"http://en.wikipedia.org/wiki/Presheaf#Definition_of_a_presheaf">presheaf</a>"
of commutative rings; something also satisfying condition 2 is called
a "<a href =
"http://en.wikipedia.org/wiki/Sheaf_%28mathematics%29">sheaf</a>"
of commutative rings.

So, Spec(R) is not just a topological space, it's equipped with a
sheaf of commutative rings.  People call this a "<a href =
"http://en.wikipedia.org/wiki/Ringed_space">ringed space</A>".

Whenever we have a ringed space, we can ask if it comes from a
commutative ring R in the way I just sketched.  If so, we call it an
"affine scheme".  Affine schemes are just a fancy
geometrical way of talking about commutative rings!

More interestingly, whenever we have a ringed space, we can ask if
it's \emph{locally} isomorphic to one coming from a commutative ring.  
In other words: does every point have a neighborhood that, as a
ringed space, looks like Spec(R) for some commutative ring R?
Or in other words: is our ringed space \emph{locally} isomorphic to an
affine scheme?  If so, we call it a "scheme".  

A classic example of a scheme that's not an affine scheme is the
Riemann sphere.  There aren't any rational functions defined on the
whole Riemann sphere, except for constants - the rest all blow up 
somewhere.  So, it's hopeless trying to think of the Riemann sphere
as an affine scheme.

But, for any open set U in the Riemann sphere there's a commutative
ring O(U) consisting of rational functions that are defined on U.  So,
the Riemann sphere becomes a ringed space.  And, it's \emph{locally}
isomorphic to the complex plane, which is the affine scheme
corresponding to the commutative ring of complex polynomials in one
variable.  So, the Riemann sphere is a scheme!

For more on schemes, try this nice introduction, which actually
has lots of pictures:

11) David Eisenbud and Joe Harris, 
The Geometry of Schemes, Springer, Berlin, 2000.

Now, we can talk about schemes "over a field F", meaning
that each commutative ring O(U) is also a vector space over F, in a
well-behaved way, giving us a "sheaf of commutative rings over
F".  For example, the Riemann sphere is a scheme over C.

There's a secret 3-dimensional aspect to the affine scheme Spec(Z),
where Z is the commutative ring of integers.  As explained in the
Addenda to "<a href = "week257.html">week257</A>", we might
understand this if we could see Spec(Z) as a scheme over the field
with one element!  For more, see this:

12) M. Kapranov and A. Smirnov, Cohomology determinants and 
reciprocity laws: number field case, available at 
<a href = "http://wwwhomes.uni-bielefeld.de/triepe/F1.html">http://wwwhomes.uni-bielefeld.de/triepe/F1.html</a>

So, we really need a theory of schemes over the field with one
element.  The problem is, F_{1} isn't really a field.  In
Deitmar's approach, it's just a commutative monoid.

So, let me sketch how Deitmar gets around this.  In a nutshell, he takes
advantage of the fact that a lot of basic algebraic geometry only requires 
multiplication, not addition!

He starts by defining a "commutative ring over
F_{1}" to be simply a commutative monoid.  The simplest
example is F_{1} itself.

Now, watch how he gets away with never using addition:

He defines an "ideal" in a commutative monoid R to be a
subset I for which the product of something in I with anything in R
again lies in I.  He says an ideal P is "prime" if whenever
a product of two elements in R is in P, at least one of them is in P.

He defines the "spectrum" Spec(R) of a commutative monoid R
to be the set of its prime ideals.  He gives this the "Zariski
topology".  That's the topology where the closed sets are the
whole space, or any set of prime ideals that contain a given ideal.

He then shows how to get a sheaf of commutative monoids on Spec(R).
He defines a "scheme" to be a space equipped with a sheaf of 
commutative monoids that's \emph{locally} isomorphic to one of this sort. 

If you know algebraic geometry, these definitions should seem very
familiar.  And if you don't, you can just replace the word "monoid"
by "ring" everywhere in the previous three paragraphs, and you'll get
the standard definitions in algebraic geometry!

Deitmar shows how to build a scheme over F_{1} called the
"projective line".  The projective line over C is just the
Riemann sphere.  The projective line over F_{1} has just two
points (or more precisely, two closed points).  This is good, because
the projective line over the field with q elements has

[2] = 1 + q

points, and we're doing the q = 1 case.  

Deitmar's construction seems like a lot of work to get ahold of the 
2-element set, if that's all it secretly is.  But, I need to think 
about this more.  After all, he doesn't just get a space; he gets a 
sheaf of commutative monoids on this space!  And what's that like?
I should work it out.

Deitmar also shows how to relate schemes over F_{1} to the usual sort
of schemes.

From a commutative ring, we can always get a commutative monoid just
by forgetting the addition.  This process has a kind of reverse, too.
Namely, from a commutative monoid, we can get a commutative ring 
simply by taking formal integral linear combinations of elements.  
Using this, Deitmar shows how we can turn ordinary schemes into 
schemes over F_{1}... and conversely.  He says an ordinary scheme is 
"defined over F_{1}" if it arises in this way from a scheme over F_{1}.  

Okay, that's a taste of Deitmar's approach.  For Durov's approach, 
try this mammoth 568-page paper:

13) Nikolai Durov, New approach to Arakelov geometry, available as
<a href = "http://arxiv.org/abs/0704.2030">arXiv:0704.2030</a>.

or read our discussions of it at the \emph{n}-Category Caf&eacute;, 
starting here:

14) David Corfield, The field with one element, 
<a href = "http://golem.ph.utexas.edu/category/2007/04/the_field_with_one_element.html">http://golem.ph.utexas.edu/category/2007/04/the_field_with_one_element.html</a>

Durov defines a "generalized ring" to be what Lawvere much
earlier called an "algebraic theory".  What is it?  Nothing
scary!  It's just a gadget with a bunch of abstract n-ary operations
closed under composition, permutation, duplication and deletion of
arguments, and equipped with an identity operation.

So, for example, if our gadget has a binary operation 

(x,y) |\to  f(x,y)

we can compose this with itself to get the ternary operation

(x,y,z) |\to  f(f(x,y),z)

and the 4-ary operation

(w,x,y,z) |\to  f(w,f(x,f(y,z)))

and so on.  We can then permute arguments in our 4-ary operation
to get one like this:

(w,x,y,z) |\to  f(z,f(x,f(w,y)))

or duplicate some arguments to get a binary operation like this

(x,y) |\to  f(x,f(x,f(y,y)))

From this we can then form a 3-ary operation by deleting an argument,
for example like this:

(x,y,z) |\to  f(x,f(x,f(y,y)))

If you know about 
"<a href = "http://en.wikipedia.org/wiki/Operad_theory">operads</a>",
this kind of gadget is just a specially 
nice operad where we can duplicate and delete operations.  
 
Now, a generalized ring is said to be "commutative" if all the 
operations commute in a certain sense.  (I'll let you guess what
it means for an n-ary operation to commute with an m-ary operation.)
We get an example of a commutative generalized ring from a commutative 
ring R if we let the n-ary operations be "n-ary R-linear 
combinations", like this:

(x_{1}, ..., x_{n}) |\to  r_{1} x_{1} + 
... + r_{n} x_{n}           

We also get a very similar example from any commutative "<a href =
"http://en.wikipedia.org/wiki/Semiring">rig</a>",
which is a gizmo satisfying rules like those of a commutative ring,
but without negatives:


\begin{verbatim}

 x + y = y + x        (x + y) + z = x + (y + z)   x + 0 = x

 xy = yx              (xy)z = x(yz)               1x = x

 x(y + z) = xy + xz 
\end{verbatim}
    

And, we get an example from any commutative monoid, where we only
have 1-ary operations, coming from multiplication by elements of
our monoid:

(x_{1}) |\to  r x_{1}

So, Durov's framework generalizes Deitmar's!  But, it includes a lot
more examples: exotic hothouse flowers like the "tropical rig", the
"real integers", and more.  He develops a theory of schemes for all
these generalized rings, and builds it "up to construction of algebraic 
K-theory, intersection theory and Chern classes" - fancy things that
algebraic geometers like.

What I don't yet see is how either Deitmar's or Durov's approach
helps us understand the secret 3-dimensional nature of Spec(Z).  
I may just need to read their papers more carefully and think about
them more.

Finally, here's yet another approach to the field with one element:

15) Bertrand Toen and M. Vaquie, Under Spec(Z), available as 
<a href = "http://arxiv.org/abs/math/0509684">arXiv:math/0509684</a>.

16) Shai Haran, Non-additive geometry, Composito Mathematica 143
(2007), 613-638.  

Toen describes interesting relations between algebra over F_{1} and
stable homotopy theory.  Haran even suggests that the Riemann
Hypothesis could be proved if we understood enough about the geometry
of schemes over F_{1}!  This is fascinating... I don't understand it,
but I want to.

In short, a mathematical phantom is gradually taking solid form before our
very eyes!   In the process, a grand generalization of algebraic geometry
is emerging, which enriches it to include some previously scorned 
entities: rigs, monoids and the like.  And, this enrichment 
holds the promise of shedding light on some otherwise impenetrable
mysteries: for example, the deep inner meaning of 
q-deformation, and the 3-dimensional nature of Spec(Z).


\par\noindent\rule{\textwidth}{0.4pt}
\textbf{Addenda}: I thank Thomas Riepe and David Corfield for drawing my
attention to the paper by Shai Haran.  Thomas Riepe also recommends
the following online introduction to schemes:

17) Marc Levine, Summer course in motivic homotopy theory, available at 
<a href = "http://www.math.neu.edu/~levine/publ/SummerSchoolAG.pdf">
http://www.math.neu.edu/~levine/publ/SummerSchoolAG.pdf</a>

Kevin Buzzard has a word of advice about the "generic point":

\begin{quote}

\begin{verbatim}
  We can think of elements of a commutative ring R as functions on
  certain space called the "spectrum" of R, Spec(R).
\end{verbatim}
    

So this is the set of all \emph{prime} ideals of R, right? Not just
the maximal ones? So...

\begin{verbatim}
  So, the Riemann sphere is a scheme!
\end{verbatim}
    

Well, you have to throw in a mystical extra "generic point" if you really
want to make it a scheme :-) Corresponding to the zero ideal. My
impression is that most non-algebraic geometers think that the generic
point is either confusing or just plain daft. But believe me, it's a
\emph{really} good idea! For decades in the literature in algebraic geometry
people were using the word "generic" to mean "something that was true most
of the time" - in fact a "generic point" is probably another really good
example of a phantom! For example a meromorphic function on the Riemann
sphere that wasn't zero would be "generically non-zero" to people like
Borel and Weil, and if you asked them for a definition they would say
that it just meant something like "the zero locus in the space had a
smaller dimension than the whole space" or "the zero locus was
nowhere dense in any component of the space" or something, and of course
people could even make rigorous definitions that worked in particular
cases and so on, but then Grothendieck came along with his "generic
point", corresponding to the zero ideal [note to sub: check to see whether the
idea was in the literature pre-Grothendieck!] and suddenly a function that
was "generically non-zero" was just a function which was non-zero on the
generic point! Such a cool way of doing it :-)

Kevin
\end{quote}

If you get stuck on my puzzle "what does it mean for
an n-ary operation to commute with an m-ary operation?", let me just
show you what it means for a binary operation f to commute with a 
ternary operation g.  It means:

g(f(x_{1},x_{2}), 
f(x_{3},x_{4}),
f(x_{5},x_{6})) = 
f(g(x_{1},x_{2},x_{3}),
g(x_{4},x_{5},x_{6}))

I hope this example gives away the general pattern.

If this is confusing, look
at the case where we start with a ring R and take as our
n-ary operations the "n-ary R-linear combinations"

(x_{1}, ..., x_{n}) |\to  r_{1} x_{1} +
... + r_{n} x_{n}

with r_{i} in R.  Here an example of a binary operation is addition:

(x_{1}, x_{2}) |\to  x_{1} + x_{2}

while every unary operation is multiplication by some element of R:

x_{1} |\to  r x_{1}

To say "addition commutes with multiplication by an element of R"
means that

r(x_{1} + x_{2}) = rx_{1} + rx_{2}

This is just the distributive law so it holds for any ring R.

But, for the unary operations to commute with each other, we need R
to be commutative, since this says:

r(s x_{1}) = s (r x_{1})

(In the calculations I just did, we can either think of the x_{i}
as elements of a specific R-module, or more abstractly as "dummy variables"
used to describe the ring R as a generalized ring in Durov's sense -
what Lawvere calls an algebraic theory.)

For more discussion, go to the
<a href = "http://golem.ph.utexas.edu/category/2007/12/this_weeks_finds_in_mathematic_19.html">\emph{n}-Category Caf&eacute;</a>.



\par\noindent\rule{\textwidth}{0.4pt}
<em>The analogy between number fields and function fields finds a 
basic limitation with the lack of a ground field.  One says that Spec(Z)
(with a point at infinity added, as is familiar in Arakelov geometry) is 
like a (complete) curve, but over which field?</em> - Christophe Soul&eacute;

\par\noindent\rule{\textwidth}{0.4pt}

% </A>
% </A>
% </A>
