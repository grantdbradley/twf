
% </A>
% </A>
% </A>
\week{July 23, 1997 }

Well, it seems I want to talk one more time about octonions before
moving on to other stuff.  I'm a bit afraid this obsession with
octonions will mislead the nonexperts, fooling them into thinking
octonions are more central to mainstream mathematical physics than they
actually are.  I'm also worried that the experts will think I'm spend
all my time brooding about octonions when I should be working on
practical stuff like quantum gravity.  But darn it, this is summer
vacation!  The only way I'm going to keep on cranking out "This Week's
Finds" is if I write about whatever I feel like, no matter how
frivolous.  So here goes.  

First of all, let's make sure everyone here knows what projective  space
is.  If you don't, I'd better explain it.  This is honest mainstream
stuff that everyone should know, good nutritious  mathematics, so I
won't need to feel too guilty about serving the extravagant octonionic
dessert which follows.  

Start with R^{n}, good old n-dimensional Euclidean space.  We
can imagine wanting to "compactify" this so that if you go
sailing off to infinity in some direction you'll come sailing back from
the other side like Magellan.  There are different ways to do this.  A
well-known one is to take R^{n} and add on one extra "point
at infinity", obtaining the n-dimensional sphere S^{n}.
Here the idea is that start anywhere in R^{n} and start sailing
in any direction, you are sailing towards this "point at
infinity".

But there is a sneakier way to compactify R^{n}, which gives us not the
n-dimensional sphere but "projective n-space".  Here we add on a lot of
points, one for each line through the origin.  Now there are \emph{lots} of
points at infinity, one for every direction!  The idea here is that if
you start at the origin and start sailing along any straight line, you
are sailing towards the point at infinity corresponding to that line.
Sailing along any parallel line takes you twoards the same point at
infinity.  It's a bit like a perspective picture where different
families of parallel lines converge to different points on the horizon
- the points on the horizon being points at infinity.  

Projective n-space is also called RP^{n}.   The R is for "real", since
this is actually "real projective n-space".  Later we'll see what
happens if we replace the real numbers by the complex numbers,
quaternions, or octonions.

There are some other ways to think about RP^{n} that are useful either
for visualizing it or doing calculations.  First a nice way to visualize
it.  First take R^{n} and squash it down so it's just the ball of radius 1,
or more precisely, the "open ball" consisting of all vectors of length
less than 1.  We can do this using a coordinate transformation like:

x |\to  x' = x/(1 + |x|^{2})^{\frac{1}{2} }

Here x stands for a vector in R^{n} and |x| is its length.  Dividing the
vector x by (1 + |x|^{2})^{\frac{1}{2} } 
gives us a vector x' whose length never
quite gets to 1, though it can get as close at it likes.  So we have
squashed R^{n} down to the open ball of radius 1.  

Now say you start at the origin in this squashed version of R^{n} and 
sail off in any direction in a straight line.  Then you are secretly
heading towards the boundary of the open ball.   So the points an the
boundary of the open ball are like "points at infinity".  

We can now compactify R^{n} by including these points at infinity. In
other words, we can work not with the open ball but with the "closed
ball" consisting of all vectors x' whose length is less than or equal
to 1.

However, to get projective n-space we also have to decree that antipodal
points x' and -x' with |x'| = 1 are to be regarded as the same.   In
other words, we need to "identify each point on the boundary of the
closed ball with its antipodal point".  The reason is that we said that
when you sail off to infinity along a particular straight line, you are
approaching a particular point in projective n-space.  Implicit in this 
is that it doesn't matter which \emph{way} you sail along that straight line.
Either direction takes you towards the same point in projective n-space!

This may seem weird: in this world, when the cowboy says "he went
thataway" and points at a particular point on the horizon, you gotta
remember that his finger points both ways, and the villian could equally
well have gone in the opposite direction.   The reason this is good is
that it makes projective space into a kind of geometer's paradise: any
two lines in projective space intersect in a \emph{single} point.  No more
annoying exceptions: even "parallel" lines intersect in a single point,
which just happens to be a point at infinity.   This simplifies life
enormously.

Okay, so RP^{n} is the space formed by taking a closed n-dimensional ball
and identifying pairs of antipodal points on its boundary.

A more abstract way to think of RP^{n}, which is incredibly useful in
computations, is as the set of all lines through  the origin in R^{n+1}.
Why is this the same thing?  Well, let me illustrate it in an example.
What's the space of lines through the origin in R^{3}?  To keep track of
these lines, draw a sphere around the origin.  Each line through the
origin intersects this sphere in two points.  Either one point is in the
northern hemisphere and the other is in the southern hemisphere, or 
both are on the equator.  So we can keep track of all our lines using
points on the northern hemisphere and the equator, but identifying
antipodal points on the equator.  This is just the same as taking the
closed 2-dimensional ball and identifying antipodal points on the
boundary!  QED.  The same argument works in higher dimensions too.

Now that we know a point in RP^{n} is just a line through the origin in
R^{n+1}, it's easy to put coordinates on RP^{n}.  
There's one line through
the origin passing through any point in R^{n+1}, but if we 
multiply the
coordinates (x_{1},...,x_{n+1}) of this point by any 
nonzero number we 
get the same line.  Thus we can use a list of n+1 real numbers 
to describe a point in RP^{n}, with the proviso that we get the same
point in RP^{n} if someone comes along and multiplies them all by 
some nonzero number!  These are called "homogeneous coordinates".   

If you don't like the ambiguity of homogeneous coordinates, you can go
right ahead and divide all the coordinates by the real number x_{1},
getting

(1, x_{2}/x_{1}, ... , x_{n+1}/x_{1}) 

which lets us describe a point in RP^{n} by n real numbers, as befits an
n-dimensional real manifold.  Of course, this won't work if x_{1}
happens to be zero!  But we can divide by x_{2} if x_{2} is nonzero, and
so on.  \emph{One} of them has to be nonzero, so we can cover RP^{n} with n+1
different coordinate patches corresponding to the regions where
different x_{i}'s are nonzero.  It's easy to change coordinates, too.

This makes everything very algebraic, which makes it easy to
generalize RP^{n} by replacing the real numbers with other number
systems.  For example, to define "complex projective n-space" or CP^{n},
just replace the word "real" by the word "complex" in the last two
paragraphs, and replace "R" by "C".  CP^{n} is even more of a geometer's
paradise than RP^{n}, because when you work with complex numbers you can
solve all polynomial equations.  Also, now there's no big difference
between an ellipse and a hyperbola!  This sort of thing is why CP^{n} is
so widely used as a context for "algebraic geometry".

We can go even further and replace the real numbers by the
quaternions, H, defining the "quaternionic projective n-space" HP^{n}.
If we are careful about writing things in the right order, it's no
problem that the quaternions are noncommutative... we can still divide
by any nonzero quaternion, so we can cover HP^{n} with n+1 different
coordinate charts and freely change coordinates as desired.

We can try to go even further and use the octonions, O.  Can we define
"octonionic projective n-space", OP^{n}?  Well, now things get tricky!
Remember, the octonions are nonassociative.  There's no problem
defining OP^{1}; we can cover it with two coordinate charts,
corresponding to homogeneous coordinates of the form

(x, 1) 

and 

(1, y),

and we can change coordinates back and forth with no problem.  This
amounts to taking O and adding a single point at infinity, getting the
8-dimensional sphere S^{8}.  This is part of a pattern:
RP^{1} = S^{1}
CP^{1} = S^{2}
HP^{1} = S^{4}
OP^{1} = S^{8}
I discussed the implications of this pattern for Bott periodicity in 
"<A HREF = "week105.html">week105</A>".  

We can also define OP^{2}.  Here we have 3 coordinate charts
corresponding to homogeneous coordinates of the form

(1, y, z), 

(x, 1, z), 

and 

(x, y, 1). 

We can change back and forth between coordinate systems, but now we
have to \emph{check} that if we start with the first coordinate system,
change to the second coordinate system, and then change back to the
first, we wind up where we started!  This is not obvious, since
multiplication is not associative.  But it works, thanks to a couple
of identities that are not automatic in the nonassociative context,
but hold for the octonions:

(xy)^{-1} = y^{-1} x^{-1}

and 

(xy)y^{-1} = x.

Checking these equations is a good exercise for anyone who wants to
understand the octonions.

Now for the cool part: OP^{2} is where it ends!  

We can't define OP^{n} for n greater than 2, because the
nonassociativity keeps us from being able to change coordinates a
bunch of times and get back where we started!  You might hope that we
could weasel out of this somehow, but it seems that there is a real
sense in which the higher-dimensional octonionic projective spaces
don't exist.

So we have a fascinating situation: an infinite tower of RP^{n}'s, an
infinite tower of CP^{n}'s, an infinite tower of HP^{n}'s, but an abortive
tower of OP^{n}'s going only up to n = 2 and then fizzling out.  This
means that while all sorts of geometry and group theory relating to
the reals, complexes and quaternions fits into infinite systematic
patterns, the geometry and group theory relating to the octonions is
quirky and mysterious.

We often associate mathematics with "classical" beauty, patterns
continuing ad infinitum with the ineluctable logic of a composition by
some divine Bach.  But when we study OP^{2} and its implications, we see
that mathematics also has room for "exceptional" beauty, beauty that
flares into being and quickly subsides into silence like a piece by
Webern.  Are the fundamental laws of physics based on "classical"
mathematics or "exceptional" mathematics?  Since our universe seems
unique and special - don't ask me how would we know if it weren't
- Witten has suggested the latter.  Indeed, it crops up a lot in
string theory.  This is why I'm trying to learn about the octonions: a
lot of exceptional objects in mathematics are tied to them.

I already discussed this a bit in "<A HREF = "week64.html">week64</A>", where I sketched how there
are 3 infinite sequences of "classical" simple Lie groups corresponding
to rotations in R^{n}, C^{n}, and H^{n}, and 5 "exceptional" simple Lie groups
related to the octonions.   After studying it all a bit more, I can now
go into some more detail.  

In order of increasing dimension, the 5 exceptional Lie groups are
called G2, F4, E6, E7, and E8.  The smallest, G2, is easy to
understand in terms of the octonions: it's just the group of
symmetries of the octonions as an algebra.  It's a marvelous fact that
all the bigger ones are related to OP^{2}.  This was discovered by
people like Freudenthal and Tits and Vinberg, but a great place to
read about it is the following fascinating book:

1) Boris Rosenfeld, Geometry of Lie Groups, Kluwer Academic Publishers,
1997.

The space OP^{2} has a natural metric on it, which allows us to measure
distances between points.  This allows us to define a certain symmetry
group OP^{2}, the group of all its "isometries", which are
transformations preserving the metric.  This symmetry group is F4!

However, there is another bigger symmetry group of OP^{2}.
As in real projective n-space, the notion of a "line" makes
sense in OP^{2}.  One has to be careful: these are octonionic
"lines", which have 8 real dimensions.  Nonetheless, this lets
us define the group of all "collineations" of OP^{2},
that is, transformations that take lines to lines.  This symmetry group
is E6!  (Technically speaking, this is a "noncompact real
form" of E6; the rest of the time I'll be talking about compact
real forms.)

To get up to E7 and E8, we need to take a different viewpoint, which
also gives us another way to get E6.  The key here is that the tensor
product of two algebras is an algebra, so we can tensor the octonions
with R, C, H, or O and get various algebras:

<UL>
<LI>
The algebra (R \otimes  O) is just the octonions.  

<LI>
The algebra (C \otimes  O) is called the "bioctonions".  

<LI>
The algebra (H \otimes  O) is called the "quateroctonions". 

<LI>
The algebra (O \otimes  O) is called the "octooctonions".
</UL>

I'm not making this up: it's all in Rosenfeld's book!  The poet Lisa
Raphals suggested calling the octooctonions the "high-octane
octonions", which I sort of like.  But compared to Rosenfeld, I'm a
model of restraint: I won't even mention the dyoctonions, duoctonions,
split octonions, semioctonions, split semioctonions, 1/4-octonions or
1/8-octonions - for the definitions of these, you'll have to read his
book.

Apparently one can define projective planes for all of these algebras,
and all these projective planes have natural metrics on them, all of
them same general form.  So each of these projective planes has a group
of isometries.  And, lo and behold:

<UL>
<LI>
The group of isometries of the octonionic projective plane is F4.

<LI>
The group of isometries of the bioctonionic projective plane is E6.

<LI>
The group of isometries of the quateroctonionic projective plane is E7.

<LI>
The group of isometries of the octooctonionic projective plane is E8.
</UL>
Now I still don't understand this as well as I'd like to - I'm not
sure how to define projective planes for all these algebras (though I
have my guesses), and Rosenfeld is unfortunately a tad reticent on
this issue.  But it looks like a cool way to systematize the study of
the expectional groups!  That's what I want: a systematic theory of
exceptions.

I want to say a bit more about the above, but first let me note that
there are lots of other ways of thinking about the exceptional groups.
A great source of information about them is the following posthumously
published book by the great topologist Adams:

2) John Frank Adams, Lectures on Exceptional Lie Groups, eds. Zafer
Mahmud and Mamoru Mimura, University of Chicago Press, Chicago, 1996.

He has a bit about octonionic constructions of G2 and F4, but mostly he
concentrates on constructions of the exceptional groups using classical
groups and spinors.  

In "<A HREF = "week90.html">week90</A>" I explained Kostant's constructions of F4 and E8 using
spinors in 8 dimensions and triality - which, as noted in "<A HREF = "week61.html">week61</A>",
is just another way of talking about the octonions.  Unfortunately I
don't yet see quite how this relates to the above stuff, nor do I see
how to get E6 and E7 in a beautiful way using Kostant's setup. 

There's also a neat construction of E8 using spinors in 16 dimensions!
Adams gives a nice explanation of this, and it's also discussed in the
classic tome on string theory:

3) Michael B. Green, John H. Schwarz, and Edward Witten, Superstring
Theory, two volumes, Cambridge U. Press, Cambridge, 1987.

The idea here is to take the direct sum of the Lie algebra so(16) and
its 16-dimensional left-handed spinor representation S+ to get the
Lie algebra of E8.  The bracket of two guys in so(16) is defined as
usual, the bracket of a guy in so(16) and a guy in S+ is defined to
be the result of acting on the latter by the former, and the bracket
of two guys in S+ is defined to be a guy in S+ by dualizing the 
map 

so(16) \otimes  S+ \to  S+ 

to get a map 

S+ \otimes  S+ \to  so(16).

This is a complete description of the Lie algebra of E8!  

Anyway, there are lots of different ways of thinking about exceptional
groups, and a challenge for the octonionic approach is to systematize
all these ways.

Now I want to wrap up by saying a bit about how the exceptional Jordan
algebra fits into the above story.  Jordan algebras were invented as a
way to study the self-adjoint operators on a Hilbert space, which
represent observables in quantum mechanics.  If you multiply two
self-adjoint operators A and B the result needn't be self-adjoint, but
the "Jordan product"

A o B = (AB + BA)/2  

is self-adjoint.  This suggests seeing what identities the Jordan
product satisfies, cooking up a general notion of "Jordan algebra",
seeing how much quantum mechanics you can do with an arbitrary Jordan
algebra of observables, and classifying Jordan algebras if possible.

We can define a "projection" in a Jordan algebra to be an element A
with A o A = A.  If our Jordan algebra consists of self-adjoint
operators on the complex Hilbert space C^{n}, a projection is a
self-adjoint operator whose only eigenvalues are zero and one.
Physically speaking, this corresponds to a "yes-or-no question" about
our quantum system.  Geometrically speaking, such an operator is a
projection onto some subspace of our Hilbert space.  All this stuff also
works if we start with the real Hilbert space R^{n} or the quaternionic
Hilbert space H^{n}.  


 In these special cases, one can define a "minimal
projection" to be a projection on a 1-dimensional subspace of our
Hilbert space.  Physically, minimal projections correspond to "pure
states" - states of affairs in which the answer to some 
maximally informative question is "yes", like "is
the z component of the angular momentum of this spin-1/2 particle
equal to 1/2?"  Geometrically, the space of minimal projections
is just the space of "lines" in our Hilbert space.  This is
either RP^{n-1}, or CP^{n-1}, or HP^{n-1},
depending on whether we're working with the reals, complexes or
quaternions.  So: the space of pure states of this sort of quantum
system is also a projective space!  The relation between quantum theory
and "projective geometry" has been intensively explored for
many years.  You can read about it in:

4) V. S. Varadarajan, Geometry of Quantum Theory, Springer-Verlag,
Berlin, 2nd ed., 1985.

Most people do quantum mechanics with complex Hilbert spaces.  Real
Hilbert spaces are apparently too boring, but some people have 
considered the quaternionic case:

5) Stephen L. Adler, Quaternionic Quantum Mechanics and Quantum Fields, 
Oxford U. Press, Oxford, 1995.

If our Hilbert space is the complex Hilbert space C^{n}, its group of
symmetries is usually thought of as U(n) - the group of n\times n unitary
matrices.  This group also acts as symmetries on the Jordan algebra of
self-adjoint n\times n complex matrices, and also on the space CP^{n-1}.

Similarly, if we start with R^{n}, we get the group of orthogonal n\times n
matrices O(n), which acts on the Jordan algebra of real self-adjoint
n\times n matrices and on RP^{n-1}.  

Likewise, if we start with H^{n}, we get
the group Sp(n), which acts on the Jordan algebra of quaternionic
self-adjoint n\times n matrices and on HP^{n-1}.  

This pretty much explains
how the classical groups are related to different flavors of quantum
mechanics.

Now what about the octonions?  Well, here we can only go up to n = 3,
basically for the reasons explained before: the same stuff that keeps
us from defining octonionic projective spaces past a certain point
keeps us from getting Jordan algebras!  The interesting case is the
Jordan algebra of 3\times 3 self-adjoint octonionic matrices.  This is
called the "exceptional Jordan algebra", J.  The group of symmetries
of this is - you guessed it, F4.  One can also define a "minimal
projection" in J and the space of these is OP^{2}.

Is it possible that octonionic quantum mechanics plays some role in
physics?

I don't know.

Anyway, here is my hunch about the bioctonionic, quateroctonionic, and
octooctonionic projective planes.  I think to define them you should
probably tensor the exceptional Jordan algebra with C, H, and O,
respectively, and take the space of minimal projections in the
resulting algebra.  Rosenfeld seems to suggest this is the way to go.
However, I'm vague about some important details, and it bugs me,
because the special identities I needed above to define OP^{2} are
related to O being an alternative algebra, but C \otimes  O, H \otimes  O
and O \otimes  O are not alternative.

I should add that in addition to octonionic projective geometry, one
can do octonionic hyperbolic geometry.  One can read about this in
Rosenfeld and also in the following:

6) Daniel Allcock, Reflection groups on the octave hyperbolic plane,
University of Utah Mathematics Department preprint.


\par\noindent\rule{\textwidth}{0.4pt}

<strong>Addenda:</strong> Here's an email from David Broadhurst,
followed by various remarks.

\begin{quote}
John:

Shortly before his death I spent a charming afternoon with Paul Dirac.
Contrary to his reputation, he was most forthcoming.

Among many things, I recall this: Dirac explained that while trained
as an engineer and known as a physicist, his aesthetics were mathematical.
He said (as I can best recall, nearly 20 years on): At a young age,
I fell in love with projective geometry.
I always wanted to use to use it in physics, but never found a place for it.

Then someone told him that the difference between complex and quaternionic
QM had been characterized as the failure of theorem in
classical projective geometry.

Dirac's face beamed a lovely smile: Ah he said, it was just such a thing
that I hoped to do.

I was reminded of this when bactracking to your &quot;week106&quot;, today.

Best, <br>
David
\end{quote}
    

The theorem that fails for quaternions but holds for R and C is
the "Pappus theorem", discussed in 
"<A HREF = "week145.html">week145</A>".

Next, a bit about OP^{n}.
There are different senses in which we can't define OP^{n} 
for n greater than 2.  One is that if we try to define coordinates
on OP^{n} in a similar way to how we did it for OP^{2}, 
nonassociativity 
keeps us from being able to change coordinates a bunch of times and 
get back where we started!  It's definitely enlightening to see how 
the desired transition functions g_{ij} fail to satisfy the 
necessary
cocycle condition g_{ij} g_{jk} = g_{ik} 
when we get up to OP^{3}, which 
would require 4 charts.

But, a deeper way to think about this emerged in conversations 
I've had with James Dolan.  Stasheff invented a notion of 
"A_{\infty } space", which is a pointed topological space with 
a product that is associative up to homotopy which satisfies the 
pentagon identity up to... etc.  Any A_{\infty } space G has a 
classifying space BG such that 

\Omega (BG) ~ G. 

In other words, 
BG is a pointed space such that the space of loops based at this 
point is homotopy equivalent to G.  One can form this space BG by 
the Milnor construction: sticking in one 0-simplex, one 1-simplex 
for every point of G, one 2-simplex for every triple (g,h,k) with 
gh = k, one 3-simplex for every associator, and so on.  If we do 
this where G is the group of length-one elements of R (i.e. Z/2) 
we get RP^{\infty }, as we expect, since 

RP^{\infty } = B(Z/2).  

Even better, at the nth
stage of the Milnor construction we get a space homeomorphic to 
RP^{n}.  Similarly, if we do this where G is the group of 
length-one elements of C 
or H we get CP^{\infty } or HP^{\infty }.  
But if we take G to be
the units of O, which has a product but is not even homotopy-associative, 
we get OP^{1} = S^{7} at the first 
step, OP^{2} at the
second step, ... but there's no way to perform the third step!

Next: here's a little more information on the octonionic, bioctonionic,
quateroctonionic and octooctonionic projective planes.  
Rosenfeld
claims that the groups of isometries of these planes are F4, E6, E7,
and E8, respectively.  The problem is, I can't quite understand
how he constructs these spaces, except for the plain octonionic
case.  

It appears that these spaces can also be constructed using the 
ideas in Adams' book.  Here's how it goes.   

<UL>
<LI>
The Lie algebra F4 has a subalgebra of maximal rank isomorphic to
so(9).  The quotient space is 16-dimensional - twice the dimension
of the octonions.  It follows that the Lie group F4 mod the subgroup
generated by this subalgebra is a 16-dimensional Riemannian manifold
on which F4 acts by isometries.

<LI>
The Lie algebra E6 has a subalgebra of maximal rank isomorphic to
so(10) \oplus  u(1).  The quotient space is 32-dimensional - twice the
dimension of the bioctonions.  It follows that the Lie group E6 mod
the subgroup generated by this subalgebra is a 32-dimensional
Riemannian manifold on which E6 acts by isometries.
  
<LI>
The Lie algebra E7 has a subalgebra of maximal rank isomorphic to
so(12) \oplus  su(2).  The quotient space is 64-dimensional - twice the
dimension of the quateroctonions.  It follows that the Lie group E6
mod the subgroup generated by this subalgebra is a 64-dimensional
Riemannian manifold on which E7 acts by isometries.

<LI>
The Lie algebra E8 has a subalgebra of maximal rank isomorphic to so(16).  
The quotient space is 128-dimensional - twice the dimension of
the octooctonions.  It follows that the Lie group E6 mod the 
subgroup generated by this subalgebra is a 128-dimensional Riemannian
manifold on which E8 acts by isometries.  
</UL>

According to:

6) Arthur L. Besse, Einstein Manifolds, Springer, Berlin, 1987, pp.
313-316.

the above spaces are the octonionic, bioctonionic, quateroctonionic
and octooctonionic projective planes, respectively.  However, I don't
yet fully understand the connection.

I thank Tony Smith for pointing out the reference to Besse
(who, by the way, is apparently a cousin of the famous Bourbaki).
Thanks also go to Allen Knutson for showing me a trick for finding
the maximal rank subalgebras of a simple Lie algebra.  

Next, here's some more stuff about the
biquaternions, bioctonions, quaterquaternions, quateroctonions
and octooctonions!  
I wrote this extra stuff as part of a post to sci.physics.research on
November 8, 1999....

\begin{quote}
One reason people like these algebras 
is that some of them - the associative ones - are also Clifford algebras.  
I talked a bit about Clifford algebras in 
&quot;<A HREF = "week105.html">week105</A>&quot;, but just remember that
we define the 
Clifford algebra C<sub>p,q</sub> to be the associative algebra you get 
by taking the real numbers and throwing in p square roots of -1 
and q square roots of 1, all of which anticommute with each other.  
This algebra is very important for understanding spinors in spacetimes 
with p space and q time dimensions.   (It's also good for studying 
things in other dimensions, so things can get a bit tricky, but I 
don't want to talk about that now.) 

For example: if you just thrown in one square root of -1 and no 
square roots of 1, you get C<sub>1,0</sub> - the complex numbers!
Similarly, one reason people like the quaternions is because
they are C<sub>2,0</sub>.  Start with the real numbers, throw in two
square roots of -1 called I and J, make sure they anticommute 
(IJ = -JI) and voila - you've got the quaternions!   
Similarly, one reason people like the biquaternions is because
they are C<sub>2,1</sub>.  You take the quaternions and complexify 
them - this amounts to throwing in an extra number i that's a square 
root of -1 and <em>commutes</em> with the quaternionic I and J - and you get
an algebra which is also generated by I, J, and K = iI.  Note
that I, J, and K all anticommute, and K is a square root of 1.  
Thus the biquaternions are C<sub>2,1</sub>!   
Similarly, one reason people like the quaterquaternions is because
they are C<sub>2,2</sub>.  You take the quaternions and quaternionify
them - this amounts to throwing in two square roots of -1, say i
and j, which anticommute but which <em>commute</em> with the quaternionic
I and J - and you get an algebra which is also generated by I, J,
K = iI, and L = jI.  Note that I, J, K, and L all anticommute, and
K and L are square roots of 1.  Thus the quaterquaternions are 
C<sub>2,2</sub>!
Now, as soon as we thrown the octonions into the mix we don't
get Clifford algebras anymore, since octonions aren't associative,
while Clifford algebras are.  However, there are still relationships
to Clifford algebras.  For example, suppose we look at all the linear
transformations of the octonions generated by the left multiplication
operations

x |&rarr; ax

This is an associative algebra, and it turns out to be <em>all</em>
linear transformations of the octonions, regarded as an 8-dimensional
real vector space.  In short, it's just the algebra of 8x8 real
matrices.  And this is C<sub>6,0</sub>.  

If you do the same trick for the bioctonions, quateroctonions and
octooctonions, you get other Clifford algebras... but I'll leave 
the question of <em>which ones</em> as a puzzle for the reader.  If you
need some help, look at the "Footnote" in &quot;<a href = "week105.html">week105</a>&quot;.

Perhaps the fanciest example of this trick concerns the biquateroctonions.  
Now actually, I've never heard anyone use this term for the algebra 
C &otimes; H &otimes; O!  The main person interested in this algebra is 
Geoffrey Dixon, and he just calls it T.  But anyway, if we look at 
the algebra of linear transformations of C &otimes; H &otimes; O generated 
by left multiplications, we get something isomorphic to the algebra of 
16 x 16 complex matrices.  And this in turn is isomorphic to C<sub>9,0</sub>.

The biquateroctonions play an important role in Dixon's grand unified 
theory of the electromagnetic, weak and strong forces.  There are lots 
of nice things about this theory - for example, it gets the right 
relationships between weak isospin and hypercharge for the fermions in 
any one generation of the Standard Model (though, as in the Standard 
Model, the existence of 3 generations needs to be put in "by hand").  
It may or may not be right, but at least it comes within shooting distance!  
You can read a bit more about his work in &quot;<a href = "week59.html">week59</A>&quot;.
\end{quote}
    


 <HR>
<em> "Mainstream mathematics" is a name given to mathematics that 
more fittingly belongs on Sunset Boulevard </em> - Gian-Carlo Rota, Indiscrete 
Thoughts


<HR>

% </A>
% </A>
% </A>


% parser failed at source line 845
