
% </A>
% </A>
% </A>
\week{October 29, 2000 }


Today I want to continue talking about 11d supergravity. 
I mainly want to describe this paper:

1) Yi Ling and Lee Smolin, Eleven dimensional supergravity as a
constrained topological field theory, available as <A HREF = "http://xxx.lanl.gov/abs/hep-th/0003285">hep-th/0003285</A>.

This paper gives an elegant new formulation of 11d supergravity by
starting from a kind of BF theory and then imposing constraints,
very much like Plebanski's formulation of ordinary gravity in 4d
spacetime.  Recall that in Plebanski's formalism, we start with:

a) a Lorentz connection A, which can locally be thought of as a 
1-form taking values in the Lie algebra of the Lorentz group, 

and:

b) a field B, which can locally be thought of as a 2-form valued
in the Lie algebra of the Lorentz group.

We get a topological field theory by using the Lagrangian 

tr(B ^ F)

where F is the curvature of A.  The equations of motion say that both
the curvature of A and the exterior covariant derivative of B vanish. 
All solutions of these equations are locally gauge-equivalent, so there
are no local degrees of freedom - that's what I mean by saying we get a
topological field theory.

But if we impose the constraint that 

B = e ^ e

where e is a "cotetrad" - which locally amounts to a 1-form taking
values in R^4 - we get the equations of general relativity!  We can
impose this constraint by throwing an extra term into the Lagrangian,
involving an extra "Lagrange multiplier" field.  The sole purpose of
this extra field is to ensure that when we compute the variation of the
action with respect to it, we get zero iff B = e ^ e. 

Similarly, in Ling and Smolin's formulation of 11d supergravity we start
with:

a) a super-Poincare superconnection A, which can locally be described 
as a 1-form taking values in the super-Lie algebra of the super-Poincare
group - or "super-Poincare algebra", for short.

b) a 3-form C.

c) a 6-form D.

We think of all three of these as "gauge fields".  I already
mentioned in "<A HREF = "week157.html">week157</A>" how a
p-form can be viewed as a generalization of the electromagnetic vector
potential which couples naturally to a membrane that traces out a
p-dimensional surface in spacetime: we just integrate the p-form over
this surface to get the action.  Annoyingly, physicists call a membrane
that traces out a p-dimensional surface in spacetime a
"(p-1)-brane", so a string is a 1-brane, a point particle is a
0-brane...  and an instanton is a -1-brane.  They should have remembered
to count spacetime dimensions instead of space dimensions!  Then we
wouldn't have this nasty "minus one" stuff.

But anyway, the usual formulation of 11d supergravity (see "<A HREF
= "week157.html">week157</A>") involves a 3-form field, which
couples naturally to 2-branes.  This is nice because there's lots of
evidence that M-theory has a lot to do with 2-branes.  The nice thing
about Ling and Smolin's formulation is that it also includes a 6-form
field, which couples to 5-branes.  There's also a lot of evidence that
M-theory is related to 5-branes, but these have always been a bit more
mysterious than the 2-branes.  Now, however, they're staring us in the
face right from the start!

Next, before I go further, I should say what the "super-Poincare
algebra" is!  

In fact, I've been pretty coy all along about explaining supersymmetry.
Let me quickly try to remedy that.  The basic idea of supersymmetry is
that we should build the distinction between bosons and fermions into
all the math we ever do.  So instead of doing math with vector spaces,
we should do it with "supervector spaces".  A supervector
space is just a direct sum of two vector spaces, called the
"even" or "bosonic" space and the "odd" or
"fermionic" space.  So, for example, the Hilbert space of a
quantum system built out of bosons and fermions will always be a
supervector space.

Supervector spaces work a lot like ordinary vector spaces, so we can
redo all of math replacing vector spaces by supervector spaces.  To do
this, we just copy all the usual stuff, EXCEPT that whenever we switch
two vectors past each other in our formulas, we stick in an extra minus
sign when they're both odd!  This reflects the way fermions actually
work in nature: when you exchange two of them, their wavefunction picks
up a phase of -1.  

Supervector spaces are also an obvious idea if you've studied enough 
math.  For example, differential forms of odd degree anticommute with
each other, while forms of even degree commute with everything.  So
the differential forms on a manifold really form a supervector space,
and in fact, a "supercommutative algebra".  For reasons like this, 
mathematicians and physicists got together back in the 1980s and figured
out how to redo huge wads of algebra in the context of supervector spaces.
It's actually very easy if you use a little category theory....

Anyway, using this trick we can come up with the notion of a "super-Lie
algebra".  It's almost like a Lie algebra, except that the bracket [A,B]
of two odd elements A and B behaves like an anticommutator AB+BA instead
of the usual commutator AB-BA.  This means we need to throw in suitable
signs into the Jacobi identity and other Lie algebra axioms: an extra
minus sign whenever two odd elements get switched!

Now, how about the super-Poincare algebra?  

As you probably know, the Lie algebra of the Poincare group has
translation generators P_{a} and rotation/boost generators 
L_{ab}, where
the indices go from 1 to n if spacetime has n dimensions.  I won't bother
writing down the well-known commutation relations between these guys.

The super-Lie algebra of the super-Poincare group contains all this
stuff as its even part, but it also has an odd part!  The odd part has a
basis of "supertranslation generators" Q_{A}, where A
ranges over a basis of real spinors.  Now, spinors are like "square
roots of vectors": there's a natural symmetric bilinear map taking
a pair of spinors to a vector.  So it's natural to define the bracket of
two supertranslations by:

[Q_{A},Q_{B}] = \Gamma _{AB}^{a} P_{a}

where the so-called "\gamma  matrix"
\Gamma _{AB}^{a} is just the physicist's coordinate-ridden
way of describing this map taking a pair of spinors to a vector.  Since
this map is symmetric, we have

[Q_{A},Q_{B}] = [Q_{B},Q_{A}] 

If you're used to Lie algebras, this equation must look like it's
missing a minus sign - but we're doing super-Lie algebras, and
the supertranslation generators are odd, so we expect that!

To complete the definition, we need to describe the brackets between
supertranslations and the even elements of our super-Lie algebra.  This
is easy.  The bracket of an ordinary translation and a supertranslation
is zero.  The bracket of a rotation/boost and a supertranslation is
defined using the usual action of the Lie algebra of the Lorentz group
on spinors.

Okay, now let's go back and think a minute about what the
"superconnection" in Ling and Smolin's formulation of 11d
supergravity is really like.  If we work locally, we can think of this
as a 1-form taking values in the super-Poincare algebra.  Thus it really
consists of 3 parts:

a) a 1-form taking values in the Lorentz Lie algebra so(10,1).  
This is secretly the "spin connection" in the usual formulation of 
11d supergravity, as described last week.

a') a 1-form taking values in the translation Lie algebra R^{11}.    
This is secretly the "elfbein" in the usual formulation of 11d
supergravity, as described last week.

a'') a 1-form taking values in the space of real spinors.  This is 
just the "gravitino" in the usual formulation of 11d supergravity,
as described last week.

So you see, this fancy-schmancy super-baloney really helps simplify 
our description of what's going on!

I'm getting a little worn out, so I'll just summarize the rest of
the story.  First, how do Ling and Smolin get their 11d topological
field theory?  Like I said, it's a kind of BF theory, where the
Lagrangian is like tr(B ^ F).  But there are a bunch of F fields - 
i.e., curvatures - and thus a bunch of B fields.  Namely, we've got 
the curvature of the superconnection A, the curvature dC of the 3-form C,
and the curvature dD of the 6-form D.  And if you analyze it, the
curvature of the superconnection consists of 3 separate parts.  So we
really have five F fields.  Each one has its corresponding B field, and
the Lagrangian is a sum of terms of the form tr(B ^ F).  

To get 11d supergravity, we have to impose a bunch of constraints by
throwing extra terms into the Lagrangian.  There is one term like this
for each F field.   We also have to throw in a term which gives the
analog of Maxwell's equations for the 3-form field C.  So the paper's
title is a mild lie!  We're not seeing 11d supergravity as simply a
constrained topological field theory - there's also an extra interaction.

By the way, if you've never seen the Plebanski formulation of 4d gravity
as a constrained BF theory, here's the original paper:

2) M. J. Plebanski, On the separation of Einsteinian substructures,
J. Math. Phys. 18 (1977), 2511.

Ling and Smolin's formulation of 11d supergravity is related to some
work of Fre and collaborators, which I haven't read yet:

3) Pietro Fre, Comments on the six index photon in D = 11, preprint
TH-3884-CERN.

4) R. D'Auria and P. Fre, Geometric supergravity in D = 11 and its hidden
supergroup, Nucl. Phys. B201 (1982), 101.  Erratum, Nucl. Phys. B206 (182),
496.

5) L. Castellani, P. Fre and P. van Nieuwenhuizen, A review of the group 
manifold approach and its applications to conformal supergravity, Ann.
Phys. 136 (1981), 398.

Here's another formulation of 11d supergravity I'd like to check out:

6) Martin Cederwall, Ulf Gran, Mikkel Nielsen, and Bengt Nillson,
Generalised 11-dimensional supergravity,  available as <A HREF =
"http://xxx.lanl.gov/abs/hep-th/0010042">hep-th/0010042</A>.

Cederwall has done interesting work on octonions and physics, so I want
to look here for clues that 11d supergravity is related to the octonions.

Actually, now that I've said a bit about supersymmetry, I can explain a
bit about how it's related to division algebras and exceptional groups.
All this stuff will be described in more detail in my review article on
octonions, which I'll finish by March of next year.  But I can't resist
saying a little right now....

As we've seen, a crucial part of the super-Poincare algebra is the map 
taking a pair of real spinors to a vector.  Abstractly we can write 
this as follows:

m: S x S \to  V.

In certain dimensions we can split the spinor space S into spaces of
left- and right-handed spinors, say S+ and S-.  Then we get a map

m: S+ x S- \to  V.

This stuff works both for Minkowski spacetime and for Euclidean space.  
If we do it for Euclidean space, we find a marvelous fact....

In certain special cases - namely dimensions 1 and 2 - the dimension of
V matches the dimension of S.  This lets us identify V with S.  Then the
map 

m: S x S \to  V

turns out to be \emph{multiplication} for the real and complex numbers,
respectively.

In other special cases - namely dimensions 4 and 8 - the dimension of V
matches the dimension of S+, and also S-.  This lets us identify V with
S+ and S-.  Then the map 

m: S+ x S- \to  V

turns out to be \emph{multiplication} for the quaternions and octonions, 
respectively.  

In other words, the vector-spinor interaction which plays such an
important role in physics:


\begin{verbatim}

                  \
                   \
                    \
                     ~~~~~~~~~
                    /
                   /
                  /

\end{verbatim}
    
also gives rise to all the division algebras!  (Here I've drawn the
usual picture of a spinor particle and a spinor antiparticle annihilating 
to form a vector boson: this is a physics application of the map m.)

Another crucial part of the super-Poincare algebra is the action of the 
Lorentz Lie algebra on spinors.  Again, this has a Euclidean analogue,
where the Lie algebra of the Lorentz group gets replaced by that of 
the rotation group.  In n dimensions, we thus get an action

so(n) x S \to  S

which we can also dualize to get a map

S x S \to  so(n).

Of course, we also have the Lie bracket

so(n) x so(n) \to  so(n).

So it's natural to ask: can we use all three of these maps to define
a Lie bracket on the direct sum of so(n) and the spinor space S?

And the answer is: yes, but only if n = 9.  Then we get the exceptional
Lie algebra F4.  

Spurred on by our success, we can ask: what if we use right-handed
spinors instead?  If we restrict the above maps to right-handed spinors,
can we define a Lie bracket on the direct sum of so(n) and the space S+?

And the answer is: yes, but only if n = 16.  Then we get the exceptional
Lie algebra E8.

And then we ask: can we get the other exceptional Lie algebras by some
variant of this trick?  

And the answer is: yes, at least for E6 and E7.  

If n = 10, the spinor space S is naturally a complex vector space, so
u(1) acts on it.  Using this and the above maps, we can make the direct
sum of so(10), S and u(1) into a Lie algebra, which turns out to be E6.

If n = 12, the right-handed spinor space S+ is naturally a quaternionic
vector space, so su(2) acts on it.  Using this and the above maps, we
can make the direct sum of so(12), S+ and su(2) into a Lie algebra, which
turns out to be E7.

In short, we have the following story:


\begin{verbatim}

natural maps involving vectors and spinors give: R in dimension 1
                                                 C in dimension 2
                                                 H in dimension 4
                                                 O in dimension 8

natural maps involving so(n) and spinors give:  F4 in dimension 9
                                                E6 in dimension 10
                                                E7 in dimension 12
                                                E8 in dimension 16
\end{verbatim}
    
And you'll note that the dimensions in the second list are 8 more than
the corresponding dimensions in the first list.  This is no coincidence!
It has to do with the octonions.  But I'm too tired to explain that now....

Anyway, my main point was just that the natural maps involving 
rotation/boost generators (i.e. the Lorentz Lie algebra, or rotation Lie
algebra), translation generators (i.e. vectors) and supertranslation
generators (i.e. spinors) are the essential ingredient for constructing:

a) the super-Poincare algebra
b) the division algebras R, C, H and O
c) the exceptional Lie algebras F4, E6, E7 and E8

So it's not really odd to expect relations between these three things!

Of course, I've shown how items b) and c) are related to rotations,
spinors and vectors in Euclidean space, while item a) is related to
rotations/boosts, spinors and vectors in Minkowski spacetime.  To round
off the picture, I'd have to describe the relation between spinors in 
n-dimensional Euclidean space to spinors in (n+2)-dimensional Minkowski
spacetime.  It's this relation that gives the isomorphisms

so(2,1) = sl(2,R)
so(3,1) = sl(2,C)
so(5,1) = sl(2,H)
so(9,1) = sl(2,O)

which I mentioned already in "<A HREF =
"week104.html">week104</A>".  This is what lets us write down the
super-Yang-Mills Lagrangians and superstring Lagrangians in spacetimes
of dimension 3, 4, 6, and 10 - i.e., 2 more than the magic numbers 1, 2,
4, and 8.  Adding 8, we can guess there should also be fun stuff in
spacetimes of dimensions 11, 12, 14 and 18, related to F4, E6, E7 and
E8, respectively.  Is this true?  Is the 11d case related to 11d
supergravity - or M-theory?  I don't know.





 \par\noindent\rule{\textwidth}{0.4pt}

% </A>
% </A>
% </A>
