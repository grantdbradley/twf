
% </A>
% </A>
% </A>
\week{October 10, 2001 }

There isn't a Nobel prize for mathematics.  You've probably heard why: 
Alfred Nobel was annoyed that the famous mathematician Mittag-Leffler 
had an affair with his wife.  Well, that's what they say, anyway.  
It makes a great story.  The only problem is, Nobel was never married!  
So it's just another urban legend.  For more details, see:

1) Urban legends reference pages, The Prize's Rite,
<A HREF = "http://www.snopes2.com/science/nobel.htm">http://www.snopes2.com/science/nobel.htm</A>

More likely, Nobel just didn't consider mathematics sufficiently
practical.  In any event, mathematicians have always felt a bit grumpy
about this slight.  Their adoption of the Fields Medal as a kind of
substitute has never been completely satisfying.  For one thing, the
Fields Medal is only for work done before the age of 40 - a condition
that seems ever more silly with the wisdom of age.  For another, the
Fields prize gives you a measly 15,000 Canadian dollars, while the Nobel
prize keeps going up: this year, it was 10 million Swedish crowns, or
almost a million bucks.

Anyway, now there may be a better substitute: the Abel Prize.


 Norway Establishes Abel Prize in Mathematics, 
<A HREF = "http://www.maa.org/news/abel_prize.html">http://www.maa.org/news/abel_prize.html</A>

It even almost rhymes with Nobel!  Abel, of course, was a famous
Norwegian mathematician, and this prize will be awarded annually by the
government of Norway, starting in 2003.  It will have a value of about
$500,000, at least initially.  Even better, it will be awarded on a
first-come, first-serve basis... so send in your application now.

When I was in Cambridge this summer, I visited Tom Leinster and Eugenia
Cheng, who showed me around the new mathematics buildings.  The
Cambridge system is too Byzantine for a mere American to understand, but
there are two main things resembling a "mathematics department": DPMSS,
the Department of Pure Mathematics and Mathematical Statistics, and
DAMTP, the Department of Applied Mathematics and Theoretical Physics.
They used to be in separate dilapidated buildings downtown on Silver
Street, but now they occupy two towers in a huge complex near the Newton
Institute, on the outskirts of town.  

The new setup is pretty cool.  Parts of it are still under construction,
but you can get the idea already.  Different breeds of mathematicians
will be housed in different towers, all surrounding a central building
resembling an airplane hanger, which is actually an enormous cafeteria.
The univeral human interest in food will lure otherwise aloof
specialists to mingle and chat.  I even saw Hawking there one day.
However, there is also a separate coffee lounge at the base of each
tower, so the different groups can have slightly more private chats.
Futuristic light sensors lower curtains in the cafeteria whenver the sun
comes out, to enhance the visitor's impression that it's always cloudy
in England.  But the really cool thing is that every tower has a door on
the second floor which opens out to the \emph{roof} of the cafeteria.  The
roof is covered with grass, like a little park!  Finally, people working
on fluid dynamics are kept in the basement, which gurgles mysteriously
with the sound of experiments.  

Leinster and Cheng are both students of Martin Hyland, and they
both work on n-categories.   I've talked about their work before
in "<A HREF = "week165.html">week165</A>".  Leinster has just come out with a nice paper on
n-categories:

1) Tom Leinster, A survey of definitions of n-category, available
at <A HREF = "http://xxx.lanl.gov/abs/math.CT/0107188">math.CT/0107188</A>.

By now, there are lots of definitions of "weak n-category", and
our job is to understand how they're related.  This paper is required
reading for anyone interested in this business: it goes through 10
different definitions, giving each definition in two pages and then
using two more pages to show how it works for n less than or equal to 2.
It also has a nice annotated bibliography giving some of the history
of the subject.

While I'm talking about review articles, here are some review
articles on quantum gravity:

2) Steve Carlip, Quantum gravity: a progress report,
Rep. Prog. Phys. 64 (2001) 885-942, also available at <A HREF = "http://xxx.lanl.gov/abs/gr-qc/0108040">gr-qc/0108040</A>.

This is an excellent \emph{long} description of where we stand on quantum
gravity, with a strong focus on the big conceptual problems.  Again,
it's required reading for anyone in this field.  It doesn't do justice
to string theory, which is a mammoth subject in its own.  For that, 
you might try this article which I bumped into in the same journal:

3) Ulf Daniellson, Introduction to string theory, 
Rep. Prog. Phys. 64 (2001) 51-96.  

It seems to do a pretty good job of the impossible - explaining
all of string theory in less than 50 pages.  Of course, if you
want to get serious, you'll eventually have to read some of the
string theory textbooks listed in "<A HREF = "week124.html">week124</A>" and elsewhere.

There is also a new introduction to loop quantum gravity available
online.  It's more of a book than an article:

4) Thomas Thiemann, Introduction to modern canonical quantum general
relativity, 301 pages, available at <A HREF = "http://xxx.lanl.gov/abs/gr-qc/0110034">gr-qc/0110034</A>.

This is really \emph{the} place to go if you want to catch up on the last 15
years of work on loop quantum gravity.  It's truly impressive.  It'll
make fairly substantial demands on the average physicist's mathematical
know-how: for example, not just differential geometry, which everyone
into gravity must know, but also functional analysis.  Luckily, it has
an appendix over 40 pages long which explains much of the needed math.
For the would-be grad student or postdoc, a very helpful feature is the
list of institutions where loop quantum gravity is studied, in the
Introduction.

Speaking of loop quantum gravity, here are a few interesting
new papers on that subject:

5) Rodolfo Gambini and Jorge Pullin, Consistent discretizations for
classical and quantum general relativity, available as <A HREF = "http://xxx.lanl.gov/abs/gr-qc/0108062">gr-qc/0108062</A>.

6) Luca Bombelli, Statistical geometry of random weave states, available
as <A HREF = "http://xxx.lanl.gov/abs/gr-qc/0101080">gr-qc/0101080</A>.

7) Michael Seifert, Angle and volume studies in quantized space, 85 pages,
available as <A HREF = "http://xxx.lanl.gov/abs/gr-qc/0108047">gr-qc/0108047</A>.

The paper by Gambini and Pullin argues that good spin foam models will
come from quantizing "consistent" discretizations of general
relativity, that is, those where the discretized equations of motion
preserve the constraints on initial data, and where the solutions
converge to solutions of the continuum equations in the limit where the
discretization is made ever more fine.

The paper by Bombelli presents a proposal for states of loop quantum
gravity that should be good approximations to classical geometries.  The
idea is to take a Riemannian manifold, sprinkle points on it randomly
form the corresponding Voronoi diagram, and label the edges with spins
in a certain way to get a spin network.  If we then average over all
possible ways of randomly sprinkling these points, we get Bombelli's
"random weave state" - a kinematical state of quantum gravity that
approximates of the Riemannian geometry we started with.

I don't know if that made sense to you.  Do you at least know what a
Voronoi diagram is?  To explain that, a picture is worth a thousand
words, so I won't explain the concept - I'll just urge you to play 
with this applet:

9) Paul Chew, Voronoi/Delaunay Applet, 
<A HREF = "http://www.cs.cornell.edu/Info/People/chew/Delaunay.html">http://www.cs.cornell.edu/Info/People/chew/Delaunay.html</A>

If you click the mouse to sprinkle the rectangle with points, you'll see
a bunch of edges appear, which intersect in vertices, forming a graph
called the Voronoi diagram.  By epxerimenting a bit you can figure out
how it works - or else you can cheat and read the text.  You'll see that
generically the vertices of this graph are trivalent: they have three
edges coming out of them.  If you click on the button that says
"Delaunay Triangulation", you'll see the dual graph, which
generically consists of a bunch of triangles.  Each edge of these
triangles intersects exactly one edge of the Voronoi diagram.

In the theory of quantum gravity where space is just 2-dimensional (a
toy model), we can take the Voronoi diagram and label its edges by spins
j = 0, 1/2, 1, ... which match, as well as possible, the lengths of the
edge of the Delaunay triangulation which it intersects.  This will give
us a spin network.  Averaging over all ways of sprinking the points, we
then get Bombelli's "random weave state".  The same sort of idea works
in higher dimensions, too.

Finally, Michael Seifert's paper is an excellent undergraduate thesis on
loop quantum gravity, done with the help of Seth Major.  After a nice
review of the basics, it studies some operators that act on the Hilbert
space of states of a single spin network vertex: in particular, the
volume operator and some less familiar operators that measure the angles
between spin network edges.  He proves some nice things about these, and
also gets some interesting numerical results - which someone should make
into theorems.  The relation between 3d geometry and the representation
theory of SU(2) still has unexplored wrinkles!






<p> <hr>

% </A>
% </A>
% </A>


% parser failed at source line 259
