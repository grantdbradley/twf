
% </A>
% </A>
% </A>
\week{April 20, 2005 }


What common English slang phrase alludes to the number 168?

I won't tell you - not right away.  But, I'll tell you a bunch of 
other cool stuff about this number, and eventually the answer should 
jump out at you. 

Okay:

Start with a bunch of equilateral triangles.  Glue them together
so that 3 meet at each corner.  You get a regular tetrahedron.

Next, take a bunch of squares.  Glue them together so that 3
meet at each corner.  You get a cube.

Next, take a bunch of regular pentagons.  Glue them together so 
that three meet at each corner.  You get a regular dodecahedron.

This is fun!  We're getting a series of Platonic solids.

Next, take a bunch of regular hexagons and glue them together 
so that three meet at each corner.  Now the angles of the hexagons 
add up to 360 degrees, so we don't get a Platonic solid.  Instead, we 
get a tiling of the plane.  It looks like a honeycomb that stretches 
out forever in all directions.

But, if you want something finite in size, you can cut out a portion 
of this honeycomb and curl it up to get a doughnut, or torus.  There 
are actually lots of ways to do this.  You might have fun figuring out 
what they all are.  Can you take just \emph{one} regular hexagon and curl 
it up to form a torus?

Anyway, these tori deserve to be called "Platonic surfaces", since
they are surfaces tiled with regular polygons, with the same number
meeting at each vertex.   

Next, let's take a bunch of regular \emph{heptagons} and glue them 
together so that three meet each corner.  Now the angles add up
to more than 360 degrees, so we get a tiling of the "hyperbolic
plane".  The hyperbolic plane is like the opposite of a sphere,  
since it's saddle-shaped at every point instead of bulging 
out at every point.  In fact the sphere and the hyperbolic plane
are the two most symmetrical forms of non-Euclidean geometry.  
The sphere is "positively curved", while the hyperbolic plane is 
"negatively curved".

You may have trouble visualizing the hyperbolic plane tiled with 
regular heptagons, but if we distort it, it fits into a disk and 
looks really pretty!  Here it is:

1) Don Hatch, Hyperbolic planar tesselations,
<A HREF = "http://www.hadron.org/~hatch/HyperbolicTesselations">http://www.hadron.org/~hatch/HyperbolicTesselations/</A>

<BR>
<A HREF ="http://www.hadron.org/~hatch/HyperbolicTesselations/">
<DIV ALIGN = CENTER>
<IMG SRC = 7_3.gif>
</DIV>
% </A>


It's called "{7,3}", since it's made of 7-sided figures with 3 
meeting at each corner.  

In this picture there's one heptagon at the center, surrounded by 
rings of heptagons that appear smaller (but aren't really - that's
just an effect of the distortion).  

Can we cut out a portion of this tiling and curl it up to get
a torus?  No!  But, we can curl up a portion to get a 3-holed
torus - like the surface of a doughnut with three holes.  But, 
we can only do this if we use precisely 24 heptagons!

Here's how we do it.  Here's a picture of 24 heptagons, taken
from an old paper by Klein and Fricke but prettied up a bit:

2) Tony Smith, Klein's quartic surface,
<A HREF = "http://www.valdostamuseum.org/hamsmith/cdomain.html#tesselations">http://www.valdostamuseum.org/hamsmith/cdomain.html#tesselations</A>

<BR>
<A HREF = "http://www.valdostamuseum.org/hamsmith/Klein168.gif">
<DIV ALIGN = CENTER>
<IMG SRC = Klein168.gif>
</DIV>
% </A>


You'll notice they're drawn in a fancy style: each heptagon has 
been "barycentrically subdivided" into 14 right triangles.  
But don't worry about that yet; concentrate on the heptagons.

There's a blue heptagon in the middle, 7 red ones touching that,
7 yellow ones touching those, then 7 green ones falling off the 
edge of the picture, and 2 blue ones broken into bits all around 
the corners of the picture.  That's a total of 24 heptagons.

We wrap this thing up into a 3-holed torus using the numbers on 
the edges of the picture:

<UL>
<LI>
connect edges 1 and 6
<LI>
connect edges 3 and 8 
<LI>
connect edges 5 and 10
<LI>
connect edges 7 and 12
<LI>
connect edges 9 and 14
<LI>
connect edges 11 and 2
<LI>
connect edges 13 and 4
</UL>

In other words, connect edges 2n+1 and 2n+6 mod 14.
To connect them the right way, make sure that triangles
of the same color never touch each other.

Here's how to see if you get the idea.   Ignore the little
triangles; just pay attention to the heptagons!   Then:

Start on any edge of any heptagon and march along in either direction. 

Then, when you get to the end, turn left.  

Then, when you get to the end, turn right.  

Then, when you get to the end, turn left.  

Then, when you get to the end, turn right.  

Then, when you get to the end, turn left.  

Then, when you get to the end, turn right.  

Then, when you get to the end, turn left.  

Then, when you get to the end, turn right.  

You should now be back where you started!!!

These are like the driving directions the devil gives people who ask 
the way out of hell.  LRLRLRLR and you're right back where you started.

But the resulting Platonic surface is heavenly.   It has lots of
symmetries.  Each of the 24 heptagons has 7-fold rotational symmetry - 
and amazingly, all these rotations extend to a symmetry of the Platonic 
surface!   

Now let's talk about those little triangles.  Since our surface 
is made of 24 heptagons, each chopped into 14 right triangles, 
there are a total of 

24 \times  14 = 336

triangles.  And this number is also the number of symmetries of the 
Klein quartic, including reflections!   

This is no coincidence.  We can specify a symmetry by saying where 
it sends our favorite right triangle.  Since it can go to any other
triangle, there are 336 possibilities.  If we exclude reflections,
we get half as many symmetries: 24 \times  7 = 168.

By the way, this trick works for ordinary Platonic solids as well.  
For example, if we take a dodecahedron and barycentrically subdivide 
all 12 pentagons, we get 10 x 12 = 120 right triangles.    If we pick 
one of these as the "identity element", we can specify any 
symmetry by saying which triangle this triangle gets sent to.  So, the set of 
triangles becomes a vivid \textbf{picture} of the 120-element symmetry group of
 the dodecahedron.  It's called the "Coxeter complex".  This idea 
generalizes in many directions, and is incredibly useful.

Anyway... there is much more to say about this stuff.  For example, 
if we take our hyperbolic plane tiled with heptagons and count them 
grouped according to how far they are from the central one, we get 
the sequence

7, 7, 14, 21, 35, 56, 91, ....

These are 7 times the Fibonacci numbers!

To dig a bit deeper, though, it helps to think about complex analysis.

If we think of the hyperbolic plane as the unit disc in the complex 
plane, this surface becomes a "Riemann surface", meaning that it
gets equipped with a complex structure.  This was Felix Klein's
viewpoint when he discovered all this stuff in about 1878.  He 
showed this surface could be described by an incredibly symmetrical 
quartic equation in 3 complex variables:

u^{3} v + v^{3} w + w^{3} u = 0

where we count two solutions as the same if they differ by an overall 
factor.  So, it's called "Klein's quartic curve".  

(Why a "curve" and not a surface?  Because it takes one 
\emph{complex}
number to say where you are on it.  We have 3 unknowns and one 
equation, but we mod out by an overall factor, so we get something
locally parametrized by one complex number... so algebraic geometers
call it a curve.)

You can read Klein's original article translated into English.  
It's available online as part of a whole \emph{book} about his incredible 
quartic: 

3) Silvio Levy, The Eightfold Way: the Beauty of Klein's Quartic Curve,
MSRI Research Publications 35, Cambridge U. Press, Cambridge 1999.
Also available as <A HREF = "http://www.msri.org/publications/books/Book35/">
http://www.msri.org/publications/books/Book35/</A>

This book was put out by the Mathematical Sciences Research Institute 
in Berkeley, to celebrate the completion of sculpture of Klein's
quartic curve made by Helaman Ferguson.  I must admit that the 
sculpture leaves me unmoved.  But the curve itself - ah, that's 
another story!

For example, Klein's quartic curve turns out to have the maximum 
number of symmetries of any 3-holed Riemann surface. 

Let's back up a minute and think about a Riemann surface with no holes:
a sphere.  There's only one way to make a sphere into a Riemann 
surface - it's called the Riemann sphere.  You can think of it as
the complex numbers plus a point at infinity.  This has \emph{infinitely}
many symmetries.  They're called conformal transformations, and they
all look like this:


$$

           az + b
z |--->   --------
           cz + d
$$
    
They form a group called PSL(2,C), since it's the same as the group of
2x2 complex matrices with determinant 1, mod scalars.  It's also the 
same as the Lorentz group!  

There are different ways to make a torus into a Riemann surface,
some with more symmetries than others (see "<A HREF = "week124.html">week124</A>").  But, there 
are always translation symmetries in both directions, so the symmetry 
group is always infinite. 

On the other hand, a Riemann surface with 2 or more holes can only 
have a \emph{finite} group of conformal transformations.  In fact, in 1893 
Hurwitz proved that a Riemann surface with g holes has at most 

84(g - 1)

For g = 3, this is 168.  So, Klein's quartic surface is as symmetrical 
as possible!  (We don't count reflections here, since they don't 
preserve the complex structure - they're like complex conjugation.)

Now I should break down and give the best description of Klein's
quartic curve as a Riemann surface.  Sitting inside PSL(2,C) is
PSL(2,Z), where we only use integers a,b,c,d in our fractional linear 
transformation


$$

           az + b
z |--->   --------
           cz + d 
$$
    
This subgroup acts on the upper half-plane H, which is just another
way of thinking about the hyperbolic plane.

Sitting inside PSL(2,Z) is a group \Gamma (7) consisting of guys where 
the matrix 


\begin{verbatim}

a  b

c  d 
\end{verbatim}
    
is congruent to the identity:


\begin{verbatim}

1  0

0  1
\end{verbatim}
    
modulo 7.  This is an example of a "congruence subgroup"; these serve
to relate complex analysis to number theory in lots of cool ways.
In particular, Klein's quartic curve is just

H/\Gamma (7)

Since \Gamma (7) is a normal subgroup of PSL(2,Z), the quotient
group 

PSL(2,Z)/\Gamma (7) = PSL(2,Z/7) 

acts as symmetries of Klein's quartic curve.  And, this group has
168 elements!  

In fact, this group is the second smallest nonabelian simple group.
The smallest one is the rotational symmetry group of the icosahedron,
which has 60 elements.  This group is actually PSL(2,Z/5), and
Klein had run into it in his work on the icosahedron and 
quintic equations (see "<A HREF = "week213.html">week213</A>").  
So, it's actually far from 
sheer luck that he then moved on to PSL(2,Z/7) and ran into his 
wonderful quartic curve.

By the way, this 168-element group is also known as PSL(3,Z/2) - 
the symmetry group of the "Fano plane".  This is a name for the 
projective plane over Z/2.  The Fano plane is closely related to 
the octonions:

3) John Baez, The Fano plane,
<A HREF = "http://math.ucr.edu/home/baez/octonions/node4.html">
http://math.ucr.edu/home/baez/octonions/node4.html</A>

<BR>
<DIV ALIGN = CENTER>
<IMG SRC = "fano.jpg">
</DIV>


So in fact, our 168-element group acts on the set of octonion
multiplication tables:

4) Tony Smith, Octonion products,
<A HREF = "http://www.valdostamuseum.org/hamsmith/480op.html">http://www.valdostamuseum.org/hamsmith/480op.html</A>

5) Geoffrey Dixon, Octonion multiplication tables,
<A HREF = "http://www.7stones.com/Homepage/octotut0.html">http://www.7stones.com/Homepage/octotut0.html</A>

And, as James Dolan just noted today, and Tony Smith seems to have known
all along, there's a way to draw the Fano plane that even \emph{looks} like 
the diagram Klein and Fricke used to build the Klein quartic.  Here's
a picture drawn by Burkard Polster, author of "The Mathematics
of Juggling" and "Geometries on Surfaces":

<BR>
<DIV ALIGN = CENTER>
<IMG SRC = symfano.gif>
</DIV>
<BR>

So, something interesting is going on, and I want to know what it is!

By the way, fans of the quaternions and octonions may like this
review of Conway and Smith's book:

6) John Baez, review of "On Quaternions and Octonions: Their Geometry,
Arithmetic and Symmetry", by John H. Conway and Derek A. Smith,
Bull. Amer. Math. Soc. 42 (2005), 229-243.  Available at 
<A HREF = "http://www.ams.org/bull/2005-42-02/">http://www.ams.org/bull/2005-42-02/</A> and 
<A HREF = "http://math.ucr.edu/home/baez/octonions/node24.html">http://math.ucr.edu/home/baez/octonions/node24.html</A>

It's packed with cool pictures and weird facts - a more refined
version of the material in "<A HREF = "week193.html">week193</A>" and "<A HREF = "week194.html">week194</A>".  

It builds up to a kind of crazy climax in which I describe how when you 
pack spheres as densely as possible in 8 dimensions, each sphere touches 
240 others... and if you look at the 240 neighbors of a given sphere, each 
one of those neighbors touches 56 other neighbors.  Then I explain how this 
gives rise to a 56-dimensional representation of the exceptional group E7 -
its smallest nontrivial representation!  And, how it
gives rise to a 57-dimensional manifold on which the exceptional group E8 
acts - the smallest space on which it acts nontrivially!   

Bertram Kostant is one of the real gurus of Lie theory.  He teaches at
MIT, and he has a strong fondness for exceptional Lie groups.  When he saw 
this review of mine, he mentioned a couple of other papers that construct 
the 57-dimensional space on which E8 acts:

7) Ranee Brylinski and Bertram Kostant, Lagrangian models of minimal
representations of E6, E7, and E8, in Functional Analysis on the Eve
of the 21st Century, vol. 1, Progress in Math. 131, Birkhauser, Boston, 
1995, pp. 13-53.

Bertram Kostant, Minimal coadjoint orbits and symplectic induction,
in The Breadth of Symplectic and Poisson geometry, 391-422, 
Progress in Math. 232, Birkhauser, Boston, 2005.  Also
available as <A HREF = "http://www.arXiv.org/abs/math.SG/0312252">
http://www.arXiv.org/abs/math.SG/0312252</A>

I've got to read these sometime.

Having the number 56 on my brain, I can't resist nothing that if
you take Klein's quartic curve tiled by heptagons, and you count 
the vertices, you get 

24 \times  7 / 3 = 56

since each vertex is shared by 3 heptagons.  I'm hoping this is
not a coincidence!  

Okay, that's all for this week, except for some silly stuff....

First of all, speaking of octonions, Geoff Corbishley just told me 
that their inventor, John Thomas Graves, is a relative of Robert 
Graves - the author of "I Claudius". 

Second of all, I hope you've figured out the puzzle I gave at the 
beginning of this Week.  The phrase is "24-7", as in "we're working 
on it 24-7".  24 hours a day, 7 days a week, makes 168 hours per week!

Finally, speaking of numerology, this number 168 is related to why the 
days of the week have the names they do!  I explained why in "<A HREF = "week175.html">week175</A>", 
but I'll remind you:

Astrologers liked to list the planets in order of decreasing orbital 
period, counting the sun as having a period of one year, and the moon 
as period of one month:


\begin{verbatim}

Saturn    (29 years)  
Jupiter   (12 years) 
Mars      (687 days)  
Sun       (365 days)
Venus     (224 days)
Mercury   (88 days)
Moon      (29.5 days)
\end{verbatim}
    
For the purposes of astrology they wanted to assign a planet to each 
hour of each day of the week.  To do this, they assigned Saturn to 
the first hour of the first day, Jupiter to the second hour of the 
first day, and so on, cycling through the list of planets over and 
over, until each of the 24 \times  7 = 168 hours was assigned a planet.  
Each day was then named after the first hour in that day.  Since 
24 mod 7 equals 3, this amounts to taking the above list and cycling
around it, reading off every third planet:
 

\begin{verbatim}

Saturn  (Saturday)  
Sun     (Sunday)   
Moon    (Monday)  
Mars    (Tuesday)  
Mercury (Wednesday)
Jupiter (Thursday)
Venus   (Friday)
\end{verbatim}
    

And that's how they got listed in this order!  At least, this is what 
the Roman historian Dion Cassius (AD 150-235) claims.  Nobody knows for
sure.


\par\noindent\rule{\textwidth}{0.4pt}

\textbf{Addendum:}
Mike Stay took Don Hatch's picture and drew numbers
from 1 to 24 showing how to identify heptagons in order 
to get the Klein quartic curve:

<BR>
<DIV ALIGN = CENTER>
<IMG SRC = 7-3.gif>
</DIV>

Gerard Westendorp had some interesting comments on what
I wrote:

\begin{quote}
If you take Euler's formula

$$

     V+F-E = 2-2\times holes
$$
    
then you can figure out that for a (7,3) tiling with N heptagons,
you have

\begin{verbatim}

     V = 7N/3
     F = N
     E = 7N/2
\end{verbatim}
    
so that

$$

     N = 12\times (holes-1)
$$
    
Here's the table of solutions:

\begin{verbatim}

   holes    N
    0      -12
    1       0
    2       12
    3       24
    4       36
    .       . 
    .       . 
\end{verbatim}
    
So indeed, there are no solutions for 0 holes (sphere)
or 1 hole (torus).  But a 2-holed torus should be possible,
as well as the 24-faced 3-holed one.

Anyway, see if I can visualise the 3-holed one.

If you start with a sphere, i.e. genus 0, and drill
a tunnel through it, you will get a genus 1 object. On
the outer surface, you see 2 holes, one for each side
of the "tunnel". (I use the word "tunnel"
for something into a 3D object, and "hole" for something
in a 2D surface.)

Next, you can drill a second tunnel, and get a genus 2 object,
and you would see 4 holes on the outer surface.
But a nice trick is to drill not to the outer surface,
but to a secret "cave" in the middle where you meet the
first tunnel. Here you stop drilling. To complete
the genus-3 object, you drill the third tunnel again not
to the outer surface, but to the central cave. Thus, you
get and object with genus 3, which has 4 holes on its
outer surface, each leading to a central cave.

Confusingly, 4 tunnels to a central cave is topologically
the same as 3 separate tunnels!  The trick is that tunnels
do not have to end on the outer surface, the inner surface
is topologically the same.

OK, so we have an object with 4 holes on its outer surface.
4 holes \to  tetrahedron...

I built a cardboard model of a tetrahedron with a central
cave.  Truncated tetrahedrons together with tetrahedrons
can fill space.  So you can stack 4 truncated tetrahedra
on top of each other, leaving a hole in the shape of
an imaginary 5th one. Then use tetrahedra to fill up
some gaps. This was basically the shape I built outof
cardboard. Then, I spent rather a long time trying to
tile this with heptagons. A clue to a solution was
that a triangulation of the surface I made had 120
triangle, and 120 = 24\times 5. What is so good about 5?
Well, 5 triangles stuck together have 7 outer sides,
sop they are a kind of pseudo heptagons. Anyway, I
got a bit frustrated, and did not find a nice tiling.

As I was trying to figure it out, I found this site:
<A HREF = "http://www.math.uni-siegen.de/wills/klein/">http://www.math.uni-siegen.de/wills/klein/</A>
It has some nice pictures.


\begin{verbatim}

  > These are like the driving directions the devil gives people who ask
  > the way out of hell.  LRLRLRLR and you're right back where you started.
\end{verbatim}
    
Btw, this works the same on other polyhedra, e.g. the cube.


\begin{verbatim}

  > Saturn  (Saturday)
  > Sun     (Sunday)
  > Moon    (Monday)
  > Mars    (Tuesday)
  > Mercury (Wednesday)
  > Jupiter (Thursday)
  > Venus   (Friday)
\end{verbatim}
    

My French is not so good, but in French some names look more
convincing:


\begin{verbatim}

    tuesday = mardi (Mars?)
    wednesday = mercredi (Mercury?)
    thursday = jeudi (Jove?)
    friday = vendredi (Venus?)
\end{verbatim}
    

Gerard
\end{quote}

I replied:

\begin{quote}
Gerard Westendorp wrote:


\begin{verbatim}

 > John Baez wrote:

 >  > It's called "{7,3}", since it's made of 7-sided figures with 3
 >  > meeting at each corner.

 >  > Can we cut out a portion of this tiling and curl it up to get
 >  > a torus?  No!  But, we can curl up a portion to get a 3-holed
 >  > torus - like the surface of a doughnut with three holes.  But,
 >  > we can only do this if we use precisely 24 heptagons!

 >If you take Euler's formula [....]

 >So indeed, there are no solutions for 0 holes (sphere),
 >1 hole( torus). But a 2-holed torus should be possible,
 >as well as the 24-faced 3-holed one.
\end{verbatim}
    

I was going to talk about this, but I figured my article
was getting too long.  

Indeed, Euler's formula also allows the possibility of a 
\emph{2-holed} torus tiled with 12 heptagons meeting 3 at each corner.  

But this does not prove such a tiling is possible.  I don't know 
if it is!  Someone should try it.

However: even if such a tiling exists, it's not possible for each 
rotational symmetry of each heptagon to extend to a symmetry 
of the whole tiled surface.   What's marvelous about the 3-holed
case is that they all do - at least if you do things correctly.
This is what makes the Klein quartic a full-fledged "Platonic 
surface".

If you look here:

<UL>
<LI>
Hermann Karcher and Mattias Weber, The Geometry of Klein's Riemann Surface,
in The Eightfold Way: the Beauty of Klein's Quartic Curve, 
ed. Silvio Levy, MSRI Research Publications 35, Cambridge U. Press, 
Cambridge 1999. Also available as
<A HREF = "http://www.msri.org/publications/books/Book35/files/karcher.pdf">
PDF</A> and
<A HREF = 
"http://www.msri.org/publications/books/Book35/files/karcher.ps.gz">gzipped
Postscript</A>.
</UL>

you'll see that Karcher and Weber study Platonic surfaces using Euler's 
formula.

On pages 13-19 they consider Platonic surfaces with 2 holes.
On page 19 they give a clever proof that no tiling of the 2-holed torus 
by heptagons meeting 3 at each corner can be a Platonic surface. 
The proof is so clever that I don't understand it.

(Warning: their article starts on page 9.)  


\begin{verbatim}

 >Anyway, see if I can visualise the 3-holed one.
\end{verbatim}
    

I wish I could visualize it myself.


\begin{verbatim}

 >As I was trying to figure it out, I found this site:
 ><A HREF = "http://www.math.uni-siegen.de/wills/klein/">http://www.math.uni-siegen.de/wills/klein/</A>
 >It has some nice pictures.
\end{verbatim}
    

These pictures are interesting, but what I'd really like
is a nice picture of a 3-holed torus, not weird or crumpled up, 
which is tiled by 24 heptagons just like the Klein quartic.
 
The heptagons can't all be regular if the torus is embedded in
R^{3}, since there's no way to embed a compact surface of constant
negative curvature in R^{3}.  
But, you \emph{can} get the \emph{topology}
correct and still have the torus embedded in R^{3}.

If anyone draws such a picture, and I think it looks nice, I'd love
to put it on This Week's Finds!

If anyone wants instructions on how such a surface should be made, 
look above, where Mike Stay has kindly drawn numbers from 1-24 on a 
portion of the hyperbolic plane tiled with heptagons.  These numbers 
indicate how to identify heptagons to get the Klein quartic.  For example, 
all the heptagons labelled "21" are really the same 
heptagon in the Klein quartic.


\begin{verbatim}

 >  > Saturn  (Saturday)
 >  > Sun     (Sunday)
 >  > Moon    (Monday)
 >  > Mars    (Tuesday)
 >  > Mercury (Wednesday)
 >  > Jupiter (Thursday)
 >  > Venus   (Friday)

 >My French is not so good, but in French some names look more
 >convincing:
 >
 >    tuesday = mardi (Mars?)
 >    wednesday = mercredi (Mercury?)
 >    thursday = jeudi (Jove?)
 >    friday = vendredi (Venus?)
\end{verbatim}
    

Yes, this because most of the English names for planets come from 
Latin, and French is more like Latin.  

English is more complicated, but I'm so used to it that I forgot 
people might find the connection to Latin mysterious:

<UL>
<LI>
"Tuesday" comes from 
"Tiu" or "Tyr", an ancient Germanic god of war whom 
the Romans identified with Mars.  We see traces of this in the German 
"Dienstag" as well.

<LI>
"Wednesday" comes from "Woden" or 
"Odin", a Germanic god whom the Romans
identified with Mercury.  Modern German uses "Mittwoch" 
instead, which means "mid-week".
 
<LI>
"Thursday" comes from "Thor", 
a Germanic thunder god whom the Romans 
identified with Jupiter.  Modern German uses "Donnerstag" instead,
with "Donner" meaning "thunder".

<LI>
"Friday" comes from "Freya" or 
"Frigga", a Germanic goddess of married love whom
the Romans identified with Venus.  The German "Freitag" 
is very similar.
</UL>

\end{quote}


\par\noindent\rule{\textwidth}{0.4pt}
% </A>
% </A>
% </A>
