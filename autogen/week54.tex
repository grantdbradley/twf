
% </A>
% </A>
% </A>
\week{June 2, 1995 }

I just got back from a quantum gravity conference in Warsaw, and I'm dying 
to talk about some of the stuff I heard there, but first I should describe 
some work on topology and higher-dimensional algebra that I have been 
meaning to discuss for some time now.

1) Timothy Porter, Abstract homotopy theory: the interaction of category 
theory and homotopy theory, lectures from the school on "Categories and 
Topology", Department of Mathematics, Universita di Genova, report #199, 
March 1992.

Timothy Porter is another expert on higher-dimensional algebra whom I met in 
Bangor, Wales, where he teaches.  As paper 3) below makes clear, he is 
very interested in the relationship between traditional themes in 
topology and the new-fangled topological quantum field theories (TQFTs) 
people have been coming up with these days.  The above paper does not 
mention TQFTs; instead, it is an overview of various approaches that 
people have used to study homotopy theory in an algebraic way.  But
anyone seriously interested in the intersection of physics and topology 
would do well to get ahold of it, since it's a pleasant way to get 
acquainted with some of the beautiful techniques algebraic topologists 
have been developing, which many physicists are just starting to 
catch up with.

What's homotopy theory?  Well, roughly, it's the study of the properties of 
spaces that are preserved by a wide class of stretchings and squashings, 
called "homotopies".  

For example, a closed disc D and a one-point set {p} are quite
different as topological spaces, in that there is no continuous map
from one to the other having a continuous inverse.  (This is obvious
because they have a different number of points!)  But there is clearly
something similar about them, because you can squash a disc down to a
point without crushing any holes in the process (since the disc has no
holes).  To formalize this, note that we can find continuous functions
f: D \to  {p} 
and 
g: {p} \to  D 
that are inverses "up to
homotopy".  For example, let f be the only possible function from D to
{p}, taking every point in D to p, and let g be the map that sends p
to the point 0, where we think of D as the unit disc in the plane.
Now if we first do g and then do f we are back where we started from,
so gf is the identity on {p}.  But if we first do f and then g we are
NOT necessarily back where we started from: instead, the function fg
takes every point in D to the point 0 in D.  So fg is not the
identity.  But it is "homotopic" to the identity, by which I mean that
there is a continuously varying family of continuous functions
F_{t} from D to itself, such that
F_{0} = fg 
and F_{1} is the identity on D.  Simply let F_{t} be
scalar multiplication by t!  As t goes from 1 to 0, we see that
F_{t} squashes the disc down to a point.

A bit more precisely, and more generally too, if we have two
topological spaces X and Y we say that two continuous functions f,g: X
\to  Y are homotopic if there is a continuous function 
F: [0,1] \times  X \to  Y 
such that 
F(0,x) = f(x) 
and 
F(1,x) = g(x).
Intuitively, this means that f can be "continuously deformed" into
g.  Then we say that two spaces X and Y are homotopic if there are
continuous functions f: X \to  Y, g: Y \to  X which are inverse up
to homotopy, i.e., such that gf and fg are homotopic to the identity
on X and Y, respectively.

The main goal in homotopy theory is to understand when functions are
homotopic and when spaces are homotopic.  This is incredibly hard in
\emph{general}, but in special cases a huge amount is known.  To take a
random (but important) example, people know that all maps from the
sphere to the circle are homotopic.  Remember that algebraists call the
sphere S^{2} since its surface is 2-dimensional, and call the
circle S^{1}; in general the unit sphere in R^{n+1} is
called S^{n}.  So for short, one says that all maps from
S^{2} to S^{1} are homotopic.  But: there are infinitely
many different nonhomotopic maps from S^{3} to S^{2}!
In fact there is a nice way to label all these "homotopy classes" of
maps by integers.  And then: there are only two homotopy classes of maps
from S^{4} to S^{3}.  There are also only two homotopy
classes of maps from S^{5} to S^{4}, and from
S^{6} to S^{5}, and so on.
 
Now, the famous topologist J. H. C. Whitehead put forth an important 
program in 1950, as follows: "The ultimate aim of \emph{algebraic homotopy} 
is to construct a purely algebraic theory, which is equivalent to 
homotopy theory in the same way that `analytic' is equivalent to `pure' 
projective geometry."  Since then a lot of people have approached 
this program from various angles, and Porter's paper tours some of the
key ideas involved.  


Part of the reason for pursuing this program is simply to get good at
computing things, in a manner similar to how analytic geometry helps
you solve problems in "pure" geometry.  This is not my main
interest; if I want to know how many homotopy classes of maps there
are from S^{9} to S^{6}, or something, I know where to
look it up, or whom to ask - which is infinitely more efficient than
trying to figure it out myself!  And indeed, there is a formidable
collection of tools out there for solving various sorts of specific
homotopy-theoretic problems, not all of which rely crucially on a
\emph{general} purely algebraic theory of homotopy.

I'm more interested in this program for another reason, which is
simply to find an algebraic language for talking about things being
true "up to homotopy".  As I've tried to explain in recent "weeks",
there are many situations where equations should be replaced by some
weaker form of equivalence.  Taking this seriously leads naturally to
the study of n-categories, in which equations between j-morphisms can
be replaced by specified (j+1)-morphisms.  But Porter describes a host
of different (though related) formalisms set up to handle this sort of
issue.  A few of the main ones are: simplicial sets, simplicial
objects in more general categories, Kan complexes, Quillen's "model
categories", Cat^{n} groups, and homotopy coherent diagrams.
Understanding how all these formalisms are related and what they are
good for is quite a job, but this paper helps one get started.

So far everything I've actually said is quite elementary - I've made
reference to some impressive buzzwords without explaining them, but
that doesn't count.  So I should put in something for the folks who
want more!  Let me say a word or two about Cat^{n} groups.
The definition of these is a typical mind-blowing piece of
higher-dimensional algebra, so I can't resist explaining it.  (After a
while these definitions stop seeming so mind-boggling, and then one is
presumably beginning understand the point of the subject!)  In "<A
HREF = "week53.html">week53</A>" I gave a definition of a category
using category theory.  This might seem completely circular and
useless, but of course I was illustrating quite generally how one
could define a "model" of a "finite limit theory" using category
theory.  The idea was that a category is a \emph{set} of objects, a
\emph{set} of morphisms, together with various \emph{functions}
like the source and target functions which assign to any morphism (or
"arrow") its source and target (or "tail" and "tip").  These sets and
functions needed to satisfy various axioms, of course.

Now \emph{sets} and \emph{functions} are the objects and morphisms
in the category of sets, which folks call Set.  So in "<A HREF =
"week53.html">week53</A>" I cooked up a little category Th called "the
theory of categories", which has objects called "ob" and "mor",
morphisms called "s" and "t", etc..  These were completely abstract
gizmos, not actual sets and functions.  But we required them to
satisfy the exact same laws that the sets of objects and morphisms,
and the source and target functions, and so on, satisfy in an actual
category.  Then a functor from Th to Set which preserves finite limits
is called a "model" of the theory of categories, because it assigns to
the completely abstract gizmos actual sets and functions satisfying
the same laws.  In other words, if we have a functor 
F: Th \to  Set,
we have an actual set F(ob) of objects, an actual set F(mor) of
morphisms, an actual function F(s) from F(ob) to F(mor), and so on.
In short, we have an actual category!


Now to get this trick to work we didn't need much to be true about the
category Set: all we needed was that it had finite limits.  (Ignore this
technical stuff about limits if you don't get it; you can still get the
basic idea here.)  And there are lots of categories with this property,
like the category Grp of groups.  So we can also talk about a model of
the theory of categories in the category of groups!  What is this?
Well, it's just a functor from Th to Grp that preserves finite limits.
More concretely, it's exactly like a category, except everywhere in the
definition of category where you see the word "set", scratch that out
and write in "group", and everywhere you see the word "function",
scratch that out and write in "homomorphism".  So you have a
\emph{group} of objects, a \emph{group} of morphisms, together with
various \emph{homomorphisms} like the source and target, and so
on... satisfying laws perfectly analogous to those in the definition of
a category!

Folks call this kind of thing a "categorical group", a "category
object in Grp" or an "internal category in Grp".  From the point of
view of sheer audacity alone, it's a wonderful concept: we have taken
the definition of a category and transplanted it from the soil in
which it was born, namely the category Set, into new soil, namely the
category Grp!  But more remarkably still, the study of these
"categorical groups" is equivalent to the study of "homotopy 2-types"
- that is, topological spaces, but where you only care about them up
to homotopy, and even more drastically, where nothing above dimension
2 concerns you.  This result is due (as far as I can tell) to Ronnie
Brown and C. B. Spencer, building on earlier work of Mac Lane and
Whitehead.

But why stop here?  Consider the category Cat(Grp) of these
category objects in Grp.  Take my word for it, such a thing exists and
it has finite limits.  That means we can look for models of the theory
of categories in Cat(Grp) - i.e., functors from Th to Cat(Grp),
preserving finite limits.  In fact, \emph{there} things form a
category, say Cat^{2}(Grp), and \emph{this} category has
finite limits, so we can look for models of the theory of categories
in \emph{this} category, and \emph{these} form a category
Cat^{3}(Grp), which also has finite limits... etc.  So we can
construct an insanely recursive hierarchy:

\begin{verbatim}
groups
category objects in the the category of groups
category objects in the category of (category objects in the category of groups)
etc....

\end{verbatim}
    
Now, truly wonderfully, L. Loday showed that the study of
Cat^{n}(Grp) is 
equivalent (in a certain precise sense) to the study of homotopy 
n-types - i.e., homotopy theory where you don't care about phenomena
above dimension n:

2) L. Loday, Spaces with finitely many non-trivial homotopy groups, Jour. 
Pure Appl. Algebra 24 (1982), 179-202.

Subsequently, Ronnie Brown, Loday and others have done some interesting 
topology using this fact.  But the most remarkable thing, in a way, is how 
taking a perfectly basic concept, the concept of GROUP, and then doing 
category theory "internally" in the category of groups in an iterated 
fashion, winds up being very much the same as doing topology - or at least 
homotopy theory.  This suggests that there is something deeply algebraic 
about homotopy theory in the first place.

3) Timothy Porter, Interpretations of Yetter's notion of G-coloring: 
simplicial fibre bundles and non-abelian cohomology, available at
<a href = "http://citeseer.ist.psu.edu/100965.html">http://citeseer.ist.psu.edu/100965.html</a>

Physicists know and love the Dijkgraaf-Witten model, a 2+1-dimensional TQFT 
that depends on a finite group G.  It's easy to compute the Hilbert space of 
states for any compact oriented 2-manifold in this TQFT.  Just triangulate 
your 2-manifold and let your Hilbert space have as a basis the set of all 
possible ways of labelling the edges with elements of G such that 
g_{1} g_{2} g_{3} = 1 whenever we have 3 edges going counterclockwise
around any triangle.  Yetter figured out how to generalize this to get an 
interesting TQFT from any finite categorical group:

4)  David N. Yetter, Topological quantum field theories associated to 
finite groups and crossed G-sets, Journal of Knot Theory and its 
Ramifications 1 (1992), 1-20.

TQFTs from homotopy 2-types, Journal of Knot Theory and its Ramifications 2 
(1993), 113-123.

This should be the beginning of some bigger pattern relating homotopy theory
and TQFTs.   Jim Dolan and I have our own theories as to how this pattern 
should work (see "<A HREF = "week49.html">week49</A>") but they are still a rather long ways from being 
theorems.  Porter, who is an expert in simplicial methods, makes the 
relationship (or ONE of the relationships) very clear in this special case.

5) Justin Roberts, Skein theory and Turaev-Viro invariants, preprint.

Refined state-sum invariants of 3- and 4-manifolds, preprint.

Skeins and mapping class groups, Math. Proc. Camb. Phil. Soc. 115 (1994),  
53-77.

G. Masbaum and Justin Roberts, On central extensions of mapping class groups,
Mathematica Gottingensis, Schriftenreihe des Sonderforschungsbereichs Geometrie
und Analysis, Heft 42 (1993).

The first two papers here might be the most interesting for physicists.  
They both deal with 3d and 4d TQFTs constructed using quantum SU(2): 
in particular, the Turaev-Viro theory in dimension 3, and the 
Crane-Yetter-Broda theory in dimension 4.  The first theory is interesting 
physically because it corresponds to 3d Euclidean quantum gravity with 
cosmological constant.  The second theory is interesting mainly because 
it's one of the few 4d TQFTs for which the Atiyah axioms have been shown; 
sometime I will explain the Lagrangian for this theory, which seems not to 
be well-known.   

In Roberts' first paper, which was already discussed in "<A HREF =
"week14.html">week14</A>", he gave a simple proof that the partition
function for the Turaev-Viro theory was the absolute value squared of
that for Chern-Simons theory, perhaps the most famous of TQFTs.  He also
showed that the partition function for the Crane-Yetter-Broda theory was
a function of the signature and Euler characteristic (classical
invariants of 4-manifolds).  In the second paper, he considers
observables for the TV and CYB theories depending on cohomology classes
in the manifold.  I wish I had energy now to explain a bit more about
these observables, since they are very interesting, but I don't!

6) Lawrence Breen, On the Classification of 2-Gerbes and 2-Stacks, 
Asterisque 225, 1994.

Suffice it to say that if gerbes and stacks - which are, very roughly, 
sort of like sheaves of categories - are too simple to interest
you, it's time to read about 2-gerbes and 2-stacks - which are 
roughly like sheaves of 2-categories.  


\par\noindent\rule{\textwidth}{0.4pt}
% </A>
% </A>
% </A>


% parser failed at source line 380
