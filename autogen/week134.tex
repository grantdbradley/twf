
% </A>
% </A>
% </A>
\week{June 8, 1999 }


My production of "This Week's Finds" has slowed to a trickle
as I've been struggling to write up a bunch of papers.  Deadlines,
deadlines!  I hate deadlines, but when you write things for other
people, or with other people, that's what you get into.  I'll do my best
to avoid them in the future.  Now I'm done with my chores and I want to
have some fun.

I spent last weekend with a bunch of people talking about quantum 
gravity in a hunting lodge by a lake in Minnowbrook, New York:

1) Minnowbrook Symposium on Space-Time Structure, program and
transparencies of talks available at
<A HREF = "http://www.phy.syr.edu/research/he_theory/minnowbrook/#PROGRAM">http://www.phy.syr.edu/research/he_theory/minnowbrook/#PROGRAM</A>

The idea of this get-together, organized by Kameshwar Wali and some other
physicists at Syracuse University, was to bring together people working
on string theory, loop quantum gravity, noncommutative geometry, and
various discrete approaches to spacetime.  People from these different
schools of thought don't talk to each other as much as they should, so
this was a good idea.  People gave lots of talks, asked lots of tough
questions, argued, and learned what each other were doing.  But I came
away with a sense that we're quite far from understanding quantum
gravity: every approach has obvious flaws.  

One big problem with string theory is that people only know how to study
it on a spacetime with a fixed background metric.  Even worse, things
are poorly understood except when the metric is static - that is, roughly
speaking, when geometry of space does not change with the passage of time.  


For example, people understand a lot about string theory on spacetimes
that are the product of Minkowski spacetime and a fixed Calabi-Yau
manifold.  There are lots of Calabi-Yau manifolds, organized in
continuous multi-parameter families called moduli spaces.  This suggests
the idea that the geometry of the Calabi-Yau manifold could change with
time.  This idea is lurking behind a lot of interesting work.  For
example, Brian Greene gave a nice talk on "mirror symmetry".
Different Calabi-Yau manifolds sometimes give the same physics; these
are called "mirror manifolds".  Because of this, a curve in
one moduli space of Calabi-Yau manifolds can be physically equivalent to
a curve in some other moduli space, which sometimes lets you continue
the curve beyond a singularity in the first moduli space.  Physicists
like to think of these curves as representing spacetime geometries where
the Calabi-Yau manifold changes with time.  The problem is, there's no
fully worked out version of string theory that allows for a
time-dependent Calabi-Yau manifold!

There's a good reason for this: one shouldn't expect anything so simple
to make sense, except in the "adiabatic approximation" where
things change very slowly with time.  The product of Minkowski spacetime
with a fixed Calabi-Yau manifold is a solution of the 10-dimensional
Einstein equations, and this is part of why this kind of spacetime
serves as a good background for string theory.  But we do not get a
solution if the geometry of the Calabi-Yau manifold varies from point to
point in Minkowski spacetime - except in the adiabatic approximation.

There are also problems with "unitarity" in string theory when
the geometry of space changes with time.  This is already familiar from
ordinary quantum field theory on curved spacetime.  In quantum field
theory, people usually like to describe time evolution using unitary
operators on a Hilbert space of states.  But this approach breaks down
when the geometry of space changes with time.  People have studied this
problem in detail, and there seems to be no completely satisfactory way
to get around it.  No way, that is, except the radical step of ceasing
to treat the geometry of spacetime as a fixed "background".
In other words: stop doing quantum field theory on spacetime with a
pre-established metric, and invent a background-free theory of quantum
gravity!  But this is not so easy - see "<A HREF =
"week132.html">week132</A>" for more on what it would entail.

Apparently this issue is coming to the attention of string theorists now
that they are trying to study their theory on non-static background
metrics, such as anti-de Sitter spacetime.  Indeed, someone at the
conference said that a bunch of top string theorists recently got
together to hammer out a strategy for where string theory should go
next, but they got completely stuck due to this problem.   I think
this is good: it means string theorists are starting to take the
foundational issues of quantum gravity more seriously.  These issues
are deep and difficult.

However, lest I seem to be picking on string theory unduly, I should
immediately add that all the other approaches have equally serious
flaws.  For example, loop quantum gravity is wonderfully
background-free, but so far it is almost solely a theory of kinematics,
rather than dynamics.  In other words, it provides a description of the
geometry of \emph{space} at the quantum level, but says little about
\emph{spacetime}.  Recently people have begun to study dynamics with the help
of "spin foams", but we still can't compute anything well
enough to be sure we're on the right track.  So, pessimistically
speaking, it's possible that the background-free quality of loop quantum
gravity has only been achieved by simplifying assumptions that will
later prevent us from understanding dynamics.

Alain Connes expressed this worry during Abhay Ashtekar's talk, as did
Arthur Jaffe afterwards.  Technically speaking, the main issue is that
loop quantum gravity assumes that unsmeared Wilson loops are sensible
observables at the kinematical level, while in other theories, like
Yang-Mills theory, one always needs to smear the Wilson loops.  Of
course these other theories aren't background-free, so loop quantum
gravity probably \emph{should} be different.  But until we know that loop
quantum gravity really gives gravity (or some fancier theory like 
supergravity) in the large-scale limit, we can't be sure it should be
different in this particular way.  It's a legitimate worry... but only
time will tell!

I could continue listing approaches and their flaws, including Connes'
own approach using noncommutative geometry, but let me stop here.  The
only really good news is that different approaches have \emph{different}
flaws.  Thus, by comparing them, one might learn something!  

Some more papers have come out recently which delve into the 
philosophical aspects of this muddle: 


 2) Carlo Rovelli, Quantum spacetime: what do we know?, to appear in 
Physics Meets Philosophy at the Planck Scale, eds. Craig Callender and Nick
Huggett, Cambridge U. Press.  Preprint available as <A HREF =
"http://xxx.lanl.gov/abs/gr-qc/9903045">gr-qc/9903045</A>.

3) J. Butterfield and C. J. Isham, Spacetime and the philosophical 
challenge of quantum gravity, to appear in Physics Meets Philosophy 
at the Planck Scale, eds. Craig Callender and Nick Huggett, Cambridge 
U. Press.  Preprint available as 
<A HREF = "http://xxx.lanl.gov/abs/gr-qc/9903072">gr-qc/9903072</A>.

Rovelli's paper is a bit sketchy, but it outlines ideas which I find
very appealing - I always find him to be very clear-headed about the
conceptual issues of quantum gravity.  I found the latter paper a bit
frustrating, because it lays out a wide variety of possible positions
regarding quantum gravity, but doesn't make a commitment to any one of
them.  However, this is probably good when one is writing to an audience
of philosophers: one should explain the problems instead of trying to
sell them on a particular claimed solution, because the proposed
solutions come and go rather rapidly, while the problems remain.  Let me
quote the abstract:

\begin{quote}
We survey some philosophical aspects of the search for a quantum theory
of gravity, emphasising how quantum gravity throws into doubt the
treatment of spacetime common to the two `ingredient theories' (quantum
theory and general relativity), as a 4-dimensional manifold equipped
with a Lorentzian metric.  After an introduction, we briefly review the
conceptual problems of the ingredient theories and introduce the
enterprise of quantum gravity.  We then describe how three main research
programmes in quantum gravity treat four topics of particular
importance: the scope of standard quantum theory; the nature of
spacetime; spacetime diffeomorphisms, and the so-called problem of time.
By and large, these programmes accept most of the ingredient theories'
treatment of spacetime, albeit with a metric with some type of quantum
nature; but they also suggest that the treatment has fundamental
limitations.  This prompts the idea of going further: either by
quantizing structures other than the metric, such as the topology; or by
regarding such structures as phenomenological.  We discuss this in
Section 5.
\end{quote}

Now let me mention a few more technical papers that have come out in
the last few months:

4) John Baez and John Barrett, The quantum tetrahedron in 3
and 4 dimensions, preprint available as 
<A HREF = "http://xxx.lanl.gov/abs/gr-qc/gr-qc/9903060">gr-qc/9903060</A>.

The idea here is to form a classical phase whose points represent
geometries of a tetrahedron in 3 or 4 dimensions, and then apply
geometric quantization to obtain a Hilbert space of states.  These
Hilbert spaces play an important role in spin foam models of quantum
gravity.  The main goal of the paper is to explain why the quantum
tetrahedron has fewer degrees of freedom in 4 dimensions than in 3
dimensions.  Let me quote from the introduction:

\begin{quote}
State sum models for quantum field theories are constructed by giving
amplitudes for the simplexes in a triangulated manifold. The simplexes
are labelled with data from some discrete set, and the amplitudes depend
on this labelling.  The amplitudes are then summed over this set of
labellings, to give a discrete version of a path integral. When the
discrete set is a finite set, then the sum always exists, so this
procedure provides a bona fide definition of the path integral.

State sum models for quantum gravity have been proposed based on the Lie
algebra so(3) and its q-deformation.  Part of the labelling scheme
is then to assign irreducible representations of this Lie algebra to
simplexes of the appropriate dimension.  Using the q-deformation, the
set of irreducible representations becomes finite.  However, we will
consider the undeformed case here as the geometry is more elementary.

Irreducible representations of so(3) are indexed by a non-negative
half-integers j called spins.  The spins have different interpretations
in different models.  In the Ponzano-Regge model of 3-dimensional
quantum gravity, spins label the edges of a triangulated 3-manifold, and
are interpreted as the quantized lengths of these edges.  In the
Ooguri-Crane-Yetter state sum model,  spins label triangles of a
triangulated 4-manifold, and the spin is interpreted as the norm of a
component of the B-field in a BF  Lagrangian.  There is also a state sum
model of 4-dimensional quantum gravity in which spins label triangles. 
Here the spins are interpreted as areas.  

Many of these constructions have a topologically dual formulation.  The
dual 1-skeleton of a triangulated surface is a trivalent graph, each of
whose edges intersect exactly one edge in the original triangulation. 
The spin labels can be thought of as labelling the edges of this graph, 
thus defining a spin network.  In the Ponzano-Regge model, transition
amplitudes between spin networks can be computed as a sum over
labellings of faces of the dual 2-skeleton of a triangulated 3-manifold.
Formulated this way, we call the theory a `spin foam model'.

A similar dual picture exists for 4-dimensional quantum gravity.  The
dual 1-skeleton of a triangulated 3-manifold is a 4-valent graph each of
whose edges intersect one triangle in the original triangulation.  The
labels on the triangles in the 3-manifold can thus be thought of as
labelling the edges of this graph.  The graph is then called a
`relativistic spin network'.  Transition amplitudes between relativistic
spin networks can be computed using a spin foam model.  The path
integral is then a sum over labellings of faces of a 2-complex
interpolating between two relativistic spin networks.

In this paper we consider the nature of the quantized geometry of a
tetrahedron which occurs in some of these models, and its relation to
the phase space of geometries of a classical tetrahedron in 3 or 4
dimensions.  Our main goal is to solve the following puzzle: why does
the quantum tetrahedron have fewer degrees of freedom in 4 dimensions
than in 3 dimensions?  This seeming paradox turns out to have a simple
explanation in terms of geometric quantization. The picture we develop 
is that the four face areas of a quantum tetrahedron in four dimensions 
can be freely specified, but that the remaining parameters cannot, due
to the uncertainty principle.
\end{quote}

Naively one would expect the quantum tetrahedron to have the same number
of degrees of freedom in 3 and 4 dimensions (since one is considering
tetrahedra mod rotations).  However, quantum mechanics is funny about
these things!  For example, the Hilbert space of two spin-1/2 particles
whose angular momenta point in opposite directions is smaller than the
Hilbert space of a single spin-1/2 particle, even though classically you
might think both systems have the same number of degrees of freedom.  
In fact a very similar thing happens for the quantum tetrahedron in 3
and 4 dimensions.

5) Abhay Ashtekar, Alejandro Corichi and Kirill Krasnov, Isolated horizons:
the classical phase space, preprint available as <A HREF =
"http://xxx.lanl.gov/abs/gr-qc/9905089">gr-qc/9905089</A>.


 This paper explains in more detail the classical aspects of the
calculation of the entropy of a black hole in loop quantum gravity (see
"<A HREF = "week112.html">week112</A>" for a description of
this calculation).  Let me quote the abstract:

\begin{quote}
A Hamiltonian framework is introduced to encompass non-rotating (but
possibly charged) black holes that are "isolated" near future
time-like infinity or for a finite time interval.  The underlying
space-times need not admit a stationary Killing field even in a
neighborhood of the horizon; rather, the physical assumption is that
neither matter fields nor gravitational radiation fall across the
portion of the horizon under consideration.  A precise notion of
non-rotating isolated horizons is formulated to capture these ideas.
With these boundary conditions, the gravitational action fails to be
differentiable unless a boundary term is added at the horizon.  The
required term turns out to be precisely the Chern-Simons action for the
self-dual connection.  The resulting symplectic structure also acquires,
in addition to the usual volume piece, a surface term which is the
Chern-Simons symplectic structure.  We show that these modifications
affect in subtle but important ways the standard discussion of
constraints, gauge and dynamics.  In companion papers, this framework
serves as the point of departure for quantization, a statistical
mechanical calculation of black hole entropy and a derivation of laws of
black hole mechanics, generalized to isolated horizons.  It may also have
applications in classical general relativity, particularly in the
investigation of analytic issues that arise in the numerical studies
of black hole collisions.
\end{quote}

The following are some review articles on spin networks, spin
foams and the like:


 6) Roberto De Pietri, Canonical "loop" quantum gravity and spin
foam models, to appear in the proceedings of the XXIIIth Congress of the
Italian Society for General Relativity and Gravitational Physics
(SIGRAV), 1998, preprint available as <A HREF =
"http://xxx.lanl.gov/abs/gr-qc/9903076">gr-qc/9903076</A>.


 7) Seth Major, A spin network primer, to appear in Amer. Jour. Phys.,
preprint available as <A HREF =
"http://xxx.lanl.gov/abs/gr-qc/9905020">gr-qc/9905020</A>.

8) Seth Major, Operators for quantized directions, preprint available
as <A HREF = "http://xxx.lanl.gov/abs/gr-qc/9905019">gr-qc/9905019</A>.

9) John Baez, An introduction to spin foam models of BF theory and quantum
gravity, in Geometry and Quantum Physics, eds. Helmut Gausterer and
Harald Grosse, Lecture Notes in Physics, Springer-Verlag, Berlin, 2000,
pp. 25-93.  Preprint available as <A HREF =
"http://xxx.lanl.gov/abs/gr-qc/9905087">gr-qc/9905087</A>.

By the way, Barrett and Crane have come out with a paper sketching a
spin foam model for Lorentzian (as opposed to Riemannian) quantum
gravity:

10) John Barrett and Louis Crane, A Lorentzian signature model for quantum
general relativity, preprint available as <A HREF =
"http://xxx.lanl.gov/abs/gr-qc/9904025">gr-qc/9904025</A>.

However, this model is so far purely formal, because it involves
infinite sums that probably diverge.  We need to keep working on this!
Now that I'm getting a bit of free time, I want to tackle this issue.
Meanwhile, Iwasaki has come out with an alternative spin foam model of
Riemannian quantum gravity:

11) Junichi Iwasaki, A surface theoretic model of quantum gravity, preprint
available as <A HREF =
"http://xxx.lanl.gov/abs/gr-qc/9903112">gr-qc/9903112</A>.

Alas, I don't really understand this model yet.  Finally, to wrap
things up, something completely different:

12) Richard E. Borcherds, Quantum vertex algebras, preprint available
as <A HREF = "http://xxx.lanl.gov/abs/math.QA/9903038">math.QA/9903038</A>.
 

I like how the abstract of this paper starts: "The purpose of this paper
is to make the theory of vertex algebras trivial".  Good!  Trivial is
not bad, it's good.  Anything one understands is automatically trivial.





 \par\noindent\rule{\textwidth}{0.4pt}

% </A>
% </A>
% </A>
