
% </A>
% </A>
% </A>
\week{December 5, 2009}

A while back, my friend Dan Christensen drew a picture of all the
roots of all the polynomials of degree at most 5 with integer
coefficients ranging from -4 to 4:

<div align = center>
<a href = "http://math.ucr.edu/home/baez/roots/deg5.png">
<img border = "none" width = "600" src = "http://math.ucr.edu/home/baez/roots/deg5.png">
% </a>
</div>

1) Dan Christensen, Plots of roots of polynomials with integer
coefficients, <a href = "http://jdc.math.uwo.ca/roots/">http://jdc.math.uwo.ca/roots/</a>

2) John Baez, The beauty of roots, <a href = "http://math.ucr.edu/home/baez/roots/">http://math.ucr.edu/home/baez/roots/</a>

Click on the picture for bigger view.  Roots of quadratic polynomials
are in grey; roots of cubics are in cyan; roots of quartics are in red
and roots of quintics are in black.  The horizontal axis of symmetry
is the real axis; the vertical axis of symmetry is the imaginary axis.
The big hole in the middle is centered at 0; the next biggest holes
are at &plusmn;1, and there are also holes at &plusmn;i and all the
cube roots of 1.

You can see lots of fascinating patterns here, like how the roots of
polynomials with integer coefficients tend to avoid integers and roots
of unity - except when they land \emph{right on} these points!  You
can see more patterns if you zoom in:

<div align = "center">
<img src= "http://math.ucr.edu/home/baez/roots/deg5_closeup.jpg">
</div>

Now you see beautiful feathers surrounding the blank area
around the point 1 on the real axis, a hexagonal star
around exp(i \pi  / 6), a strange red curve from this point to 
1, smaller stars around other points, and more....

People should study this sort of thing!  Let's define the set
C(d,n) to be the set of all roots of all polynomials of
degree d with integer coefficients ranging from -n to n.  Clearly
C(d,n) gets bigger as we make either d or n bigger.  It
becomes dense in the complex plane as n \to  \infty , as long as
d \ge  1.  We get all the rational complex numbers if we fix d \ge  1
and let n \to  \infty , and all the algebraic complex numbers if let
both d,n \to  \infty .  

But based on the above picture, there seem to be a lot of interesting
conjectures to make about this set as d \to  \infty  for fixed n.

Inspired by the pictures above, Sam Derbyshire decided to to make a
high resolution plot of some roots of polynomials.  After some
experimentation, he decided that his favorite were polynomials whose
coefficients were all 1 or -1 (not 0).  He made a high-resolution plot
by computing all the roots of all polynomials of this sort having
degree 24.  That's 2^{24} polynomials, and about 24 \times 
2^{24} roots &mdash; or about 400 million roots!  It took
Mathematica 4 days to generate the coordinates of the roots, producing
about 5 gigabytes of data.  He then used some Java programs to create
this amazing image:  

<br/><br/>
<div align = center>
<a href = "http://math.ucr.edu/home/baez/roots/polynomialrootssmall.png">
<img border = "none" width = "600" src = "http://math.ucr.edu/home/baez/roots/polynomialrootssmall.png">
% </a>
</div>
<br/><br/>

The coloring shows the density of roots, from black to dark red to
yellow to white.  The picture above is a low-resolution version of the
original image, which is available as a 90-megabyte file on Dan's
website.  We can zoom in to get more detail:

<br/><br/><br/>
<div align = center>
<a href = "http://math.ucr.edu/home/baez/roots/polynomialroots_closeup.jpg">
<img border = "none" width = "600" src = "http://math.ucr.edu/home/baez/roots/polynomialroots_closeup.jpg">
% </a>
</div>

<br/><br/>
Note the holes at certain roots of unity, and wondrously intricate
patterns as we move inside the unit circle.  To make all this clearer,
Sam Derbyshire zoomed in on certain regions, marked here:

<br/><br/><br/>
<div align = center>
<a href = "http://math.ucr.edu/home/baez/roots/polynomialrootscrops.png">
<img border = "none" width = "600" src = "http://math.ucr.edu/home/baez/roots/polynomialrootscrops.png">
% </a>
</div>

<br/><br/><br/>
Here's a closeup of the hole at 1:  

<br/><br/><br/>
<div align = center>
<a href = "http://math.ucr.edu/home/baez/roots/polynomialroots1.png">
<img border = "none" width = "500" src = "http://math.ucr.edu/home/baez/roots/polynomialroots1.png">
% </a>
</div>

<br/><br/><br/>
Note the white line along the real axis.  That's because lots more of
these polynomials have real roots than \emph{nearly} real roots.

Next, here's the hole at i:

<br/><br/><br/>
<div align = center>
<a href = "http://math.ucr.edu/home/baez/roots/polynomialrootsi.png">
<img border = "none" width = "400" src = "http://math.ucr.edu/home/baez/roots/polynomialrootsi.png">
% </a>
</div>

<br/><br/><br/>
And here's the hole at exp(i\pi /4) = (1 + i)/\sqrt 2:

<br/><br/><br/>
<div align = center>
<a href = "http://math.ucr.edu/home/baez/roots/polynomialrootsexpi025p.png">
<img border = "none" width = "400" src = "http://math.ucr.edu/home/baez/roots/polynomialrootsexpi025p.png">
% </a>
</div>

<br/><br/><br/>
Note how the density of roots increases as we get closer to
this point, but then suddenly drops off right next to it.  Note
also the subtle patterns in the density of roots.

But the feathery structures as move inside the unit circle are
even more beautiful!  Here is what they look near the real axis &mdash;
this plot is centered at the point 4/5:

<br/><br/><br/>
<div align = center>
<a href = "http://math.ucr.edu/home/baez/roots/polynomialroots08.png">
<img border = "none" width = "500" src = "http://math.ucr.edu/home/baez/roots/polynomialroots08.png">
% </a>
</div>

<br/><br/><br/>
They have a very different character near the point (4/5)i:

<br/><br/><br/>
<div align = center>
<a href = "http://math.ucr.edu/home/baez/roots/polynomialroots08i.png">
<img border = "none" width = "600" src = "http://math.ucr.edu/home/baez/roots/polynomialroots08i.png">
% </a>
</div>

<br/><br/><br/> But I think my favorite is the region near the point
(1/2)exp(i/5).  This image is almost a metaphor of how mathematical
patterns emerge from confusion like sharply defined figures looming
from the mist:

<br/><br/><br/>
<div align = center>
<a href = "http://math.ucr.edu/home/baez/roots/polynomialroots05expi02.png">
<img border = "none" width = "750" src = "http://math.ucr.edu/home/baez/roots/polynomialroots05expi02.png">
% </a>
</div>

<br/><br/><br/>
Dan and Sam were not the first to explore these issues, but there's
a lot left to do: conjectures to make, theorems to prove, and 
pictures to draw!  If you come up with some pretty pictures, I'd
love to include them on my webpage - and cite you.   For previous 
research, see:

3) Loki Joergenson, Zeros of polynomials with constrained coefficients,
<a href = "http://www.cecm.sfu.ca/~loki/Projects/Roots/">http://www.cecm.sfu.ca/~loki/Projects/Roots/</a>

4) Eric W. Weisstein, MathWorld, Polynomial roots, 
<a href = "http://mathworld.wolfram.com/PolynomialRoots.html">http://mathworld.wolfram.com/PolynomialRoots.html</a>

My colleague the knot theorist Xiao-Song Lin - sadly no longer with us
- plotted the zeros of the Jones polynomial for prime alternating
knots with up to 13 crossings, and you can see his pictures here:

5) Xiao-Song Lin, Zeros of the Jones polynomial,
<a href = "http://math.ucr.edu/~xl/abs-jk.pdf">http://math.ucr.edu/~xl/abs-jk.pdf</a>

You'll see that \emph{some} of the patterns in his pictures just come from
the patterns we see in the roots of polynomials with integer
coefficients... since the Jones polynomial has integer coefficients.

This paper is also interesting:

6) Andrew M. Odlyzko and B. Poonen, Zeros of polynomials with 0,1
coefficients, L'Enseignement Math. 39 (1993), 317-348. Also available
at <a href = "http://dx.doi.org/10.5169/seals-60430">http://dx.doi.org/10.5169/seals-60430</a>

Odlyzko and Poonen proved some interesting things about the set of all
roots of all polynomials with coefficients 0 or 1.  If we define the
set C(d,p,q) to be the set of roots of all polynomials of degree d 
with coefficients ranging from p to q, Odlyzko and Poonen are 
studying C(d,0,1) in the limit d \to  \infty .  They mention some 
known results and prove some new ones: this set is contained in 
the half-plane Re(z) < 3/2 and contained in the annulus 

1/\Phi  < |z| < \Phi 

where \Phi  is the golden ratio, (\sqrt 5 + 1)/2.  In fact they trap it,
not just between these circles, but between two subtler curves.  They
also show that the closure of this set is path connected but not
simply connected.

But from the pictures above, these ideas just scratch the surface
of the wealth of patterns to be found and theorems to be proved!

Next, let me say a bit about the talks from the second day of the 
AMS session on homotopy theory and higher algebraic structures at 
UC Riverside.  You can see videos of these talks here, or by clicking
on the talk titles below:

7) Special session on homotopy theory and higher algebraic structures,
AMS Western Section Meeting, November 7-8, 2009.  Talks available as
Quicktime videos at <a href =
"http://math.ucr.edu/~jbergner/amsriverside09.htm">http://math.ucr.edu/~jbergner/amsriverside09.htm</a>

It's been good for me trying to discuss all these talks - it's forced
me to think about them a lot harder.  I'm not sure how good it is for
you, though: lots of ideas are flashing past without adequate
explanation.  Each talk could be the basis for a whole This Week's
Finds.  But I'm happy to get a chance to at least \emph{mention} all sorts
of ideas that would be fun to explore more deeply someday.

Eric Malm of Stanford University started the show bright and early
8:30 am on Sunday with a talk on 
"<a href = "ams_2009/Malm_AMS_UCR_2009.mov">String topology and the based loop
space</a>".  You can see the slides here:

8) Eric Malm, String topology and the based loop space,
<a href = "http://math.ucr.edu/~jbergner/ucr-st-present.pdf">http://math.ucr.edu/~jbergner/ucr-st-present.pdf</a>

In its original form, string topology studies the \emph{unbased} loop space
of an oriented d-dimensional manifold M.  This is the space LM of all
maps from a circle into M.  In their fundamental paper on the subject:

9) Moira Chas and Dennis Sullivan, String topology, available
as <a href = "http://arXiv.org/abs/arXiv:math/9911159">arXiv:math/9911159</A>.

Chas and Sullivan showed that the homology groups of LM with degrees
shifted by d:

A_{i} = H_{i+d}(LM)

are equipped with a graded-commutative product:

o: A_{i} \otimes  A_{j} \to  A_{i+j}

together with an operator

D: A_{i} \to  A_{i+1}

with D^{2} = 0.  These satisfy a bunch of equations, which
makes the degree-shifted homology of LM into a gadget called a
"Gerstenhaber algebra".  I explained such gadgets in
"<a href = "week220.html">week220</A>": they're precisely
algebras of the homology of the little 2-cubes operad.

But the homology of the loop space has even more structure: it's a
"Batalin-Vilkovisky algebra".  That means that in addition
to the above stuff, it has a Lie bracket of degree 1:

[&middot;,&middot;]: A_{i} \otimes  A_{j} \to  A_{i+j+1}

which gets along with the rest in a nice way.  I also talked about
these in "<a href = "week220.html">week220</A>": they're
precisely algebras of homology of the framed little 2-discs operad!

This is just the beginning of a big story.  Malm's talk surveys this
story and adapts the ideas of string topology to the \emph{based} loop 
space of a manifold, using its relations to Hochschild homology.
For some useful background here, try this book - or at least the
very informative review:

10) Ralph L. Cohen, Kathryn Hess, and Alexander A. Voronov, String
Topology and Cyclic Homology, Birkhauser, Boston, 2006.  Review by
Janko Latschev at <a href = "http://www.ams.org/bull/0000-000-00/S0273-0979-09-01265-8/">http://www.ams.org/bull/0000-000-00/S0273-0979-09-01265-8/</a>

Next, Laura Scull of Fort Lewis College spoke on 
"<a href = "ams_2009/Scull_AMS_UCR_2009.mov">Orbifolds 
and equivariant homotopy
theory</a>".  This is joint work she's doing with Dorette Pronk.
Laura is an expert on equivariant homotopy theory: that's the kind of
homotopy theory you do for spaces on which a group acts.  Dorette is
an expert on category theory.  So it was natural for them to team up
and tackle orbifolds!

Why?  And what's an orbifold?  Well, just as a manifold is built up
from patches that look like R^{n}, an orbifold is built up
from patches that look like R^{n} modulo the linear action of
a finite group.  So, most places it looks like a manifold, but it can
have singularities of a mild sort here and there.

When people tried to define maps of orbifolds, they ran into a lot of
trouble.  Naive approaches led to a mess.  It turns out there's a good
reason for this.  There's not really a nice category of orbifolds.
But there's a nice \emph{2-category} of orbifolds!

The reason is that we shouldn't think of an orbifold as a set with
extra structure.  We should think of it as a \emph{groupoid} with extra
structure.  The points of the orbifold are the objects of this
groupoid.  For a plain old manifold, we'd only have identity
morphisms - so it's basically just a set.  But for a more interesting
orbifold, the singular points have extra automorphisms.

Everything likes to live in something bigger and fancier than itself.
Groupoids, being categories, want to be objects of a 2-category.  The
same is true for orbifolds.  However, there are extra subtleties due
to the \emph{smooth structure} on our orbifold.  To deal with these,
it's nice to treat orbifolds as "Lie groupoids" or
"stacks".  I could say a lot more, but instead I'll just
refer you to this very readable paper:

11) Eugene Lerman, Orbifolds as stacks?  Available as <a href =
"http://arxiv.org/abs/0806.4160">arXiv:0806.4160</a>.

It begins by discussing easy approaches, then describes their flaws,
and so on, leading up to the current state of the art.  After this
warmup, try:

12) Dorette Pronk and Laura Scull, Translation groupoids and orbifold
Bredon cohomology, <a href =
"http://arxiv.org/abs/0705.3249">arXiv:0705.3249</a>.

Then Anssi Lahtinen of Stanford University spoke about "<a href =
"ams_2009/Lahtinen_AMS_UCR_2009.mov">The Atiyah-Segal completion
theorem in twisted K-theory</a>".

Twisted K-theory is fascinating to folks who like categorification,
because it involves "U(1) gerbes", which are categorified
U(1) bundles.  Just as a U(1) bundle over a space can be defined by
chopping a space into open sets U_{i} and giving U(1)-valued
"transition functions" on the intersections U_{i}
\cap  U_{j}, a U(1) gerbe over a space can be defined by
chopping a space into open sets and giving transition functions

h_{ijk} : U_{i} \cap  U_{j} \cap  U_{k}
\to  U(1)

If you have a U(1) gerbe, you can define "twisted vector bundles".
These are like vector bundles, but where the transition functions
g_{ij} satisfy the usual cocycle conditions only up to a phase, given
by h_{ijk}.  In other words, instead of the famous formula

g_{ij} g_{jk} = g_{ik}

we just have

g_{ij} g_{jk} h_{ijk} = g_{ik}

Given a space X, we can form its K-theory K(X) by taking the category
of vector bundles and forming its Grothendieck group.  We saw some
Grothendieck group constructions last time!  Here's how it goes this
time.  Take the category of vector bundles over X, say Vect(X).  Then
take the set of isomorphism classes of vector bundles.  Then take
formal linear combinations of these to get an abelian group, and then
impose the equivalence relation

[M \oplus  N] = [M] + [N]

The result is an abelian group K(X) called the "K-theory" of
X.  And in fact it's a ring, since we can also take tensor products of
vector bundles!

The Atiyah-Segal completion theorem concerns K(X) when X is the
classifying space of a topological group G.  As explained in "<a
href = "week151.html">week151</A>", this is a space BG with a
principal G-bundle over it:

EG \to  BG

with the property that every other principal G-bundle over every other
space is a pullback of this one.  Given any representation of G, we
can use the "associated bundle" trick to create a vector bundle over
BG.  So, we get a functor from the category of representations of G to
the category of vector bundles over BG:

Rep(G) \to  Vect(BG)

Applying the Grothendieck group construction, this functor in turn
gives a ring homomomorphism

R(G) \to  K(BG) 

where R(G), the so-called "representation ring" of G, is the
Grothendieck group of Rep(G).

The Atiyah-Segal theorem explains how this map from R(G) to K(BG) is
almost - though not quite - an isomorphism.  It's tempting to
generalize this from K-theory to twisted K-theory... and that's what
Anssi Lahtinen spoke about!

Next, Konrad Waldorf of UC Berkeley spoke on "<a href =
"ams_2009/Waldorf_AMS_UCR_2009.mov">String connections and
supersymmetric sigma models</a>":

13) Konrad Waldorf, String connections and supersymmetric sigma models,
<a href = "http://www.konradwaldorf.de/docs/riverside.pdf">http://www.konradwaldorf.de/docs/riverside.pdf</a>

14) Konrad Waldorf, String connections and Chern-Simons theory,
available as <a href =
"http://arxiv.org/abs/0906.0117">arxiv:0906.0117</a>.

His talk was a great introduction to some things I know and love, 
and some others that I'd never quite understood before... but loved at 
first sight now.

U(1) bundles over a space are classified by elements of its second 
cohomology with integer coefficients.  U(1) gerbes are similarly 
classified by the third integral cohomology group.  This story 
keeps on going!  U(1) 2-gerbes are classified by the fourth 
cohomology, and so on.  If you don't know what a 2-gerbe is, don't
panic: just go back to my description of bundles and gerbes, and you
can guess how the story continues.

But when M is a manifold, there's a nice way to get an element of its
fourth integral cohomology group!  If it's an oriented manifold, its
oriented frame bundle is a principal SO(n) bundle.  This has
"characteristic classes"; the first interesting one is the
"first Pontryagin class", which is an element in 
the fourth integral cohomology group of M.  You can get a
representative of this in deRham cohomology by picking a connection,
taking its curvature 2-form F and multiplying the closed 2-form

tr(F ^ F)

by the right number.  But in fact the first Pontryagin class lives in
integral cohomology.  So, any oriented 4-manifold automatically gives 
a 2-gerbe... but that's not the one we need here!

If M is equipped with a spin structure, its oriented frame bundle is
equipped with a double cover that's a principal Spin(n) bundle.  This
too has characteristic classes.  The first interesting one lives
in the fourth integral cohomology group of M, and it has the 
property that when you multiply it by 2 you get the first Pontryagin 
class.  (In integral cohomology there can be various different
classes with this property, coming from different spin structures.)

So: every spin structure on M gives an element of the fourth integral
cohomology group of M, and thus a 2-gerbe.  This is called the
"Chern-Simons 2-gerbe".  The reason for this term is
explained here:

15) Urs Schreiber, States of Chern-Simons theory,
<a href = "http://golem.ph.utexas.edu/category/2008/02/states_of_chernsimons_theory.html">http://golem.ph.utexas.edu/category/2008/02/states_of_chernsimons_theory.html</a>

There are lots of ways to think about "string structures" on a spin
manifold M, but Waldorf advocated thinking of them as <i>choices of
trivialization</i> of its Chern-Simons 2-gerbe.  There may of course be
none, or many.  But the really nice thing about this viewpoint is that it
gives a nice approach to "string connections".  

Next, S&oslash;ren Galatius of Stanford University gave a talk on "<a
href = "ams_2009/Galatius_AMS_UCR_2009.mov">Monoids of moduli spaces
of manifolds</a>", explaining a paper with the same title:

16) S&oslash;ren Galatius and Oscar Randal-Williams, Monoids of
moduli spaces of manifolds, available as <a href = "http://arxiv.org/abs/0905.2855">arXiv:0905.2855</a>.

The goal of their work was to create a title with as many
words beginning with "M" as possible... no, not really.
In fact, it's a kind of continuation of this famous paper:

17) S&oslash;ren Galatius, Ib Madsen, Ulrike Tillmann, and Michael
Weiss, The homotopy type of the cobordism category, available as <a
href = "http://arxiv.org/abs/math/0605249">arXiv:math/0605249</a>.

In item K of "<a href = "week117.html">week117</a>" I explained
how to turn any category into a topological space called its
"classifying space".  This construction has a nice
generalization to "topological categories" - that is,
categories where the set of morphisms from any object to any other is
a topological space, and composition is continuous.  

For example, a topological group G is the same as a topological
category with one object and all morphisms being invertible.  If we
apply the construction to this example, we get the classifying space
BG that I mentioned a while back.

The Galatius-Madsen-Tillmann-Weiss paper determined the homotopy type 
of the classifying space of the topological category of n-dimensional 
oriented cobordisms!  The new work constructs a topological monoid
that has the same classifying space - a nice simplification.   

After lunch, Alissa Crans of Loyola Marymount University spoke on
"<a href = "ams_2009/Crans_AMS_UCR_2009.mov">2-Quandles:
categorified quandles</a>".  A "quandle" is the sort of
algebraic gadget when you axiomatize the properties of conjugation in
a group.  So, start with a group and define an operation of "left
conjugation":

g > h = g h g^{-1}

and an operation of "right conjugation":

h < g = g^{-1} h g

Then, figure out all the equations these obey, regardless of what
group you've got!  Clearly these operations are inverses of
each other:

g > (h < g) = h = (g > h) < g

If you conjugate anything by itself, nothing happens:

g > g = g = g < g

But more interestingly, we also have

g > (h > k) = (g > h) > (g > k)

(k < h) < g = (k < g) < (h < g)

Conjugation distributes over itself!  Do the calculation yourself and
see!  As far as I know, all equations obeyed by these operations
follow from the ones I've listed... though I've never seen a proof,
and I'd like to.  These equations form the definition of a
"quandle".

So, we may define a quandle in a very conceptual way as an algebraic
structure where each element acts as a symmetry of that structure, and
every element acts trivially on itself.  Think about it.

But the magical thing about quandles is that they give invariants of
tangles!  The easiest way to start seeing this is by pondering braids.
Given a quandle Q there's a way to turn any n-strand braid into a
function

Q^{n} \to  Q^{n}

Here's how.  In braids we can have two kinds of crossings:


\begin{verbatim}

 \   /
  \ /
   /
  / \
 /   \
\end{verbatim}
    

and


\begin{verbatim}

 \   /
  \ /
   \
  / \
 /   \
\end{verbatim}
    

Let's think of each as giving a function from Q^{2} to itself.  To
do this, we let the quandle element labelling one strand act on the 
quandle element labelling the other, using our two kinds of 
conjugation:


$$

g     h
 \   /
  \ /
   /
  / \
 /   \
h    g<h
$$
    
and


$$

g     h
 \   /
  \ /
   \
  / \
 /   \
g>h   g
$$
    

The strand above acts on the strand below, following the general
principle that the people higher up cause trouble for the people below
them.  Now, look at the third Reidemeister move, which says:


\begin{verbatim}

|   /    |      |    \   |
 \ /     |      |     \ / 
  \      |      |      \
 / \     |      |     / \
|   \   /        \   /   |
|    \ /          \ /    |
|     \      =     \     |	
|    / \          / \    |
|   /   \        /   \   |
 \ /     |      /     \ / 
  \      |      |      \  
 / \     |      |     / \ 
|   \    |      |    /   |
|    \   |      |   /    |
\end{verbatim}
    

If we feed in three quandle elements on top, look what happens:


$$

g    h   k      g   h    k
|   /    |      |    \   |
 \ /     |      |     \ / 
  \      |      |      \
 / \     |      |  h>k/ \
|   \   /        \   /   |
|    \ /          \ /    |
|g>h  \      =     \     |	
|    / \          / \    |
|   /   \        /   \   |
 \ /g>k  |      /     \ / 
  \      |      |      \  
 / \     |      |     / \ 
|   \    |      |    /   |
|    \   |      |   /    |
*   g>h  g      *  g>h   g
$$
    

Look at the strand marked with an asterisk!  On the left it should be
labelled by

(g > h) > (g > k)

On the right it should be labelled by 

g > (h > k)

But thanks to the self-distributive law, these are equal!  Similarly,
the equation 

g > (h < g) = h = (g > h) < g

handles the second Reidemeister move:


\begin{verbatim}

 \   /      |    |       \   / 
  \ /       |    |        \ /
   /        |    |         \  
  / \       |    |        / \  
 /   \   =  |    |   =   /   \  
 \   /      |    |       \   /
  \ /       |    |        \ /
   \        |    |         / 
  / \       |    |        / \ 
 /   \      |    |       /   \ 
\end{verbatim}
    
while the equation 

g > g = g = g < g

handles the first Reidemeister move.  The first Reidemeister move is
not really about about braids - it's about tangles: 


\begin{verbatim}

|      /\           |         |      /\
|     /  \          |         |     /  \
 \   /    \         |          \   /    \
  \ /      |        |           \ /      |
   \       |   =    |    =       /       |
  / \      |        |           / \      |
 /   \    /         |          /   \    /
|     \  /          |         |     \  / 
|      \/           |         |      \/
|                   |         |    
\end{verbatim}
    

So, there's a deep relation between crossings in tangles and
conjugation in groups, captured by the quandle axioms.  And the
quandle axioms also cover \emph{Lie algebras}, with self-distributivity
corresponding to the Jacobi identity:

16) J. Scott Carter, Alissa Crans, Mohamed Elhamdadi and Masahico
Saito, Cohomology of categorical self-distributivity, available
as <a href = "http://arXiv.org/abs/arXiv:math/0607417">arXiv:math/0607417</A>.

It's possible to explain this relation a lot more deeply than I just
did... but anyway, what Alissa did is start \emph{categorifying} this
relation.  Together with the topologists Carter and Saito, she's
studying "2-quandles", which should relate 2-tangles to
conjugation in 2-groups.

Next, Chad Giusti of the University of Oregon spoke on "<a href =
"ams_2009/Giusti_AMS_UCR_2009.mov">Unstable Vassiliev
theory</a>":
 
17) Chad Giusti, Unstable Vassiliev theory,
<a href = "http://math.ucr.edu/~jbergner/RiversideTalk.pdf">http://math.ucr.edu/~jbergner/RiversideTalk.pdf</a>

The goal here is to understand the space of "long knots".  A
long knot is a curve in R^{3} that goes on forever and is a
vertical straight line outside some compact set.  So, it can get
knotted around in the middle.  One nice thing about long knots is that
there's a multiplication defined on them, by sticking them end-to-end.

If you know about tangles, a long knot is just another way of thinking
about a one-strand tangle with a strand coming in at top and going
out at bottom.  Then the multiplication on long knots is a special 
case of the composition of tangles.

(We can define even more operations if we work with
"thickened" long knots.  In fact, the space of these forms
an algebra of the little 2-cubes operad!  This gives a mystical
relation between thickened long knots and Gerstenhaber algebras.  I
explained this near the end of "<a href =
"week220.html">week220</A>".)

Anyway, the part of Giusti's talk that I understood best, and
therefore liked the most, was a neat combinatorial description
of the space of long knots. He calls them "plumbers' knots",
because they're like pipes that move only along the x, y, or z
directions... for details, see his slides!

Then Robin Koytcheff of Stanford University gave a somewhat related
talk on "<a href = "ams_2009/Koytcheff_AMS_UCR_2009.mov">A homotopy-theoretic view of Bott-Taubes integrals and knot
spaces</a>":

18) Robin Koytcheff, A homotopy-theoretic view of Bott-Taubes integrals
and knot spaces, <a href = "http://math.ucr.edu/~jbergner/RKslidesUCR.pdf">http://math.ucr.edu/~jbergner/RKslidesUCR.pdf</a>

He began with a nice introduction to the Bott-Taubes approach to
Vassiliev theory.  Then he gave a great description of how the
little 2-cubes operad acts on the space of thickened long knots, 
and how one can use this to underand the homology of this space.
Then he discussed how to combine these ideas.  For more details, see:

19) Robin Koytcheff, A homotopy-theoretic view of Bott-Taubes integrals
and knot spaces, Alg. Geom. Top. 9 (2009), 1467-1501.  Also available as
<a href = "http://arxiv.org/abs/0810.1785">arXiv:0810.1785</a>.

Next, Chris Douglas of U.C. Berkeley gave talk charmingly entitled
"3-categories for the working mathematician" - 
unfortunately no video for this one.  It's great to see how weak
3-categories are making their way into applications.  Douglas is
working with Arthur Bartels and Andre Henriques on their applications
to "conformal nets" - that is, algebras of local observables
in conformal field theory.  The bulk of Douglas' talk involved a kind
of hieroglyphic notation for operations and equations in a definition
of weak 3-category.  This definition is close to the existing
definitions of "tricategory", but not exactly the same - at
least, not superficially.  It's probably equivalent.

Finally, <a href = "ams_2009/Morrison_AMS_UCR_2009.mov">Scott
Morrison</a> and <a href =
"ams_2009/KWalker_AMS_UCR_2009.mov">Kevin Walker</a> gave a
2-part talk on "blob homology" - a great introduction to
their big paper in progress:

20) Scott Morrison and Kevin Walker, Blob homology slides:
<a href = "http://tqft.net/UCR-blobs1">http://tqft.net/UCR-blobs1</a>
and 
<a href = "http://tqft.net/UCR-blobs2">http://tqft.net/UCR-blobs2</a>

21) Scott Morrison and Kevin Walker, The blob complex.  Draft
available at <a href = "http://tqft.net/papers/blobs.pdf">http://tqft.net/papers/blobs.pdf</a>

The clever idea here is to use manifolds to provide a quick and
practical definition of "n-categories with duals" - thus
short-circuiting, at least temporarily, the need to prove some big
conjectures linking this algebraic concept to topology.  With this
definition, they're able to define and study "blob
homology": that is, a kind of homology for manifolds with
coefficients in a linear n-category with duals!

This includes ordinary TQFTs and also Hochschild homology as special
cases.  So, it's a big deal, and I'm sure we'll be seeing more of it
in the years to come.

Next week I'll start a series of This Week's Finds on rational
homotopy theory.  This is a great subject with connections to 
pretty much everything: deformation theory, Lie n-groups and Lie
n-algebras, classical mechanics, supergravity and more!  So stay
tuned....

\par\noindent\rule{\textwidth}{0.4pt}
\textbf{Addenda:} I thank Toby Bartels for some improvements, and
Ralf Bader for a link to Odlyzko and Poonen's paper.

For more discussion visit the <a href =
"http://golem.ph.utexas.edu/category/2009/12/this_weeks_finds_in_mathematic_46.html">\emph{n}-Category
Caf&eacute;</a>.

\par\noindent\rule{\textwidth}{0.4pt}
<em>The author feels that this technique of deliberately lying will
actually make it easier for you to learn the ideas.</em> - Donald Knuth

\par\noindent\rule{\textwidth}{0.4pt}

% </A>
% </A>
% </A>
