
% </A>
% </A>
% </A>
\week{November 20, 2006 }

I've been working too hard, and running around too much, to write 
This Week's Finds for a while.  A bunch of stuff has built up 
that I want to explain.  Luckily I've been running around explaining 
stuff - higher gauge theory, and tales of the dodecahedron.

This weekend I went to Baton Rouge.  I was invited to Louisiana 
State University by Jorge Pullin of loop quantum gravity fame, and 
I used the opportunity to get a look at LIGO - the Laser 
Interferometry Gravitational-Wave Observatory!

I described this amazing experiment back in "<A HREF =
"week189.html">week189</A>", so I won't rehash all that.  Suffice
it to say that there are two installations: one in Hanford Washington,
and one in Livingston Louisiana.  Each consists of two evacuated tubes
4 kilometers long, arranged in an L shape - here's the one in Livingston:


<div align = center>
<img src = "http://touro.ligo-la.caltech.edu/%7Ebonnie/publish/aerials/aerials-Images/3.jpg">
</div>

Laser beams bounce back
and forth between mirrors suspended at the ends of the tubes, looking
for tiny changes in their distance that would indicate a gravitational
wave passing through, stretching or squashing space.  And when I say
"tiny", I mean smaller than the radius of a proton!  This is
serious stuff.

Jorge drove me in his SUV to Livingston, a tiny town about 20 minutes 
from Baton Rouge.  While he runs the gravity program at Louisiana 
State University, which has links to LIGO, he isn't officially part 
the LIGO team.  His wife is.  When I first met Gabriela Gonzalez, she 
was studying the Brownian motion of torsion pendulums.  The mirrors in
LIGO are hung on pendulums made of quartz wire, to minimize the effect 
of vibrations.  But, the random jittering of atoms due to thermal 
noise still affects these pendulums.  She was studying this noise to 
see its effect on the accuracy of the experiment.  

This was way back when LIGO was just being planned.  Now that LIGO is
a reality, she's doing data analysis, helping search for gravitational
waves produced by pairs of neutron stars and/or black holes as they 
spiral down towards a sudden merger.  Together with an enormous 
pageful of authors, she helped write this paper, based on data taken 
from the "first science run" - the first real LIGO experiment, back 
in 2002:

1) The LIGO Scientific Collaboration, Analysis of LIGO data for
gravitational waves from binary neutron stars, Phys. Rev. D69 (2004),
122001.  Also available at <A HREF =
"http://xxx.lanl.gov/abs/gr-qc/0308069">gr-qc/0308069</A>.

She's one of the folks with an intimate knowledge of the experimental 
setup, who keeps the theorists' feet on the ground while they stare 
up into the sky.

On the drive to Livingston, Jorge pointed out the forests that 
surround the town.  These forests are being logged:

<div align = center>
<img src = "ligo_0.jpg">
</div>

I asked him 
about this - when I last checked, the vibrations from falling trees 
were making it impossible to look for gravitational waves except at 
night!  He said they've added a "hydraulic external pre-isolator" to 
shield the detector from these vibrations - basically a super-duper 
shock absorber:

<div align = center>
<img src = "hepi.jpg">
</div>

Now they can operate LIGO day and night.  

I also asked him how close LIGO had come to the sensitivity levels
they were seeking.  When I wrote "<A HREF =
"week189.html">week189</A>", during the first science run, they
still had a long way to go.  That's why the above paper only sets
upper limits on neutron star collisions within 180 kiloparsecs.  This
only reaches out to the corona of the Milky Way - which includes the
Small and Large Magellanic Clouds.  We don't expect many neutron star
collisions in this vicinity: maybe one every 3 years or so.  The first
science run didn't see any, and the set an upper limit of about 170
per year: the best experimental upper limit so far, but definitely
worth improving, and nowhere near as fun as actually \emph{seeing}
gravitational waves.

But Jorge said the LIGO team has now reached its goals: they should be
able to see collisions out to 15 megaparsecs!  By comparison, the
center of the Virgo cluster is about 20 megaparsecs away.  In fact,
they should already be able to see about half the galaxies in this
cluster.

They're now on their seventh
science run, and they'll keep upping the sensitivity in future
projects called "Enhanced LIGO" and "Advanced
LIGO".  The latter should see neutron star collisions out to 300
megaparsecs:

2) Advanced LIGO, <a href =
"http://www.ligo.caltech.edu/advLIGO/">http://www.ligo.caltech.edu/advLIGO/</a>

When we arrived at the gate, Jorge spoke into the intercom and got 
us let in.  Our guide, Joe Giaime, was running a bit a late, 
so we walked over and looked at the interferometer's arms, each 
of which stretched off beyond sight, 2.5 kilometers of concrete 
tunnel surrounding the evacuated piping - the world's largest vacuum
facility:

<div align = center>
<img src = "ligo_1.jpg">
</div>

Schoolkids have been invited to paint pictures on some of these
pipes:

<div align = center>
<img src = "ligo_2.jpg">
</div>

One can tell this is the South.  The massive construction caused
pools of water to form in the boggy land near the facility, and
these pools then attracted alligators.  These have been dealt with firmly.  
The game hunters who occasionally fired potshots at the facility were 
treated more forgivingly: instead of feeding them to the alligators, 
the LIGO folks threw a big party and invited everyone from the local 
hunting club.  Hospitality works wonders down here.

The place was pretty lonely.  During the week lots of scientists work
here, but this was Saturday, and on weekends there's just a skeleton 
crew of two.  There's usually not much to do now that the experiment 
is up and running.   As Joe Giaime later said, there have been no "Jodie 
Foster moments" like in the movie Contact, where the scientists on 
duty suddenly see a signal, turn on the suspenseful background music, 
and phone the President.   There's just too much data analysis 
required to see any signal in real time: data from both Livingston and 
Hanford is sent to Caltech, and then people grind away at it.  So, 
about the most exciting thing that happens is when the occasional tiny
earthquake throws the laser beam out of phase lock.

When Joe showed up, I got to see the main control room, which 
is dimly lit, full of screens indicating noise and sensitivity
levels of all sorts - and even some video monitors showing the view 
down the laser tube:

<div align = center>
<img src = "ligo_3.jpg">
</div>

This is where the people on duty hang out - you can see those video
monitors on top:

<div align = center>
<img src = "ligo_4.jpg">
</div>

One of them had brought his sons, in a feeble attempt to dispose of
the huge supply of Halloween candy that had somehow collected here.

I also got to see a sample of the 400 "optical baffles" which have 
been installed to absorb light spreading out from the main beam
before it can bounce back in and screw things up.  The interesting
thing is that these baffles and their placement were personally 
designed by Kip Thorne and some other godlike LIGO figure.   Moral: 
unless they've gone soft, even bigshot physicists like to actually 
think about physics now and then, not just manage enormous teams.

But overall, there was surprisingly little to see, since the innermost 
workings are all sealed off, in vacuum.  The optics are far more
complicated than my description - "a laser bouncing between two 
suspended mirrors" - could possibly suggest.  But, all I got to
see was a chart showing how they work:

<div align = center>
<img src = "ligo_5.jpg">
</div>

Oh well.  I'm glad I don't need to understand this stuff in detail.  
It was fun to get a peek.

By the way, I wasn't invited to Louisiana just to tour LIGO and eat
<a href = "http://www.crescentcitybeignets.com/beignet.html">beignets</a> 
and <a href = "http://www.servingsushi.com/CatSubCat/CatSubCatDisplay2.asp?p1=9&p2=Y&p7=2&p8=528&p9=CSC1">alligator sushi</a>.  My real reason for going there was to 
talk about higher gauge theory - a generalization of gauge theory 
which studies the parallel transport not just of point particles, but 
also strings and higher-dimensional objects:

3) John Baez, Higher gauge theory, <a href =
"http://math.ucr.edu/home/baez/highergauge">http://math.ucr.edu/home/baez/highergauge</a>

This is a gentler introduction to higher gauge theory than my previous
talks, some of which I inflicted on you in "<A HREF =
"week235.html">week235</A>".  It explains how BF theory can be
seen as a higher gauge theory, and briefly touches on Urs Schreiber's
work towards exhibiting Chern-Simons theory and 11-dimensional
supergravity as higher gauge theories.  The webpage has links to more
details.

I was also travelling last weekend - I went to Dartmouth and gave this 
talk: 

4) John Baez, Tales of the Dodecahedron: from Pythagoras through Plato
to Poincare, <a href = "http://math.ucr.edu/home/baez/dodecahedron/">http://math.ucr.edu/home/baez/dodecahedron/</a>

It's full of pictures and animations - fun for the whole family!

<div align = center>
<a href = "http://en.wikipedia.org/wiki/Image:Dodecahedron.gif">
<img style = "border:none;" src = "http://math.ucr.edu/home/baez/dodecahedron/Dodecahedron.gif">
% </a>
</div>

I started with the Pythagorean fascination with the pentagram, and how
you can use the pentagram to give a magical picture proof of the
irrationality of the golden ratio.

I then mentioned how Plato used four of the so-called Platonic solids 
to serve as atoms of the four elements - earth, air, water and fire - 
leaving the inconvenient fifth solid, the dodecahedron, to play the
role of the heavenly sphere.  This is what computer scientists call
a "kludge" - an awkard solution to a pressing problem.  Yes, there 
are twelve constellations in the Zodiac - but unfortunately, they're 
arranged quite differently than the faces of the dodecahedron.

This somehow led to the notion of the dodecahedron as an atom of
"aether" or "quintessence" - a fifth element
constituting the heavenly bodies.  If you've ever seen the science
fiction movie The Fifth Element, now you know where the
title came from!  But once upon a time, this idea was quite
respectable.  It shows up as late as Kepler's Mysterium
Cosmographicum, written in 1596.

I then went on to discuss the 120-cell, which gives a way of chopping 
a spherical universe into 120 dodecahedra. 

<div align = center>
<img width = 400 src = "http://math.ucr.edu/home/baez/dodecahedron/Stereographic_polytope_120cell_faces.png">
</div>

This leads naturally to
the Poincare homology sphere, a closely related 3-dimensional manifold 
made by gluing together opposite sides of \emph{one} dodecahedron.  

The Poincare homology sphere was briefly advocated as a model
of the universe that could explain the mysterious weakness of the 
longest-wavelength ripples in the cosmic background radiation -
the ripples that only wiggle a few times as we scan all around the
sky:

5) J.-P. Luminet, J. Weeks, A. Riazuelo, R. Lehoucq, and J.-P.  Uzan,
Dodecahedral space topology as an explanation for weak wide-angle
temperature correlations in the cosmic microwave background, Nature
425 (2003), 593.  Also available as <A HREF =
"http://xxx.lanl.gov/abs/astro-ph/0310253">astro-ph/0310253</A>.

The idea is that if we lived in a Poincare homology sphere, we'd 
see several images of each very distant point in the universe.  So, 
any ripple in the background radiation would repeat some minimum
number of times: the lowest-frequency ripples would be suppressed.

Alas, this charming idea turns out not to fit other data.  We just
don't see the same distant galaxies in several different directions:

6) Neil J. Cornish, David N. Spergel, Glenn D. Starkman and Eiichiro
Komatsu, Constraining the topology of the universe,
Phys. Rev. Lett. 92 (2004) 201302.  Also available as <A HREF =
"http://xxx.lanl.gov/abs/astro-ph/0310233">astro-ph/0310233</A>.

For a good review of this stuff, see:

7) Jeffrey Weeks, The Poincare dodecahedral space and the mystery
of the missing fluctuations, Notices of the AMS 51 (2004), 610-619. 
Also available at <a href = "http://www.ams.org/notices/200406/fea-weeks.pdf">http://www.ams.org/notices/200406/fea-weeks.pdf</a>

In the abstract of my talk, I made the mistake of saying that 
the regular dodecahedron doesn't appear in nature - that instead, 
it was invented by the Pythagoreans.  You should never say things 
like this unless you want to get corrected!

Dan Piponi pointed out this dodecahedral virus:

<div align = center>
<a href = "http://www.nature.com/nsmb/journal/v8/n1/pdf/nsb0101_77.pdf">
<img style = "border:none;" src = "pariacoto_virus.jpg">
% </a>
</div>

8) Liang Tang et al, The structure of Pariacoto virus reveals a 
dodecahedral cage of duplex RNA, Nature Structural Biology 8
(2001), 77-83.  Also available at
<a href = "http://www.nature.com/nsmb/journal/v8/n1/pdf/nsb0101_77.pdf">http://www.nature.com/nsmb/journal/v8/n1/pdf/nsb0101_77.pdf</a>


The first black line is 100 angstroms long (10^{-8} meters),
while the second is 50 angstroms long.

Garett Leskowitz pointed out the molecule "dodecahedrane", with
20 carbons at the vertices of a dodecahedron and 20 hydrogens bonded
to these:

<div align = center>
<a href = "http://en.wikipedia.org/wiki/Image:Dodecahedran.png">
<img style = "border:none;" src = "http://upload.wikimedia.org/wikipedia/commons/thumb/4/4b/Dodecahedran.png/200px-Dodecahedran.png">
% </a>
</div>

9) Wikipedia, Dodecahedrane, <a href = "http://en.wikipedia.org/wiki/Dodecahedrane">http://en.wikipedia.org/wiki/Dodecahedrane</a>

This molecule hasn't been found in nature yet, but chemists can 
synthesize it using reactions like these:

<div align = center>
<a href = "http://www.pubmedcentral.nih.gov/articlerender.fcgi?artid=346698">
<img style = "border:none;" src = "dodecahedrane_synthesis_1.jpg">
% </a>
</div>

<div align = center>
<a href = "http://www.pubmedcentral.nih.gov/articlerender.fcgi?artid=346698">
<img style = "border:none;" src = "dodecahedrane_synthesis_2.jpg">
% </a>
</div>

10) Robert J. Ternansky, Douglas W. Balogh and Leo A. Paquette, 
Dodecahedrane, J. Am. Chem. Soc. 104 (1982), 4503-4504.

11) Leo A. Paquette, Dodecahedrane - the chemical transliteration of 
Plato's universe (a review), Proc. Nat. Acad. Sci. USA 14 part 2 
(1982), 4495-4500.  Also available at
<a href = "http://www.pubmedcentral.nih.gov/articlerender.fcgi?artid=346698">http://www.pubmedcentral.nih.gov/articlerender.fcgi?artid=346698</a>

So, there's probably a bit somewhere in our galaxy.  

Of course, what I \emph{meant} was that people didn't come up with 
regular dodecahedra after seeing them in nature - that instead, 
the Pythagoreans dreamt them up, possibly after seeing pyrite 
crystals that look sort of similar.  These crystals are called 
"pyritohedra".  

But, even here I made a mistake.   The Pythagoreans seem not to have been 
the first to discover the dodecahedron.  John McKay told me that
stone spheres with Platonic solids carved on them have been found
in Scotland, dating back to around 2000 BC!  

<div align = center>
<a href = "http://arxiv.org/abs/math-ph/0303071">
<img style = "border:none;" src = "blocks.gif">
% </a>
</div>


13) Michael Atiyah and Paul Sutcliffe, Polyhedra in physics, chemistry
and geometry, available as <a href = "http://arxiv.org/abs/math-ph/0303071">math-ph/0303071</a>.

14) Dorothy N. Marshall, Carved stone balls, Proc. Soc. Antiq. 
Scotland, 108 (1976/77), 40-72.  Available at
<a href = "http://ads.ahds.ac.uk/catalogue/library/psas/">http://ads.ahds.ac.uk/catalogue/library/psas/</a>

Indeed, stone balls with geometric patterns on them have been found 
throughout Scotland, and occasionally Ireland and northern England.  
They date from the Late Neolithic to the Early Bronze age: 2500 BC to 
1500 BC.  For comparison, the megaliths at Stonehenge go back to
2500-2100 BC.  

Nobody knows what these stone balls were used for, though the article
by Marshall presents a number of interesting speculations.

The pyritohedron is interesting in itself, so before I turn to some
really fancy math, let me talk a bit about this guy.  Since pyrite is 
fundamentally a cubic crystal, the pyritohedron is basically made out 
of little cubic cells, as shown here:

<div align = center>
<a href = "http://www.uwgb.edu/dutchs/symmetry/isometuc.htm">
<img style = "border:none;" src = "pyritohedron.gif">
% </a>
</div>

12) Steven Dutch, Building isometric crystals with unit cells, 
<a href = "http://www.uwgb.edu/dutchs/symmetry/isometuc.htm">http://www.uwgb.edu/dutchs/symmetry/isometuc.htm</a>

It has 12 pentagonal faces, orthogonal to these vectors:


\begin{verbatim}

   (2,1,0)   (2,-1,0)   (-2,1,0)   (-2,-1,0)
   (1,0,2)   (-1,0,2)   (1,0,-2)   (-1,0,-2)
   (0,2,1)   (0,2,-1)   (0,-2,1)   (0,-2,-1)
\end{verbatim}
    

You can see how this works by going here:

13) mindat.org, Pyrite, <A HREF = "http://www.mindat.org/min-3314.html">http://www.mindat.org/min-3314.html</A>

If your webbrowswer can handle Java, go to this webpage and click on
"Pyrite no. 3" to see a rotating pyritohedron.  Then, while
holding your left mouse button down when the cursor is over the
picture of the pyritohedron, type "m" to see the vectors
listed above.

Why "m"?  These vectors are called "Miller
indices".  In general, Miller indices are
outwards-pointing vectors orthogonal to the faces of a crystal; we can
use them to classify crystals.

The Miller indices for the pyritohedron have a nice property.  If you
think of these 12 vectors as points in space, they're the corners of
three 2\times 1 rectangles: a rectangle in the xy plane, a rectangle in the
xz plane, and a rectangle in the yz plane.  

These points are also corners of an icosahedron!  It's not a regular
icosahedron, though.  It's probably the "pseudoicosahedron" shown
in Steven Dutch's site above:

<div align = center>
<a href = "http://www.uwgb.edu/dutchs/symmetry/isometuc.htm">
<img style = "border:none;" src = "pseudoicosahedron.gif">
% </a>
</div>

Apparently iron pyrite can also form 
a pseudoicosahedron - see "Pyrite No. 7" on 
the mindat.org website above.   Does anyone have actual photos?

To get the corners of a regular icosahedron, we just need to replace 
the number 2 by the golden ratio \Phi  = (\sqrt  5 + 1)/2:

$$
(\Phi ,1,0)   (\Phi ,-1,0)   (-\Phi ,1,0)   (-\Phi ,-1,0)
(1,0,\Phi )   (-1,0,\Phi )   (1,0,-\Phi )   (-1,0,-\Phi )
(0,\Phi ,1)   (0,\Phi ,-1)   (0,-\Phi ,1)   (0,-\Phi ,-1)
$$
    

Now our rectangles are golden rectangles:

<div align = center>
<a href = "http://en.wikipedia.org/wiki/Icosahedron#Cartesian_coordinates">
<img style = "border:none;" src = "Icosahedron-golden-rectangles.png">
% </a>
</div>  

Since the pseudoicosahedron does a cheap imitation of this trick,
with the number 2 replacing the golden ratio, 
the number 2 deserves to be called the "fool's golden ratio".
I thank Carl Brannen for explaining this out to me!

The regular docahedron is "dual" to the regular icosahedron: 
the vertices of the icosahedron are Miller indices for the
dodecahedron.  Similarly, I bet the pyritohedron is dual to the 
pseudoicosahedron.  

So, we could call the pyritohedron the "fool's
dodecahedron", and the pseudoicosahedron the "fool's
icosahedron".  Fool's gold may have fooled the Greeks into
inventing the regular dodecahedron, by giving them an example of a
fool's dodecahedron.

As pointed out by Noam Elkies and James Dolan, there is a sequence
of less and less foolish dodecahedra whose faces have normal vectors


\begin{verbatim}

   (B,A,0)   (B,-A,0)   (-B,A,0)   (-B,-A,0)
   (A,0,B)   (-A,0,B)   (A,0,-B)   (-A,0,-B)
   (0,B,A)   (0,B,-A)   (0,-B,A)   (0,-B,-A)
\end{verbatim}
    

where A and B are the nth and (n+1)st Fibonacci numbers, respectively.  
As n \to  \infty , these dodecahedra approach a regular dodecahedron
in shape, because the ratio of successive Fibonacci numbers approaches 
the golden ratio.

When A = 1 and B = 2, we get the fool's dodecahedron, since only a 
fool would think 2/1 is the golden ratio.  

However, this is not the most foolish of all dodecahedra!   The 
case A = 1 and B = 1 gives the rhombic dodecahedron, which doesn't
even have pentagonal faces:

<div align = center>
<a href = "http://en.wikipedia.org/wiki/Rhombic_dodecahedron">
<img style = "border:none;" src = "180px-Rhombicdodecahedron.gif">
% </a>
</div>  


14) Wikipedia, Rhombic dodecahedron, 
<a href = "http://en.wikipedia.org/wiki/Rhombic_dodecahedron">http://en.wikipedia.org/wiki/Rhombic_dodecahedron</a>

So, the rhombic dodecahedron deserves to be called the "moron's 
dodecahedron" - at least for people who think it's actually a 
regular dodecahedron.

But actually, even this dodecahedron isn't the dumbest.  The 
Fibonacci numbers start with 0:

0, 1, 1, 2, 3, 5, 8, 13, 21, 34, ...

So, even more foolish is the case A = 0 and B = 1.  Here our
12 vectors reduce to just 6 different ones:


\begin{verbatim}

   (1,0,0)   (1,-0,0)   (-1,0,0)   (-1,-0,0)
   (0,0,1)   (-0,0,1)   (0,0,-1)   (-0,0,-1)
   (0,1,0)   (0,1,-0)   (0,-1,0)   (0,-1,-0)
\end{verbatim}
    

These are normal to the faces of a cube.  So, the cube 
deserves to be called the "half-wit's dodecahedron": it 
doesn't even have 12 faces, just 6.

Moving in the direction of increasing wisdom, we can consider 
the case A = 2, B = 3.  This gives a dodecahedron which is
closer to regular than the pyritohedron.  And, apparently it
exists in nature!  It shows up as number 12 in this list of crystals:

15) Ian O. Angell and Moreton Moore, Projections of cubic crystals,
section 4: The diagrams, 
<a href = "http://www.iucr.org/education/pamphlets/12/full-text">http://www.iucr.org/education/pamphlets/12/full-text</a>

They also call this guy a pyritohedron, so presumably some 
pyrite forms these less foolish crystals!  You can compare it
with the A = 1, B = 2 case here:

16) Ian O. Angell and Moreton Moore, Projections of cubic crystals,
Graphical index of figures, 
<a href = "http://www.iucr.org/education/pamphlets/12/graphical-index">http://www.iucr.org/education/pamphlets/12/graphical-index</a>

The A = 1, B = 2 pyritohedron is <a href = "http://www.iucr.org/education/pamphlets/12/graphical-index/figures/individual?show=14180">figure 9</a>, while the A = 2, B = 3
pyritohedron is <a href = "http://www.iucr.org/education/pamphlets/12/graphical-index/figures/individual?show=14144">figure 12</a>.  It's noticeably better!

Let me wrap up by mentioning a fancier aspect of the dodecahedron
which has been intriguing me lately.  I already mentioned it in "<A
HREF = "week230.html">week230</A>", but in such a general setting
that it may have whizzed by too fast.  Let's slow down a bit and enjoy
it.

The rotational symmetries of the dodecahedron form a 60-element 
subgroup of the rotation group SO(3).  So, the "double cover" of 
the rotational symmetry group of the dodecahedron is a 120-element 
subgroup of SU(2).  This is called the "binary dodecahedral group".
Let's call it G.  

The group SU(2) is topologically a 3-sphere, so G acts as left 
translations on this 3-sphere, and we can use a dodecahedron sitting 
in the 3-sphere as a fundamental domain for this action.  This gives 
the 120-cell.  The quotient SU(2)/G is the Poincare homology sphere!  

But, we can also think of G as acting on C^{2}.  The quotient C^{2}/G
is not smooth: it has an isolated singular coming from the origin 
in C^{2}.  But as I mentioned in "<A HREF = "week230.html">week230</A>", we can form a "minimal 
resolution" of this singularity.  This gives a holomorphic map

p: M \to  C^{2}/G 

where M is a complex manifold.   If we look at the points in M
that map to the origin in C^{2}/G, we get a union of 8 Riemann spheres, 
which intersect each other in this pattern:


\begin{verbatim}

   /\   /\   /\   /\   /\   /\   /\ 
  /  \ /  \ /  \ /  \ /  \ /  \ /  \
 /    \    \    \    \    \    \    \
/    / \  / \  / \  / \  / \  / \    \ 
\    \ /  \ /  \ /  \ /  \ /  \ /    / 
 \    \    \    \    \ /\ \    \    /
  \  / \  / \  / \  / \  \ \  / \  /
   \/   \/   \/   \/ / \/ \ \/   \/
                    /      \ 
                    \      /
                     \    /
                      \  /
                       \/ 
\end{verbatim}
    
Here I've drawn linked circles to stand for these intersecting 
spheres, for a reason soon to be clear.  But, already you can 
see that we've got 8 spheres corresponding to the dots in this
diagram:



\begin{verbatim}

   o----o----o----o----o----o----o
                       |
                       |
                       o
\end{verbatim}
    
where the spheres intersect when there's an edge between the
corresponding dots.  And, this diagram is the Dynkin diagram for 
the exceptional Lie group E_{8}! 

I already mentioned the relation between the E_{8} Dynkin
diagram and the Poincare homology sphere in "<A HREF =
"week164.html">week164</A>", but now maybe it fits better into a
big framework.  First, we see that if we take the unit ball in
C^{2}, and see what points it gives in C^{2}/G, and
then take the inverse image of these under

p: M \to  C^{2}/G, 

we get a 4-manifold whose boundary is the Poincare homology 
3-sphere.  So, we have a cobordism from the empty set to the 
Poincare homology 3-sphere!  Cobordisms can be described using
"surgery on links", and the link that describes this particular
cobordism is:


\begin{verbatim}

   /\   /\   /\   /\   /\   /\   /\ 
  /  \ /  \ /  \ /  \ /  \ /  \ /  \
 /    \    \    \    \    \    \    \
/    / \  / \  / \  / \  / \  / \    \ 
\    \ /  \ /  \ /  \ /  \ /  \ /    / 
 \    \    \    \    \ /\ \    \    /
  \  / \  / \  / \  / \  \ \  / \  /
   \/   \/   \/   \/ / \/ \ \/   \/
                    /      \ 
                    \      /
                     \    /
                      \  /
                       \/ 
\end{verbatim}
    

Second, by the "McKay correspondence" described in "<A
HREF = "week230.html">week230</A>", all this stuff also works for
other Platonic solids!  Namely:

If G is the "binary octahedral group" - the double cover of the 
rotational symmetry group of the octahedron - then we get a minimal 
resolution

p: M \to  C^{2}/G

which yields, by the same procedure as above, a cobordism from the 
empty set to the 3-manifold SU(2)/G.  

This cobordism can be described using surgery on this link:



\begin{verbatim}

         /\   /\   /\   /\   /\   /\ 
        /  \ /  \ /  \ /  \ /  \ /  \
       /    \    \    \    \    \    \   
      /    / \  / \  / \  / \  / \    \ 
      \    \ /  \ /  \ /  \ /  \ /    / 
       \    \    \    \ /\ \    \    /
        \  / \  / \  / \  \ \  / \  /
         \/   \/   \/ / \/ \ \/   \/
                     /      \ 
                     \      /
                      \    /
                       \  /
                        \/ 
\end{verbatim}
    
which encodes the Dynkin diagram of E_{7}:


\begin{verbatim}

       o----o----o----o----o----o
                      |
                      |
                      o
\end{verbatim}
    

And, if G is the "binary tetrahedral group" - the double
cover of the rotational symmetry group of the tetrahedron - then a
minimal resolution

p: M \to  C^{2}/G

yields, by the same procedure as above, a cobordism from the 
empty set to the 3-manifold SU(2)/G.  This cobordism can be 
described using surgery on this link:



\begin{verbatim}

              /\   /\   /\   /\   /\ 
             /  \ /  \ /  \ /  \ /  \
            /    \    \    \    \    \   
           /    / \  / \  / \  / \    \ 
           \    \ /  \ /  \ /  \ /    / 
            \    \    \ /\ \    \    /
             \  / \  / \  \ \  / \  /
              \/   \/ / \/ \ \/   \/
                     /      \ 
                     \      /
                      \    /
                       \  /
                        \/ 
\end{verbatim}
    
which encodes the Dynkin diagram of E_{6}:



\begin{verbatim}

            o----o----o----o----o
                      |
                      |
                      o

\end{verbatim}
    
I don't fully understand this stuff, that's for sure.  But, I 
want to.  The Platonic solids are still full of mysteries.  

\par\noindent\rule{\textwidth}{0.4pt}

\textbf{Addenda:} Someone with the handle "Dileffante" has
found another nice example of the dodecahedron in nature - and even
in Nature:

\begin{quote}
While perusing a Nature issue I found this short notice on a paper, and I
remembered that in your talk (which I saw online) you mentioned that the
dodecahedron was not found in nature.  Now I see in "week241" 
that there are some things dodecahedral after all, but nevertheless, I 
send this further dodecahedron which was missing there.

Nature commented in issue 7075:


15) The complete Plato, Nature 439 (26 January 2006), 372-373. 

\begin{quote}
According to Plato, the heavenly ether and the classical elements - earth,
air, fire and water - were composed of atoms shaped like polyhedra whose
faces are identical, regular polygons. Such shapes are now known as the
Platonic solids, of which there are five: the tetrahedron, cube, octahedron,
icosahedron and dodecahedron. Microscopic clusters of atoms have already
been identified with all of these shapes except the last.

Now, researchers led by Jos&eacute; Luis Rodr&iacute;guez-L&oacute;pez 
of the Institute for
Scientific and Technological Research of San Luis Potos&eacute; in Mexico and
Miguel Jos&eacute;-Yacam&eacute;n of the University of Texas, Austin, 
complete the set.
They find that clusters of a gold-palladium alloy about two nanometres
across can adopt a dodecahedral shape.
\end{quote}

The article is in:

16) Juan Mart&iacute;n Montejano-Carrizales, Jos&eacute;
Luis Rodr&iacute;guez-L&oacute;pez, 
Umapada Pal, Mario Miki-Yoshida and Miguel Jos&eacute;-Yacam&aacute;n,
The completion of the Platonic atomic polyhedra: the dodecahedron,
Small, 2 (2006), 351-355.


Here's the abstract: 

\begin{quote}
Binary AuPd nanoparticles in the 1-2 nm size range are synthesized. Through
HREM imaging, a dodecahedral atomic growth pattern of five fold axis is
identified in the round shaped (85%) particles. Our results demonstrate the
first experimental evidence of this Platonic atomic solid at this size range
of metallic nanoparticles. Stability of such Platonic structures are
validated through theoretical calculations.
\end{quote}

Either there is some additional value in the construction, or the authors
(and Nature editors) were unaware of dodecahedrane.
\end{quote}

Dodecahedrane is a molecule built from carbon and hydrogen - a bit
different from an "atomic cluster" of the sort discussed
here.  It's a matter of taste whether that's important, but I bet
these gold-palladium nanoparticles occur in nature, while dodecahedrane
seems to be unstable.

My friend Geoffrey Dixon contributed these pictures of Platonic
life forms:

<div align = center>
<img src = "platonic_lifeforms.jpg">
</div>

They look a bit like Ernst Haeckel's pictures from his book
"Kunstformen der Natur" (artforms of nature).

Finally, here's a really important addendum: in March 2009, Lieven le
Bruyn discovered that the ancient Scots did <i>not</i> carve stone
balls to look like Platonic solids!  The whole story is something
between a hoax and a series of misunderstandings:

17) Lieven le Bruyn, The Scottish solids hoax, from his blog
neverendingbooks, March 25, 2009,
<a href = "http://www.neverendingbooks.org/index.php/the-scottish-solids-hoax.html">http://www.neverendingbooks.org/index.php/the-scottish-solids-hoax.html</a>

18) John Baez, Who discovered the icosahedron?, talk at the Special
Session on History and Philosophy of Mathematics, 2009 Fall Western
Section Meeting of the AMS, November 7, 2009.  Available at
<a href = "http://math.ucr.edu/home/baez/icosahedron/">http://math.ucr.edu/home/baez/icosahedron/</a>

You can read a bunch of freewheeling discussions triggered by
this Week's Finds at the <a href = "http://golem.ph.utexas.edu/category/2006/11/this_weeks_finds_in_mathematic_2.html">n-Category Caf&eacute;</a>.

\par\noindent\rule{\textwidth}{0.4pt}
\emph{The essence of mathematics lies in its freedom.} - Georg Cantor

\par\noindent\rule{\textwidth}{0.4pt}

% </A>
% </A>
% </A>
