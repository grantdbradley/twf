
% </A>
% </A>
% </A>
\week{March 29, 2008 }


I'm done with teaching until fall, and now I'll be travelling a lot.
I just got back from Singapore.  It's an incredibly diverse place.  I
actually had to buy a book to understand all the foods!  I'm now
acquainted with the charms of <a href =
"http://en.wikipedia.org/wiki/Appam">appam</a>, <a href =
"http://en.wikipedia.org/wiki/Kaya_toast">kaya toast</a>, and <a href
= "http://pachome1.pacific.net.sg/~ccchia/recipe03.html">babi buah
keluak</a>.  But I didn't get around to trying a <a href =
"http://en.wikipedia.org/wiki/Cendol">chendol</a>, a <a href =
"http://en.wikipedia.org/wiki/Bandung_%28drink%29">bandung</a>, or a
<a href = "http://en.wikipedia.org/wiki/Milo_%28drink%29#Use">Milo
dinosaur</a>, even though they're all available in every <a href =
"http://en.wikipedia.org/wiki/Hawker_centre">hawker center</a>.

Today I'll talk about quantum technology in Singapore, atom 
chips, graphene transistors, nitrogen-vacancy pairs in diamonds, 
a new construction of e_{8}, and a categorification of 
quantum sl(2).

But first - the astronomy pictures of the week!  

First another planetary nebula - the "Southern Ring Nebula": 


<div align = "center">
<a href = "http://heritage.stsci.edu/1998/39/big.html">
<img width = "500" src = "NGC3132.jpg">
% </a>
</div>

1) Hubble Heritage Project, Planetary Nebula NGC 3132,
<a href = "http://heritage.stsci.edu/1998/39/index.html">http://heritage.stsci.edu/1998/39/index.html</a>

This bubble of hot gas is .4 light years in diameter.  You 
can see \emph{two} stars near its center.  The faint one is the 
white dwarf remnant of the star that actually threw off the 
gas forming this nebula.  The gas is expanding outwards at 
about 20 kilometers per second.  The intense ultraviolet 
radiation from the white dwarf is ionizing this gas and making 
it glow.

The Southern Ring Nebula is 2000 light years from us.  Much closer to
home, here's a new shot of the frosty dunes of Mars:

<div align = "center">
<a href = "http://hirise-pds.lpl.arizona.edu/PDS/EXTRAS/RDR/PSP/ORB_007000_007099/PSP_007043_2650/PSP_007043_2650_RGB.NOMAP.browse.jpg">
<img width = "600" src = "mars_barchans_1.jpg">
% </a>
</div>

2) HiRISE (High Resolution Imaging Science Experiment), 
Defrosting polar sand dunes, <a href = "http://hirise.lpl.arizona.edu/PSP_007043_2650">http://hirise.lpl.arizona.edu/PSP_007043_2650</a>

These horn-shaped dunes are called "barchans"; you can read
more about them at "<a href = "week228.html">week228</A>".
The frost is carbon dioxide, evaporating as the springtime sun warms
the north polar region.  Here's another photo, taken in February:

<div align = "center">
<a href = "http://hirise-pds.lpl.arizona.edu/PDS/EXTRAS/RDR/PSP/ORB_007100_007199/PSP_007193_2640/PSP_007193_2640_RGB.NOMAP.browse.jpg">
<img width = "600" src = "mars_barchans_2.jpg">
% </a>
</div>

3) HiRISE (High Resolution Imaging Science Experiment), 
Defrosting northern dunes, <a href = "http://hirise.lpl.arizona.edu/PSP_007193_2640">http://hirise.lpl.arizona.edu/PSP_007193_2640</a>

The dark stuff pouring down the steep slopes reminds me of
water, but they say it's dust!

(If you click on these Mars photos, you'll get some amazing
larger views.)

Meanwhile, down here on Earth, I had some good conversations with 
mathematicians and physicists at the National University of Singapore 
(NUS), and also with Artur Ekert and Valerio Scarani, who work here:

4) Centre for Quantum Technologies, <a href = "http://www.quantumlah.org/">http://www.quantumlah.org/</a>

I like the name "quantumlah".  "Lah" is perhaps
the most famous word in Singlish: you put it at the end of a sentence
for emphasis, to convey "acceptance, understanding, lightness,
jest, and a medley of other positive feelings".  Unfortunately I
didn't get to hear much Singlish during my visit.

The Centre for Quantum Technologies is hosted by NUS but is somewhat
independent.  It reminds me a bit of the Institute for Quantum
Computing - see "<a href = "week235.html">week235</A>" - but
it's smaller, and still getting started.  They hope to take advantage
of the nearby semiconductor fabrication plants, or "fabs",
to build stuff.

They've got theorists and experimentalists.  Being overly theoretical
myself, I asked: what are the most interesting real-life working
devices we're likely to see soon?  Ekert mentioned "quantum
repeaters" - gadgets that boost the power of a beam of entangled
photons while still maintaining quantum coherence, as needed for
long-distance quantum cryptography.  He also mentioned "atom
chips", which use tiny wires embedded in a silicon chip to trap
and manipulate cold atoms on the chip's surface:

5) Atomchip Group, <a href =
"http://www.atomchip.org/">http://www.atomchip.org/</a>

6) Atom Optics Group, Laboratoire Charles Fabry, Atom-chip experiment,
<a href =
"http://atomoptic.iota.u-psud.fr/research/chip/chip.html">http://atomoptic.iota.u-psud.fr/research/chip/chip.html</a>

There's also a nanotech group at NUS:

7) Nanoscience and Nanotechnology Initiative, National 
University of Singapore, <a href = "http://www.nusnni.nus.edu.sg/">http://www.nusnni.nus.edu.sg/</a>

who are doing cool stuff with "graphene" - 
hexagonal sheets of carbon atoms, like individual
layers of a graphite crystal:

<div align = "center">
<a href = "http://en.wikipedia.org/wiki/Graphene">
<img src = "graphene.jpg">
% </a>
</div>

Graphene is closely related to buckyballs (see "<a href =
"week79.html">week79</a>") and polycyclic aromatic hydrocarbons (see
"<a href = "week258.html">week258</a>").

Some researchers believe that graphene
transistors could operate in the terahertz range, about 1000 times
faster than conventional silicon ones.  The reason is that electrons
move much faster through graphene.  Unfortunately the difference in
conductivity between the "on" and "off" states is
less for graphene.  This makes it harder to work with.  People think
they can solve this problem, though:

8) Kevin Bullis, Graphene transistors, Technology Review,
January 28, 2008, <a href = "http://www.technologyreview.com/Nanotech/20119/">http://www.technologyreview.com/Nanotech/20119/</a>

Duncan Graham-Rowe, Better graphene transistors, Technology
Review, March 17, 2008, <a href = "http://www.technologyreview.com/Nanotech/20424/">http://www.technologyreview.com/Nanotech/20424/</a>

Ekert also told me about another idea for carbon-based computers:
"nitrogen-vacancy centers".  These are very elegant
entities.  To understand them, it helps to know a bit about diamonds.
You really just need to know that diamonds are crystals made of
carbon.  But I can't resist saying more, because the geometry of these
crystals is fascinating.

A diamond is made of carbon atoms arranged in tetrahedra, which 
then form a cubical structure, like this:

<div align = "center">
<a href = "http://newton.ex.ac.uk/research/qsystems/people/sque/diamond/structure/">
<img style = "border:none;" src = "diamond-conventional-unit-cell.gif">
% </a>
</div>
9) Steve Sque, Structure of diamond, 
<a href = "http://newton.ex.ac.uk/research/qsystems/people/sque/diamond/structure/">http://newton.ex.ac.uk/research/qsystems/people/sque/diamond/structure/</a>

Here you see 4 tetrahedra of carbon atoms inside a cube.  
Note that there's one carbon at each corner of the cube, and 
also one in the middle of each face.  If that was all, we'd 
have a "face-centered cubic".  But there are also 4 more 
carbons inside the cube - one at the center of each tetrahedron!

If you look really carefully, you can see that the full 
pattern consists of two interpenetrating face-centered 
cubic lattices, one offset relative to the other along the 
cube's main diagonal!

While the math of the diamond crystal is perfectly beautiful, 
nature doesn't always get it quite right.  Sometimes a carbon 
atom will be missing.  In fact, sometimes a cosmic ray will 
knock a carbon out of the lattice!  You can also do it yourself 
with a beam of neutrons or electrons.  The resulting hole is 
called a "vacancy".  If you heat a diamond to about 900 
kelvin, these vacancies start to move around like particles.

Diamonds also have impurities.  The most common is nitrogen, 
which can form up 1% of a diamond.  Nitrogen atoms can take 
the place of carbon atoms in the crystal.  Sometimes these 
nitrogen atoms are isolated, sometimes they come in pairs. 

When a lone nitrogen encounters a vacancy, they stick together!  
We then have a "nitrogen-vacancy center".  It's also common for 
4 nitrogens to surround a vacancy.  Many other combinations are 
also possible - and when we get enough of these nitrogen-vacancy 
combinations around, they form larger structures called 
"platelets".

10) R. Jones and J. P. Goss, Theory of aggregation of nitrogen 
in diamond, in Properties, Growth and Application of Diamond, 
eds. Maria Helena Nazare and A. J. Neves, EMIS Datareviews 
Series, 2001, 127-130.

A nice thing about nitrogen-vacancy centers is that they act 
like spin-1 particles.  In fact, these spins interact very 
little with their environment, thanks to the remarkable properties 
of diamond.  So, they might be a good way to store quantum 
information: they can last 50 microseconds before losing 
coherence, even at room temperature.   If we could couple them
to each other in interesting ways, maybe we could do some 
"spintronics", or even quantum computation:

11) Sankar das Sarma, Spintronics, American Scientist
89 (2001), 516-523.  Also available at 
<a href = "http://www.physics.umd.edu/cmtc/earlier_papers/AmSci.pdf">http://www.physics.umd.edu/cmtc/earlier_papers/AmSci.pdf</a>

Lone nitrogens are even more robust carriers of quantum 
information: their time to decoherence can be as much as a 
millisecond!  The reason is that, unlike nitrogen-vacancy 
centers, lone nitrogens have "dark spins" - their spin 
doesn't interact much with light.  But this can also makes 
them harder to manipulate.  So, it may be easier to use 
nitrogen-vacancy centers.  People are busy studying the options:

12) R. J. Epstein, F. M. Mendoza, Y. K. Kato and D. D. 
Awschalom, Anisotropic interactions of a single spin and 
dark-spin spectroscopy in diamond, Nature Physics 1 (2005), 
94-98.  Also available as <a href = "http://arxiv.org/abs/cond-mat/0507706">arXiv:cond-mat/0507706</a>.

13) Ph. Tamarat et al, The excited state structure of the 
nitrogen-vacancy center in diamond, available as 
<a href = "http://arxiv.org/abs/cond-mat/0610357">arXiv:cond-mat/0610357</a>.

14) R. Hanson, O. Gywat and D. D. Awschalom, Room-temperature 
manipulation and decoherence of a single spin in diamond,
Phys. Rev. B74 (2006) 161203.  Also available as 
<a href = "http://arxiv.org/abs/quant-ph/0608233">arXiv:quant-ph/0608233</a>

But regardless of whether anyone can coax them into quantum
computation, I like diamonds.  Not to own - just to contemplate!  I
told you about the diamond rain on Neptune back in "<a href =
"week160.html">week160</A>".  And in "<a href =
"week193.html">week193</A>", I explained how diamonds are the
closest thing to the E8 lattice you're likely to see in this
3-dimensional world.

The reason is that in any dimension you can define a checkerboard
lattice called D_{n}, consisting of all n-tuples of integers
that sum to an even integer.  Then you can define a set called
D_{n}^{+} by taking two copies of the D_{n}
lattice: the original and another shifted by the vector (1/2,...,1/2).
D_{8}^{+} is the E_{8} lattice, but
D_{3} is the face-centered cubic, and
D_{3}^{+} is the pattern formed by carbons in a
diamond!

In case you're wondering: in math, unlike crystallography,
we reserve the term "lattice" for 
a discrete subgroup of R^{n} that's isomorphic to
Z^{n}.  The set D_{n}^{+} is only closed under 
addition when n is even.  So, the carbons in a diamond don't form a 
lattice in the strict
mathematical sense.  On the other hand, the face-centered cubic 
really is a lattice, the D_{3} lattice - and this is
secretly the same as the A_{3} lattice, familiar from stacking
oranges.  It's one of the densest ways to pack spheres, with a density
of

\pi  /(3\sqrt 2)   &asymp;   0.74

The D_{3}^{+} pattern, on the other hand, has a
density of just

(\pi \sqrt 3)/16   &asymp;   0.34  

This is why ice becomes denser when it melts: it's packed in 
a close relative of the D_{3}^{+} pattern, with
an equally low density.

(Do diamonds become denser when they melt?  Or do they always turn
into graphite when they get hot enough, regardless of the pressure?
Inquiring minds want to know.  These days inquiring minds use search
engines to answer questions like this... but right now I'd rather talk
about E_{8}.)

As you probably noticed, Garrett Lisi stirred up quite a media 
sensation with his attempt to pack all known forces and particles 
into a theory based on the exceptional Lie group E_{8}:

15) Garrett Lisi, An exceptionally simple theory of everything,
available as <a href = "http://arxiv.org/abs/0711.0770">arXiv:0711.0770</a>

Part of his idea was to use Kostant's triality-based description of
E_{8} to explain the three generations of leptons - see
"<a href = "week253.html">week253</A>" for more.
Unfortunately this part of the idea doesn't work, for purely
group-theoretical reasons:

16) Jacques Distler, A little group theory,
<a href = "http://golem.ph.utexas.edu/~distler/blog/archives/001505.html">http://golem.ph.utexas.edu/~distler/blog/archives/001505.html</a>  <br/>
A little more group theory, 
<a href = "http://golem.ph.utexas.edu/~distler/blog/archives/001532.html">http://golem.ph.utexas.edu/~distler/blog/archives/001532.html</a>

There would also be vast problems trying get all the dimensionless
constants in the Standard Model to pop out of such a scheme - or 
to stick them in somehow.

Meanwhile, Kostant has been doing new things with E_{8}.  He's
mainly been using the complex form of E_{8}, while Lisi needs
a noncompact real form to get gravity into the game. So, the
connection between their work is somewhat limited.  Nonetheless,
Kostant enjoys the idea of a theory of everything based on
E_{8}.

He recently gave a talk here at UCR:

17) Bertram Kostant, On some mathematics in Garrett Lisi's
"E_{8} theory of everything", February 12, 2008,
UCR.  Video and lecture notes at
<a href = "http://math.ucr.edu/home/baez/kostant/">http://math.ucr.edu/home/baez/kostant/</a>

He did some amazing things, like chop the 248-dimensional Lie 
algebra of E_{8} into 31 Cartan subalgebras in a nice way, thus 
categorifying the factorization

248 = 8 \times  31

To do this, he used a copy of the 32-element group (Z/2)^{5}
sitting in E_{8}, and the 31 nontrivial characters of this group.

Even more remarkably, this copy of (Z/2)^{5} sits inside a
copy of SL(2,F_{32}) inside E_{8}, and the centralizer
of a certain element of SL(2,F_{32}) is a product of two
copies of the gauge group of the Standard Model!  What this means - if
anything - remains a mystery.

Indeed, pretty much everything about E_{8} seems mysterious to me,
since nobody has exhibited it as the symmetry group of anything
more comprehensible than E_{8} itself.  This paper sheds some
new light this puzzle:

17) Jos&eacute; Miguel Figueroa-O'Farrill, A geometric construction
of the exceptional Lie algebras F_{4} and E_{8}, available as
<a href = "http://arxiv.org/abs/0706.2829">arXiv:0706.2829</a>.

The idea here is to build the Lie algebra of E_{8} using Killing 
spinors on the unit sphere in 16 dimensions. 

Okay - what's a Killing spinor?  

Well, first I need to remind you about Killing vectors.  Given
a Riemannian manifold, a "Killing vector" is a vector field that 
generates a flow that preserves the metric!  A transformation 
that preserves the metric is called an "isometry", and these 
form a Lie group.  Killing vector fields form a Lie algebra 
if we use the ordinary Lie bracket of vector fields, and this 
is the Lie algebra of the group of isometries.

Now, if our manifold has a spin structure, a "Killing spinor" is
a spinor field \psi  such that 

D_{v}\psi  = k v\psi 

for some constant k for every vector field v.  Here D_{v}\psi 
is the covariant derivative of \psi  in the v direction, while
v\psi  is defined using the action of vectors on spinors.
Only the sign of the constant k really matters, since rescaling 
the metric rescales this constant.  

It's a cute equation, but what's the point of it?  Part
of the point is this: the action of vectors on spinors

V \otimes  S \to  S

has a kind of adjoint

S \otimes  S \to  V

This lets us take a pair of spinor fields and form a vector
field.  This is what people mean when they say spinors are 
like the "square root" of vectors.  And, if we do this to 
two \emph{Killing} spinors, we get a \emph{Killing} vector!  You can 
prove this using that cute equation - and that's the main point 
of that equation, as far as I'm concerned. 

Under good conditions, this fact lets us define a "Killing 
superalgebra" which has the Lie algebra of Killing vectors 
as its even part, and the Killing spinors as its odd part.

In this superalgebra, the bracket of two Killing vectors 
is just their ordinary Lie bracket.  The bracket of a Killing 
vector and a Killing spinor is defined using a fairly obvious
notion of the "Lie derivative of a spinor field".  And, the 
bracket of two Killing spinors is defined using the map 

S \otimes  S \to  V

which, as explained, gives a Killing vector. 

Now, you might think our "Killing superalgebra" should be a
Lie superalgebra.  But in some dimensions, the map

S \otimes  S \to  V

is skew-symmetric.  Then our Killing superalgebra has a chance
at being a plain old Lie algebra!  We still need to check 
the Jacobi identity.  And this only works in certain special cases:

If you take S^{7} with its usual round metric, the isometry
group is SO(8), so the Lie algebra of Killing vectors is so(8).
There's an 8-dimensional space of Killing spinors, and the action of
so(8) on this gives the real left-handed spinor representation
S_{8}^{+}.  The Jacobi identity holds, and you 
get a Lie algebra structure on

so(8) \oplus  S_{8}^{+} 

But then, thanks to triality, you knock yourself on the head and say
"I could have had a V_{8}!" After all, up to an
outer automorphism of so(8), the spinor representation
S_{8}^{+} is the same as the 8-dimensional vector
representation V_{8}.  So, your Lie algebra is the same as

so(8) \oplus  V_{8} 

with a certain obvious Lie algebra structure.  This is just so(9).
So, it's nothing exceptional, though you arrived at it by a 
devious route.

If you take S^{8} with its usual round metric, the Lie algebra
of Killing vector fields is so(9).  Now there's a 16-dimensional space
of Killing spinor fields, and the action of so(9) on this gives the
real (non-chiral) spinor representation S_{9}.  The Jacobi
identity holds, and you get a Lie algebra structure on

so(9) \oplus  S_{9}

This gives the exceptional Lie algebra f_{4}!

Finally, if you take S^{15} with its usual round metric, the Lie 
algebra of Killing vector fields is so(16).  Now there's a 
128-dimensional space of Killing spinor fields, and the action of 
so(16) on this gives the left-handed real spinor representation 
S_{16}^{+}.  The Jacobi identity holds, and you 
get a Lie algebra structure on

so(16) \oplus  S_{16}^{+}

This gives the exceptional Lie algebra e_{8}!

In short, what Figueroa-O'Farrill has done is found a nice geometrical
interpretation for some previously known algebraic constructions of
f_{4} and e_{8}.  Unfortunately, he still needs to
verify the Jacobi identity in the same brute-force way.  It would be
nice to find a slicker proof.  But his new interpretation is
suggestive: it raises a lot of new questions.  He lists some of these
at the end of the paper, and mentions a really big one at the
beginning.  Namely: the spheres S^{7}, S^{8} and
S^{15} all show up in the Hopf fibration associated to the
octonionic projective line:

S^{7} \to  S^{15} \to  S^{8}

Does this give a nice relation between so(9), f_{4} and
e_{8}?  Can someone guess what this relation should be?  Maybe
e_{8} is built from so(9) and f_{4} somehow.

I also wonder if there's a Killing superalgebra interpretation
of the Lie algebra constructions

e_{6} = so(10) \oplus  S_{10} \oplus  u(1)

and

e_{7} = so(12) \oplus  S_{12}^{+} \oplus  su(2)

These would need to be trickier, with the u(1) showing up from
the fact that S_{10} is a complex representation, and the su(2)
showing up from the fact that S_{12}^{+} is a quaternionic 
representation.  The algebra is explained here:

18) John Baez, The octonions, section 4.3: the magic square,
available at <a href = "http://math.ucr.edu/home/baez/octonions/node16.html">http://math.ucr.edu/home/baez/octonions/node16.html</a>

A geometrical interepretation would be nice!

Finally - my former student Aaron Lauda has been working with 
Khovanov on categorifying quantum groups, and their work is starting 
to really take off.  I'm just beginning to read his new papers, but 
I can't resist bringing them to your attention:

19) Aaron Lauda, A categorification of quantum sl(2), 
available as <a href = "http://arxiv.org/abs/0803.3652">arXiv:0803.3652</a>.

Aaron Lauda, Categorified quantum sl(2) and equivariant cohomology 
of iterated flag varieties, 
available as <a href = "http://arxiv.org/abs/0803.3848">arXiv:0803.3848</a>.
 
He's got a \emph{2-category} that decategorifies to give the quantized
universal enveloping algebra of sl(2)!   And similarly for all the
irreps of this algebra!

There's more to come, too....

\par\noindent\rule{\textwidth}{0.4pt}
\textbf{Addenda:} Starting this Week, you can see 
more discussion and also <i>questions I'm dying to know
the answers to</i> over at the
<a href = "http://golem.ph.utexas.edu/category/2008/03/this_weeks_finds_in_mathematic_23.html">\emph{n}-Category Caf&eacute;</a>.  Whenever I 
write This Week's Finds, I come up with lots of questions.  If you
can help me with some of these, I'll be really grateful.

Jos&eacute; Figueroa-O'Farrill sent an email saying:

\begin{quote}

About the geometric constructions of exceptional Lie algebras, you are
totally spot on in that what is missing is a more conceptual
understanding of the construction which would render the odd-odd-odd
component of the Jacobi identity 'trivial', as is the case for the
remaining three components.  One satisfactory way to achieve this
would be to understand of what in, say, the 15-sphere is E8 the
automorphisms.  I'm afraid I don't have an answer.

As for E_{6} and E_{7}, there is a similar geometric
construction for E_{6} and one for E_{7} is in the
works as part of a paper with Hannu Rajaniemi, who was a student of
mine.  The construction is analogous, but for one thing.  One has to
construct more than just the Killing vectors out of the Killing
spinors: in the case of E_{6}, it is enough to construct a
Killing 0-form (i.e., a constant) which then acts on the Killing
spinors via a multiple of the Dirac operator.  (This is consistent
with the action of 'special Killing forms' a.k.a. 'Killing-Yano
tensors' on spinors.)  The odd-odd-odd Jacobi identity here is even
more mysterious: it does not simply follow from representation theory
(i.e., absence of invariants in the relevant representation where the
'jacobator' lives), but follows from an explicit calculation.  The
case of E_{7} should work in a similar way, but we still have
not finished the construction.  (Hannu has a real job now and I've
been busy with other projects of a less 'recreational' nature.)  In

20) Jos&eacute; Figueroa-O'Farrill, A geometrical construction of exceptional
Lie algebras, talk at Leeds, February 13, 2008, available at <a href = "http://www.maths.ed.ac.uk/~jmf/CV/Seminars/Leeds.pdf">http://www.maths.ed.ac.uk/~jmf/CV/Seminars/Leeds.pdf</a>

you'll find the PDF version of a Keynote file I used for a geometry
seminar I gave recently on this topic in Leeds.

This geometric construction has its origin, as does the notion of
Killing spinor itself, in the early supergravity literature.  Much of
the early literature on supergravity backgrounds was concerned with
the so-called Freund-Rubin backgrounds: product geometries L \times  R, with
L a lorentzian constant curvature spacetime and R a riemannian
homogeneous space and the only nonzero components of the flux were
proportional to the volume forms of L and/or R.  For such backgrounds,
supergravity Killing spinors, which are in bijective correspondence
with the supersymmetries of a (bosonic) background, reduce to
geometric Killing spinors.

To any supersymmetric supergravity background one can associate a Lie
superalgebra, called the Killing superalgebra.  This is the
superalgebra generated by the Killing spinors; that is, if we let
K = K_{0} \oplus  K_{1} denote the Killing superalgebra, then

K_{1} = {Killing spinors} 

and    

K_{0} = [K_{1},K_{1}]

This is a Lie superalgebra, due to the odd-odd Lie bracket being
symmetric, as is typical in lorentzian signature in the physically
interesting dimensions.

I gave a triangular seminar in London about this topic and you can
find slides here:

21) Jos&eacute; Figueroa-O'Farrill, Killing superalgebras in supergravity,
talk at University of London, February 27, 2008, available at
<a href = "http://www.maths.ed.ac.uk/~jmf/CV/Seminars/KSA.pdf">http://www.maths.ed.ac.uk/~jmf/CV/Seminars/KSA.pdf</a>

There is some overlap with the one in Leeds, but not too much.

Cheers, Jos&eacute;

\end{quote}

These comments by Thomas Fischbacher should also fit into the big
picture somehow:

\begin{quote}

   As you know, there is a nice triality symmetric construction of E8
   that starts from SO(8)\times SO(8). But, considering the maximally
   split real form E8(8), did you also know that this
   SO(8)\times SO(8) is best regarded as SO(8,C+), with C+ being the
   split-complex numbers with i^{2}=+1? There also are
   56-dimensional real subgroups such as SO(8,C) (2 different
   embeddings - "IIA" and "IIB") - and there also is SO(8,C0).

   Basically, the way this works is that you can extend
   SO(8)\times SO(8) to SO(16) or SO(8,8) - depending on whether you
   add the V\times V or S\times S 8\times 8-block. But if you take
   diagonal SO(8) subgroups, then the 8\times 8 all split into
   28+35+1, and you can play nice games with these 28's...

   See: 

   22) T. Fischbacher, H. Nicolai and H. Samtleben, Non-semisimple and
   complex gaugings of N = 16 supergravity, available as <a href = 
   http://arxiv.org/abs/hep-th/0306276">arXiv:hep-th/0306276</a>.

\end{quote}

\par\noindent\rule{\textwidth}{0.4pt}
<em>A knowledge of the existence of something we cannot penetrate, of the 
manifestations of the profoundest reason and the most radiant beauty, 
which are only accessible to our reason in their most elementary forms.
It is this knowledge and this emotion that constitute the truly religious
attitude; in this sense, and in this alone, I am a deeply religious man.</em>
- Albert Einstein


\par\noindent\rule{\textwidth}{0.4pt}

% </A>
% </A>
% </A>
