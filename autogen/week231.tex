
% </A>
% </A>
% </A>
\week{May 9, 2006 }

<A HREF = "http://en.wikipedia.org/wiki/Enceladus_(moon)">Enceladus</A> 
is a moon of Saturn with a cracked icy surface, twisted 
and buckled by tidal forces, hinting at mysteries beneath:

<BR>
<DIV ALIGN = CENTER>
<A HREF = "http://www.nasa.gov/mission_pages/cassini/multimedia/pia07800.html">
<IMG SRC = "enceladus.jpg">
% </A>
</DIV>
<BR>

1) NASA, Enceladus the storyteller, 
<A HREF = "http://www.nasa.gov/mission_pages/cassini/multimedia/pia07800.html">http://www.nasa.gov/mission_pages/cassini/multimedia/pia07800.html</A>

Recently the NASA space probe Cassini has been getting a good look 
at Enceladus.  In March, Cassini discovered that it has geysers 
among the cracks near its south pole - 
geysers that spray water right out into space!

<BR>
<DIV ALIGN = CENTER>
<A HREF = "http://saturn.jpl.nasa.gov/news/press-release-details.cfm?newsID=639">
<IMG HEIGHT = 250 WIDTH = 300 SRC = "enceladus_geyser.jpg">
% </A>
</DIV>
<BR>

2) NASA's Cassini discovers potential liquid water on Enceladus,
<A HREF = "http://saturn.jpl.nasa.gov/news/press-release-details.cfm?newsID=639">http://saturn.jpl.nasa.gov/news/press-release-details.cfm?newsID=639</A>
3) Special issue on Enceladus, Science 311 (March 10th 2006). 

The water freezes in microscopic crystals, which replenish Saturn's 
<A HREF = "http://saturn.jpl.nasa.gov/multimedia/images/image-details.cfm?imageID=2091">E ring</A> - a diffuse bluish ring that was previously a mystery.

The currently popular theory for the geysers looks like this:

<BR>
<DIV ALIGN = CENTER>
<A HREF = "http://www.nasa.gov/mission_pages/cassini/multimedia/pia07799.html">
<IMG SRC = "enceladus_geyser_model.jpg">
% </A>
</DIV>
<BR>

4) NASA, Enceladus "cold geyser" model,
<A HREF = "http://www.nasa.gov/mission_pages/cassini/multimedia/pia07799.html">http://www.nasa.gov/mission_pages/cassini/multimedia/pia07799.html</A>

Enceladus is now the the only place besides Earth where liquid water
has been seen - though people believe Jupiter's moon 
<A HREF = "http://en.wikipedia.org/wiki/Europa_(moon)">Europa</A>
has oceans
under a layer of ice, and maybe 
<A HREF = "http://en.wikipedia.org/wiki/Ganymede_(moon)">Ganymede</A>
and 
<A HREF = "http://en.wikipedia.org/wiki/Callisto_(moon)">Callisto</A> 
do too.

While we tend to take it for granted, water is a very strange chemical:

5) Martin Chaplin, Forty-one anomalies of water, 
<A HREF = "http://www.lsbu.ac.uk/water/anmlies.html">http://www.lsbu.ac.uk/water/anmlies.html</A>

As you probably know, the specific heat of water is unusually high,
which stabilizes the Earth's temperature.  And no other simple compound
exhibits so many different forms.  There at least 18 forms of ice!  
You can tour them here:

6) Martin Chaplin, The phase diagram of water,
<A HREF = "http://www.lsbu.ac.uk/water/phase.html">http://www.lsbu.ac.uk/water/phase.html</A>

The hexagonal form of ice we find here on earth is called 
<A HREF = "http://www.lsbu.ac.uk/water/ice1h.html">ice Ih</A>.  
There's also a slightly denser cubic phase, 
<A HREF = "http://www.lsbu.ac.uk/water/ice1c.html">ice Ic</A>,
which forms when
water vapor is condensed on a cold substrate.   Below -130 Celsius, 
a low-density amorphous solid form called 
<A HREF = "http://www.lsbu.ac.uk/water/amorph.html#lda">LDA</A> 
is possible.  By 
compressing ordinary 
<A HREF = "http://www.lsbu.ac.uk/water/ice1h.html">ice Ih</A>
to high pressures, you get a different 
higher-density amorphous form, called 
<A HREF = "http://www.lsbu.ac.uk/water/amorph.html#hda">HDA</A>. 
And there's an even denser
amorphous form called
<A HREF = "http://www.lsbu.ac.uk/water/amorph.html#vhda">VHDA</A>. 

(It's unusual for a crystal to become amorphous when you compress it 
or cool it, but ordinary ice is unusually light: it floats on liquid
water!  That's because the powerful hydrogen bonds of water allow it 
to maintain a very sparse hexagonal crystal structure - so sparse you 
could even fit extra water molecules in the gaps.  When you crush this, 
it becomes amorphous.)

There are also crystal forms called 
<A HREF = "http://www.lsbu.ac.uk/water/ice_ii.html">ice II</A>
through 
<A HREF = "http://www.lsbu.ac.uk/water/ice_xii.html#icexiv">ice XIV</A>,
in order 
of discovery.  It would take a few weeks to discuss all these, but 
luckily Chaplin's website has a separate page on each kind, with nice
explanations and pictures of the crystal structures.

Kurt Vonnegut wrote a novel called "Cat's Cradle" starring a
substance called ice IX, which was supposedly more stable than liquid
water at ordinary temperatures and pressures.  When it got loose, it
destroyed the world.  Luckily the actual 
<A HREF = "http://www.lsbu.ac.uk/water/ice_ix.html">ice IX</A>
isn't like that, and it couldn't be: the most stable form of water 
already prevails.

But enough about ice IX.  I want to talk about 
<A HREF = "http://www.lsbu.ac.uk/water/ice_vii.html#iceten">ice X</A>!

This is one of the most extreme forms of ice known.  It's only stable
at pressures of about 50 gigapascals - in other words, roughly 500,000 
atmospheres.

Hmm.  Do those quantities mean as little to you as they do to me?  A
"pascal" is a unit of pressure, or force per area, equal to
one newton per square meter.  An "atmosphere" is another
unit of pressure, basically the average air pressure at sea level here
on Earth.  This has the annoying value of 101,325 pascals.  Personally
I have some trouble getting a feel for how much pressure this is,
since a newton per square meter isn't much, but 101,325 of them sounds
like a lot.  So for me, being an American, it's helpful to know that
an atmosphere equals 2116 pounds per square foot.  If you're a metric
sort of person, that's about the weight of 1 kilogram pushing down on
each square centimeter.  That's a lot of pressure we're under!  No
wonder we feel stressed sometimes.

(Yes, I know a kilogram is not a unit of weight.  I mean the weight
corresponding to a mass of a kilogram in the Earth's gravitational field 
at sea level.  Sheesh!)

But I digress.  I was saying that 
<A HREF = "http://www.lsbu.ac.uk/water/ice_vii.html#iceten">ice X</A>
only forms at a pressure 
of about 50 gigapascals.  But I've actually read figures ranging
from 44 to 80 gigapascals.  This raises the question: how do people 
know these things?  Do they actually know, or just guess?  

Well, some overgrown kids get paid to study these issues by actually
squashing water to enormous pressures using "diamond anvil 
cells".  
Not many substances can withstand such huge pressures, but diamonds 
can: as you know, they're really hard!  They're also transparent,
so you can see what's going on while you're squashing something.  
You basically just stick something between two carefully carved
diamonds, surrounded by a metal foil gasket, and squash the heck
out of it:

7) Diamond anvil cell, Wikipedia, 
<A HREF = "http://en.wikipedia.org/wiki/Diamond_Anvil_Cell">http://en.wikipedia.org/wiki/Diamond_Anvil_Cell</A>

Apparently they can get pressures of up to 360 gigapascals this way, 
which is the pressure at the center of the Earth.

Another method, which sounds even more fun, is to use a "light
gas gun".  Here you explode a few kilograms of gunpowder to shoot
a piston down a tube.  As it shoots forwards, the piston pushes some
gas down the tube.  The tube narrows to a tiny tip at the end, so the
gas is going really fast by the time it shoots out.  It shoots out
into a much narrower tube, where it pushes a projectile.  You can then
fire the projectile into something, to generate very high pressures
for a very short time.

8) Light gas gun, Wikipedia, <A HREF = "http://en.wikipedia.org/wiki/Light_Gas_Gun">http://en.wikipedia.org/wiki/Light_Gas_Gun</A>

It's not called a "light" gas gun because it's wimpy - in
fact they're huge, and everyone evacuates the lab when they run the
one at NASA!  It's called that because the speed of the projectile is
limited only by the speed of sound in the gas, which is higher for a
light gas like helium - or even better, hydrogen.  Even better, that
is, you don't mind exploding gunpowder near highly flammable hydrogen!
But, as you can imagine, people who do this stuff are precisely the
sort who don't mind.  You may enjoy reading how folks at Lawrence
Livermore National Laboratory used a light gas gun to compress
hydrogen to pressures of up to 200 gigapascals, enough to convert it
into a metal:

9) Robert C. Cauble, Putting more pressure on hydrogen,
<A HREF = "http://www.llnl.gov/str/Cauble.html">
http://www.llnl.gov/str/Cauble.html</A>

This supports the theory that the hydrogen at Jupiter's core is in 
metallic form, which would explain its enormous magnetic field.
They know their hydrogen became a metal because they fired a laser 
at it and saw it was shiny!   In fact, they fired three lasers at 
it simultaneously, just for kicks.

(By the way, this article erroneously says a "megabar" is
100 pascals.  It's a million atmospheres, or 100 gigapascals.)

But I'm digressing again.  I was saying 
<A HREF = "http://www.lsbu.ac.uk/water/ice_vii.html#iceten">ice X</A>
forms at a pressure
of around 50 gigapascals.  It's pretty far-out stuff.  It's a cubic 
crystal with density 2.5 times that of ordinary liquid water.  
It's so compressed that separate water molecules no longer exist!   
Instead, the oxygen atoms form a body-centered cubic.  This means 
they lie at the corners of a lattice of cubes, but with one at the 
center of each cube too, like the red dots in this picture by
Cavazzoni:

<DIV ALIGN = CENTER>
<IMG SRC = "http://math.ucr.edu/home/baez/cavazzoni_ice_X.jpg">
</DIV>

10) Carlo Cavazzoni, Large scale first-principles simulations of water
and ammonia at high pressure and temperature, Ph.D. thesis, Scuola
Internazionale Superiore di Studi Avanzati, October 1998.  Figure
4.10: symmetric and super-ionic ice X structures, p. 57.  Available at
<A HREF =
"http://sirio.cineca.it/~acv0/thesis.html">http://sirio.cineca.it/~acv0/thesis.html</A>


Hydrogen ions - in other words, protons - sit at the midpoints of half 
the edges connecting cube corners to cube centers.  There are two ways 
they can do this, illustrated by the yellow and gray dots shown above.
They can form a right-side-up tetrahedron, or an upside-down tetrahedron. 

A body-centered cubic can also be visualized as two interpenetrating 
cubic lattices, labelled A and B here:

<DIV ALIGN = CENTER>
<IMG SRC = "http://math.ucr.edu/home/baez/cavazzoni_ice_X_2.jpg">
</DIV>

Each oxygen has 4 hydrogens next to it.  If you compress water a bit
less than enough to make 
<A HREF = "http://www.lsbu.ac.uk/water/ice_vii.html#iceten">ice X</A>,
you get 
<A HREF = "http://www.lsbu.ac.uk/water/ice_vii.html">ice VII</A>.
This is almost the
same, but two of those hydrogens are closer to the oxygen than the
other two, so there are still separate water molecules!  It's completely 
random which two hydrogens are closer than the other two.  But if you 
cool down 
<A HREF = "http://www.lsbu.ac.uk/water/ice_vii.html">ice VII</A>,
you get 
<A HREF = "http://www.lsbu.ac.uk/water/ice_viii.html">ice VIII</A>,
where it's \emph{not} random.

So, Nature explores all the options.

Recently people have gotten interested in ice at even higher pressures
- and also higher temperatures, to understand the interiors of planets
like Neptune and Uranus.  Here pressures range from 20 to 800
gigapascals, and temperatures from 2000 to 8000 kelvin.  In "<A
HREF = "week160.html">week160</A>" I mentioned that on Neptune
it may rain diamonds, formed by methane in the atmosphere.  But what
happens to the water, and the ammonia?  If they became good electrical
conductors, that might explain the magnetic fields of these planets.
 
People have done computer simulations to study this:

12) C. Cavazzoni, G. L. Chiarotti, S. Scandolo, E. Tosatti, M. Bernasconi
and M. Parrinello, Superionic and metallic states of water and ammonia 
at giant planet conditions, Science 283 (January 1999), 44-46.  
Also available at <A HREF = "http://www.sciencemag.org/cgi/content/full/283/5398/44">http://www.sciencemag.org/cgi/content/full/283/5398/44</A>

<DIV ALIGN = CENTER>
<IMG SRC = "cavazzoni_ice_phases.png">
</DIV>

It seems that when you heat up 
<A HREF = "http://www.lsbu.ac.uk/water/ice_vii.html#iceten">ice X</A>,
it goes into a "superionic"
state where the little tetrahedra of hydrogen ions in each cube are
constantly randomizing themselves, instead of remaining fixed. 
It's a curious hybrid of a solid and a liquid, since the hydrogens 
are moving around, while the oxygens stay in their body-centered
cubic crystal.  

But if you heat it even more, the oxygen melts too!  As you can see 
from the phase diagram above, it then becomes an ionic fluid. 

As you heat it even more, you enter the region labelled "gap
closure", where the water starts to act like a metallic plasma.
Then it's a really good conductor of electricity.

The curve labelled "Neptune isentrope" describes the pressures 
and
temperatures you'd experience if you unwisely jumped into Neptune!

As you fell in, it would keep getting hotter and the pressure would 
keep rising until you entered this chart, at a temperature of about
2000 kelvin.  At this point you'd see molecular fluid water - I say 
this because at temperatures above 650 kelvin (the critical point 
for water), there's no sharp difference between liquid and gas.  
Then the fluid would become ionic... and then you'd start drifting
towards gap closure and the metallic plasma phase.  Down deep,
metallic plasmas of water and ammonia might explain the magnetic
field of this planet.

Recently people have done some experiments with water at extremely
high pressures, checking what theorists like Cavazzoni and company
predict.  For example, this paper says that using "extremely large
lasers", people have studied water at pressures near a terapascal -
1000 gigapascals:

13) P. M. Celliers et al, Electronic conduction in shock-compressed
water, Plasmas 11 (2004), L41-L48.

They also mention that "a single datum at 1.4 terapascals from an
underground nuclear experiment has never been repeated."  Some people
just don't know when to stop in the quest for higher pressures.

While I'm at it, I should mention a few more interesting articles
on weird forms of ice.  There's a lot of research on this subject!
Here's a quick overview:

14) Nancy McGuire, The many phases of water, American Chemical Society, 
<A HREF = "https://web.archive.org/web/20051201104533/http://www.chemistry.org/portal/a/c/s/1/feature_pro.html?id=c373e9fbed0a01c78f6a4fd8fe800100">https://web.archive.org/web/20051201104533/http://www.chemistry.org/portal/a/c/s/1/feature_pro.html?id=c373e9fbed0a01c78f6a4fd8fe800100</a>

Here's a webpage with some nice pictures and an interesting story:

15) J. L. Finney, The phase diagram of water and a new metastable form of 
ice, <A HREF = "http://www.cmmp.ucl.ac.uk/people/finney/soi.html">http://www.cmmp.ucl.ac.uk/people/finney/soi.html</A>

And finally, there's a paper that talks about how ordinary 
ice Ih but also silica and ice XI become amorphous when you 
squeeze them enough:

16) Koichiro Umemoto, Renata M. Wentzcovitch, Stefano Baroni and Stefano
de Cironcoli, Anomalous pressure-induced transition(s) in ice XI, Physical
Review Letters 92 (2004), 105502-1.   Also available at
<A HREF = "http://www.cems.umn.edu/research/wentzcovitch/papers/Phys._Rev._Lett._92_105502_(2004).pdf">http://www.cems.umn.edu/research/wentzcovitch/papers/Phys._Rev._Lett._92_105502_(2004).pdf</A>

There's some interesting math in here, because they do computer
simulations of the transition from a crystal to an amorphous 
substance, which is interesting to study using Fourier analysis.  
The idea is that certain vibrational modes of the crystal 
"go soft", so they get easily excited.  When a bunch of modes 
go soft that have wavelengths not equal to the crystal lattice 
spacing, the crystal structure becomes unstable, and there can 
be a transition to an amorphous state.

There's also interesting math lurking in Cavazzoni et al's 
models of ice X!  If you think particle physics is hard, just
wait until you try understanding something complicated, like water.

I've been sort of <A HREF = "diary/diary_2006.html">obsessed with 
ice</A> lately.  If you like it too,
I recommend this book for general information:

16) Mariana Gosnell, Ice: The Nature, the History, and the Uses of
an Astonishing Substance, Alfred A. Knopf, New York, 2005.

but I bought this one, because it tells an interesting history of
the science of climate change as seen from icy peaks:

17) Mark Bowen, Thin Ice: Unlocking the Secrets of Climate in the
World's Highest Mountains, Henry Holt & Co., 2005.

Now for some math.
Last week I said a bit about quivers, the McKay correspondence,
and string theory.   I want to dig deeper into the relation
between these subjects, because Urs Schreiber has some interesting 
ideas about them, which he's mentioned here:

18) Urs Schreiber, A note on RCFT and quiver reps, 
<A HREF = "http://golem.ph.utexas.edu/string/archives/000794.html">
http://golem.ph.utexas.edu/string/archives/000794.html</A>

But, I'm not feeling sufficiently energetic to explain these ideas
right now, especially since he already has!  For some more clues, try
this:

19) Paul Aspinwall, D-branes on Calabi-Yau manifolds, section 
7.3.1, The McKay correspondence, p. 101 and following.  Available
as 
<A HREF = "http://xxx.lanl.gov/abs/hep-th/0403166">hep-th/0403166</A>

For more on the the representation theory of
quivers, see the references in the "Addenda" to "<A
HREF = "week230.html">week230</A>", and also this excellent
book:

20) David J. Benson, Representations and Cohomology I, 
Cambridge U. Press, Cambridge 1991.  

You'll see how the non-simply-laced Dynkin diagrams get into the
act!  A more thorough treatment, fascinating but somewhat quirky,
can be found here:

21) P. Gabriel and A. V. Roiter, Representations of Finite-Dimensional 
Algebras, Enc. of Math. Sci., 73, Algebra VIII, Springer, Berlin 1992.

If you like category theory you may enjoy this book, because
it's all about representations of categories, i.e. functors

F: C \to  Vect

where C is a category.  It's full of nontrivial theorems about these,
starting with Gabriel's classification of quivers into those of finite
representation type (see "<A HREF =
"week230.html">week230</A>"), the tame quivers (which have an
infinite but still manageable set of indecomposable representations), 
and the wild ones.   
But, you may be puzzled when you read about "svelte"
categories, or functors that "preserve heteromorphisms".

I might as well say what those are.  A category is "svelte"
if its isomorphism classes of objects form a mere set instead of a
proper class, like the category of finite-dimensional vector spaces.
Most people would say such a category is "essentially
small".

And, a functor "preserves heteromorphisms" if it maps
heteromorphisms to heteromorphisms.  Well, duh!  But what's a
"heteromorphism"?  It's their term for a morphism that's not
an isomorphism.  Most people would say such a functor "reflects
isomorphisms".

You may also be interested in what a "locular" category is,
or a "spectroid"... but I won't tell you!  Read the book.

Speaking of category theory, this is my last week in Chicago, which
is really sad, because Steve Lack is just starting to give us a
crash course on "Australian category theory".  Australia, 
you see, is the center of macho category theory, where 
they're heavy on the calculus of mates,
doctrinal adjunctions are a dime a dozen, and everything should be
V-enriched if not W-enriched.  But
Chicago is starting to get macho too: tomorrow Nick Gurski defends
his Ph.D. thesis on "Algebraic Tricategories"!    So, the Chicago
gang wants to learn some tricks from the Australians.  But next
Monday I'm off to the Perimeter Institute, to indulge the physics
side of my personality....

\par\noindent\rule{\textwidth}{0.4pt}
\textbf{Addenda:} 
I thank Colin Rust for correcting a serious typo.  Uncle Al
points out that a newton is, quite appropriately, about the weight 
of an average apple.  Aaron Bergman had this to say:



% parser failed at source line 555
