
% </A>
% </A>
% </A>
\week{November 28, 2001 }

Groups are how mathematicians and physicists talk about symmetry, and
Lie groups are how they talk about \emph{continuously varying}
symmetries, like rotations, translations and the like.  Sophus Lie
helped start the subject of Lie groups in the late 1800s, and it's been
in constant growth ever since.  I spend lots of time studying it, and I
probably will all my life - there's a lot to learn!  To really
understand it, it helps to know the history.  And for that, this is the
book to read:

1) Thomas Hawkins, The Emergence of the Theory of Lie Groups: 
an Essay in the History of Mathematics, 1869-1926, Springer,
New York, 2000.

You have to know your Lie groups pretty well to enjoy this book, but
if you do, you'll find it's full of interesting facts.  For example:
folks often complain about Wilhelm Killing's original classification 
of simple Lie algebras - it wasn't rigorous, he made some mistakes, 
and so on.  Elie Cartan came along later and cleaned it up, and 
many people applaud Cartan's work and sneer at poor old Killing,
even though he was the one who came up with the original ideas.  
But in this book, it becomes clear that Killing was pretty much 
\emph{pushed} into publishing his ideas in a half-baked state by 
mathematicians who were dying to know his results!  Now I feel
even more sorry for him.  

There's also a lot of interesting stuff about Hermann Weyl's 
approach to representation theory via tensors and Young diagrams, 
and why he liked it better than Cartan's approach via roots and 
weights.  Basically, Weyl liked his approach because it stuck closer
to Felix Klein's original "Erlanger program" - a program for 
understanding geometry via symmetry groups.  But it's interesting
to see how Weyl studied and respected Cartan's approach, and tried to
bridge the gap between the two.

Okay... so much for gossip!  Now I'm going to dive in and pick up 
right where I left off in my discussion of the ideas behind this paper:

2) Michael Mueger, From subfactors to categories and topology I:
Frobenius algebras in and Morita equivalence of tensor categories,
available at 
<A HREF = "http://xxx.lanl.gov/abs/math.CT/0111204">math.CT/0111204</A>.

My ultimate goal is to take you to an elegant understanding of
Frobenius algebras by means of a 2-category called the "walking
ambidextrous adjunction", but first I'll play around a bit with a
simpler but more famous 2-category called the "walking
adjunction".  This may sound scary, but if you can stick with it,
you'll see that I'm really just using these 2-categories to describe fun
games that you can play with certain 2-dimensional pictures.  Even if
you don't read the words, please stare at the pictures - I spend my
Thanksgiving weekend drawing them, and I don't want that work to go to
waste!

Category theorists love to talk about adjoint functors, but 2-category
theorists know that these are just a special example of an
"adjunction".  An adjunction is something that makes sense in
any 2-category; if we take the 2-category to be Cat we get adjoint
functors.  There are lots of other nice examples that make this
generalization worthwhile.  For example, in "<A HREF =
"week83.html">week83</A>" I explained how a pair of dual vector
spaces is also an example of an adjunction.


To study adjunctions, it suffices to study the "walking
adjunction".  This is a little 2-category containing exactly the
stuff any adjunction in any 2-category must have: not a jot more, not a
tiddle less!  It was first studied by Schanuel and Street:

3) Stephen Schanuel and Ross Street, The free adjunction,
Cah. Top. Geom.  Diff. 27 (1986), 81-83.

In a bit more detail, the walking adjunction is the 2-category freely
generated by two objects:

a and b,

two morphisms:

L: a \to  b  and  R: b \to  a,

and two 2-morphisms, called the "unit" and "counit":

i: 1_{a} => LR  and  e: RL => 1_{b}

satisfying two relations, called the "triangle equations".  


I wrote down these equations already last week, but let me do it again
using "string diagrams", as explained in "<A HREF =
"week79.html">week79</A>" and "<A HREF =
"week92.html">week92</A>".  In a 2-categorical string diagram,
objects are denoted by 2d regions in the plane, morphisms are denoted by
1d edges, and 2-morphisms are denoted by 0d points.  If the dimensions
look sort of upside-down, you're right - that's exactly the point!

Instead of explaining the whole theory, I'll just plunge in with
the example at hand.  The unit i looks like this:


\begin{verbatim}

                     i
                    / \
                   L   R
                  /     \
              a  /   b   \  a
\end{verbatim}
    
while the counit e looks like this:


\begin{verbatim}

              b  \   a   /  b 
                  R     L  
                   \   / 
                    \ /
                     e
\end{verbatim}
    

Note that as you cross a line labelled "L" from left to right,
you go from region a to region b, which is our way of saying that L: a
-> b.  Similarly, as you cross a line labelled "R" from
left to right, you go from region b to region a, since R: b \to  a.

In terms of string diagrams, the triangle equations just say that we can
straighten out a zig-zag:


\begin{verbatim}

                     |                     |  
           i         |                     |
          / \        L                     |
    a    /   \       |                     |
        /     \      |                     |
       |       R     /         =      a    L    b
       |        \   /                      | 
       L         \ /    b                  |
       |          e                        |
       |                                   |  
\end{verbatim}
    
or a zag-zig:
 

\begin{verbatim}

         |                                  |   
         |          i                       |
         R         / \                      |
         |        /   \   a                 |
         |       /     \                    |
          \     L       |       =      b    R    a          
           \   /        |                   |
       b    \ /         R                   |
             e          |                   |
                        |                   |
\end{verbatim}
    
We can build any 2-morphism in the walking adjunction by vertically
and horizontally composing units and counits, which corresponds to
sticking together string diagrams in a vertical or horizontal way.
Thus, a typical 2-morphism looks like this:


\begin{verbatim}

      \     \   a   /   \   a   /      /               |
       \     R     L     R     L      /       i        |
        \     \   /       \   /      /       / \       L
         \     \ /         \ /      /   a   /   R      |    b
          \     e           e      /       /     \     |
    a      L                      R        \      \   / 
            \         b          /     i    \      \ / 
             \                  /     / \    L      e
              \                /     L   R    \       
               \              /     /  b  \    \  

\end{verbatim}
    
By the triangle equations, we could straighten out the zig-zag without
changing the 2-morphism.

As you may know, the word "anaranjado" means
"orange" in Spanish - there was no word in English for
"orange" before people in England started importing oranges
from Spain.  And this is a nice mnemonic, because if we take the above
picture and paint the regions labelled "a" orange, and paint
the regions labelled "b" black, the above picture has a
roughly tiger-striped appearance.  In fact, these tiger stripes tell you
everything you need to know about the 2-morphism!  For example, starting
from just this:


\begin{verbatim}

      \     \   a   /   \   a   /      /               |
       \     \     /     \     /      /       _        |
        \     \   /       \   /      /       / \       |
         \     \_/         \_/      /   a   /   \      |    b
          \                        /       /     \     |
    a      \                      /        \      \   / 
            \         b          /     _    \      \_/ 
             \                  /     / \    \      
              \                /     /   \    \       
               \              /     /  b  \    \  
\end{verbatim}
    
you can figure out where everything else should go.


By the way, note that orange stripes can disappear as we go down the
page, and they can split, but they can't appear or merge.  Black stripes
can appear or merge, but they can't disappear or split.  As a result,
there can never be any orange or black \emph{spots}.  We'll change
these rules later, when we talk about the walking "ambidextrous
adjunction".

Okay, so we've got this 2-category, the walking adjunction: let's call
it Ad for short.  It's pretty simple.  How can we understand it better?


Well, for any two objects a and b in a 2-category we get a
"hom-category" hom(a,b), whose objects are the morphisms from
a to b, and whose morphisms are the 2-morphisms between those.  If we
work out these hom-categories in Ad, we get some cool stuff.

First let's look at the hom-category hom(a,a).   In this category,
the objects are 

1_{a}, LR, LRLR, LRLRLR, ....

and all the morphisms are built by sticking these two basic
generators together vertically or horizontally:


\begin{verbatim}

                     \  \    a    /  /  
                      \  \       /  /
                       L  R     L  R
                        \  \   /  /
                  a      \  \ /  /      a
                          \  e  /
                           \   /
                           | b |
                           |   |
                           L   R
                           |   |
                           |   |   
\end{verbatim}
    
and

\begin{verbatim}

                              i
                             / \
                     a      |   |    a
                            | b |     
                            |   |
                            L   R   
                            |   |
                            |   |
\end{verbatim}
    
In tiger language, we're talking about pictures of black stripes on an
orange background.  The two basic generators are the merging of two
black stripes and the appearance of a black stripe.  

If you read "<A HREF = "week89.html">week89</A>", you'll know
another way to describe this!  Our ability to stick together pictures
vertically and horizontally makes hom(a,a) into a "monoidal
category".  LR is a "monoid object", with merging of two
black stripes being "multiplication", and the appearance of a
black stripe being the "multiplicative identity".  Being a
"monoid object" simply means that these operations satisfy the
left unit law:



\begin{verbatim}

                                 / /                 | |
                                / /                  | |
                               / /                   | |
                    /\        / /                    | |
                    \ \      / /                     | |
                     \ \    / /                      | |
                      \ \  / /                  a    | |
                       \ \/ /                        |b|
                        |  /          =              | |
             a          | |                          | |      a
                        | |                          | |
                        |b|                          | |
                        | |     a                    | |
                        | |                          | |
                        | |                          | |
                        | |                          | |
\end{verbatim}
    
and its mirror image, called the right unit law, together with the
associative law:


\begin{verbatim}

            \ \  a / /    / /      \ \    \ \  a / /
             \ \  / /  a / /        \ \  a \ \  / /
              \ \/ /    / /          \ \    \ \/ /
               \  /    / /            \ \    \  /
                \ \   / /              \ \   / /
                 \ \_/ /                \ \_/ /
                  \   /                  \   /
                   | |                    | |
              a    | |   a            a   | |   a
                   | |          =         | |
                   |b|                    |b|
                   | |                    | |
                   | |                    | |
                   | |                    | |
                   | |                    | |
\end{verbatim}
    
There aren't any other laws, so hom(a,a) is the "free monoidal
category on a monoid object", or if you prefer, the "walking
monoid"!


I touched upon the immense consequences of this fact for algebraic
topology in "<A HREF = "week117.html">week117</A>" and
"<A HREF = "week118.html">week118</A>".  They mainly rely on
another way of thinking about hom(a,a): it's the category of
order-preserving maps between finite ordinals!

For example, these black tiger stripes on an orange background:


\begin{verbatim}

         0          1           2                     3
    --------------------------------------------------------
   |  \     \   a |   |  a  /      /               |    |   |
   |   \     \    |   |    /      /       _        |    |   |
   |    \     \   |   |   /      /       / \       |    |   |
   |     \     \_/     \_/      /   a   /   \      |    |   |
   |      \                    /        \    \     |    |   |
   | a     \                  /          \    \   /    /    |
   |        \       b        /     _      \    \_/    /     |
   |         \              /     / \      \         /      |
   |          \            /     / b \      \   b   /   a   |
   |           \          /     /     \      \     |        |
    --------------------------------------------------------
                     0             1            2
\end{verbatim}
    
correspond to the order-preserving map 

f: {0,1,2,3} \to  {0,1,2}

with 

f(0) = 0, f(1) = 0, f(2) = 0, f(3) = 2.  

Just read the stripes down!

A more geometrical way to say the same thing is to call hom(a,a) the
category of "simplices", usually denoted \Delta .   Here the object 


\begin{verbatim}

                         |---n+1 of them---|
                          LRLR..........LRLR
\end{verbatim}
    
corresponds to the n-simplex, and these morphisms:


$$

                                 -i.LRLR-->
                 --i.LR->        -LR.i.LR->
1_{a}  --i-->  LR  --LR.i->  LRLR  -LRLR.i-->  LRLRLR ...
                 <-L.e.R-        <-L.e.RLR-
                                 <-LRL.e.R-

$$
    
are the basic "face" and "degeneracy" maps between
simplices, which you'll find in any book on algebraic topology.  The
n-simplex is a face of the (n+1)-simplex in n+1 ways, and there are n
basic degenerate ways to map the (n+1)-simplex down to the
n-simplex. These aren't \emph{all} the morphisms; just enough to generate all
the rest by composition - i.e., sticking together pictures vertically,
but \emph{not} horizontally.

Perhaps I should explain the notation here a bit more.  Readers of
"<A HREF = "week80.html">week80</A>" will know that I use a dot to denote horizontal composition of
2-morphisms.  For example, when we have a couple of 2-morphisms like
this:


$$

                      f           f'
                  ---->----   ---->----  
                 /   ||    \ /   ||    \              S: f => g
                x    || S   y    || T   z             T: f' => g'
                 \   \/    / \   \/    /
                  ---->----   ---->----
                      g           g'
$$
    
we get a 2-morphism like this:
 


$$

                         ff'
                  -------->-------
                 /       ||       \
                x        || S.T    z                S.T: ff' => gg'
                 \       \/       /
                  -------->-------
                         gg'
$$
    
But sometimes we can also horizontally compose a morphism and a
2-morphism!  We can do it whenever our morphism f looks like a little
"whisker" f sticking out of the 2-morphism T:


$$

                                  f'
                              ---->----  
                      f      /   ||    \              
                x----->-----y    || T   z             T: f' => g'
                             \   \/    /
                              ---->----
                                  g'
$$
    
and what we get is a 2-morphism f.S like this:
 


$$

                         ff'
                  -------->-------
                 /       ||       \
                x        || f.T    z                f.T: ff' => fg'
                 \       \/       /
                  -------->-------
                         fg'
$$
    
This process, called "whiskering", is not really a new operation.
f.S is really just the horizontal composite of these 2-morphisms:


$$

                      f           f'
                  ---->----   ---->----  
                 /   ||    \ /   ||    \              
                x    ||1_f  y    || S   z             
                 \   \/    / \   \/    /
                  ---->----   ---->----
                      f           g'
$$
    
Similarly we can define T.f in this sort of situation:


$$

                      f'           
                  ---->----   
                 /   ||    \      f                   T: f' => g'
                x    || T   y----->-----z             T.f: f'f => g'f
                 \   \/    /
                  ---->---- 
                      g'     
$$
    
Anyway, once you're an expert on this 2-categorical yoga, you can
easily see that these morphisms in hom(a,a), which are really 2-morphisms
in Ad:


$$

                                 -i.LRLR-->
                 --i.LR->        -LR.i.LR->
1_{a}  --i-->  LR  --LR.i->  LRLR  -LRLR.i-->  LRLRLR ...
                 <-L.e.R-        <-L.e.RLR-
                                 <-LRL.e.R-

$$
    
are obtained by taking our basic tiger stripe operations - the
"merging of two black stripes", or L.e.R, and the
"appearance of a black stripe", or i - and drawing some extra
black stripes on both sides.  That's what those LR's are for.  After
all, no tiger is complete without whiskers!

Okay.  Now, having understood hom(a,a) in all these ways, let's turn
to hom(b,b).  Luckily, this is very similar!  Here the objects are

1_{b}, RL, RLRL, RLRLRL, ....

and morphisms are pictures of \emph{orange} stripes on a
\emph{black} background:


\begin{verbatim}

           \   a   /   \   a   /      /               |
            \     /     \     /      /       _        |
             \   /       \   /      /       / \       |
              \_/         \_/      /   a   /   \      |    b
                                  /       /     \     |
                                 /        \      \   / 
       b                        /     _    \      \_/ 
                               /     / \    \      
                              /     /   \    \       
                             /     /  b  \    \  
\end{verbatim}
    
These orange stripes can only split:


\begin{verbatim}

                           |   |
                           |   |   
                           R   L
                           |   |
                           | a |
                           /   \                           
                          /  i  \
                  b      /  / \  \      b
                        /  /   \  \
                       R  L     R  L
                      /  /       \  \
                     /  /    b    \  \ 
\end{verbatim}
    
or disappear:


\begin{verbatim}

                            |   |    
                     b      | a |     b 
                            |   |
                            R   L   
                            |   |
                            |   |
                             \ /
                              e
\end{verbatim}
    
as we march down the page.  This means is that hom(b,b) is
\Delta ^{op}: the \emph{opposite} of the category of simplices, the 
\emph{opposite} of the category of finite ordinals, or the walking
\emph{comonoid} - which is just like a monoid, only upside down!  

Here is another picture of hom(b,b):


$$

                                  --R.i.LRL->
                 --R.i.L->        --RLR.i.L->
1_{b}  <--e--  RL  <--e.RL--  RLRL  <--e.RLRL--  RLRLRL ...
                 <--RL.e--        <--RL.e.RL-
                                  <--RLRL.e--
$$
    
If you're a devoted reader of This Week's Finds, you'll know I secretly
drew this category already in section N of "<A HREF = "week118.html">week118</A>".  There I was
talking about specific adjoint functors instead of the walking
adjunction, so as not to prematurely blow your mind.   I was also
writing horizontal composites backwards, for certain old-fashioned
reasons.  But the idea is exactly the same!  The morphisms above give
the usual "face and degeneracy maps" we always have in a simplicial set,
since a simplicial set is a functor

F: \Delta ^{op} \to  Set.

By the way, you may have noticed that to get from hom(a,a) to hom(b,b),
we had to switch the colors orange and black AND read the pictures
upside-down.  The reason is that if we turn around all the 1-morphisms
AND 2-morphisms in the walking adjunction, we get the walking adjunction
again.  Ponder that!  


We can summarize what we've learned so far using the "Platonic
idea" jargon I introduced last week:

The Platonic idea of a monoid and the Platonic idea of a comonoid are
the hom-categories hom(a,a) and hom(b,b) sitting inside the Platonic
idea of an adjunction!


 (By the way, to round this off we should really describe hom(a,b)
and hom(b,a), too.  I think hom(a,b) is the Platonic idea of "an
object with a left action of a monoid and a right coaction of a
comonoid, in a compatible way".  If so, hom(b,a) would be the
Platonic idea of "an object with a right action of a monoid and a
left coaction of a comonoid, in a compatible way".  By
"compatible" I'm saying that we can act on one side and coact
on the other side in either order, and get the same thing.  Filling in
the details requires concepts I'm not eager to discuss right now, so I
leave this as an exercise for the highly energetic reader.  The less
energetic reader can just study the tiger-stripe descriptions of these
categories.)

Finally, here's Mueger's new twist on all these ideas!  Better than
an adjunction is an "ambidextrous" adjunction.  This has some
extra structure, which turns out to explain all sorts of fancy-sounding
stuff people look at in the study of subfactors and TQFTs and the
like....

But what's an "ambidextrous adjunction"?


 A ambidextrous adjunction is where you have a morphism
L: a \to  b
in a 2-category that is both left and right adjoint to 
R: b \to  a.
More precisely, it is a setup

(a,b,L,R,i,e,j,f) 

where 

(a,b,L,R,i,e) 

and

(b,a,R,L,j,f) 

are both adjunctions.  

In terms of string diagrams, our generating 2-morphisms look like this:




\begin{verbatim}

                  i                             j
                 / \                           / \
                L   R                         R   L
               /     \                       /     \
           a  /   b   \  a               b  /   a   \  b




           b  \   a   /  b               a  \   b   /  a
               R     L                       L     R
                \   /                         \   /
                 \ /                           \ /
                  e                             f
\end{verbatim}
    
and the triangle equations say all possible zig-zags can be straightened
out.

Now let's study the "walking ambidextrous adjunction",
AmbAd.  As before, 2-morphisms in AmbAd can be described using pictures
with orange and black stripes - but now \emph{both} kinds of stripes
can appear, disappear, merge or split as we march down the page:


\begin{verbatim}

  -------------------------------------------------------
 |   \     \   a |   |  a  /      /             |       |
 |    \     \    |   |    /      /              |       |
 |     \     \__/     \__/      /      a        |       |
 |      \        _____         /     _____      |       |
 |       \      /  a  \       /     /     \     |       |
 |  a    /     /  ___  \     /     /       \   /        |
 |      /     /  /   \  \   /     /    __   \_/         |
 |     /     /   \ b /  /  /     /    /  \              |
 |    /  b   \    \_/  /  /     /    / a  \  b          |
 |   /        \       /  /     /    /      \            |
  -------------------------------------------------------
\end{verbatim}
    
This allows for quite arbitrary ways of cutting up a rectangle into
regions of orange and black, with piecewise linear boundaries, subject
to the condition that each vertical border has the same color all along
it.  The triangle equations and the rules for 2-categories say that we
can warp such a picture around without changing the 2-morphism that it
defines... I don't want to be too precise here, since it would be
boring.  Hopefully you get the idea: AmbAd has a purely topological 
description!  

Now for the punchline: in AmbAd, what is the category hom(a,a) like?
As in Ad, the objects are

1_{a}, LR, LRLR, LRLRLR, ...

but now the object LR is equipped not only with multiplication:


$$

                     \  \    a    /  /  
                      \  \       /  /
                       L  R     L  R
                        \  \   /  /
                  a      \  \ /  /      a      
                          \  e  /                     multiplication:
                           \   /                     L.e.R: LRLR => LR 
                           | b |
                           |   |
                           L   R
                           |   |
                           |   |   
$$
    
and multiplicative identity:


$$

                             i
                            / \
                    a      |   |    a                 multiplicative
                           | b |                         identity:
                           |   |                        i: 1_{a} => LR
                           L   R   
                           |   |
                           |   |

$$
    
but also a "comultiplication":
 

$$

                           |   |
                           |   |   
                           L   R
                           |   |
                           | b |
                           /   \                           
                          /  j  \                    comultiplication:
                  a      /  / \  \      a            L.j.R: LR => LRLR
                        /  /   \  \
                       L  R     L  R
                      /  /       \  \
                     /  /    b    \  \ 
$$
    
and "comultiplicative coidentity":




$$

                            |   |    
                     a      | b |     a 
                            |   |                    comultiplicative
                            L   R                       coidentity:
                            |   |                      f: LR => 1_{a}
                            |   |
                             \ /
                              f
$$
    
which make it into a monoid object \emph{and} a comonoid object.  Even
better, there are some extra relations between the multiplication and
comultiplication, which make LR into a so-called "Frobenius object"!

In short, hom(a,a) is the walking Frobenius object!  So is hom(b,b),
since there is no real asymmetry between the objects a and b in an
ambidextrous adjunction, as there was with an adjunction.  I haven't
thought much about hom(a,b) and hom(b,a) yet, but one obvious thing is
that they're isomorphic.


Next time I'll talk about examples of Frobenius objects and why they are
so important in subfactors, TQFTs and the like.  This is what Mueger is
really interested in.  Right now, I want to wrap up by saying exactly
what it means to say LR is a "Frobenius object".  What are the
extra relations between multiplication and comultiplication?

There are various ways of describing these relations.   Mueger uses
a pair of equations that are popular in the TQFT literature:

                                      

\begin{verbatim}

               \ \     / /                | |        | |
                \ \   / /                 | |        | |
                 \ \_/ /                  | |        | |
                  \   /                   |  \   a   | |
                   | |                    |   \      | |
              a    | |   a           a    | |\ \     | |   a
                   | |                    | | \ \    | |
                   |b|                    | |  \ \   | |
                   | |          =         | |   \ \  | |
                   | |                    | |    \ \ | |
                   | |                    | |  a  \ \| |
                   | |                    | |      \   |
                  / _ \                   | |       \ b|
                 / / \ \                  | |        | |
                / /   \ \                 | |        | | 
               / /     \ \                | |        | |
\end{verbatim}
    
and its mirror image.  People sometimes call these the "I = N"
equations, for the obvious reason.  So: one definition of a "Frobenius
object" in a monoidal category is that it's a monoid object / comonoid
object satisfying the I = N equations. 

Where can you read about this?  Well, besides Mueger's paper,
there are these:

4) Frank Quinn, Lectures on axiomatic quantum field theory, in 
Geometry and Quantum Field Theory, Amer. Math. Soc., Providence,
RI, 1995.  

5) Lowell Abrams, Two-dimensional topological quantum field theories
and Frobenius algebras, J. Knot Theory and its Ramifications 5 (1996),
569-587.

A "Frobenius algebra" is just a Frobenius object in the category
of vector spaces.  I seem to recall that this is equivalent to what
Quinn calls an "ambialgebra".  For any TQFT in any dimension, the 
vector space associated to the sphere is a commutative Frobenius
algebra.  The proof consists of playing with pictures very much 
like the ones above, but in higher dimensions.

The I = N equations are cute, but personally I prefer a more conceptual
description of a Frobenius object.  This may be a bit mindblowing to the
uninitiated, so if you're just barely hanging on, please stop now.

Hmm!  If you're still reading this, you must be brave!  Okay - don't
say I didn't warn you.  Let's start by pondering LR a bit more.  
This guy is its own adjoint, with the unit and counit as follows:


\begin{verbatim}

                      _
                a    / \      
                    |   |                     
                    |   |                      unit for LR =
                    | b |           multiplicative identity composed with
                   /  _  \                    comultiplication                
                  /  / \  \
                 /  /   \  \
                /  /  a  \  \



                \  \  a  /  /
                 \  \   /  /                 
                  \  \_/  /                   counit for LR =
                   \     /              multiplication composed with 
               a    | b |                comultiplicative coidentity
                    |   |
                    |   |
                     \_/
\end{verbatim}
    
It's easy to check the triangle equations by straightening out
the relevant zig-zags.


Now, whenever a monoid object has a right or left adjoint, that
right or left adjoint automatically becomes a comonoid object, by the
magic of duality.  But if a monoid object is its \emph{own} adjoint,
it becomes a comonoid object in \emph{two} ways, because it is both
its own left \emph{and} right adjoint!  So, our guy LR is a comonoid
object in \emph{three} ways!  Huh?  Well, we already knew LR was a comonoid
object before this devilish paragraph began, but since LR is its own
adjoint, it becomes a comonoid object in two other ways.  Amazingly, the
I = N equations are equivalent to the fact that all three comonoid
structures agree!  I leave this as an exercise for the insanely
energetic reader... I've worked it out before, and I rechecked it this
morning in bed.  I don't know if a proof exists in the literature, but
from what Mueger writes, I suspect maybe you can catch glimpses of it in
Appendix A3 of this book:

6) L. Kadison, New Examples of Frobenius Extensions, University 
Lecture Series #14, Amer. Math. Soc., Providence RI, 1999.

Anyway, the upshot is that we can equivalently define a Frobenius
object in a monoidal category as follows: it's a monoid object / 
comonoid object which becomes its own adjoint by letting

unit   = multiplicative identity composed with comultiplication
counit = multiplication composed with comultiplicative coidentity

and has the property that the resulting 3 comonoid structures agree.

Or, equivalently, that the resulting 3 monoid structures agree!

There is much more to say about this, but let's stop here.

\par\noindent\rule{\textwidth}{0.4pt}
Postscript - Oswald Wyler had this correction to make:

\begin{quote}
The walking adjunction is much older than the 1986 paper by Schanuel and
Street.  Back in 1970, Pumpl&uuml;n published a paper: Eine Bemerkung
&uuml;ber
Monaden und adjungierte Funktoren, Math. Annalen 185 (1970), 329-377.
The small bicategory "walking adjunction" definitely was in that paper,
but I don't recall whether it was explicitly formulated or not.
\end{quote}

Andree Ehresmann added:
\begin{quote}
On the "walking adjunction"
I don't know the Pumplun's paper cited by Wyler. But there is another
reference at about the same time; indeed, the "walking adjunction" has been
explicitly constructed and studied in the paper of Auderset:
\begin{quote}
         "Adjonction et monade au niveau des 2-categories"
\end{quote}
published in "Cahiers de Top. et Geom. Diff." XV-1 (1974), 3-20.
More formally it could also be called "the 2-sketch of an adjunction" in
the terminology in my paper with Charles Ehresmann:
\begin{quote}
"Categories of sketched structures", in the "Cahiers" XIII-2 (1972),
\end{quote}
reprinted in
"Charles Ehresmann: Oeuvres completes et commentees" Part IV-2.
\end{quote}

Bill Lawvere added:
\begin{quote}

<h4>ONE MORE HISTORICAL CITATION</h4>
    The Pumplun paper cited by Wyler as well as the Auderset paper cited
by Mme Ehresmann illustrate that the study of generic structures in
2-categories has been going on for some time.  My own paper
ORDINAL SUMS AND EQUATIONAL DOCTRINES, SLNM 80 (1969) 141-155
shows that the augmented simplicial category \Delta  serves as the generic
monad, but moreover goes on to actually apply this to show that the
Kleisli construction is a tensor product left-adjoint to the Eilenberg-
Moore construction which is an enriched Hom. The Hom/tensor formalism
appropriate to the case of strict monoid objects is all that is required
here, as I will explain below.

<h4>AN EXTENSION AND A RESTRICTION
</h4>
    The important special case of FROBENIUS monads is explicitly
characterized in three ways in my paper.
    Concerning the IDEMPOTENT case discussed a few days ago by Grandis
and Johnstone, note that the publication of Schanuel and Street proves
among other things that the monoid \Delta  in Cat has very few quotients
(see below for significance of the monoid structure).
<h4>
THE GENERAL HOM/TENSOR FORMALISM AND A VERY PARTICULAR MONOID
</h4>
    In any cartesian-closed category with finite limits and co-limits, a
non-linear version of the Cartan-Eilenberg Hom/tensor formalism applies
to actions and biactions of monoid objects.  In Cat, \Delta  is a (strict)
monoid and its actions are precisely monads on arbitrary categories.  A
crucial part of the formalism is that categories of actions are
automatically enriched in the basic cartesian-closed category, which in
this case is Cat.  There is a particular biaction of \Delta , which I called
\Delta  plus, with the property that the enriched Hom of it into an
arbitrary \Delta -action is exactly the Eilenberg-Moore category of
"algebras", automatically equipped with its structure as a \Delta ^op action
(co-monad).  The left-adjoint tensor assigns to any category equipped with
a co-monad its Kleisli category, as a category with monad.  Not only are
the calculations in this particular case quite explicit, but the enriched
Hom tensor formalism has a lot of content which is still under-exploited.
<h4>
SKETCHES VERSUS PLATONISM
</h4>
     The often repeated slander that mathematicians think "as if" they
were "platonists" needs to be combatted rather than swallowed. What
mathematicians and other scientists use is the objectively developed human
instrument of general concepts. (The plan to misleadingly use that fact as
a support for philosophical idealism may have been an honest mistake by
Plato, or it may have been part of his job as disinformation officer for
the Athenian CIA organization; it probably would not have survived until
now had it not been for the special efforts of Cosimo de' Medici.)
It seems that a general concept has two related aspects, as I began to
realize more explicitly in connection with my paper Adjointness in
foundations, Dialectica vol. 23, 1969 281-296;  I later learned that
some philosophers refer to these two aspects as
"abstract general vs. concrete general".  For example, there is the
algebraic theory of rings  vs. the category of all rings, or
a particular abstract group  vs. the category of all permutation
representations of the group.  While it is "obvious" that, at least in
mathematics, a concrete general should have the structure of a category,
because all the instances embody the same abstract general and hence
any two instances can be compared in preferred ways, by contrast it was
not until the late fifties that one realized that an abstract general can
also be construed as a category in its own right. That realization
essentially made explicit the fact that substitution is a logical
operation and indeed is the most fundamental logical operation.
     Thus an abstract general is essentially a special algebraic structure
indeed a category with additional structure such as finite limits or
still richer doctrines.  As with other algebraic structures there are
again two aspects, the structures themselves and their presentations which
are closely related, yet quite distinct;  for example, more than one
presentation may be needed for efficient calculations determining features
of the same algebraic structure. What is meant by a presentation depends 
on the doctrine: for example \Delta  as a mere category has an infinite
presentation used in topology, but as a strict monoidal category it has
a finite presentation.
     The notion of SKETCH is the most efficient scheme yet devised for the
general construction of PRESENTATIONS OF ABSTRACT GENERALS. The fact that
particular abstract generals and the idea of sketches exist within the
historically developed objective science does not mean that they somehow
always existed; to call them "platonic" seems to detract from the honor of
their actual discoverers.
     Bill Lawvere
\end{quote}



 \par\noindent\rule{\textwidth}{0.4pt}

% </A>
% </A>
% </A>
