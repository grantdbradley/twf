
% </A>
% </A>
% </A>
\week{June 4, 1995}

I recently went to a workshop on canonical quantum gravity 
in Warsaw, organized by Jerzy Kijowski and Jerzy Lewandowski,
and I learned some interesting things.  I'll talk about some of them
in this issue, and some in the next.

Conferences are a funny thing.  On science newsgroups on the net,
there is very little talk about conferences.  This is probably because 
the people who really understand conferences are too busy flying from one 
conference to the next to post to newsgroups very often.  Academic success is
in part measured by the number of conference invitations one receives,
the prestige of the conferences, and the type of invitation.  For example,
a big plenary lecture on an impressive stage, preceded by a little warmup
where someone explains how great you are, counts for infinitely many talks 
in those parallel sessions where dozens of people get 10 minutes each to 
explain their work before the moderator begins to make little coughs 
indicating that it's time for the next one, while all the while people drift 
in and out in a feeble attempt to find the really interesting talks.  
Still, giving any sort of talk is regarded as better than giving none, so
academics spend a lot of time doing this sort of thing.  

One of the great dangers of being a successful academic, in fact, is that 
one may get invited to so many conferences that one never has time to think.
Winning the Nobel prize is purported to be the kiss of death in this respect.
Of course, it's a universal platitude that the real thinking at conferences 
gets done not during the talks, but informally in small groups.  But the 
funny thing is that at most conferences people are so worn out after going 
to a day's worth of talks that they have limited energy for serious 
conversation afterwards: they usually seem more interested in
finding the good local restaurants and scenic attractions.  If people could
have conferences with no lectures whatsoever, or maybe one a day, it would
probably be more productive.  But the idea that a bunch of people 
could figure something out just by sitting around and chatting informally 
is absolutely foreign to our conception of "work".  People expect to receive
money from bureaucrats to go to conferences, but to convince a bureaucrat
that you are deserve the money, you need to give a lecture, so of course 
all conferences have too many lectures.

Turning back towards Warsaw, a city with a marvelous mathematical 
history, I am reminded of Gian-Carlo Rota's biographical sketch of 
Stanislaw Ulam, in which (as a master of irony) he talks about how lazy
Ulam was: all he wanted to do was sit around in cafes and come up with
interesting conjectures and research rograms, and leave it to others to 
work them out.  And this in turn reminds me of the Scottish Cafe, where 
Polish mathematicians used to hang out and write on the tablecloths, 
until the owner provided them with a notebook, in which many
famous conjectures were formulated, and I believe prizes like bottles of 
wine were offered for their solutions.  Was the Scottish Cafe in Warsaw?
[No, Lwow.]  Does it still exist?  I completely forgot to check while 
I was there.  The Banach Center, in which the conference participants 
stayed, comes from a later stratum of Polish mathematical history; 
it was built after the war, and one room still contains a portrait 
of Lenin.  I know that because a film crew used it to shoot a 
scene for a historical movie!

Anyway, I enjoyed this conference in Warsaw quite a bit, because a lot
of people working on the loop representation of quantum gravity were
there, and I managed to have a fair number of serious conversations.
Before going into what I learned there, I should say that I just found a 
fun thing for people to read who are interested in quantum gravity, but 
are not necessarily specialists:

1)  Gary Au, The quest for quantum gravity, available as gr/qc-9506001.

This consists mainly of interviews with Chris Isham, Abhay Ashtekar and
Edward Witten.  What's nice is that the interviews are conducted by someone
who knows physics. The questions and answers are technical enough to
convey some of the real substance of the subject, while still (I hope) non-
technical enough so that you don't have to be an expert to get a lot out of 
them.  Isham talks mainly about the "problem of time" in quantum gravity, 
Ashtekar talks mainly about the loop representation of quantum gravity, 
and Witten talks about string theory.

Anyway, Ashtekar and a bunch of other good people were at this Warsaw 
conference, which is why I went.  The main topics of conversation were 
spin networks and their use in studying the area and volume operators in
quantum gravity.  As I explained earlier in "<A HREF = "week43.html">week43</A>", one may very roughly 
think of a spin network as a graph whose edges are labelled with "spins" 
0,1/2,1,3/2, and so on, and who vertices are labelled with certain gadgets 
called "intertwining operators" (which in the simplest case are just the 
Clebsch-Gordon coefficients you learn about when studying angular momentum 
in quantum mechanics).  Penrose introduced these as abstract graphs (see
"<A HREF = "week22.html">week22</A>" and "<A HREF = "week41.html">week41</A>"), as a kind of substitute for thinking of space as a 
manifold, but more recently Rovelli and Smolin started thinking of them 
as graphs embedded into 3d space, and saw that these were a really natural 
way to describe states of quantum gravity: even better than loops, because 
they form an orthonormal basis!  Actually, it was mainly me who proved 
in a really rigorous way that they form an orthonormal basis, but 
Rovelli and Smolin had already been doing calculations using this idea 
for a while.  One thing they computed was the eigenvalues of the 
observables in quantum gravity corresponding to the area of a 
surface in space, or the volume of a region.  

Now there are all sorts of technical caveats and subtleties that I don't 
want to get into here, but in a really rough sort of sense, what their 
answers suggest is that IF the loop representation of quantum gravity is right,
and we are on the right track about how it works, then the area of surfaces 
comes in certain (not integer, but discrete) multiples of the Planck length 
squared, and the volume of regions comes in multiples of the Planck length 
cubed.  Note: that was a big "IF".  This is especially interesting because it
doesn't arise by assuming from the start that spacetime has a discrete
structure.  In fact, their computations assume spacetime is a continuous
manifold.  Nonetheless this discreteness pops out.  It's not completely 
surprising: after all, Schrodinger's equation for the hydrogen
atom is a perfectly "continuous" sort of thing, a partial differential equation,
but the energy of the bound states winds up being a discrete sort of thing.
Still, it's sort of exciting and new.

An interesting thing happened at the conference.  Renate Loll, who works on
the loop representation of gauge theories and also lattice gauge theory,
has recently developed a lattice formulation of quantum gravity closely
modelled after the loop representation:

2) Renate Loll, Nonperturbative solutions for lattice quantum gravity, 
preprint available as <A HREF = "http://xxx.lanl.gov/abs/gr-qc/9502006">gr-qc/9502006</A>.

This has the wonderful feature that it's perfectly rigorous and also
one can start using computers to start calculating things with it.
For example, the most subtle aspect of the loop representation of
quantum gravity is the Wheeler-DeWitt equation


$$

                                 H \psi  = 0
$$
    

where H is an operator called the "Hamiltonian constraint".  More on
this later; my point here is just that physical states of quantum
gravity need to satisfy this equation.  Getting H to be well-defined
is tricky when space is a continuum, but in Loll's lattice version of
theory (which is an approximation to the full continuum theory) she has
already done this, so one can now start trying numerically to find
solutions and see what they look like.  She has also found some explicit
solutions.  

\emph{Also}, she did some work on the volume operator in her lattice approach, 
and came up with a result in contradiction to Rovelli and Smolin's
paper on the subject (cited in "<A HREF = "week43.html">week43</A>").  They had said that states
corresponding to trivalent spin networks ---- spin networks with only
3 edges at each vertex --- could have nonzero volume.  But using her version 
of the theory she computed that trivalent states --- states with only 3 nonzero
spins at the edges of the lattice incident to any vertex --- all had
zero volume, and that she needed at least 4 nonzero spins to get
volume!  The volume operator, in case you're wondering, acts as a certain
sum over vertices: each one winds up contributing a certain finite
amount of volume, which the theory allows you to compute.

This led to a whole lot of discussion and scribbling on the blackboards
of the Banach center.  I found it truly delightful to see all these 
physicists drawing pictures of spin networks and doing graphical computations
just the way a knot theorist like Kauffman does all the time.  It was as
if the universe had this spin network aspect to it, and everyone was
finally starting to catch on.  Either that or mass delusion!  I hadn't
quite gotten the hang of how to compute these volume operators before,
so it was a great chance to learn: one person would do a computation,
then someone else would do it a different way and get a different answer,
then someone else would do it yet another way and get yet another answer,
and so on, so you could ask lots of questions without seeming too dumb.  
Even I did a computation after a while, and got zero volume for at least
a certain class of trivalent vertices.  The votes in favor of trivalent 
vertices having zero volume kept piling up.  Finally Smolin noticed 
that he and Rovelli had made a sign mistake.  This is incredibly easy to do, 
since there are lots of tricky sign conventions in spin network
theory.  Fundamentally these are due to the fact that spin-1/2 particles
are fermions... but I don't think people fully understand the physical
implications of this.  (There is also a marvelous category-theoretic explanation
of it, but I fear that if I go into that all the physicists will stop
reading.  Maybe some other time.)  Rovelli and Smolin got pretty depressed
about this for a while, but I tried to reassure them that only people
who write really interesting papers ever get anybody to find the
mistakes.   

So perhaps we know a little more about the meaning of volume in a
quantum theory of spacetime.

Spin networks are very beautiful and simple things.  To learn about them, 
in addition to the various papers listed in the "weeks" above, one can 
now turn to Rovelli and Smolin's paper:

3) C. Rovelli and L. Smolin, Spin networks in quantum gravity, preprint
available in LaTeX form as gr/qc-9505006.

If you are more of a mathematician, or less of an expert on quantum gravity,
you might also try a review article I wrote about them, which starts with a
quick summary of what the heck canonical quantum gravity is about, why it's
hard to do, and why the loop representation seems to help:

4) J. Baez, Spin networks in nonperturbative canonical quantum gravity,
preprint available in LaTeX form as gr-qg/9504036, or via ftp from 
math.ucr.edu, as the file net.tex in the directory baez.

Now so far I have been trying to make things sound simple, but here I 
should point out that when one talks about "states of quantum gravity" 
there are at least three quite different things one might mean.  This is 
because the loop representation follows Dirac's general philosophy of 
quantizing systems with constraints, with some extra twists here and there.  
As I've repeatedly explained (e.g. "<A HREF = "week43.html">week43</A>"), Einstein's equation for 
general relativity has 10 components, and if you split spacetime up into 
space and time (more or less arbitrarily --- there's no "best" way) 4 of 
these can be seen as constraints that the metric on space and its first time 
derivative must satisfy (at any given time), while the remaining 6 describe 
how the metric on space evolves in time (which makes sense, because the 
metric has 6 components).  When you follow Dirac's procedure for quantizing
the equations what you do is this.  First you forget about the constraint
and get a big space of states, the "kinematical state space".  There are lots
of mathematical choices involved here, but Ashtekar and Lewandowski
came up with a particular nice way of doing this rigorously, and one calls this
space of states "L^2 of the space of SU(2) connections modulo gauge
transformations with respect to the Ashtekar-Lewandowski generalized 
measure".  Spin networks form an orthonormal basis of this Hilbert space.  
All the stuff about area and volume operators above refers to operators on this 
space.

Then, however, you need to deal with the constraints.  Now
3 of the 4 constraints simply amount to requiring that your states be invariant
under all diffeomorphisms of space, so these are usually dealt with first,
and called the "diffeomorphism constraint".  Imposing these constraints
are a bit tricky; naively one would first guess that this "diffeomorphism-
invariant state space" is just a subspace of the original kinematical state
space, but actually it's not quite so simple.  In any event, there are also
spin network states at the diffeomorphism-invariant level, corresponding
not to \emph{particular} embeddings of graphs in space, but to diffeomorphism
equivalence classes thereof.  This again has been used by Rovelli, Smolin and
others for a while now, but it was first rigorously shown in the following 
paper:

5) Abhay Ashtekar, Jerzy Lewandowski, Don Marolf, Jose Mourao, and 
Thomas Thiemann, Quantization of diffeomorphism invariant theories of 
connections with local degrees of freedom, to appear in the November 1995 
Jour. Math. Phys. special issue on diffeomorphism-invariant field theory, 
preprint available as <A HREF = "http://xxx.lanl.gov/abs/gr-qc/9504018">gr-qc/9504018</A>.

This paper is nice in part because it doesn't assume you already have read
every previous paper about this stuff; instead, it describes the general plan
of the loop representation before constructing the diffeomorphism-
invariant spin network states.  Also, buried in an appendix somewhere, it
gives nice conceptual formulas for the area and volume operators, which 
serve as a complement to Rovelli and Smolin's explicit computations of 
their matrix elements in terms of the spin network basis.  

Anyway, after taking care of the diffeomorphism constraint, one
finally needs to take care of the Hamiltonian constraint, meaning
one needs to find states satisfying the Wheeler-DeWitt equation.  This is the
hardest thing to make rigorous, and the most exciting aspect of
the whole subject, because it expresses the fact that "physical
states" of quantum gravity are invariant under diffeomorphisms
of space-TIME, not just space.  There is much more to say about this,
but I won't go into it here.

Now besides Loll and Rovelli and Smolin, all the authors of the above
paper except Mourao were at the conference in Warsaw, so there was a 
large contingent of spin network fans around, not even counting some
other folks whose work I will get to in a while.  This is why I was so
eager to go there, especially because my own talk was on a rather esoteric
subject which only these experts could be expected to give a darn about.  
Namely....

The breakthrough of Ashtekar and Lewandowski, when it came to making the 
loop representation rigorous, involved working with piecewise real-analytic 
loops rather than piecewise smooth loops.  (Actually Penrose suggested this 
idea.)  This is because piecewise smooth loops can intersect in crazy ways, 
like in a Cantor set, which nobody could figure out how to handle.  But the 
price of this breakthrough was that one had to assume the 3-manifold 
representing space was real-analytic, and things then only work nicely for 
real-analytic diffeomorphisms, as opposed to smooth ones.  This always 
bugged me, so I have been working away for about a year trying to deal with 
smooth loops, and finally I got smart and teamed up with Steve Sawin, and 
we recently figured out how to get things to work with smooth loops (at least a 
bunch of things, like the Ashtekar-Lewandowski generalized measure).  Our 
paper will be out pretty soon, but for now anyone who wants a taste of the 
mathematical technology involved should look at:

6)  Steve Sawin, Path integration in two-dimensional topological quantum 
field theory, to appear in the October 1995 Jour. Math. Phys. issue on 
diffeomorphism-invariant field theory, preprint available as gr/qc-9505040.

Loop representation ideas are applicable not only to canonical quantum
gravity but also to path integrals in gauge theory, because in both cases
one is doing integrals over a space of connections mod gauge 
transformations.  Here Sawin uses them to give a rigorous formulation 
of 2d TQFTs in terms of path integrals.  There aren't many unitary 
2d TQFTs, and all of them are isomorphic to 2-dimensional quantum 
gravity with the usual Einstein-Hilbert action, with different values of the 
coupling constant, or else direct sums of such theories.    

Next "week" I'll talk about cool new idea Smolin has about TQFTs,
quantum gravity, and Bekenstein's bound on the entropy of a
physical system in terms of its surface area.
\par\noindent\rule{\textwidth}{0.4pt}

% </A>
% </A>
% </A>
