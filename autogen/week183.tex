
% </A>
% </A>
% </A>
\week{July 30, 2002 }

I'm now in England, visiting the category theorists in Cambridge.
Before coming here I went to a wonderful conference in honor of
Graeme Segal's 60th birthday.  Most of the talks there described 
the marvelous different ways in which ideas from string theory
are spreading throughout mathematics.  I should really tell you
about this stuff... it's very cool... but right now I'm in the
mood for talking about something simpler: some ways in which ideas
from \emph{quantum theory} are spreading throughout mathematics.

Quantum theory is digging its way ever deeper into the mathematical
psyche: for every branch of math, people seem to be developing a
corresponding quantized version, from "quantum groups" to
"quantum cohomology".  Now there is even a textbook on
"quantum calculus", suitable for undergraduates:

1) Victor Kac and Pokman Cheung, Quantum Calculus, Springer, Berlin, 2002.

Indeed, we'll soon see that quantum calculus is based on an even simpler
subject that deserves to be called "quantum arithmetic"!


This book talks about two modified versions of calculus: the
"h-calculus" and the "q-calculus".  The letter h
stands for Planck's constant, while the letter q stands for
quantum. They are adjustable parameters related by the formula

q = exp(ih).

In particular, these modified versions of calculus reduce to
Sir Isaac Newton's good old "classical calculus" in the limit where


$$

h \to  0
$$
    
or alternatively,


$$

q \to  1.
$$
    
One eerie thing about these modified versions of calculus is that people
discovered them before quantum mechanics - and they even used the
letters "h" and "q" in their formulas!  In
particular, the use of the letter "q" seems to go back all the
way to Gauss, who wrote about a q-analogue of the binomial formula and
other things.

So what's the idea?   Like many great ideas, it's pathetically simple. 
To get the h-calculus, you just leave out the limit in the definition of
the derivative, using this instead:


\begin{verbatim}

              f(x+h) - f(x)
              -------------
                    h
\end{verbatim}
    
In the limit as h \to  0, this reduces to the usual derivative.


There is a lot to say about this, but deeper and more mysterious
mathematics arises from the q-calculus, where we use the
"q-derivative":


\begin{verbatim}

              f(qx) - f(x)
              ------------
                 qx - x
\end{verbatim}
    
This reduces to the usual derivative as q \to  1.  Note that the
h-derivative says how f(x) changes when you \emph{add} something to x, while
the q-calculus says how it changes when you \emph{multiply} x by something.

Some choices of q are more interesting than others.   If q is a complex
number with |q| = 1, we can take the q-derivative of a function that's
defined only on the unit circle in the complex plane!  Multiplying by a
unit complex number rotates the unit circle a bit, just as adding a real
number translates the real line.   If you think about this for a while
you'll see the relationship between the h-calculus and the q-calculus, 
and how it's especially nice when we set q = exp(ih).

Alternatively, if q is an integer, we can take the q-derivative of a
function that's defined only on the integers!  This is especially cool
when q is a prime or a power of a prime; then there are nice connections
to algebra.

Pretty much anything you can do with calculus, you can do with the
q-calculus.  There are q-integrals, q-trigonometric functions, 
q-exponentials, and so on.   If you try books like this:

2) George E. Andrews, Richard Askey, Ranjan Roy, Special Functions,
Cambridge U. Press, Cambridge, 1999.

you'll see there are even q-analogues of all the special functions you
know and love - Bessel functions, hypergeometric functions and so on. 
And like I said, the really weird thing is that people invented them
\emph{before} their relation to quantum mechanics was understood.

I can't possibly explain all this stuff here, but a good way to get
started is to look at the q-analogue of Taylor's formula.  In ordinary
calculus this formula says how to reconstruct any sufficiently nice
function from its derivatives at zero:


$$

f(x) = f(0) + f'(0) x + f''(0) x^{2}/2! + ... 
$$
    
In q-calculus we can write down the \emph{exact} same formula using
q-derivatives and q-factorials!  The nth q-derivative of a function is
defined in the obvious way, by taking the q-derivative over and over.  
Let's do this to the function x^{n}.  
If we take its q-derivative \emph{once} we get:


\begin{verbatim}

     (qx)^{n} - x^{n}         q^{n} - 1
     -----------   =   ------- x^{n-1} 
       qx - x           q - 1
\end{verbatim}
    
We can make this look almost like the usual derivative of x^{n} if
we define the "q-integer" [n] by


\begin{verbatim}

            q^{n} - 1
  [n]  =   -------   
            q - 1

       = 1 + q + q^{2} + ... + q^{n-1}
\end{verbatim}
    
Then the q-derivative of x^{n} is just 


$$

[n] x^{n-1}
$$
    
This implies that the nth q-derivative of x^{n} is the "q-factorial"
 

\begin{verbatim}

[n]! = [1] [2] ... [n] 
\end{verbatim}
    
This in turn means that the usual Taylor formula still works if we
replace derivatives by q-derivatives and factorials by q-factorials.

Now, starting with q-factorials we can define q-binomial coefficients:


\begin{verbatim}

            [n]! 
        -----------
        [m]! [n-m]!
\end{verbatim}
    
and then cook up a q-Pascal's triangle, prove a q-binomial theorem, and 
so on.  It's not just a matter of recapitulating ordinary calculus, 
either: eventually we run into lots of cool identities that have no 
classical analogues, like the "Jacobi triple product formula":


\begin{verbatim}

                 sum_{n}  q^{n(n+1)/2} x^{n}  =  

        product_{i}  (1 + xq^{i}) (1 + x^{-1}q^{i-1}) (1 - q^{i}) 
\end{verbatim}
    
where the sum is over all integers n, while the product is over positive
integers i.  Now, personally I'm not a big fan of identities just for
the sake of identities.  However, I like taking identities and trying to 
find their "secret inner meaning" - mainly by seeing how they come 
from isomorphisms between interesting mathematical structures.  The 
mysterious identities of q-mathematics provide an ample playground 
for this game, especially since they're all related in intricate ways.  

If you ever get stuck on a desert island you can have lots of fun
reinventing quantum calculus, and if you \emph{don't}, you can read Kac 
and Cheung's book.  So either way, there's no point in me describing 
its contents further; instead, I want to say more about how q-mathematics
is related to physics.

For starters, let's see how the canonical commutation relations
change when we use a q-derivative to define the momentum operator,
instead of an ordinary derivative.  Remember what Schroedinger said:
a particle on a line is described by a "wavefunction", which is a 
complex function on the line, say \psi .  The position operator Q 
multiplies a wavefunction by x:


$$

(Q \psi )(x) = x \psi (x)
$$
    
while the momentum operator P basically takes their derivative:


$$

(P \psi )(x) = -i \psi '(x)
$$
    
The canonical commutation relations say that


\begin{verbatim}

PQ - QP = -i
\end{verbatim}
    
Now, how does this change if we define the momentum operator using the
q-derivative instead?  I could do this calculation for you, but you'll
be a much better person if you do it yourself - it's incredibly easy, 
so \emph{please} do it.  The answer is


\begin{verbatim}

PQ - qQP = -i
\end{verbatim}
    

In other words, we must replace the commutator PQ - QP by a
"q-commutator".  This is the tip of a big iceberg: the whole
theory of Lie algebras has a "q-deformed" version where
q-commutators of various sorts take the place of commutators - and just
as Lie algebras go along with Lie groups, these q-deformed Lie algebras
go along with "quantum groups".

Now let's check to see if you're paying attention.  The alert reader
should have already noticed an incredible glaring contradiction in what
I've said!  I put it there on purpose, to make an important point.

No?  It's simple.  I said that making q different from 1 is like making
Planck's constant different from 0 - going from classical to quantum. 
People working on quantum groups often say this.  But look what we just
did!  We took the canonical commutation relations, which are \emph{already}
quantum-mechanical, and modified \emph{them} by making q different from 1. 
This is blatantly obvious if we put Planck's constant where it belongs
in the above formulas, instead of hiding it by setting it equal to 1. 
The momentum operator is really


$$

(P \psi )(x) = -i \hbar  \psi '(x)
$$
    
so the canonical commutation relations are 


$$

PQ - QP = -i \hbar 
$$
    
and when we use a q-derivative in the momentum operator they become


$$

PQ - qQP = -i \hbar .
$$
    
So there really are \emph{two} adjustable parameters floating around:
Planck's constant and this mysterious new "q"!

In fact, I've been complaining about this for years: it's only in
certain special contexts that you can think of the "q" or
"h" in quantum calculus as related to Planck's constant;
here's one in which they're obviously distinct.  So what's the physical
meaning of q-deformation?

One person to take a stab at this is Shahn Majid:

3) Shahn Majid, Foundations of Quantum Group Theory, Cambridge U.
Press, Cambridge, 2000.

In this book he says q is related to Newton's gravitational constant.
This would be cool, because then you could take your theory of quantum 
gravity, full of formulas like


$$

PQ - qQP = -i \hbar ,
$$
    
and make the quantum effects small by letting \hbar  \to  0, or make the
gravitational effects go away by setting q \to  1.  The problem is, I've
never seen a theory of quantum gravity like this!  Neither loop quantum
gravity nor string theory work this way.  

In fact, both loop quantum gravity people and string theorists agree  
on how to quantize gravity without matter in 3 spacetime dimensions.  
This is about the \emph{only} thing they agree on.  Quantum gravity in 3
dimensions is full of q-mathematics, and in this theory q is the
exponential of something involving the \emph{cosmological constant}.   
When q = 1 you get the quantum theory of flat 3d spacetime, since then
Einstein's equations say that spacetime is flat - this is a peculiarity
of 3 dimensions.  But when q is different from 1, you get the quantum
theory of a spacetime having constant curvature: a nonzero cosmological
constant means the vacuum has energy density, which curves spacetime!  

For some interesting new insights into this, see:

4) John Barrett, Geometrical measurements in three-dimensional
quantum gravity, available as <A HREF = "http://xxx.lanl.gov/abs/gr-qc/0203018">gr-qc/0203018</A>.


When we make the cosmological constant nonzero in 3d quantum gravity we
must replace the group SU(2) by the quantum group SU_{q}(2).  Based on
this, one can argue that quantum groups are misnamed - they should
really be called "cosmological groups".  Another way to put it
is this: ordinary groups are already perfectly sufficient for most of
quantum theory; quantum groups show up only in certain special contexts.

This goes to show that the deep inner meaning of the "q" in
quantum groups is still up for debate.  Mathematically it has a lot to
do with replacing groups by non-cocommutative Hopf algebras, whose
representations form a braided rather than symmetric monoidal category.
Here Majid and I agree completely: Planck's constant is about deviations
from commutativity while this "q" stuff is about deviations
from co-commutativity, or the failure of braidings to be symmetric.
Still, I think one should try to understand this more deeply.  The
amazing things that happen when q is a power of a prime number have got
to be an important clue!  I'll talk about this more next week.


\par\noindent\rule{\textwidth}{0.4pt}
\textbf{Addendum:} Toby Bartels brought up an important point in a reply
on the newsgroup sci.physics.research:


\begin{verbatim}

John Baez wrote in small part:

>In fact, I've been complaining about this for years: it's only in
>certain special contexts that you can think of the "q" or "h" in
>quantum calculus as related to Planck's constant; here's one in
>which they're obviously distinct.  So what's the physical meaning
>of q-deformation?

If q = exp h, then h couldn't possibly be Planck's constant,
because Planck's constant is not dimensionless.
(Or when you make it dimensionless, you generally fix its value,
and then it makes no sense to speak of varying q.)
To get a dimensionless constant for h, use (\hbar  G \Lambda /c^3),
where \hbar  = Planck's constant, G = Newton's constant,
\Lambda  = cosmological constant, and c = speed of light.

If you're coming from the POV where you only had 3 of these before,
with the 4th equal to 0 (or infinite in the case of c),
then you're going to view changing from q = 1 to some other q
as varying the value of the 4th constant.
Thus John (a quantum gravity theorist that often sets
\hbar , G, and c to fixed values) thinks that it's \Lambda ,
while Majid (who studied quantum field theory, which fixes \hbar  and c
and thinks of \Lambda  as a fixed QFT effect)[*] thinks that it's G.
But it is the dimensionless ratio that matters to everybody.

[*]I'm being presumptuous here.

-- Toby Bartels


\end{verbatim}
    

I replied:

\begin{verbatim}



Toby Bartels wrote:

>John Baez wrote in small part:

>>In fact, I've been complaining about this for years: it's only in
>>certain special contexts that you can think of the "q" or "h" in
>>quantum calculus as related to Planck's constant; here's one in
>>which they're obviously distinct.  So what's the physical meaning
>>of q-deformation?

>If q = exp h, then h couldn't possibly be Planck's constant,
>because Planck's constant is not dimensionless.
>(Or when you make it dimensionless, you generally fix its value,
>and then it makes no sense to speak of varying q.)

It might make sense to treat Planck's constant as dimensionless 
and still talk of varying its value.  

However, you're certainly right about this: in applications of 
q-mathematics to quantum gravity, we make Planck's constant 
dimensionless by combining it with Newton's gravitational 
constant, the speed of light, and the cosmological constant in this way:

>To get a dimensionless constant for h, use (\hbar  G \Lambda /c^3),
>where \hbar  = Planck's constant, G = Newton's constant,
>\Lambda  = cosmological constant, and c = speed of light.

... or something like that.  I think the formula depends on 
the dimension of spacetime, and so far it's in (2+1)d spacetime
that all the really solid applications of q-mathematics to
quantum gravity arise.  But the basic idea is robust, and it
doesn't depend on the dimension of spacetime:

We get a dimensionless constant by measuring the density of
the vacuum in Planck masses per Planck volume!

In other words: using \hbar  G and c we can construct units of
length, time, mass and so on - and then we can talk about the
energy density of the vacuum, \emph{measured in those units}, and
get something dimensionless.

This explains why q-mathematics only shows up when we do 
quantum gravity with a nonzero cosmological constant (or perhaps
matter).

>If you're coming from the POV where you only had 3 of these before,
>with the 4th equal to 0 (or infinite in the case of c),
>then you're going to view changing from q = 1 to some other q
>as varying the value of the 4th constant.
>Thus John (a quantum gravity theorist that often sets
>\hbar , G, and c to fixed values) thinks that it's \Lambda  [...]

Right.  Actually, the real reason I like to claim it's \Lambda 
is that this is the most surprising of the four alternatives.

\end{verbatim}
    
\par\noindent\rule{\textwidth}{0.4pt}
<em> When we try to pick out anything by itself, we find it hitched 
to everything else in the Universe.</em> - John Muir

\par\noindent\rule{\textwidth}{0.4pt}

% </A>
% </A>
% </A>
