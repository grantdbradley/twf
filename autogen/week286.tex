
% </A>
% </A>
% </A>
\week{December 8, 2009 }

This Week I'd like to start telling you about "rational 
homotopy theory".  But first: can you guess what this is a picture
of?  

<div align = "center">
<img border = "2" src = "abalos_undae.jpg">
</div>

I'll explain it at the end.  

So, what's "rational homotopy theory"?  One might naively
define it as the study of spaces whose homotopy groups are vector
spaces over the rational numbers.

But if you think about it, that's pretty weird!  

For example, the first homotopy group of a space X, usually called the
"fundamental group" of X and denoted \pi _{1}(X),
consists of equivalence classes of loops in X that start and end at
our favorite point.  Two loops count as equivalent if you can
continuously deform one until it looks like the other.  If you can do
this, we say these loops are "homotopic".

The fundamental group of the circle is Z, the group of integers.
The reason is that two loops drawn on the circle are homotopic
if and only if they wind around the same number of times - and
that number must be an integer!  You can walk around the block once 
and get back home.  You can walk around the block twice and get back
home.  You can even walk around the block -5 times and get back
home: the negative number just means you walk around the other way.
But you can't walk \emph{halfway} around the block and be back home!

But suppose you had a space whose fundamental group was Q, the
rational numbers.  Then you \emph{could} walk halfway around the
block and get back home.  That sounds pretty weird - nay, downright
impossible!

But part of why it sounds so weird is that it's not right.  We really
need some other "block" such that walking around \emph{that}
block \emph{twice} is homotopic to walking around the original
block \emph{once}.  This sounds more complicated...  but also more
possible.

Later in this post I'll describe a space called "the rational
circle", whose fundamental group is indeed Q.  Then you can see
how it actually works.

Anyway: spaces whose homotopy groups are rational vector spaces are
weird.  Why should we care about them?

We shouldn't!  In fact, the real point of rational homotopy theory
lies elsewhere. 

It's better not to think of rational homotopy theory as the study
of weird spaces whose homotopy groups are rational vector spaces.
It's better to think of it as the study of \emph{ordinary} spaces -
but viewed in a way that doesn't let us see their homotopy
groups, only their homotopy groups tensored with Q.  This process
turns their homotopy groups into rational vector spaces!

This is a common theme in algebraic topology.  We can think of
various kinds of homotopy theory either as the completely precise
study of rather strange spaces, or as the study of ordinary spaces
as seen through a blurry lens.  A blurry lens can be a good thing,
because it simplifies a complicated picture.

However, even \emph{this} way of thinking about rational homotopy
theory misses the point.  The real point is that rational vector
spaces come from the land of linear algebra, so rational homotopy
blends topology and linear algebra.  So does rational \emph{homology}
theory, but rational homotopy theory is deeper.  When we get into it,
we'll take lots of important concepts from linear algebra - like
commutative algebras, and Lie algebras, and Hopf algebras - and
study very interesting "homotopy versions" of these concepts.

By doing this, we'll vastly generalize linear algebra.  We'll wind up
with a whole new perspective... and we'll see applications to physics
ranging from classical field theory, to quantization, to supergravity!

And you should not be surprised that we're doing here is really
\emph{categorifying} linear algebra.

But more on that later.  Today, I want to start with the naive
viewpoint that rational homotopy theory is about spaces whose homotopy
groups are rational vector spaces.  

In algebraic topology, the really hard part is \emph{torsion}.  A group
element is "torsion" if you can add it to itself a bunch of times
and get zero.  So, for example, every element of a finite group
is torsion, but the group of integers is "torsion-free".

Look at some homotopy groups of spheres and you'll see what I mean:


$$

\pi _{3}(S^{2})   = Z  <br/>
\pi _{5}(S^{3})   =     Z/2 <br/>
\pi _{7}(S^{4})   = Z x Z/12 <br/>
\pi _{9}(S^{5})   =     Z/2  <br/>
\pi _{11}(S^{6})  = Z  <br/>
\pi _{13}(S^{7})  =     Z/2  <br/>
\pi _{15}(S^{8})  = Z x Z/120  <br/>
\pi _{17}(S^{9})  =     Z/8   <br/>
\pi _{19}(S^{10}) = Z x Z/8  <br/>
$$
    

These are the homotopy groups \pi _{2n-1}(S^{n}).  If
you were asked to make a guess about the torsion-free part of these
groups, you could easily formulate a conjecture: it's Z when n is
even, and trivial when n is odd.  And this is true.

But if you were asked to make a guess about the torsion part of these
groups, you'd find it a lot harder.  And indeed, nobody knows the full
story here.

This suggests trying to do a version of algebraic topology where we
systematically get rid of torsion.  We'll lose a lot of important
information, but things will get easy and fun - and still far from
trivial!

This is "rational homotopy theory".

How can we get rid of torsion?

Well, the nth homotopy group of a compact manifold, like a sphere, is
always finitely generated - and abelian when n > 1.  A finitely
generated abelian group always looks like Z^{n} \times  T
where T is finite.  All the torsion is in T, so to get rid of torsion
we can just throw out T.

But that doesn't work in general.  In general, the nth homotopy group 
of a space can be \emph{any} group when n = 1 - and any abelian group when 
n > 1.  

For an arbitrary abelian group, the torsion elements always form a 
subgroup, called the "torsion subgroup".   It's not true in general 
that an abelian group is the product of its torsion subgroup and 
some other group!  But, we can still kill off the torsion by modding
out by the torsion subgroup.

For a nonabelian group, the torsion elements don't necessarily form a
subgroup!  For example, take the free group generated by x and y, and
mod out by the relations x^{2} = y^{2} = 1.  Then x
and y are torsion elements, but xy is not.

I don't know any good way to kill off the torsion for an arbitrary
nonabelian group.  A lot of work on rational homotopy theory sidesteps
this issue by working only with "1-connected" spaces.  These are
spaces that are path-connected and also simply connected.  That means the
fundamental group is trivial - and the higher homotopy groups are
always abelian, so we don't have to worry about nonabelian groups.

Now, I've made it sound like the right way to "kill off
torsion" in an abelian group is to mod out by its torsion
subgroup.  This makes me wonder if there's a systematic way to take a
space X and turn it into a space X' such that \pi _{n}(X') is
\pi _{n}(X) mod its torsion subgroup.  Does anyone know?

But anyway, this is \emph{not} how we kill off torsion in rational homotopy 
theory!

Instead, here's what we do.  Abelian groups are the same as Z-modules
where Z is the ring of integers.  Since Z is commutative, we can take
tensor products of Z-modules.  In other words, we can take tensor
products of abelian groups.  And to kill off the torsion in an abelian
group, we just tensor it with the rational numbers!

I hope you see what this accomplishes.  Tensoring an abelian group G 
with the rational numbers gives a new abelian group Q \otimes  G that 
includes elements like

q \otimes  g

where g &isin; G and q is a rational number.   Any element g of G gives
an element of Q \otimes  G, namely 

1 \otimes  g

But we also get elements like

(1/2) \otimes  g

which acts like "half of g".  More generally, given any element of Q
tensor G, we're allowed to multiply it by any fraction.

Now, suppose g is a torsion element of G.  Then ng = 0 for some n, so

1 \otimes  ng = 0,

If we multiply both sides by 1/n, we get

1 \otimes  g = 0  

So, torsion elements of G get sent to zero in Q \otimes  G.  We've killed
the torsion.

But the great thing about this trick is that Q \otimes  G is even better
than a torsion-free abelian group.  It's a vector space over the
rational numbers!  So, we're not just killing off torsion.  We're
going from the world of abelian groups to the world of <em>linear
algebra</em>, which is notoriously well-behaved.

Next let me sketch how we can take a 1-connected space X and 
"rationalize" it, obtaining a new space X_{Q} with

\pi _{n}(X_{Q}) = \pi _{n}(X) \otimes  Q

for all n.  

Since we're doing homotopy theory, we can assume X is a "CW complex".
A space of this sort is built from balls.  To build a CW complex, we
start with some 0-balls - that is, points.  Then we take some 1-balls
- that is, intervals - and glue their boundaries to the 0-balls.  We
get a space that's just a graph.  Then we take some 2-balls - that is,
disks - and glue their boundaries to the space we've got so far.  Then
we take some 3-balls and glue their boundaries to what we've got so
far.  And so on, ad infinitum.  Any space is "weakly homotopy
equivalent" to a space of this sort, and that's good enough for us.

So, to rationalize X we should rationalize this whole procedure!  This
procedure relies on balls - and also spheres, since the boundary of a
ball is a sphere.  So, we should define a "rational n-ball"
and a "rational n-sphere", and then make sure that given a
CW complex, we can build a new space where each ball or sphere we used
has been replaced by a "rational" one!

I'll describe the rational n-sphere, since that's the fun part.
Even though we don't need it here, let's start with the case n = 1:
the "rational circle".  As mentioned earlier, this is a space whose 
fundamental group is Q.  Here's one way to build it.  

First, take an ordinary circle, and take a cylinder, and glue your
circle to the bottom of that cylinder.  But: make sure the 
circle goes around the bottom of the cylinder \emph{twice!} See what
this accomplishes?  It means that walking around your original circle
\emph{once} is homotopic to walking around the top of the cylinder
\emph{2 times}.

This solves our problem of how walking once around the block can be
the same as walking twice around some other block.

Then take another cylinder, and glue the top of your first cylinder 
to that.  But: make sure the top of your first cylinder winds around 
the bottom of this new one \emph{3 times}.  

Then take yet another cylinder.   Glue the top of your second
cylinder to the bottom of that - but make sure it wraps around
the bottom \emph{4 times}.  

And then do this forever... 

...and then take a little rest, since it's
very tiring to do an infinite amount of work.  Sit back and admire
your handiwork.  
The space you've built has Q as its fundamental
group, because for any loop g and any integer n, we've created a new
loop h such that g = nh.

Mathematicians call this general type of space a "telescope".
An ordinary hand telescope - the kind that pirates use - is built 
from cylinders of metal that fit into each other:

<div align = "center">
<img border = "2" src = "pirate_telescope.gif">
</div>

A mathematician's telescope is similar - but it's built from infinitely
many cylinders, and you can't collapse it, because they're attached to
each other in a complicated way.  This makes it really easy to spot a
mathematician in a roomful of pirates.

We can easily generalize this telescope idea to construct the
"rational n-sphere".  The point is that for each integer k,
there's a way to wrap the n-sphere around itself k times.  So, we can
use these to build an infinite telescope, just as we did for the
rational circle.  This telescope is a space whose homotopy groups are
those of the n-sphere, but tensored with the rational numbers.

A similar trick produces a rational n-ball, but this is less exciting,
since all the homotopy groups of the n-ball were trivial already -
it's contractible, after all.  The rational n-ball is still
contractible, but it's been modified so that its "boundary"
is a rational n-sphere.

Having rationalized our spheres and balls, we also need to check 
that the maps we used to build our CW complex extend in a canonical 
way from the spheres to the rational spheres.  But let's skip
this: in This Week's Finds we only do the fun part!

As you can see, the rationalized version of a nice little CW complex
is usually a huge nightmarish space.  This is a familiar tradeoff in
algebra topology: spaces with comprehensible homotopy groups almost
always look big and scary when we try to build them by gluing balls
together.  But it's a tradeoff algebraic topologists have learned to
accept.  There's more to life than whether a space \emph{looks} nice.

In particular, this rationalization process has a very nice abstract
characterization.  Suppose X is any 1-connected pointed space.  Then
we can define "a rationalization" of X to be any 1-connected pointed
space X' equipped with a map

X \to  X' 

satisfying two properties.  First, X' is a "rational space": a
1-connected pointed space whose homotopy groups are rational 
vector spaces.  Second, the induced map
   
Q \otimes  \pi _{n}(X) \to  Q \otimes  \pi _{n}(X')

is isomorphism for all n. 

It turns out that the rationalization of a space is unique up to weak
homotopy equivalence.  And we can construct it for CW complexes as I
just explained.

Okay.  So far I've been treating rational homotopy theory as the study
of weird "rational" spaces.  And I've showed you how to turn
any space into one of these.  But as I already admitted, this
misses the point.

To come closer to the point, we need to recall an amazing old
theorem due to J. H. C. Whitehead, which says a map

f: X \to  Y 

between connected CW complexes is a homotopy equivalence if and only 
if the induced maps

\pi _{n}(f): \pi _{n}(X) \to  \pi _{n}(Y)

are isomorphisms for all n.  This is why for more general connected
spaces we define any map that induces isomorphisms on homotopy groups
to be a "weak homotopy equivalence".  Even better, every space is
weakly homotopy equivalent to a CW complex!  So, if we formally throw
in inverses to all weak homotopy equivalences, we get a category
called where every space is \emph{isomorphic} to a CW complex.  This is
called the "homotopy category".

These ideas are very powerful, so it's good to generalize them to
rational homotopy theory.  So now suppose X and Y are 1-connected
pointed spaces.  And let's say a map

f: X \to  Y

is a "rational homotopy equivalence" if the induced maps
on rational homotopy groups

Q \otimes  \pi _{n}(f): Q \otimes  \pi _{n}(X) \to  Q
\otimes  \pi _{n}(Y)

are isomorphisms for all n.  There's a nice category where we formally
throw in inverses to all rational homotopy equivalences.  This is 
called the "rational homotopy category".  

In the rational homotopy category, we're studying ordinary spaces 
through a blurry lens, because two spaces that have a rational
homotopy equivalence between them look the same.  

But the rational homotopy category is equivalent to a subcategory of
the usual homotopy category: namely, the subcategory consisting of
rational spaces and all morphisms between those!  So, we can also
say we're studying strange spaces, but with complete precision - or
at least, the usual level of precision in homotopy theory.

To learn more, I urge you to grab this and take a look:

2) Kathryn Hess, Rational homotopy theory: a brief introduction,
in Interactions Between Homotopy Theory and Algebra, ed. 
Luchezar L. Avramov, Contemp. Math 436, AMS, Providence, Rhode
Island, 2007.  Also available as <a href = "http://arxiv.org/abs/math.AT/0604626">arXiv:math.AT/0604626</a>.

For even more detail, I recommend:

3) Yves Felix, Stephen Halperin and Jean-Claude Thomas, Rational
Homotopy Theory, Springer, Berlin, 2005.

I'll give more references later.  In the weeks to come, we'll see two
very different descriptions of the rational homotopy category, which
were both discovered by Daniel Quillen back in 1969:

4) Daniel Quillen, Rational homotopy theory, Ann. Math. 90 (1969), 
205-295.   

It's these other descriptions that make the subject really interesting.
One is based on a homotopy version of Lie algebras.  The other is
based on a homotopy version of commutative algebras!

In the first sentence of his paper, Quillen defines the rational
homotopy category.  But he does it a bit differently than I just did.
This is worth understanding.  He says it's "the category obtained from
the category of 1-connected pointed spaces by localizing with respect
to the family of maps which are isomorphisms modulo the class in the
sense of Serre of torsion abelian groups".

Let me say this with less jargon.  Start with the category of
1-connected pointed spaces.  Throw in formal inverses of all maps

f: X \to  Y

for which the induced maps

\pi _{n}(f): \pi _{n}(X) \to  \pi _{n}(Y)

have kernels and cokernels where all elements are torsion.  This gives
the rational homotopy category!

I'll conclude with a few remarks that would have been a bit too
distracting earlier.

First: I discussed rational homotopy theory only for 1-connected
spaces.  This is great as far as the connection to algebra goes.  But
in terms of topology it's a bit sad.  Sometimes people go a step
further and work with "nilpotent" spaces.  These are spaces where the
fundamental group is nilpotent and acts nilpotently on the higher
homotopy groups.

Second: the rational circle is an interesting space.  As we've seen,
it's a space with the rational numbers as its fundamental group.  All
its other homotopy groups are trivial, since that's already true for
the circle.

Any space with G as its nth homotopy group and every other homotopy
group being trivial is called "the Eilenberg-Mac Lane space K(G,n)".
We're allowed use the word "the", since this space is unique up to
weak homotopy equivalence.  So, the rational 1-sphere is K(Q,1).

I've talked about lots of different Eilenberg-Mac Lane spaces in 
This Week's Finds, and they're all collected here:

3) John Baez, Generalized cohomology theories, Eilenberg-Mac Lane
spaces, classifying spaces and K(Z,n), and other examples of
classifying spaces.  Available at
<a href = "http://math.ucr.edu/home/baez/calgary/BG.html">http://math.ucr.edu/home/baez/calgary/BG.html</a>

Now you can add K(Q,1) to your collection!

Third: in case you're wondering about Quillen's jargon: by
"localizing" he means the process of taking a category and throwing in
formal inverses to a bunch of morphisms.  This is an important way of
simplifying categories.  It lets us count slightly different objects
as the same.  

A "Serre class" of abelian groups is a bunch of abelian groups such
that whenever A and C are in this class, and the sequence

0 \to  A \to  B \to  C \to  0

is exact, then B is in this class too.  The idea is that we think of
abelian groups in the Serre class as "small", or "negligible".  For
example: the class of finite abelian groups, or the class of torsion
abelian groups.  We can localize the category of abelian groups by
throwing in an inverse for any morphism whose kernel and cokernel are
in the Serre class.

If you like abelian categories, you can generalize this "Serre
class" idea from the category of abelian groups to other abelian
categories.  

There's also much more to say about localization!  Try this for
starters:
 
5) nLab, Localization, <a href = "http://ncatlab.org/nlab/show/localization">http://ncatlab.org/nlab/show/localization</a>

Besides doing rational homotopy theory, we can use localization to
take homotopy theory and "localize at the prime p".  This is a way to
focus special attention on the "p-torsion" in our homotopy groups:
that is, the elements that give zero when you multiply them by a power
of p.

Finally, what about the picture at the beginning of this Week's Finds?
It shows sand dunes in a region called Abalos Undae near the north
pole of Mars:

5) HiRISE (High Resolution Imaging Science Experiments), 
Dunes in Abalos Undae, <a href = "http://hirise.lpl.arizona.edu/PSP_010219_2785">http://hirise.lpl.arizona.edu/PSP_010219_2785</a>

The photo covers a strip about 1.2 kilometers across.  As the HiRISE
satellite sweeps over Mars it takes incredibly detailed photos like
this.  Here's the description on the HiRISE website:

\begin{quote}
     The Abalos Undae dune field stretches westward, away from a
     portion (Abalos Colles) of the ice-rich north polar layered
     deposits that is separated from the main Planum Boreum dome by
     two large chasms.  These dunes are special because their sands 
     may have been derived from erosion of the Rupes Tenuis unit (the
     lowest stratigraphic unit in Planum Boreum, beneath the icier
     layers) during formation of the chasms.  Some researches have
     argued that these chasms were formed partially by melting of the
     polar ice.

     The enhanced color data illuminate differences in composition.  
     The dunes appear blueish because of their basaltic composition, 
     while the reddish-white areas are probably covered in dust.  
     Upon close inspection, tiny ripples and grooves are visible on 
     the surface of the dunes; these features are formed by wind 
     action, as are the dunes themselves.

     It is possible that the dunes are no longer migrating (the
     process of dune formation forces dunes to move in the direction
     of the main winds) and that the tiny ripples are the only active
     parts of the dunes today.
\end{quote}

\par\noindent\rule{\textwidth}{0.4pt}
\textbf{Addenda:} 
The rational circle is pretty hard to draw, but Kenneth Baker did
a great job of illustrating some early stages of its construction:


1) Kenneth Baker, A (reverse) rational circle?, on his blog Sketches
of Topology at <a href =
"http://sketchesoftopology.wordpress.com/2009/12/10/a-rational-circle/">http://sketchesoftopology.wordpress.com/2009/12/10/a-rational-circle/</a>

<br/><br/>
<div align = "center">
<img border = "2" src = "rational_circle_kenneth_baker_1a.jpg">
</div>
<br/><br/>

The right edge of the red band is our original circle, drawn in a
tricky way to make the whole picture more manageable.  The left edge
of the red band is homotopic to 2 times the loop traced out by this
original circle.  The left of the orange band is homotopic to 6 times
it, and the left edge of the green band is homotopic to 24 times it!

If we remove the red band we see how the orange one wrapped around
it 3 times:

<br/><br/>
<div align = "center">
<img border = "2" src = "rational_circle_kenneth_baker_1b.jpg">
</div>
<br/><br/>

and if we remove the yellow band we see how the green one wrapped
around it 4 times:

<br/><br/>
<div align = "center">
<img border = "2" src = "rational_circle_kenneth_baker_1c.jpg">
</div>
<br/><br/>

Here's a kind of cross-section that reveals more about what's going
on:

<br/><br/>
<div align = "center">
<img border = "2" src = "rational_circle_kenneth_baker_2a.jpg">
</div>
<br/><br/>

Or in stages:

<br/><br/>
<div align = "center">
<img border = "2" src = "rational_circle_kenneth_baker_2b.jpg">
</div>
<br/><br/>

You're probably curious about how Kenneth Baker drew these pictures.
Here's how:

\begin{quote}

These pictures are done using Rhino 3D. Actually I'm using the beta
version of their port to OS X. There's a function (called Flow) that
lets you map a "spine" of an object to another curve to tell it how to
deform the object. This is how I went from the chopped open version to
the round one. It's also how I managed to make the orange wrap around
the green and the red wrap around the orange.

\end{quote}

On the \emph{n}-Category Caf&eacute;, Tom Leinster raised a useful point:

\begin{quote}

Something that bothered me for a while about rational homotopy, as an
outsider, was this phrase "the homotopy groups are rational vector
spaces".  A priori the (higher) homotopy groups are abelian groups.
So does this mean that there exists a rational vector space structure?
That there exists a unique one?  That one is somehow specified?


In fact, these questions are unnecessary, for the following reason.
(I think this was explained to me by the James who sometimes comments
here.)  Fact:

\begin{quote}
    Let A be an abelian group.  Then A has the structure of a rational
    vector space in at most one way.
\end{quote}

So, despite appearances, being a rational vector space is a property
of abelian groups, not extra structure.

The proof is fairly straightforward, I think.  If A admits a rational
vector space structure then

\begin{quote}
    for all a &isin; A and all positive integers n, there exists a unique b 
&isin; A such that nb=a.
\end{quote}

And this condition clearly determines what the scalar multiplication
over Q must be.  (In fact, it's an 'iff': an abelian group admits the
structure of a rational vector space if and only if it satisfies this
condition.)

\end{quote}

Todd Trimble added:

\begin{quote}

Yes. A rational vector space is the same as a divisible torsionfree
abelian group. Incidentally, an abelian group is divisible if and only
if it is injective in the category of abelian groups, and is
torsionfree if and only if it is flat in the category of abelian
groups.

\end{quote}

For more discussion visit the <a href =
"http://golem.ph.utexas.edu/category/2009/12/this_weeks_finds_in_mathematic_47.html">\emph{n}-Category
Caf&eacute;</a>.

\par\noindent\rule{\textwidth}{0.4pt}
<em>...the pursuit of science is more than the pursuit of understanding. 
It is driven by the creative urge, the urge to construct a vision, a 
map, a picture of the world that gives the world a little more beauty 
and coherence than it had before.  Somewhere in the child that urge 
is born.</em> - John Archibald Wheeler

\par\noindent\rule{\textwidth}{0.4pt}

% </A>
% </A>
% </A>
