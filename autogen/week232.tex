
% </A>
% </A>
% </A>
\week{May 18, 2006 }

I'm at the Perimeter Institute now.  It's great to see how it's
developed since I first saw their <A HREF = "PI1.jpg">new building</A> 
back in 2004 (see "<A HREF = "week208.html">week208</A>" 
for the story).

There's now a busy schedule of seminars and weekly colloquia, with
string theorists and loop quantum gravity people coexisting happily.
Their program of Superstring Quartets features some really hot bands,
like the Julliard and Emerson - unfortunately not playing while I'm
here.  The Black Hole Bistro serves elegant lunches and dinners, there
are at least two espresso machines on each floor, and my friend
Eugenia Cheng will be happy to hear that they still have a piano
available (after 6 pm).

But don't get the impression that it's overly sophisticated: 
there are also a couple of guys constantly playing 
<A HREF = "foosball.jpg">foosball</A> in the Feynman Lounge.

Since I'm here, I should talk about quantum gravity - so I will.
But first, let's have the astronomy picture of the week.  

This week it comes, not from outer space, but beneath the surface of 
the South Pole:

<DIV ALIGN = CENTER>
<A HREF = "http://icecube.wisc.edu/gallery/detector_concepts/ceren_hires">
<IMG SRC = "ice_cube.jpg">
% </A>
</DIV>
1) Steve Yunck / NSF, Cerenkov light passing through the IceCube neutrino
detector, <A HREF = "http://icecube.wisc.edu/gallery/detector_concepts/ceren_hires">http://icecube.wisc.edu/gallery/detector_concepts/ceren_hires</A>

This is an artist's impression of a huge neutrino observatory called
"IceCube".  (Maybe they left out the space here so the rap
star of that name doesn't sue them for trademark infringement, or go
down there and shoot them.)

IceCube is being built in the beautifully clear 18,000-year old ice deep
beneath the Amundsen-Scott South Pole Station.  When a high-energy 
neutrino hits a water molecule, sometimes the collision produces a 
muon zipping faster than the speed of light in ice.  This in turn 
produces something like a sonic boom, but with light instead of sound.  
It's called "Cerenkov radiation", and it's the blue light in the picture.  
This will be detected by an array of 5000 photomultiplier tubes - those 
gadgets hanging on electrical cables.  

One thing the artist's impression doesn't show is that 
IceCube is amazingly large.  The whole
array is a cubic kilometer in size!  It will encompass the already
existing AMANDA detector, itself 10,000 meters
tall, shown as a yellow cylinder here with
a neutrino zipping through:

<DIV ALIGN = CENTER>
<A HREF = 
"http://icecube.wisc.edu/gallery/detector_concepts/icecubeencomp_300">
<IMG SRC = "ice_cube_AMANDA.jpg">
% </A>
</DIV>

2) Darwin Rianto / NSF, Comparison of AMANDA and IceCube, 
<A HREF = "http://icecube.wisc.edu/gallery/detector_concepts/icecubeencomp_300">http://icecube.wisc.edu/gallery/detector_concepts/icecubeencomp_300</A>

Even the very top of IceCube is 1.4 kilometers beneath the snowy 
Antarctic surface, to minimize the effect of stray cosmic rays.  
The station on top looks like this - not very cozy, I'd say:

<DIV ALIGN = CENTER>
<A HREF = 
"http://icecube.wisc.edu/gallery/antarctica/PC140287_300">
<IMG SRC = "south_pole_station.jpg">
% </A>
</DIV>

3) Robert G. Stokstad / NSF, South Pole Station, 
<A HREF = "http://icecube.wisc.edu/gallery/antarctica/PC140287_300">http://icecube.wisc.edu/gallery/antarctica/PC140287_300</A>

I heard about IceCube from Adrian Burd, one of the old-timers 
who used to post a lot on sci.physics, a former cosmologist turned
oceanographer who recently visited Antarctica as part of an 
NSF-run field course.  He ran into some people working on IceCube. 
It sounds like an interesting community down there!  You can read 
about it in their newspaper, the Antarctic Sun.  For example:

4) Ice Cube turns up the heat, The Antarctic Sun, January 29, 2006,
<A HREF = "http://antarcticsun.usap.gov/2005-2006/contentHandler.cfm?id=959">http://antarcticsun.usap.gov/2005-2006/contentHandler.cfm?id=959</A>

For more on IceCube and Amanda, these are fun to read:

5) Francis Halzen, Ice fishing for neutrinos, 
<A HREF = "http://icecube.berkeley.edu/amanda/ice-fishing.html">http://icecube.berkeley.edu/amanda/ice-fishing.html</A>

6) Katie Yurkiewicz, Extreme neutrinos, Symmetry, volume 1 issue 1, 
November 2004, <A HREF = "http://symmetrymagazine.org/cms/?pid=1000014">http://symmetrymagazine.org/cms/?pid=1000014</A>

For some of AMANDA's results, including a map of the sky as
seen in neutrinos, try this:

7) M. Ackermann et al, Search for extraterrestrial point sources of
high energy neutrinos with AMANDA-II using data collected in 2000-2002,
available as <A HREF = "http://xxx.lanl.gov/abs/astro-ph/0412347">astro-ph/0412347</A>.

For much more, try these:

8) AMANDA II Project, <A HREF = "http://amanda.uci.edu/">http://amanda.uci.edu/</A>

9) Welcome to IceCube, <A HREF = "http://icecube.wisc.edu/">http://icecube.wisc.edu/</A>

And now, on to gravity.  

You may have heard of the gravitational 3-body problem.  Well, 
Richard Montgomery (famous from "<A HREF = "week181.html">week181</A>") recently pointed out 
this movie of the 60-body problem:

10) Davide L. Ferrario, Periodic orbits for the 60-body problem,
<A HREF = "http://www.matapp.unimib.it/~ferrario/mov/index.html">http://www.matapp.unimib.it/~ferrario/mov/index.html</A>

60 equal masses do a complicated dance while always preserving 
icosahedral symmetry!  First 12 groups of 5 swing past each other, 
then 20 groups of 3.  If you want to know how he found these solutions, 
read this:

11) Davide L. Ferrario and S. Terracini, On the existence of collisionless 
equivariant minimizers for the classical n-body problem.  Invent. Math.
155 (2004), 305-362.

It's quite math-intensive - though just what you'd expect if you 
know this sort of thing: they use the G-equivariant topology of loop 
spaces, where G is the symmetry group in question (here the icosahedral 
group), to prove the existence of action-minimizing loops with given 
symmetry properties.

Next, I'd like to say a little about point particles in 3d quantum 
gravity, and some recent work with Alissa Crans, Derek Wise and Alejandro 
Perez on string-like defects in 4d topological gravity:

12) John Baez, Derek Wise and Alissa Crans, Exotic statistics for strings
in 4d BF theory, available as <A HREF = "http://xxx.lanl.gov/abs/gr-qc/0603085">gr-qc/0603085</A>.

13) John Baez and Alejandro Perez, Quantization of strings and branes 
coupled to BF theory, available as <A HREF = "http://xxx.lanl.gov/abs/gr-qc/0605087">gr-qc/0605087</A>.

(Jeffrey Morton is also involved in this project, a bit more on the 
n-category side of things, but that aspect is top secret for now.)

In "<A HREF = "week222.html">week222</A>" I listed a bunch of cool papers on 3d quantum gravity,
but I didn't really explain them.  What we're trying to do now is 
generalize this work to higher dimensions.  But first, let me start
by explaining the wonders of 3d quantum gravity.  

The main wonder is that we actually understand it!  The classical
version of general relativity is exactly solvable when spacetime has
dimension 3, and so is the quantum version.  Most of the wonders I 
want to discuss are already visible in the classical theory, where 
they are easier to understand, so I'll focus on the classical case.

A nice formulation of general relativity in 3 dimensions uses a
"Lorentz connection" A and a "triad field" e.  This is a gauge theory
where the gauge group is SO(2,1), the Lorentz group for 3d spacetime.
If we're feeling lowbrow we can think of both A and e as so(2,1)-valued 
1-forms on the 3-manifold M that describes spacetime.  The action for 
this theory is:

&int;_{M}  tr(e ^ F)

where F is the curvature of A.  If you work out the equations of
motion one of them says that F = 0, so our connection A is flat.  The
other, d_{A}e = 0, says A is basically just the
Levi-Civita connection.

This is exactly what we want, because in the absence of matter, 
general relativity in 3 dimensions says spacetime is \emph{flat}.

A fellow named Phillipp de Sousa Gerbert came up with an interesting
way to couple point particles to this formulation of quantum gravity:

14) Phillipp de Sousa Gerbert, On spin and (quantum) gravity in 2+1
dimensions, Nuclear Physics B346 (1990), 440-472.

He actually did it for particles with spin, but I'll just do the 
spin-zero case.  

The idea is to fix a 1-dimensional submanifold W in our 3-manifold M 
and think of it as the worldlines of some particles.  Put so(2,1)-valued 
functions p and q on these worldlines - think of these as giving the 
particles' momentum and position as a function of time.  

Huh?  Well, normally we think of position and momentum as vectors.  
In special relativity, "position" means "position in spacetime",
and "momentum" means "energy-momentum".  We can think of both of
these as vectors in Minkowksi spacetime.  But in 3 dimensions, 
Minkowski spacetime is naturally identified with the Lorentz Lie 
algebra so(2,1).  So, it makes sense to think of q and p as elements
of so(2,1) which vary from point to point along the particle's 
worldline.

To couple our point particles to gravity, we then add a term to 
the action like this:

S   =   &int;_{M}  tr(e ^ F)   -   
&int;_{W}  tr((e + d_{A}q) ^ p)

Now if you vary the e field you get a field equation saying that 

F = p \delta _{W}

Here \delta _{W} is like the Dirac delta function 
of the worldline W;
it's a distributional 2-form defined by requiring that

&int;_{W}  X   =  &int;_{M}  (X ^ \delta _{W}) 

for any smooth 1-form X on W.  This sort of "distributional
differential form" is also called a "current", and you
can read about them in the classic tome by Choquet-Bruhat et al.  But
the main point is that the field equation

F = p \delta _{W}

says our connection on spacetime is flat except along the worldlines
of our particles, where the curvature is a kind of "\delta 
function".  This is nice, because that's what we expect in 3d
gravity: if you have a particle, spacetime will be flat everywhere
except right at the particle, where it will have a singularity like
the tip of a cone.

A cone, you see, is intrinsically flat except at its tip: that's why 
you can curl paper into a cone without crinkling it!

So, our spacetime is flat except along the particles' worldlines, and
there it's like a cone.  The "deficit angle" of this cone -
the angle of the slice you'd need to cut out to curl some paper into
this cone - is specified by the particle's momentum p.

Since delta functions are a bit scary, it's actually better to work with 
an "integrated" form of the equation

F = -p \delta _{W}

The integrated form says that if we parallel transport a little tangent
vector around a little loop circling our particle's worldline, it gets
rotated and/or Lorentz transformed by the element

exp(p) 

in SO(2,1).  This will be a rotation if the particle's momentum p is 
timelike, as it is for normal particles.   Again, that's just as it 
should be: if you parallel transport a little arrow around a massive 
particle in 3d gravity, it gets rotated!

If p is timelike, our particle is a tachyon and exp(p) is a Lorentz
boost.  And so on... we get the usual classification of particles
corresponding to various choices of p:

<DIV ALIGN = CENTER>
<IMG SRC = "loopbraid/lightcone.jpg">
</DIV>


There are other equations of motion, obtained by varying other fields,
but all I want to note is the one you get by varying q:

d_{A} p = 0

This says that the momentum p is covariantly constant along the
particles' worldlines.  So, momentum is conserved!

The really cool part is the relation between the Lie algebra element p 
and the group element exp(p).  Originally we thought of p as momentum - 
but there's a sense in which exp(p) is the momentum that really counts!  

First, exp(p) is what we actually detect by parallel transporting a 
little arrow around our particle.  

Second, suppose we let two particles collide and form a new one:


$$

    p       p'
     \     /
      \   /  
       \ /  
        |   
        |   
        |    
        p"
$$
    

Now our worldlines don't form a submanifold anymore, but if we keep 
our wits about us, we can see that everything still makes sense, and
we get momentum conservation in this form:

exp(p") = exp(p) exp(p')

since little loops going around the two incoming particles can fuse to 
form a loop going around the outgoing particle.  Note that we're getting
conservation of the \emph{group-valued} momentum, not the Lie-algebra-valued
momentum - we don't have

p" = p + p'

So, conservation of energy-momentum is getting modified by gravitational
effects!  This goes by the name of "doubly special relativity":

15) Laurent Freidel, Jerzy Kowalski-Glikman and Lee Smolin, 
2+1 gravity and doubly special relativity, Phys. Rev. D69 (2004) 
044001.  Also available as <A HREF = "http://xxx.lanl.gov/abs/hep-th/0307085">hep-th/0307085</A>.

This effect is a bit less shocking if we put the units back in.  I've
secretly been setting 4\pi G = 1, where G is Newton's gravitational 
constant.  If we put that constant back in - let's call it k instead 
of 4\pi G - we get

exp(kp") = exp(kp) exp(kp') 

or if you expand things out:

p" = p + p' + (k/2) [p,p'] + terms of order k^{2} and higher... 

So, as long as the momenta are small compared to the Planck mass, 
the usual law of conservation of momentum

p" = p + p'

\emph{almost} holds!   But, for large momenta this law breaks down - we 
must think of momentum as group-valued if we want it to be conserved!

I think this is incredibly cool: as we turn on gravity, the usual
"flat" momentum space curls up into a group, and we need to
\emph{multiply} momenta in this group, instead of \emph{add} them
in the Lie algebra.  We can think of this group has having a
"radius" of 1/k, so it's really big and almost flat when the
strength of gravity is small.  In this limit, multiplication in the
group reduces to addition in the Lie algebra.

I should point out that this effect is purely classical!  It's still
there when we quantize the theory, but it only depends on the
gravitational constant, not Planck's constant.  Indeed, in 3d quantum
gravity, we can build a unit of mass using just G and c: we don't need
\hbar .  This unit is the mass that curls space into an infinitely
skinny cone!  It would be a bit misleading to call it "Planck
mass", but it's the maximum possible mass.  Any mass bigger than
this acts like a \emph{negative} mass.  That's because the corresponding
group-valued momenta "wrap around" in the group SO(2,1).

We also get another cool effect - exotic statistics.  In the absence 
of gravitational or quantum effects, when you switch two particles, 
you just switch their momenta:

(p, p') |\to  (p', p) 

But in 3d gravity, you can think of this process of switching 
particles as a braid: 


\begin{verbatim}

          \   /
           \ / 
            \
           / \
          /   \
\end{verbatim}
    

and if you work out what happens to their group-valued momenta,
say  

g  = exp(kp) <br>
g' = exp(kp')

it turns out that one momentum gets conjugated by the other:

(g, g') |\to  (gg'g^{-1}, g)

To see this, remember that we get these group elements by doing
parallel transport around loops that circle our particles.  When
we move our particles, the loops get dragged along, like this:

<DIV ALIGN = CENTER>
<IMG SRC = "loopbraid/particles_braiding.jpg">
</DIV>

Note that the left-hand red loop moves until it looks just
like the right one did initially, but the right-hand one gets
wrapped around the left one.  If you ponder this carefully, 
and you know some math, you can see it yields this:

(g, g') |\to  (gg'g^{-1}, g)

So, the process of braiding two particles around each other has a
nontrivial effect on their momenta.  In particular, if you braid two
particles around other twice they don't wind up in their original
state!  

Thus, our particles are neither bosons nor fermions, but
"nonabelian anyons" - the process of switching them is
governed not by the permutation group, but by the braid group.  
But again, if you expand things out in powers of k you'll see this 
effect is only noticeable for large momenta:

(p, p') |\to  (p' + k[p,p'] + higher order terms..., p)

Summarizing, we see quantum gravity is lots of fun in 3 dimensions:
it's easy to introduce point particles, and they have group-valued
momentum, which gives rise to doubly special relativity and braid
group statistics.

Now, what happens when we go from 3 dimensions to 4 dimensions?

Well, we can write down the same sort of theory:

S   =   
&int;_{M}  tr(B ^ F)   -  
&int;_{W}  tr((B + d_{A}q) ^ p)

The only visible difference is that what I'd been calling
"e" is now called "B", so you can see why folks
call this "BF theory". 

But more importantly, now M is an 4-dimensional spacetime and W is an 
2-dimensional "worldsheet".  A is again a Lorentz connection, which 
we can think of as an so(3,1)-valued 1-form.   B is an so(3,1)-valued 
2-form.  p is an so(3,1)-valued function on the worldsheet W.  q is an 
so(3,1)-valued 1-form on W.

So, only a few numbers have changed... so everything works very
similarly!  The big difference is that instead of spacetime having a
conical singularity along the worldline of a \emph{particle}, now
it's singular along the worldsheet of a \emph{string}.  When I call
it a "string", I'm not trying to say it behaves like the
ones they think about in string theory - at least superficially, it's
a different sort of theory, a purely topological theory. But, we've
got these closed loops that move around, split and join, and trace out
surfaces in spacetime.

They can also braid around each other in topologically nontrivial
ways, as shown in this "movie": 

<DIV ALIGN = CENTER>
<IMG SRC = "loopbraid/loopbraid.jpg">
</DIV>

(By the way, all the math pictures this week were drawn by 
Derek for our paper.)

So, we get exotic statistics as before, but now they are governed not by
the braid group but by the "loop braid group", which keeps
track of all the ways we can move a bunch of circles around in 3d
space.  Let's take our spacetime M to be R^{4}, to keep things simple.
Then our circles can move around in R^{3}... and there are 
two basic ways we can switch two of them: move them around each other, 
or pass one \emph{through} the other, like this:

<DIV ALIGN = CENTER>
<IMG SRC = "loopbraid/loop_switch.jpg">
</DIV>

If we just move them around each other, they might as well have been 
point particles: we get a copy of the permutation group, and all we see 
are ordinary statistics.  But when we consider all the ways of passing
them through each other, we get a copy of the braid group!  

When we allow ourselves both motions, we get a group called the
"loop braid group" or "braid permutation group" -
and one thing Alissa Derek and I did was to get a presentation of this
group.  This is an example of a "motion group": just as the
motion group of point particles in the plane is the braid group, and
motion group of point particles in R^{3} is the permutation group, the
motion group of strings in R^{3} is the loop braid group.

As before, our strings have group-valued momenta: we can get an 
element of the Lorentz group SO(3,1) by parallel transporting a
little tangent vector around a string.  And, we can see how 
different ways of switching our strings affect the momenta.  
When we move two strings around each other, their momenta switch
in the usual way:

(g, g') |\to  (g', g)

but when we move one through the other, one momentum gets conjugated
by the other:

(g, g') |\to  (gg'g^{-1}, g)

So, we have exotic statistics, but you can only notice them if you
can pass one string through another!

In the paper with Alejandro, we go further and begin the project of
quantizing these funny strings, using ideas from loop quantum gravity.
Loop quantum gravity has its share of problems, but it works perfectly
well for 3d quantum gravity, and matches the spin foam picture of this
theory.  People have sort of believed this for a long time, but 
Alejandro demonstrated this quite carefully in a recent paper with 
Karim Noui:

15) Karim Noui and Alejandro Perez, 
Dynamics of loop quantum gravity and spin foam models in three dimensions,
to appear in the proceedings of
the Third International Symposium on Quantum Theory and Symmetries 
(QTS3), available as <A HREF = "http://xxx.lanl.gov/abs/gr-qc/0402112">gr-qc/0402112</A>.

The reason everything works so nicely is that the equations of motion
say the connection is flat.  Since the same is true in BF theory in 
higher dimensions, we expect that the loop quantization and spin foam
quantization of the theory I'm talking about now should also work well.

We find that we get a Hilbert space with a basis of "string spin networks",
meaning spin networks that can have loose ends on the stringy defects.

So, there's some weird blend of loop quantum gravity and strings going
on here - but I don't really understand the relation to ordinary
string theory, if any.  It's possible that I can get a topological
string theory (some sort of well-defined mathematical gadget) which
describes these stringy defects, and that would be quite interesting.

But, I spoke about this today at the Perimeter Institute, and Malcolm
Perry said that instead of "strings" I should call these
guys (n-2)-branes, because the connection has conical singularities on
them, "which is what one would expect for any respectable
(n-2)-brane".

I will talk to him more about this and try to pick his, umm, branes.
In fact I took my very first GR course from him, back when he was a
postdoc at Princeton and I was a measly undergraduate.  I was too
scared to ask him many questions then.  I'm a bit less scared now,
but I've still got a lot to learn.  Tomorrow he's giving a talk about
this:

17) David S. Berman, Malcolm J. Perry, M-theory and the string genus
expansion, Phys. Lett. B635 (2006) 131-135.  Also available as 
<A HREF = "http://arxiv.org/abs/hep-th/0601141">hep-th/0601141</A>.


\par\noindent\rule{\textwidth}{0.4pt}
\textbf{Addenda:} Here's an email from Greg Egan, and my reply:

\begin{quote}
  John Baez wrote:


\begin{verbatim}

  > The really cool part is the relation between the Lie algebra 
  > element p and the group element exp(p).  Originally we thought 
  > of p as momentum - but there's a sense in which exp(p) is the 
  > momentum that really counts!
\end{verbatim}
    

  Would it be correct to assume that the ordinary tangent vector p still
  transforms in the usual way?  In other words, suppose I'm living in a
  2+1 dimensional universe, and there's a point particle with rest mass m
  and hence energy-momentum vector in its rest frame of p=me_{0}.  If I
  cross its world line with a certain relative velocity, there's an
  element g of SO(2,1) which tells me how to map the particle's tangent
  space to my own.  Would I measure the particle's energy-momentum to be
  p'=gp?  (e.g. if I used the particle to do work in my own rest frame)
  Would there still be no upper bound on the total energy, i.e. by making
  our relative velocity close enough to c, I could measure the particle's
  kinetic energy to be as high as I wished?

  I guess I'm trying to clarify whether the usual Lorentz transformation
  of the tangent space has somehow been completely invalidated for
  extreme boosts, or whether it's just a matter of there being a second
  definition of "momentum" (defined in terms of the Hamiltonian) which
  transforms differently and is the appropriate thing to consider in
  gravitational contexts.

  In other words, does the cut-off mass apply only to the deficit angle,
  and do boosts still allow me to measure (by non-gravitational means)
  arbitrarily large energies (at least in the classical theory)?
\end{quote}

I replied:

\begin{quote}
  Greg Egan wrote:


\begin{verbatim}

  >John Baez wrote:

  >>The really cool part is the relation between the Lie algebra 
  >>element p and the group element exp(p).  Originally we thought 
  >>of p as momentum -  but there's a sense in which exp(p) is the 
  >>momentum that really counts!

  >Would it be correct to assume that the ordinary tangent vector p 
  >still transforms in the usual way?  
\end{verbatim}
    

  Hi!  Yes, it would.


\begin{verbatim}

  >In other words, suppose I'm living in a 2+1 dimensional universe, 
  >and there's a point particle with rest mass m and hence 
  >energy-momentum vector in its rest frame of p=m e_0.  If I
  >cross its world line with a certain relative velocity, there's 
  >an element g of SO(2,1) which tells me how to map the particle's 
  >tangent space to my own.  Would I measure the particle's 
  >energy-momentum to be p'=gp?  (e.g. if I used the particle to 
  >do work in my own rest frame)  Would there still be no upper 
  >bound on the total energy, i.e. by making our relative velocity 
  >close enough to c, I could measure the particle's kinetic energy 
  >to be as high as I wished?
\end{verbatim}
    

  To understand this, it's good to think of the momenta as 
  elements of the Lie algebra so(2,1) - it's crucial to the
  game.  

  Then, if you have momentum p, and I zip past you, so you 
  appear transformed by some element g of the Lorentz group 
  SO(2,1), I'll see your momentum as

  p' = g p g^{-1}

  This is just another way of writing the usual formula for 
  Lorentz transforms in 3d Minkowski space.  No new physics 
  so far, just a clever mathematical formalism.

  But when we turn on gravity, letting Newton's constant k 
  be nonzero, we should instead think of momentum as group-valued, 
  via 

  h = exp(kp)

  and similarly 

  h' = exp(kp')

  Different choices of p now map to the same choice of h.
  In particular, a particle of a certain large mass - the 
  Planck mass- will turn out to act just like a particle 
  of zero mass!   

  So, if we agree to work with h instead of p, we are now 
  doing new physics.  This is even more obvious when we decide
  to multiply momenta instead of adding them, since multiplication
  in SO(2,1) is noncommutative!

  But, if we transform our group-valued momentum in the correct
  way:

  h' = ghg^{-1}

  this will be completely compatible with our previous transformation
  law for vector-valued momentum!


\begin{verbatim}

  >I guess I'm trying to clarify whether the usual Lorentz transformation
  >of the tangent space has somehow been completely invalidated for
  >extreme boosts, or whether it's just a matter of there being a second
  >definition of "momentum" (defined in terms of the Hamiltonian) which
  >transforms differently and is the appropriate thing to consider in
  >gravitational contexts.
\end{verbatim}
    
  Good question!  Amazingly, the usual Lorentz transformations still 
  work EXACTLY - even though the rule for adding momentum is new (now 
  it's multiplication in the group).  We're just taking exp(kp) instead 
  of p as the "physical" aspect of momentum.  

  This effectively puts an upper limit on mass, since as
  we keep increasing the mass of a particle, eventually it "loops 
  around" SO(2,1) and act exactly like a particle of zero mass.  

  But, it doesn't exactly put an upper bound on energy-momentum, 
  since SO(2,1) is noncompact.  Of course energy and momentum don't
  take real values anymore, so one must be a bit careful with this
  "upper bound" talk.


\begin{verbatim}

  >In other words, does the cut-off mass apply only to the deficit 
  >angle, and do boosts still allow me to measure (by non-gravitational 
  >means) arbitrarily large energies (at least in the classical theory)?
\end{verbatim}
    

  There's some sense in which energy-momenta can be arbitrarily 
  large.  That's because the space of energy-momenta, namely SO(2,1), 
  is noncompact.  Maybe you can figure out some more intuitive way 
  to express this.
\end{quote}

\par\noindent\rule{\textwidth}{0.4pt}
<em>I was sitting in a chair in the patent office in Bern when all of a
sudden a thought occurred to me.  If a person falls freely, he will
not feel his own weight.</em> - Albert Einstein

\par\noindent\rule{\textwidth}{0.4pt}
% </A>
% </A>
% </A>
