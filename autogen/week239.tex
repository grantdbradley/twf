
% </A>
% </A>
% </A>
\week{September 8, 2006 }


David Corfield, Urs Schreiber and I have started up a new blog!

David is a philosopher, Urs is a physicist, and I'm a mathematician,
but one thing we all share is a fondness for n-categories.  We 
also like to sit around and talk shop in a public place where our 
friends can drop by.  Hence the title of our blog:

1) The n-Category Caf&eacute;, <a href =
"http://golem.ph.utexas.edu/category/">
http://golem.ph.utexas.edu/category</a>

Technologically speaking, the cool thing about this blog is that it
uses itex and MathML to let us (and you) write pretty equations in TeX.
For this we thank Jacques Distler, who pioneered the technology on
his own blog:


2) Jacques Distler, Musings, <a href = "http://golem.ph.utexas.edu/~distler/blog/">http://golem.ph.utexas.edu/~distler/blog/</a>

It's great!

Urs began by posting about 11d supergravity and higher gauge theory
(see "<A HREF = "week237.html">week237</A>").  Now he's
discussing Barrett and Connes' new work on the Standard Model.
Meanwhile, I've been obsessed with the categorical semantics of
quantum computation, and David has been running discussions on
categorifying Klein's Erlangen program (see "<A HREF =
"week213.html">week213</A>"), the differences between
mathematicians and historians when it comes to writing histories of
math, and so on.

And, it's all free.

Meanwhile, in the bad old world of extortionist math publishers,
we see a gleam of hope.  The entire editorial board of the journal 
Topology resigned to protest Reed-Elsevier's high prices!

3) Topology board of editors, letter of resignation,
<a href = "http://math.ucr.edu/home/baez/topology-letter.pdf">http://math.ucr.edu/home/baez/topology-letter.pdf</a>

The board includes some topologists I respect immensely.  It takes 
some guts for full-fledged memmbers of the math establishment to 
do something like this, and I congratulate them for it.  It'll be
fun to see what stooges Reed-Elsevier rounds up to form a new board 
of editors.  I can't imagine they'll just declare defeat and let the 
journal fold.

This is part of trend where journal editors "declare independence"
from their publishers and move toward open access:

4) Open Access News, Journal declarations of independence,
<a href = "http://www.earlham.edu/%7Epeters/fos/lists.htm#declarations">http://www.earlham.edu/%7Epeters/fos/lists.htm#declarations</a>

Speaking of open access, you can now get the notes from the course 
Freeman Dyson taught on quantum electrodynamics when he first
became a professor of physics at Cornell: 

5) Freeman J. Dyson, 1951 Lectures on Advanced Quantum Mechanics,
second edition, available as <A HREF = "http://xxx.lanl.gov/abs/quant-ph/0608140">quant-ph/0608140</A>.  For historical
context and original mimeographs, see 
http://hrst.mit.edu/hrs/renormalization/dyson51-intro/ 

These notes are from an exciting period in physics, shortly after
the 1947 Shelter Island conference where Feynman and Schwinger 
presented their approaches to quantum electrodynamics to an audience
of luminaries including Bohr, Oppenheimer, von Neumann, and Weisskopf.
Nobody understood Feynman's diagrams except Schwinger and maybe 
Feynman's thesis advisor, John Wheeler.

Every true fan of physics loves reading about this heroic era and 
its figures, especially Feynman.  So, if you haven't read these yet, 
run to the bookstore and buy them now!  

6) James Gleick, Genius: the Life and Science of Richard Feynman,
Vintage Press, 1993.

7) Jagdish Mehra, The Beat of a Different Drum: the Life and Science
of Richard Feynman, Oxford U. Press, 1996. 

8) Silvan S. Schweber, QED and the Men Who Made It, Princeton U.
Press, Princeton, 1994.

The first book is a barrel of fun but doesn't get into the nitty-gritty
details of Feynman's work.  The second more scholarly treatment also
has lots of Feynman anecdotes - even some new ones!  But, it covers
his work in enough detail to intimidate any non-physicist.  The third
offers a broader panorama of the development of quantum electrodynamics.
Taken together, they add up to quite a nice story. 

Of course, I'm \emph{assuming} you've read these: 

9) Richard P. Feynman, Surely You're Joking, Mr. Feynman! (Adventures
of a Curious Character), W. W. Norton and Company, New York, 1997.

10) Richard P. Feynman, What Do \emph{You} Care What Other People Think?
(Further Adventures of a Curious Character), W. W. Norton and Company, 
New York, 2001.

They're more fun than everything else I've ever recommended on This
Week's Finds, combined.  If you haven't read them, don't just
\emph{run} to the nearest bookstore - get in a time machine, go
back, and make sure you \emph{did} read them.

Today I'd like to wrap up the discussion of Koszul duality which I 
began last Week.  As we'll see, this gives a really efficient way
of categorifying the theory of Lie algebras and defining "Lie 
n-algebras".  And, as Urs Schreiber notes, these seem to be just 
what we need to understand 11-dimensional supergravity in a nice
geometric way.

But before I dive into this heavy stuff, something fun.   Thanks to
Christine Dantas' blog, I just saw a webpage on the origins of math 
and writing in Mesopotamia:

11) Duncan J. Melville, Tokens: the origin of mathematics,
from his website Mesopotamian Mathematics,
<a href = "http://it.stlawu.edu/%7Edmelvill/mesomath/">http://it.stlawu.edu/%7Edmelvill/mesomath/</a>

Before people in the Near East wrote on clay tablets, there were "tokens":

<div align = center>
<img width = 500 src = "tokens.jpg">
</div>

12) The Sch&oslash;yen Collection,
MS 5067/1-8, Neolithic plain counting tokens possibly representing 1 measure of grain, 1 animal and 1 man or 1 day's labour, respectively, 
<a href = "http://www.nb.no/baser/schoyen/5/5.11/index.html">http://www.nb.no/baser/schoyen/5/5.11/index.html</a>

These are little geometric clay figures that represented things like
sheep, jars of oil, and various amounts of grain.  They are found
throughout the Near East starting with the agricultural revolution in
about 8000 BC.  Apparently they were used for contracts!  

Eventually groups of tokens were sealed in clay envelopes, so any 
attempt to tamper with them would be visible.

But, it's annoying to have to break a clay envelope just to see what's 
in it.  So, after a while, they started marking the envelopes to say 
what was inside.  At first, they did this simply by pressing the tokens
into the soft clay of the envelopes.

Later, these marks were simply drawn on tablets.  Eventually they gave
up on the tokens - a triumph of convenience over security.  The marks 
on tablets then developed into the Babylonian number system!  The 
transformation was complete by 3000 BC.

<div align = center>
<a href = "http://www.gap-system.org/~history/HistTopics/Babylonian_numerals.html">
<img src = "Babylonian_symbols.gif">
% </a>
</div>


So, five millennia of gradual abstraction led to the writing of numbers!
From three tokens representing jars of oil, we eventually reach the 
abstract number "3" applicable to anything.  

Of course, all history is detective work.  The story I just told is 
an interpretation of archaeological evidence.  It could be wrong.  
This particular interpretation is due to Denise Schmandt-Besserat.  
It seems to be fairly well accepted in broad outline, but scholars 
are still arguing about it.  

For more on her ideas, try this:

13) Denise Schmandt-Besserat, Accounting with tokens in the 
ancient Near East, 
<a href = "http://www.utexas.edu/cola/centers/lrc/numerals/dsb/dsb.html">http://www.utexas.edu/cola/centers/lrc/numerals/dsb/dsb.html</a>

For a bibliography of her many papers, try:

14) Denise Schmandt-Besserat, Publications,
<a href = "http://www.utexas.edu/cola/centers/lrc/iedocctr/ie-pubs/dsb-pubs.html">http://www.utexas.edu/cola/centers/lrc/iedocctr/ie-pubs/dsb-pubs.html</a>

For more work on this subject - I want to read more! - try:

15) Eleanor Robson, Bibliography of Mesopotamian mathematics,
<a href = "http://it.stlawu.edu/~dmelvill/mesomath/erbiblio.html">http://it.stlawu.edu/~dmelvill/mesomath/erbiblio.html</a>

And for a fun intro to writing on clay tablets, try this:

16) John Heise, Cuneiform writing system,
<a href = "http://xoomer.alice.it/bxpoma/akkadeng/cuneiform.htm">http://xoomer.alice.it/bxpoma/akkadeng/cuneiform.htm</a>

Next, from 8000 BC, let's shoot forward ten millennia straight into the 20th
century.  Last week I gave three examples of Koszul duality:

\begin{quote}
 Making the free graded-commutative algebra on SL* into a differential
 graded-commutative algebra is the same as making L into a Lie algebra.
\end{quote}
    

\begin{quote}
 Making the free graded Lie algebra on SL* into a differential
 graded Lie algebra is the same as making L into a commutative algebra.
\end{quote}
    

\begin{quote}
 Making the free graded associative algebra on SL* into a differential
 graded associative algebra is the same as making L into an associative
 algebra.
\end{quote}
    

Here L is a vector space, which we think of as a graded vector space
concentrated in degree zero.  L* is its dual, and SL* is the
"shifted" or "suspended" version of L*, where we
add one to the degree of everything.

Now, what if we replace L by a graded vector space that can have stuff
of any degree?  We get a fancier version of Koszul duality, which goes
like this:

\begin{quote}
 Making the free graded-commutative algebra on SL* into a differential
 graded-commutative algebra is the same as making L into an L<sub>&infin;</sub>-algebra. 
\end{quote}
    

\begin{quote}
 Making the free graded Lie algebra on SL* into a differential
 graded Lie algebra is the same as making L into a C<sub>&infin;</sub>-algebra.
\end{quote}
    

\begin{quote}
 Making the free graded associative algebra on SL* into a differential
 graded associative algebra is the same as making L into an A<sub>&infin;</sub>-algebra.
\end{quote}
    


Here an "L_{\infty }-algebra" is a chain complex
that's like a Lie algebra, except the Jacobi identity holds up to a
chain homotopy called the "Jacobiator", which in turn
satisfies its own identity up to a chain homotopy called the
"Jacobiatorator", and so on ad infinitum.  Keeping track of
all these higher homotopies is quite a chore.  Well, it's sort of fun
when you get into it, but the great thing about Koszul duality is that
you don't need to remember any fancy formulas: all the higher
homotopies are packed into the \emph{differential} on SL*.
 
Similarly, a "C_{\infty }-algebra" is a chain
complex that's like a graded-commutative algebra up to homotopy, ad
infinitum.

Similarly, an "A_{\infty }-algebra" is a chain
complex that's like an associative algebra up to homotopy, ad
infinitum.  Here you can read off all the higher homotopies from the
Stasheff associahedra, which you know and love from "<A HREF =
"week144.html">week144</A>" - but again, Koszul duality means you
don't have to!

As mentioned last week, all this stuff generalizes to any kind of
algebraic gadget in Vect - the category of vector spaces - which is
defined by a "quadratic operad" O.  Any such operad has a
"Koszul dual" operad O* such that:

\begin{quote}
 Making the free graded O*-algebra on SL* into a differential
 graded O*-algebra is the same as making L into an O<sub>&infin;</sub>-algebra.
\end{quote}
    

Here O_{\infty } is an operad in the category of chain complexes
defined by "weakening" O in a systematic way - replacing all the
laws by chain homotopies, ad infinitum.  We can define O_{\infty }
using the "bar construction", as nicely described here:

16) Todd Trimble, Bar constructions,
<a href = "http://math.ucr.edu/home/baez/trimble/bar.html">http://math.ucr.edu/home/baez/trimble/bar.html</a>

or in the book by Markl, Schnider and Stasheff:

17) Martin Markl, Steve Schnider and Jim Stasheff, Operads in 
Algebra, Topology and Physics, AMS, Providence, Rhode Island, 2002.

See "<A HREF = "week191.html">week191</A>" for more on this
book, and what the heck an "operad" is.  By the way, we have

O** &cong; O

so we can also say:

\begin{quote}
 Making the free graded O-algebra on SL* into a differential
 graded O-algebra is the same as making L into an O*<sub>&infin;</sub>-algebra.
\end{quote}
    



Anyway, I don't have much intuition for how Koszul duality lets
us magically sidestep the bar construction of O_{\infty }. Someday
I hope I'll understand this.

But, once we have the concept of "L_{\infty }-algebra", 
we can restrict ourselves to chain complexes that
vanish except for their first n terms - that is, degrees 0, 1, ...,
n-1 - and get the concept of "Lie n-algebra".

In fact, a Lie n-algebra is like a hybrid of a Lie algebra and an
n-category!  The definition I just gave says a Lie n-algebra is 
an L_{\infty }-algebra which as a chain complex vanishes above 
degree n-1.  But, such chain complexes are equivalent to strict 
n-category objects in Vect!  So, we can think of Lie n-algebras as 
strict n-categories that do their best to act like Lie algebras, but 
with some of the laws holding up to isomorphism, with these isomorphisms 
satisfying their own laws up to isomorphism, etcetera.  

But, the really cool part is that we can do \emph{gauge theory} using
Lie n-algebras instead of Lie algebras, and taking n = 3 we get an
example that seems to explain the geometry of 11d supergravity...
that is, the classical limit of that mysterious thing called M-theory.

For this, you really need to read Urs Schreiber's stuff:

18) Urs Schreiber, Castellani on free differential algebras in
supergravity: gauge 3-group of M-theory, <a href =
"http://golem.ph.utexas.edu/string/archives/000840.html">http://golem.ph.utexas.edu/string/archives/000840.html</a>

19) Urs Schreiber, SuGra 3-connection reloaded, <a href =
"http://golem.ph.utexas.edu/category/2006/08/sugra_3connection_reloaded.html">http://golem.ph.utexas.edu/category/2006/08/sugra_3connection_reloaded.html</a>

and many other things he's been writing on the n-Category Caf&eacute; lately.

\par\noindent\rule{\textwidth}{0.4pt}

\textbf{Addenda:} You can see discussion of this Week's Finds at the
<a href = "http://golem.ph.utexas.edu/category/2006/09/this_weeks_finds_in_mathematic.html">\emph{n}-Category
Caf&eacute;</a>.  In his blog 
<a href = "http://www.math.columbia.edu/~woit/wordpress/?p=456">
Not Even Wrong</a>, Peter Woit
has more to say about the open access movement
and a questionable plan broached by CERN to pay for-profit journals to
make their papers freely available.
<a href = "http://johncarlosbaez.wordpress.com/2011/12/02/babylon-and-the-square-root-of-2/#comment-11178">Some comments</a> on this blog article
dig deeper into the evolution of Babylonian numerals:

19) John Baez and Richard Elwes, Babylon and the square root of 2,
Azimuth, December 2nd, 2011, <a href = "http://johncarlosbaez.wordpress.com/2011/12/02/babylon-and-the-square-root-of-2/">http://johncarlosbaez.wordpress.com/2011/12/02/babylon-and-the-square-root-of-2/</a>.
</p>

In particular, Duncan Melville points out that when number systems
first evolved in Babylonia, they had about a dozen <i>different</i>
systems for different kinds of products!  A base-60 system called the
S system, was used to count most discrete objects, such as sheep or
people.  For 'rations' such as cheese or fish, they used a base 120
system, the B system.  Another system, the &#352;E system, was used to
measure quantities of grain, and so on.  So, number systems were a bit
more like business software today, with different kinds used by
different trades.

\par\noindent\rule{\textwidth}{0.4pt}
<em>I never once doubted that I would eventually succeed in getting to
the bottom of things.</em> - Alexander Grothendieck

<HR>

% </A>
% </A>
% </A>


% parser failed at source line 473
