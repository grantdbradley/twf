
% </A>
% </A>
% </A>
\week{December 24, 2007 }

Since it's Christmas Eve, I thought I'd list some free books you 
can download.  I'm a big fan of giving the world presents...
and I'm not the only one.

But first, this week's nebulae!  Here's one called the Retina:

<div align = "center">
<a href = "retina_nebula.jpg">
<img width = "500" src = "retina_nebula.jpg">
% </a>
</div>

1) Retina Nebula, Hubble Heritage Project, <a href =
"http://heritage.stsci.edu/2002/14/">http://heritage.stsci.edu/2002/14/</a>

This is actually a tube of ionized gas about a quarter of a light-year 
across and one light-year long.  It's a planetary nebula produced
by a dying star.   If you zoom in and look closely, you can see this
star lurking in the middle, now a mere white dwarf.

The blue light is the most energetic, so it's really hot where you see
blue.   This blue light comes from singly ionized helium - helium where
one electron has been knocked off.  The green light is a bit less 
energetic: that's from doubly ionized oxygen.  The red light comes from 
even cooler regions: that's from singly ionized nitrogen.

You can also see a lot of "dust lanes" in this photo.  They're
beautiful.  And they're big!  The width of each one is about 160 times
the distance between the Sun and the Earth.  The gas and dust in these
lanes is about 1000 times higher than elsewhere.  But what creates them?   

Apparently, when the fast-moving glowing hot gas from the star crashes
into the invisible gas in the surrounding interstellar space, the
boundary gets sort of crumpled, and these dust lanes form.  It's
vaguely similar to the puffy surface of a cumulus cloud.  But here the
mechanism is different, because it involves a "shock wave":
the hot gas is moving faster than the speed of sound as it hits the
cold gas!

This effect is called a "Vishniac instability", since in
1983, the astrophysicist Ethan Vishniac showed that a shock wave
moving in a sufficiently compressible medium would be subject to an
instability of this sort, growing as the square root of time.  I've
never seen how Vishniac's calculations work, so the mathematics
underlying this beautiful phenomenon will have to wait for another
day.

Note that this planetary nebula, like the others I've shown you, is 
far from spherically symmetric.  Astrophysicists used to pretend stars 
were spherically symmetric.  But, that's a bad approximation whenever 
anything really exciting happens... just like in the 
<a href = "http://en.wikipedia.org/wiki/Spherical_cow">old joke</a> where 
the punchline is "consider a spherical cow".

As I said, the Retina Nebula is actually shaped like a tube.  Viewed 
from either end, this tube would look very different - probably like 
the Ring Nebula:

<div align = "center">
<a href = "ring_nebula.jpg">
<img width = "500" src = "ring_nebula.jpg">
% </a>
</div>

2) Ring Nebula, Hubble Heritage Project, 
<a href = "http://heritage.stsci.edu/1999/01/">http://heritage.stsci.edu/1999/01/</a>

This is one light-year across.  Again we see He II blue light with a
wavelength of 4686 
angstroms, then O III green light at 5007
angstroms,
then N II red light at 6584 
angstroms.  You can also see
the white dwarf as a tiny dot in the center; it's about 100,000 kelvin
in temperature.

(In case you're wondering, an "angstrom" is an 
obsolete but popular unit of distance, equal to 10^{-10} meters.  
Just like
the "parsec", it's a sign that astronomy is an old science.
Anders Jonas &Aring;ngstr&ouml;m was one of the founders of spectroscopy, back
around 1860.  Archaic conventions may also explain why singly ionized
helium is called "He II", and so on.  Maybe the number zero
hadn't fully caught on.)

Next: free books!

At least around here, Christmas seems to be all about buying stuff and
giving it away.  Giving is good.  But I think gifts have more soul if
you make them yourself.  This is one of the great things about the
internet: it lets us create things and give them to <em>everyone in
the world</em> - or more precisely: everybody who wants them, and
nobody who doesn't.

In this spirit, here's a roundup of free books on math and physics: 
gifts from their authors to you.  There are lots out there.   I'll 
only list a few.  For more, try these sites:

3) George Cain, Online Mathematics Textbooks, 
<a href = "http://www.math.gatech.edu/~cain/textbooks/onlinebooks.html">http://www.math.gatech.edu/~cain/textbooks/onlinebooks.html</a>

4) Free Online Mathematics Books, 
<a href = "http://www.pspxworld.com/book/mathematics/">http://www.pspxworld.com/book/mathematics/</a>

5) Alex Stefanov, Textbooks in Mathematics, 
<a href = "http://users.ictp.it/~stefanov/mylist.html">http://users.ictp.it/~stefanov/mylist.html</a>
or (with annoying ads,
but more permanent) <a href = "http://us.geocities.com/alex_stef/mylist.html">http://us.geocities.com/alex_stef/mylist.html</a>

Despite its title, Stefanov's excellent site includes a lot of 
books on physics.  I can't find lists \emph{specifically} devoted to
free physics books, but there are a lot out there - including a lot on 
the arXiv.  

Anyway, let's dive in!

What if you're dying to learn physics, but don't know where to start?  
Start here:

6) Physics Books Online, <a href = "http://www.sciencebooksonline.info/physics.html">http://www.sciencebooksonline.info/physics.html</a>. 

You'll find plenty of free online books, starting from the basics
and working up to advanced topics.  But to dig deeper into these
mysteries, you'll eventually need to learn a bunch of math.  Do you 
remember what 
<a href = "http://globetrotter.berkeley.edu/conversations/Weisskopf/">Victor
Weisskopf</a> said when a student asked how much math a physicist needs to
know?  

<center>
"More."
</center>

This can be scary when you're just getting started.
What if you don't know calculus, for example?

Simple: learn calculus!  This book is a classic - and it's free:

7) Gilbert Strang, Calculus, Wellesley-Cambridge Press, Cambridge,
1991.  Also available at 
<a href = "http://ocw.mit.edu/ans7870/resources/Strang/strangtext.htm">http://ocw.mit.edu/ans7870/resources/Strang/strangtext.htm</a>

It really explains things clearly.  I may use it the next time I 
teach calculus.  We professors need to quit making our students
buy expensive textbooks, and switch to free online books!  We could
join forces and make wiki textbooks that are a lot better and
more flexible than the budget-busting, back-breaking mammoths we 
currently inflict on our kids.  But there are already a lot of good
texts available free online.  

Or: what if you know calculus, but you're still swimming through the
undergraduate sea of differential equations, Fourier transforms, 
matrices, vectors and tensors?  Then this should be really helpful:

8) James Nearing, Mathematical Tools for Physics, available at
<a href = "http://www.physics.miami.edu/~nearing/mathmethods/">http://www.physics.miami.edu/~nearing/mathmethods/</a>

Unlike the usual dry and formal textbook, it reads like a friendly 
uncle explaining things in plain English, trying to cut through the red
tape and tell you how to actually think about this stuff.

For example, on page 3 he introduces the hyperbolic trig functions:

\begin{quote}
  Where do hyperbolic functions come from?  If you have a mass in
  equilibrium, the total force on it is zero.  If it's in
  \emph{stable} equilibrium then if you push it a little to one side
  and release it, the force will push it back to the center.  If it is
  \emph{unstable} then when it's a bit to one side it will be pushed
  farther away from the equilibrium point.  In the first case, it will
  oscillate about the equilibrium position and the function of time
  will be a circular trigonometric function - the common sines or
  cosines of time, Acos(\omega t).  If the point is unstable, the
  motion will be described by hyperbolic functions of time,
  sinh(\omega t) instead of sin(\omega t).  An ordinary ruler held at
  one end will swing back and forth, but if you try to balance it at
  the other end it will fall over.  That's the difference between cos
  and cosh.  
\end{quote}

He goes into more detail later, after introducing the complex numbers.
This book also features some great animations of Taylor series and 
Fourier series, like this movie of the Taylor series of the sine
function:

<div align = "center">
<img width = "350" src = "http://www.physics.miami.edu/~nearing/mathmethods/power-sine-series.gif">
</div>

There are free online books at all levels... so let's soar a bit 
higher.  How about if you're a more advanced student trying to learn 
general relativity?  Here you go:

9) Sean M. Carroll, Lecture Notes on General Relativity, available as
 <a href = "http://arXiv.org/abs/arXiv:gr-qc/9712019">arXiv:gr-qc/9712019</A>

How about quantum field theory?  Then you're in luck - there are 
\emph{two} detailed books available online:

10) Warren Siegel, Fields, available as <a href = "http://arXiv.org/abs/arXiv:hep-th/9912205">arXiv:hep-th/9912205</A>

10) Mark Srednicki, Quantum Field Theory, Cambridge U. Press, 
Cambridge, 2007.  Also available at 
<a href = "http://www.physics.ucsb.edu/~mark/qft.html">http://www.physics.ucsb.edu/~mark/qft.html</a>

Or what about algebraic topology?   Again you're in luck, since you 
can read both Allen Hatcher's gentle introduction and Peter May's 
high-powered "concise course":

11) Allen Hatcher, Algebraic Topology, Cambridge U. Press, Cambridge,
2002.  Also available at 
<a href = "http://www.math.cornell.edu/~hatcher/AT/ATpage.html">http://www.math.cornell.edu/~hatcher/AT/ATpage.html</a>

12) Peter May, A Concise Course in Algebraic Topology, U. of Chicago 
Press, Chicago, 1999.  Also available at 
<a href = "http://www.math.uchicago.edu/~may/CONCISE/ConciseRevised.pdf">http://www.math.uchicago.edu/~may/CONCISE/ConciseRevised.pdf</a>

May has a lot of more advanced topology books available at his website, 
too - like this classic, where he used operads to solve important 
problems involving loop spaces:

13) Peter May, The Geometry of Iterated Loop Spaces, Lecture Notes 
in Mathematics 271, Springer, Berlin, 1972.   Also available at 
<a href = "http://www.math.uchicago.edu/~may/BOOKS/gils.pdf">http://www.math.uchicago.edu/~may/BOOKS/gils.pdf</a>

Or say you want to learn about vector bundles and how they show up
in physics, from the basics all the way to fancy stuff like D-branes 
and K-theory?  Try this - it's a great sequel to Husemoller's classic
intro to fiber bundles:

14) Dale Husemoller, Michael Joachim, Branislav Jurco and Martin
Schottenloher, Basic Bundle Theory and K-Cohomology Invariants,
Lecture Notes in Physics 726, Springer, Berlin, 2008.  Also
available at 
<a href = "http://www.mathematik.uni-muenchen.de/~schotten/Texte/978-3-540-74955-4_Book_LNP726.pdf">http://www.mathematik.uni-muenchen.de/~schotten/Texte/978-3-540-74955-4_Book_LNP726.pdf</a>

The list goes on and on!  The American Mathematical Society will give 
you books for free if you prove that you're not a robot by solving a 
little puzzle:

15) American Mathematical Society, Books Online By Subject,
<a href = "http://www.ams.org/online_bks/online_subject.html">http://www.ams.org/online_bks/online_subject.html</a>

Apparently they don't want robots learning advanced math and putting
us professors out of business by teaching with more charisma and flair.
(By the way: make sure to let them put cookies on your
web browser, or they'll send you an endless succession of these
puzzles, without explaining why!)

Since James Dolan and I plan to explain symmetric groups and their 
Hecke algebras in our online seminar, this particular book from the AMS
caught my eye:

16) David M. Goldschmidt, Group Characters, Symmetric Functions, 
and the Hecke Algebra, AMS, Providence, Rhode Island, 1993.
Also available as <a href = "http://www.ams.org/online_bks/ulect4/">http://www.ams.org/online_bks/ulect4/</a>

Since we're also struggling to understand the Langlands program, 
this looks good too:

17) Armand Borel, Automorphic Forms, Representations, and L-functions,
AMS, 2 volumes, Providence, Rhode Island, 1979.  Also available at
<a href = "http://www.ams.org/online_bks/pspum331/">http://www.ams.org/online_bks/pspum331/</a> and
<a href = "http://www.ams.org/online_bks/pspum332/">http://www.ams.org/online_bks/pspum332/</a>

It's a serious collection of expository papers by bigshots like 
Borel, Cartier, Deligne, Jacquet, Knapp, Langlands, Lusztig, Tate, 
Tits, Zuckerman, and many more.

"Motives" are the mysterious virtual building blocks that algebraic
varieties are built from.  If you're ready to learn about motives -
I'm not sure I am - try this:

18) Marc Levine, Mixed Motives, AMS, Providence, Rhode Island, 1998.
Also available at <a href = "http://www.ams.org/online_bks/surv57/">http://www.ams.org/online_bks/surv57/</a>

Or, if you're interested in using category theory to make analysis
clearer and more beautiful, try this:

19) Andreas Kriegl and Peter W. Michor, The Convenient Setting of
Global Analysis, AMS, Providence, Rhode Island, 1997.  Also available
at <a href =
"http://www.ams.org/online_bks/surv53/">http://www.ams.org/online_bks/surv53/</a>

The focus is on getting and working with a "convenient
category" of infinite-dimensional manifolds.  The idea of a
"convenient category" goes back to topology: at some point,
people realized they wanted this property to hold:

C(X \times  Y, Z) \cong  C(X, C(Y, Z))

Here C(X,Y) is the space of maps from X to Y.  So, the isomorphism
above says that a map from X \times  Y to Z should correspond to a map
from X to C(Y,Z).  A category with this property is called
"cartesian closed".  While it may not be obvious why,
this property is so wonderful that people threw out the
category of topological spaces and continuous maps and replaced it
with a slightly different one, just to get this to hold.

Another sort of "convenient category" for differential
geometry uses infinitesimals.  Again, you can learn about this in a
free book:

20) Anders Kock, Synthetic Differential Geometry, Cambridge U. Press,
Cambridge, 2006.  Also available at <a href = "http://home.imf.au.dk/kock/">http://home.imf.au.dk/kock/</a>

This category is not just cartesian closed - it's a topos!

If you don't know what a topos is, never fear - more free books are
coming to your rescue:

21) Robert Goldblatt, Topoi, the Categorial Analysis of Logic, 
Dover, 1983.  Also available at 
<a href = "http://historical.library.cornell.edu/cgi-bin/cul.math/docviewer?did=Gold010">http://historical.library.cornell.edu/cgi-bin/cul.math/docviewer?did=Gold010</a>

22) Michael Barr and Charles Wells, Toposes, Triples and Theories,
Springer, Berlin, 1983.   Also available at
<a href = "http://www.case.edu/artsci/math/wells/pub/ttt.html">http://www.case.edu/artsci/math/wells/pub/ttt.html</a>

The first one is so gentle it makes a good introduction to category 
theory as a whole.  The second scared the bejeezus out of me for a 
decade, but now I like it.

I like Jordan algebras, so I was also pleased to see this classic
offered for free at the AMS website:

23) Nathan Jacobson, Structure and Representations of Jordan Algebras,
AMS, Providence, Rhode Island, 1968.  Also available at 
<a href = "http://www.ams.org/online_bks/coll39/">http://www.ams.org/online_bks/coll39/</a>

Fans of exceptional Lie algebras will like the last two chapters, on
"connections with Lie algebras" and "exceptional Jordan algebras".

Speaking of Lie algebras, I'd never seen this textbook before:

24) Shlomo Sternberg, Lie Algebras, 
<a href = "http://www.math.harvard.edu/~shlomo/docs/lie_algebras.pdf">http://www.math.harvard.edu/~shlomo/docs/lie_algebras.pdf</a>

It's a somewhat quirky introduction, not for beginners I think, but
it features some nice special topics: character formulas, the Kostant
Dirac operator, and a detailed study of the center of the universal
enveloping algebra.  

This intro to Lie groups is also a bit quirky, but if you like Feynman 
diagrams or spin networks, it's irreplaceable:

25) Predrag Cvitanovic, Birdtracks, Lie's, and Exceptional Groups,
available at <a href = "http://www.nbi.dk/GroupTheory/">http://www.nbi.dk/GroupTheory/</a>

One of the great things about this book is that it classifies simple
Lie groups according to their "skein relations" - properties
of their representations, written out diagrammatically.  In so doing,
Cvitanovic realized that there's a "magic triangle"
containing all the exceptional Lie groups.  This subsumes the
"magic square" of Freudenthal and Tits, which I discussed in
"<a href = "week145.html">week145</A>" and my 
<a href = "http://math.ucr.edu/home/baez/octonions/node16.html">octonion
webpages</a>.

This idea of Cvitanovic is closely related to the "exceptional
series" of Lie groups - a pattern whose existence was conjectured
by Deligne.  I love the term "exceptional series".   It's 
an oxymoron, since the exceptional groups were defined as those that 
don't fit into any series.  But, it makes sense!   

To see the exceptional series, it helps to do a mental backflip called
"Tannaka-Krein duality", where you focus on the category of
representations of the Lie group, instead of the group itself.  Then,
draw the morphisms in that category as diagrams, like Feynman
diagrams!  Then see what identities they satisfy.  New patterns leap
out: new series unify what had been "exceptions".

Very briefly, the idea goes like this.  Suppose we have a Lie group
G with Lie algebra L.  The Lie bracket takes two elements x and y and spits out 
one element [x,y], and it's linear in each variable, so it gives a 
linear operator

L \otimes  L \to  L

which is actually a morphism in the category of representations of G.

So, following the philosophy of Feynman diagrams, we can draw the 
bracket operation like this:
\begin{verbatim}
                \     /
                 \   /
                  \ /
                   |
                   |
                   |
\end{verbatim}
    
We can even use this to state the definition of a Lie algebra using 
diagrams!  To say the bracket is antisymmetric:

[y,x] = -[x,y]

we just draw this:
\begin{verbatim}
                 \   /              |       |
                  \ /               |       |
                   /                |       |
                  / \               |       |
                 /   \               \     /
                 \   /     =    -     \   /
                  \ /                  \ /
                   |                    |
                   |                    |
                   |                    |
\end{verbatim}
    

To say the Jacobi identity:
[x,[y,z]] = [[x,y],z] + [y,[x,z]]

we just draw this:
\begin{verbatim}
  \     \     /          \     /     /            \    /      /   
   \     \   /            \   /     /              \  /      /
    \     \ /              \ /     /                \       / 
     \     /                \     /                  \     /
      \   /        =         \   /         +       /  \   /
       \ /                    \ /                 /    \ /
        |                      |                  \     /
        |                      |                   \   /
        |                      |                    \ /
        |                      |                     |
\end{verbatim}
    
If that's too cryptic, maybe this will explain what I'm doing:
\begin{verbatim}
 x     y       z        x       y     z          x      y      z
  \     \     /          \     /     /            \    /      /   
   \     \   /            \   /     /              \  /      /
    \     \ /              \ /     /                \       / 
     \     /                \     /                  \     /
      \   /        =         \   /         +       /  \   /
       \ /                    \ /                 /    \ /
        |                      |                  \     /
        |                      |                   \   /
        |                      |                    \ /
        |                      |                     |
   [x,[y,z]]              [[x,y],z]              [y,[x,z]]
\end{verbatim}
    
But in fact, people usually massage this picture to make it even
more cryptic, and call it the "IHX" identity - since the three terms
look like the letters I, H, and X by the time they're done twisting
them around.  For a good explanation, with pretty pictures, see:

26) Greg Muller, Chord diagrams and Lie algebras, 
<a href = "http://cornellmath.wordpress.com/2007/12/25/chord-diagrams-and-lie-algebras/">http://cornellmath.wordpress.com/2007/12/25/chord-diagrams-and-lie-algebras/</a>
 
It then turns out that the exceptional Lie algebras 
F_{4}, E_{6}, E_{7} and E_{8} satisfy 
\emph{yet another} identity:


\begin{verbatim}

 \        /                
  \      /                
   \----/                
   |    |                       
   |    |           = 
   /----\            
  /      \          
 /        \                
                       
 
  \        /           \   /
   \      /             \ /   
    \    /               |   
A    ----     +     A    |       +
    /    \               |  
   /      \             / \  
  /        \           /   \  


  \     /       \        /         \        /
   \   /         \      /           \      /
    \ /           \    /             \____/ 
B    /     +  B    |  |       +  B    ____    
    / \           /    \             /    \
   /   \         /      \           /      \
  /     \       /        \         /        \
\end{verbatim}
    


for various choices of the constants A and B.  So, they fit into a
"series"!

I believe the main point
of this identity, going back to Vogel's paper "Algebraic structures 
on modules of diagrams", is that for these Lie algebras,
the square of the quadratic Casimir is the only degree-4 Casimir.

I think there's a lot more to be discovered here, in part by taking
the gnarly computations people have done so far and making them
more beautiful and conceptual.  
So, I urge all fans of exceptional mathematics, diagrams, and
categories to look at these:

27) Pierre Deligne, La serie exceptionnelle des groupes de Lie, 
C. R. Acad. Sci. Paris Ser. I Math 322 (1996), 321-326.

Pierre Deligne and R. de Man, The exceptional series of Lie groups II,
C. R. Acad. Sci. Paris Ser. I Math 323 (1996), 577-582.

Pierre Deligne and Benedict Gross, On the exceptional series, and its 
descendants, C. R. Acad. Sci. Paris Ser. I Math 335 (2002), 877-881.
Also available as <a href = "http://www.math.ias.edu/~phares/deligne/ExcepSeries.ps">http://www.math.ias.edu/~phares/deligne/ExcepSeries.ps</a>

28) Pierre Vogel, Algebraic structures on modules of diagrams, 1995.
Available at <a href = "http://www.institut.math.jussieu.fr/~vogel/">http://www.institut.math.jussieu.fr/~vogel/</a>
or
<a href = "http://citeseer.ist.psu.edu/469395.html">http://citeseer.ist.psu.edu/469395.html</a>

The universal Lie algebra, 1999.  Available at
<a href = "http://www.institut.math.jussieu.fr/~vogel/">http://www.institut.math.jussieu.fr/~vogel/</a>

Vassiliev theory and the universal Lie algebra, 2000.  
Available at <a href = "http://www.institut.math.jussieu.fr/~vogel/">http://www.institut.math.jussieu.fr/~vogel/</a>

For a good overview, try this:

28) J. M. Landsberg and L. Manivel, Representation theory and projective
geometry, 2002.  Available at <a href = "http://arXiv.org/abs/math/0203260">arXiv:math/0203260</A>.

Alas, they avoid drawing Feynman diagrams, though they talk about them
in section 4.  They prefer to use ideas from algebraic geometry:

29) J. M. Landsberg and L. Manivel, The projective geometry of 
Freudenthal's magic square, J. Algebra 239 (2001), 477-512.  Also 
available as <a href = "http://arXiv.org/abs/math/9908039">arXiv:math/9908039</A>.

J. M. Landsberg and L. Manivel, Triality, exceptional Lie algebras and
Deligne dimension formulas, Adv. Math. 171 (2002), 59-85.  Also 
available as <a href = "http://arXiv.org/abs/math/0107032">arXiv:math/0107032</A>.

J. M. Landsberg and L. Manivel, Series of Lie groups, available 
as <a href = "http://arXiv.org/abs/math/0203241">arXiv:math/0203241</A>.

Bruce Westbury, whom longtime readers of This Week's Finds will
remember as John Barrett's collaborator, has also worked on this
subject.  He has pointed out that both the magic square and the
magic triangle can be given an extra row and column if we introduce
a 6-dimensional algebra halfway between the quaternions and the
octonions:

30) Bruce Westbury, Sextonions and the magic square, available
as <a href = "http://arXiv.org/abs/math/0411428">arXiv:math/0411428</A>.

For even more references, try this:

31) Bruce Westbury, References on series of Lie groups,
<a href = "http://www.mpim-bonn.mpg.de/digitalAssets/2763_references.pdf">http://www.mpim-bonn.mpg.de/digitalAssets/2763_references.pdf</a>

This stuff has been on my mind recently, since I've been working on
exceptional groups and grand unified theories with my student
John Huerta.  Also, my friend Tevian Dray has a student who just 
finished a thesis on a related topic:

32) Aaron Wangberg, The structure of E6,
available as <a href = "http://arXiv.org/abs/0711.3447">arXiv:0711.3447</a>.

In a nutshell: E_{6} is secretly SL(3,O).  Octonions rock!

Happy holidays.  Keep learning cool stuff.

\par\noindent\rule{\textwidth}{0.4pt}
<B>Addenda:</B> Thomas Riepe listed some more free online math books.
Tony Smith pointed out something I already knew, but didn't make clear
above: the idea that E_{6} is secretly SL(3,O) is far from new.

Thomas wrote:

\begin{quote}
  Some more links:

  Milne's <a href = "http://www.jmilne.org/math/index.html">great
  collection </a> (incl. the famous LNM 900), leading the reader from
  basic algebra through algebraic number theory, class fields, modular
  forms, arithmetic groups,... up to etale cohomology, Shimura
  varieties etc.

<a href = "http://www.math.uni-bielefeld.de/~fw/">Friedhelm Waldhausen's lectures</a> on algebraic topology and K-theory.  

<a href = "http://www.mathematik.uni-bielefeld.de/~rehmann/DML/dml_links_author_A.html">DML: Digital Mathematics Library</a> 

<a href = "http://www.math.uni-bonn.de/people/harder/">G. Harder's math-links</a>

<a href = "http://www.msri.org/publications/books/">MSRI online books</a> 

Finally:

"Nearly three and a half centuries of scientific study and achievement is now available online in the <a href = "http://www.pubs.royalsoc.ac.uk/archive">Royal Society Journals Digital Archive</a>. This is the longest-running and arguably most influential journal archive in Science, including all the back articles of both Philosophical Transactions and Proceedings."
\end{quote}


Tony Smith wrote:


\begin{quote}
  Thanks for an interesting list of stuff in week 260,
  but I have some questions about this:

\begin{quote}
  32) Aaron Wangberg, The structure of E6, available as <a href = "http://www.arxiv.org/abs/0711.3447">arXiv:0711.3447</a>.
  In a nutshell: E_{6} is secretly SL(3,O). Octonions rock! 
\end{quote}

  Not only from your brief list descrption, but also from reading the  
  paper at pages 96 ff
  I get the impression that Wangberg is claiming the result E_{6} = SL(3,O).
  Do you get the same impression?
  I hope not, and I hope that my impression is somehow mistaken,
  because
  the result E_{6} = SL(3,O) is (and has been for some time) well known  
  and in the literature.
  For example, in <a href = "http://arxiv.org/abs/hep-th/9309030">hep-th/9309030</a> Martin Cederwall and Christian R.  
  Preitschopf said:
\begin{quote}

  ... It should be possible to realize E_{6} = SL(3;O) [18,24] on them in  
  a "spinor-like" manner, much like SO(10) = SL(2;O) acts on its 16-dimensional spinor representations that play the role of homogeneous  
  coordinates for OP^{1} ...

  ...<br/>
  18. H. Freudenthal, Adv. Math. 1 (1964) 145. <br/>
  ...<br/>
  24. A. Sudbery, J. Phys. A17 (1984) 939. ....<br/>
\end{quote}

  Although that Freudenthal Adv. Math. is listed as a reference in  
  Wangberg's paper (as reference 5), I did not see the Sudbery paper  
  listed, and I did not see the Freudenthal reference on page 96.
  Please don't misunderstand this message. I think that Wangberg's  
  thesis is very interesting. I am just trying to get a correct  
  historical record.
  Tony
  PS - In Sudbery's 1984 paper, he not only says (at page 950)
  "... sl(3,K) ... When K = O, this Lie algebra is a n on-compact form  
  of the exceptional Lie algebra E_{6}, the maximal compact subalgebra  
  being F_{4} ..."
  but he goes on to say
  "... sp(6,K) ... when K = O it is a non-compact form of E_{7}, the  
  maximal compact subalgebra being E6 \oplus  so(2). ...".
\end{quote}



For more discussion, go to the
<a href = "http://golem.ph.utexas.edu/category/2007/12/this_weeks_finds_in_mathematic_20.html">\emph{n}-Category Caf&eacute;</a>.


\par\noindent\rule{\textwidth}{0.4pt}
<em>If nature has made any one thing less susceptible than all others 
 of exclusive property, it is the action of the thinking power called 
 an idea, which an individual may exclusively possess as long as he 
 keeps it to himself; but the moment it is divulged, it forces itself 
 into the possession of every one, and the receiver cannot dispossess 
 himself of it.  Its peculiar character, too, is that no one possesses 
 the less, because every other possesses the whole of it. </em> -
Thomas Jefferson

\par\noindent\rule{\textwidth}{0.4pt}

% </A>
% </A>
% </A>
