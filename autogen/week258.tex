
% </A>
% </A>
% </A>
\week{November 25, 2007 }




Happy Thanksgiving!  Today I'll talk about a conjecture by Deligne
on Hochschild cohomology and the little 2-cubes operad.  

But first I'll talk about... dust!

I began "<A HREF = "week257.html">week257</A>" with some
chat about about dust in a binary star system called the Red
Rectangle.  So, it was a happy coincidence when shortly thereafter, I
met an expert on interstellar dust.

I was giving some talks at James Madison University in Harrisonburg, 
Virginia.  They have a lively undergraduate physics and astronomy 
program, and I got a nice tour of some labs - like Brian Utter's 
granular physics lab.  

It turns out nobody knows the equations that describe the flow of 
grainy materials, like sand flowing through an hourglass.  It's a 
poorly understood state of matter!  Luckily, this is a subject where 
experiments don't cost a million bucks.  

Brian Utter has a nice apparatus consisting of two clear plastic
sheets with a bunch of clear plastic disks between them - big
"grains".  And, he can make these grains "flow".
Since they're made of a material that changes its optical properties
under stress, you can see "force chains" flicker in and out
of existence as lines of grains get momentarily stuck and then come
unstuck!

These force chains look like bolts of lightning:

<div align = "center">
<img src = "utter_force_chains.jpg">
</div>

1) Brian Utter and R. P. Behringer, Self-diffusion in dense
granular shear flows, Physical Review E 69, 031308 (2004).
Also available as <a href = "http://arXiv.org/abs/cond-mat/0402669">
arXiv:cond-mat/0402669</a>.

I wonder if conformal field theory could help us understand these 
simplified 2-dimensional models of granular flow, at least near some 
critical point between "stuck" and "unstuck" flow.  
Conformal field theory tends to be good at studying critical points in 2d 
physics.  

% <a name = "PAH">
Anyway, I'm digressing.  After looking at a chaotic double pendulum 
in another lab, I talked to Harold Butner about his work using radio 
astronomy to study interstellar dust.  

He told me that the dust in the Red Rectangle contains a lot of PAHs -
"polycyclic aromatic hydrocarbons".  These are compounds made of 
hexagonal rings of carbon atoms, with some hydrogens along the edges.

<div align = "center">
<img src = "PAH_molecules.jpg">
</div>

On Earth you can find PAHs in soot, or the tarry stuff that forms 
in a barbecue grill.  Wherever carbon-containing materials suffer
incomplete combustion, you'll find PAHs.  

Benzene has a single hexagonal ring, with 6 carbons and 6 hydrogens.
You've probably heard about naphthalene, which is used for mothballs.
This consists of two hexagonal rings stuck together.  True PAHs have
more.  Anthracene and phenanthrene consist of
three rings:

<div align = "center">
<img src = "80px-Anthracene_acsv.svg.png">
<br/>
<font size = "-1">
<a href = "http://en.wikipedia.org/wiki/Anthracene">Anthracene</a>
</font>
</div>

<div align = "center">
<img src = "80px-Phenanthrenec1.png">
<br/>
<font size = "-1">
<a href = "http://en.wikipedia.org/wiki/Phenanthrene">Phenanthrene</a>
</font>
</div>

Napthacene, pyrene,
triphenylene and chrysene consist of four:

<div align = "center">
<img src = "100px-Tetracene.png">
<br/>
<font size = "-1">
<a href = "http://en.wikipedia.org/wiki/Tetracene">
Napthacene</a> (also known as tetracene)
</font>
</div>

<div align = "center">
<img src = "100px-Pyrene.png">
<br/>
<font size = "-1">
<a href = "http://en.wikipedia.org/wiki/Pyrene">Pyrene</a>
</font>
</div>

<div align = "center">
<img src = "100px-Triphenylene_chemical_structure.png">
<br/>
<font size = "-1">
<a href = "http://en.wikipedia.org/wiki/Triphenylene">Triphenylene</a>
</font>
</div>

<div align = "center">
<img src = "100px-Chrysene.png">
<br/>
<font size = "-1">
<a href = "http://en.wikipedia.org/wiki/Chrysene">Chrysene</a>
</font>
</div>



and so on:

2) Wikipedia, Polycyclic aromatic hydrocarbon, 
<a href = "http://en.wikipedia.org/wiki/Polycyclic_aromatic_hydrocarbon">http://en.wikipedia.org/wiki/Polycyclic_aromatic_hydrocarbon</a>

In 2004, a team of scientists discovered anthracene and pyrene in the 
Red Rectangle!  This was first time such complex molecules had been 
found in space:

3) Uma P. Vijh, Adolf N. Witt, and Karl D. Gordon, Small polycyclic 
aromatic hydrocarbons in the Red Rectangle, The Astrophysical 
Journal, 619 (2005) 368-378.  Also available as 
<a href = "http://arXiv.org/abs/astro-ph/0410130">arXiv:astro-ph/0410130</a>.

By now, lots of organic molecules have been found in interstellar 
or circumstellar space.  There's a whole "ecology" of organic 
chemicals out there, engaged in complex reactions.  Life on planets
might someday be seen as just an aspect of this larger ecology.  

I've read that about 10% of the interstellar carbon is in the form 
of PAHs - big ones, with about 50 carbons per molecule.  They're 
common because they're incredibly stable.  They've even been found 
riding the shock wave of a supernova explosion!

PAHs are also found in meteorites called "carbonaceous chondrites".  
These space rocks contain just a little carbon - about 3% by weight.  
But, 80% of this carbon is in the form of PAHs.

Here's an interview with a scientist who thinks PAHs were important 
precursors of life on Earth:

5) Aromatic world, interview with Pascale Ehrenfreund,
Astrobiology Magazine, available at
<a href = "http://www.astrobio.net/news/modules.php?op=modload\text{\&} name=News\text{\&} file=article\text{\&} sid=1992">http://www.astrobio.net/news/modules.php?op=modload\text{\&} name=News\text{\&} file=article\text{\&} sid=1992</a>

And here's a book she wrote, with a chapter on organic molecules 
in space:

6) Pascale Ehrenfreud, editor, Astrobiology: Future Perspectives,
Springer Verlag, 2004.

Harold Butner also told me about dust disks that have been seen around 
the nearby stars Vega and Epsilon Eridani.  By examining these disks, 
we may learn about planets and comets orbiting these stars.  Comets 
emit a lot of dust, and planets affect its motion.

Mathematicians will be happy to know that \emph{symplectic geometry}
is required to simulate the motion of this dust:

7) A. T. Deller and S. T. Maddison, Numerical modelling of 
dusty debris disks, Astrophys. J. 625 (2005), 398-413.
Also available as <A HREF = "http://xxx.lanl.gov/abs/astro-ph/0502135">arXiv:astro-ph/0502135</A>

Okay... now for a bit about Hochschild cohomology.   I want to
outline a conceptual proof of Deligne's conjecture that the cochain 
complex for Hochschild cohomology is an algebra for the little 
2-cubes operad.   There are a bunch of proofs of this by now.
Here's a great introduction to the story:

8) Maxim Kontsevich, Operads and motives in deformation quantization,
available as <a href =
"http://arXiv.org/abs/math/9904055">arXiv:math/9904055</a>.

I was inspired to seek a more conceptual proof by some conversations 
I had with Simon Willerton in Sheffield this summer, and this paper 
of his:

9) Andrei Caldararu and Simon Willerton, The Mukai pairing, I: a
categorical approach, available as <a href =
"http://arXiv.org/abs/0707.2052">arXiv:0707.2052</a>.

But, while trying to write up a sketch of this more conceptual 
proof, I discovered that it had already been worked out:

10) Po Hu, Igor Kriz and Alexander A. Voronov, On Kontsevich's Hochschild
cohomology conjecture, available at <a href =
"http://arXiv.org/abs/math.AT/0309369">arXiv:math.AT/0309369</a>.

This was a bit of a disappointment - but also a relief.  It
means I don't need to worry about the technical details: you 
can just look them up!  Instead, I can focus on sketching the 
picture I had in mind.

If you don't know anything about Hochschild cohomology, don't worry!
It only comes in at the very end.  In fact, the conjecture
follows from something simpler and more general.  So, what you 
really need is a high tolerance for category theory, homological 
algebra and operads.

First, suppose we have any monoidal category.  Such a category
has a tensor product and a unit object, which we'll call I.  Let 
end(I) be the set of all endomorphisms of this unit object. 

Given two such endomorphisms, say 

f: I \to  I

and

g: I \to  I

we can compose them, getting

f o g: I \to  I

This makes end(I) into a monoid.  But we can also tensor f and
g, and since I \otimes  I is isomorphic to I in a specified way, 
we can write the result simply as

f \otimes  g: I \to  I

This makes end(I) into a monoid in another, seemingly different
way.  

Luckily, there's a thing called the Eckmann-Hilton argument which 
says these two ways are equal.  It also says that end(I) is a
\emph{commutative} monoid!  It's easiest to understand this argument 
if we write f o g vertically, like this:

\begin{verbatim}
     f

     g
\end{verbatim}
    

and f \otimes  g horizontally, like this:

\begin{verbatim}
    f g
\end{verbatim}
    

Then the Eckmann-Hilton argument goes as follows:

\begin{verbatim}
     f       1 f                 g 1       g
         =         =   g f   =         = 
     g       g 1                 1 f       f
\end{verbatim}
    
Here 1 means the identity morphism 1: I \to  I.  Each step in the 
argument follows from standard stuff about monoidal categories.  
In particular, an expression like

\begin{verbatim}
     f g

     h k
\end{verbatim}
    

is well-defined, thanks to the interchange law

(f \otimes  g) o (h \otimes  k) = (f o h) \otimes  (g o k)

If we want to show off, we can say the interchange law says we've got
a "monoid in the category of monoids" - and the Eckmann-Hilton
argument shows this is just a monoid.  See "<a href =
"week100.html">week100</a>" for more.

But the cool part about the Eckmann-Hilton argument is that we're 
just moving f and g around each other.  So, this argument has a 
topological flavor!  Indeed, it was first presented as an argument
for why the second homotopy group is commutative.  It's all about 
sliding around little rectangles... or as we'll soon call 
them, "little 2-cubes".  

Next, let's consider a version of this argument that holds only
"up to homotopy".  This will apply when we have not a \emph{set}
of morphisms from any object X to any object Y, but a <em>chain 
complex</em> of morphisms.  

Instead of getting a set end(I) that's a commutative monoid, we'll 
get a cochain complex END(I) that's a commutative monoid "up to 
coherent homotopy".  This means that the associative and
commutative laws hold up to homotopies, which satisfy their
own laws up to homotopy, ad infinitum.

More precisely, END(I) will be an "algebra of the little 2-cubes 
operad".  This implies that for every configuration of n little 
rectangles in a square:

\begin{verbatim}
        ---------------------
       |                     |
       |           -----     |
       | -----    |     |    | 
       ||     |   |     |    |
       ||     |   |     |    |           
       | -----    |     |    |            
       |           -----     |     
       |   ----------------  |
       |  |                | |
       |   ----------------  |
       |                     |
        ---------------------
\end{verbatim}
    

we get an n-ary operation on END(I).  For every homotopy between
such configurations:

$$
        ---------------------            ---------------------
       |                     |          | -----               |
       |           -----     |          ||     |    ----      |
       | -----    |     |    |          ||     |   |    |     |
       ||     |   |     |    |          ||     |   |    |     |
       ||     |   |     |    |          ||     |   |    |     |        
       | -----    |     |    |   --->   | -----    |    |     |  
       |           -----     |          |           -----     |   
       |   ----------------  |          |   -------           |     
       |  |                | |          |  |       |          |
       |   ----------------  |          |   -------           |
       |                     |          |                     |
        ---------------------            ---------------------
$$
    
we get a chain homotopy between n-ary operations on END(I).  And
so on, ad infinitum.  

For more on the little 2-cubes operad, see "<A HREF = "week220.html">week220</A>".  In fact, 
what I'm trying to do now is understand some mysteries I described 
in that article: weird relationships between the little 2-cubes
operad and Poisson algebras.

But never mind that stuff now.  For now, let's see how easy it is to 
find situations where there's a cochain complex of morphisms between 
objects.  It happens throughout homological algebra!  

If that sounds scary, you should refer to a book like this as you 
read on:

10) Charles Weibel, An Introduction to Homological Algebra, 
Cambridge U. Press, Cambridge, 1994. 

Okay.  First, suppose we have an abelian category.  This provides a 
context in which we can reason about chain complexes and cochain 
complexes of objects.  A great example is the category of R-modules 
for some ring R.  

Next, suppose every object X in our abelian category has an 
"projective resolution" - that is, a chain complex 

$$
    d_{0}      d_{1}      d_{2}
X_{0} <--- X_{1} <--- X_{2} <--- ...
$$
    
where each object X^{i} is 
<a href = "http://en.wikipedia.org/wiki/Projective_module">projective</a>, and the homology groups

\begin{verbatim}
          ker (d<sub>i</sub>)  
H<sub>i</sub> =    ------------
          im (d<sub>i-1</sub>)
\end{verbatim}
    
are zero except for H^{0}, which equals X.  You should think of
a projective resolution as a "puffed-up" version of X that's
better for mapping out of than X itself.  

Given this, besides the usual set hom(X,Y) of morphisms from the 
object X to the object Y, we also get a cochain complex which I'll 
call the "puffed-up hom":

HOM(X,Y)

How does this work?  Simple: replace X by a chosen projective
resolution 

$$
X_{0} <--- X_{1} <--- X_{2} <--- ...
$$
    

and then map this whole thing to Y, getting a cochain complex

$$
hom(X_{0},Y) ---> hom(X_{1},Y) ---> hom(X_{2},Y) --- >
$$
    
This cochain complex is the puffed-up hom, HOM(X,Y).  

Now, you might hope that the puffed-up hom gives us a new category 
where the hom-sets are actually cochain complexes.  This is morally
true, but the composition

o: HOM(X,Y) \times  HOM(Y,Z) \to  HOM(X,Z)

probably isn't associative "on the nose".  However, I think
it should be associative up to homotopy!  This homotopy probably won't
satisfy the law you'd hope for - the pentagon identity.  But, it
should satisfy the pentagon identity up to homotopy!  In fact, this
should go on forever, which is what we mean by "up to coherent
homotopy".  This kind of situation is described by an infinite
sequence of shapes called "associahedra", discovered by
Stasheff (see "<A HREF = "week144.html">week144</A>").

If this is the case, instead of a category we get an
"A_{\infty }-category": a gadget where the hom-sets
are cochain complexes and the associative law holds up to coherent
homotopy.  I'm not sure the puffed-up hom gives an
A_{\infty }-category, but let's assume so and
march on.

Suppose we take any object X in our abelian category.  Then we get 
a cochain complex 

END(X) = HOM(X,X)

equipped with a product that's associative up to coherent 
homotopy.  Such a thing is known as an
"A_{\infty }-algebra".  It's just an
A_{\infty }-category with a single object, namely X.

Next suppose our abelian category is monoidal.  (To get the tensor 
product to play nice with the hom, assume tensoring with any object 
is <a href = "http://en.wikipedia.org/wiki/Exact_functor">right exact</a>.)  
Let's see what happens to the Eckmann-Hilton
argument.   We should get a version that holds "up to coherent
homotopy".  

Let I be the unit object, as before.  In addition to composition:

o: END(I) \times  END(I) \to  END(I)

tensoring should give us another product: 

\otimes : END(I) \times  END(I) \to  END(I)

which is also associative up to coherent homotopy.  So, END(I) should
be an A_{\infty }-algebra in two ways.  But, since composition
and tensoring in our original category get along nicely:

(f \otimes  g) o (h \otimes  k) = (f o h) \otimes  (g o k)

END(I) should really be an A_{\infty }-algebra in the category of 
A_{\infty }-algebras!

Given this, we're almost done.  A monoid in the category of monoids 
is a commutative monoid - that's another way of stating what the
Eckmann-Hilton argument proves.  Similarly, an A_{\infty }-algebra in 
the category of A_{\infty }-algebras is an algebra of the little 
2-cubes operad.  So, END(I) is an algebra of the little 2-cubes 
operad.

Now look at an example.  Fix some algebra A, and take our 
monoidal abelian category to have:

<ul>
<li>
    A-A bimodules as objects
<li>
    A-A bimodule homomorphisms as morphisms
</ul>
Here the tensor product is the usual tensor product of bimodules,
and the unit object I is A itself.  And, as Simon Willerton pointed 
out to me, END(I) is a cochain complex whose cohomology is familiar: 
it's the "Hochschild cohomology" of A.   

So, the cochain complex for Hochschild cohomology is an algebra of 
the little 2-cubes operad!  But, we've seen this as a consequence 
of a much more general fact.

To wrap up, here are a few of the many technical details I glossed
over above.

First, I said an projective resolution of X is a puffed-up version of X
that's better for mapping out of.  This idea is made precise in the
theory of model categories.  But, instead of calling it a
"puffed-up version" of X, they call it a "cofibrant
replacement" for X.  Similarly, a puffed-up version of X that's
better for mapping into is called a "fibrant replacement".

For a good introduction to this, try:

11) Mark Hovey, Model Categories, American Mathematical Society,
Providence, Rhode Island, 1999.

Second, I guessed that for any abelian category where every object has
a projective resolution, we can create an A_{\infty }-category
using the puffed-up hom, HOM(X,Y).  Alas, I'm not really sure this is true.

Hu, Kriz and Voronov consider a more general situation, but
what I'm calling the "puffed-up hom" should be a special case of their
"derived function complex".  However, they don't seem to say
what weakened sort of category you get using this
derived function complex - maybe an A_{\infty }-category, or
something equivalent like a quasicategory or Segal category?  They
somehow sidestep this issue, but to me it's interesting in its
own right.

At this point I should mention something well-known that's similar
to what I've been talking about.  I've been talking about the 
"puffed-up hom" for an abelian category with enough projectives.   
But most people talk about "Ext", which is the cohomology of the 
puffed-up hom:

Ext^{i}(X,Y) = H^{i}(HOM(X,Y))

And, while I want 

END(X) = HOM(X,X) 

to be an A_{\infty }-algebra, most people seem happy to have

Ext(X) = H(HOM(X,X))

be an A_{\infty }-algebra.  Here's a reference:

12) D.-M. Lu, J. H. Palmieri, Q.-S. Wu and J. J. Zhang,
A_{\infty }-structure on Ext-algebras, available as 
<a href = "http://arXiv.org/abs/math.KT/0606144">arXiv:math.KT/0606144</a>.

I hope they're secretly getting this A_{\infty }-structure on 
H(HOM(X,X)) from an A_{\infty }-structure on HOM(X,X).  They don't
come out and say this is what they're doing, but one promising 
sign is that they use a theorem of Kadeishvili, which says that 
the cohomology of an A_{\infty }-algebra is an A_{\infty }-algebra.  

Finally, the really interesting part: how do we make an
A_{\infty }-algebra in the category of
A_{\infty }-algebras into an algebra of the little 2-cubes
operad?  This is the heart of the "homotopy Eckmann-Hilton argument".

I explained operads, and especially the little k-cubes operad, back in
"<A HREF = "week220.html">week220</A>".  The little k-cubes
operad is an operad in the world of topological spaces.  It has one
abstract n-ary operation for each way of sticking n little
k-dimensional cubes in a big one, like this:

\begin{verbatim}
        ---------------------
       |                     |
       |           -----     |
       | -----    |     |    | 
       ||     |   |     |    |
       ||     |   |     |    |           typical  
       | -----    |     |    |     3-ary operation in the       
       |           -----     |     little 2-cubes operad
       |   ----------------  |
       |  |                | |
       |   ----------------  |
       |                     |
        ---------------------
\end{verbatim}
    

A space is called an "algebra" of this operad if these abstract 
n-ary operations are realized as actual n-ary operations on the 
space in a consistent way.  But, when we study the homology 
of topological spaces, we learn that any space gives a chain complex.
This lets us convert any operad in the world of topological spaces 
into an operad in the world of chain complexes.  Using this, it also 
makes sense to speak of a \emph{chain complex} being an algebra of the
little k-cubes operad.  Or, for that matter, a cochain complex.

Let's use "E(k)" to mean the chain complex version of the little 
k-cubes operad.  

An "A_{\infty }-algebra" is an algebra of a certain
operad called A-infinity.  This isn't quite the same as the operad
E(1), but it's so close that we can safely ignore the difference here:
it's "weakly equivalent".

Say we have an A_{\infty }-algebra in the category of A_{\infty }-algebras.  How do we get an algebra of the little 2-cubes operad,
E(2)?

Well, there's a way to tensor operads, such that an algebra of P\otimes Q 
is the same as a P-algebra in the category of Q-algebras.  So, an 
A_{\infty }-algebra in the category of A_{\infty }-algebras is
the same as an algebra of

A_{\infty } \otimes  A_{\infty }

Since A_{\infty } and E(1) are weakly equivalent, we can turn this
algebra into an algebra of 

E(1) \otimes  E(1)

But there's also an obvious operad map

E(1) \otimes  E(1) \to  E(2)

since the product of two little 1-cubes is a little 2-cube.
This too is a weak equivalence, so we can turn our algebra of
E(1) \otimes  E(1) into an algebra of E(2).

The hard part in all this is showing that the operad map

E(1) \otimes  E(1) \to  E(2)

is a weak equivalence.  In fact, quite generally, the map

E(k) \otimes  E(k') \to  E(k+k') 

is a weak equivalence.  This is Proposition 2 in the paper by
Hu, Kriz and Voronov, based on an argument by Gerald Dunn:

13) Gerald Dunn, Tensor products of operads and iterated loop
spaces, Jour. Pure Appl. Alg 50 (1988), 237-258.

Using this, they do much more than what I've sketched: they
prove a conjecture of Kontsevich which says that the Hochschild 
complex of an algebra of the little k-cubes
operad is an algebra of the little (k+1)-cubes operad!

That's all for now.  Sometime I should tell you how this is related
to Poisson algebras, 2d TQFTs, and much more.  But for now, you'll
have to read that in Kontsevich's very nice paper.

\par\noindent\rule{\textwidth}{0.4pt}

\textbf{Addenda:} 
Over at the \emph{n}-Category Caf&eacute;, 
Michael Batanin made some comments on the
difficulties in making my proposed argument rigorous, his own work in
doing just this (long before I came along), and the history of Deligne's
conjecture (which I deliberately didn't go into, since it's such a
long story).  Mikael Vejdemo Johansson explained more about the
A_{\infty }-structure on Ext.

Modulo some typographical changes, Michael Batanin wrote:

\begin{quote}

   Hi, John.
   Just a few remarks about your stuff on Deligne's conjecture. 
   Unfortunately, technical details are important in this business.

   First, we have to be careful about tensor product of operads. A
   very long standing question is: Let A be a E<sub>1</sub>-operad and
   B be a cofibrant E<sub>1</sub>-operad.  Is it true that their
   tensor product A &otimes; B is an E<sub>2</sub>-operad?  The answer
   is unknown, even though Dunn's argument is correct and the tensor
   product of two little 1-cube operads is equivalent to the little
   2-cube operad.  Unfortunately, the theorem from Hu, Kriz and
   Voronov is based implicitly on an affirmative answer to the above
   question.

   I think the history of Deligne's conjecture is quite remarkable and 
   complicated and still developing. The most conceptual and correct 
   proof I know is provided by Tamarkin in 

   14) Dmitry Tamarkin, What do DG categories form?, available as <a
   href =
   "http://arXiv.org/abs/math.CT/0606553">arXiv:math.CT/0606553</a>.

   And it uses my up to homotopy Eckmann--Hilton argument. This argument 
   is based on a techniques of compactification of configuration spaces 
   and first was proposed by Getzler and Jones.  I think I already wrote 
   about it in a post to n-category cafe where Dolgushev's work was 
   discussed. Here is the reference to my lecture about Deligne's conjecture:

   15) Michael A. Batanin, Deligne's conjecture: an interplay between
   algebra, geometry and higher category theory, talk at ANU Canberra,
   November 3 2006, available at 
   <a href = "http://www.math.mq.edu.au/~street/BataninMPW.pdf">http://www.math.mq.edu.au/~street/BataninMPW.pdf</a>
 
   Concerning your idea to construct an A-infinity category using 
   Hom(PX,Y), where PX is a projective resolution: it's been done by 
   me many years ago and in a more general situation.  It is long story 
   to tell but more or less I prove that your Hom functor is equivalent 
   as a simplicially coherent bimodule to the homotopy coherent left 
   Kan extension of the inclusion functor 

   Projective bounded chain complex &rarr; Bounded chain complex 

   along itself.  Then the Kleisli category of this distributor has a 
   canonical A-infinity structure and this Kleisli category is equivalent 
   in an appropriate sense to your 'puffed' category.  In fact, the 
   situation I consider in my paper is much more general and includes 
   simplicial Quillen categories as a very special example.  The paper is:

   16) Michael A. Batanin, Categorical strong shape theory, Cahiers de 
   Topologie et Geom. Diff., V.XXXVIII-1 (1997), 3-67. 

   and its companion 

   17) Michael A. Batanin, Homotopy coherent category theory and 
   A<sub>&infin;</sub> structures in monoidal categories, Jour. Pure Appl. 
   Alg. 123 (1998), 67-103.

  Regards,<br/>
  Michael 
\end{quote}
    

Batanin's talk has a very nice introduction to his "derived 
Eckmann-Hilton argument", which is a precise version of what I was 
attempting to sketch in this Week's Finds.  Here's the paper by
Getzler and Jones:

18) Ezra Getzler and J. D. S. Jones, Operads, homotopy algebra and 
iterated integrals for double loop spaces, available as 
<a href = "http://arXiv.org/abs/hep-th/9403055">arXiv:hep-th/9403055</a>.

It's very interesting, but it was never published, perhaps because 
of some subtle flaws caught by Tamarkin. 

Modulo some typographical changes and extra references, Mikael Vejdemo
Johansson wrote:

\begin{quote}
   I could try to claim that I'm starting to become an expert on
   things A<sub>&infin;</sub>, but given that Jim Stasheff is an avid
   commenter here, I don't quite dare to. :)

   However, I have read the Lu-Palmieri-Wu-Zhang [LPWZh] paper
   mentioned in the exposition backwards and forwards.  On the face, what
   LPWZh try to do is to take the survey articles by Bernhard Keller:

   19) Bernhard Keller, Introduction to A<sub>&infin;</sub>-algebras and modules,
   available as <a href = "http://arxiv.org/abs/math/9910179">arXiv:math/9910179</a>.

   A brief introduction to A<sub>&infin;</sub>-algebras, notes from a talk at 
   the workshop on Derived Categories, Quivers and Strings, Edinburgh, 
   August 2004.  Available at
   <a href = "http://www.institut.math.jussieu.fr/~keller/publ/index.html">http://www.institut.math.jussieu.fr/~keller/publ/index.html</a>

   A<sub>&infin;</sub>-algebras in representation theory, contribution to the 
   Proceedings of ICRA IX, Beijing 2000.  Available at 
   <a href = "http://www.institut.math.jussieu.fr/~keller/publ/index.html">http://www.institut.math.jussieu.fr/~keller/publ/index.html</a>
  
   A<sub>&infin;</sub>-algebras, modules and functors, available as 
   <a href = "http://arxiv.org/abs/math/0510508">arXiv:math/0510508</a>.

   outlining the use of A<sub>&infin;</sub>-algebras in representation theory,
   and widening the scope of their proven usability while actually
   proving the many unproven and interesting statements that Keller
   makes.

   At the core of this lies two different theorems. One is the
   Kadeishvili theorem (which in various guises has been proven by
   everyone involved with A<sub>&infin;</sub>-algebras, and a few more, in my
   impression ;) that says that you can carry A<sub>&infin;</sub>-algebras
   across taking homology.  Kadeishvili's argument specializes to the 
   case where you start with an A<sub>&infin;</sub>-algebra with only m<sub>1</sub> and 
   m<sub>2</sub> are non-trivial - i.e. a plain old dg-algebra.  For higher 
   generality, you'd probably want to turn to the Homology Perturbation 
   Theory crowd with Stasheff, Gugenheim and Huebschmann among the more 
   famous names...

   Hence, if we take graded endomorphism algebra of a resolution 
   of M and introduce the "homotopy differential": 

   &part; f = d f + f d

   then cycles are chain maps and the homology picks out exactly the
   algebra cohomology over the appropriate module category. Thus, we 
   get Ext as the homology of a dg-algebra, and thus, Ext has an
   A<sub>&infin;</sub>-algebra structure.

   The second cornerstone of these papers is the Keller higher
   multiplication theorem: if the ring R is sufficiently nice, then 
   the A<sub>&infin;</sub>-algebra structure on Ext<sub>R</sub>*(M,M) for some 
   appropriate module M will allow you to recover a presentation of 
   R explicitly.

   I hope this answers your question about the origin of their
   A<sub>&infin;</sub>-algebra structure.
\end{quote}
    

Note the great technical simplification of working with what 
I called hom(PX,PX) instead of hom(PX,X) - composition becomes
strictly associative!



For more discussion, go to the
<a href = "http://golem.ph.utexas.edu/category/2007/11/this_weeks_finds_in_mathematic_22.html">\emph{n}-Category Caf&eacute;</a>.

\par\noindent\rule{\textwidth}{0.4pt}
<em>We need a really short and convincing argument for this very 
fundamental fact about the Hochschild complex.</em> - Maxim Kontsevich
<em>Higher category theory provides us with the argument Kontsevich
was looking for.</em> - Michael Batanin

<HR>

% </A>
% </A>
% </A>


% parser failed at source line 1007
