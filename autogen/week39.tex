
% </A>
% </A>
% </A>
\week{eptember 24, 1994}

I want to say a bit about Alain Connes' book, newly out in English, and
then some about Yang-Mills theory in 2 dimensions.

1) Noncommutative Geometry, by Alain Connes, Academic Press, 640 pp.

You know something is up when a prominent mathematical physicist (Daniel
Kastler) says "Alain is great.  I am just his humble prophet."  (This
happened at a conference at Penn State I just went to.)  What is
noncommutative geometry and what's so great about it?

Basically, the idea of noncommutative geometry is to generalize geometry
to "quantum spaces".  For example, the ordinary plane has two functions
on it, the coordinate functions x and y, which commute: xy = yx.  We can
think of x and y as representing the position and momentum of a
classical particle.  But when we consider a quantum-mechanical particle,
we must give up commutativity and instead impose the "canonical
commutation relations" xy - yx = i \hbar , where \hbar  is Planck's
constant.  Now x and y are not really functions on any space at all, but
simply elements of a noncommutative algebra.  Still, we can try our best
to \emph{pretend} that they are functions on some mysterious sort of
"quantum space" in which knowing one coordinate of a point
precisely precludes us from knowing the other coordinate exactly, by the
Heisenberg uncertainty principle.  Mathematically, noncommutative
geometry consists of 1) expressing the geometry of spaces algebraically
in terms of the commutative algebra of functions on them, and 2) then
generalizing the results to classes of noncommuative algebras.

The main trick invented by Connes was to come up with a substitute for
the "differential forms" on a space.  Differential forms are the bread
and butter of modern geometry.  If we start with a commutative algebra A
(say the algebra of smooth functions on some manifold like the plane),
we can form the algebra of differential forms over A by introducing, for
each element f in A, a formal symbol df, and imposing the following
rules:


\begin{verbatim}

d(f+g) = df + dg
d(cf) = c df       (c a constant)
d(fg) = (df)g + f dg
fdg = (dg)f
df dg = -dg df.
\end{verbatim}
    

More precisely, the differential forms over A are the algebra generated
by A and these differentials df, modulo the above relations.  This gives
a purely algebraic way of understanding what those mysterious things
like dx dy dz in integral signs are.  

Now, the last two of the five rules listed above fit nicely with the
commutative of A when it \emph{is} commutative, but they jam up the works
horribly otherwise.  So: how to generalize differential forms to the
noncommutative case?  There are various things one can do if A is
commutative in some generalized sense, such as "supercommutative" or
"braided commutative" (which I call "R-commutative" in some papers on this
subject).  However, if A is utterly noncommutative, it seems that the
best approach is Connes', which is first to \emph{throw out} the last two
relations, obtaining something folks call the "differential envelope" of
A or the "universal differential graded algebra" over A --- which is
pleasant but quite boring by itself --- and then to consider "chains"
which are linear maps F from this gadget to the complex numbers (or
whatever field you're working in) satisfying the cyclic property

F(uv) = (-1)^{ij} F(vu)

where u is something that looks like f_0 df_1 df_2 .... df_i, and
v is something like g_0 dg_1 dg_2 .... dg_j.  There are charming things
one can do with chains that wind up letting one do most of what one
could do with differential forms.  More precisely, just as differential
forms allow you entry into the wonderful world of DeRham cohomology, 
chains let you develop something similar called cyclic homology (and
there is a corresponding cyclic cohomology that's even more like the
DeRham theory).   

Connes, being extremely inventive and ambitious, has applied
noncommutative differential geometry to many areas: index theory,
K-theory, foliations, Penrose tilings, fractals, the quantum Hall
effect, and even elementary particle physics.  Perhaps the most
intriguing result is that if one develops the Yang-Mills equations 
using the techniques of noncommutative geometry, but with a very simple
"commutative" model of spacetime, namely a two-sheeted cover of ordinary
spacetime, the Higgs boson falls out rather magically on its own.  This
has led Kastler and other physicists to pursue a reformulation of the
whole Standard Model in terms of noncommutative geometry, hoping to
simplify it and even make some new predictions.  It is far too early to
see if this approach will get somewhere useful, but it's certainly
interesting.  

I haven't read this book, just part of the French version on which it's
based (with extensive additions), but my impression is that it's quite
easy to read given the technical nature of the subject.


2) 2d Yang-Mills theory and topological field theory, by Gregory Moore,
available as <A HREF = "http://xxx.lanl.gov/abs/hep-th/9409044">hep-th/9409044</A>.  

This is a nice review of recent work on 2d Yang-Mills theory.  While
Yang-Mills theory in 4 dimensions is the basis of our current theories
of the strong, weak, and electromagnetic forces, and mathematically
gives rise to a cornucopia of deep results about 4-dimensional topology,
2d Yang-Mills theory has traditionally been considered "trivial" in that
one can exactly compute pretty much whatever one wants.  However,
Witten, in "On quantum gauge theories in two dimensions" (see "<A HREF = "week36.html">week36</A>"),
showed that precisely because 2d Yang-Mills theory was exactly soluble,
one could use it to study a lot of interesting mathematics problems
relating to "moduli spaces of flat connections."  (More about those
below.)  And Gross, Taylor and others have recently shown that 2d
Yang-Mills theory, at least working with gauge groups like SU(N) or
SO(N) and taking the "large N limit", could be formulated as a string
theory.  So people respect 2d Yang-Mills theory more these days; its
complexities stand as a strong clue that we've just begun to tap the
depths of 4d Yang-Mills theory!

I can't help but add that Taylor and I did some work a while back in
which we formulated SU(N) 2d Yang-Mills theory for \emph{finite} N as a
string theory.  This was meant as evidence for my proposal that the loop
representation of quantum gravity is a kind of string theory, a
proposal described in "<A HREF = "week18.html">week18</A>".  For more on this sort of thing, try
my paper in the book Knots and Quantum Gravity (see "<A HREF = "week23.html">week23</A>") --- which
by the way is finally out --- and also the following:

3) Strings and two-dimensional QCD for finite N, by J. Baez and W.
Taylor, 19 pages in LaTeX format available as <A HREF = "http://xxx.lanl.gov/abs/hep-th/9401041">hep-th/9401041</A>, or by ftp
from math.ucr.edu as "baez/string2.tex", to appear in Nuc. Phys. B.


When it comes to "moduli spaces of flat connections", it's hard to say
much without becoming more technical, but I certainly recommend starting
with the beautiful work of Goldman:

4) The symplectic nature of fundamental groups of surfaces, by W.
Goldman, Adv. Math. 54 (1984), 200-225.

Invariant functions on Lie groups and Hamiltonian flows of surface
group representations, by W. Goldman, Invent. Math. 83 (1986), 263-302.

Topological components of spaces of representations, by W. Goldman, Invent.
Math. 93 (1988), 557-607.


The basic idea here is to take a surface S with a particular G-bundle on
it, and carefully study the space of flat connections modulo gauge
transformations, which will be a finite-dimensional stratified space.
If you fix G and S, no matter what bundle you pick, this space will
appear as a subspace of a bigger space called the moduli space of flat
connections, which is the same as Hom(\pi _1(S),G)/Ad G.  There is an open
dense set of this space, the "top stratum", which is a symplectic
manifold.  Geometric quantization of this manifold has everything in the
world to do with Chern-Simons theory, as summarized so deftly by Atiyah:

5) "The Geometry and Physics of Knots," by Michael Atiyah, Cambridge U.
Press, Cambridge, 1990. 

On the other hand, lately people have been using 2d Yang-Mills theory, BF
theory, and the like (see "<A HREF = "week36.html">week36</A>") to get a really thorough handle on
the cohomology of the moduli space of flat connections.  For a
mathematical approach to this problem that doesn't talk much about gauge
theory, try:

6) Group cohomology construction of the cohomology of moduli spaces of
flat connections on 2-manifolds, by Lisa C. Jeffrey, preprint available
from Princeton U. Mathematics Department.  
\par\noindent\rule{\textwidth}{0.4pt}

% </A>
% </A>
% </A>
