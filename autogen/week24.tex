
% </A>
% </A>
% </A>
\week{October 31, 1993}

This Week's Finds in Mathematical Physics (Week 24)
John Baez

I will now revert to topics more directly connected to physics and start
catching on the papers that have been accumulating.  First, two very
nice review papers:

1) Prima facie questions in quantum gravity, by Chris Isham, lecture at
Bad Honeff, September 1993, preprint available in LaTeX form as
<A HREF = "http://xxx.lanl.gov/abs/gr-qc/9310031">gr-qc/9310031</A>. 

If one wants to know why people make such a fuss about quantum gravity,
one could not do better than to start here.  There are many approaches
to the project of reconciling quantum mechanics with gravity, all of
them rather technical, but here Isham focuses on the "prima facie"
questions that present themselves no matter \emph{what} approach one uses.  
He even explains why we should study quantum gravity - a nontrivial
question, given how difficult it has been and how little practical
payoff there has been so far!  Let me quote his answers and urge you to 
read the rest of this paper:

\par\noindent\rule{\textwidth}{0.4pt}
\textbf{We must say something.}  The value of the Planck length suggests that
quantum gravity should be quite irrelevant to, for example, atomic
physics.  However, the non-renormalisability of the perturbative theory
means it is impossible to actually compute these corrections, even if
physical intuition suggests they will be minute.  Furthermore, no
consistent theory is known in which the gravitational field is left
completely classical.  Hence we are obliged to say \emph{something} about
quantum gravity, even if the final results will be negligible in all
normal physical domains.

\textbf{Gravitational singularities.}  The classical theory of
general relativity is notorious for the existence of unavoidable
spacetime singularities.  It has long been suggested that a quantum
theory of gravity might cure this disease by some sort of `quantum
smearing'.

\textbf{Quantum cosmology.}  A particularly interesting singularity
is that at the beginning of a cosmological model described by, say, a
Robertson-Walker metric.  Classical physics breaks down here, but one
of the aims of quantum gravity has always been to describe the
`origin' of the universe as some type of quantum event.

\textbf{The end state of the Hawking radiation process.}  One of the most
striking results involving general relativity and quantum theory is
undoubtedly Hawking's famous discovery of the quantum thermal radiation
produced by a black hole.  Very little is known of the final fate of such
a system, and this is often taken to be another task for a quantum
theory of gravity.

\textbf{The unification of fundamental forces.}  The weak and electromagnetic
forces are neatly unified in the Salam-Weinberg model, and there has
also been a partial unification with the strong force.  It is an
attractive idea that a consistent quantum theory of gravity \emph{must}
include a unification of all the fundamental forces.

\textbf{The possibility of a radical change in basic physics.}  The deep
incompatibilities between the basic structures of general relativity and
of quantum theory have lead many people to feel that the construction of
a consistent theory of quantum gravity requires a profound revision of
the most fundamental ideas of modern physics.  The hope of securing such
a paradigm shift has always been a major reason for studying the
subject.
\par\noindent\rule{\textwidth}{0.4pt}

2) Lectures on 2d gauge theories: topological aspects and path integral
techniques, by Matthias Blau and George Thompson, 70 pages, preprint
available in LaTeX form as <A HREF = "http://xxx.lanl.gov/abs/hep-th/9310144">hep-th/9310144</A>.

Most of the basic laws of physics appear to be gauge theories.  Gauge
theories are tricky to deal with because they are inherently nonlinear.
(At least the "nonabelian" ones are - the main example of an abelian
gauge theory is Maxwell's equations.)  People have been working hard for
quite some time trying to develop tools to study gauge theories on their
own terms, and \emph{one} reason for the interest in gauge theories in
2-dimensional spacetime is that life is simple enough in this case to
exactly solve the theories and see precisely what's going on.  Another
reason is that in string theory one becomes interested in gauge fields
living on the 2-dimesional "string worldsheet."  

This paper is a thorough review of two kinds of gauge theories in 2
dimensions: topological Yang-Mills theory (also called BF theory) and
the G/G gauged Wess-Zumino-Witten model.  Both of these are of great
mathematical interest in addition to their physical relevance.  Studying
the BF theory gives a way to do integrals on the moduli space of flat
connections on a bundle over a Riemann surface, while studying the G/G
model amounts to a geometric construction of the categories of
representations of quantum groups at roots of unity.  (Take my word for
it, mathematicians find these important!)

I have found this review a bit rough going so far because the authors
like to use supersymmetry to study these models.  But I will continue
digging in, since the authors consider the following topics (and I
quote): solution of Yang-Mills theory on arbitrary surfaces; calculation
of intersection numbers of moduli spaces of flat connections; coupling
of Yang-Mills theory to coadjoint orbits and intersection numbers of
moduli spaces of parabolic bundles; derivation of the Verlinde formula
from the G/G model; derivation of the shift k to k+h in the G/G model
via the index of the twisted Dolbeault complex.

3) Semi-classical limits of simplicial quantum gravity, by J. W. Barrett
and T. J. Foxon, preprint available as <A HREF = "http://xxx.lanl.gov/abs/gr-qc/9310016">gr-qc/9310016</A>.

This paper looks at quantum gravity in 3 spacetime dimesions formulated
along the lines of Ponzano and Regge, that is, with the spacetime
manifold replaced by a bunch of tetrahedra (a "simplicial complex").  
I describe some work along these lines in "<A HREF = "week16.html">week16</A>".  Here the Feynman
path integral is replaced by a discrete sum over states, in which the
edges of the tetrahedra are assigned integer or half-integer lengths,
which really correspond to "spins," and the formula for the action is
given in terms of 6j-symbols.  The authors look for stationary points of
this action and find that some correspond to Riemannian metrics and some
correspond to Lorentzian metrics.  This is strongly reminiscent of
Hartle and Hawking's work on quantum cosmology,

4) Wave function of the universe, by J. B. Hartle and S. W. Hawking,
Phys. Rev. D28 (1983), 2960.

in which there is both a Euclidean and a Lorentzian regime (providing a
most fascinating answer to the old question, "what came before the big
bang!).  Here, however, the path integral is oscillatory in the
Euclidean regime and exponential in the Lorentzian one - the opposite of
what Hartle and Hawking had.  This puzzles me.

5) Generalized measures in gauge theory, by John Baez, available in
LaTeX as <A HREF = "http://xxx.lanl.gov/abs/hep-th/9310201">hep-th/9310201</A>.

Path integrals in gauge theory typically invoke the concept of Lebesgue
measure on the space of connections.  This is roughly an
infinite-dimensional vector space, and there \emph{is} no ``Lebesgue
measure'' on an infinite-dimensional vector space.  So what is going on?
Physicists are able to do calculations using this concept and get useful
answers - mixed in with infinities that have to be carefully
``renormalized.''  Some of the infinities here are supposedly due to the
fact that one should really be working no on the space of connections,
but on a quotient space, the connections modulo gauge transformations.
But not all the infinities are removed this way, and mathematically the
whole situation is enormously mysterious.

Recently Ashtekar, Isham, Lewandowski and myself have been looking at a
way to generalize the concept of measure, suggested by earlier work on
the ``loop representation'' of gauge theories.  Ashtekar and Lewandowski
managed to rigorously construct a kind of ``generalized measure'' on the
space of connections modulo gauge transformations that acts formally
quite a bit like what might hope for.  In this paper I show how can
define generalized measures directly on the space of connections.  All
of these project down to generalized measures on the space of
connections modulo gauge transformations, but even when one is
interested in gauge-invariant quantities, it is sometimes easier to work
``upstairs.''  In particular, when the gauge group is compact, there is
a ``uniform'' generalized measure on the space of connections that
projects down to the measure constructed by Ashtekar and Lewandowski.
This generalized measure is in some respects a rigorous substitute for
the ill-defined ``Lebesgue measure,'' but it is actually built using
Haar measure on G.  I also define generalized measures on the group of
gauge transformations (which is an infinite-dimensional group), and when
G is compact I construct a natural example that is a rigorous substitute
for Haar measure on the group of gauge transformations .  As an
application of this ``generalized Haar measure'' I show that any
generalized measure on the space of connections can be averaged against
generalized Haar measure to give a gauge-invariant generalized measure
on the space of connections.

This doesn't, by the way, mean the problems I mentioned at the beginning
are solved!
\par\noindent\rule{\textwidth}{0.4pt}

% </A>
% </A>
% </A>
