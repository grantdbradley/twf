
% </A>
% </A>
% </A>
\week{July 15, 2000 }

This one is going to be a bit rough at the edges, because in a
few hours I'm taking a plane to London.  I'm going to the International
Congress on Mathematical Physics, where I'll get to hear talks by 
Ashtekar, Atiyah, Buchholz, Connes, Dijkgraaf, Donaldson, Faddeev, 
Freed, Froehlich, Kreimer, Ruelle, Schwartz, Shor, Thirring, 't Hooft, 
and other math/physics heavyweights.  I'm also gonna talk a bit myself -
they'd have to pay me to shut up!  I hope to report on this stuff in 
future issues.  

But today, I want to say a bit about counting.

Archimedes loved to count.  In his Sand Reckoner, he invented a notation 
for enormous numbers going far beyond what the Greeks had previously 
considered.  He made up a nice problem to showcase these large numbers:
how many grains of sand would it take to fill the universe?   He then
computed an upper bound, based on assumptions such as these:

A) No more than 10,000 grains of sand can fit into a sphere whose 
diameter was 1/40th a finger-width.

B) The circumference of the earth is no more than 3,000,000 stades.
A "stade" is about 160 meters - different Greek cities used different
stades, so it difficult to be very precise about this.

C) The diameter of the earth is greater than the diameter of the moon.

D) The diameter of the sun is no more than 30 times the diameter of the 
moon.  (Of course this one is way off!)

E) The diameter of the sun is greater than the side of a regular chiliagon 
inscribed in a great circle in the sphere of the universe.  A chiliagon 
is a thousand-sided polygon.


 He concluded that no more than 10^{63} grains of sand would
be needed to fill the universe.  Of course, he didn't use modern
exponential notation!  Instead, he used a system of his own devising.
The largest number the Greeks had a notation for was a "myriad
myriads", or 10^{8}, since a "myriad" means
10,000.  Archimedes called 10^{8} a number of the "first
order".  He then invented a number of the "second order",
namely 10^{16}, and the "third
order", namely 10^{24} - and so on, up to the
myriad-myriadth order, i.e. 10^{8} to the 10^{8}th
power.

He then said all these numbers were of the "first period", and went on to
define higher periods of numbers, up to a number of the myriadth period, 
which was 10^{80,000,000,000,000,000}.   After this exercise, the number
of grains of sand in the universe must have seemed rather puny - merely 
a thousand myriads of numbers of the eighth order!

Actually, this counting exercise is one of Archimedes' lesser feats.
He pioneered many of the concepts of mechanics and calculus.  He also
had the neat idea to use mechanical methods to do calculations and 
"prove theorems".  He wrote about this in a treatise called "Methods of 
Mechanical Theorems".  There is only one surviving copy of this treatise,
and that is a fascinating story in itself.  It is part of the "Archimedes 
Palimpsest", a copy of various works of Archimedes which dates back to the 
10th century A.D..   A "palimpsest" is a parchment which was reused and
written over - in this case, by Greek monks.   The Archimedes palimpsest
has a long and complicated history, and only in 1998 was it made publicly
accessible at the Walters Art Gallery.  For more on this, see:

1) Reviel Netz, The origins of mathematical physics: new light on
an old question, Physics Today, June 2000, 32-37.

2) The Walters Art Gallery, Archimedes Palimpsest website,
<A HREF = "http://www.thewalters.org/archimedes/frame.html">http://www.thewalters.org/archimedes/frame.html</A>

For more on Archimedes, try:

3) Chris Rorres, Archimedes website,
<A HREF = "http://www.mcs.drexel.edu/~crorres/Archimedes/contents.html">http://www.mcs.drexel.edu/~crorres/Archimedes/contents.html</A>

Anyway, back to counting.  These days I'm interested in generalizations of 
"cardinality".  The cardinality of a set S is just its number of elements, 
which I'll denote by |S|.  The great thing about this is that if you know 
the cardinality of a set, you know that set up to isomorphism: any two 
sets with the same number of elements are isomorphic.   Of course, this 
is no coincidence: it's exactly what numbers were invented for!  

I explained this using the "parable of the shepherd" in "<A HREF = "week121.html">week121</A>", so I
won't run through that spiel again.  Instead, I'll just remind you 
of the basic facts: there's a category FinSet whose objects are
finite sets and whose morphisms are functions.  We can "decategorify" 
any category by forming the set of isomorphism classes of objects.  
When we do this to FinSet we get the set of natural numbers, N.  So 
given any finite set S, its isomorphism class |S| is just a natural 
number - its cardinality!

Via this trick the natural numbers inherit all their basic operations
from corresponding operations in FinSet.   For example, given two 
finite sets S and T we can form their disjoint union S + T and their
Cartesian product, and these operations give birth to addition and 
multiplication of natural numbers, via these formulas:

|S + T| = |S| + |T|
|S x T| = |S| x |T|

Now the advantage of this rather esoteric view of basic arithmetic is
that it suggests vast generalizations which unify all sorts of seemingly
disparate stuff.  For example, we can play this
"decategorification" game to categories other than FinSet.
For example, we can do it to the category Vect whose objects are vector
spaces and whose morphisms are linear functions - and what do we get?
The set N again!  But this time we don't call the isomorphism class of a
vector space its "cardinality" - we call it the
"dimension".  And this time, addition and multiplication of
natural numbers correspond to direct sum and tensor product of vector
spaces.

Well, this example is so familiar that it may seem that we're still not
getting anywhere interesting.  But suppose we consider the category of
Vect(X) of vector \emph{bundles} over a topological space X.  If we take X
to be a single point this is just Vect - a vector bundle over a point is
a vector space.  But if we take X to be more interesting, when we 
decategorify Vect(X) we get an interesting set that depends on X.  Since 
we can take direct sums and tensor products of vector bundles, this 
set has addition and multiplication operations.  Like the natural numbers,
this set is not a ring, since it doesn't have additive inverses.  It's a 
mere "rig" - a "ring without negatives".

But just as we created the integers by making up additive inverses for
the natural numbers, we can take this set and throw in formal additive
inverses to get a ring.  What ring do we get?  Well, it depends on X:
it's called the "K-theory of X", and denoted K(X).  Studying
this ring K(X) is a wonderful way to understand the space X.  K-theory
is a great example of a generalized cohomology theory (see "<A HREF
= "week149.html">week149</A>" and "<A HREF =
"week150.html">week150</A>").  To explain it in detail would
require a book.  Luckily, such books already exist.  In fact there are a
bunch!  Here are 3 of my favorites:

4) Raoul Bott, Lectures on K(X), Harvard University, Cambridge, 1963.

5) Michael Atiyah, K-theory, W. A. Benjamin, New York, 1967.

6) Max Karoubi, K-theory: an Introduction, Springer, Berlin, 1978.

There are a million variations on this decategorification trick: for 
example, we can decategorify the category of complex line bundles on 
the space X, and get a set called H^{2}(X) - 
the "second cohomology 
group of X".   This is an abelian group thanks to the fact that we 
can take tensor products of line bundles.  The isomorphism class of 
any complex line bundle gives an element of H^{2}(X) 
called the "first 
Chern class" of the line bundle.  For more about this see "<A HREF = "week149.html">week149</A>"....
my point here is that this is just a generalization of the idea of
cardinality!

Or, we can start with the category of finite-dimensional representations
of a group G.  When we decategorify this we get a rig, since we can take
direct sums and tensor products of representations.  If we throw in
additive inverses, we get a ring R(G) called the "representation
ring" of G.  The isomorphism class of any representation gives an
elemnet of R(G) which people call the "character" of that
representation.

Or start with the category where an object is an action of G on a finite 
set!  Decategorifying and then throwing in additive inverses, we get 
something called the "Burnside ring" of G.

In fact, the last two examples are special cases of something more
general: we can start with the category hom(G,C) where the objects 
are actions of G on objects in some category C!  Different choices 
of C give different views of the group G, and different structures
on C will give us a group, or a rig, when we decategorify hom(G,C).
I am tempted to launch into a detailed disquisition on how this works, 
but I fear such generality will exhaust the patience of all but the 
true lovers of abstraction - who can figure it out for themselves anyway!  
So let me descend earthwards a few hundred meters and let the winds 
hasten me towards my ultimate goal, which is... elliptic cohomology.

Suppose we decategorify the category of compact oriented smooth manifolds!
What are the morphisms in this category?  Well, let's take them to be
cobordisms.  And to simplify life let's throw in formal inverses to all
these morphisms, so manifolds with a cobordism between them get counted
as isomorphic.  We get a category where all the morphisms are isomorphisms.
And when decategorify this, we get a big set.  This set becomes a rig thanks
to our ability to take disjoint unions and Cartesian products of compact
oriented smooth manifolds.   In fact it's a ring, because the orientation-
reversed version of any manifold serves as its additive inverse.  This
ring is obviously commutative.  People call it the "oriented cobordism ring".  
And believe or not, people know quite a bit about this ring.

To simplify this ring a bit, let's tensor it with the complex numbers.
We get an algebra that's easy to describe: it's just the algebra of
complex polynomials in countably many variables!  These variables correspond 
to the complex projective spaces CP^{2}, CP^{4}, 
CP^{6}, etcetera - so folks
sometimes write this algebra as follows:

C[CP^{2},CP^{4},CP^{6},...]

Now, using this algebra we can cook up various notions analogous to the
"cardinality" of a compact oriented smooth manifold.  But
people don't say "cardinality", they say "genus".
Don't be fooled - if you know about the genus of a surface, this isn't
that!  In this definition, a "genus" assigns to each compact
oriented manifold M a complex number |M| such that

|M + N| = |M| + |N|

|M x N| = |M| x |N| 

and |M| = |M'| if there is a cobordism from M to M'.  If you stare
at this definition carefully, you'll see that a genus is really just
a homomorphism from C[CP^{2},CP^{4},CP^{6},...] 
to the complex numbers.  

As any classicist will tell you, the plural of genus is "genera".  
Examples of genera include the signature and A-hat genus, both 
beloved by topologists and differential geometers.   The Euler 
characteristic is \emph{not} a genus since it is not cobordism invariant - 
very much a pity, since it's so much like the cardinality in so 
many ways (see "<A HREF = "week146.html">week146</A>".)  

Since the algebra C[CP^{2},CP^{4},CP^{6},...] 
is generated by the guys CP^{2n},
all the information to describe a genus is
contained in the "logarithm"

LOG(x) = sum |CP^{2n}| x^{2n+1} / (2n+1)

Classifying genera is hard, but it gets easier if we impose some
extra conditions.  Suppose 

F ---> E ---> B

is a fiber bundle with compact connected structure group.  The space
E is like a "twisted product" of F and B, so it makes sense to demand
that 

|E| = |F| |B|.

In this case we say we have an "elliptic genus".  And in this case 
Ochanine proved that in this case the logarithm is an elliptic integral:

LOG(x) = integral_{0}^{x}  dt / sqrt(1 - 2dt^{2} + 
et^{4}) 

for some numbers d and e.  This is the inverse of an elliptic function,
and this elliptic function is periodic with respect to some lattice L in 
the complex plane.  

(You don't remember what elliptic functions are, and what they have to
do with lattices?  Then go back to "<A HREF =
"week13.html">week13</A>".)

We can think of the elliptic genus as a function of the lattice L.   
If we do this, something nice happens: if we rescale (d,e) to (c^{2d},c^{4e}), this changes the 
lattice L to L/c and changes the genus |M| to c^{dim(M)/2} |M|.  Folks 
summarize this and some other stuff by saying that the elliptic genus |M|, 
thought of as a function of the lattice L, is a "modular form of 
weight dim(M)/2".

Now for the final punchline: if we think of our elliptic genus as taking
values in a ring where d and e are formal variables, the resulting
"universal elliptic genus" has a nice interpretation in terms
of elliptic cohomology - a generalized cohomology theory that I
discussed in "<A HREF = "week151.html">week151</A>".  To
compute the universal elliptic genus |M|, we just take the fundamental
class of M (in elliptic cohomology) and push it forwards via the map
from M to a point!

(We can do this "pushforward" because elliptic cohomology is a
complex oriented cobordism theory and acts very much like ordinary
cohomology or K-theory.)

It's very interesting how elliptic functions, modular forms and the like
appear out of the blue in what I've just been talking about.  Why???
The explanation seems to involve loop groups, vertex operator algebras and
that sort of stuff... but alas, I don't have time to even \emph{try} 
to explain this now!  For now, I just urge you to read these:

7) Graeme Segal, Elliptic cohomology, Asterisque 161-162 (1988), 187-201.  

8) Hirotaka Tamanoi, Elliptic Genera and Vertex Operator Super-Algebras,
Springer Lecture Notes in Mathematics 1704, Springer, Berlin, 1999.





<p> <hr>
<em>It is like walking through a constantly shifting illusion, 
routes appearing and decaying, the solvable and the utterly impossible
snuggled so close together that they cannot be told apart.</em> - 
Craig Childs, Soul of Nowhere
<HR>

% </A>
% </A>
% </A>


% parser failed at source line 396
