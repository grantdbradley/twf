
% </A>
% </A>
% </A>
\week{September 18, 2005 }

After going to the Streetfest this summer, I wandered around China.  
I began by going to a big conference in Beijing, the 22nd 
International Congress on the History of Science.  I learned some 
interesting stuff.  For example:  

You may have heard of <a href = 
"http://en.wikipedia.org/wiki/Andalusia">Andalusia,</a> 
that fascinating melting-pot
of cultures that formed when southern Spain was invaded by Muslims.
The eleventh century was the golden age of Andalusian astronomy 
and mathematics, with a lot of innovation in astrolabes.  During 
the <A HREF = "http://en.wikipedia.org/wiki/Timeline_of_the_Muslim_Occupation_of_the_Iberian_Peninsula">Caliphate</A> (929-1031), three quarters of all mathematical 
manuscripts were produced in 
<A HREF =" http://lexicorient.com/spain/photos/cordoba02.jpg">
Cordoba</A>, most of the rest in 
<A HREF =" http://images.encarta.msn.com/xrefmedia/sharemed/targets/images/pho/t054/T054900A.jpg">Sevilla</A>, and 
only a few in 
<A HREF =" http://lexicorient.com/spain/photos/granada02.jpg">
Granada</A> and 
<A HREF =" http://hopf.chem.brandeis.edu/Milos/PhotoMD/Images/Spain2001/Toledo_Alcazar.JPG">
Toledo</A>.   

I didn't understand the mathematical predominance of Cordoba when 
I first heard about it, but the underlying reason is simple.  
The first great Muslim dynasty were the 
<A HREF =" http://en.wikipedia.org/wiki/Umayyads">Ummayyads</A>,
who ruled from Damascus.   They were massacred by the
<A HREF =" http://en.wikipedia.org/wiki/Abbasid">Abbasids</A> 
in 750, who then moved the capital to Baghdad.
When <A HREF =" http://en.wikipedia.org/wiki/Abd-ar-Rahman_I">Abd 
ar-Rahman</A> fled Damascus in 750 as the only Ummayyad
survivor of this massacre, he went to Spain, which 
had already been <A HREF = "http://en.wikipedia.org/wiki/Muslim_conquest_of_Iberia">invaded by Muslim Berbers</A> in 711.

Abd ar-Rahman made Cordoba his capital.  And, by enforcing a certain 
level of religious tolerance, he made this city into <em>the place to 
be</em> for Muslims, Jews and Christians - the "ornament of the 
world", 
and a beacon of learning - until it was sacked by Berber troops in 
1009.  

Other cities in Andalusia became important later.  The great 
philosopher <A HREF = "http://en.wikipedia.org/wiki/Averroes">Ibn 
Rushd</A> - known to Westerners by the Latin name 
"Averroes" - was born in Cordoba in 1128.  He later became a judge 
there.  He studied mathematics, medicine, and astronomy, and wrote 
detailed line-by-line commentaries on the works of Aristotle.  It 
was through these commentaries that most of Aristotle's works, 
including his Physics, found their way into Western Europe!  By 1177, 
the bishop of Paris had banned the teaching of many of these new
ideas - but to little effect. 

Toledo seems to have only gained real prominence after 
<A HREF = "http://en.wikipedia.org/wiki/Alfonso_VI_of_Castile">Alfonso VI</A> 
made it his capital upon capturing it in 1085 as part of the 
Christian "reconquista".  By the 1200s, it became a lively center 
for translating Arabic and Hebrew texts into Latin.   
  
Mathematics also passed from the Arabs to Western Europe in other
ways.  <A HREF = "http://www-groups.dcs.st-and.ac.uk/~history/Mathematicians/Fibonacci.html">Fibonacci</A> (1170-1250) studied Arabic accounting methods in 
North Africa where his father was a diplomat.  His book \emph{Liber Abaci}
was important in transmitting the Indian system of numerals 
(including zero) from the Arabs to Europe.  However, he wasn't the 
first to bring these numbers to Europe.  They'd been around for over 
200 years!

For example: 
<A HREF =" http://en.wikipedia.org/wiki/Gerbert_of_Aurillac">Gerbert 
d'Aurillac</A> (940-1003) spent years studying
mathematics in various Andalusian cities including Cordoba.  On
his return to France, he wrote a book about a cumbersome sort of 
"abacus" labelled by a Western form of the Indian numerals -
close to what we now call "Arabic numerals".  This 
remained popular in intellectual circles until the mid-12th century.

Amusingly, Arabic numerals were also called "dust numerals" since 
they were used in calculations on an easily erasable "dust board".
Their use was described in the \emph{Liber Pulveris}, 
or "book of dust".

I want to learn more about Andalusian science!  I found this book 
a great place to start - it's really fascinating:

1) Maria Rose Menocal, The Ornament of the World: How Muslims, Jews
and Christians Created a Culture of Tolerance in Medieval Spain,
Little, Brown and Co., 2002.

For something quick and pretty, try this:

2) Steve Edwards, Tilings from the Alhambra, 
<A HREF = "http://www2.spsu.edu/math/tile/grammar/moor.htm">
http://www2.spsu.edu/math/tile/grammar/moor.htm</A>

Apparently 13 of the 17 planar symmetry groups can be found in tile
patterns in the Alhambra, a Moorish palace built in Granada in the
1300s.

To dig deeper into the splendors of Arabic mathematics, try these:

3) John J. O'Connor and Edmund F. Robertson, Arabic mathematics: 
forgotten brilliance?, 
<A HREF = "http://www-groups.dcs.st-and.ac.uk/~history/HistTopics/Arabic_mathematics.html">
http://www-groups.dcs.st-and.ac.uk/~history/HistTopics/Arabic_mathematics.html
% </A>

John J. O'Connor and Edmund F. Robertson, 
Biographies of Arab/Islamic mathematicians,
<A HREF = "http://www-groups.dcs.st-and.ac.uk/~history/Indexes/Arabs.html">
http://www-groups.dcs.st-and.ac.uk/~history/Indexes/Arabs.html</A>
% </A>

For more on Fibonacci and Arabic mathematics, try this paper by
Charles Burnett, who spoke about the history of
"Arabic numerals" in Beijing:

4) Charles Burnett, Leonard of Pisa and Arabic Arithmetic,
<A HREF = "http://muslimheritage.com/topics/default.cfm?ArticleID=472">
http://muslimheritage.com/topics/default.cfm?ArticleID=472</A>

Another interesting talk in Beijing was about the role of the 
Syriac language in the transmission of Greek science to Europe.
Many important texts didn't get translated directly from Greek to 
Arabic!  Instead, they were first translated into \emph{Syriac}.

I don't understand the details yet, but luckily there's a great
book on the subject, available free online:

5) De Lacy O'Leary, How Greek Science Passed to the Arabs,
Routledge \text{\&}  Kegan Paul Ltd, 1949.  Also available at
<A HREF = "http://www.aina.org/books/hgsptta.htm">
http://www.aina.org/books/hgsptta.htm</A>

So, medieval Europe learned a lot of Greek science by reading Latin 
translations of Arab translations of Syriac translations of 
second-hand copies of the original Greek texts!

George Baloglu recommends this book:

6) Dimitri Gutas, Greek Thought, Arabic Culture: The Graeco-Arabic 
Translation Movement in Baghdad and Early 'Abbasid Society 
(2nd-4th/8th-10th Centuries), Routledge, 1998.

I want to read this book, too:

7) Scott L. Montgomery, Science in Translation: Movements of 
Knowledge through Cultures and Time, U. of Chicago Press, 2000.
Review by William R. Everdell available at MAA Online,
<A HREF = "http://www.maa.org/publications/maa-reviews/science-in-translation-movements-of-knowledge-through-cultures-and-time">
http://www.maa.org/publications/maa-reviews/science-in-translation-movements-of-knowledge-through-cultures-and-time</A>

The historian of science John Stachel, famous for his studies of
Einstein, says this book "strikes a blow at one of the founding
myths of 'Western Civilization'" - namely, that Renaissance Europeans
single-handedly picked up doing science where the Greeks left off.  
As Everdell writes in his review:

\begin{quote}
    Perhaps the best of the book's many delightful challenges 
    to conventional wisdom comes in the first section on the 
    translations of Greek science.  Here we learn why it is 
    ridiculous to use a phrase like &quot;the Renaissance recovery 
    of the Greek classics&quot;; that in fact the Renaissance recovered 
    very little from the original Greek and that it was long before 
    the Renaissance that Aristotle and Ptolemy, to name the two most 
    important examples, were finally translated into Latin.  What 
    the Renaissance did was to create a myth by eliminating all the 
    intermediate steps in the transmission.  To assume that Greek 
    was translated into Arabic &quot;still essentially erases centuries 
    of history&quot; (p. 93).  What was translated into Arabic was 
    usually Syriac, and the translators were neither Arabs (as 
    the great Muslim historian <A HREF = "http://en.wikipedia.org/wiki/Ibn_Khaldun">Ibn Khaldun</A> admitted) nor Muslims. 
    The real story involves Sanskrit compilers of ancient Babylonian 
    astronomy, Nestorian Christian Syriac-speaking scholars of 
    Greek in the Persian city of <A HREF = "http://en.wikipedia.org/wiki/Jundishapur">Jundishapur</A>, and Arabic- and 
    Pahlavi-speaking Muslim scholars of Syriac, including the 
    Nestorian Hunayn <A HREF = "http://www.nestorian.org/hunein_ibn_ishak.html">Ibn Ishak</A> (809-873) of Baghdad, &quot;the greatest 
    of all translators during this era&quot; (p. 98).
\end{quote}
    

And now for something completely different: the Langlands program!
I want to keep going on my gradual quest to understand and explain 
this profoundly difficult hunk of mathematics, which connects 
number theory to representations of algebraic groups.  I've found 
this introduction to be really helpful:

8) Stephen Gelbart: An elementary introduction to the Langlands 
program, Bulletin of the AMS 10 (1984), 177-219.

There are a lot of more detailed sources of information on the
Langlands program, but the problem for the beginner (me) is that
the overall goal gets swamped in a mass of technicalities.  
Gelbart's introduction does the best at avoiding this problem.

I've also found parts of this article to be helpful:

9) Edward Frenkel, Recent advances in the Langlands program, available
at <A HREF = "http://arxiv.org/abs/math.AG/0303074">math.AG/0303074</A>.

It focuses on the "geometric Langlands program", which I'd rather 
not talk about now.  But, it starts with a pretty clear introduction 
to the basic Langlands stuff... at least, clear to me after I've 
battered my head on this for about a year!  

If you know some number theory or you followed recent issues
This Week's Finds (especially "<A HREF = "week217.html">week217</A>"
and "<A HREF = "week218.html">week218</A>") it should 
make sense, so I'll quote it:

\begin{quote}
  The Langlands Program has emerged in the late 60's in the form of 
  a series of far-reaching conjectures tying together seemingly 
  unrelated objects in number theory, algebraic geometry, and the 
  theory of automorphic forms.  To motivate it, recall the classical 
  Kronecker-Weber theorem which describes the maximal abelian extension 
  Q<sup>ab</sup> of the field Q of rational numbers (i.e., the maximal extension 
  of Q whose Galois group is abelian).  This theorem states that Q<sup>ab</sup>
  is obtained by adjoining to Q all roots of unity; in other words, 
  Q<sup>ab</sup> is the union of all cyclotomic fields Q(1<sup>1/N</sup>) 
  obtained by adjoining to Q a primitive Nth root of unity 

  1<sup>1/N</sup>

  The Galois group Gal(Q(1<sup>1/N</sup>)/Q) of automorphisms of 
  Q(1<sup>1/N</sup>) 
  preserving Q is isomorphic to the group (Z/N)* of units of the 
  ring Z/N.  Indeed, each element m in (Z/N)*, viewed as an integer 
  relatively prime to N, gives rise to an automorphism of Q(1<sup>1/N</sup>) 
  which sends

  1<sup>1/N</sup>

  to 

  1<sup>m/N</sup>.

  Therefore we obtain that the Galois group Gal(Q<sup>ab</sup>/Q), or, 
  equivalently, the maximal abelian quotient of Gal(Q<sup>-</sup>/Q), 
  where Q<sup>-</sup> is an algebraic closure of Q, is isomorphic to the 
  projective limit of the groups (Z/N)* with respect to the system 
  of surjections

  (Z/N)* &rarr; (Z/M)* 

  for M dividing N. This projective limit is nothing but the direct 
  product of the multiplicative groups of the rings of p-adic 
  integers, Z<sub>p</sub>*, where p runs over the set of all primes.  Thus, 
  we obtain that

  Gal(Q<sup>ab</sup>/Q) = &prod; Z<sub>p</sub>*.

  The abelian class field theory gives a similar description for the
  maximal abelian quotient Gal(F<sup>ab</sup>/F) of the Galois group 
  Gal(F<sup>-</sup>/F), 
  where F is an arbitrary global field, i.e., a finite extension of 
  Q (number field), or the field of rational functions on a smooth 
  projective curve defined over a finite field (function field).  
  Namely, Gal(F<sup>ab</sup>/F) is almost isomorphic to the quotient A(F)*/F*, 
  where A(F) is the ring of adeles of F, a subring in the direct 
  product of all completions of F.  Here we use the word &quot;almost&quot; 
  because we need to take the group of components of this quotient 
  if F is a number field, or its profinite completion if F is a 
  function field.

  When F = Q the ring A(Q) is a subring of the direct product of the 
  fields Q<sub>p</sub> of p-adic numbers and the field R of real numbers, and 
  the quotient A(F)*/F* is isomorphic to 

  R<sup>+</sup> &times; &prod;<sub>p</sub> Z<sub>p</sub>*. 

  where R<sup>+</sup> is the multiplicative group of positive real numbers.
  Hence the group of its components is

  &prod;<sub>p</sub> Z<sub>p</sub>*

  in agreement with the Kronecker-Weber theorem.

  One can obtain complete information about the maximal abelian 
  quotient of a group by considering its one-dimensional 
  representations.  The above statement of the abelian class field 
  theory may then be reformulated as saying that one-dimensional 
  representations of Gal(F<sup>-</sup>/F) are essentially in bijection with 
  one-dimensional representations of the abelian group 

  A(F)* = GL(1,A(F)) 

  which occur in the space of functions on

  A(F)*/F* = GL(1,A(F))/GL(1,F)

  A marvelous insight of Robert Langlands was to conjecture that 
  there exists a similar description of <em>n-dimensional 
  representations</em> of Gal(F<sup>-</sup>/F). Namely, he proposed that those 
  may be related to irreducible representations of the group 
  GL(n,A(F)) which are <em>automorphic</em>, that is those occurring in 
  the space of functions on the quotient 

  GL(n,A(F))/GL(n,F)

  This relation is now called the <em>Langlands correspondence</em>.

  At this point one might ask a legitimate question: why is it 
  important to know what the n-dimensional representations of the 
  Galois group look like, and why is it useful to relate them to 
  things like automorphic representations?  There are indeed many 
  reasons for that.  First of all, it should be remarked that 
  according to the Tannakian philosophy, one can reconstruct a 
  group from the category of its finite-dimensional representations, 
  equipped with the structure of the tensor product.  Therefore 
  looking at n-dimensional representations of the Galois group is 
  a natural step towards understanding its structure.  But even 
  more importantly, one finds many interesting representations of 
  Galois groups in &quot;nature&quot;.  

  For example, the group Gal(Q<sup>-</sup>/Q) will act on the geometric
  invariants (such as the etale cohomologies) of an algebraic variety
  defined over Q.  Thus, if we take an elliptic curve E over Q,
  then we will obtain a two-dimensional Galois representation on its
  first etale cohomology.  This representation contains a lot of
  important information about the curve E, such as the number of
  points of E over Z/p for various primes p.

  The point is that the Langlands correspondence is supposed to 
  relate n-dimensional Galois representations to automorphic 
  representations of GL(n,A(F)) in such a way that the data on 
  the Galois side, such as the number of points of E over Z/p, 
  are translated into something more tractable on the automorphic 
  side, such as the coefficients in the q-expansion of the modular 
  forms that encapsulate automorphic representations of GL(2,A(Q)).

  More precisely, one asks that under the Langlands correspondence
  certain natural invariants attached to the Galois representations 
  and to the automorphic representations be matched. These 
  invariants are the <em>Frobenius conjugacy classes</em> on the Galois 
  side and the <em>Hecke eigenvalues</em> on the automorphic side. 
\end{quote}
    

Since I haven't talked about Hecke operators yet, I'll stop here!

But, someday I should really explain the ideas behind the baby 
"abelian" case of the Langlands philosophy in simpler terms than 
Frenkel does here.  The abelian case goes back way before Langlands:
it's called "class field theory".  And, it's all about exploiting 
this analogy, which I last mentioned in "<A HREF = "week218.html">week218</A>":

\begin{verbatim}
 NUMBER THEORY                 COMPLEX GEOMETRY                                 

 Integers                      Polynomial functions on the complex plane
 Rational numbers              Rational functions on the complex plane
 Prime numbers                 Points in the complex plane            
 Integers mod p^n              (n-1)st-order Taylor series
 p-adic integers               Taylor series
 p-adic numbers                Laurent series
 Adeles for the rationals      Adeles for the rational functions
 Fields                        One-point spaces
 Homomorphisms to fields       Maps from one-point spaces
 Algebraic number fields       Branched covering spaces of the complex plane
                                       
\end{verbatim}
    

\par\noindent\rule{\textwidth}{0.4pt}
\textbf{Addendum:} 
I thank Fabien Besnard for some suggestions on how to improve
this Week's Finds.  
Bruce Smith, Noam Elkies, and 
Miguel Carri&oacute;n-&Aacute;lvarez 
had some things to say about 
the history of science.  In response to this comment of mine:

\begin{quote}
So, medieval Europe learned a lot of Greek science by reading Latin 
translations of Arab translations of Syriac translations of 
second-hand copies of the original Greek texts!
\end{quote}
    

my friend Bruce wrote:

\begin{quote}
This all seems so precarious a process that it makes me wonder whether
there was ten times as much valuable ancient math and philosophy as we
know about, most of which got <em>completely</em> lost.
\end{quote}
    

Something like this almost certainly true.  

Like Plato, Aristotle is believed to have written dialogs which presented 
his ideas in a polished form.  They were all lost.  His extant writings 
are just "lecture notesquot; for courses he taught!

Euripides wrote at least 75 plays, of which only 19 survive in their
full form.  We have fragments or excerpts of some more.  This isn't
philosophy or math, but it's still incredibly tragic (pardon the pun).

The mathematician Apollonius wrote a book on \emph{Tangencies} which is lost.
Only four of his eight books on \emph{Conics} survive in Greek.  
Luckily, the first seven survive in Arabic.  

The burning of the library of Alexandria is partially to blame for
these losses.

There's some good news, though:

Archimedes did more work on calculus than previously believed!
We know this now because a manuscript of his on mechanics that had been 
erased and written over has recently been read with the help of a 
synchrotron X-ray beam!  This is a great example of modern science
helping the history of science.  

This manuscript, called the Archimedes Palimpsest, also reveals for 
the first time that he did work on combinatorics:

10) Nova, The Archimedes Palimpsest, 
<A HREF = "http://www.pbs.org/wgbh/nova/archimedes/palimpsest.html">
http://www.pbs.org/wgbh/nova/archimedes/palimpsest.html</A>

11) Heather Rock Woods, Placed under X-ray gaze, Archimedes 
manuscript yields secrets lost to time, Stanford Report, May 19, 2005, 
<A HREF = "http://news-service.stanford.edu/news/2005/may25/archimedes-052505.html">http://news-service.stanford.edu/news/2005/may25/archimedes-052505.html</A>

12) Erica Klarreich, Glimpses of genius: mathematicians and
historians piece together a puzzle that Archimedes pondered,
Science News 165 (2004), 314.  Also available at
<A HREF = "http://www.sciencenews.org/articles/20040515/bob9.asp">
http://www.sciencenews.org/articles/20040515/bob9.asp</A>

Also: a team using "multispectral imaging" has recently been able 
to read parts of a Roman library in the town of 
<A HREF = "http://en.wikipedia.org/wiki/Herculaneum">Herculaneum</A>.  
The books in this library were "roasted in place" -
heavily carbonized - during the eruption of Vesuvius that destroyed 
<A HREF = "http://en.wikipedia.org/wiki/Pompeii">Pompeii</A>. 
By distinguishing between different shades of 
black, researchers were able to reconstruct the entire book <em>On 
Piety</em> by one Philodemus:

13) Julie Walker, A library of mud and ashes, BYU Magazine, Spring
2001, <A HREF = "http://magazine.byu.edu/?act=view&a=43">http://magazine.byu.edu/?act=view&a=43</A>

I can't resist quoting a bit:

\begin{quote}
A sister city to Pompeii that was also buried in the volcanic eruption 
of A.D. 79, Herculaneum was a seaside town that sat between Vesuvius' 
fertile foot and the gleaming Bay of Naples. The collection of 2,000 
carbonized Greek and Latin scrolls, primarily Epicurean philosophical 
writings, was found in a luxurious Herculaneum house known as the Villa 
of the Papyri, which was discovered in 1752.

The scrolls have endured a destructive path through history: first, rain 
soaked the papyri, then a 570-degree swell of molasses-thick mud engulfed 
the villa and charred the scrolls. They would remain buried under 65 feet 
of mud for hundreds of years.

As a result, many of the fragile scroll cylinders are pressed into 
trapezoidal columns; some are bowed and snaked into half-moons, others 
folded into v-shapes.

After their discovery the mortality rate for the scrolls continued to climb 
as would-be conservators struggled to find a way to unroll the fragile 
manuscripts. Some scrolls were turned to mush when they were painted with 
mercury; many were sliced down the middle and cut into fragments. Early 
transcribers would copy the visible outer layer of a scroll, then scrape it 
off and discard it to read the next layer.

Even today, scholars use metaphors of near impossibility to describe the 
scroll unrolling process. It is like &quot;flattening out a potato chip&quot; 
without destroying it, or like &quot;separating (burned) layers of two-ply 
tissue,&quot; says Jeffrey Fish of Baylor University.

The current unrolling methoddeveloped by a team of Norwegian conservators
involves applying a gelatin-based adhesive to the scroll's outer surface. 
As the adhesive dries, the outer shell - which bears the text on its interior -
can be slowly peeled off. It can take days to remove a single fragment, 
months or years to process a complete scroll. Some 300 of the library's 
scrolls have yet to be unrolled, and many more scrolls are in various 
stages of conservation and repair.

On the Herculaneum project, CPART researchers Steve and Susan Booras 
conducted multispectral imaging (MSI) on 3,100 trays of papyrus fragments 
and photographed them with a high-quality digital camera. The images will 
be used to create a digital library that can be accessed by scholars 
worldwide. Developed for NASA scientists, the imaging technique has only 
recently been applied to the study of ancient texts. Rather than focusing 
on light that is seen at wave lengths visible to the eye, MSI uses 
filters to focus on nonvisible portions of the light spectrum. In the 
nonvisible infrared spectrum, the black ink on a blackened scroll can be 
clearly differentiated. In some cases clear, legible writings have been 
found on fragments that researchers believed were completely blank.
\end{quote}
    

The same team is now studying over 400,000 fragments of papyrus found 
in an ancient garbage dump in the old Egyptian town of 
<A HREF = "http://en.wikipedia.org/wiki/Oxyrhynchus">Oxyrhynchus</A>.  They've
pieced together new fragments of plays by Euripides, Sophocles and Menander,
lost lines from the poets Sappho, Hesiod, and Archilocus, and most of
a book by Hesiod:

14) Oxyrhynchus Online, multispectral imaging,
<A HREF = "http://www.papyrology.ox.ac.uk/multi/procedure.html">
http://www.papyrology.ox.ac.uk/multi/procedure.html</A>

If you just want to look at a nice "before and after" 
movie of what multispectral imaging can do, try this link.

Finally, in response to this remark of mine:

\begin{quote}
Amusingly, Arabic numerals were also called "dust numerals" since 
they were used in calculations on an easily erasable "dust board".
Their use was described in the Liber Pulveris, or "book of dust".
\end{quote}
    

Noam Elkies wrote:

\begin{quote}
This is even more amusing than you may realize: the word "abacus"
comes from a Greek word "abax, abak-" for "counting board", which
conjecturally might come from the Hebrew word (or a cognate word
in another semitic language) for "dust"!  See for instance:

<A HREF = "http://education.yahoo.com/reference/dictionary/entry/abacus">
http://education.yahoo.com/reference/dictionary/entry/abacus</A>

So these &quot;dust numerals&quot; replaced a reckoning device whose name
may also originate with calculation a dust board...
\end{quote}
    

Interesting!  While "calculus" refers back to pebbles.  

My erstwhile student
Miguel Carri&oacute;n-&Aacute;lvarez 
clarified the issue somewhat:

\begin{quote}
The first abaci were drawn in the sand with sticks. The next
step was to carve grooves in a board (wooden, or clay: think
cuneiform tablets) and place beads in them. Pierced beads
moving on beams (wood, later metal) must have been a pretty
recent development, relatively speaking.

Remember that Archimedes was studying geometry by drawing
figures in the sand when he was slain. If a sand abacus is the
precursor of the modern calculator, Archimedes' sandbox is the
precursor of GUI geometry software. 

One of Archimedes' most fanciful works is &quot;The Sand Reckoner&quot;.
Here the reckoner can be understood to be himself, as he is
counting the grains of sand which fit inside the sphere of
fixed stars, but it can also refer to a sand abacus (reckoner
= calculator). In fact, romance translations of this title
that I've seen (French: L'arenaire, Spanish: El arenario, etc.)
unambiguously refer to an object, not a person. It is easy to
imagine Archimedes inventing his positional number system on a
sand abacus, and using the counting of grains of sand as an
excuse to write about it.

\end{quote}
    


\par\noindent\rule{\textwidth}{0.4pt}
<em>We avail ourselves of what our predecessors may have said.  
That they were or were not our coreligionists is of no account....
Whatever accords with the truth, we shall happily and gratefully
accept, and whatever conflicts, we shall scrupulously but generously
point out.</em> - Averroes 

<HR>

% </A>
% </A>
% </A>


% parser failed at source line 712
