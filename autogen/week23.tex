
% </A>
% </A>
% </A>
\week{October 24, 1993}

I will soon revert to my older style, in which I list piles of new
papers as they accumulate on my desk.  This time, though, I want to
describe Frank Quinn's work on the Andrews-Curtis conjecture using
topological quantum field theories (TQFTs), as promised.  Then, if
you'll pardon me, I'll list the contents of a book I've just finished
editing.  It is such a relief to be done that I cannot resist.

So -

1) Topological quantum invariants and the Andrews-Curtis conjecture
(Progress report), by Frank Quinn, preprint, Sept. 1993.

2) Lectures on axiomatic topological quantum field theory, by Frank
Quinn, to appear in the proceedings of the Park City Geometry Institute.

3) On the Andrews-Curtis conjecture and related problems, by Wolfgang
Metzler, in Combinatorial Methods in Topology and Algebraic Geometry,
Contemporary Mathematics 44, AMS, 1985. 

Last week I described - in a pretty sketchy way - how the 4-color
theorem and the Beraha conjecture are related to TQFTs.  These can be
regarded as two very hard problems in 2-dimensional topology - one
solved by a mixture of cleverness and extreme brute force, the other
still open.  There is another hard problem in 2-dimensional topology
called the Andrews-Curtis conjecture, which Quinn is working on using TQFT
methods, which I'll talk about this time.  I don't know too much about
this stuff, so I hope any experts out there will correct my inevitable
mistakes.  

Actually, this conjecture is easiest to describe in a purely algebraic
way, so I'll start there.  Hopefully most of you know the concept of a
"presentation" of a group in terms of generators and relations.  For
example, the group Z_n (integers mod n) has the presentation <x|x^n>.
This means, roughly, that we form all products of the "generator" x and
its inverse, and then mod out by the "relation" x^n = 1.  A bit more
interesting is the dihedral group D_n of symmetries of a regular n-gon,
counting rotations and reflections, with presentation <x, y|x^n, y^2,
(xy)^2>.  Here x corresponds to a clockwise rotation by 1/n-th of a
turn, and y corresponds to a reflection.  

A group always has lots of different presentations, so a natural 
problem is to decide whether two different presentations give the
same group (or, strictly speaking, isomorphic groups).  It'd be nice to
have an algorithm for deciding this question.  But it's a famous result
of mathematical logic that there is no such algorithm!  

If two presentations give the same group, one can get from one to the
other by a sequence of the following easy steps, called Tietze moves:


\begin{verbatim}

1)   Throw in an extra new generator x together with the extra new
relation xg^{-1} where g is a product of the previous generators and
their inverses.
2)  The inverse of 1) - remove a generator x together with the relation
xg^{-1}, if possible (the relation xg^{-1} needs to be there!).
3)  Throw in a new relation that's a consequence of existing relations.
4)  The inverse of 2) - remove a relation that's a consequence of other
relations.  
\end{verbatim}
    

So if one has two presentations and wants to see if they give same
group, you could always set up a program that blindly tries using these
Tietze moves in all possible ways to transform one presentation into the
other.  If they are the same it'll eventually catch on!  But if they're
not it'll chug on forever.  There's no algorithmic way to tell \emph{when} it
should give up and admit the two presentations give different groups! -
which is why we say there is no "decision procedure" for this problem.  

In one form, the Andrews-Curtis conjecture goes as follows.  Remember
that the trivial group is the group with just the identity element; it
has a presentation <x|x>.  Suppose we have some other "balanced" presentation
of the trivial group, that is a presentation with just as many
generators as relations: <x_1,...,x_n|r_1,...,r_n>.  Then the conjecture
is that it can be reduced to the presentation <x|x> by a sequence of the
following moves that keep the presentation balanced:


\begin{verbatim}

1)  Throw in an extra new generator x together with the extra new
relation x.
2)  The inverse of 1).
3)  Permute the relations
4)  Change r_1 to r_1^{-1}
5)  Change r_1 to r_1r_2
6)  Change r_1 to gr_1g^{-1} for any g.
\end{verbatim}
    

The experts seem to think this conjecture is probably false - but
nobody has disproved it.   Metzler lists a few presentations of the
trivial group that might be counterexamples: nobody has ever found a way
to use moves 1)-6) to boil them down to the presentation <x|x>.  For example,


$$

<a, b|b^5a^{-4}, aba(bab)^{-1}>.
$$
    

Try it!

The Andrews-Curtis conjecture is interesting mainly for its implications
in topology.  When they first stated their conjecture they noted a
number of topological consequences, and the referee of the paper noted
one more.  For example, it would shed some light on the Poincare
conjecture (although not settle it) as follows.  Recall that the
Poincare conjecture says every 3-dimensional manifold homotopic to a
3-sphere is homeomorphic to a 3-sphere.  The Andrews-Curtis conjecture
implies that if the Poincare conjecture is false, any counterexample can
in fact be embedded (topologically) in R^4!

It was the referee (does anyone know who that was) who noted that 
the Andrews-Curtis conjecture can be formulated in terms of
"CW complexes."  This is how Quinn thinks about it, so I suppose I
should say what those are.   

A 0-complex is simply a set of points given the discrete topology.  We
call the points "0-cells."   To get a 1-complex, we take a set of
"1-cells," that is, closed unit intervals, and glue their ends on to the
0-cells in any way we want.  In other words, we get a graph, possibly
with some edges having both ends at the same vertex.  To get a
2-complex, we take a set of "2-cells," that is, 2-dimensional closed
disks, and glue their boundaries onto our 1-complex by any continuous
map.  And so on, with the "n-cells" being just copies of the closed unit
ball in R^n. 

CW complexes were invented by J. H. C. Whitehead in 1949 and are a key tool
in algebraic topology.  (The word "CW," by the way, seems to come from
"closure-finite" and "weak" -- as in "weak topology.")  They are a
nice class of topological spaces since on the one hand, being built up
by gluing simple pieces together, one can really understand them, and on
the other hand, they are actually quite general.  In fact, if one is
interested in the usual invariants studied in algebraic topology
(homology and cohomology groups, homotopy groups and the like), CW
complexes are pretty much good enough. More precisely, Whitehead proved
a "CW approximation theorem" saying that any halfway decent topological
space (i.e., any "compactly generated" space) is "weakly homotopy
equivalent" to a CW complex.  I won't burden you with the definitions
here; I learned this stuff once upon a time from 

4) Elements of Homotopy Theory, by George W. Whitehead, Springer-Verlag,
Berlin, 1978.  ISBN 0-387-90336-4

Anyway, the Andrews-Curtis conjecture can be thought of as being about
2-complexes.  In fact, a group presentation can be regarded as
instructions for building up a 2-complex -- start with a point, glue on
1-cells, one for each generator (obtaining a "bouquet of circles") and
then glue on 2-cells, one for each relation, attaching their boundaries
to the 1-cells in the manner presecribed by the relation.  This
2-complex will have fundamental group equal to the group given by the
presentation.  The moves 1)-6) above can be thought of as operations on
these 2-complexes.  So one can translate the Andrews-Curtis conjecture
into a statement about 2-complexes.  And at this point I guess I'm going
to start getting more technical...

One topological statement of the Andrews-Curtis conjecture is that "if
two 2-complexes are simply equivalent then one can be 2-deformed to the
other."  I don't understand this as well as I want, so I won't explain
it; instead, I'll briefly explain the (weaker?) version corresponding
more closely to the algebraic statement above, namely "if X is a
contractible 2-complex, it can be 2-deformed to a point."  Being
"contractible" means that as far as homotopy theory goes X is just like
a point.  (E.g., the unit disk is contractible, while the circle is
not.)  And a "2-deformation" roughly means a sequence of moves
consisting of adding or deleting 1-cells or 2-cells in a way that
doesn't affect things as far as homotopy goes, or doing homotopies of
attaching maps of 2-cells.  The interesting thing about these
formulations of the Andrews-Curtis conjecture is that their analogs for
n > 2 are true and in fact were shown by J H C Whitehead in 1939!

Quinn's goal is to cook up invariants of 2-complexes that might detect
counterexamples to the Andrews-Curtis conjecture, i.e., invariants under
2-deformation.  He wants to do it using 1+1-dimensional TQFTs of a sort
that assign vector spaces to 1-complexes and linear maps to 2-complexes.
Traditionally, TQFTs assign vector spaces to n-manifolds and linear maps to
(n+1)-manifolds.  Quinn calls his TQFTs "modular" because they have a
lot of formal similarities to the kind of TQFTs that come up in string
theory (where the modular group reigns supreme).  He gives a thorough
axiomatic description of modular TQFTs in his lecture notes, and this is
actually the most fascinating aspect for me, more so than the
Andrews-Curtis conjecture per se, since it bears on physics.  

The problem with coming up with an TQFT invariant that can catch
counterexamples to the Andrews-Curtis conjecture is an interesting
"stabilization" property that 2-complexes have.  Namely, if two
2-complexes are simply equivalent, one can can wedge them both with some
large number k of 2-spheres and get complexes which are 2-deformable to
each other.  It turns out that this means we want to find a TQFT such
that Z(S^2)^k = 0.   And so Quinn considers TQFTs based, not on the
complex numbers, but on integers mod p.  

A TQFT of his sort amounts to finding a symmetric tensor category of
vector spaces and an object A in this category with some special
properties corresponding to the fact that it is the vector space
corresponding to the unit interval [0,1], which is the basic 1-complex
from which one can build up more fancy ones.  The kind of category he
uses has been described by:

5)  S. Gelfand and D. Kazhdan, Examples of tensor categories, Invent.
Math. 109 (1992) 595-617.

It is formed by starting with the category of representations of an
algebraic group in characteristic p, and then making a semisimple
category out of this in a manner strongly reminiscent of what they do in
the theory of quantum groups at roots of unity.  (See "<A HREF = "week5.html">week5</A>" for a bit
more about this.)  The object A is taken to be the sum of one copy of
each irreducible representation.  (Again, this is strikingly
reminiscent, and no doubt based on, what occurs in the physics of the
Wess-Zumino-Witten model, where quantum groups at roots of unity play
the role a finite group is playing here.)  

So, to round off a long story, Quinn and Ivelini Bobtcheva are currently
engaged in some rather massive computer calculations in order to
actually explicitly obtain the data necessary to calculate in the TQFTs
of this form.  They have been looking at the groups SL(2), SL(3), Sp(4)
and G_2 over Z_p, where p is small (up to 19 for the SL(2) case).  They
are finding some interesting stuff just by calculating the TQFT
invariants of the 2-complexes corresponding to the presentations
<x|x^n>.  (Note that n = 0 gives a space that's a wedge of a circle and
S^2, while n = 1 gives a disk.)  Namely, they are finding periodicity in
n.  

But they haven't found any counterexamples to the Andrews-Curtis
conjecture yet!



6) Knots and Quantum Gravity, ed. John Baez, Oxford University Press (to
appear).  

This is the proceedings of a workshop held at U.C. Riverside; a large
percentage of the papers contain new results.  Let me simply list them:


\begin{verbatim}

The Loop Formulation of Gauge Theory and Gravity, by Renate Loll

Representation Theory of Analytic Holonomy C* Algebras, by Abhay
Ashtekar and Jerzy Lewandowski 

The Gauss Linking Number in Quantum Gravity, by Rodolfo Gambini and Jorge
Pullin (currently available as <A HREF = "http://xxx.lanl.gov/abs/gr-qc/9310025">gr-qc/9310025</A>)

Vassiliev Invariants and the Loop States in Quantum Gravity, by Louis H.
Kauffman (soon to be on gr-qc)

Geometric Structures and Loop Variables in (2+1)-Dimensional Gravity,
by Steven Carlip (currently available as <A HREF = "http://xxx.lanl.gov/abs/gr-qc/9309020">gr-qc/9309020</A>)

From Chern-Simons to WZW via Path Integrals, by Dana S. Fine

Topological Field Theory as the Key to Quantum Gravity, by Louis Crane
(currently available as <A HREF = "http://xxx.lanl.gov/abs/hep-th/9308126">hep-th/9308126</A>)

Strings, Loops, Knots and Gauge Fields, by John Baez (currently
available as <A HREF = "http://xxx.lanl.gov/abs/hep-th/9309067">hep-th/9309067</A>)

BF Theories and 2-knots, by Paolo Cotta-Ramusino and Maurizio Martellini

Knotted Surfaces, Braid Movies, and Beyond, by J. Scott Carter and
Masahico Saito 
\end{verbatim}
    
\par\noindent\rule{\textwidth}{0.4pt}

% </A>
% </A>
% </A>
