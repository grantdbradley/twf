
% </A>
% </A>
% </A>
\week{August 31, 2005 }


Work on quantum gravity has seemed stagnant and stuck for the 
last couple of years, which is why I've been turning more
towards pure math.  

Over in string theory they're contemplating a vast "landscape" of 
possible universes, each with their own laws of physics - one or
more of which might be ours.  Each one is supposed to correspond
to a different "vacuum" or "background" for the 
marvelous unifying
M-theory that we don't completely understand yet.  They can't 
choose the right vacuum except by the good old method of fitting 
the experimental data.  But these days, this time-honored method 
gets a lot less airplay than the "anthropic principle":

1) Leonard Susskind, The anthropic landscape of string theory,
available as <A HREF = "http://xxx.lanl.gov/abs/hep-th/0302219">hep-th/0302219</A>.

Perhaps this is because it's more grandiose to imagine choosing 
one theory out of a multitude by discovering that it's among the 
few that supports intelligent life, than by noticing that it 
correctly predicts experimental results.  Or, perhaps it's because
nobody really knows how to get string theory to predict experimental
results!  Even after you chose a vacuum, you'd need to see how
supersymmetry gets broken, and this remain quite obscure.

There's still tons of beautiful math coming out of string theory,
mind you: right now I'm just talking about physics.
 
What about loop quantum gravity?  This line of research has always 
been less ambitious than string theory.  Instead of finding the 
correct theory of everything, its goal has merely been to find 
\emph{any} theory that combines gravity and quantum mechanics in a 
background-free way.  But, it has major problems of its own:
nobody knows how it can successfully mimic general relativity
at large length scales, as it must to be realistic!  Old-fashioned
perturbative quantum gravity failed on this score because it 
wasn't renormalizable.  Loop quantum gravity may get around this
somehow... but it's about time to see exactly how.

Loop quantum gravity follows two main approaches: the so-called
"Hamiltonian" or "spin network" approach, which focuses on the 
geometry of space at a given time, and the so-called "Lagrangian" 
or "spin foam" approach, which focuses on the geometry of 
spacetime. 

In the last couple of years, the most interesting new work in the 
Hamiltonian approach has focussed on problems with extra symmetry,
like black holes and the big bang.  Here's a nontechnical 
introduction:

2) Abhay Ashtekar, Gravity and the quantum, available as
<A HREF = "http://xxx.lanl.gov/abs/gr-qc/0410054">gr-qc/0410054</A>.

and here's some new work that treats the information loss
puzzle:

3) Abhay Ashtekar and Martin Bojowald, Black hole evaporation:
a paradigm, Class. Quant. Grav. 22 (2005) 3349-3362.  Also
available as <A HREF = "http://xxx.lanl.gov/abs/gr-qc/0504029">gr-qc/0504029</A>.

However, by focusing on solutions with extra symmetry, one puts 
off facing the hardest aspects of renormalization, or whatever
its equivalent might be in loop quantum gravity.

The other approach - the spin foam approach - got stalled when 
the most popular model seemed to give spacetimes made mostly of 
squashed-flat "degenerate 4-simplexes".  Various papers have
found an effect like this: see "<A HREF = "week198.html">week198</A>" for more details.  So,
there's definitely a real phenomenon going on here.   However, 
its physical significance remains a bit obscure.  The devil is
in the details. 

In particular, even though the \emph{amplitude} for a single large 
4-simplex in the Barrett-Crane model is dominated by degenerate 
geometries, certain \emph{second derivatives} of the amplitude might not -
and this may be what really matters.  Carlo Rovelli has recently
come out with a paper on this:

4) Carlo Rovelli, Graviton propagator from background-independent 
quantum gravity, available as <A HREF = "http://xxx.lanl.gov/abs/gr-qc/0508124">gr-qc/0508124</A>.

If the idea holds up, I'll be pretty excited.  If not, I'll be
bummed.  But luckily, I've already gone through the withdrawal 
pains of switching my focus away from quantum gravity.   When you 
do theoretical physics, sometimes you feel the high of discovering 
hidden truths about the physical universe.  Sometimes you feel the
agony of suspecting that those "hidden truths" were probably just
a bunch of baloney... or, realizing that you may never know.  
Ultimately nature has the last word.  

Math is, at least for me, a less nerve-racking pursuit, since 
the truths we find can be confirmed simply by discussing them: 
we don't need to wait for experiment.  Math is just as grand as 
physics, or more so.  But it's more wispy and ethereal, since it's
about pure pattern in general - not the particular magic patterns 
that became the world we see.  So, the stakes are lower, but the 
odds are higher.

Speaking of math, I really want to talk about the Streetfest - the 
conference in honor of Ross Street's 60th birthday.  It was a real 
blast: over sixty talks in two weeks in two cities, Sydney and 
Canberra.  However, I accidentally left my notes from those talks 
at home before zipping off to Calgary for a summer school on 
homotopy theory:

5) Topics in Homotopy Theory, graduate summer school at the 
Pacific Institute of Mathematics run by Kristine Bauer and Laura 
Scull.  Recommended reading material available at 
<A HREF = "http://www.pims.math.ca/science/2005/05homotopy/reading.html">
http://www.pims.math.ca/science/2005/05homotopy/reading.html</A>

So, I'll say a bit about what I learned at this school.  

Dan Dugger spoke about motivic homotopy theory, which was
\emph{great}, because I've been trying to understand stuff
from number theory and algebraic geometry like the Weil 
conjectures, etale cohomology, motives, and Voevodsky's proof 
of the Milnor conjecture... and thanks to his wonderfully 
pedagogical lectures, it's all starting to make some sense!

I hope to talk about this someday, but not now.

Alejandro Adem spoke about orbifolds and group cohomology.
Purely personally, the most exciting thing here was seeing
that orbifolds can also be seen as certain kinds of topological 
groupoids, or stacks, or topoi... so that various versions of
"categorified topology" are actually different faces of the 
same thing!
  
I may talk about this someday, too, but not now.

I spoke about higher gauge theory and its relation to Eilenberg-Mac 
Lane spaces.  I may talk about that too someday, but not now.

Dev Sinha spoke about operads, and besides explaining the basics, 
he said a couple of things that really blew me away.  So, I want
to talk about this now.

For one, the homology of the little k-cubes operad is a graded 
version of the Poisson operad!  For two, the little 2-cubes 
operad acts on the space of thickened long knots!  

But for this to thrill you like it thrills me, I'd better say a
word about operads - and especially little k-cubes operads.

Operads, and especially the little k-cubes operads, were 
invented by Peter May in the early 1970s to formalize the 
algebraic structures lurking in "infinite loop spaces".  In 
"<A HREF = "week149.html">week149</A>" I explained what infinite loop spaces are, and how 
they give generalized cohomology theories, but let's not get 
bogged down in this motivation now, since operads are actually
quite simple.  

In its simplest form, an operad is a gizmo that has for each 
n = 0,1,2,... a set O(n) whose elements are thought of as n-ary 
operations - operations with n inputs.  It's good to draw such 
operations as black boxes with n input wires and one output:  


\begin{verbatim}

                  \    |    /
                   \   |   / 
                    \  |  /
                     -----
                    |  f  | 
                     -----
                       |
                       |
\end{verbatim}
    
For starters these operations are purely abstract things that
don't actually operate on anything.  Only when we consider a
"representation" or "action" of an operad do they get incarnated
as actual n-ary operations on some set.   The point of operads is 
to study their actions. 

But, for completeness, let me sketch the definition of an operad.
An operad tells us how to compose its operations, like this:


\begin{verbatim}

         \    /     \  |  /       | 
          \  /       \ | /        |
          -----      -----      -----
         |  b  |    |  c  |    |  d  |
          -----      -----      -----
              \        |        /  
               \       |       /
                \      |      /
                 \     |     /
                  \    |    / 
                   \   |   /  
                    \  |  /
                     -----
                    |  a  | 
                     -----
                       |
                       |
\end{verbatim}
    
Here we are composing a with b,c, and d to get an operation with 6 
inputs called a o (b,c,d).

An operad needs to have a unary operation serving as the identity 
for composition.  It also needs to satisfy an "associative law" 
that makes a composite of composites like this well-defined:



\begin{verbatim}

              \    /   |  \  |  /   \     / 
               \  /    |   \ | /     \   / 
                ---   ---   ---       ---
               |   | |   | |   |     |   |
                ---   ---   ---       ---
                   \   |   /          /
                    \  |  /          / 
                     \ | /          / 
          -----      -----      -----
         |     |    |     |    |     |
          -----      -----      -----
              \        |        /  
               \       |       /   
                \      |      /
                 \     |     / 
                  \    |    / 
                   \   |   /   
                    \  |  /
                     -----
                    |     | 
                     -----
                       |
                       |
\end{verbatim}
    
(This picture has a 0-ary operation in it, just to emphasize
that this is allowed.)  

That's the complete definition of a "planar operad".  In a 
full-fledged operad we can do more: we can permute the inputs 
of any operation and get a new operation:
                      

\begin{verbatim}

                      \ /   /
                       /   /
                      / \ /    
                     /   /   
                    /   / \
                    \  |  /
                     -----
                    |     | 
                     -----
                       |
                       |
\end{verbatim}
    
This gives actions of the permutation groups on the sets O(n).  
We also demand that these actions be compatible with composition, 
in a way that's supposed to be obvious from the pictures.   For 
example:



\begin{verbatim}

       \  |  /   |   \   /               \\\ /   / /
        \ | /    |    \ /                 \\/   / /
         ---    ---   ---                  /\\ / /
        | a |  | b | | c |                / \\/ /
         ---    ---   ---                /   / /
           \     /   /                  /   / /\\
            \   /   /                  /   | | \\\  
             \ /   /                  /    | |  \\\
              /   /                  ---   ---   ---
             / \ /           =      | b | | c | | a |
            /   /                    ---   ---   ---
           /   / \                      \   |   /
           \  |  /                       \  |  /
            -----                         -----
           |  d  |                       |  d  | 
            -----                         -----
              |                             |
              |                             |
\end{verbatim}
    
and similarly for permuting the inputs of the black boxes on
top.

\emph{Voil&agrave;!}

Now, operads make sense in various contexts.  So far we've been 
talking about operads that have a \emph{set} O(n) of n-ary operations 
for each n.  These have actions on \emph{sets}, where each guy in O(n) 
gets incarnated as a \emph{function} that eats n elements of some set 
and spits out an element of that set. 

But historically, Peter May started by inventing operads that have 
a \emph{topological space} of n-ary operations for each n.  These like 
to act on \emph{topological spaces}, with the operations getting 
incarnated as \emph{continuous maps}.     

Most importantly, he invented an operad called the "little k-cubes 
operad".  Here O(n) is the space of ways of putting n nonoverlapping
little k-dimensional cubes in a big one.  We don't demand that
the little cubes are actually cubes: they can be rectangular
boxes.  We do demand that their walls are nicely lined up with 
the walls of the big cube:



\begin{verbatim}

        ---------------------
       |                     |
       |           -----     |
       | -----    |     |    | 
       ||     |   |     |    |
       ||     |   |     |    |           typical  
       | -----    |     |    |     3-ary operation in the       
       |           -----     |     little 2-cubes operad
       |   ----------------  |
       |  |                | |
       |   ----------------  |
       |                     |
        ---------------------

\end{verbatim}
    
This is an operation in O(3), where O is the little 2-cubes
operad.  Or, at least it would be if I labelled each of the 3 
little 2-cubes - we need that extra information.  

We compose operations by sticking pictures like this into 
each of the little k-cubes in another picture like this!  
I should draw you an example, but I'm too lazy.  So, figure
it out yourself and check the associative law.

The reason this example is so important is that we get an action
of the little k-cubes operad whenever we have a "k-fold loop 
space".  

Starting from a space S equipped with a chosen point *, the 
k-fold loop space \Omega ^{k}(S) is the space of all maps from
a k-sphere into S that send the north pole to the point *.  But
this is also the space of all maps from a k-cube into S sending 
the boundary of the k-cube to the point *.  

So, given n such such maps, we can glom them together using an 
n-ary operation in the little k-cubes operad:


\begin{verbatim}

        ---------------------
       |*********************|
       |***********-----*****|
       |*-----****|     |****| 
       ||     |***|     |****|
       ||     |***|     |****|
       | ----- ***|     |****|
       |***********-----*****|
       |***----------------**|
       |**|                |*|
       |***----------------**|
       |*********************|
        ---------------------
\end{verbatim}
    
where we map all the shaded stuff to the point *.  We get 
another map from the k-cube to S sending the boundary to *.
So: 

<div align = center>
             ANY k-FOLD LOOP SPACE HAS AN ACTION OF <br>
                  THE LITTLE k-CUBES OPERAD
</div>

But the really cool part is the converse: 

<div align = center>
           ANY CONNECTED POINTED SPACE WITH AN ACTION OF <br>
                THE LITTLE k-CUBES OPERAD IS  <br>
         HOMOTOPY EQUIVALENT TO A k-FOLD LOOP SPACE
</div>

This is too technical to make a good bumper sticker, so if you 
want people in your neighborhood to get interested in operads, 
I suggest combining both the above slogans into one:

<div align = center>
            A k-FOLD LOOP SPACE IS THE SAME AS <br>
          AN ACTION OF THE LITTLE k-CUBES OPERAD                  
</div>
Like any good slogan, this leaves out some important fine print, 
but it <A HREF = "churchsign.jpg">gets the basic idea across</A>.  
Modulo some details, being a
k-fold loop space amounts to having a bunch of operations: one
for each way of stuffing little k-cubes in a big one!

By the way:

Speaking of bumper stickers, I'm in Montreal now, and there's
a funky hangout on the Boulevard Saint-Laurent called Cafe \pi 
where people play chess - and they sell T-shirts, key rings, 
baseball caps and coffee mugs decorated with the Greek letter \pi !  
The T-shirts are great if you're going for a kind of math-nerd/punk 
look; I got one to wow the students in my undergraduate 
courses.  I don't usually provide links to commercial websites, 
but I made an exception for Acme Klein Bottles, and I'll make an 
exception for Cafe \pi :

6) Cafe \pi , <A HREF = "http://www.cafepi.ca/">http://www.cafepi.ca/</A>

Unfortunately they don't sell bumper stickers.

But where were we?  Ah yes - the little k-cubes operad.

The little k-cubes operad sits in the little (k+1)-cubes operad 
in an obvious way.  Indeed, it's a "sub-operad".  So, we can 
take the limit as k goes to \infty  and form the "little 
\infty -cubes operad".  Any infinite loop space gets an action 
of this... and that's why Peter May invented operads!

You can read more about these ideas in May's book:

7) J. Peter May, The Geometry of Iterated Loop Spaces, 
Lecture Notes in Mathematics 271, Springer, Berlin, 1972.

or for a more gentle treatment, try this expository article:

8) J. Peter May, Infinite loop space theory, Bull. Amer. Math. 
Soc. 83 (1977), 456-494.

But Dev Sinha told us about some subsequent work by Fred 
Cohen, who computed the homology and cohomology of the little
k-cubes operad.

For this, we need to think about operads in the world of linear 
algebra.  Here we consider operads that have a \emph{vector space} of 
n-ary operations for each n, which get incarnated as <em>multilinear


% parser failed at source line 511
