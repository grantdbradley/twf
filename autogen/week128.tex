
% </A>
% </A>
% </A>
\week{January 4, 1999}

This week I'd like to catch you up on the latest developments
in quantum gravity.  First, a book that everyone can enjoy:  

1) John Archibald Wheeler and Kenneth Ford, Geons, Black Holes, and
Quantum Foam: A Life in Physics, Norton, New York, 1998.

This is John Wheeler's autobiography.  If Wheeler's only contribution to
physics was being Bohr's student and Feynman's thesis advisor, that in 
itself would have been enough.  But he did much more.  He played a
crucial role in the Manhattan project and the subsequent development of
the hydrogen bomb.  He worked on nuclear physics, cosmic rays, muons and
other elementary particles.  And he was also one of the earlier people
to get really excited about the more outlandish implications of general
relativity.  For example, he found solutions of Einstein's equation that
correspond to regions of gravitational field held together only by their
own gravity, which he called "geons".  He was not the first to study
black holes, but he was one of the first people to take them seriously,
and he invented the term "black hole".  And the reason he is <em>
my</em> hero
is that he took seriously the challenge of reconciling general
relativity and quantum theory.  Moreover, he recognized how radical the
ideas needed to accomplish this would be - for example, the idea that
spacetime might not be truly be a continuum at short distance scales,
but instead some sort of "quantum foam".  

Anyone interested in the amazing developments in physics during the 20th
century should read this book!  Here is the story of how he first met
Feynman:

\begin{quote}
     Dick Feynman, who had earned his bachelor's degree at MIT, 
     showed up at my office door as a brash and appealing twenty-one-
     year-old in the fall of 1939 because, as a new student with a
     teaching assistantship, he had been assigned to grade papers for
     me in my mechanics course.  As we sat down to talk about the course 
     and his duties, I pulled out and placed on the table between us
     a pocket watch.  Inspired by my father's keenness for time-and-
     motion studies, I was keeping track of how much time I spent on 
     teaching and teaching-related activities, how much on research, 
     and how much on departmental or university chores.  This meeting
     was in the category of teaching-related.  Feynman may have been 
     a little taken aback by the watch but he was not one to be 
     intimidated.  He went out and bought a dollar watch (as I learned
     later), so he would be ready for our next meeting.  When we got
     together again, I pulled out my watch and put it on the table 
     between us.  Without cracking a smile, Feynman pulled out his
     watch and put it on the table next to mine.  His theatrical sense
     was perfect.  I broke down laughing, and soon he was laughing as
     hard as I, until both of us had tears in our eyes.  It took quite
     a while for us to sober up and get on with our discussion.  This 
     set the tone for a wonderful friendship that endured for the rest
     of his life.
     
\end{quote}
Next for something a wee bit more technical:
    
2) Steven Carlip, Quantum Gravity in 2+1 Dimensions, Cambridge 
University Press, 1998.

If you want to learn about quantum gravity in 2+1 dimensions this is
the place to start, because Carlip is the world's expert on this subject, 
and he's pretty good at explaining things.  
 
(By the way, physicists write "2+1 dimensions", not because they can't
add, but to emphasize that they are talking about 2 dimensions of space
and 1 dimension of time.)
 
Quantum gravity in 2+1 dimensions is just a warmup for what physicists 
are really interested in - quantum gravity in 3+1 dimensions.   Going 
down a dimension really simplifies things, because Einstein's equations 
in 2+1 dimensions say that the energy and momentum flowing through a 
given point of spacetime completely determine the curvature there, 
unlike in higher dimensions.  In particular, spacetime is \emph{flat}
in the vacuum in 2+1 dimensions, so there's no gravitational radiation. 
Nonetheless, quantum gravity in 2+1 dimensions is very interesting, for
a number of reasons.  Most importantly, we can solve the equations
exactly, so we can use it as a nice testing-ground for all sorts of
ideas people have about quantum gravity in 3+1 dimensions.   

Quantum gravity is hard for various reasons, but most of all it's hard
because, unlike traditional quantum field theory, it's a "background-free" 
theory.  What I mean by this is that there's no fixed way of measuring
times and distances.  Instead, times and distances must be measured with
the help of the geometry of spacetime, and this geometry undergoes
quantum fluctuations.  That throws most of our usual methods for doing
physics right out the window!  Quantum gravity in 2+1 dimensions gives
us, for the first time, an example of a background-free theory where we
can work out everything in detail.  

Here's the table of contents of Carlip's book:
\begin{quote}
 1. Why (2+1)-dimensional gravity?
 2. Classical general relativity in 2+1 dimensions
 3. A field guide to the (2+1)-dimensional spacetimes
 4. Geometric structures and Chern-Simons theory
 5. Canonical quantization in reduced phase space
 6. The connection representation
 7. Operator algebras and loops
 8. The Wheeler-DeWitt equation
 9. Lorentzian path integrals
10. Euclidean path integrals and quantum cosmology
11. Lattice methods
12. The (2+1)-dimensional black hole
13. Next steps
 A. Appendix: The topology of manifolds
 B. Appendix: Lorentzian metrics and causal structure
 C. Appendix: Differential geometry and fiber bundles

\end{quote}
And now for some stuff that's available online.   First of all, anyone
who wants to keep up with research on gravity should remember to read
"Matters of Gravity".  I've talked about it before, but here's the
latest edition:

3) Jorge Pullin, editor, Matters of Gravity, vol. 12, available at 
<A HREF = "http://xxx.lanl.gov/abs/gr-qc/9809031">gr-qc/9809031</A> and at
<A HREf = "http://vishnu.nirvana.phys.psu.edu/mog.html">http://vishnu.nirvana.phys.psu.edu/mog.html</A>


There's a lot of good stuff in here.  Quantum gravity buffs will
especially be interested in Gary Horowitz's article "A nonperturbative
formulation of string theory?" and Lee Smolin's "Neohistorical
approaches to quantum gravity".  The curious title of Smolin's article
refers to \emph{new} work on quantum gravity involving a sum over 
\emph{histories}
- or in other words, spin foam models.

Even if you can't go to a physics talk, these days you can sometimes 
find it on the world-wide web.  Here's one by John Barrett:

4) John W. Barrett, State sum models for quantum gravity, Penn State
relativity seminar, August 27, 1998, audio and text of transparencies
available at 
<A HREF = "http://vishnu.nirvana.phys.psu.edu/online/Html/Seminars/Fall1998/Barrett/">http://vishnu.nirvana.phys.psu.edu/online/Html/Seminars/Fall1998/Barrett/</A>


Barrett and Crane have a theory of quantum gravity, which I've also
worked on; I discussed it last in "<A HREF = "week113.html">week113</A>" and "<A HREF = "week120.html">week120</A>".  Before I
describe it I should warn the experts that this theory deals with 
Riemannian rather than Lorentzian quantum gravity (though Barrett and
Crane are working on a Lorentzian version, and I hear Friedel and
Krasnov are also working on this).   Also, it only deals with vacuum
quantum gravity - empty spacetime, no matter.  

In this theory, spacetime is chopped up into 4-simplices.  A 4-simplex 
is the 4-dimensional analog of a tetrahedron.  To understand what I'm 
going to say next, you really need to understand 4-simplices, so let's 
start with them.

It's easy to draw a 4-simplex.  Just draw 5 dots in a kind of circle and
connect them all to each other!   You get a pentagon with a pentagram
inscribed in it.   This is a perspective picture of a 4-simplex
projected down onto your 2-dimensional paper.  If you stare at this
picture you will see the 4-simplex has 5 tetrahedra, 10 triangles, 
10 edges and 5 vertices in it.  

The shape of a 4-simplex is determined by 10 numbers.  You can take
these numbers to be the lengths of its edges, but if you want to be
sneaky you can also use the areas of its triangles.  Of course, there
are some constraints on what areas you can choose for there to \emph{exist} 
a
4-simplex having triangles with those areas.   Also, there are some
choices of areas that fail to make the shape \emph{unique}: for one of these
bad choices, the 4-simplex can flop around while keeping the areas of
all its triangles fixed.  But generically, this non-uniqueness doesn't
happen.  

In Barrett and Crane's theory, we chop spacetime into 4-simplices and 
describe the geometry of spacetime by specifying the area of each
triangle.  But the geometry is "quantized", meaning that the area
takes a discrete spectrum of possible values, given by 

                          sqrt(j(j+1))

where the "spin" j is a number of the form 0, 1/2, 1, 3/2, etc.   This
formula will be familiar to you if you've studied the quantum mechanics
of angular momentum.  And that's no coincidence!  The cool thing about
this theory of quantum gravity is that you can discover it just by 
thinking a long time about general relativity and the quantum mechanics 
of angular momentum, as long as you also make the assumption that
spacetime is chopped into 4-simplices.  

So: in Barrett and Crane's theory the geometry of spacetime is described
by chopping spacetime into 4-simplices and labelling each triangle with
a spin.  Let's call such a labelling a "quantum 4-geometry".  Similarly,
the geometry of space is described by chopping space up into tetrahedra
and labelling each triangle with a spin.  Let's call this a "quantum 
3-geometry".  

The meat of the theory is a formula for computing a complex number
called an "amplitude" for any quantum 4-geometry.  This number plays the
usual role that amplitudes do in quantum theory.  In quantum theory, if
you want to compute the probability that the world starts in some state
\psi  and ends up in some state \psi ', you just look at all the ways the
world can get from \psi  to \psi ', compute an amplitude for each way, add
them all up, and take the square of the absolute value of the result. 
In the special case of quantum gravity, the states are quantum 3-geometries, 
and the ways to get from one state to another are quantum 4-geometries. 

So, what's the formula for the amplitude of a quantum 4-geometry?  It
takes a bit of work to explain this, so I'll just vaguely sketch how it
goes.  First we compute amplitudes for each 4-simplex and multiply all
these together.  Then we compute amplitudes for each triangle and
multiply all these together.  Then we multiply these two numbers.

(This is analogous to how we compute amplitudes for Feynman diagrams
in ordinary quantum field theory.  A Feynman diagram is a graph whose
edges have certain labellings.  To compute its amplitude, first we 
compute amplitudes for each edge and multiply them all together.  Then 
we compute amplitudes for each vertex and multiply them all together.
Then we multiply these two numbers.  One goal of work on "spin 
foam models" is to more deeply understand this analogy with Feynman
diagrams.)

Anyway, to convince oneself that this formula is "good", 
one would like
to relate it to other approaches to quantum gravity that also involve
4-simplices.  For example, there is the Regge calculus, which is a
discretized version of \emph{classical} general relativity.  In this approach
you chop spacetime into 4-simplices and describe the shape of each
4-simplex by specifying the lengths of its edges.  Regge invented a
formula for the "action" of such a geometry which approaches the usual
action for classical general relativity in the continuum limit.  I 
explained the formula for this "Regge action" in "<A HREF = "week120.html">week120</A>".   

Now if everything were working perfectly, the amplitude for a 4-simplex
in the Barrett-Crane model would be close to exp(iS), where S is the
Regge action of that 4-simplex.   This would mean that the Barrett-Crane
model was really a lot like a path integral in quantum gravity.  Of
course, in the Barrett-Crane model all we know is the areas of the triangles 
in each 4-simplex, while in the Regge calculus we know the lengths of
its edges.  But we can translate between the two, at least generically,
so this is no big deal.

Recently, Barrett and Williams
came up with a nice argument saying that in the limit
where the triangles have large areas, the amplitude for a 4-simplex in
the Barrett-Crane theory is proportional, not to exp(iS), but to cos(S):

5) John W. Barrett and Ruth M. Williams, The asymptotics of an amplitude 
for the 4-simplex, preprint available as 
<A HREF = "http://xxx.lanl.gov/abs/gr-qc/9809032">gr-qc/9809032</A>.

This argument is not rigorous - it uses a stationary phase approximation
that requires further justification.  But Regge and Ponzano used a
similar argument to show the same sort of thing for quantum gravity in 3
dimensions, and their argument was recently made rigorous by Justin
Roberts, with a lot of help from Barrett:

6) Justin Roberts, Classical 6j-symbols and the tetrahedron, preprint
available as 
<A HREF = "http://xxx.lanl.gov/abs/math-ph/9812013">math-ph/9812013</A>.

So one expects that with work, one can make Barrett and
Williams' argument rigorous.

But what does it mean?  Why does he get cos(S) instead of exp(iS)? 
Well, as I said, the same thing happens one dimension down in the
so-called Ponzano-Regge model of 3-dimensional Riemannian quantum
gravity, and people have been scratching their heads for decades trying
to figure out why.  And by now they know the answer, and the same
answer applies to the Barrett-Crane model.  

The problem is that if you describe 4-simplex using the areas of its
triangles, you don't \emph{completely} know its shape.  (See, I lied to you
before - that's why you gotta read the whole thing.)  You only know it
\emph{up to reflection}.  You can't tell the difference between a 4-simplex
and its mirror-image twin using only the areas of its triangles!  When
one of these has Regge action S, the other has action -S.  The Barrett-
Crane model, not knowing any better, simply averages over both of them, 
getting 

(1/2)(exp(iS) + exp(-iS)) = cos(S)

So it's not really all that bad; it's doing the best it can under
the circumstances.  Whether this is good enough remains to be seen.

(Actually I didn't really \emph{lie} to you before; I just didn't tell you 
my definition of "shape", so you couldn't tell whether mirror-image
4-simplices should count as having the same shape.  Expository prose
darts between the Scylla of overwhelming detail and the Charybdis of 
vagueness.)

Okay, on to a related issue.  In the Barrett-Crane model one describes a
quantum 4-geometry by labelling all the triangles with spins.  This
sounds reasonable if you think about how the shape of a 4-simplex is
almost determined by the areas of its triangles.  But if you actually
examine the derivation of the model, it starts looking more odd.  What
you really do is take the space of geometries of a \emph{tetrahedron}
embedded in R^{4}, and use a trick called geometric quantization to get
something called the "Hilbert space of a quantum tetrahedron in 4
dimensions".   You then build your 4-simplices out of these quantum
tetrahedra.  

Now the Hilbert space of a quantum tetrahedron has a basis labelled by
the eigenvalues of operators corresponding to the areas of its 4
triangular faces.  In physics lingo, it takes 4 "quantum numbers" to
describe the shape of a quantum tetrahedron in 4 dimensions.

But classically, the shape of a tetrahedron is \emph{not} determined by the
areas of its triangles: it takes 6 numbers to specify its shape, not
just 4.  So there is something funny going on.  

At first some people thought there might be more states of the quantum 
tetrahedron than the ones Barrett and Crane found.  But Barbieri came up
with a nice argument suggesting that Barrett and Crane had really found
all of them: 

7) Andrea Barbieri, Space of the vertices of relativistic
spin networks, preprint available as 
<A HREF = "http://xxx.lanl.gov/abs/gr-qc/9709076">gr-qc/9709076</A>.

While convincing, this argument was not definitive, since it
assumed something plausible but not yet proven - namely, that the
"6j symbols don't have too many exceptional zeros".  Later, 
Mike Reisenberger came up with a completely rigorous argument:

8) Michael P. Reisenberger, On relativistic spin network vertices,
preprint available as <A HREF = "http://xxx.lanl.gov/abs/gr-qc/9809067">gr-qc/9809067</A>.

But while this settled the facts of the matter, it left open the
question of "why" --- why does it take \emph{6} 
numbers to describe the
shape of classical tetrahedron in 4 dimensions but only \emph{4} numbers
to describe the shape of a quantum one?  John Barrett and I have 
almost finished a paper on this, so I'll give away the answer.

Not surprisingly, the key is that in quantum mechanics, not all
observables commute.   You only use the eigenvalues of \emph{commuting}
observables to label a basis of states.  The areas of the quantum
tetrahedron's faces commute, and there aren't any other independent
commuting observables.  It's a bit like how in classical mechanics
you can specify both the position and momentum of a particle, but
in quantum mechanics you can only specify one.

This isn't news, of course.  And indeed, people knew perfectly well 
that for this reason, it takes only \emph{5} numbers to describe the shape of
a quantum tetrahedron in 3 dimensions.  The real puzzle was why it takes
even fewer numbers when your quantum tetrahedron lives in 4 dimensions! 
It seemed bizarre that adding an extra dimension would reduce the number
of degrees of freedom!  But it's true, and it's just a spinoff of the
uncertainty principle.  Crudely speaking, in 4 dimensions the fact that
you know your tetrahedron lies in some hyperplane makes you unable to
know as much about its shape.

Here are some other talks available on the web:

9) Abhay Ashtekar, Chris Beetle and Steve Fairhurst, Mazatlan lectures 
on black holes, slides available at 
<A HREF = "http://vishnu.nirvana.phys.psu.edu/online/Html/Conferences/Mazatlan/">http://vishnu.nirvana.phys.psu.edu/online/Html/Conferences/Mazatlan/</A>

These explain a new concept of "nonrotating isolated horizon"
which allow one to formulate and prove the zeroth and first laws
of black hole mechanics in a way that only refers to the geometry
of spacetime near the horizon.  For more details try:

10) Abhay Ashtekar, Chris Beetle and S. Fairhurst, Isolated horizons:
a generalization of black hole mechanics, preprint available as
<A HREF = "http://xxx.lanl.gov/abs/gr-qc/9812065">gr-qc/9812065</A>.  

This concept also serves as the basis for a forthcoming 2-part paper
where Ashtekar, Corichi, Krasnov and I compute the entropy of a quantum
black hole (see "<A HREF = "week112.html">week112</A>" for more on this).  

Finally, here are a couple more papers.  I don't have time to say much 
about them, but they're both pretty neat:

11) Matthias Arnsdorf and R. S. Garcia, Existence of spinorial states in
pure loop quantum gravity, preprint available as <A HREF = "http://xxx.lanl.gov/abs/gr-qc/9812006">gr-qc/9812006</A>.

I'll just quote the abstract:

     We demonstrate the existence of spinorial states in a theory of
     canonical quantum gravity without matter. This should be regarded
     as evidence towards the conjecture that bound states with particle
     properties appear in association with spatial regions of
     non-trivial topology. In asymptotically trivial general relativity
     the momentum constraint generates only a subgroup of the spatial
     diffeomorphisms. The remaining diffeomorphisms give rise to the
     mapping class group, which acts as a symmetry group on the phase
     space. This action induces a unitary representation on the loop
     state space of the Ashtekar formalism. Certain elements of the
     diffeomorphism group can be regarded as asymptotic rotations of
     space relative to its surroundings. We construct states that
     transform non-trivially under a 2 \pi  rotation: gravitational
     quantum states with fractional spin.

14) Steve Carlip, Black hole entropy from conformal field theory in any 
dimension, preprint available as 
<A HREF = "http://xxx.lanl.gov/abs/hep-th/9812013">hep-th/9812013</A>.

Again, here's the abstract:

     When restricted to the horizon of a black hole, the 'gauge'
     algebra of surface deformations in general relativity contains a
     physically important Virasoro subalgebra with a calculable central
     charge. The fields in any quantum theory of gravity must transform
     under this algebra; that is, they must admit a conformal field
     theory description. With the aid of Cardy's formula for the
     asymptotic density of states in a conformal field theory, I use
     this description to derive the Bekenstein-Hawking entropy. This method
     is universal - it holds for any black hole, in any dimension, and
     requires no details of quantum gravity - but it is also explicitly
     statistical mechanical, based on the counting of microscopic
     states.

On Thursday I'm flying to Schladming, Austria to attend a workshop on 
geometry and physics organized by Harald Grosse and Helmut Gausterer.
Some cool physicists will be there, like Daniel Kastler and Julius Wess.  
If I understand what they're talking about I'll try to explain it here.  
Happy new year!


 \par\noindent\rule{\textwidth}{0.4pt}
\textbf{Addendum:} Above I wrote:

\begin{quote}
Recently, Barrett and Williams
came up with a nice argument saying that in the limit
where the triangles have large areas, the amplitude for a 4-simplex in
the Barrett-Crane theory is proportional, not to exp(iS), but to cos(S)....

This argument is not rigorous - it uses a stationary phase approximation
that requires further justification. But  
similar argument to show the same sort of thing for quantum gravity in 3
dimensions, and their argument was recently made rigorous by Justin
Roberts, with a lot of help from Barrett....

So one expects that with work, one can make Barrett and Williams' 
argument rigorous.
\end{quote}

In fact one can't make it rigorous: it's wrong!  In the limit of large
areas the amplitude for a 4-simplex in the Barrett-Crane model is
wildly different from cos(S), or exp(iS), or anything like that.  Dan
Christensen, Greg Egan and I showed this in a couple of papers that I
discuss in "<a href = "week170.html">week170</a>" and
"<a href = "week198.html">week198</a>".  Our results were
confirmed by John Barrett, Chris Steel, Laurent Friedel and David
Louapre.  

By now &mdash; I'm writing this in 2009 &mdash; it's
generally agreed that the Barrett-Crane model is wrong and another
model is better.  To read about this new model, see "<a href =
"week280.html">week280</a>".



 \par\noindent\rule{\textwidth}{0.4pt}

% </A>
% </A>
% </A>
