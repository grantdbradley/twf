
% </A>
% </A>
% </A>
\week{October 11, 2002 }



I've been talking about q-mathematics, and last week the story reached 
a kind of climax when I combined the themes of q-deformation,
categorification, and Dynkin diagrams.  These are three of my favorite
things, but I can't expect everyone to enjoy them as much as I do, so
now I'll back down and talk about something simpler - but related. 
Let's see what happens when you put Pascal's triangle in a magnetic field!

You've probably seen Pascal's triangle.  It has a 1 on top, and
each other number in the triangle is gotten by adding the number 
or numbers above it:


\begin{verbatim}

                               1
                              / \
                             1   1
                            / \ / \
                           1   2   1
                          / \ / \ / \
                         1   3   3   1
                        / \ / \ / \ / \
                       1   4   6   4   1
\end{verbatim}
    
and so on.  

The number at each place tells you how many paths there are zig-zagging
down from the top of the triangle to that place - since each number is
the sum of the numbers above it.  

Let's think about one of these paths.  To get to the kth place in the
nth row of the triangle, the path must go to the right a total of k
times and to the left a total of (n-k) times.  Here I'm counting the
rows starting from zero.  For example, the "6" in Pascal's triangle is
at the 2nd place in the 4th row, so any path down to the "6" must go
left twice and right twice:


\begin{verbatim}

                               1
                              / 
                             1   1
                              \ 
                           1   2   1
                              /   
                         1   3   3   1
                              \   
                       1   4   6   4   1
\end{verbatim}
    
Now, the number of ways to choose k things out of n is the binomial
coefficient


\begin{verbatim}

(n choose k) = n! / k! (n-k)!
\end{verbatim}
    
so this is the number at the kth place of the nth row.  And since
each number in the triangle is the sum of those above it, we get


\begin{verbatim}

(n choose k) = (n-1 choose k) +  (n-1 choose k-1)
\end{verbatim}
    
To illustrate how these things work, you can actually build a machine
where you drop a little ball into the top of a triangle, designed so 
that the ball has a 50% chance of zigging to the left or zagging to the
right at each step of its fall.  By the time it gets to the nth row, the 
chance of its being in the kth place will be proportional to (n choose k).  

If you drop a bunch of balls in the top and catch them as they fall out
the bottom, you'll get an approximately Gaussian distribution of balls,
like this:


\begin{verbatim}

                               1
                              / \
                             1   1
                            / \ / \
                           1   2   1
                          / \ / \ / \
                         1   3   3   1
                        / \ / \ / \ / \
                       1   4   6   4   1


                               o
                               o
                           o   o   o   
                           o   o   o   
                           o   o   o   
                       o   o   o   o   o
\end{verbatim}
    
This illustrates how the famous "bell curve" shows up whenever you 
add a big bunch of independent random numbers - as made precise by
the "central limit theorem".

This stuff is old - very old.  In fact, Pascal's triangle had already
been around for centuries before he wrote his book about it in 1654.  
I think we need a more up-to-date version of Pascal's triangle for the
21st century.

So: let's suppose the little ball we drop into the machine is an
ELECTRON, and let's turn on a MAGNETIC FIELD!

In quantum mechanics, if you have a charged particle in a static
magnetic field, its wavefunction gets multiplied by a phase when 
you move it around a loop.  This phase is just


\begin{verbatim}

q = exp(iF)
\end{verbatim}
    
where F is the flux of the magnetic field though any surface bounded 
by the loop.  Here I'm working in units where Planck's constant and the 
particle's charge both equal 1.  

Suppose our particle is confined to a plane, and there's a constant
magnetic field perpendicular to this plane.   If we tile this plane 
with squares, moving the particle counterclockwise around any one of 
these squares multiplies its wavefunction by the same phase q:


\begin{verbatim}

              ---<---
             |       |
             v       ^
             |       |
              --->---
\end{verbatim}
    
Here's another way to think about it.  Let U be the operation of moving
our particle one unit across to the right, and let V be the operation of
moving it one unit upwards.  Then the operation of moving it around the
above square is


$$

U V U^{-1} V^{-1} 
$$
    
and we've just seen this equals q - that is, it acts to multiply
the particle's wavefunction by q.  So:


$$

U V U^{-1} V^{-1} = q
$$
    
This means that the operations U and V don't commute: instead, 
they "q-mute":


\begin{verbatim}

UV = qVU
\end{verbatim}
    
Now suppose we have a grid of squares and our particle takes any path
from the lower left corner to the upper right corner:


\begin{verbatim}

                ---------------
               |       |       |
               |       |       ^
               |       |       |
               |----------->---|
               |       |       |
               |       ^       |
               |       |       | 
                --->-----------
\end{verbatim}
    
The phase it acquires by the end of its journey depends on the path it 
takes.  Since 


\begin{verbatim}

UV = qVU, 
\end{verbatim}
    
its phase gets an extra factor of q each time it goes first right and
then up, as compared to going up and then right.  So compared to the
path that goes all the way up and then across, the \emph{relative} phase 
of the particle at the end of its journey will be q^{n}, where n is the
number of squares above and to the left of its path:


\begin{verbatim}

                ---------------
               |       |       |
               |   2   |   3   ^
               |       |       |
               |----------->---|
               |       |       |
               |   1   ^       |
               |       |       | 
                --->-----------
\end{verbatim}
    
In this example our particle gets a relative phase of q^{3}.

Now let's rotate our grid of squares so it look like diamonds arranged
in Pascal's triangle, with the lower right corner of our grid becoming
the top of the triangle.  And suppose we drop in a charged particle at
the top!  Then the particle can take lots of paths from the top to any
given spot, but it will get a different phase depending on which path 
it takes.  The rules of quantum mechanics say we have to add up these
phases to get the amplitude for the particle to wind up at that spot. 
This sum over paths is a simple version of what physicists call a 
"path integral".  

For example, in this new picture the above path looks like:


\begin{verbatim}

                               o
                              / 
                             o   x
                              \   
                           x   o   x
                              /       
                         x   o   x   x
                              \    
                       x   x   o   x   x
\end{verbatim}
    
and we can tell it gets a phase of q^{3}, because this shape is
made of 3 diamonds:


\begin{verbatim}

                               o
                              / \
                             o 1 x
                              \ / \
                           x   o 2 x
                              / \ / 
                         x   o 3 x   x
                              \ /
                       x   x   o   x   x
\end{verbatim}
    
If for each spot we sum these phases over all paths that reach
that spot, we get the "q-deformed Pascal's triangle":


\begin{verbatim}

                               1
                              / \
                             /   \
                            /     \
                           /       \
                          /         \
                         1           1
                        / \         / \
                       /   \       /   \
                      /     \     /     \
                     /       \   /       \                
                    1         1+q         1
                   / \        / \        / \
                  /   \      /   \      /   \
                 /     \    /     \    /     \
                /       \  /       \  /       \
               /         \/         \/         \
              1        1+q+q^{2}      1+q+q^{2 }       1
             / \         /\         /\         / \
            /   \       /  \       /  \       /   \
           /     \     /    \     /    \     /     \
          /       \   /      \   /      \   /       \
         /         \ /        \ /        \ /         \
        1    1+q+q^{2}+q^{3}    1+q+2q^{2}+q^{3}+q^{4}  1+q+q^{2}+q^{3 }     1

\end{verbatim}
    
Let's call the entry in the kth place of the nth row 


\begin{verbatim}

[n choose k]
\end{verbatim}
    
The square brackets tell us that these are "q-binomial coefficients"
instead of the ordinary ones.

Using the fact that every path to reach a spot must have come 
from the left or the right, you can show that


$$

[n choose k]  =  [n-1 choose k]  +  q^{n-k} [n-1 choose k-1]
$$
    
It helps to draw some pictures to see where that factor of q^{n-k} is
coming from... but I'll leave that as a fun little exercise for you!   
And starting with this recursion relation, it's easy to check inductively 
that:


\begin{verbatim}

                   [n]!
[n choose k] = -----------
               [k]! [n-k]!
\end{verbatim}
    
where [n]! is the q-factorial


\begin{verbatim}

[n]! = [1] [2] ... [n]
\end{verbatim}
    
and [n] is the q-integer


\begin{verbatim}

[n] = 1 + q + ... + q^{n-1}
\end{verbatim}
    
So the quantum Pascal's triangle is a lot like the ordinary one.  
In particular, the formula 


\begin{verbatim}

                   [n]!
[n choose k] = -----------
               [k]! [n-k]!
\end{verbatim}
    
makes it obvious that this triangle is symmetric around its axis, just
like the ordinary one, even though I defined it in way that obscured
this fact a bit.  This symmetry gives us the mirror-image recursion
relation:


$$

[n choose k]  =  q^{k} [n-1 choose k]  +  [n-1 choose k-1]
$$
    
But now let's see if you've really been paying attention.  Do you really
understand these q-binomial coefficients?  If so, you should instantly
know what the coefficient of any power of q in [n choose k] signifies. 
Say, the coefficient of q^{i}.  Think about it.

Right!  It's the number of paths to the kth place of the nth row
of Pascal's triangle that create a shape like this with i diamonds:


\begin{verbatim}

                               o
                              / \
                             o   x
                              \ / \                 n = 4
                           x   o   x                k = 2
                              / \ /                 i = 3
                         x   o   x   x
                              \ /
                       x   x   o   x   x
\end{verbatim}
    
But these shapes are famous!  They're usually drawn like this:



\begin{verbatim}

          ---------------
         |       |       |
         |       |       |
         |       |       |
         |---------------
         |       |       
         |       |       
         |       |       
          -------
\end{verbatim}
    

and they're called "Young diagrams" - or, with more historical
accuracy, "Ferrers diagrams".  So, the coefficient of q^i in
[n choose k] is also the number of Young diagrams with i boxes, k
columns and n-k rows.

Now, Young diagrams show up all over the place in mathematics.  I
described a few of their most fundamental applications in "<A HREF
= "week157.html">week157</A>".  One thing I \emph{didn't} go into is
their relation to Grassmannians.  But we're pretty well-positioned to do
that now, so let's give it a shot.

The Grassmannian Gr(n,k) is the set of all k-dimensional subspaces 
of an n-dimensional vector space.  This makes sense over any field.
However, it's particularly fun to work over the field with q elements, 
where q is any prime power - because then our Grassmannian has [n choose k] 
elements!  

In fact, we already saw this in "<A HREF =
"week184.html">week184</A>".  But now I want to give a proof that
uses Young diagrams.  This will forge yet another link between the two
flavors of q-mathematics: the sort where q is a unit complex number, and
the sort where q is a prime power.  So far this week we've been doing
quantum mechanics and thinking of q as a unit complex number.  Now we'll
turn into algebraists, take what've done, and apply it when q is a power
of a prime number!

In "<A HREF = "week184.html">week184</A>" I showed that we can
read off a lot of information about the Grassmannian Gr(n,k) from the
q-binomial coefficient [n choose k].  For example,


$$

[4 choose 2] = 1 + q + 2q^{2} + q^{3} + q^{4}
$$
    
and if we define our Grassmannian over the field F, we have

$$

     Gr(4,2) = 1 + F + 2F^{2} + F^{3} + F^{4}

$$
    
meaning that Gr(4,2) is a disjoint union of a point, a copy of the field
F, 2 copies of F^{2}, a copy of F^{3}, and a copy of
F^{4}.  Each copy of F^{i} is called an
"i-cell" - or in this context a "Schubert cell of
dimension i", because there may be many ways to decompose a space
into cells, but there's a particularly nice one for Grassmannians.

If F is the real or complex numbers, the cells F^{i} are actually open
balls, and we can use this to study the topology of the Grassmannian. 
But if F is the field with q elements, we can use the cell decomposition
to work out the \emph{cardinality} of the Grassmannian.  For example, the
number of points in Gr(4,2) is


\begin{verbatim}

|Gr(4,2)| = |1  +  F  + 2F^{2}   +  F^{3} +   F^{4}|

          = |1| + |F| + 2|F|^{2} + |F|^{3} + |F|^{4}

          =  1  +  q  +  2q^{2}  +  q^{3}  +  q^{4}
 
          = [4 choose 2]
\end{verbatim}
    
I already said all this in "<A HREF =
"week184.html">week184</A>".  But now, I want to use Young diagrams
to get ahold of these cell decompositions.

The answer should be staring us in the face.  We know that the
coefficient of any power of q in a q-binomial coefficient is the number
of Young diagrams of a certain sort.  And we also know it's the number of
Schubert cells of a certain sort.  There should be some way to see this
directly.

To do this, we just have to find a way to decompose the Grassmannian Gr(n,k)
into cells, where each i-cell corresponds to a Young diagram with i boxes, 
k rows and n-k columns.

The idea is simple.  A point in Gr(n,k) is a k-dimensional subspace of
the vector space F^{n}.  
We can describe such a thing by writing a list of
row vectors that form a basis for this subspace.  This gives a matrix.
Of course, the same subspace can have lots of different bases.  But we
can always find a nice "standard" basis where our matrix is in
"row echelon form".

This is one of those things you're supposed to learn in linear algebra... 
but if you forget how it goes, don't worry.  The idea is just to take a
matrix and keep subtracting multiples of any row from the rows below it,
and stuff like that, until your matrix looks something like this:
               

\begin{verbatim}

               1 0 X 0 X X X 0 X X            n = 10     (10 columns)
               0 1 X 0 X X X 0 X X            k = 4      (4 rows)
               0 0 0 1 X X X 0 X X            i = 19     (19 X's)
               0 0 0 0 0 0 0 1 X X
\end{verbatim}
    
Here the X's are arbitrary numbers.  


The set of matrices of a given shape like this is isomorphic to
F^{i}, where i is the number of X's.  So, each shape gives us an
i-cell in our Grassmannian!  To finish the job, we just need to think of
each "shape" as being a Young diagram with i boxes, k rows and
n-k columns.

And that's easy: we just remove the 0's and 1's from the above picture
and make a Young diagram out of the X's:
               

\begin{verbatim}

                     X X X X X X              k = 4      (4 rows)
                     X X X X X X            n-k = 6      (6 columns)
                       X X X X X              i = 19     (19 boxes)
                             X X
\end{verbatim}
    
Voila!

Just for the heck of it, I'll work out the Schubert cell decomposition
of Gr(4,2) by this technique.   I'll write out the various shapes of 
row echelon form for matrices with 4 columns and 2 rows, and next to 
them the corresponding Young diagrams and the kind of i-cells we get:


\begin{verbatim}

0 0 1 0                           0-cell
0 0 0 1         
  
0 1 X 0             X             1-cell
0 0 0 1                 

1 X X 0           X X             2-cell
0 0 0 1     

0 1 0 X             X             2-cell
0 0 1 X             X

1 X 0 X           X X             3-cell
0 0 1 X             X

1 0 X X           X X             4-cell
0 1 X X           X X

\end{verbatim}
    
This is nicely consistent with what we already know:


$$

[4 choose 2] = 1 + q + 2q^{2} + q^{3} + q^{4}
$$
    
Okay, enough of this... now for some references.  

I've already given lots of Young diagram references in "<A HREF =
"week157.html">week157</A>", and lots of references on q-binomial
coeffients and the like in "<A HREF =
"week183.html">week183</A>" - "<A HREF =
"week185.html">week185</A>".  So, I'll just give some references to
Pascal's triangle and this "UV = qVU" business.

Here's the best place to learn the history of Pascal's triangle:

1) A. W. F. Edwards, Pascal's Arithmetical Triangle, Charles Griffin
and Co., London, 1987.

You'll see that the basic idea of Pascal's triangle goes back to
Mersenne in 1636, and Tartaglia in 1556, and the Hindu mathematician
Bhaskara in 1150, and the Jain mathematician Mahavira in 850... and
so on into the mists of time.  

You'll also see that around 1655, Wallis came up with his wonderful
formula:


\begin{verbatim}

       2 4 4 6 6 8 8 
\pi /4 = - - - - - - - ...
       3 3 5 5 7 7 9 
\end{verbatim}
    
by relating Pascal's triangle to the integral for the area of a 
quarter-circle!  His method was ingenious and daring: it consisted
of taking an integral formula for binomial coefficients, extrapolating 
it to guess that


$$

4/\pi  = (1 choose 1/2)
$$
    
and then using properties of Pascal's triangle to express the right-hand
side as an infinite product!  One reason I like this is that I want
to categorify the number \pi .  Let me explain....

Jim Dolan and I have a way to assign non-integral cardinalities to
groupoids (see "<A HREF = "week147.html">week147</A>").  The
cardinality of the groupoid of finite sets is e, and this actually
explains lots of things about e once you understand it.  So, I'm always
on the lookout for a really nice groupoid whose cardinality is related
to \pi .  It's easy to find groupoids that do the job; the hard part is
finding one that does so in an enlightening way.  Of course \pi  is a
subtler number than e, so this may be hard.

The philosopher David Corfield has gotten interested in this challenge.
Recently he took a crack at it by looking for a structure type whose
generating function was 
 

\begin{verbatim}

                  1          1.3          1.3.5
arcsin x =  x +  --- x^{3}  +  -----  x^{5} +  -------  x^{7} + ...
                 2.3        2.4.5        2.4.6.7
\end{verbatim}
    
and evaluating this at the 1-element set to get a groupoid whose
cardinality is \pi /4.  (If this is utterly mystifying, see "<A HREF = "week185.html">week185</A>" 
and the references there.)  I've discussed this with him, and I also talked 
about it with my friend the combinatorist Bill Schmitt, and we made 
a little progress, but not enough.  So, right now I like the idea
of going back to Wallis' original formula for \pi , and seeing how
it relates to Pascal's triangle, and seeing if I can get anywhere
with that!  
Embarrassingly, I don't know how the formula for arcsin(1)
is related to Wallis' formula, even though they look sort of similar.
Next:
If you like the "UV = qVU" idea, you need to study the "noncommutative
torus".  This is a name for the C*-algebra generated by two unitaries U
and V satisfying UV = qVU.  The quantum Pascal triangle is built right
in, because


$$

(U + V)^{n} = sum_{k} [n choose k] U^{k} V^{n-k}
$$
    
As we've seen, the noncommutative torus shows up naturally when we have
a charged particle on the plane in a magnetic field.  Jean Belissard has
used this to relate the fractional quantum Hall effect to the K-theory
of the noncommutative torus:

2) Jean Bellisard, K-theory of C*-algebras in solid state physics, in
Lecture Notes in Physics vol. 237, Springer, Berlin, 1986, pp. 99-156.

Connes has also studied these matters:

3) Alain Connes, Noncommutative Geometry, Academic Press, New
York, 1994.  

More recently, string theorists have done a bunch of physics on the
noncommutative torus!  The reason is that string theory includes a
2-form field "B" which is similar in some ways to the magnetic field.
For an overview of this with lots of references try:

4) Richard Szabo, Quantum field theory on noncommutative spaces,
available as <A HREF = "http://xxx.lanl.gov/abs/hep-th/0109162">hep-th/0109162</A>.  

On the other hand, if you're more of a pure mathematician you
might like this:

5) Marc Rieffel, Noncommutative tori: a case study of noncommutative
differential manifolds, in Geometric and topological invariants of
elliptic operators Contemp. Math. 105, American Mathematical Society,
1990, pp. 191-211.

By the way, the concept of "noncommutative manifold" has not so far
received a precise definition, but someone told me Connes is working on
such a definition and is all excited about it.  That sounds promising!
Whatever the definition, the noncommutative torus must be an example.

\par\noindent\rule{\textwidth}{0.4pt}
<em>This Table has truly exceptional and admirable properties;
for besides concealing within itself the mysteries of Combinations,
as we have seen, it is known by those expert in the higher parts of
Mathematics also to hold the foremost secrets of the whole of the
rest of the subject.</em> - James Bernoulli, 1713.

\par\noindent\rule{\textwidth}{0.4pt}

% </A>
% </A>
% </A>
