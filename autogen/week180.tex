
% </A>
% </A>
% </A>
\week{April 19, 2002 }

First, a news flash: they may have found a quark star... or two!

In case you're wondering, a "quark star" is a hypothetical entity 
smaller and denser than a neutron star, where the pressure is so 
high that the neutrons get crushed into a mess of quarks.  Nobody 
really knows if this is possible without the darn thing collapsing 
all the way into a black hole.  However, if it happened, a bunch 
of the down quarks in the neutron would turn into strange quarks, 
which are somewhat more massive, but energetically favored nonetheless 
in situations like this where the Pauli exclusion principle reigns 
supreme.  Folks refer to this phenomenon as "strangeness enhancement".
It sounds like some sort of surgical operation undergone by Michael
Jackson, doesn't it?   But anyway, for this reason, quark stars are 
also known as "strange stars".  

Back in "<A HREF = "week117.html">week117</A>" I described evidence 
for strangeness enhancement
in quark-gluon plasma experiments at Brookhaven and elsewhere, 
but it would be really cool to see it in nature.   People have been 
looking for quark stars for some time, with no success, but NASA
has just announced the discovery of two entities that \emph{might} be 
quark stars:

1) Cosmic X-rays reveal evidence for new form of matter, 
<A HREF = "http://www1.msfc.nasa.gov/NEWSROOM/news/releases/2002/02-082.html">
http://www1.msfc.nasa.gov/NEWSROOM/news/releases/2002/02-082.html</A>


The Chandra X-ray observatory (see "<A HREF = "week143.html">week143</A>" for info on this marvelous
satellite) has seen two stars, romantically entitled RXJ1856 and 3C58, 
that look sort of like neutron stars... but apparently too small or 
too cool to \emph{be} neutron stars!  There's always the possibility that 
something else is going on, but folks are thinking they look like 
strange stars.  Stay tuned.

Okay, now for some math.  First some news on topos theory, and then I'll 
return to the theme of "<A HREF = "week178.html">week178</A>": Lie 
groups and geometry... leading up to a taste of twistors. 

Peter Johnstone is a category theorist who can often be seen playing
backgammon in the common room of the Department of Pure Mathematics and
Mathematical Statistics at Cambridge University.  He also selects the
wines at St. Johns.  But he must have been working dreadfully hard for
the last decade or so, because he's produced a book of mammoth proportions:

2) Peter Johnstone, Sketches of an Elephant: a Topos Theory Compendium,
Cambridge U. Press.  Volume 1, comprising Part A: Toposes as Categories, 
and Part B: 2-categorical Aspects of Topos Theory, 720 pages, to appear 
in June 2002.  Volume 2, comprising Part C: Toposes as Spaces, and Part D: 
Toposes as Theories, 880 pages, to appear in June 2002.  Volume 3, 
comprising Part E: Homotopy and Cohomology, and Part F: Toposes as 
Mathematical Universes, in preparation.

I can't wait to dig into this.  A topos is a kind of generalization of
the universe of set theory that we all know and love, but topos theory
is really a wonderful way to unify and generalize vast swathes of
mathematics - you could say it's the way that logic and topology merge
when you take category theory seriously.  I've really just begun to get
a glimmering of what it's all about, so I'm curious to see Johnstone's
overall view of the subject.

If you're wondering what a topos actually \emph{is}, and you're too impatient
to wait for Johnstone's books to come out, I suggest that you start with
my quick online summary:

3) John Baez, Topos theory in a nutshell, <A HREF =
"http://math.ucr.edu/home/baez/topos.html">http://math.ucr.edu/home/baez/topos.html</A>

and then try the books I recommended in "<A HREF = "week68.html">week68</A>", along with
this one:

4) Colin McLarty, Elementary Categories, Elementary Toposes, 
Oxford University Press, Oxford, 1992.

which I only learned about later, when McLarty sent me a copy.  I wish
I'd known about it much sooner: it's very nice!  It starts with a great
tour of category theory, and then it covers a lot of topos theory,
ending with a bit on various special topics like 
the "effective topos",
which is a kind of mathematical universe where only effectively
describable things exist - roughly speaking.

Now, in "<A HREF = "week178.html">week178</A>" I described some 
things James Dolan and I were learning
about Lie groups and geometry.  In the meantime we've learned so much
that I sort of despair of conveying it all... beautiful, wonderful 
stuff!  We're even beginning to understand the theory of "buildings",
which I had long considered an impenetrable bastion of incomprehensibility.

But instead of rhapsodizing, let me dive in and explain as much as
I can.  Last time I explained that every complex simple Lie group G gives
rise to a generalization of projective geometry.  When we take G = SL(n,C)
we get ordinary projective geometry, and I focussed on this case, but
I described how things work in general.  Today I want to dig a little
deeper into the general theory then consider a bunch of examples, 
leading up to Penrose's theory of twistors.

First, remember how this game goes.  Every complex simple Lie group G has 
a bunch of maximal solvable subgroups, all basically the same as each
other - so people pick one and call it "the Borel subgroup", or B for
short.  When G = SL(n,C) we can take B to be the subgroup of upper
triangular matrices.  When doing geometry with some symmetry group G,
any subgroup should be thought of as the group of transformations that
preserves some sort of "figure" - some geometrical object.  The 
importance of the Borel subgroup is that it 
preserves a "maximal flag".  For G = SL(n,C)
acting on complex projective space, this is just:


\begin{verbatim}

a point lying on a line lying on a plane lying on a 3-space lying on...
\end{verbatim}
    
For other complex simple groups we'll get other concepts of "maximal flag",
which I'll describe later.

Having chosen a Borel subgroup B, there is a finite set of subgroups
containing B and smaller than G - people call these "parabolic
subgroups".  These preserve all the various smaller kinds of flag, 
which in the SL(n,C) case are things like 


\begin{verbatim}

a point lying on a plane lying on a 5-space
\end{verbatim}
    
or simply 


\begin{verbatim}

a line
\end{verbatim}
    
For any parabolic subgroup P, the quotient space G/P is called 
a "flag manifold", since it's the space of all  flags of the given type.

The parabolics range in size from B to the "maximal parabolic subgroups".
The bigger the subgroup, the less it preserves, so the maximal parabolics
preserve the simplest flags, like "a point", "a line", "a plane", and so
on.  In this case the flag manifold G/P is usually called a "Grassmannian".  

Now, the cool part is that you can read off the parabolic subgroups from
the Dynkin diagram of a simple Lie group: they correspond to subsets
of the dots!  The maximal parabolics correspond to the dots themselves.
For SL(n,C) it works like this... I'll illustrate with the case n = 4:


\begin{verbatim}

                           o------o------o
                        points  lines  planes       
\end{verbatim}
    
So, to pick out a flag manifold, you just mark the dots you want.  
For example, 


\begin{verbatim}

                           x------o------x
                        points  lines  planes       
\end{verbatim}
    
gives the parabolic P such that SL(4,C)/P is the space of all "points
lying on a plane" in CP^{3}.  As I explained earlier, 
this P is the 
subgroup of SL(4,C) consisting of all matrices of the form


\begin{verbatim}

                               * * * *
                               0 * * *          
                               0 * * *
                               0 0 0 *
\end{verbatim}
    
If you look at the pictures in "<A HREF = "week178.html">week178</A>", you should be able to figure
out the recipe for getting this subgroup from a subset of dots in the 
Dynkin diagram, at least in the SL(n,C) case.

Even better, this game lets you get all the finite-dimensional irreducible 
representations of your complex simple group G.  I'll say how it goes
without explaining why it works.  To get an irrep, just label each
Dynkin diagram dot with a natural number!  The subset of dots labelled
by \emph{nonzero} numbers determines a parabolic subgroup P.  The numbers
themselves pick out a complex line bundle over G/P.  The group G acts
on G/P, of course, and it also acts on this line bundle.  Now, G/P is 
always a complex manifold since G and P are complex, so it makes sense
to talk about \emph{holomorphic} sections of this line bundle.  The space
of these forms a finite-dimensional irrep of G!  

To really understand this deeply, you should learn a bit about geometric
quantization.  However, let's just assume it works and see what happens
in some examples.  

First consider G = SL(n,C).  Here we've already seen that the maximal 
parabolics are the subgroups preserving various obvious figures in 
complex projective space:


\begin{verbatim}

                           o------o------o
                        points  lines  planes       
\end{verbatim}
    
The irrep corresponding to this numbering:


\begin{verbatim}

                           1------0------0
\end{verbatim}
    
is the obvious representation of SL(n,C) on C^{n}.  This irrep:


\begin{verbatim}

                           0------1------0
\end{verbatim}
    
is the obvious rep of SL(n,C) on the 2nd exterior power of C^{n} - 
or in physics lingo, rank two antisymmetric tensors.  This irrep:


\begin{verbatim}

                           0------0------1
\end{verbatim}
    
is the obvious rep of SL(n,C) on the 3rd exterior power of C^{n}.
And so on, if there are more dots.   Note what we're really saying 
here: if you take the Grassmannian of all j-dimensional subspaces
in C^{n}, there's a god-given complex line bundle on it whose space
of holomorphic sections is the jth exterior power of C^{n}.  

In general, the irreps we get by labelling one dot with a 1 and the
rest with 0 are the most exciting: they're called the "fundamental"
reps.  In math jargon, they generate the representation ring of G.  Even
better, there's a simple recipe for taking a Dynkin diagram with dots
labelled by numbers and finding the corresponding irrep inside a
tensor product of symmetrized tensor powers of these fundamental reps,
where the numbers labelling the dots tell you which powers to use.  
For SL(n,C) this is just the theory of Young diagrams, which I discussed 
in "week 157".  So, we're just generalizing the heck out of that.

Even if you don't understand what I just said, you can rest assured
knowing that we can completely master \emph{all} the irreps of G once we
figure out the fundamental ones.  So, we'll focus on those.

We've more or less beat SL(n,C) to death, so let's see what happens with
some other simple Lie groups... for example, the groups Spin(n,C).  If
you don't know these guys, first think about SO(n,C).  This is the group
of all linear transformations of C^{n} preserving the symmetric bilinear
form

x.y = x_{1} y_{1} + ... + x_{n} y_{n}

Unfortunately SO(n,C) is not simply connected, so not all reps of its
Lie algebra give reps of the group.  So, to get group representations
from ways of labelling the Dynkin diagram by numbers, we need to work
with its double cover, the "spin" group Spin(n,C).  

You may be more familiar with the compact real forms of these groups.
The compact real form of SO(n,C) is the good old rotation group in n
dimensions, SO(n).  The compact real form of Spin(n,C) is the double
cover of SO(n), called Spin(n).  The irreps of Spin(n,C) give unitary
irreps of Spin(n), so you can think about them that way if you prefer.

The Dynkin diagram of Spin(n,C) looks really different depending on 
whether n is even or odd.  It takes a while for the pattern to
become clear - it's obscured by lots of delightful coincidences 
in low dimensions.  I'll work through these low dimensions and then
say the general pattern.  If you're the sort who can't stand reading
long lists of facts until you've seen the pattern they fit, jump 
ahead to where I talk about Spin(9,C) and Spin(10,C).  I'm gonna
climb my way up there slowly, taking my time to smell the flowers.

The Dynkin diagram of Spin(3,C) is just a single dot:


\begin{verbatim}

                      o
\end{verbatim}
    
just like the Dynkin diagram for SL(2,C).  That's because they're
isomorphic:

Spin(3,C) = SL(2,C).

The fundamental representation corresponding to the single dot in the
Dynkin diagram is called the "spinor" representation of Spin(3,C): 
it's just the obvious rep of SL(2,C) on C^{2}.
This fact is crucial for understanding spin-1/2 particles in 3d space.


The Dynkin diagram of Spin(4,C) is two dots, not connected by
an edge:


\begin{verbatim}

                      o



  
                      o
\end{verbatim}
    
just like the Dynkin diagram for SL(2,C) x SL(2,C).  That's because 
they're isomorphic:


\begin{verbatim}

Spin(4,C) = SL(2,C) x SL(2,C).
\end{verbatim}
    
The fundamental reps coresponding to the two dots are called 
the "left-handed" and "right-handed" spinor 
representations of Spin(4,C): they're just the obvious reps of 
SL(2,C) x SL(2,C) on C^{2}.  This fact is crucial for
understanding spin-1/2 particles in 4d spacetime. 

The Dynkin diagram of Spin(5,C) is two dots connected by a
double edge:



$$

                      o===>===o 
$$
    
For an explanation of the double edge and the arrow 
see "<A HREF = "week62.html">week62</A>"
and "<A HREF = "week64.html">week64</A>", 
where I also explained that this Dynkin diagram is the
same as that of Sp(4,C), the group of transformations preserving
a symplectic structure on C^4.  That's because these groups are isomorphic:

Spin(5,C) = Sp(4,C).

The fundamental rep corresponding to the left dot in the Dynkin diagram
comes from the obvious rep of SO(5,C) on C^{5} - what physicists would
call the "vector" rep.  The fundamental rep corresponding to the right
dot comes from the obvious rep of Sp(4,C) on C^{4} - it's called the
"spinor" rep of Spin(5,C).  This would be fundamental for
studying spin-1/2 particles in 5-dimensional spacetime if anyone were
interested... but not many people are.  

The Dynkin diagram of Spin(6,C) has three dots:

\begin{verbatim}

                          o
                         /
                        /
                       /
                      o
                       \
                        \
                         \
                          o
\end{verbatim}
    
This is the same as that of SL(4,C), though I've drawn it differently.
That's because these groups are isomorphic:

Spin(6,C) = SL(4,C).

The fundamental rep corresponding to the left dot comes from the 
obvious rep of SO(6,C) on C^{6} - the "vector" rep again.  
The reps corresponding to the other dots are the left- and right-handed
spinor reps of Spin(6,C), coming from the obvious rep of SL(4,C)
on C^{4} and its dual.  This is fundamental for understanding
spin-1/2 particles in 6-dimensional space - for example, the 6 extra
curled-up dimensions in string theory.  And as we'll see, it's also
basic to Penrose's theory of twistors!


At this point we're done with all the cute isomorphisms, so let us line
them up and admire them before bidding them farewell:


\begin{verbatim}

Spin(3,C) = SL(2,C)
Spin(4,C) = SL(2,C) x SL(2,C)
Spin(5,C) = Sp(2,C)
Spin(6,C) = SL(4,C).
\end{verbatim}
    
They give rise to isomorphisms of their maximal compact subgroups, 
so let's say goodbye to those too:


\begin{verbatim}

Spin(3) = SU(2)
Spin(4) = SU(2) x SU(2)
Spin(5) = Sp(2)
Spin(6) = SU(4).
\end{verbatim}
    
Sometime we should return and learn to know them better... they've
barely begun to display their many charms!  But today we must 
sail on to higher dimensions....

The Dynkin diagram of Spin(7,C) has three dots:



$$

                     o-------o===>===o 
$$
    
The fundamental rep corresponding to the left dot comes from
the vector rep of SO(7,C) on C^{7}.  The rep corresponding to the
middle dot is the second exterior power of the vector rep.  The
rep corresponding to the right dot is the spinor rep, which is
no longer so easy to describe without using Clifford algebras -
see "<A HREF = "week93.html">week93</A>" or "<A HREF = "week105.html">week105</A>" for more about those.

The Dynkin diagram of Spin(8,C) has four dots:


\begin{verbatim}

                                  o
                                 /
                                /
                               /
                      o-------o
                               \
                                \
                                 \
                                  o
\end{verbatim}
    
The fundamental rep corresponding to the left dot comes from
the vector rep of SO(8,C) on C^{8}.  The middle dot corresponds
to the second exterior power of the vector rep.  The top and
bottom dots correspond to the left- and right-handed spinor reps.
Like the vector rep, these are also 8-dimensional.  This coincidence
arises from the symmetry of the diagram, which is called "triality".

I've said a lot about triality 
in "<A HREF = "week61.html">week61</A>", 
"<A HREF = "week91.html">week91</A>" and elsewhere,
but right now it's just a distraction - I'm trying to get you to
see the pattern of Spin(n,C) Dynkin diagrams, and I'm hoping that
by now it's apparent: an alternation between odd and even dimensions,
and so on....

But just to be clear, let's look at SO(n,C) for n = 9 and n = 10, which 
illustrate the pattern even more clearly.   I'll also explain how
how it's all related to incidence geometry.

The Dynkin diagram of SO(9,C) has 4 = (9-1)/2 dots:



$$

                 o-------o-------o====>====o 
$$
    
The fundamental rep corresponding to the ith dot is the
ith exterior power of vector rep, \emph{except} for the last dot, 
which corresponds to the spinor rep.  

To see how the dots correspond to different types of geometrical figures
in some incidence geometry, first remember that we're starting with 
C^{n} equipped with a symmetric bilinear form:

x&sdot;y = x_{1} y_{1} + ... + x_{n} y_{n}

This is really different than R^{n} with its usual inner product, since
it's perfectly possible for a vector to have x&sdot;x = 0, and we can even
get big subspaces that are orthogonal to themselves.  A subspace of 
C^{n} is called "isotropic" if all vectors in this 
subspace are orthogonal to 
each other with respect to this form.  


The idea of a subspace orthogonal to itself seems really weird at first!
If you've never thought about this, you should probably skip ahead to
the "addendum" at the end of this article, where I explain it in more
detail.  It's closely related to the fact that lightlike vectors in
Minkowski spacetime are always orthogonal to themselves.  In other
words, they have x&sdot;x = 0.

To construct an incidence geometry for SO(n,C) and make it as similar to
projective geometry as possible, we work not with C^{n} but with the
subspace of CP^{n-1} coming from vectors in C^{n} with 
x&sdot;x = 0.
Algebraic geometers call this subspace a "quadric".  
In physics it arises naturally from taking (n-2)-dimensional
Minkowski spacetime, compactifying it in a certain way, and then 
complexifying it - we'll talk about this more later!
Inside this quadric there are various types of geometrical figures:



$$

                 o-------o-------o====>====o 
            points     null     null      null
                      lines    planes    3-spaces
$$
    
A "point" in the quadric is really a 1-dimensional isotropic subspace 
of C^{n}; a "null line" is a 2-dimensional isotropic 
subspace, and 
so on.  We can talk about a point lying on a line, or a line lying on a plane,
and they mean the obvious things.  This gives the incidence geometry
associated to Spin(n,C).

Putting together everything I've said so far: for n odd, the ith dot in
the Dynkin diagram of Spin(n,C) corresponds to a maximal parabolic P
such that Spin(n,C)/P is the manifold consisting of all isotropic
i-dimensional subspaces in C^{n} - or in other words, 
all null (i-1)-spaces in the corresponding quadric.  And this manifold, 
called an "orthogonal Grassmannian", has a complex line 
bundle on it whose space of holomorphic sections is the ith fundamental 
rep of Spin(n,C).

For n even, let's look at SO(10,C).

The Dynkin diagram of SO(10,C) has 5 = 10/2 dots:
   

\begin{verbatim}

                                    o 
                                   /
                                  /
                                 /
                 o------o-------o 
                                 \
                                  \
                                   \
                                    o
\end{verbatim}
    
The fundamental rep corresponding to the ith dot is the 
ith exterior power of the vector rep, \emph{except} for the last 
two dots, which correspond to the left- and right-handed
spinor reps.

In the language of incidence geometry, the dots again 
correspond to different types of figures in a quadric:



\begin{verbatim}

                                    o left-handed 4-spaces
                                   /
                                  /
                                 /
                 o------o-------o  null planes
              points   null      \
                      lines       \
                                   \
                                    o right-handed 4-spaces 
\end{verbatim}
    
The big difference from the odd-dimensional case is that there are two
kinds of spaces of the highest dimension listed, and we leave out the
next-highest dimension.  In our example we get:

<ul>
<li>
\emph{points} in the quadric, which are 
1-dimensional isotropic subspaces of C^{10}
</li>
<li>
\emph{null lines} in the quadric, which are 
2-dimensional isotropic subspaces of C^{10}
</li>
<li>
\emph{null planes} in the quadric, which are 
3-dimensional isotropic subspaces of C^{10}
</li>
<li>
\emph{left-handed 4-spaces} in the quadric, which are 
left-handed 5-dimensional subspaces of C^{10}
</li>
<li>
\emph{right-handed 4-spaces} in the quadric, which are 
right-handed 5-dimensional subspaces of C^{10}
</li>
</ul>

But what are these left- and right-handed subspaces?  The answer involves
the Hodge star operator, so if you don't know what that is, skip this
paragraph, because it will only make matters worse!  Any oriented
p-dimensional subspace of C^{10} determines a p-form w, namely its
volume form.  If you hit this with the Hodge star operator, you get a
(10-p)-form *w which corresponds to the orthogonal complement of your
subspace.  In particular, the Hodge star operator maps 5-forms to
5-forms, and satisfies

** = -1

This means that its eigenvalues are i and -i.   Thus there are
"self-dual" 5-forms with 

*w = iw 

and "anti-self-dual" ones with 

*w = -iw,

which give two kinds of 5-dimensional subspaces of C^{10}
that are their
own orthogonal complement: the so-called "right-handed" 
and "left-handed" 
ones.  There's nothing special about the number 10 here; any even number 
n will do, though we should leave out the factor of "i" in the above 
formulas when n is a multiple of 4, since then the square of the Hodge 
star operator on n/2-forms is 1 instead of -1.

Okay, that pretty much concludes my story for Spin(n,C).  I could
do some other examples, but we're probably both getting worn out;
if you want, you can read about them in section 23.3 of this book:

5)  William Fulton and Joe Harris, Representation Theory - a First
Course, Springer Verlag, Berlin, 1991.

So instead, let me conclude with a few remarks about twistors.
taken from here:

6) Robert J. Baston and Michael G. Eastwood, The Penrose Transform:
its Interaction with Representation Theory, Clarendon Press, Oxford,
1989.

The field equations for massless particles like photons are conformally
invariant.  The group SO(2,4) acts as conformal transformations of 4d
Minkowski spacetime.  To be precise, we should admit that some of these
are just partially defined, like conformal inversion:

x |\to  x/x&sdot;x

However, they become everywhere defined if we switch to a slightly
bigger space, the "conformal compactification" of Minkowski spacetime.

The great realization of Roger Penrose was that it's nice to go even
further and \emph{complexify} this conformal compactification, getting a
4-dimensional complex manifold M with a \emph{complex} metric.  Minkowski
spacetime sits inside this wonderful space M just like the real line
sits inside the Riemann sphere.  A lot of physics becomes easier on M,
just like a lot of math is easier to do on the Riemann sphere than on
the real line.

Now, since SO(2,4) is a real form of SO(6,C), the whole group 
SO(6,C) acts as symmetries of M.  Of course the double cover 
Spin(6,C) also acts on M, so let's use that.  Here's the cool part:

M = Spin(6,C)/P 

where P is the maximal parabolic corresponding to this dot on
the Dynkin diagram for Spin(6,C):


$$

                          o
                         /
                        /
                       /
     we live here \to   o
                       \
                        \
                         \
                          o
$$
    
We've seen this diagram before.  In the language of incidence geometry,
the dots correspond to different figures in a quadric:



\begin{verbatim}

                          o left-handed planes
                         /
                        /
                       /
               points o 
                       \
                        \
                         \
                          o right-handed planes
\end{verbatim}
    
so points of M are just points of this quadric!

If you unravel some of the definitions, this says that

M = {1-dimensional isotropic subspaces of C^{6}},

so in physics lingo, M is the space of lightlike lines through the
origin in C^{6}.... but remember, these are \emph{complex} lines.

So far, this stuff actually works in any dimension: the space of 
1-dimensional isotropic subspaces of C^{n} is the same
as what you get by complexifying the conformal compactification of 
(n-2)-dimensional Minkowski spacetime, and so on.

But now we can use one of those charming coincidences:

Spin(6,C) = SL(4,C)!

This means we can also write

M = SL(4,C)/P

where now we think of P as a parabolic in SL(4,C).  Let's see
what M looks like in these terms.  SL(4,C) acts on CP^{3}, and
we've seen that the dots in the Dynkin diagram for SL(4,C) 
correspond to these different types of geometrical figures in CP^{3}:


\begin{verbatim}

                          o points
                         /
                        /
                       /
                lines o 
                       \
                        \
                         \
                          o planes

\end{verbatim}
    
So, we get yet another description of our marvelous spacetime:

M = {lines in CP^{3}}

or if you prefer:

M = {2-dimensional subspaces of C^{4}}

Whew!  What's the point?  Well, these descriptions of the
complexification of conformally compactified Minkowski spacetime let
Penrose use incidence geometry methods to solve conformally invariant
field equations, like Maxwell's equations or the Yang-Mills equations.

But what's a twistor?  That's easy: it's just a spinor for Spin(6),
either left-handed or right-handed.  In other words, twistors are
the fundamental reps corresponding to these dots on the Dynkin
diagram:


$$

                          o <--- left-handed twistors live here
                         /
                        /
                       /
                      o
                       \
                        \
                         \
                          o <--- right-handed twistors live here
$$
    
In the language of incidence geometry, these dots correspond to the two
sorts of null planes in M.  Penrose likes to think of these null planes
as more fundamental than points....

There's a lot more to say, but I'll stop here!  If you want more,
try this:

7) S. A. Huggett and K.P. Tod, An Introduction to Twistor Theory,
Cambridge U. Press, Cambridge, 1994.

\par\noindent\rule{\textwidth}{0.4pt}
<strong>Addendum:</strong> 
Someone who prefers to remain anonymous asked me to give some examples
of "isotropic" subspaces of C^{n}.  I really should have done this
earlier, because isotropic subspaces seem very mysterious before you've
seen them, but very simple afterwards.  They have a beautiful connection
with special relativity, especially the geometry of \emph{light}.


So, let me give some examples.  But since complex numbers are weird, let's
start with R^{n} 
equipped with a metric of some signature or other, and look
at the isotropic subspaces in there.  An isotropic subspace is just a
vector subspace where all vectors are orthogonal to each other.  This
is the same as a subspace in which all vector have x&cdot;x = 0 - or in 
physics lingo, one where all vectors are \emph{lightlike}.


For starters consider good old Minkowski space, R^{3,1}.  This has 3
space directions and 1 time direction, and it has a bunch of 1-dimensional
isotropic subspaces.  Why?  Simple: these are just light rays through the 
origin.


Are there any 2-dimensional isotropic subspaces in Minkowski spacetime?
No!  To find one of these, we'd need two light rays through the origin
that were orthogonal to each other.  And this is impossible, basically
because all lightlike vectors have a nonzero time component.  To find
two orthogonal light rays, we'd need to have two different time directions!


So, in R^{3,1} the biggest isotropic subspaces are 1-dimensional.
But if we had a spacetime like R^{2,2}, with two space directions
and two time directions, we could find 2-dimensional isotropic subspaces.
For example, if the metric on R^{2,2} looks like this:


(x,y,s,t).(x',y',s',t') = xx' + yy' - ss' - tt'


then here are two lightlike vectors that are orthogonal to each
other:


(1,0,1,0)


and


(0,1,0,1).


Since they are orthogonal, every linear combination of them is
lightlike as well.   So, these vectors span a 2d isotropic subspace.


Hopefully you get the picture now: to get an n-dimensional isotropic
subspace in R^{p,q} we need at least n time dimensions and at least
n space dimensions.  So, there will be isotropic subspaces of dimensions
going from zero on up to the \emph{minimum} of p and q.


Now we're ready to bring the complex numbers into the story!
We can take a real vector space with a metric on it and "complexify"
it by letting our vectors have complex coefficients instead of real
ones, and using the same formula for the metric.  But the funny thing 
about "complexifying" is that it actually \emph{simplifies} things in 
certain ways.  Since i^{2} = -1, you can turn a vector from timelike
to lightlike or vice versa just by multiplying it by i!  This means
the distinction between space and time isn't such a big deal anymore.
In particular, it doesn't matter how many space or time directions we 
had to begin with; after complexifying them, all the spaces R^{p,q} 
look just like C^{n} (n = p+q) with the metric 


x.y = x_{1} y_{1} + ... + x_{n} y_{n}

In other words, all these spaces R^{p,q} are sitting inside 
C^{n} as different "real parts". 


It's also easy to see that if we start with an isotropic subspace of
R^{p,q}, and take \emph{complex} linear combinations of the
vectors in that subspace, we get an isotropic subspace of C^{n}.
This means all the stuff we just learned about the "real world" has
ramifications for the "complex world".


For example, we instantly know that C^{n} has isotropic subspaces of 
dimension up to the minimum of p and q, where p and q are \emph{any}
numbers with p+q = n.  To get this minimum as big as possible, we should
take p = q = n/2.  Then we'll get isotropic subspaces of dimensions
going all the way up to n/2.  But we can only do this when n is even!  
When n is odd, the best we can do is (n-1)/2.


This shows that isotropic subspaces of C^{n} work differently depending
on whether n is odd or even.   I described this in more detail above,
where I separately treated SO(n,C) for n odd and n even.

\par\noindent\rule{\textwidth}{0.4pt}
\textbf{Addendum:}
Here are two posts on sci.physics.research which address
this mysterious fact: there's no dot in the Dynkin diagram
for SO(2n,C) corresponding to the (n/2 - 1)-dimensional
isotropic subspaces of C^{2n}, even though there is
one for every \emph{other} dimension from 1 to n/2.

\begin{verbatim}

From: James Dolan
Subject: Re: This Week's Finds in Mathematical Physics (Week 180)
Date: Thu, 13 Jun 2002 


marc bellon wrote:

| John Baez writes:

|> Borcis wrote:

|> >John Baez wrote:
 
|> >>Boris Borcic wrote:
 
|> >> >None of the diagrams for C^n, n even, shows an entry for n/2-1 
|> >> >dimensional isotropic subspaces - how should we read this fact ?
 
|> >> I don't know what it means.  Isotropic subspaces of this
|> >> dimension certainly <em>exist</em>, but for some reason the theory
|> >> I am describing here does not regard them as important.  It's
|> >> not an arbitrary decision on anyone's part; it's built into the 
|> >> logic of the subject - but I don't understand it.
 
|> >No doubt a temporary phenomenon :)

|> Let's hope so.  

|Let me propose an explanation.
|It is sufficient to consider the four dimensional case.
|In the two-dimensional case, there are two isotropic lines,
|one of which is self-dual and the other anti-self-dual, so that 
|the configuration is completely fixed, consistent with the abelian 
|character of SO(2).
|Now when I choose an isotropic line is C^4, its orthogonal is a three
|dimensional subspace which contains it, so that the extension of the
|isotropic line to an isotropic plane is equivalent to choosing an 
|isotropic line in a two-dimensional space.  But in view of the 
|two-dimensional case, no choice has to be made, so that an isotropic
|line uniquely define two isotropic plane, one self-dual, the other
|anti-self-dual.  Reciprocally, a self-dual isotropic plane and an 
|anti-self-dual one evidently cannot coincide, but they cannot either 
|be complementary: in this case they are dual to each other using the 
|metric and are of the same self-duality.
|
|To give an isotropic line in C^4 is therefore equivalent to give a
|pair of isotropic planes, one self-dual and the other anti-self-dual.
|AFAIK, it is the property used in the twistor program of Penrose:
|you parameterize the light rays (null lines) by the isotropic planes
|it lies on.  More generally, when considering SO(2n), you do not need 
|to consider the (n-1)-dimensional isotropic plane, since they are 
|uniquely defined by the combination of a self-dual n-space and an
|anti-self dual one, if they have a (n-2)-dimensional space in common.


this seems like a good explanation.  extrapolating from this case,
maybe whenever we have a dynkin diagram corresponding to a particular
sort of incidence geometry, and a chosen dot in the diagram
corresponding to a particular sort of "point" in the geometry, then
for any "anti-chain" in the dynkin diagram, the type of partial flag
corresponding to the anti-chain is uniquely determined by (and thus
representable as) the intersection of the subspaces in the partial
flag.

thus in the case described by marc bellon, the dynkin diagram is a
"d" series diagram such as d_5:

      *
     /
*-*-*
     \
      *

and the chosen dot (actually an asterisk in the above picture) is the
leftmost one.  labeling the dots by letters and placing the chosen dot
at top we have:

  a
  |
 b
  |
  c
 / \
d   e

.  then a so-called "anti-chain" with respect to the partial order "x
is in more direct proximity to a than y is" is a dot-set s such that
no member of s is subordinate to any other member; thus for example
{}, {b}, and {d,e} are anti-chains but since e is subordinate to c,
{c,e} isn't an anti-chain.

{d,e} is in fact the only anti-chain in the above partial order with
more than one dot.  arranging the anti-chains in order from those with
a larger collection of subordinates to those with a smaller
collection, we have:

  {a}
   |
 {b}
   |
 {c}
   |
 {d,e}
  / \
{d} {e}
  \ /
  { }

.  now for each anti-chain we can try to calculate the dimension of
the intersection of all of the subspaces in a partial flag of the type
corresponding to the anti-chain (that is, containing one subspace of
each type corresponding to a dot in the anti-chain).  according to
marc bellon we get these dimensions:

  0
  |
 1
  |
 2
  |
  3
 / \
4   4
 \ /
  8

and this more or less explains the mystery which boris borcic and john
baez were discussing, as to why it seemed at first that 3-dimensional
subspaces play no interesting role in the incidence geometry of the
d_5 dynkin diagram (and correspondingly for other "d" series
diagrams): it turns out that 3-dimensional subspaces _do_ play an
interesting role here, but they're related to a multi-dot anti-chain
in the dynkin diagram instead of to a single dot.  the importance of
anti-chains here comes as a bit of a surprise if your intuition about
incidence geometry is based on classical projective geometry, where
the dynkin diagram is in the "a" series and the chosen dot is an
end-dot, because in that case there are no multi-dot anti-chains.

now we can take an arbitrary dynkin diagram and an arbitrary chosen
dot in it and try to calculate for the corresponding incidence
geometry the dimensions of the types of subspaces corresponding to the
anti-chains in the partial order, making some optimistic assumptions.
consider for example the dynkin diagram e_7:

    *
    |
*-*-*-*-*-*

with the rightmost dot as the chosen dot.  then we have:

  a
  |
 b
  |
 c
  |
  d
 / \
e   f
    |
    g

and the anti-chains for the partial order are:

  {a}
   |
 {b}
   |
 {c}
   |
 {d}
   |
 {e,f}
  / \
{f}{e,g}
  \ / \
  {g} {e}
    \ /
    { }

.  using an optimistic method of calculation related to methods
mentioned by john baez in some previous posts in this thread but not
really explained there either, we obtain for the dimensions of the
corresponding types of subspace:

  0
  |
 1
  |
 2
  |
 3
  |
  4
 / \
5   5
 \ / \
 10   6
   \ /
   27

.  so that's what this calculation predicts: that e_7 geometry
involves a compact 27-dimensional manifold of "points", with types of
special subspaces of dimensions 1, 2, 3, 4, 6, and 10, plus two
different types of special subspaces of dimension 5.  the special
4-dimensional subspaces and one of the types of special 5-dimensional
subspaces are evidently of "anti-chain" type.  i'd be interested to
know whether e_7 geometry has ever been described along these lines,
or more generally whether special subspaces of the "anti-chain" type
have been studied or at least noticed, beyond the cases described by
marc bellon.

\end{verbatim}
    
\par\noindent\rule{\textwidth}{0.4pt}

\begin{verbatim}

From: James Dolan
Subject: Re: This Week's Finds in Mathematical Physics (Week 180)
Date: Sat, 15 Jun 2002 


i wrote:

|now we can take an arbitrary dynkin diagram and an arbitrary chosen
|dot in it and try to calculate for the corresponding incidence
|geometry the dimensions of the types of subspaces corresponding to the
|anti-chains in the partial order, making some optimistic assumptions.
|consider for example the dynkin diagram e_7:
|
|    *
|    |
|*-*-*-*-*-*
|
|with the rightmost dot as the chosen dot.  then we have:
|
|  a
|  |
 b
|  |
 c
|  |
|  d
| / \
|e   f
|    |
|    g
|
|and the anti-chains for the partial order are:
|
|  {a}
|   |
 {b}
|   |
 {c}
|   |
 {d}
|   |
| {e,f}
|  / \
|{f}{e,g}
|  \ / \
|  {g} {e}
|    \ /
|    { }
|
|.  using an optimistic method of calculation related to methods
|mentioned by john baez in some previous posts in this thread but not
|really explained there either, we obtain for the dimensions of the
|corresponding types of subspace:
|
|  0
|  |
 1
|  |
 2
|  |
 3
|  |
|  4
| / \
|5   5
| \ / \
| 10   6
|   \ /
|   27
|
|.  so that's what this calculation predicts: that e_7 geometry
|involves a compact 27-dimensional manifold of "points", with types of
|special subspaces of dimensions 1, 2, 3, 4, 6, and 10, plus two
|different types of special subspaces of dimension 5.  the special
|4-dimensional subspaces and one of the types of special 5-dimensional
|subspaces are evidently of "anti-chain" type.


having thought about it some more, i now think that we can give much
more specific information about the nature of the geometry here, and
in a much simpler way.


given a dotted dynkin diagram, this time for example say:

    a
    |
b-c-d-*-f-g-h

, we can consider the partially ordered set of all connected
sub-diagrams including the chosen dot, in this case:



                              *




                          d-*  *-f


                   a
                   |
                   d-*  c-d-*  d-*-f  *-f-g 


           a             a
           |             |
         c-d-*  b-c-d-*  d-*-f  c-d-*-f  d-*-f-g  *-f-g-h


            a      a
            |      |
        b-c-d-*  c-d-*-f  b-c-d-*-f  c-d-*-f-g  d-*-f-g-h


      a        a                       a
      |        |                       |
  b-c-d-*-f  c-d-*-f-g  b-c-d-*-f-g  c-d-*-f-g-h  b-c-d-*-f-g-h


                 a          a
                 |          |
             b-c-d-*-f-g  c-d-*-f-g-h  b-c-d-*-f-g-h


                             a
                             |
                         b-c-d-*-f-g-h



.  then each sub-diagram in the partial order can be interpreted as a
type of special subspace of the space of points in the

    a
    |
b-c-d-*-f-g-h

geometry, with the partial order (not completely explicit in the above
picture) indicating the containment relationships between the
subspaces in a complete so-called "flag" configuration, including
subspaces generated by intersection from the "principal" subspaces in
the flag.  furthermore, intersection of sub-diagrams corresponds
perfectly to intersection of subspaces in the flag.

thus in this case the space of points of the geometry contains special
subspaces that look like projective lines (since

*

is the dotted dynkin diagram for projective line geometry), two kinds
of special subspaces that look like projective planes (since

d-*

and

*-f

are slightly different ways of drawing the dotted dynkin diagram for
projective plane geometry), three kinds of subspaces that look like 
projective 3-spaces (since

a
|
d-*

and

c-d-*

and

*-f-g

are isomorphic to the dotted dynkin diagram for projective 3-space
geometry), and so forth.  since the

d-*

sub-diagram and the

*-f

sub-diagrams intersect in

*

, the intersection of special projective planes of the two different
types will be a special projective line if the two special projective
planes lie in a single flag.  and so forth.

(one minor defect in this treatment is that the semi-lattice of
connected sub-diagrams containing the chosen dot needs to be
supplemented by one extra element at the top to account for the
singleton subspaces of the geometry; i'm not going to worry about that
for now.)

now let's return to the example discussed by marc bellon.  we have a
d-series dynkin diagram dotted at the boring end, thus for example
d_5:

      d
     /
*-b-c
     \
      e

.  the semi-lattice of connected sub-diagrams containing the chosen
dot is:


           *


          *-b


         *-b-c


      d         *-b-c
     /               \ 
*-b-c                 e


              d
             /
        *-b-c
             \
              e


.  we see that a flag in this geometry includes a projective line
corresponding to

*

, a larger projective plane corresponding to

*-b

, a larger projective 3-space corresponding to

*-b-c

, two larger projective 4-spaces corresponding to

      d         *-b-c
     /               \ 
*-b-c                 e

whose intersection is the projective 3-space, and finally the space of
all points in the geometry, corresponding to


      d
     /
*-b-c
     \
      e


.  since the projective 3-space appears as the intersection of the two
projective 4-spaces, it's in some sense redundant and thus not one of
the "principal" subspaces in the flag.  but it's there nevertheless,
thus more or less resolving boris borcic's mystery of the missing
isotropic subspace.


this all seems simple enough (in principle) now that it must be
well-known, but i don't know where it might be discussed in reasonably
plain language.
\end{verbatim}
    




 \par\noindent\rule{\textwidth}{0.4pt}
<em>If you want to get a view 
of the world you live in, climb a little rocky mountain with 
a neat small peak. But the big snowpeaks pierce the world of clouds and 
cranes, rest in the zone of five colored banners and writhing crackling 
dragons in veils of ragged mist and frost crystals, into a pure transparency 
of blue.</em> - Gary Snyder

\par\noindent\rule{\textwidth}{0.4pt}

% </A>
% </A>
% </A>
