
% </A>
% </A>
% </A>
\week{arch 23, 2003   }


In 1999 a Canadian businessman named Mike Lazaridis donated 
$100 million to set up the Perimeter Institute for Theoretical
Physics.  Right now it's housed in a red stone building with 
a big clock tower on King Street in Waterloo, Ontario.  It's 
funky and comfortable: the place used to be a restaurant, and 
there's still an espresso bar, a pool table and an out-of-tune 
piano in a big room on the second floor.   You can make yourself 
coffee, and get sandwich fixings and soft drinks from the refrigerator
whenever you want... and they show movies on Friday!  Right now
the institute is focused on quantum gravity and quantum computation,
but eventually it will move to a big new building and expand quite 
a bit, perhaps including some cosmology and particle physics.

For details see:

1) Perimeter Institute, <A HREF =
"http://perimeterinstitute.ca/">http://perimeterinstitute.ca/</A>

I've been talking to lots of people here, including Lee Smolin, 
who just came out with this review article on quantum gravity:

2) Lee Smolin, How far are we from the quantum theory of gravity?,
available as <A HREF = "http://xxx.lanl.gov/abs/hep-th/0303185">hep-th/0303185</A>.

He compares all the main approaches, with an emphasis on loop quantum 
gravity and string theory.  This is great, because he's one of the few
people who has thought hard about both loops and strings.  He comes down
rather critical of string theory, pointing out a number of issues which 
had escaped my attention.  In fact, he told me he wasn't feeling so 
critical when he started writing this review article; he says writing it 
pushed him further in that direction.

For example, people often claim the great thing about string theory is
that it's "finite": that is, one can compute how strings scatter off
each other as an infinite (but possibly divergent) sum of well-defined
terms, one for each different number of holes in the string worldsheet:


\begin{verbatim}

     o  o        o  o  
     \\//        \\//
      ||          ||  
      ||         //\\  
      ||    +   ||  ||   +  ....
      ||         \\//
      ||          ||
     //\\        //\\ 
     o  o        o  o  

\end{verbatim}
    
But there are different string theories to consider here: bosonic 
string theory and 5 the different superstring theories (see "<A HREF = "week72.html">week72</A>").  


The bosonic string is indeed finite, but it has other
problems.  For example, the sum diverges, and you can't even get a
finite answer for it using the trick called "Borel
summation".  Bosonic string theory also predicts a tachyon, which
is a sign that the theory is unstable.

Most importantly, bosonic string theory doesn't predict fermions, which 
we need in any theory of particle physics.  So for physics, what really 
matters are the superstring theories.  And for these, it turns out people 
have only figured out how to compute the amplitudes for worldsheets with 
at most 2 holes in them: the so-called 2-loop or genus-2 case.  Moreover, 
this was done only in 2001!   It was done by Eric D'Hoker and D. H. Phong
in a series of 4 long technical papers.

In the first of these papers, they wrote:

\begin{quote}
 Despite great advances in superstring theory, multiloop 
 amplitudes are still unavailable, almost twenty years after the
 derivation of the one-loop amplitudes by Green and Schwarz
 for Type II strings and by Gross et al for heterotic strings.  
 The main obstacle is the presence of supermoduli for worldsheets
 of non-trivial topology.  Considerable efforts had been made by
 many authors in order to overcome this obstacle, and a chaotic
 situation ensued, with many competing prescriptions proposed in 
 the literature.  These prescriptions drew from a variety of 
 fundamental principles such as BRST invariance and the picture-
 changing formalism, descent equations and Cech cohomology, 
 modular invariance, the light-cone gauge, the global geometry
 of the Teichmueller curve, the unitary gauge, the operator
 formalism, group theoretic methods, factorization, and algebraic
 supergeometry.  However, the basic problem was that gauge-fixing
 required a local gauge slice, and the prescriptions ended up 
 depending on the choice of such slices, violating gauge invariance.
\end{quote}
I hope the techniques they devised for the 2-loop case speed up 
progress on higher-loop amplitudes!  It would be nice to know if 
superstring theory really lives up to its promise of finiteness.


Smolin's paper also gives a critical summary of various standard
conjectures in string theory, along with the evidence for these.  This
makes good reading for anyone wondering how much of what one hears
about string theory is hype and how much is solid.  To make this
clear, Smolin states an amusing "minimal string theory
conjecture" describing the worst possible scenario consistent
with everything that's actually been shown so far!  The gap between
this and the more optimistic scenarios one usually hears is truly
vast.

My only complaint about Smolin's review article is that it's not 
sufficiently critical of loop quantum gravity.  It does mention 
that nobody knows whether this theory reduces to general relativity 
at distance scales much larger than the Planck length, but it doesn't 
make clear how severe this problem is.  For example, it doesn't point
out that nobody agrees on the correct dynamics for this theory!  Given 
this, the issue of whether loop quantum gravity reduces to general 
relativity at large distance scales is not a mere yes-or-no question: 
we need to \emph{find a version} of the theory that gives general relativity 
as a limiting case.  


Along similar lines, when Smolin mentions Thiemann's theory of loop
quantum gravity coupled to the Standard Model, he doesn't emphasize
that nobody knows if this theory really reduces to the Standard Model
in a suitable limit: Thiemann has a specific proposal for the
dynamics, but it hasn't been tested in this way.  Finally, I think
Smolin is overly optimistic about Olaf Dreyer's method of computing
the Immirzi parameter in loop quantum gravity.  For a useful
corrective, see "<A HREF = "week189.html">week189</A>" and
especially "<A HREF = "week192.html">week192</A>".


Of course, you can't really expect a harsh list of the flaws of loop
quantum gravity from one of that theory's inventors any more than you
can expect string theorists to tear into \emph{their} theory!  As
A. J. Tolland has pointed out, Steve Carlip's review article is more
even-handed (see "<A HREF = "week171.html">week171</A>").
But Smolin's is still very much worth reading - especially if you want
something not too technical.

Here's a good review of D'Hoker and Phong's proof that heterotic
and type II superstring theory are finite up to 2 loops:


3) Eric D'Hoker and D.H. Phong, Lectures on two-loop superstrings,
available as <A HREF =
"http://xxx.lanl.gov/abs/hep-th/0211111">hep-th/0211111</A>.

It summarizes four long papers of theirs:

4) Eric D'Hoker and D.H. Phong, Two-loop superstrings: I, The main formulas, 
Phys. Lett. B529 (2002), 241-255.  Also available as <A HREF = "http://xxx.lanl.gov/abs/hep-th/0110247">hep-th/0110247</A>.  

II, The chiral measure on moduli space, Nucl. Phys. B636 (2002), 3-60.
Also available as <A HREF = "http://xxx.lanl.gov/abs/hep-th/0110283">hep-th/0110283</A>.  

III, Slice independence and absence of ambiguities, Nucl. Phys. B636 
(2002), 61-79.  Also available as <A HREF = "http://xxx.lanl.gov/abs/hep-th/0111016">hep-th/0111016</A>.  

IV, The cosmological constant and modular forms, Nucl. Phys. B639 (2002), 
129-181.  Also available as <A HREF = "http://xxx.lanl.gov/abs/hep-th/0111040">hep-th/0111040</A>.

The quote above is taken from part I.  

After looking at these, I got a bit curious about the exact state of 
the art in perturbative quantum gravity.  In physics, folklore often
gets exaggerated with each retelling.  If superstring theory is not 
really known to be finite, despite all the folklore to the
contrary, is perturbative quantum gravity \emph{really} known to be 
nonrenormalizable?   

I got some clues here:

5) Zvi Bern, Perturbative quantum gravity and its relation to gauge 
theory, Living Rev. Relativity 5 (2002), available at
<A HREF = "http://www.livingreviews.org/Articles/Volume5/2002-5bern/index.html">http://www.livingreviews.org/Articles/Volume5/2002-5bern/index.html</A>

Zvi Bern, The S-matrix reloaded: twistors, unitarity, gauge theories 
and gravity, talk at the KITP Program: Mathematical Structures in 
String Theory, Sept. 29, 2005.  Video, audio and transparencies available
at <A HREF = "http://online.kitp.ucsb.edu/online/strings05/bern/">http://online.kitp.ucsb.edu/online/strings05/bern/</A> 

It turns out the current best method for understanding perturbative 
quantum gravity is to connect it to Yang-Mills theory via the
"Kawai-Lewellen-Tye relations", whatever those are.  
(Twistor methods have also come into fashion, after I wrote the
original version of this article.)  Apparently 
the state of the art is like this - though I sure haven't checked
these things myself:

<UL>
<LI>
In 4 dimensions, pure gravity without matter is renormalizable to 
  1 loop, but not 2.

<LI>
  In 4 dimensions, pure gravity with non-supersymmetric matter is 
  generically not renormalizable even to 1 loop.

<LI>
  In 4 dimensions, supergravity theories are renormalizable up to
  2 loops.  It is believed that most of these theories are not
  renormalizable to 3 loops, since a candidate divergent term is
  known.  However, "no explicit calculations have as yet been 
  performed to directly verify the existence of the three-loop 
  supergravity divergences." 

<LI>
  Maximally supersymmetric supergravity theories behave better
  than people had expected.  In 4 dimensions, it \emph{seems} that
  so-called "N = 8 supergravity" is renormalizable up to 4 loops,
  but not 5.  However, neither of these have been proved, and 
  this theory could even be renormalizable to all orders: 
  see pages 33-35 in Zvi Bern's transparencies above.

<LI>
  11-dimensional supergravity is renormalizable to 1 loop but not 2.

</UL>
Since M-theory is supposed to reduce to 11-dimensional supergravity in
some sort of limit, the last point is important.  Indeed this
nonrenormalizability is why people stopped working on 11d supergravity
for a while - until evidence started coming in that it sheds a lot of
light on string theory (see "<A HREF =
"week72.html">week72</A>").

For more readable stuff about the nonrenormalizability of 11d
supergravity, try these review articles:


6) Stanley Deser, Nonrenormalizability of (last hope) D=11
supergravity, with a terse survey of divergences in quantum gravities,
available as <A HREF =
"http://xxx.lanl.gov/abs/hep-th/9905017">hep-th/9905017</A>.
 

7) Stanley Deser, Infinities in quantum gravities, Annalen Phys.  9
(2000) 299-307.  Also available as <A HREF =
"http://xxx.lanl.gov/abs/gr-qc/9911073">gr-qc/9911073</A>.

Speaking of M-theory and the like, I've been reading Acharya's 
article on "G2 manifolds", which I mentioned last week, and I've 
been talking to various people about it on sci.physics.research, 
especially Robert Helling and Urs Schreiber.  Here's a bit of what 
I have learned.   

First of all, let me say some basic stuff about why string theorists
like G2 manifolds.  M-theory lives in 11 dimensions, and 4 + 7 = 11,
so it's interesting to study M-theory on a spacetime of the form
R^{4} x N where N is a 7-dimensional manifold.  The kind of
7-dimensional manifold that works is called a "G2 manifold".
Or at least this might be true if anyone knew what M-theory was!  What
people really understand is 11-dimensional supergravity, which is
supposed to be some sort of limiting case of the mysterious mess
called M-theory.  So, Acharya talks about 11d supergravity on
Minkowski spacetime times a G2 manifold, and what sort of physics this
gives.

People also like to study superstring theory on spacetimes of the 
form R^{4} x O.  But superstring theory lives in 10 dimensions, and 
4 + 6 = 10, so here O should be a 6-dimensional manifold.  The kind 
of 6-dimensional manifold that works is called a "Calabi-Yau manifold".  

These ideas are related, because M-theory on R^{4} x N is sort of 
like heterotic string theory on R^{4} x O when N = O x [0,1].  But, 
M-theory on R^{4} x N has an extra adjustable parameter due to the 
length of the interval [0,1].  This lets you make gravity weaker 
than the other forces, which you can't do in heterotic string theory.

At least this is what my sources tell me!   I don't understand all 
of this, so it could be a bit wrong.  But I think I understand how
G2 manifolds and Calabi-Yau manifolds are related, and why O being a
Calabi-Yau manifold makes O x [0,1] into a G2-manifold.  So, I'll explain 
that.

The key principle to keep in mind is that any type of structure you
can put on a real inner product space yields a type of Riemannian 
manifold.  Each tangent space of a Riemannian manifold is a real inner
product space, and there's a god-given way to parallel transport tangent 
vectors on a Riemannian manifold.  So, if X is some type of structure 
you can put on a real inner product space, you can define an "X-manifold" 
to be a Riemannian manifold where each tangent space has an X-structure... 
in a way that's preserved by parallel transport!

For example, X could be a "Hermitian structure" - a way of making a
real inner product space into a \emph{complex} inner product space.  Then 
an X-manifold is called a "Kaehler manifold".  

When we parallel transport a vector around a loop in a n-dimensional
Riemannian manifold, it can be rotated or reflected.  In more jargonesgue
jargon, the holonomy around a loop defines an element of the group O(n).  
But when your manifold is a Kaehler manifold, each tangent space becomes 
a complex inner product space of dimension n/2, in a way that's preserved 
by parallel transport.  So, the holonomy around any loop must lie in 
the unitary group U(n/2).   

There's a converse to this, as well!  So a Kaehler manifold is just a 
Riemannian manifold where the holonomies all lie in U(n/2).  

And this is how it usually works - or \emph{always}, if you take care
to include all the necessary fine print.  Thus many sorts of
X-manifolds are called "manifolds with special holonomy".  See:

8) Dominic Joyce, Compact Manifolds with Special Holonomy, Oxford U. Press, 
Oxford, 2000. 

For example, suppose X is a "quaternionic structure" - a way
of making a real inner product space into a quaternionic inner product
space.  Then an X-manifold is called a "hyperKaehler
manifold", and this just one where the holonomies lie in the
quaternionic unitary group Sp(n/4).

Or, suppose X is a Hermitian structure together with an n/2-form.
Then an X-manifold is called a "Calabi-Yau manifold".  This concept 
of Calabi-Yau manifold works in any even dimension, while before I was 
just talking about 6-dimensional ones!  For parallel transport around 
a loop to preserve an n/2-form as well as a Hermitian structure, 
the holonomy must lie in SU(n/2).  So, a Calabi-Yau manifold is the 
same as one where the holonomies lie in SU(n/2).  

We can define G2-manifolds in a similar way.  But to do this, and
to see how they're related to 6-dimensional Calabi-Yau manifolds,
we need a detour into the theory of spinors.  The reason is that 
"N = 1 supersymmetric theories" work nicely when you can pick a
spinor at each point of space in a way that's preserved by parallel 
transport.  We call such a thing a "covariantly constant spinor 
field".  Actually, this spinor field needs to be nonzero to be
of any use, but that's so obvious people often don't mention it.

Now, a nonzero spinor isn't exactly an extra structure you can put 
on a real inner product space, since spinors are representations
not of O(n) or even SO(n) but of the double cover Spin(n).  
However, if you start with a \emph{spin} manifold, you can think of a
nonzero covariantly constant spinor field as some extra structure
that reduces the holonomy group from Spin(n) down to some subgroup.

So, let's see what this extra structure is like in some examples!

For the examples I'll talk about, the key is that spinors in 5-, 6-, 7- 
and 8-dimensional space are all very related, and all very related to 
the octonions.  You can see this from looking at the even part of 
the Clifford algebra, because spinors are defined to be irreducible
representations of this algebra.  Here's what the even part of the 
Clifford algebra looks like in various dimensions:


$$

dimension 1:  R^{ }
dimension 2:  C^{ }
dimension 3:  H^{ }
dimension 4:  H + H^{ }
dimension 5:  H(2) ^{ }
dimension 6:  C(4)^{ }
dimension 7:  R(8)^{ }
dimension 8:  R(8) + R(8)^{ }
$$
    
Here K = R, C, H stands for the real numbers, complex numbers and 
quaternions, while K(n) means n x n matrices with entries in K. 

I'll always be interested in \emph{real} spinors, which are the irreducible 
\emph{real} representations of these algebras.  I won't even keep saying
the word "real" from now on.  If you eyeball the above chart, you'll 
see that in dimensions 4 and 8 we get two kinds of spinor - called
left- and right-handed spinors - while in the other dimensions there's 
just one kind.  The way these spinors work is sort of obvious:


$$

dimension 1:  R^{ }
dimension 2:  C^{ }
dimension 3:  H^{ }
dimension 4:  left and right, both H^{ }
dimension 5:  H^{2}
dimension 6:  C^{4}
dimension 7:  R^{8}
dimension 8:  left and right, both R^{8}
$$
    

Now the cool part is that H^{2}, C^{4} and
R^{8} are all secretly the same 8-dimensional real vector
space equipped with various amounts of extra structure - i.e. the
structure of a 4-dimensional complex vector space, or a 2-dimensional
quaternionic vector space.  And you'll probably be more bored than
shocked when I tell you that this 8-dimensional real vector space is
yearning to become the OCTONIONS.

Let's see how we can use this to study specially nice manifolds
in 8, 7, 6 and 5 dimensions.   We'll start in dimension 8 and climb
our way down by a systematic process.  In 7 dimensions we'll get G2 
manifolds, while in 6 dimensions we'll get Calabi-Yau manifolds.

Okay:

In 8 dimensions there are three different 8-dimensional irreps of the 
spin group (the double cover of the rotation group):

<UL>
<LI>
the vector rep V 
<LI>
the left-handed spinor rep S+    
<LI>
the right-handed spinor rep S-  
</UL>
You can build a vector from a left-handed spinor and a
right-handed spinor, so we have an intertwining operator: 


$$

S+ tensor S- \to  V
$$
    
The cool part is that this map tells us how to multiply octonions!

More precisely, suppose we pick a unit vector 1+ in S+ and a unit vector 
1- in S-.   It turns out that multiplying by 1+ defines an isomorphism 
from S- to V.  Similarly, multiplying by 1- gives an isomorphism from 
S+ to V.  This lets us think of all three spaces as the same: THE OCTONIONS, 
with m as the octonion product and 1+ (or 1- if you prefer) as its unit.

In fact, there's nothing special about writing our operator as


$$

S+ tensor S- \to  V
$$
    
since all three of these reps are their own dual.  This lets us
permute these guys and work with


$$

V tensor S+ \to  S-
$$
    
or whatever we like.   So, picking unit vectors in any 2 out of these
3 spaces gives us a unit vector in the third and makes all 3 into an
algebra isomorphic to the octonions.

This instantly implies that if we have an 8-dimensional spin manifold M 
with nonzero covariantly constant sections of 2 of these 3 bundles:
 
<UL>
<LI>
the left-handed spinor bundle
<LI>
the right-handed spinor bundle
<LI>
the tangent bundle
</UL>
we get a way to make all 3 of these bundles into "octonion bundles" -
meaning that each fiber is an algebra in a covariantly constant way, 
where this algebra is isomorphic to the octonions.

This in turn implies that the holonomy group of the metric on M must 
be a subgroup of G2 - the automorphism group of the octonions.  

Let's call a manifold like this M an "octonionic manifold".

How do we get manifolds like this?
 
The easiest way is to take a 7-dimensional spin manifold N and let 
M = N x R.  The special 8th direction in M gives us a nonzero 
covariantly constant vector field on M.  So, to get the above 
"2 out of 3" trick to work, we just need a nonzero covariantly 
constant section of either the left- or right-handed spinor bundle 
of M.

But as we've seen, spinors in 7 dimensions are secretly the same
as either left- or right-handed spinors in 8 dimensions.  So, it
suffices to have a nonzero covariantly constant spinor field on N.

Thus, when N is a 7-dimensional spin manifold with a nonzero covariantly 
constant spinor field, its spinor bundle automatically becomes an octonion 
bundle!  


Its tangent bundle doesn't become an octonion bundle, because it's
just 7-dimensional.  But if you think about what I've said, you'll see
the tangent bundle plus a trivial line bundle becomes an octonion
bundle.  This trivial line bundle corresponds to the \emph{real}
octonions, while the tangent bundle of N corresponds to the
\emph{imaginary} octonions.


The imaginary octonions are 7-dimensional, and they have a "dot
product" and "cross product" rather like those in 3
dimensions.  Since you can use these to recover the octonion product,
the group of transformations of the imaginary octonions preserving the
dot product and cross product is again G2.
 
So, the tangent bundle of N becomes an "imaginary octonion bundle", 
meaning that each fiber gets a dot product and cross product in a 
covariantly constant way, making it isomorphic to the imaginary octonions.

This in turn implies that the holonomy group of the metric on N must 
be a subgroup of G2.  

People call a manifold like this N a "G2 manifold".  

How do we get manifolds like this?

The easiest way is to take a 6-dimensional spin manifold O and let 
N = O x R.   To make N into a G2 manifold, we need a nonzero covariantly 
constant spinor field on N.  

But as we've seen, spinors in 6 dimensions are secretly the same 
as spinors in 7 dimensions.  So, it suffices to have a nonzero 
covariantly constant spinor field on O.  

Thus, when O is a 6-dimensional spin manifold with a nonzero 
covariantly constant spinor field, its spinor bundle automatically
becomes an octonion bundle!

Its tangent bundle doesn't become an imaginary octonion bundle,
because it's just 6-dimensional.  But if you think about what I've
said, you'll see the tangent bundle plus a trivial line bundle 
becomes an imaginary octonion bundle.  This trivial line bundle
corresponds to a particular direction in the imaginary octonions.

This in turn implies that the holonomy group of O must lie in the 
subgroup of G2 fixing a direction in the imaginary octonions.  
This subgroup is SU(3), so the holonomy group of O must be a 
subgroup of SU(3).  

People call a manifold like this O a "Calabi-Yau manifold".

How do we get manifolds like this?

The easiest way is to take a 5-dimensional spin manifold P and let 
O = P x R.   To make O into a Calabi-Yau manifold, we need a nonzero 
covariantly constant spinor field on O.

But as we've seen, spinors in 5 dimensions are secretly the same as
spinors in 6 dimensions.   So, it will suffice to have a nonzero
covariantly constant spinor field on P.

Thus, when P is a 5-dimensional spin manifold with a nonzero
covariantly constant spinor field, its spinor bundle automatically
becomes an octonion bundle!

Its tangent bundle doesn't become an imaginary octonion bundle,
because it's just 5-dimensional.  But if you think about what I've
said, you'll see the tangent bundle plus two trivial line bundles
becomes an imaginary octonion bundle.  These trivial line bundles
correspond to two orthogonal directions in the imaginary octonions.

This in turn implies that the holonomy group of P must lie in the 
subgroup of G2 fixing two orthogonal directions in the imaginary 
octonions.  This subgroup is SU(2).  

I'll call a manifold like this P an "SU(2) manifold".

Does my prose style seem stuck in a loop?  That's on purpose;
I'm trying to make a certain pattern very clear.  But the loop stops
here, or at least changes flavor drastically, because spinors stop 
being 8-dimensional when we get down to 4-dimensional space.   

Summary: 

<UL>
<LI>
When M is an 8-dimensional spin manifold with 2 out of these 3 things: 

<UL>
<LI>
a nonzero covariantly constant vector field
<LI>
a nonzero covariantly constant left-handed spinor field
<LI>
a nonzero covariantly constant right-handed spinor field
</UL>
it automatically gets all three - and its tangent bundle, 
left-handed spinor bundle and right-handed spinor bundle all
become octonion bundles.   We call M an octonionic manifold.

<LI>
When N is a 7-dimensional spin manifold with a nonzero
covariantly constant spinor field, its spinor bundle becomes an
octonion bundle, while its tangent bundle becomes an imaginary 
octonion bundle.   We call N a G2 manifold.  

<LI>
When O is a 6-dimensional spin manifold with a nonzero
covariantly constant spinor field, its spinor bundle becomes an
octonion bundle, while its tangent bundle plus a trivial line bundle
becomes an imaginary octonion bundle.  We call O a Calabi-Yau
manifold.

<LI>
When P is a 5-dimensional spin manifold with a nonzero
covariantly constant spinor field, its spinor bundle becomes an
octonion bundle, while its tangent bundle plus two trivial line 
bundles becomes an imaginary octonion bundle.   We call O an 
SU(2) manifold.  
</UL>

An SU(2) manifold times R is a Calabi-Yau manifold;
a Calabi-Yau manifold times R is a G2 manifold;
a G2 manifold times R is an octonionic manifold.  

You may not like how the 8-dimensional case on the above list is
different from the rest.  Don't worry; people also study 8-dimensional
spin manifolds that admit just a nonzero covariantly constant
left-handed \emph{or} right-handed spinor field.  The holonomy group of
such a manifold must like in Spin(7), and such a manifold is called a
Spin(7) manifold.

You may wonder how I knew that the subgroup of G2 fixing
one direction in the imaginary octonions is SU(3).  You may
also wonder how I knew that the subgroup of G2 fixing two 
orthogonal directions in the imaginary octonions is SU(2).

This is very pretty!  I mainly just used two facts we've already 
seen: the even part of the Clifford algebra in 6 dimensions is C(4), 
while in 5 dimensions it's H(2).  

The first of these facts implies that so(6) must sit inside
the traceless skew-adjoint matrices in C(4).  In other words, 
so(6) sits inside su(4).  But 


\begin{verbatim}

dim(so(6)) = dim(su(4)) = 15
\end{verbatim}
    
so in fact so(6) = su(4).  Indeed, SU(4) is the double
cover of SO(6), and it acts on the space of spinors, C^{4}, 
in the obvious way.  The subgroup fixing a unit spinor is 
thus SU(3).   

The second of these facts implies that so(5) must sit inside
the traceless skew-adjoint matrices in H(2).  In other words,
so(5) sits inside sp(2).  But


\begin{verbatim}

dim(so(5)) = dim(sp(2)) = 10
\end{verbatim}
    
so in fact so(5) = sp(2).  Indeed, Sp(2) is the double
cover of SO(5), and it acts on the space of spinors, H^{2},
in the obvious way.  The subgroup fixing a unit spinor is
thus Sp(1)... which being the unit quaternions, is isomorphic to SU(2).

If you think about it a while, these results do the job.

If you wish you had some pictures to help you with all this
higher-dimensional geometry, here's the best I can do.  Start
with the octonion multiplication triangle I keep drawing - 
I explained it in "<A HREF = "week104.html">week104</A>": 



\begin{verbatim}

                            e6

 
                        e4      e1
                            e7

                    e3       e2     e5
\end{verbatim}
    
This is really the Fano plane: the projective plane over the
field with two elements.  The 3d vector space over this field
looks like a cube, and the Fano plane is just a flattened-out
picture of this cube:
  

\begin{verbatim}

                           e6
                          /  \
                        e4    e1
                         |\  /| 
                         | e7 | 
                        e3  | e5
                          \ |/
                           e2     
\end{verbatim}
    
The hidden corner of this cube corresponds to the octonion "1". 
If rotate the cube so that corner is on top, and blow it up a bit,
it looks like this:


\begin{verbatim}

                            1
                           /|\
                          / | \
                         /  |  \         
                        /   |   \       
                      e3   e6    e5   
                      |\   / \   /| 
                      | \ /   \ / |
                      |  \     /  |
                      | / \   / \ |
                      e4   \ /   e1 
                       \   e2    / 
                        \   |   /                       
                         \  |  /
                          \ | /
                           \|/
                           e7
\end{verbatim}
    
Now, this cube has an an obvious Z_{3} symmetry that we get by
holding it between our thumb and finger and rotating it about 
the vertical axis.  This Z_{3} group acts as automorphisms
of the octonions that fix the elements 1 and e7.  Of course,
every automorphism fixes 1, so the interesting part is that
they fix a unit imaginary octonion, e7.  

But Z_{3} is a subgroup of SO(3) in an obvious way, since any 
cyclic permutation of the x,y,z axes gives a rotation.
And SO(3), in turn, is a subgroup of SU(3) in an obvious way. 
And we already know that SU(3) is the group of \emph{all} automorphisms 
of the octonions that fix a unit imaginary octonion, say e7.  

Or if you prefer: octonions are the same as spinors in 7 dimensions, 
and SU(3) is the subgroup of Spin(7) that fixes two orthogonal 
unit spinors, namely those corresponding to 1 and e7.

Either way, you can think of SO(3) and SU(3) as souped-up versions
of the obvious Z_{3} symmetry of the octonion cube.  Here's how
the octonions decompose as a representation of SO(3):



\begin{verbatim}

 1d real rep                   1
 of SO(3)                     /|\
                             / | \
 ......................................................................
                           /   |   \          
 3d real rep             e3   e6    e5   
 of SO(3)                |\   / \   /| 
                         | \ /   \ / |
 ......................................................................
                         |  \     /  |
 3d real rep             | / \   / \ |
 of SO(3)                e4   \ /   e1 
                          \   e2    / 
                           \   |   /                       
 ......................................................................
                             \ | /
 1d real rep                  \|/
 of SO(3)                     e7
\end{verbatim}
    
And here's how they decompose as a rep of SU(3):




\begin{verbatim}

 1d real rep                   1
 of SU(3)                     /|\
                             / | \
 ......................................................................
                           /   |   \          
                         e3   e6    e5   
                         |\   / \   /| 
 3d complex rep          | \ /   \ / |
 of SU(3)                |  \     /  |
                         | / \   / \ |
                         e4   \ /   e1 
                          \   e2    / 
                           \   |   /                       
 ......................................................................
                             \ | /
 1d real rep                  \|/
 of SU(3)                     e7
\end{verbatim}
    
I hope this makes things a bit more vivid!  

\par\noindent\rule{\textwidth}{0.4pt}
Addendum: My definition of "Kaehler manifold" above was a bit
nonstandard.  For a while, some of us on sci.physics.research
started worrying that it wasn't equivalent to the usual one!  Luckily,
it turns out that it is.  Here is some of our discussion of this issue.

John Baez wrote:

\begin{quote}

$$

Squark wrote:

>John Baez wrote:

>> [Moderator's note: a Kaehler manifold has to be complex, not
>> just "almost complex". - jb]  

>That's precisely my problem. You said that putting a Hermitian
>structure of the tangent space of a real manifold at each point
>(putting it on the tangent bundle, more accurately) makes it into
>a Kaehler manifold.

No, I did not say this!  I'll remind you of what I actually said.

>However, there's the additional condition of the almost complex
>structure resulting on the manifold being an actual complex
>structure. This cannot be ensured on the "point level", i.e. it is
>not enough to speak of the kind of structure you put on the tangent
>space at each point, but it's important how those structures "glue
>together" (except the obvious smoothness part).

Right - in math jargon, we need some "integrability conditions"
to ensure that the complex structures on each tangent space fit
together to make each little patch of the manifold look like C^n.
Only then do we get a complex manifold.  Otherwise we just have
an "almost complex manifold".  

I didn't ignore this issue, but now you've got me worried
that I may not have handled it correctly.  Here's what I wrote:

  The key principle to keep in mind is that any type of structure you
  can put on a real inner product space yields a type of Riemannian 
  manifold.  Each tangent space of a Riemannian manifold is a real
  inner product space, and there's a god-given way to parallel
  transport tangent vectors on a Riemannian manifold.  So, if X is
  some type of structure you can put on a real inner product space,
  you can define an "X-manifold" to be a Riemannian manifold where each
  tangent space has an X-structure...  in a way that's preserved by
  parallel transport!

  For example, X could be a "Hermitian structure" - a way of making a
  real inner product space into a \emph{complex} inner product space.  Then 
  an X-manifold is called a "Kaehler manifold".

See?   

I didn't say an X-manifold was a Riemannian manifold where
each tangent space is given a structure of type X.

I said it was a Riemannian manifold on which each tangent space is
given a structure of type X... IN A WAY THAT'S PRESERVED BY PARALLEL
TRANSPORT!

If I had left out that last clause, I'd obviously be in trouble.
This last clause is the only condition that relates what's going
on at different tangent spaces.

In particular, if X = "a Hermitian structure", an X-manifold is a
Riemannian manifold where each tangent space is equipped with a
complex structure J and a complex inner product h whose real part 
is the original Riemannian inner product... such that h and J are
preserved by parallel transport.

I was hoping this definition is equivalent to the usual ones.  
Now you've got me nervous... after all, before I can flame you 
for misunderstanding me, I should be sure what I actually said 
is right!  :-)

My definition is conceptually simple, but it contains
some redundancy... let's squeeze that out and see what's left.

We start with an X-manifold where X = "a hermitian structure".
Each tangent space has a complex inner product h,
whose real part g is the original Riemannian metric, 
and whose imaginary part we call w:

h = g + iw

Each tangent space also has a complex structure J on it.

We want all this stuff to be preserved by parallel transport.
So, at first it seems like we have 3 integrability conditions:

  g, w, and J are preserved by parallel transport

But g is automatically preserved by parallel transport - that's
how the Levi-Civita connection is defined!  

So, there are really just 2 integrability conditions:

  w and J are preserved by parallel transport.

But we can always recover the imaginary part of the inner
product from its real part together with the complex structure:

w(u,v) = -g(u,Jv)

So, there is really just one integrability condition:

  J is preserved by parallel transport.

Now, how does this compare to other definitions of Kaehler
manifold?  Marc Nardmann wrote:

> I assume that you know
>
> (#) that every hermitian metric h on a complex manifold X has a
>     decomposition h = g+iw, where g is a Riemannian metric on X_R
>     (and X_R is the smooth manifold X without its complex structure),
>     and w is a 2-form on X_R;
>
>(#) and that each of h,g,w determines the other two via the R-vector
>     bundle morphism J: T(X_R) \to  T(X_R) given by Jv = iv (where the
>     holomorphic tangent bundle TX is canonically identified as a real
>     vector bundle with T(X_R)). E.g. g(u,v) = w(u,Jv) up to a sign that
>     depends on our definition of hermiticity.
>
> The hermitian metric h = g+iw is K"ahler
>     if and only if w is closed,
>     if and only if J, viewed as a real (1,1)-tensor field on X, is
>     parallel with respect to the Levi-Civita connection of g.

It sounds like he's saying that a Kaehler manifold is a \emph{complex}
manifold for which J is preserved by parallel transport.  My proposed
definition is close, but it doesn't contain the crucial word \emph{complex}.

Can we safely leave it out?  I.e., is any almost complex Riemannian
manifold for which J is preserved by parallel transport automatically
complex???

I don't know.  So, now I'm nervous.

I could try to show by a calculation that if J has vanishing
covariant derivative, it satisfies the integrability condition
that forces it to be a complex structure:

[Ju,Jv] - [u,v] - J[u,Jv] - J[Ju,v] = 0

However, I'm too lazy!  I'm hoping Marc Nardmann or someone 
will step in with either the necessary theorem, or a counterexample.

Btw, there is such a thing as an "almost Kaehler manifold",
which is an almost complex manifold where each tangent space
is equipped with a complex inner product h = g+iw such that the
imaginary part w is a closed 2-form.  But, I don't see why
the existence of these things serves as a counterexample to
my hope.
$$
    
\end{quote}

Then Marc Nardmann confirmed my hope: any almost complex Riemannian
manifold for which J is preserved by parallel transport is
automatically complex, and thus a Kaehler manifold.  He wrote
(in part):

\begin{quote}

$$

John Baez wrote:

 > It sounds like he's saying that a Kaehler manifold is a \emph{complex}
 > manifold for which J is preserved by parallel transport.

Yes. I forgot to discuss this issue in the post you're citing here. In
the stringy context, there's initially just the Riemannian metric, so it
is important to know how e.g. a holonomy condition implies the existence
of a complex structure, as opposed to a mere almost complex structure.
Let's see:

 > My proposed definition
 > is close, but it doesn't contain the crucial word \emph{complex}.
 >
 > Can we safely leave it out?  I.e., is any almost complex Riemannian
 > manifold for which J is preserved by parallel transport automatically
 > complex???
 >
 > I don't know.  So, now I'm nervous.
 >
 > I could try to show by a calculation that if J has vanishing
 > covariant derivative, it satisfies the integrability condition
 > that forces it to be a complex structure:
 >
 > [Ju,Jv] - [u,v] - J[u,Jv] - J[Ju,v] = 0
 >
 > However, I'm too lazy!

It's very easy, so even laziness is no excuse :-). The hard part is
contained in the theorem you're citing here: that an almost complex
structure comes from a complex structure (which is then uniquely
determined) if (and only if) [Ju,Jv] - [u,v] - J[u,Jv] - J[Ju,v] = 0
for all vector fields u,v (in fact, the LHS of the equation is
tensorial, hence well-defined for \emph{vectors}).

We need only the fact that the Levi-Civita connection is torsion-free:

  [Ju,Jv]       -    [u,v]     -    J[u,Jv]       -       J[Ju,v]       = 

&nabla;_{Ju}Jv - &nabla;_{Jv}Ju   -   &nabla;_{u}v + &nabla;_{v}u   -   J(&nabla;_{u}Jv - &nabla;_{Jv}u)   -   J(&nabla;_{Ju}v -&nabla;_{v}Ju)  =

J(&nabla;_{Ju}v) - J(&nabla;_{Jv}u) - &nabla;_{u}v + &nabla;_{v}u    +   &nabla;_{u}v + J(&nabla;_{Jv}u)   -   J(&nabla;_{Ju}v) - &nabla;_{v}u  =  0
$$
    
\end{quote}




\par\noindent\rule{\textwidth}{0.4pt}
<em>"The series is divergent; therefore we may be able to
do something with it"</em> - Oliver Heaviside.  
\par\noindent\rule{\textwidth}{0.4pt}

% </A>
% </A>
% </A>
