
% </A>
% </A>
% </A>
\week{May 20, 2000 }

Various books are coming out to commemorate the millennium.... describing
the highlights of the math we've done so far, and laying out grand dreams 
for the future.  The American Mathematical Society has come out with one:

1) Mathematics: Frontiers and Perspectives, edited by Vladimir Arnold, 
Michael Atiyah, Peter Lax and Barry Mazur, AMS, Providence, Rhode 
Island, 2000.  

This contains 30 articles by bigshots like Chern, Connes, Donaldson,
Jones, Lions, Manin, Mumford, Penrose, Smale, Vafa, Wiles and Witten.
I haven't actually read it yet, but I want to get ahold of it.  

Springer Verlag is coming out with one, too:

2) Mathematics Unlimited: 2001 and Beyond, edited by Bjorn Engquist
and Wilfried Schmid, Springer Verlag, New York, 2000.  

It should appear in the fall.  

I don't know what the physicists are doing along these lines.  The 
American Physical Society has a nice timeline of 20th century physics 
on their website:

3) The American Physical Society: A Century of Physics, available
at <a href = "http://timeline.aps.org/APS/home_HighRes.html">
http://timeline.aps.org/APS/home_HighRes.html</a>

But I don't see anything about books.

One reason I haven't been doing many This Week's Finds lately is
that I've been buying and then moving into a new house.  Another
is that James Dolan and I have been busily writing our own millennial 
pontifications, which will appear in the Springer-Verlag book:

4) John Baez and James Dolan, From finite sets to Feynman diagrams,
preprint available as <a href = "http://arXiv.org/abs/math.QA/0004133">math.QA/0004133</a>

So let me talk about this stuff a bit....

As usual, the underlying theme of this paper is categorification.  
I've talked about this a lot already - e.g. in "<A HREF = "week121.html">week121</A>" - so I'll
assume you remember that when we categorify, we use this analogy:


\begin{verbatim}

SET THEORY                          CATEGORY THEORY

elements                            objects                       
equations between elements          isomorphisms between objects        
sets                                categories                    
functions                           functors                      
equations between functions         natural isomorphisms between functors  
\end{verbatim}
    
to take interesting \emph{equations} and see them as shorthand for even
more interesting \emph{isomorphisms}.    

To take a simple example, consider the laws of basic arithmetic, like
a+b = b+a or a(b+c) = ab+ac.  We usually think of these as
\emph{equations} between \emph{elements} of the \emph{set} of
natural numbers.  But really they arise from \emph{isomorphisms}
between \emph{objects} of the \emph{category} of finite sets.

For example, if we have finite sets a and b, and we use a+b to 
denote their disjoint union, then there is a natural isomorphism 
between a+b and b+a.   Moreover, this isomorphism is even sort of 
interesting!  For example, suppose we use 1 to denote a set consisting 
of one dot, and 2 to denote a set of two dots.  Then the natural 
isomorphism between 1+2 and 2+1 can be visualized as the process 
of passing one dot past two, like this:


\begin{verbatim}

                 .       .   .
                  \     /   /
                   \   /   /
                    \ /   /
                     /   /
                    / \ /
                   /   /
                  /   / \
                 /   /   \
                /   /     \
               /   /       \
              .   .         .

\end{verbatim}
    
This may seem like an excessively detailed "picture proof"
that 1+2 indeed equals 2+1, perhaps suitable for not-too-bright
kindergarteners.  But in fact it's just a hop, skip and a jump from here
to heavy-duty stuff like the homotopy groups of spheres.  I sketched how
this works in "<A HREF = "week102.html">week102</A>" so I
won't do so again here.  The point is, after we categorify, elementary
math turns out to be pretty powerful!

Now, let me make this idea of "categorifying the natural numbers"
a bit more precise.  Let FinSet stand for the category whose objects
are finite sets and whose morphisms are functions between these.
If we "decategorify" this category by forming the set of isomorphism
classes of objects, we get N, the natural numbers.  

All the basic arithmetic operations on N come from operations on FinSet.
I've already noted how addition comes from disjoint union.  Disjoint
union is a special case of what category theorists call the
"coproduct", which makes sense for a bunch of categories - see
"<A HREF = "week99.html">week99</A>" for the general
definition.  Similarly, multiplication comes from the Cartesian product
of finite sets, which is a special case of what category theorists call
the "product".  To get the definition of a product, you just
take the definition of a coproduct and turn all the arrows around.
There are also nice category-theoretic interpretations of the numbers 0 and 
1, and all the basic laws 0, 1, addition and multiplication.  Exponentiation 
too! 

Combinatorists have lots of fun thinking about how to take equations 
in N and prove them using explicite isomorphisms in FinSet - they call
such a proof a "bijective proof".  To read more about this, try:

5) James Propp and David Feldman, Producing new bijections from old,
Adv. Math. 113 (1995), 1-44.  Also available at 
<a href = "http://www.math.wisc.edu/~propp/articles.html">http://www.math.wisc.edu/~propp/articles.html</a>

6) John Conway and Peter Doyle, Division by three.  Available at
<a href = "http://math.dartmouth.edu/~doyle/docs/three/">http://math.dartmouth.edu/~doyle/docs/three/</a>

The latter article studies this question: if I give you an isomorphism 
between 3x and 3y, can you construct a isomorphism between x and y?  
Here of course x and y are finite sets, 3 is any 3-element set, and 
multiplication means Cartesian product.  Of course you can prove
an isomorphism exists, but can you \emph{construct} one in a natural way 
- i.e., without making any random choices?   The history of this 
puzzle turns out to be very interesting.  But I don't want to give 
away the answer!   See if you can do it or not.

Anyway, having categorified the natural numbers, we might be inclined
to go on and categorify the integers.  Can we do it?  In other words:
can we find something like the category of finite sets that includes
"sets with a negative number of elements"?  There turns out be an 
interesting literature on this subject:

7) Daniel Loeb, Sets with a negative number of elements, Adv. Math. 91 
(1992), 64-74.   

8) S. Schanuel, Negative sets have Euler characteristic and dimension, 
Lecture Notes in Mathematics 1488, Springer Verlag, Berlin, 1991, 
pp. 379-385.

9) James Propp, Exponentiation and Euler measure, available as 
<A HREF = "http://www.arXiv.org/abs/math.CO/0204009">
math.CO/0204009</A>.

10) Andre Joyal, Regle des signes en algebre combinatoire, Comptes 
Rendus Mathematiques de l'Academie des Sciences, La Societe Royale 
du Canada, VII (1985), 285-290.

See also "<A HREF = "week102.html">week102</A>" for more....

But I don't want to talk about negative sets right now!   Instead, 
I want to talk about \emph{fractional} sets.  It may seem odd to tackle
division before subtraction, but historically, the negative numbers 
were invented quite a bit \emph{after} the nonnegative rational numbers.
Apparently half an apple is easier to understand than a negative apple!  
This suggests that perhaps `sets with fractional cardinality' are 
simpler than `sets with negative cardinality'.  

The key is to think carefully about the meaning of division.  The 
usual way to get half an apple is to chop one into "two equal parts".
Of course, the parts are actually NOT EQUAL - if they were, there 
would be only one part!  They are merely ISOMORPHIC.  This suggests
that categorification will be handy.  

Indeed, what we really have is a Z/2 symmetry group acting on the 
apple which interchanges the two isomorphic parts.  In general, if a
group G acts on a set S, we can "divide" the set by the group by
taking the quotient S/G, whose points are the orbits of the action.  
If S and G are finite and G acts freely on S, this construction
really corresponds to division, since the cardinality |S/G| is
equal to |S|/|G|.   However, it is crucial that the action be free.

For example, why is 6/2 = 3?  We can take a set S consisting of six
dots in a row:


\begin{verbatim}

                   o    o    o    o    o   o
\end{verbatim}
    
let G = Z/2 act freely by reflections, and identify all the elements in 
each orbit to obtain 3-element set S/G.   Pictorially, this amounts to 
folding the set S in half, so it is not surprising that |S/G| = |S|/|G| 
in this case.  Unfortunately, if we try a similar trick starting with a 
5-element set:


\begin{verbatim}

                     o    o    o    o   o
\end{verbatim}
    
it fails miserably!  We don't obtain a set with 2.5 elements, because 
the group action is not free: the point in the middle gets mapped to 
itself.   So to define "sets with fractional cardinality", we need a 
way to count the point in the middle as "half a point".

To do this, we should first find a better way to define the quotient of
S by G when the action fails to be free.  Following the policy of
replacing equations by isomorphisms, let us define the "weak
quotient" S//G to be the category with elements of S as its
objects, with a morphism g: s \to  s' whenever g(s) = s', and with
composition of morphisms defined in the obvious way.

Next, let us figure out a good way to define the "cardinality" of a
category.   Pondering the examples above leads us to the following 
recipe: for each isomorphism class of objects we pick a representative x 
and compute the reciprocal of the number of automorphisms of this object;
then we sum over isomorphism classes. 

It is easy to see that with this definition, the point in the middle
of the previous picture gets counted as `half a point' because it has
two automorphisms, so we get a category with cardinality 2.5.  In general,

                        |S//G| = |S|/|G|    

whenever G is a finite group acting on a finite set S.  This formula 
is a simplified version of `Burnside's lemma', so-called because it
is due to Cauchy and Frobenius.  Burnside's lemma gives the cardinality 
of the ordinary quotient.   But the weak quotient is nicer, so Burnside's
lemma simplifies when we use weak quotients.
 
Now, the formula for the cardinality of a category makes sense even
for some categories that have infinitely many objects - all we need is
for the sum to make sense.  So let's try to compute the cardinality
of the category of finite sets!   Since any n-element set has n! 
automorphisms (i.e. permutations), we get following marvelous formula:

                          |FinSet| = e 

This turns out to explain lots of things about the number e.  

Now, a category all of whose morphisms are isomorphisms is called
a "groupoid".  Any category C has an underlying groupoid 
C_{0} with the
same objects but only the isomorphisms as morphisms.  The cardinality 
of a category C always equals that of its underlying groupoid C_{0}.   
This suggests that this notion should really be called "groupoid 
cardinality.   If you're a fan of n-categories, this suggests that 
we should generalize the concept of cardinality to n-groupoids, or 
even \omega -groupoids.   And luckily, we don't need to understand 
\omega -groupoids very well to try our hand at this!  Omega-groupoids 
are supposed to be an algebraic way of thinking about topological spaces 
up to homotopy.  Thus we just need to invent a concept of the `cardinality' 
of a topological space which has nice formal properties and which agrees 
with the groupoid cardinality in the case of homotopy 1-types.  In fact, 
this is not hard to do.  We just need to use the homotopy groups 
\pi _{k}(X) of the space X.  

So: let's define the "homotopy cardinality" of a 
topological space X to be the alternating product 

     |X| &nbsp; = &nbsp; |\pi _{1}(X)|^{-1} &nbsp; 
                |\pi _{2}(X)|  &nbsp;
|\pi _{3}(X)|^{-1} ....

when X is connected and the product converges; if X is not connected, 
let's define its homotopy cardinality to be the sum of the homotopy 
cardinalities of its connected components, when the sum converges.  
We call spaces with well-defined homotopy cardinality "tame".  The 
disjoint union or Cartesian product of tame spaces is again tame, 
and we have  

                     |X + Y| = |X| + |Y| ,   

                     |X \times  Y| = |X| \times  |Y| 

just as you would hope.  

Even better, homotopy cardinality gets along well with fibrations,
which we can think of as `twisted products' of spaces.  Namely, if 

                        F \to  X \to  B 

is a fibration and the base space B is connected, we have

                       |X| = |F| \times  |B|  

whenever two of the three spaces in question are tame (which implies
the tameness of the third).  

As a fun application of this fact, recall that any topological group G
has a "classifying space" BG, meaning a 
space with a principal G-bundle over it

                       G \to  EG \to  BG  

whose total space EG is contractible.  I described how to construct
the classifying space in "<A HREF =
"week117.html">week117</A>", at least in the case of a discrete
group G, but I didn't say much about why it's so great.  The main
reason it's great is that \emph{any} G-bundle over \emph{any}
space is a pullback of the bundle EG over BG.  But right now, what I
want to note is that since EG is contractible it is tame, and |EG| =
1.  Thus G is tame if and only if BG is, and

                          |BG| = 1 / |G|

so we can think of BG as the "reciprocal" of G!

This idea is already lurking behind the usual approach to "equivariant 
cohomology".  Suppose X is a space on which the topological group G
acts.  When the action of G on X is free, it is fun to calculate 
cohomology groups (and other invariants) of the quotient space X/G.
When the action is not free, this quotient can be very pathological, 
so people usually replace it by the "homotopy quotient" X//G, which is
defined as (EG x X)/G.  This is like the ordinary quotient but with
equations replaced by homotopies.  And there is a fibration

                         X \to  X//G \to  BG , 

so when X and G are tame we have

                   |X//G| = |X| \times  |BG| = |X|/|G|   

just as you would hope!

Now in the paper, Jim and I go on to talk about how all these ideas
can be put to use to give a nice explanation of the combinatorics of
Feynman diagrams.  But I don't want to explain all that stuff here - 
then you wouldn't need to read the paper!  Instead, I just want to 
point out something mysterious about homotopy cardinality.

Homotopy cardinality is formally very similar to Euler characteristic.
The Euler characteristic &chi;(X) is given by the alternating sum

  &chi;(X) = dim(H_{0}(X)) - dim(H_{1}(X)) + dim(H_{2}(X)) - ....

whenever the sum converges, where H_{n}(X) is a vector space over
the rational numbers called the nth rational homology group of X.
Just as for homotopy cardinality, we have

                     &chi;(X + Y) = &chi;(X) + &chi;(Y),

                     &chi;(X \times  Y) = &chi;(X) \times  &chi;(Y)

and more generally, whenever 

                        F \to  X \to  B

is a fibration and the base space B is connected, we have

                     &chi;(X) = &chi;(F) \times  &chi;(B)

whenever any two of the spaces have well-defined Euler characteristic,
which implies that the third does too (unless I'm confused).

So Euler characteristic is a lot like homotopy cardinality.  But
not many spaces have \emph{both} well-defined homotopy cardinality \emph{and} 
well-defined Euler characteristic.  So they're like Jekyll and Hyde - 
you hardly ever see them in the same place at the same time, so you 
can't tell if they're really the same guy.

But there are some weird ways to try to force the issue and compute
both quantities for certain spaces.  For example, suppose G is a
finite group.  Then we can build BG starting from a simplicial set 
with 1 nondegenerate 0-simplex, |G|-1 nondegenerate 1-simplices, 
(|G|-1)^{2} nondegenerate 2-simplices, and so on.  If there were only
finitely many nondegenerate simplices of all dimensions, we could 
compute the Euler characteristic of this space as the alternating sum 
of the numbers of such simplices.  So let's try doing that here!  
We get:

           c(BG) = 1  -  (|G|-1)  +  (|G|-1)^{2}  - ....

Of course the sum diverges, but if we go ahead and use the geometric
formula anyway, we get

                    c(BG) = 1/|G|

which matches our previous (rigorous) result that

                    |BG| = 1/|G|       

So maybe they're the same after all!   There are similar calculations like 
this in James Propp's paper "Exponentiation and Euler characteristic",
referred to above... though he uses a slightly different notion of Euler
characteristic, due to Schanuel.  Clearly something interesting is going
on with these "divergent Euler characteristics".  For appearances of this
sort of thing in physics, see:

11) Matthias Blau and George Thompson, N = 2 topological gauge 
theory, the Euler characteristic of moduli spaces, and the Casson 
invariant, Comm. Math. Phys. 152 (1993), 41-71. 

and the references therein.  (I discussed this paper a bit in "<A HREF = "week51.html">week51</A>".) 

However, there are still challenging tests to the theory that homotopy
cardinality and Euler characteristic are secretly the same.  Here's a
puzzle due to James Dolan.  Consider a Riemann surface of genus g > 1.
Such a surface has Euler characteristic 2 - 2g, but such a surface also
has vanishing homotopy groups above the first, which implies that it's
BG for G equal to its fundamental group.  If homotopy cardinality and
Euler characteristic were the same, this would imply

                 |G| = 1/|BG| = 1/c(S) = 1/(2 - 2g)

But the fundamental group G is infinite!  What's going on?  

Well, I'm actually sort of glad that 1/(2 - 2g) is \emph{negative}.   
Sometimes a divergent series of positive integers can be cleverly
summed up to give a negative number.  The simplest example is the
geometric series 

             1 + 2 + 4 + 8 + 16 + ... = 1/(1 - 2) = -1

but in "<A HREF = "week126.html">week126</A>" I talked about a more sophisticated example that
is very important in string theory:

             1 + 2 + 3 + 4 + 5 + ... = &zeta;(-1) = -1/12

So maybe some similar trickery can be used to count the elements
of G and get a divergent sum that can be sneakily evaluated to 
obtain 1/(2 - 2g).  Of course, even if we succeed in doing this,
the skeptics will rightly question the significance of such tomfoolery.
But there is sometimes a lot of profound truth lurking in these 
bizarre formal manipulations, and sometimes if you examine what's 
going on carefully enough, you can discover cool stuff.

To wrap up, let me mention an interesting paper on the foundations
of categorification:

12) Claudio Hermida, From coherent structures to universal properties,
available at <a href = "http://www.cs.math.ist.utl.pt/cs/s84/claudio.html">http://www.cs.math.ist.utl.pt/cs/s84/claudio.html</a>

and also two papers about 2-groupoids and topology:

13) K. A. Hardie, K. H. Kamps, R. W. Kieboom, A homotopy bigroupoid of a
topological space, in: Categorical Methods in Algebra and Topology, pp.
209-222, Mathematik-Arbeitspapiere 48, Universitaet Bremen, 1997.   Appl.
Categ. Structures, to appear.

K. A. Hardie, K. H. Kamps, R. W. Kieboom, A homotopy 2-groupoid of a Hausdorff
space, preprint.

I would talk about these if I had the energy, but it's already past
my bed-time.  Good night!


\par\noindent\rule{\textwidth}{0.4pt}

\textbf{Addenda:}

Toby Bartels had some interesting things to say about this issue
of This Week's Finds.  Here is my reply, which quotes some of his
remarks.... 

\begin{quote}

\begin{verbatim}

Toby Bartels (toby@ugcs.caltech.edu) wrote:

>>3) The American Physical Society: A Century of Physics, available
>>at <a href = "http://timeline.aps.org/APS/home_HighRes.html">http://timeline.aps.org/APS/home_HighRes.html</a>

>I like how they make the famous picture of Buzz Aldrin,
>the one that everyone thinks is a picture of Neil Armstrong,
>into a picture of Neil Armstrong after all:
>"Here he is reflected in Buzz Aldrin's visor.".
\end{verbatim}
    

Heh.  Sounds like something a doting grandmother would say!


\begin{verbatim}

>> 5) John Conway and Peter Doyle, Division by three.  
>> <a href = "http://math.dartmouth.edu/~doyle/docs/three/three/three.html">http://math.dartmouth.edu/~doyle/docs/three/three/three.html</a>

>>The latter article studies this question: if I give you an isomorphism 
>>between 3x and 3y, can you construct a isomorphism between x and y?  

>The answer must be something that won't work
>if 3 is replaced by an infinite cardinal.
>That said, I can't even figure out how to divide by 2!
>If I take the 3 copies of X or Y and put them on top of each other,
>I get a finite, 2coloured, 3valent, nonsimple, undirected graph.
>I remember from combinatorics that the 2 colours of
>a finite, 2coloured, simple, undirected graph of fixed valency
>are equipollent, but I can't remember the bijective proof.
>(Presumably it can be adopted to nonsimple graphs.)
\end{verbatim}
    

It's a tricky business.  Let me quote from the above article:

\begin{quote}
<h3> History </h3>
A proof that it is possible to divide by two was presented by Bernstein in
his Inaugural Dissertation of 1901, which appeared in Mathematische Annallen
in 1905; Bernstein also indicated how to extend his results to division
by any finite n, but we are not aware of anyone other than Bernstein himself
who ever claimed to understand this argument. In 1922 Sierpinski 
published a simpler proof of division by two, and he worked hard to extend
his method to division by three, but never succeeded.

In 1927 Lindenbaum and Tarski announced, in an infamous paper that
contained statements (without proof) of 144 theorems of set theory, that
Lindenbaum had found a proof of division by three. Their failure to give any
hint of a proof must have frustrated Sierpinski, for it appears that twenty
years later he still did not know how to divide by three.  Finally,
in 1949, in a paper `dedicated to Professor Waclaw Sierpinski in celebration
of his forty years as teacher and scholar', Tarski published a proof.
In this paper, Tarski explained that unfortunately he couldn't remember how
Lindenbaum's proof had gone, except that it involved an argument like the
one Sierpinski had used in dividing by two, and another lemma, due to
Tarski, which we will describe below. Instead of Lindenbaum's proof, he gave
another.

Now when we began the investigations reported on here, we were aware that
there was a proof in Tarski's paper, and Conway had even pored over it at
one time or another without achieving enlightenment. The problem was closely
related to the kind of question John had looked at in his thesis, and it was
also related to work that Doyle had done in the field of bijective
combinatorics. So we decided that we were going to figure out what the heck
was going on. Without too much trouble we figured out how to divide by two.
Our solution turned out to be substantially equivalent to that of
Sierpinski, though the terms in which we will describe it below will not
much resemble Sierpinski's. We tried and tried and tried to adapt the method
to the case of dividing by three, but we kept getting stuck at the same
point in the argument. So finally we decided to look at Tarski's paper, and
we saw that the lemma Tarski said Lindenbaum had used was precisely what we
needed to get past the point we were stuck on! So now we had a proof of
division by three that combined an argument like that Sierpinski used in
dividing by two with an appeal to Tarski's lemma, and we figured we must
have hit upon an argument very much like that of Lindenbaum's. This is the
solution we will describe here: Lindenbaum's argument, after 62 years.
\end{quote}


$$

>>So: let's define the "homotopy cardinality" of a topological space X to
>>be the alternating product |X| = \prod_{i>0} |\pi _i(X)|^{(-1)^i}
>>when X is connected and the product converges;

>What about divergence to 0?
>If \pi _i(X) is infinite for some odd i but no even i,
>can we say |X| is 0?
$$
    

Well, we can, but we might regret it later.  In a sense 0 is no
better than \infty  when one is doing products, so if you allow 
0 as a legitimate value for a homotopy cardinality, you should 
allow \infty , but if you allow both, you get in trouble when 
you try to multiply them.  This dilemma is familiar from the case 
of infinite sums (where +\infty  and -\infty  are the culprits), 
and the resolution seems to be either:

<UL>
<LI> disallow both 0 and \infty  as legitimate answers for the above
product,
</UL>

or

<UL>
<LI>allow both but then be extra careful when stating your theorems
so that you don't run into problems.  
</UL>


$$

>>As a fun application of this fact, recall that any topological group G
>>has a "classifying space" BG, meaning a space with a principal G-bundle 
>>over it G \to  EG \to  BG
>>whose total space EG is contractible.   I described how to construct
>>the classifying space in "week117", at least in the case of a discrete
>>group G, but I didn't say much about why it's so great.  The main 
>>reason it's great is that \emph{any} G-bundle over \emph{any} space is a pullback 
>>of the bundle EG over BG.  But right now, what I want to note is that 
>>since EG is contractible it is tame, and |EG| = 1.  Thus G is tame if 
>>and only if BG is, and |BG| = 1 / |G|,
>>so we can think of BG as the `reciprocal' of G!

>OTOH, G is already a kind of reciprocal of itself.
>If G is a discrete group, it's a topological space
>with |G|_{homotopy} = |G|_{set}.
>But G is also a groupoid with 1 object,
>and |G|_{groupoid} = 1 / |G|_{set}.
>So, |G|_{homotopy} |G|_{groupoid} = 1.
$$
    

Believe it or not, you are reinventing BG!   A groupoid can be
reinterpreted as a space with vanishing homotopy groups above the
first, and if you do this to the groupoid G, you get BG.  

More generally:

Recall that we can take a pointed space X and form a pointed space
LX of loops in X that start and end at the basepoint.  This clearly 
has

\pi _{n+1}(X) = \pi _{n}(LX)

so if X is connected and tame we'll have

|LX| = 1/|X|

Now with a little work you can make LX (or a space homotopy-equivalent
to it!) into a topological group with composition of loops as the product.
And then it turns out that BLX is homotopy equivalent to X when X is
connected.  Conversely, given a topological group G, LBG is homotopy 
equivalent to G.  
So what we're seeing is that topological groups and connected pointed 
spaces are secretly the same thing, at least from the viewpoint of 
homotopy theory.  In topology, few things are as important as this fact.
But what's really going on here?   Well, to go from a topological group G 
to a connected pointed space, you have to form BG, which has all the same 
homotopy groups but just pushed up one notch:
\pi _{n+1}(BG) = \pi _{n}(G)               
And to go from a connected point space X to a topological group, you have 
to form LX, which has all the same homotopy groups but just pushed down one
notch:
\pi _{n-1}(LX) = \pi _{n}(X)              
This is actually the trick you are playing, in slight disguise.

And the real point is that a 1-object \omega -groupoid can be 
reinterpreted as an \omega -groupoid by forgetting about the 
object and renaming all the j-morphisms "(j-1)-morphisms".  

See?  When you finally get to the bottom of it, this "BG" business
is just a silly reindexing game!!!   Of course no textbook can admit 
this openly - partially because they don't talk about \omega -groupoids.


\begin{verbatim}

>>So Euler characteristic is a lot like homotopy cardinality.  But
>>not many spaces have <em>both</em> well-defined homotopy cardinality <em>and</em> 
>>well-defined Euler characteristic.  So they're like Jekyll and Hyde - 
>>you hardly ever see them in the same place at the same time, so you 
>>can't tell if they're really the same guy.

>So, are they ever both defined but different?
\end{verbatim}
    

I don't recall any examples where they're both finite, but different.  
I know very few cases where they're both finite!   How about the point?
How about the circle?  How about the 2-sphere?  I leave you to ponder 
these cases.


$$

>>However, there are still challenging tests to the theory that homotopy
>>cardinality and Euler characteristic are secretly the same.  Here's a
>>puzzle due to James Dolan.  Consider a Riemann surface of genus g > 1.
>>Such a surface has Euler characteristic 2 - 2g, but such a surface also
>>has vanishing homotopy groups above the first, which implies that it's
>>BG for G equal to its fundamental group.  If homotopy cardinality and
>>Euler characteristic were the same, this would imply
>>
>>                 |G| = 1/|BG| = 1/&chi;(S) = 1/(2 - 2g).
>>
>>But the fundamental group G is infinite!  What's going on?  

>This doesn't seem too surprising.  1/(2 - 2g) is also infinite.
>Just use the geometric series in reverse:
>
>                  1/(2 - 2g) = (1/2) &sum;_{i} g^{i}, 
>
>which diverges since g > 1.
$$
    

Well, what I really want is a way of counting elements of the fundamental
group of the surface S which gives me a divergent sum that I can cleverly
sum up to get 1/(2 - 2g).  

\end{quote}

Later, my wish above was granted by Laurent Bartholdi and Danny Ruberman!
People have already figured out how to count the number of elements
in the fundamental group of a Riemann surface, resum, and get 1/(2 - 2g) 
in a nice way.  Here are two references:

14) William J. Floyd and Steven P. Plotnick, Growth functions on 
Fuchsian groups and the Euler characteristic, Invent. Math. 88
(1987), 1-29.

15) R. I. Grigorchuk, Growth functions, rewriting systems and Euler 
characteristic, Mat. Zametki 58 (1995), 653-668, 798.

You can read more about Euler characteristic and homotopy cardinality
here:
16) John Baez, Euler characteristic versus homotopy cardinality, 
lecture at the Fields Institute Program on Applied Homotopy Theory, 
September 20, 2003.  Available in PDF form at <A HREF = "http://www.math.ucr.edu/home/baez/cardinality/">http://www.math.ucr.edu/home/baez/cardinality/</A>






 \par\noindent\rule{\textwidth}{0.4pt}

<em>The imaginary expression \sqrt -a and the negative expression
-b resemble each other in that each one, when they seem the solution
of a problem, they indicate that there is some inconsistency or
nonsense.</em> - Augustus De Morgan, 1831.
\par\noindent\rule{\textwidth}{0.4pt}

% </A>
% </A>
% </A>
