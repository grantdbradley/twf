
% </A>
% </A>
% </A>
\week{April 8, 2007 }

As you may recall, I'm telling a long story about symmetry, geometry, 
and algebra.  Some of this tale is new work done by James Dolan, Todd 
Trimble and myself.  But a lot of it is old work by famous people which
deserves a modern explanation. 

A great example is Felix Klein's "Erlangen program" - a plan for
reducing many sorts of geometry to group theory.  Many people tip
their hat to the Erlanger program, but few seem to deeply understand 
it, and even fewer seem to have read what Klein actually wrote about it!   

<div align = center>
<a href = "http://www-history.mcs.st-andrews.ac.uk/PictDisplay/Klein.html">
<img src = "http://upload.wikimedia.org/wikipedia/commons/e/e2/Felix_Klein.jpeg">
% </a>
</div>

The problem goes back a long ways.  In 1871, while at G&ouml;ttingen, 
Klein worked on non-Euclidean geometry, and showed that hyperbolic
geometry was consistent if and only if Euclidean geometry was.
In the process, he must have thought hard about the role of symmetry
groups in geometry.  When he was appointed professor at Erlangen 
in 1872, he wrote a lecture outlining his "Erlanger Programm" for
reducing geometry to group theory.  

But, he didn't actually give this lecture as his inaugural speech!
He spoke about something else.

So, nobody ever heard him announce the Erlangen program.  And, until
recently, the lecture he wrote was a bit hard to find.  Luckily, now 
you can get it online:

1) Felix Klein, Vergleichende Betrachtungen ueber neuere geometrische 
Forschungen, Verlag von Andreas Deichert, Erlangen, 1872.  Also 
available at the University of Michigan Historical Mathematics Collection, 
<a href = "http://www.hti.umich.edu/cgi/t/text/text-idx?c=umhistmath;idno=ABN7632">http://www.hti.umich.edu/cgi/t/text/text-idx?c=umhistmath;idno=ABN7632</a>

Even better, Johan Ernst Mebius has recently prepared an HTML version,
with links to the above version:

2) Johan Ernst Mebius, Felix Klein's Erlanger Programm,
<a href = "http://www.xs4all.nl/~jemebius/ErlangerProgramm.htm">http://www.xs4all.nl/~jemebius/ErlangerProgramm.htm</a>

But what if you have the misfortune of only reading English, not 
German?  Until now the only translation was quite hard to obtain:

3) Felix Klein, A comparative review of recent researches in geometry, 
trans. M. W. Haskell, Bull. New York Math. Soc. 2, (1892-1893), 215-249.

In case you're wondering, the "New York Mathematical
Society" no longer exists!  It was founded in 1888, but in 1894
it went national and became the American Mathematical Society.

Luckily, after Thomas Love pointed out the existence of this old
translation, Chris Hillman was able to get ahold of it and scan it in!
Then Robin Houston created a PDF file of the whole thing, and Lukas-Fabian
Moser created a DjVu file.  Then Nitin C. Rughoonauth took the marvelous
step of putting it into LaTeX!  So now, you can read Klein's paper in English 
here:

4) The Erlangen program, <a href = "http://math.ucr.edu/home/baez/erlangen/">http://math.ucr.edu/home/baez/erlangen/</a>

English-speakers can read more about the Erlangen program here:

5) Felix Klein, Elementary Mathematics from an Advanced Standpoint:
Geometry, part 3: Systematic discussion of geometry and its foundations,
Dover, New York, 1939.

Luckily Dover keeps its books in print!

For more on the Erlangen program, try these:

6) Garrett Birkhoff and M. K. Bennett, Felix Klein and his
"Erlanger Programm", in History and Philosophy of Modern
Mathematics, eds.  W. Aspray and P. Kitcher, Minnesota
Stud. Philos. Sci. XI, University of Minnesota Press, Minneapolis,
1988, pp. 145-176.

7) Hans A. Kastrup, The contributions of Emmy Noether, Felix Klein and
Sophus Lie to the modern concept of symmetries in physical systems,
in Symmetries in Physics (1600-1980), ed. M. G. Doncel, World Scientific, 
Singapore, 1987, pp. 113-163.

8) I. M. Yaglom, Felix Klein and Sophus Lie: Evolution of the
Idea of Symmetry in the Nineteenth Century, trans. S. Sossinsky,
Birkhauser, Boston, 1988.

For more about Klein, try "<A HREF =
"week213.html">week213</A>" and this little biography:

9) MacTutor History of Mathematics Archive, Felix Klein,
<a href = "http://www-history.mcs.st-andrews.ac.uk/Biographies/Klein.html">http://www-history.mcs.st-andrews.ac.uk/Biographies/Klein.html</a>

But what does the Erlangen program actually amount to, in the language
of modern mathematics?  This will take a while to explain, so the best 
thing is to dive right in.  

Last week in the Tale of Groupoidification I tried to explain two slogans:

<div align = "center">
                  GROUPOIDS ARE LIKE 'SETS WITH SYMMETRIES'
</div>
<div align = "center">
          SPANS OF GROUPOIDS ARE LIKE 'INVARIANT WITNESSED RELATIONS'
</div>

They're a bit vague; they're mainly designed to give you enough intuition
to follow the next phase of the Tale, which is all about how:

<div align = "center">
                   GROUPOIDS GIVE VECTOR SPACES
</div>
<div align = "center">
              SPANS OF GROUPOIDS GIVE LINEAR OPERATORS
</div>
But before the next phase, I need to say a bit about how groupoids
and spans of groupoids fit into Klein's Erlangen program.

Groupoids are a modern way to think about symmetries.   A more 
traditional approach would use a group acting as symmetries of
some set.  And the most traditional approach of all, going back to
Galois and Klein, uses a group acting \emph{transitively} on a set.  

So, let me explain the traditional approach, and then relate it
to the modern one.

I hope you know what it means for a group G to "act" on a set X. 
It means that for any element x of X and any guy g in G, we get a 
new element gx in X.  We demand that 

1x = x

and 

g(hx) = (gh)x.

More precisely, this is a "left action" of G on X, since we write
the group elements to the left of x.  We can also define right actions,
and someday we may need those too.

We say an action of a group G on a set X is "transitive" if given 
any two elements of X, there's some guy in G mapping the first 
element to the second.  In this case, we have an isomorphism of sets

X = G/H

for some subgroup H of G.

For example, suppose we're studying a kind of geometry where the symmetry
group is G.  Then X could be the set of figures of some sort: points, or 
lines, or something fancier.  If G acts transitively on X, then all
figures of this sort "look alike": you can get from any one to any other 
using a symmetry.  This is often the case in geometry... but not always.  

Suppose G acts transitively on X.   Pick any figure x of type X and let 
H be its "stabilizer": the subgroup consisting of all guys in G that map 
x to itself.  Then we get a one-to-one and onto map

f: X \to  G/H

sending each figure gx in X to the equivalence class [g] in G/H.  

If you haven't seen this fact before, you should definitely prove it -
it's one of the big ways people use symmetry!

Here's one kind of thing people do with this fact. The 3d rotation 
group G = SO(3) acts on the sphere X = S^{2}, and the stabilizer of 
the north pole is the 2d rotation group H = SO(2), so the sphere is 
isomorphic to G/H = SO(3)/SO(2).   The same sort of result holds in any
dimension, and we can use it to derive facts about spheres from facts 
about rotation groups, and vice versa.

A grander use of this fact is to set up a correspondence between 
sets on which G acts transitively and subgroups of G.  This is
one of the principles lurking behind Galois theory.

Galois applied this principle to number theory - see "<A HREF =
"week201.html">week201</A>" for details.  But, it really has
nothing particular to do with number theory!  In his Erlangen program,
Klein applied it to geometry.

Klein's goal was to systematize a bunch of different kinds of non-Euclidean
geometry.  Each kind of geometry he was interested in had a different
group of symmetries.   For example:  

<ul>
<li>
 n-dimensional <a href =
 "http://en.wikipedia.org/wiki/Spherical_geometry">spherical
 geometry</a> has the rotation group SO(n+1) as symmetries.  (Or, if
 you want to include reflections, the bigger group O(n+1).)
</li>

<li>
 n-dimensional <a href =
 "http://en.wikipedia.org/wiki/Euclidean_geometry">Euclidean
 geometry</a> has the Euclidean group ISO(n) as symmetries.  (This
 group is built from rotations in SO(n) together with translations in
 R^{n}.) 
</li>

<li>
 n-dimensional <a href =
 "http://en.wikipedia.org/wiki/Hyperbolic_geometry">hyperbolic
 geometry</a> has the group SO(n,1) as symmetries.  (This group also
 shows up in special relativity under the name of the "Lorentz
 group": it acts on the "mass hyperboloid", and that's
 how hyperbolic geometry shows up in special relativity.)  
</li>

<li>
 n-dimensional <a href =
 "http://en.wikipedia.org/wiki/Projective_geometry">projective
 geometry</a> has the group SL(n+1) as symmetries.  (This group
 consists of (n+1)\times (n+1) matrices with determinant 1.  
 Scalar multiples of the
 identity act trivially on projective space, so it's actually better
 to use the "projective general linear group" PGL(n+1),
 consisting of invertible matrices mod scalars.  But, this has
 the same Lie algebra as SL(n+1), so people are often a bit slack about 
 which group they use.)
    
</li> 
</ul>

The details here don't matter much yet; the point is that there are lots
of interesting kinds of geometry, with interesting symmetry groups!

Klein realized that in any kind of geometry like this, a "type of
figure" corresponds to a set on which G acts transitively.  Here 
a "figure" could be a point, a line, a plane, or something 
much fancier.  Regardless of the details, the set of all figures of 
the same type can be written as G/H, and G acts transitively on this set.

The really cool part is that we can use Klein's idea to
\emph{define} a geometry for any group G.  To do this, we just say
that \emph{every} subgroup H of G gives rise to a type of figure.  So, we
work out all the subgroups of G.  Then, we work out all the incidence
relations - relations like "a point lies on a line".  To do
this, we take two sets of figures, say

X = G/H 

and 

Y = G/K

and find all the invariant relations between them: that is, subsets of
X \times  Y preserved by all the symmetries.  I'll say more about how to
do this next time - we can use something called "double
cosets".  In nice cases, like when G is a simple Lie group and H
and K are so-called "parabolic" subgroups, these let us express
all the invariant relations in terms of finitely many
"atomic" ones!  So, we can really carry out Klein's program
of thoroughly understanding geometry starting from groups - at least
in nice cases.

In short, group actions - especially transitive ones - are a traditional 
and very powerful way of using symmetry to tackle lots of problems.  

So, to bridge the gap between the traditional and the new, I should
explain how group actions give groupoids.  I'll show you that: 

<div align = "center">
             A GROUPOID EQUIPPED WITH CERTAIN EXTRA STUFF IS<br>
                     THE SAME AS A GROUP ACTION
</div>

It's not very hard to get a groupoid from a group action.  Say we have
a group G acting on a set X.   Then the objects of our groupoid are 
just elements of X, and a morphism

g: x \to  y

is just a group element g with 

gx = y.

Composing morphisms works the obvious way - it's basically just 
multiplication in the group G.

Some people call this groupoid an "action groupoid".  I often call 
it the "weak quotient" X//G, since it's like the ordinary quotient 
X/G, but instead of declaring that x and y are \emph{equal} when we have 
a group element g sending x to y, we instead declare they're 
\emph{isomorphic} via a specified isomorphism g: x \to  y.  

But for now, let's call X//G the "action groupoid".

So, group actions give action groupoids.  But, these groupoids come 
with extra stuff!

First of all, the action groupoid X//G always comes equipped with a
functor 


\begin{verbatim}

                     X//G 
                      |
                      |p
                      |
                      v
                      G
\end{verbatim}
    
sending any object of X//G to the one object of G, and any morphism 
g: x \to  y to the corresponding element of G.   Remember, a group is 
a groupoid with one object: this is the 21st century!  

Second of all, this functor p is always "faithful": given
two morphisms from x to y, if p maps them to the same morphism, then
they were equal.

And that's all!  Any groupoid with a faithful functor to G is
equivalent to the action groupoid X//G for some action of G on some 
set X.  This takes a bit of proving... let's not do it now.

So: in my slogan

<div align = "center">
             A GROUPOID EQUIPPED WITH CERTAIN EXTRA STUFF IS<br>
                     THE SAME AS A GROUP ACTION
</div>

the "certain extra stuff" was precisely a faithful functor to G.  

What if we have a \emph{transitive} group action?  Then something nice
happens.

First of all, saying that G acts transitively on X is the same as
saying there's a morphism between any two objects of X//G.   In 
other words, all objects of X//G are isomorphic.  Or in other words, 
there's just one isomorphism class of objects.  

Just as a groupoid with one object is a group, a groupoid with one 
\emph{isomorphism class} of objects is \emph{equivalent} to a group.  Here 
I'm using the usual notion of "equivalence" of categories, as 
explained back in "<A HREF = "week76.html">week76</A>".

So, G acts transitively on X precisely when X//G is equivalent 
to a group!

And what group?  Well, what could it possibly be?  It's just the 
stabilizer of some element of X!  So, in the case of a transitive 
group action, our functor


\begin{verbatim}

                     X//G 
                      |
                      |p
                      |
                      v
                      G
\end{verbatim}
    
is secretly equivalent to the inclusion


\begin{verbatim}

                      H
                      |
                      |i
                      |
                      v
                      G
\end{verbatim}
    
of the stabilizer group of this element.

So, we see how Klein's old idea of geometrical figures as
subgroups of G is being generalized.  We can start with any groupoid Y
of "figures" and "symmetries between figures", and
play with that.  It becomes an action groupoid if we equip it with a
faithful functor to some group G:


\begin{verbatim}

                      Y
                      |
                      |
                      |
                      v
                      G
\end{verbatim}
    
Then the action is transitive if all the objects of Y are isomorphic.  
In that case, our functor is equivalent to an inclusion


\begin{verbatim}

                      H
                      |
                      |
                      |
                      v
                      G
\end{verbatim}
    
and we're back down to Klein's approach to geometry.  But, it's actually 
good to generalize what Klein did, and think about arbitrary "groupoids
over G" - that is, groupoids equipped with functors to G.

So, when we blend our ideas on spans of groupoids with Klein's ideas,
we'll want to use spans of groupoids "over G" - that is,
commutative diamonds of groupoids and functors, like this:
  

\begin{verbatim}

                     S
                    / \
                   /   \
                  /     \
                 /       \
                v         v
               X           Y
                \         /
                 \       /
                  \     /
                   \   /
                    v v
                     G
\end{verbatim}
    
There's much more to say about this, but not today!

I'll say one last thing before quitting.  It's a bit more technical, 
but I feel an urge to see it in print.  

People often talk about "the" stabilizer group of a transitive action 
of some group G on some set X.  This is a bit dangerous, since every 
element of X has its own stabilizer, and they're not necessarily all equal!

However, they're all \emph{conjugate}: if the stabilizer of x is H, 
then the stabilizer of gx is gHg^{-1}.  

So, when I say above that 


\begin{verbatim}

                     X//G 
                      |
                      |p
                      |
                      v
                      G
\end{verbatim}
    
is equivalent to 


\begin{verbatim}

                      H
                      |
                      |i
                      |
                      v
                      G
\end{verbatim}
    
I could equally well have said it's equivalent to 


\begin{verbatim}

                      H
                      |
                      |i'
                      |
                      v
                      G
\end{verbatim}
    
where the inclusion i' is the inclusion i conjugated by g.  If you
know some category theory, you'll see that i and i' are naturally
isomorphic: a natural isomorphism between functors between groups 
is just a "conjugation".  Picking the specific inclusion i requires
picking a specific element x of X.

Of course, I'll try to write later issues in a way that doesn't force
you to have understood all these nuances!

\par\noindent\rule{\textwidth}{0.4pt}
\textbf{Addendum}: For more discussion, go to the <a href =
"http://golem.ph.utexas.edu/category/2007/04/this_weeks_finds_in_mathematic_10.html">\emph{n}-Category
Caf&eacute;</a>.

\par\noindent\rule{\textwidth}{0.4pt}
<em>There is no benefit today in arithmetic in Roman numerals.
There is also no benefit today in insisting that the group concept
is more fundamental than that of groupoid.</em> - Ronald Brown
\par\noindent\rule{\textwidth}{0.4pt}

% </A>
% </A>
% </A>
