
% </A>
% </A>
% </A>
\week{February 5, 2010 }

This week I want to list a bunch of recent papers and books on
n-categories.  Then I'll tell you about a conference on the math of
environmental sustainability and green technology.  And then I'll
continue my story about electrical circuits.  But first...

This column started with some vague dreams about n-categories and
physics.  Thanks to a lot of smart youngsters - and a few smart
oldsters - these dreams are now well on their way to becoming reality.
They don't need my help anymore!  I need to find some new dreams.  So,
"<a href = "week300.html">week300</a>" will be the last
issue of This Week's Finds in Mathematical Physics.

I still like learning things by explaining them.  When I start work at
the Centre for Quantum Technologies this summer, I'll want to tell you
about that.  And I've realized that our little planet needs my help a
lot more than the abstract structure of the universe does!  The deep
secrets of math and physics are endlessly engrossing - but they can
wait, and other things can't.  So, I'm trying to learn more about ecology, 
economics, and technology.  And I'd like to talk more about those.

So, I plan to start a new column.  Not completely new, just a bit
different from this.  I'll call it This Week's Finds, and drop the
"in Mathematical Physics".  That should be sufficiently
vague that I can talk about whatever I want.

I'll make some changes in format, too.  For example, I won't keep
writing each issue in ASCII and putting it on the usenet newsgroups.
Sorry, but that's too much work.

I also want to start a new blog, since the n-Category Cafe is not the
optimal place for talking about things like the melting of Arctic ice.
But I don't know what to call this new blog - or where it should
reside.  Any suggestions?

I may still talk about fancy math and physics now and then.  Or even a
lot.  We'll see.  But if you want to learn about n-categories, you don't
need me.  There's a \emph{lot} to read these days.  I mentioned Carlos 
Simpson's book in "<a href = "week291.html">week291</a>" - 
that's one good place to start.  Here's another introduction:

1) John Baez and Peter May, Towards Higher Categories, Springer, 2009.
Also available at <a href = "http://ncatlab.org/johnbaez/show/Towards+Higher+Categories">http://ncatlab.org/johnbaez/show/Towards+Higher+Categories</a>

This has a bunch of papers in it, namely:

<ul>
<li>
 John Baez and Michael Shulman, Lectures on n-categories and cohomology. 
</li>
<li>
 Julia Bergner, A survey of (\infty ,1)-categories.
</li>
<li>
 Simona Paoli, Internal categorical structures in homotopical algebra.
</li>
<li>
 Stephen Lack, A 2-categories companion.
</li>
<li>
 Lawrence Breen, Notes on 1- and 2-gerbes.
</li>
<li>
 Ross Street, An Australian conspectus of higher categories.
</li>
</ul>

After browsing these, you should probably start studying
(\infty ,1)-categories, which are \infty -categories where all the
n-morphisms for n > 1 are invertible.  There are a few different
approaches, but luckily they're nicely connected by some results
described in Julia Bergner's paper.  Two of the most important
approaches are "Segal spaces" and
"quasicategories".  For the latter, start here:

2) Andre Joyal, The Theory of Quasicategories and Its Applications,
<a href = "http://www.crm.cat/HigherCategories/hc2.pdf">http://www.crm.cat/HigherCategories/hc2.pdf</a>

and then go here:

3) Jacob Lurie, Higher Topos Theory, Princeton U. Press, 2009.
Also available at <a href = "http://www.math.harvard.edu/~lurie/papers/highertopoi.pdf">http://www.math.harvard.edu/~lurie/papers/highertopoi.pdf</a>

This book is 925 pages long!  Luckily, Lurie writes well.  After
setting up the machinery, he went on to use (\infty ,1)-categories
to revolutionize algebraic geometry:

4) Jacob Lurie, Derived algebraic geometry I: stable infinity-categories,
available as <a href = "http://arXiv.org/abs/math/0608228">arXiv:math/0608228</A>.
<br/>
Derived algebraic geometry II: noncommutative algebra, available as
<a href = "http://arXiv.org/abs/math/0702299">arXiv:math/0702299</A>.
<br/>
Derived algebraic geometry III: commutative algebra, available as 
<a href = "http://arXiv.org/abs/math/0703204">arXiv:math/0703204</A>.
<br/>
Derived algebraic geometry IV: deformation theory, available as 
<a href = "http://arXiv.org/abs/0709.3091">arXiv:0709.3091</a>.
<br/>
Derived algebraic geometry V: structured spaces, available as 
<a href = "http://arXiv.org/abs/0905.0459">arXiv:0905.0459</a>.
<br/>
Derived algebraic geometry VI: E_{k} algebras, available as <a
href = "http://arXiv.org/abs/0911.0018">arXiv:0911.0018</a>.

For related work, try these:

5) David Ben-Zvi, John Francis and David Nadler, Integral transforms
and Drinfeld centers in derived algebraic geometry available as
<a href = "http://arxiv.org/abs/0805.0157">arXiv:0805.0157</a>.

6) David Ben-Zvi and David Nadler, The character theory of a complex
group, available as 
<a href = "http://arxiv.org/abs/0904.1247">arXiv:0904.1247</a>.

Lurie is now using (\infty ,n)-categories to study topological
quantum field theory.  He's making precise and proving some old
conjectures that James Dolan and I made:

7) Jacob Lurie, On the classification of topological field theories,
available as 
<a href = "http://arxiv.org/abs/0905.0465">arXiv:0905.0465</a>.

Jonathan Woolf is doing it in a somewhat different way, which I hope
will be unified with Lurie's work eventually:

8) Jonathan Woolf, Transversal homotopy theory, available as
<a href = "http://arxiv.org/abs/0910.3322">arXiv:0910.3322</a>.

All this stuff is starting to transform math in amazing ways.  And I
hope physics, too - though so far, it's mainly helping us understand
the physics we already have.

Meanwhile, I've been trying to figure out something else to do.  Like
a lot of academics who think about beautiful abstractions and soar
happily from one conference to another, I'm always feeling a bit
guilty, wondering what I could do to help "save the planet".
Yes, we recycle and turn off the lights when we're not in the room.
If we all do just a little bit... a little will get done.  But surely
mathematicians have the skills to do more!

But what?

I'm sure lots of you have had such thoughts.  That's probably why Rachel
Levy ran this conference last weekend:

9) Conference on the Mathematics of Environmental Sustainability and
Green Technology, Harvey Mudd College, Claremont, California,
Friday-Saturday, January 29-30, 2010.  Organized by Rachel Levy.

Here's a quick brain dump of what I learned.  

First, Harry Atwater of Caltech gave a talk on photovoltaic solar
power:

10) Atwater Research Group, <a href =
"http://daedalus.caltech.edu/">http://daedalus.caltech.edu/</a>

The efficiency of silicon crystal solar cells peaked out at 24% in
2000.  Fancy "multijunctions" get up to 40% and are still improving.
But they use fancy materials like gallium arsenide, gallium indium
phosphide, and rare earth metals like tellurium.  The world currently 
uses 13 terawatts of power.  The US uses 3.  But building just 1 terawatt of 
these fancy photovoltaics would use up more rare substances than we can get our
hands on:

<div align = "center">
<img src = "element_abundances.gif">
</div>

11) Gordon B. Haxel, James B. Hedrick, and Greta J. Orris, Rare earth 
elements - critical resources for high technology, US Geological Survey
Fact Sheet 087-02, available at <a href = "http://pubs.usgs.gov/fs/2002/fs087-02/">http://pubs.usgs.gov/fs/2002/fs087-02/</a>

So, if we want solar power, we need to keep thinking about silicon and
use as many tricks as possible to boost its efficiency.

There are some limits.  In 1961, Shockley and Quiesser wrote a paper
on the limiting efficiency of a solar cell.  It's limited by 
thermodynamical reasons!  Since anything that can absorb energy 
can also emit it, any solar cell also acts as a light-emitting diode,
turning electric power back into light:

12) W. Shockley and H. J. Queisser, Detailed balance limit of 
efficiency of p-n junction solar cells, J. Appl. Phys. 32 (1961)
510-519.

13) Wikipedia, Schockley-Quiesser limit, 
<a href = "http://en.wikipedia.org/wiki/Shockley%E2%80%93Queisser_limit">http://en.wikipedia.org/wiki/Shockley%E2%80%93Queisser_limit</a>

What are the tricks used to approach this theoretical efficiency?
Multijunctions use layers of different materials to catch photons of
different frequencies.  The materials are expensive, so people use a
lens to focus more sunlight on the photovoltaic cell.  The same is true
even for silicon - see the Umuwa Solar Power Station in Australia.  
But then the cells get hot and need to be cooled.

<div align = "center">
<img width = "600" src = "solar_power_umuwa.jpg">
</div>

Roughening the surface of a solar cell promotes light trapping, by
large factors!  Light bounces around ergodically and has more chances
to get absorbed and turned into useful power.  There are theoretical
limits on how well this trick works.  But those limits were derived
using ray optics, where we assume light moves in straight lines.  So,
we can beat those limits by leaving the regime where the ray-optics
approximation holds good.  In other words, make the surface
complicated at length scales comparable to the wavelength at light.

For example: we can grow silicon wires from vapor!  They can form
densely packed structures that absorb more light:

<div align = "center">
<a href = "http://pubs.acs.org/doi/abs/10.1021/ja074897c">
<img src = "silicon_nanorod.gif">
% </a>
</div>

14) B. M. Kayes, H. A. Atwater, and N. S. Lewis, Comparison of the
device physics principles of planar and radial p-n junction nanorod
solar cells, J. Appl. Phys. 97 (2005), 114302.

James R. Maiolo III, Brendan M. Kayes, Michael A. Filler, 
Morgan C. Putnam, Michael D. Kelzenberg, Harry A. Atwater 
and Nathan S. Lewis, High aspect ratio silicon wire array 
photoelectrochemical cells, J. Am. Chem. Soc. 129 (2007),
12346-12347.  

Also, with such structures the charge carriers don't need to travel
so far to get from the n-type material to the p-type material.  This
also boosts efficiency.

There are other tricks, still just under development.  Using quasiparticles
called "surface plasmons" we can adjust the dispersion relations to 
create materials with really low group velocity.  Slow light has more
time to get absorbed!  We can also create "meta-materials" whose 
refractive index is really wacky - like n = -5!   

I should explain this a bit, in case you don't understand.  Remember,
the refractive index of a substance is the inverse of the speed of
light in that substance - in units where the speed of light in vacuum
equals 1.  When light passes from material 1 to material 2, it takes 
the path of least time - at least in the ray-optics approximation.
Using this you can show Snell's law:

sin(\theta _{1})/sin(\theta _{2}) = n_{2}/n_{1}

where n_{i} is the index of refraction in the ith material and
\theta _{i} is the angle between the light's path and the line
normal to the interface between materials:

<div align = "center">
<img src = "200px-Snells_law.svg.png">
</div>

Air has an index of refraction close to 1.  Glass has an index of
refraction greater than 1.  So, when light passes from air to glass,
it "straightens out": its path becomes closer to
perpendicular to the air-glass interface.  When light passes from
glass to air, the reverse happens: the light bends more.  But the sine
of an angle can never exceed 1 - so sometimes Snell's law has no
solution.  Then the light gets stuck!  More precisely, it's forced to
bounce back into the glass.  This is called "total internal
reflection", and the easiest way to see it is not with glass, but
water.  Dive into a swimming pool and look up from below.  You'll only
see the sky in a limited disk.  Outside that, you'll see total
internal reflection.

Okay, that's stuff everyone learns in optics.  But \emph{negative}
indices of refraction are much weirder!  The light entering such a
material will bend \emph{backwards}.

<div align = "center">
<img src = "negative refraction.gif">
</div>

Materials with a negative index of refraction also exhibit a reversed
version of the ordinary <a href =
"http://en.wikipedia.org/wiki/Goos%E2%80%93H%C3%A4nchen_effect">Goos-H&auml;nchen</a>
effect.  In the ordinary version, light "slips" a little
before reflecting during total internal reflection.  The
"slip" is actually a slight displacement of the light's wave
crests from their expected location - a "phase slip".  But
for a material of negative refractive index, the light slips
\emph{backwards}.  This allows for resonant states where light gets
trapped in thin films.  Maybe this can be used to make better solar
cells.

Next, Kenneth Golden gave a talk on sea ice, which covers 7-10% of the
ocean's surface and is a great detector of global warming.  He's a
mathematician at the University of Utah who also does measurements in
the Arctic and Antarctic.  If you want to go to math grad school
without becoming a nerd - if you want to brave 70-foot swells, dig
trenches in the snow and see emperor penguins - you want Golden as
your advisor:

<div align = "center">
<a href = "http://www.math.utah.edu/~golden/3.html">
<img width = "500" src = "kenneth_golden.jpg">
% </a>
</div>

15) Ken Golden's website, <a href =
"http://www.math.utah.edu/~golden/">http://www.math.utah.edu/~golden/</a>

Salt gets incorporated into sea ice via millimeter-scale brine
inclusions between ice platelets, forming a "dendritic platelet
structure".  Melting sea ice forms fresh water in melt ponds atop the
ice, while the brine sinks down to form "bottom water" driving the
global thermohaline conveyor belt.  You've heard of the Gulf Stream,
right?  Well, that's just part of this story.

When it gets hotter, the Earth's poles get less white, so they absorb
more light, making it hotter: this is "ice albedo feedback".  Ice
albedo feedback is \emph{largely controlled by melt ponds}.  So if you're
interested in climate change, questions like the following become 
important: when do melt ponds get larger, and when do they drain out?

Sea ice is diminishing rapidly in the Arctic - much faster than all
the existing climate models had predicted.   In the Arctic, winter sea
ice diminished in area by about 10% from 1978 to 2008.  But summer sea
ice diminished by about 40%!  It took a huge plunge in 2007, leading
to a big increase in solar heat input due to the ice
albedo effect.   

<div align = "center">
<a href = "http://www.arctic.noaa.gov/reportcard/seaice.html">
<img border = "none" width = "500" src = "arctic_sea_ice.jpg">
% </a>
<br/>
<font size = "-1">
Time series of the percent difference in ice extent in March (the month of 
ice extent maximum) and September (the month of ice extent minimum) relative 
to the mean values for the period 1979-2000. Based on a least squares linear 
regression for the period 1979-2009, the rate of decrease for the March 
and September ice extents is -2.5% and -8.9% per decade, respectively.
Figure from <a href = "http://www.arctic.noaa.gov/reportcard/seaice.html">Perovich \emph{et al}</a>.
</font>
% </a>
</div>

16) Donald K. Perovich, Jacqueline A. Richter-Menge, Kathleen
F. Jones, and Bonnie Light, Sunlight, water, and ice: Extreme Arctic
sea ice melt during the summer of 2007, Geophysical Research Letters,
35 (2008), L11501.  Also available at 
<a href = "http://www.crrel.usace.army.mil/sid/personnel/perovichweb/index1.htm">http://www.crrel.usace.army.mil/sid/personnel/perovichweb/index1.htm</a>

There's a lot less sea ice in the Antarctic than in the Arctic.
Most of it is the Weddell Sea, and there it seems to be
growing, maybe due to increased precipitation.

There's a lot of interesting math involved in understanding the
dynamics of sea ice.  The ice thickness distribution equation was
worked out by Thorndike et al in 1975.  The heat equation for ice and
snow was worked out by Maykut and Understeiner in 1971.  Sea ice
dynamics was studied by Kibler.

Ice floes have two fractal regimes, one from 1 to 20 meters, another
from 100 to 1500 meters.  Brine channels have a fractal character well
modeled by "<a href =
"http://en.wikipedia.org/wiki/Diffusion-limited_aggregation">diffusion
limited aggregation</a>".  Brine starts flowing when there's
about 5% of brine in the ice - a kind of percolation problem familiar
in statistical mechanics.  Here's what it looks like when there's 5.7%
brine and the temperature is -8 &deg;C:

<div align = "center">
<a href = "http://www.math.utah.edu/~golden/7.html">
<img width = "500" src = "kenneth_golden_brine_inclusions.jpg">
% </a>
</div>

17) Kenneth Golden, Brine inclusions in a crystal of lab-grown sea
ice, <a href =
"http://www.math.utah.edu/~golden/7.html">http://www.math.utah.edu/~golden/7.html</a>

Nobody knows why polycrystalline metals have a log-normal distribution
of crystal sizes.  Similar behavior, also unexplained, is seen in sea
ice.

A "<a href =
"http://en.wikipedia.org/wiki/Polynya">polynya</a>" is an area of
open water surrounded by sea ice.  Polynyas occupy just .001% of the
overall area in Antarctic sea ice, but create 1% of the icea.  Icy
cold <a href = "http://en.wikipedia.org/wiki/Catabatic_wind">katabatic
winds</a> blow off the mainland, pushing away ice and creating patches
of open water which then refreeze.

<div align = "center">
<a href = "http://en.wikipedia.org/wiki/Polynya">
<img src = "500px-antarctic_shelf_ice_hg.png">
% </a>
</div>

<div align = "center">
<a href = "http://en.wikipedia.org/wiki/Polynya">
<img src = "500px-katabatic-wind_hg.png">
% </a>
</div>

There was anomalous export of sea ice through Fran Strait in the 1990s,
which may have been one of the preconditions for high ice albedo feedback.

20-40% of sea ice is formed by surface flooding followed by refreezing.  
This was \emph{not included} in the sea ice models that gave such
inaccurate predictions.

The food chain is founded on diatoms.  These form "extracellular
polymeric substances"- goopy mucus-like stuff made of polysaccharides
that protects them and serves as antifreeze.  There's a lot of this
stuff; the ice gets visibly stained by it.

For more, see:

18) Kenneth M. Golden, Climate change and the mathematics of transport
in sea ice, AMS Notices, May 2009.  Also available at
<a href = "http://www.ams.org/notices/200905/">http://www.ams.org/notices/200905/</a>

19) Mathematics Awareness Month, April 2009: Mathematics and Climate,
<a href = "http://www.mathaware.org/mam/09/">http://www.mathaware.org/mam/09/</a>

Next, Julie Lundquist, who just moved from Lawrence Livermore Labs
to the University of Colorado, spoke about wind power:

20) Julie Lunquist, Department of Atmospheric and Oceanic Sciences,
University of Colorado, <a href = "http://paos.colorado.edu/people/lundquist.php">http://paos.colorado.edu/people/lundquist.php</a>

With increased reliance on wind, the power grid will need to be
redesigned to handle fluctuating power sources.  In the US, currently,
companies aren't paid for power they generate in excess of the amount
they promised to make.  So, accurate prediction is a hugely important
game.  Being off by 1% can cost millions of dollars!  Europe has
different laws, which encourage firms to maximize the amount of wind
power they generate.

If you had your choice about where to build a wind turbine, you'd
build it on the ocean or a very flat plain, where the air flows rather
smoothly.  Hilly terrain leads to annoying turbulence - but sometimes
that's your only choice.  Then you need to find the best spots, where
the turbulence is least bad.  Complete simulation of the Navier-Stokes
equations is too computationally intensive, so people use fancier tricks.
There's a lot of math and physics here.

For weather reports people use "mesoscale simulation" which
cleverly treats smaller-scale features in an averaged way - but we
need more fine-grained simulations to see how much wind a turbine will
get.  This is where "large eddy simulation" comes in.  Eddy
diffusivity is modeled by Monin-Obukhov similarity theory:

21) American Meteorological Society Glossary, Monin-Obukhov similarity theory,
<a href = "http://amsglossary.allenpress.com/glossary/search?id=monin-obukhov-similarity-theory1">http://amsglossary.allenpress.com/glossary/search?id=monin-obukhov-similarity-theory1</a>


A famous Brookhaven study suggested that the power spectrum of wind
has peaks at 4 days, 1/2 day, and 1 minute.  This perhaps justifies an
approach where different time scales, and thus length scales, are
treated separately and the results then combined somehow.  The study
is actually a bit controversial.  But anyway, this is the approach
people are taking, and it seems to work.

Night air is stable - but day air is often not, since the ground is
hot, and hot air rises.  So when a parcel of air moving along hits a
hill, it can just shoot upwards, and not come back down!  This means
lots of turbulence.

The wind turbines at Altamont Pass in California kill more raptors
than all other wind farms in the world combined!  Old-fashioned wind
turbines look like nice places to perch, spelling death to birds.
Cracks in concrete attract rodents, which attract raptors, who get
killed.  The new ones are far better.

For more:

22) National Renewable Energy Laboratory, Research needs for winds
resource characterization, available as
<a href = "http://www.nrel.gov/docs/fy08osti/43521.pdf">http://www.nrel.gov/docs/fy08osti/43521.pdf</a>

Finally, there was a talk by Ron Lloyd of Fat Spaniel Technologies.
This is a company that makes software for solar plants and other
sustainable energy companies:

23) Fat Spaniel Technologies, <a href = "http://www.fatspaniel.com/products/">http://www.fatspaniel.com/products/</a>

His talk was less technical so I didn't take detailed notes.  One big
point I took away was this: we need better tools for modelling!  This
is especially true with the coming of the "smart grid".  In its
simplest form, this is a power grid that uses lots of data - for
example, data about power generation and consumption - to regulate
itself and increase efficiency.  Surely there will be a lot of math
here.  Maybe even the topic I've been talking about lately: bond graphs!

But now I want to talk about some very simple aspects of electrical
circuits.  Last week I listed various kinds of circuits.  Now let's go
into a bit more detail - starting with the simplest kind: circuits
made of just wires and linear resistors, where the currents and
voltages are independent of time.

Mathematically, such a circuit is a graph equipped with some extra
data.  First, each edge has a number associated to it - the
"resistance".  For example:


\begin{verbatim}

               o----1----o----3----o
               |         |         | 
               |         |         | 
               2         3         2 
               |         |         | 
               |         |         | 
               o----3----o----1----o
\end{verbatim}
    
Second, we have current flowing through this circuit.  To describe this,
we first arbitrarily pick an orientation on each edge:


\begin{verbatim}

               o---->----o---->----o
               |         |         | 
               |         |         | 
               V         V         V 
               |         |         | 
               |         |         | 
               o----<----o---->----o
 
\end{verbatim}
    
Then we label each edge with a number saying how much "current"
is flowing through that edge, in the direction of the arrow:


\begin{verbatim}

                    2        3
               o---->----o---->----o
               |         |         | 
               |         |         | 
             3 V         V 1       V 3
               |         |         | 
               |         |         | 
               o----<----o---->----o
                    2       -3
\end{verbatim}
    
Electrical engineers call the current I.  Mathematically it's good
to think of I as a "1-chain": a formal linear combination of 
oriented edges of our graph, with the coefficients of the linear combination 
being the numbers shown above.  

If we know the current, we can work out a number for each vertex of
our graph, saying how much current is flowing out of that vertex,
minus how much is flowing in:


\begin{verbatim}

                         2     
             5 o---->----o---->----o 0 
               |         |         | 
               |         |         | 
               V         V         V 
               |         |         | 
               |         |         | 
            -5 o----<----o---->----o 0
                       -2   
\end{verbatim}
    

Mathematically we can think of this as a "0-chain": a formal
linear combination of the vertices of our graph, with the numbers
shown above as coefficients.  We call this 0-chain the
"boundary" of the 1-chain we started with.  Since our
current was called I, we call its boundary \delta I.

Kirchhoff's current law says that 

\delta I = 0

When this holds, let's say our circuit is a "closed".
Physically this follows from the law of conservation of electrical
charge, together with a reasonable assumption.  Current is the flow of
charge.  If the total current flowing into a vertex wasn't equal to
the amount flowing out, charge - positive or negative - would be
building up there.  But for a closed circuit, we assume it's not.

If a circuit is not closed, let's call it "open".  These are
interesting too.  For example, we might have a circuit like this:


\begin{verbatim}

               x
               |    
               |    
               V    
               |    
               |    
               o---->----o
               |         |
               |         |
               V         V
               |         |
               |         |
               x         x         

\end{verbatim}
    
where we have current flowing in the wire on top and flowing out the
two wires at bottom.  We allow \delta I to be nonzero at the ends
of these wires - the 3 vertices labelled x.  This circuit is an
"open system" in the sense of "<a href =
"week290.html">week290</A>", because it has these wires dangling
out of it.  It's not self-contained; we can use it as part of some
bigger circuit.  We should really formalize this more, but I won't now.
Derek Wise did it more generally here:

24) Derek Wise, Lattice p-form electromagnetism and chain field theory,
available as <a href = "http://arxiv.org/abs/gr-qc/0510033">gr-qc/0510033</a>.

The idea here was to get a category where chain complexes are
morphisms.  In our situation, composing morphisms amounts to gluing
the output wires of one circuit into the input wires of another.  This
is an example of the general philosophy I'm trying to pursue, where
open systems are treated as morphisms.  

We've talked about 1-chains and 0-chains... but we can also back up and
talk about 2-chains!  Let's suppose our graph is connected - it is in
our example - and let's fill it in with enough 2-dimensional
"faces" to get something contractible.  We can do this in a
god-given way if our graph is drawn on the plane: just fill in all the
holes!


\begin{verbatim}

               o---------o---------o
               |/////////|/////////| 
               |/////////|/////////| 
               |//FACE///|///FACE//| 
               |/////////|/////////| 
               |/////////|/////////| 
               o---------o---------o
\end{verbatim}
    

In electrical engineering these faces are often called
"meshes".

This give us a chain complex


$$

        \delta            \delta 
C_{0} <-------- C_{1} <-------- C_{2}
$$
    

Remember, a "chain complex" is just a bunch of vector spaces C_{i} and linear maps \delta : C_{i} \to  C_{i-1}, obeying the equation \delta ^{2} = 0.  We also get a cochain complex:


$$

       d           d
C^{0} --------> C^{1} ---------> C^{2}
$$
    

meaning a bunch of vector spaces C_{i} and linear maps d: C_{i} \to  C_{i+1}, obeying the equation d^{2} = 0. 

As I've already said, it's good to think of the current I as a 1-chain,
since then

\delta I = 0 

is Kirchoff's current law.  Since our little space is contractible the
above equation implies that

I = \delta J

for some 2-chain J called the "mesh current".  This assigns to each
face or "mesh" the current flowing around that face.

An electrical circuit also comes with a third piece of data, which I
haven't mentioned yet.  Each oriented edge should be labelled by a
number called the "voltage" across that edge.  Electrical engineers
call the voltage V.  It's good to think of V as a 1-cochain, which
assigns to each edge the voltage across that edge.  

Why a 1-cochain instead of a 1-chain?  Because then

dV = 0

is the other basic law of electrical circuits - Kirchhoff's voltage
law!  This law says that the sum of these voltages around a mesh is
zero.  Since our little space is contractible the above equation
implies that

V = d\phi 

for some 0-cochain \phi  called the "electrostatic potential".  In 
electrostatics, this potential is a function on space.  Here it
assigns a number to each vertex of our graph.  

Since the space of 1-cochains is the dual of the space of 1-chains, we
can take the voltage V and the current I, glom them together, and get
a number:

V(I)

This the "power": that is, the rate at which our network
soaks up energy and dissipates it into heat.  Note that this is just a
fancy version of formula for power that I explained in "<a href =
"week290.html">week290</A>" - power is effort times flow.

I've given you three basic pieces of data labelling our circuit: the
resistance R, the current I, and the voltage V.  But these aren't
independent!  Ohm's law says that the voltage across any edge is the
current through that times the resistance of that edge.  But this
remember: current is a 1-chain while voltage is a 1-cochain.  So
"resistance" can be thought of as a map from 1-chains to
1-cochains:

R: C_{1} \to  C^{1}

This lets us write Ohm's law like this:

V = RI

This, in turn, means the power of our circuit is

V(I) = (RI)(I)

For physical reasons, this power is always nonnegative.  In fact,
let's assume it's positive unless I = 0.  This is just another way of
saying that resistance labelling each edge is positive.  It can be
very interesting to think about circuits with perfectly conducting
wires.  These would give edges whose resistance is zero.  But that's a
bit of an idealization, and right now I'd rather allow only \emph{positive}
resistances.

Why?  Because then we can think of the above formula as the inner
product of I with itself!  In other words, then there's a unique inner
product on 1-chains with

(RI)(I) = <I,I>

In this situation

R: C_{1} \to  C^{1}

is the usual isomorphism that we get between a finite-dimensional
inner product space and its dual.  (For this statement to be true,
we'd better assume our graph has finitely many vertices and edges.)

Now, if you've studied de Rham cohomology, all this should start
reminding you of Hodge theory.  And indeed, it's a baby version
of that!  So, we're getting a little bit of Hodge theory, but in
a setting where our chain complexes are really morphisms in a category.
Or more generally, n-morphisms in an n-category!

There's a lot more to say, but that's enough for now.
Here are some references on "electrical circuits as chain
complexes":

25) Paul Bamberg and Shlomo Sternberg, A Course of Mathematics for
Students of Physics, Cambridge University, Cambridge, 1982.

Bamberg and Sternberg is a great book overall for folks wanting to 
get started on mathematical physics.  The stuff about circuits
starts in chapter 12.  

26) P. W. Gross and P. Robert Kotiuga, Electromagnetic Theory and
Computation: A Topological Approach, Cambridge University Press, 2004.

This book says just a little about electrical circuits of the sort
we're discussing, but it says a \emph{lot} about chain complexes
and electromagnetism.  It's a great place to start if you know some
electromagnetism but have never seen a chain complex.


\par\noindent\rule{\textwidth}{0.4pt}

\textbf{Addenda:} I thank Colin Backhurst, G.R.L. Cowan, David Corfield, Mikael
Vejdemo Johansson and Tim Silverman for corrections.  I thank Garett
Leskowitz for pointing out the material in Bamberg and Sternberg's
book.

Ed Allen writes:

\begin{quote}
Regarding the silicon technologies for improving efficiency of light capture, 
the Mazur lab's black silicon projects are something I've been following for a 
few years:

27) Mazur Group, Optical hyperdoping - black silicon, 
<a href = "http://mazur-www.harvard.edu/research/detailspage.php?rowid=1">
http://mazur-www.harvard.edu/research/detailspage.php?rowid=1</a>

28) Wikipedia, Black silicon, <a href =
"http://en.wikipedia.org/wiki/Black_silicon">
http://en.wikipedia.org/wiki/Black_silicon</a>

29) Anne-Marie Corley, Pink silicon is the new black, Technology
Review, Thursday, July 9, 2009.  Also available at <a href
="http://www.technologyreview.com/computing/22975/?a=f">
http://www.technologyreview.com/computing/22975/?a=f</a>

\end{quote}


David Corfield wonders if it's really true that there's a
lot less sea ice in the Antarctic than the Arctic:

\begin{quote}

Was that right? <a href = "http://arctic.atmos.uiuc.edu/cryosphere/">Cryosphere</a> has the 
<a href = 
"http://arctic.atmos.uiuc.edu/cryosphere/IMAGES/seaice.area.arctic.png">Arctic</a> sea ice oscillating on average between 5 and 14 million sq km, and the
<a href = "http://arctic.atmos.uiuc.edu/cryosphere/IMAGES/seaice.area.antarctic.png">Antarctic</a> between 2 and 15 million sq km.


Recently of course that 5 has become 3.

Best, David

\end{quote}




For more discussion, visit the
<a href = "http://golem.ph.utexas.edu/category/2010/02/this_weeks_finds_in_mathematic_54.html">\emph{n}-Category Caf&eacute;</a>.

\par\noindent\rule{\textwidth}{0.4pt}
<em>So many young people are forced to specialize in one line or another that 
a young person can't afford to try and cover this waterfront - only an old 
fogy who can afford to make a fool of himself.  If I don't, who will?</em> -
John Wheeler

\par\noindent\rule{\textwidth}{0.4pt}

% </A>
% </A>
% </A>
