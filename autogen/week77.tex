
% </A>
% </A>
% </A>
\week{March 23, 1996 }

I spent last week at Penn State visiting the CGPG - the Center for
Gravitational Physics and Geometry.  I like to visit this place whenever
I can, because I've never found anywhere else that's as good for talking
about quantum gravity.

The CGPG is run by Abhay Ashtekar, who introduced the "new variables"
for general relativity (see "<A HREF = "week7.html">week7</A>").  This formulation of general
relativity allowed Carlo Rovelli and Lee Smolin to develop a new
approach to quantum gravity, called the "loop representation".  Smolin
is at the CGPG, while Rovelli teaches at Pittsburgh, only a brief plane
ride away: he was heading back just when I showed up.  Jorge Pullin, who
has done a lot of work on knot theory and quantum gravity, is also at
the CGPG.  Roger Penrose visits it regularly, and happened to be there
last week.  There is always a peppy bunch of grad students and postdocs
wandering about the place, and some interesting mathematicians across
the street.  I have a particular interest in the work of Jean-Luc
Brylinski, since he has thought a lot about the relationships between
conformal field theory and category theory (see "<A HREF = "week25.html">week25</A>").

You can find out more about the CGPG and the new variables at the
following web sites: 

1) Center for Gravitational Physics and Geometry (CGPG) home page,
<A HREF =
"http://vishnu.nirvana.phys.psu.edu/">http://vishnu.nirvana.phys.psu.edu/</A>

Reading list on the new variables:
<A HREF =
"http://vishnu.nirvana.phys.psu.edu/readinglist/readinglist.html">http://vishnu.nirvana.phys.psu.edu/readinglist/readinglist.html</A>

I had two goals at the CGPG.  One was to get people interested in the
uses of higher-dimensional algebra in physics, and the other was to find
out where folks were heading in quantum gravity.  I made decent headway
on the first front, but let me talk about the second one.

In the last few years, Abhay Ashtekar has been working hard with a bunch
of collaborators on getting the loop representation set up on a
mathematically rigorous basis, and making good progress.  There is a
natural order in which to set things up, and the next problem to deal
with is the so-called Hamiltonian constraint (see "<A HREF = "week43.html">week43</A>").  I have
always been very worried about this, and I'm not alone, since this all
the dynamics of quantum gravity is in this operator.  Ashtekar and
Lewandowski have a paper partially written in which they rigorously
define an operator along these lines, using earlier ideas of Rovelli and
Smolin.  I have been hoping that this answer could be tested somehow...
for example, checking out its commutation relations with the other
constraints.  It turns out that they have already done this to extent that
seems possible.  So then the question is, what next?  March on, or
continue trying to make sure the Hamiltonian constraint is right?

I should add that Pullin and Gambini have another proposal regarding the
Hamiltonian constraint:

2) Rodolfo Gambini and Jorge Pullin, The general solution of the quantum
Einstein equations?, preprint in Revtex format, 7 figures included with
psfig, available as <A HREF = "http://xxx.lanl.gov/abs/gr-qc/9603019">gr-qc/9603019</A>.  

This is not as fully worked out, but it has a certain mathematical charm
to it so far.  Thus we may eventually be in a situation where there are
various competing quantizations of gravity using the loop
representation, differing mainly in their choice of Hamiltonian
constraint.  This suggests that we need further tests for what counts as
the "right" Hamiltonian constraint.  

When we spoke this time, Ashtekar was in favor of testing Hamiltonian
constraints by seeing whether they implied the "Bekenstein bound".  This
bound says that the maximal entropy of a physical system is proportional
to its surface area when we take quantum gravity into account.  There
are a number of heuristic derivations of this bound, so lots of people
hope it would follow from any good theory of quantum gravity.  Since the
"physical states" of quantum gravity must be annihilated by the
Hamiltonian constraint, and the maximal entropy of a system is just the
logarithm of the number of physical states, the Hamiltonian constraint
must have some interesting properties to get the Bekenstein bound to
work out.  So we can expect some work along these lines in the near
future.

I also talked to Lee Smolin.  He has been very interested in the
relation between the loop representation and certain simplified versions
of quantum gravity called topological quantum field theories (TQFTs).
He has ideas on how to derive the Bekenstein bound using this
relationship - see "<A HREF = "week56.html">week56</A>" and "<A HREF = "week57.html">week57</A>" for a description.

The funny thing is, some of the mathematics connecting TQFTs to the
loop representation of quantum gravity also connects TQFTs to another
well-known approach to quantum gravity - string theory!  Smolin has
been boning up on string theory lately, in part by giving a course on
the subject, and presently he is eager to bring string theory and the
loop representation closer together.  So we can also expect to see more
work on attempts to unify string and loops.  (See "<A HREF = "week18.html">week18</A>" for a bit more
on strings and loops.)


So I left feeling reinvigorated and eager to continue my own work on
higher-dimensional algebra and physics... which is what I have talking
about here ever since "<A HREF = "week73.html">week73</A>".  
In fact, I have been engaging in a
lengthy warmup, a minicourse in category theory, with an eye to the
basic themes of n-category theory.  That way, when I get around to the
really cool stuff, everyone out there will know what the heck I'm talking
about.  In theory, anyway.  You gotta work a bit to wrap your mind
around these concepts!

% <A NAME = "tale">
So, let's recall where we are in our tale of n-categories.  We were
studying increasingly subtle variations on the theme of identity and
difference.  Given two categories C and D, we can ask if they are
\emph{equal} or not.  We can also discuss \emph{isomorphisms} between C and D.  An
isomorphism is a functor F: C \to  D having an inverse: a functor G: D \to 
C such that FG is equal to the identity functor on D and GF is equal to the
identity on C. 

We can also discuss \emph{equivalences} between C and D.  An equivalence is a
functor F: C \to  D together with a functor G: D \to  C such that FG
is naturally isomorphic to the identity functor on D, and GF is
naturally isomorphic to the identity functor on C.  Remember, two
functors from one category to another are "naturally
isomorphic" if there is a natural transformation from the first to
the second, and that natural transformation has an inverse.

In math jargon we say it this way: two categories are equivalent if
there is a functor from one to the other which is invertible "up to a
natural isomorphism".  

The most useful notion of categories being "the same" turns
out to be not equality, or isomorphism, but this more supple notion of
"equivalence"!

(As we shall see later, this is because Cat is a 2-category.  Remember,
an n-category is some sort of thing with objects, morphisms,
2-morphisms, and so on up to n-morphisms.  One of the of the main themes
of n-category theory is that we may regard two things are "the
same", or "equivalent", if there is some sort of process
to get from one to the other, and this process is invertible... up to
equivalence!  More precisely, we say an n-morphism is an equivalence if
it's invertible, and then we work our way down, inductively defining a
(j-1)-morphism to be an equivalence if it's invertible up to an
equivalence.  This downwards induction leaves off when we define
equivalence for "0-morphisms", meaning objects.)

We have also begun talking about a curious situation where the
categories C and D are not at all "the same," but there are "adjoint"
functors L: C \to  D and R: D \to  C.  Let me list some examples before
defining the concept of adjoint functor and talking about it:


<OL><LI>
 First for the one we discussed in "<A HREF =
"week76.html">week76</A>".  Let Set be the category of sets, and
Grp the category of groups.  Let L: Set \to  Grp be the functor taking
each set S to the free group on S, and doing the obvious thing to
morphisms.  Let R: Grp \to  Set be the functor taking each group to its
underlying set.

<LI>
Let Ab be the category of abelian (i.e., commutative) groups.  Let L:
Set \to  Ab be the functor taking each set S to the free abelian group
on S.  The "free abelian group" on S is what we get by taking
the free group on S and imposing commutativity relations like xy = yx
for all elements x,y in S.  Let R: Ab \to  Set be the functor taking
each abelian group to its underlying set.

<LI>
Let L: Grp \to  Ab be the functor taking each group G to its
"abelianization".  The abelianization of G is what we get when
we impose the extra relations xy = yx for all elements x,y in G.  Let R:
Ab \to  Grp be the functor taking each abelian group to its underlying
group.

<LI>
Let Mon be the category of monoids, where the objects are monoids and
the morphisms are monoid homomorphisms.  (Remember that a monoid is a
set with an associative product and a unit; a monoid morphism f: M \to  N
is a function between monoids such that f(xy) = f(x)f(y) and f(1) = 1.)
Let L: Set \to  Mon be the functor taking each set S to the free monoid on
S.  This is simply the set of words whose letters are elements of S,
with the product given by concatenation of words, and the unit being the
empty word.  Let R: Mon \to  Set be the functor taking each monoid to its
underlying set.  
<LI>
Let L: Mon \to  Grp be the functor taking each monoid M to the group
obtained by throwing in formal inverses for every element of M.  The
famous example of this is when N = {0,1,2,...}, which is a monoid
whose "product" is addition.  Then L(N) = Z, the integers, since we have
thrown in the negative integers.  Let R: Grp \to  Mon be the functor
taking each group to its underlying monoid.  I.e., R simply forgets that
our group has inverses and thinks of it as a monoid.
</OL>

Note the common aspects of these examples!  In most of them, L: C \to 
D gives us a "free" object of D for every object of C, while
R: D \to  C gives us an "underlying" object of C for every
object of D.  This is an especially good way to think about it when the
objects of D are objects of C equipped with extra structure, as in
examples 1, 2, 4, and 5.  For example, a group is a set equipped with
some extra structure, the group operations.  In this case, the functor
L: C \to  D turns an object of C into an object of D by "freely
throwing in whatever extra stuff is necessary, without putting in any
relations other than those needed to get an object of D".


It's not quite the same when the objects of D are objects of C with
extra \emph{properties}, as in example 3.  In this case, the functor
L: C \to  D forces an object of C to have the properties needed to be
an object of D.  It does so in as nonviolent a manner as possible.

In either of these situations, R: D \to  C has the flavor of what we
call a "forgetful" functor.  This is not a precisely defined
term, but folks use it whenever we can simply "forget"
something about an object of D and think of it as an object of C.  For
example, we can take a group, and forget about the group operations,
thinking of it as merely a set.  Here we are forgetting extra structure;
we can also forget extra properties.

The crucial thing here is that unlike in an equivalence, there is a
built-in asymmetry here: L and R have very different flavors, and serve
different mathematical purposes.  We call L the "left adjoint"
of R, and we call R the "right adjoint" of L.

There are situations where adjoint functors L and R aren't so
immediately reminiscent of the concepts "free" and
"underlying".  But it's good to keep these ideas in mind when
learning about adjoint functors.  I used to have trouble remembering
which was supposed to be the left adjoint and which was the right.  The
honest way to do this is to remember the definition (coming up soon),
but for a cheap mnemonic, you can think of the L in a left adjoint as
standing for "liberty" - that is, freedom!

So what's the definition of "adjoint"?  Roughly speaking, it's
that for object c of C and any object d of D, we have

                        hom(Lc,d) = hom(c,Rd).

Actually this is a slight exaggeration: we don't want these to be equal.
The guy on the left is the set of morphisms from Lc to d in the category
D.  The guy on the right is the set of morphisms from c to Rd in the
category C.  So it's evil to want them to be \emph{equal}.  As you might
guess, it's enough for them to be naturally isomorphic in some sense.
Let's not worry about that too much yet, though.  Let's get the basic
idea here!

Consider example 1.  Say S is a set and G is a group.  Why is

                        hom(LS,G) 

naturally isomorphic to

                        hom(S,RG) ?
  
In other words, why is the set of homomorphisms from the free group on S
to G naturally isomorphic to the set of functions from S to the
underlying set of G?
  
Well, say we have a homomorphism f: LS \to  G.  Since LS is a free group,
we know f if we know what it does to each element of S... and it can
do whatever it wants to these elements!  So we can think of it
as just a function from S to the underlying set of G.  In other words,
we can think of it as a function f': S \to  RG.  Conversely, any function
f': S \to  RG gives us a homomorphism f: LS \to  G.

So this is the idea.  Say we have an object c of C and an object d of D.
Then:

"The set of morphisms from the free D-object on c to d is naturally
isomorphic to the set of morphisms from c to the underlying C-object of
d."

Next time I will finish off the definition of adjoint functors, by
making this "naturally isomorphic" stuff precise.  I will also begin to
explain what adjoint functors have to do with adjoint operators in
quantum mechanics.  Remember that an "observable" in quantum theory is
an operator on a Hilbert space which is its own adjoint, while a
"symmetry" in quantum theory is an operator whose adjoint is its
inverse.  I eventually hope to show that this, and many other shocking
aspects of quantum theory, become less shocking when we think of the
world in terms of categories (or n-categories) rather than sets.  The
way I think of it these days, the mysterious way quantum theory slammed
into physics in the early 20th century was just nature's way of telling
us we'd better learn n-category theory.

I'll also explain what adjoint functors have to do with the following
topological equations:



\begin{verbatim}

          /\       |       |
         /  \      |       |
        /    \     |       |
       |      \    /  =    | 
       |       \  /        |
       |        \/         |




       |       /\          |
       |      /  \         |
       |     /    \        |
       \    /      |  =    | 
        \  /       |       |
         \/        |       |



\end{verbatim}
    




<A HREF = "week78.html#tale">To continue reading the "Tale of
n-Categories", click here.</A>


\par\noindent\rule{\textwidth}{0.4pt}
% </A>
% </A>
% </A>
