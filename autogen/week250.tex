
% </A>
% </A>
% </A>
\week{April 26, 2007 }

Right now I'm in a country estate called Les Treilles in southern 
France, at a conference organized by Alexei Grinbaum and Michel 
Bitbol:

1) Philosophical and Formal Foundations of Modern Physics, 
<a href = "http://www-drecam.cea.fr/Phocea/Vie_des_labos/Ast/ast_visu.php?id_ast=762">http://www-drecam.cea.fr/Phocea/Vie_des_labos/Ast/ast_visu.php?id_ast=762</a>

<div align = "center">
<img src = "les_treilles.jpg">
</div>

It's very beautiful here, but about 20 philosophers, physicists 
and mathematicians have agreed to spend six days indoors discussing 
quantum gravity, the history of relativity, quantum information theory
and the like.  And guess what?  Now it's our afternoon off, and 
I'm spending my time writing This Week's Finds!  Some people just
don't know how to enjoy life.

In fact, I want to continue telling you The Tale of Groupoidification.  
But before I do, here's a puzzle that Jeffrey Bub raised the other night
at dinner.  It's not hard, but it's still a bit surprising.

You and your friend each flip a fair coin and then look at it.  
You can't look at your friend's coin; they can't look at yours.
You can't exchange any information while the game is being played,
though you can choose a strategy beforehand.   Each of you must guess 
whether the other's coin lands heads up or tails up.
Your goal, as a team, is to maximize the chance that you're both 
correct.  

What's the best strategy, and what's the probability that you
both guess correctly?

Here's an obvious line of thought.  

Since you don't have any information about your friend's coin flip, it
doesn't really matter what you guess.  So, you might as well guess
"heads".  You'll then have a 1/2 chance of being right.
Similarly, your friend might as well guess "heads" - or for
that matter, "tails".  They'll also have a 1/2 chance of
being right.  So, the chance that you're both right is 1/2 \times  1/2
= 1/4.

I hope that sounds persuasive - but you can actually do much better!

How?  I'll give away the answer at the end.  

Jeffrey Bub is famous for his work on the philosophy of quantum
mechanics, and in his talk today he mentioned a similar but more
sophisticated game, the Popescu-Rohrlich game.  Here you and your
friend each flip coins as before.  But now, after looking at your
coin, you each write either "yes" or "no" on a pad
of paper.  Your goal, as a team, is to give the same response when at
least one coin lands heads up, but different responses otherwise.

Classically the best you can do is both say "yes" - or, if you
prefer, both say "no".  Then you'll have a 3/4 chance of winning.
But, if before playing the game you and your friend prepare a pair 
of spin-1/2 particles in the Bell state, and you each keep one, you 
can use these to boost your chance of winning to about 85%!  

I think the underlying idea first appeared here:

1) S. Popescu and D. Rohrlich, Nonlocality as an axiom, Found. Phys. 
24 (1994), 379-385.

For the "game" version, try this:

2) Nicolas Gisin, Can relativity be considered complete?  From
Newtonian nonlocality to quantum nonlocality and beyond, available
as <A HREF = "http://xxx.lanl.gov/abs/quant-ph/0512168">quant-ph/0512168</A>.

There's a lot more to say about this - especially about the 
"Popescu-Rohrlich box", a mythical device which would let you win 
all the time at this game, but still not allow signalling.  The 
existence of such a box is logically possible, but forbidden by 
quantum mechanics.  It can only exist in certain "supra-quantum
theories" which allow even weirder correlations than quantum mechanics.

But, I don't understand this stuff, so you should just read this:

3) Valerio Scarani, Feats, features and failures of the PR-box,
available as <A HREF = "http://xxx.lanl.gov/abs/quant-ph/0603017">quant-ph/0603017</A>.

Okay - now for our Tale.  I want to explain double cosets as spans 
of groupoids... but it's best if I start with some special relativity.

Though Newton seems to have believed in some form of "absolute 
space", the idea that motion is relative predates Einstein by a 
long time.  In 1632, in his Dialogue Concerning the Two Chief World Systems,
Galileo wrote:

\begin{quote}
  Shut yourself up with some friend in the main cabin below decks on
  some large ship, and have with you there some flies, butterflies,
  and other small flying animals. Have a large bowl of water with some
  fish in it; hang up a bottle that empties drop by drop into a wide
  vessel beneath it.  With the ship standing still, observe carefully how
  the little animals fly with equal speed to all sides of the cabin.  The
  fish swim indifferently in all directions; the drops fall into the
  vessel beneath; and, in throwing something to your friend, you need
  throw it no more strongly in one direction than another, the distances 
  being equal; jumping with your feet together, you pass equal spaces 
  in every direction.  

  When you have observed all these things carefully (though doubtless 
  when the ship is standing still everything must happen in this way), 
  have the ship proceed with any speed you like, so long as the motion 
  is uniform and not fluctuating this way and that.  You will discover 
  not the least change in all the effects named, nor could you tell 
  from any of them whether the ship was moving or standing still.
\end{quote}

As a result, the coordinate transformation we use in Newtonian
mechanics to switch from one reference frame to another moving at 
a constant velocity relative to the first is called a "Galilei
transformation".  For example:

(t, x, y, z) |\to  (t, x + vt, y, z)

By the time Maxwell came up with his equations describing light, 
the idea of relativity of motion was well established.  In 1876,
he wrote:

\begin{quote}
  Our whole progress up to this point may be described as a gradual
  development of the doctrine of relativity of all physical phenomena.
  Position we must evidently acknowledge to be relative, for we cannot
  describe the position of a body in any terms which do not express
  relation.  The ordinary language about motion and rest does not so
  completely exclude the notion of their being measured absolutely, but
  the reason of this is, that in our ordinary language we tacitly assume
  that the earth is at rest.... There are no landmarks in space; one
  portion of space is exactly like every other portion, so that we
  cannot tell where we are.  We are, as it were, on an unruffled sea,
  without stars, compass, sounding, wind or tide, and we cannot tell in
  what direction we are going.  We have no log which we can case out to
  take a dead reckoning by; we may compute our rate of motion with
  respect to the neighboring bodies, but we do not know how these bodies
  may be moving in space.  
\end{quote}

So, the big deal about special relativity is \emph{not} that motion is
relative.  It's that this is possible while keeping the speed of 
light the same for everyone - as Maxwell's equations insist, and as 
we indeed see!  This is what forced people to replace Galilei
transformations by "Lorentz transformations", which have the 
new feature that two coordinate systems moving relative to each 
other will disagree not just on where things are, but \emph{when} they 
are.

As Einstein wrote in 1905:

\begin{quote}
 Examples of this sort, together with the unsuccessful attempts to
 discover any motion of the earth relative to the "light medium",
 suggest that the phenomena of electrodynamics as well as mechanics
 possess no properties corresponding to the idea of absolute rest.
 They suggest rather that, as has already been shown to the first
 order of small quantities, the same laws of electrodynamics and
 optics will be valid for all frames of reference for which the
 equations of mechanics are valid.  We will elevate this conjecture
 (whose content will be called the "principle of relativity") to
 the status of a postulate, and also introduce another postulate,
 which is only apparently irreconcilable with it, namely, that
 light is always propagated in empty space with a definite velocity
 c which is independent of the state of motion of the emitting
 body.  These two postulates suffice for attaining a simple and
 consistent theory of the electrodynamics of moving bodies based on
 Maxwell's theory for stationary bodies.
\end{quote}

So, what really changed with the advent of special relativity?  
First, our understanding of precisely which transformations count
as symmetries of spacetime.  These transformations form a \emph{group}.  
Before special relativity, it seemed the relevant group was a 
10-dimensional gadget consisting of:

<ul>
<li>
3 dimensions of spatial translations
<li>
1 dimension of time translations
<li>
3 dimensions of rotations
<li>
3 dimensions of Galilei transformations
</ul>

Nowadays this is called the "Galilei group":

With special relativity, the relevant group became the "Poincare 
group":

<ul>
<li>
3 dimensions of spatial translations
<li>
1 dimension of time translations
<li>
3 dimensions of rotations
<li>
3 dimensions of Lorentz transformations
</ul>

It's still 10-dimensional, not any bigger.  But, it acts differently
as transformations of the spacetime coordinates (t,x,y,z).  

Another thing that changed was our appreciation of the importance 
of symmetry!  Before the 20th century, group theory was not in the 
toolkit of most theoretical physicists.   Now it is.   

Okay.  Now suppose you're the only thing in the universe, floating in
empty space, not rotating.  To make your stay in this thought experiment 
a pleasant one, I'll give you a space suit.  And for simplicity, suppose
special relativity holds true exactly, with no gravitational fields 
to warp the geometry of spacetime.  

Would the universe be any different if you were moving at constant 
velocity?  Or translated 2 feet to the left or right?  Or turned 
around?  Or if it were one day later?

No!   Not in any observable way, at least!  It would seem exactly 
the same.  

So in this situation, it doesn't really make much sense to say
"where you are", or "which way you're facing", or
"what time it is".  There are no "invariant
propositions" to make about your location or motion.  In other
words, there's nothing to say whose truth value remains unchanged
after you apply a symmetry.

Well, \emph{almost} nothing to say!  The logicians in the crowd will
note that you can say "T": the tautologously true statement.
You can also say "F": the tautologously false statement.
But, these aren't terribly interesting.

Next, suppose you have a friend also floating through space.  Now
there are more interesting invariant propositions.  There's nothing
much invariant to say about just you, and nothing to say about just your
friend, but there are invariant \emph{relations}. For example, you
can measure your friend's speed relative to you, or your distance of
closest approach.

Mathematicians study invariant relations using a tool called "double
cosets".  I want to explain these today, since we'll need them soon 
in the Tale of Groupoidification.  

"Double cosets" sound technical, but that's just to keep timid
people from understanding the subject.  A double coset is secretly
just an "atomic" invariant relation: one that can't be
expressed as "P or Q" where P and Q are themselves invariant
relations - unless precisely one of P or Q is tautologously false.

So, atomic invariant relations are like prime numbers: they can't
be broken down into simpler bits.  And, as we'll see, every invariant
relation can be built out of atomic ones!  

Here's an example in the case we're considering:

\begin{quote}
  "My friend's speed relative to me is 50 meters/second, and our
   distance of closest approach is 10 meters."
\end{quote}

This is clearly an invariant relation.  It's atomic if we idealize 
the situation and assume you and your friends are points - so we 
can't ask which way you're facing, whether you're waving at each other, 
etc.

To see \emph{why} it's atomic, note that we can always find a frame of 
reference where you're at rest and your friend is moving by like this:


\begin{verbatim}

              -----FRIEND---->


                    YOU
\end{verbatim}
    
If you and your friend are points, the situation is <em>completely 
described</em> (up to symmetries) by the relative speed and distance 
of closest approach.  So, the invariant relation quoted above 
can't be written as "P or Q" for other invariant relations.  

The same analysis shows that in this example, \emph{every} atomic invariant 
relation is of this form:

\begin{quote}
  "My friend's speed relative to me is s, and our distance of 
  closest approach is d."
\end{quote}

for some nonnegative numbers s and d.

(Quiz: why don't we need to let s be negative if your friend is moving 
to the left?)

From this example, it's clear there are often infinitely many
double cosets.  But there are some wonderful examples with just 
\emph{finitely many} double cosets - and these are what I'll focus
on in our Tale.

Here's the simplest one.  Suppose we're doing projective plane 
geometry.  This is a bit like Euclidean plane geometry, but there are
more symmetries: every transformation that preserves lines is allowed.
So, in addition to translations and rotations, we also have other 
symmetries.  

For example, imagine taking a blackboard with some points and lines 
on it:



\begin{verbatim}

                 \             /
      ------------x-----------x-----------
                   \         /
                    \       /
                     \     /
                      \   /
                       \ /
                        x
                       / \
                      /   \
                     /     \
\end{verbatim}
    

We can translate it and rotate it.  But, we can also view it from
an angle: that's another symmetry in projective geometry!  This
hints at how projective geometry arose from the study of perspective 
in painting.  

We get even more symmetries if we use a clever trick.  Suppose we're
standing on the blackboard, and it extends infinitely like an endless
plain.  Points on the horizon aren't really points on the blackboard.
They're called "points at infinity".  But, it's nice to
include them as part of the so-called "projective plane".
They make things simpler: now every pair of lines intersects in a
unique point, just as every pair of points lies on a unique line.
You've probably seen how parallel railroad tracks seem to meet at the
horizon - that's what I'm talking about here.  And, by including these
extra points at infinity, we get extra symmetries that map points at
infinity to ordinary points, and vice versa.

I gave a more formal introduction to projective geometry in "<A
HREF = "week106.html">week106</A>" and "<A HREF =
"week145.html">week145</A>", and "<A HREF =
"week178.html">week178</A>".  If you read these, you'll know that
points in the projective plane correspond to lines through the origin
in a 3d space.  And, you'll know a bit about the group of symmetries
in projective geometry: it's the group G = PGL(3), consisting of 3\times 3
invertible matrices, modulo scalars.  

(I actually said SL(3), but I
was being sloppy - this is another group with the same Lie algebra.)

For some great examples of double cosets, let F be the space of
"flags".  A "flag" is a very general concept, but
in projective plane geometry a flag is just a point x on a line y:


\begin{verbatim}

      -----------------x----------------
                                y
\end{verbatim}
    
An amazing fact is that there are precisely 6 atomic invariant relations 
between a pair of flags.  You can see them all in this picture:


\begin{verbatim}

                 \             /
      ------------x-----------x'----------
                   \         /         y
                    \       /
                     \     /
                      \   /
                       \ /
                        x"
                       / \
                      /   \
                   y'/     \y"
\end{verbatim}
    
There are six flags here, and each exemplifies a different
atomic invariant relation to our favorite flag, say (x,y).

For example, the flag (x',y') has the following relation to (x,y):

\begin{quote}
   "The point of (x',y') lies on the line of (x,y), and no more."
\end{quote}
By "no more" I mean that no further incidence relations hold.

There's a lot more to say about this, and we'll need to delve into
it much deeper soon... but not yet.  For now, I just want to mention 
that all this stuff generalizes from G = PGL(3) to any other simple 
Lie group!  And, the picture above is an example of a general concept, 
called an "apartment".  Apartments are a great way to visualize 
atomic invariant relations between flags.

This "apartment" business is part of a wonderful theory due
to Jacques Tits, called the theory of "buildings".  The
space of \emph{all} flags is a building; a building has lots of apartments
in it.  Buildings have a reputation for being scary, because in his
final polished treatment, Tits started with a few rather unintuitive
axioms and derived everything from these.  But, they're actually lots
of fun if you draw enough pictures!

Next, let me explain why people call atomic invariant relations
"double cosets".

First of all, what's a relation between two sets X and Y?  We can
think of it as a subset S of X \times  Y: we say a pair (x,y) is in S
if the relation holds.

Next, suppose some group G acts on both X and Y.  What's an
"invariant" relation?  It's a subset S of X \times  Y such
that whenever (x,y) is in S, so is (gx,gy).  In other words, the
relation is preserved by the symmetries.

Now let's take these simple ideas and make them sound more complicated, 
to prove we're mathematicians.  Some of you may want to take a little
nap right around now - I'm just trying to make contact with the usual
way experts talk about this stuff.

First, let's use an equivalent but more technical way to think of an 
invariant relation: it's a subset of the quotient space G\(X \times  Y).  

Note: often I'd call this quotient space (X \times  Y)/G.  But now I'm 
writing it with the G on the left side, since we had a \emph{left} action
of G on X and Y, hence on X \times  Y - and in a minute we're gonna need 
all the sides we can get! 

Second, recall from last Week that if G acts \emph{transitively} on both 
X and Y, we have isomorphisms

X \cong  G/H

and 

Y \cong  G/K

for certain subgroups H and K of G.  Note: here we're really modding 
out by the \emph{right} action of H or K on G.  

Combining these facts, we see that when G acts transitively on both
X and Y, an invariant relation is just a subset of

G\(X \times  Y) \cong  G\(G/H x G/K) 

Finally, if you lock yourself in a cellar and think about this for a 
few minutes (or months), you'll realize that this weird-looking set is 
isomorphic to 

H\G/K

This notation may freak you out at first - I know it scared me!
The point is that we can take G, mod out by the right action of K 
to get G/K, and then mod out by the left action of H on G/K, obtaining

H\(G/K).

Or we can take G, mod out by the left action of H to get H\G, and then
mod out by the right action of K on H\G, obtaining

(H\G)/K.

And, these two things are isomorphic!  So, we relax and write

H\G/K

A point in here is called a "double coset": it's an equivalence class
consisting of all guys in G of the form 

hgk 

for some fixed g, where h ranges over H and k ranges over K.  

Since subsets of H\G/K are invariant relations, we can think of a 
point in H\G/K as an "atomic" invariant relation.   Every invariant 
relation is the union - the logical "or" - of a bunch of these. 

So, just as any hunk of ordinary matter can be broken down into atoms,
every invariant statement you can make about an entity of type X and
an entity of type Y can broken down into "atomic" invariant
relations - also known as double cosets!

So, double cosets are cool.  But, it's good to fit them into the "spans
of groupoids" perspective.  When we do this, we'll see:

<div align = center>
          A SPAN OF GROUPOIDS EQUIPPED WITH CERTAIN EXTRA STUFF IS <br/>
                       THE SAME AS A DOUBLE COSET.
</div>
This relies on the simpler slogan I mentioned last time:

<div align = center>
             A GROUPOID EQUIPPED WITH CERTAIN EXTRA STUFF IS <br/>
                     THE SAME AS A GROUP ACTION.
</div>

Let's see how it goes.  Suppose we have two sets on which G acts
transitively, say X and Y.  Pick a figure x of type X, and a figure
y of type Y.  Let H be the stabilizer of x, and let K be the
stabilizer of y.  Then we get isomorphisms

X \cong  G/H 

and 

Y \cong  G/K

The subgroup H \cap  K stabilizes both x and y, and

Z = G/(H \cap  K)

is another set on which G acts transitively.  How can we think of this 
set?  It's the set of all pairs of figures, one of type X and one of 
type Y, which are obtained by taking the pair (x,y) and applying
an element of G.  So, it's a subset of X \times  Y that's invariant under
the action of G.  In other words, it's an invariant relation between
X and Y!  

Furthermore, it's the smallest invariant subset of X \times  Y that contains
the pair (x,y).  So, it's an \emph{atomic} invariant relation - or in other
words, a double coset!  
  
Now, let's see how to get a span of groupoids out of this.  We have
a commutative diamond of group inclusions:


\begin{verbatim}

                      H\cap K
                      / \
                     /   \
                    /     \
                   v       v
                  H         K
                   \       /
                    \     /
                     \   /
                      v v
                       G
\end{verbatim}
    
This gives a commutative diamond of spaces on which G acts 
transitively:


\begin{verbatim}

                     G/(H\cap K)
                      / \
                     /   \
                    /     \
                   v       v
                 G/H      G/K
                   \       /
                    \     /
                     \   /
                      v v
                      G/G
\end{verbatim}
    
We already have names for three of these spaces - and G/G is just
a single point, say *:


\begin{verbatim}

                       Z
                      / \
                     /   \
                    /     \
                   v       v
                  X         Y
                   \       /
                    \     /
                     \   /
                      v v
                       *
\end{verbatim}
    
Now, in "<A HREF = "week249.html">week249</A>" I explained how you could form the "action groupoid"
X//G given a group G acting on a space X.  If I were maniacally 
consistent, I would write it as G\\X, since G is acting on the left.  
But, I'm not.  So, the above commutative diamond gives a commutative 
diamond of groupoids:


\begin{verbatim}

                      Z//G
                      / \
                     /   \
                    /     \
                   v       v
                 X//G     Y//G
                   \       /
                    \     /
                     \   /
                      v v
                     *//G
\end{verbatim}
    
The groupoid on the bottom has one object, and one morphism for each
element of G.  So, it's just G!  So we have this:


\begin{verbatim}

                      Z//G
                      / \
                     /   \
                    /     \
                   v       v
                 X//G     Y//G
                   \       /
                    \     /
                     \   /
                      v v
                       G
\end{verbatim}
    
So - voila! - our double coset indeed gives a span of groupoids


\begin{verbatim}

                      Z//G
                      / \
                     /   \
                    /     \
                   v       v
                 X//G     Y//G
\end{verbatim}
    

X//G is the groupoid of figures just like x (up to symmetry), Y//G
is the groupoid of figures just like y, and Z//G is the groupoid of 
\emph{pairs} of figures satisfying the same atomic invariant relation
as the pair (x,y).  For example, point-line pairs, where the point
lies on the line!  For us, a pair of figures is just a more complicated
sort of figure.

But, this span of groupoids is a span "over G", meaning it's
part of a commutative diamond with G at the bottom:


\begin{verbatim}

                      Z//G
                      / \
                     /   \
                    /     \
                   v       v
                 X//G     Y//G
                   \       /
                    \     /
                     \   /
                      v v
                       G
\end{verbatim}
    

If you remember everything in "<A HREF =
"week249.html">week249</A>" - and I bet you don't - you'll notice
that this commutative diamond is equivalent to diamond we started
with:


\begin{verbatim}

                      H\cap K
                      / \
                     /   \
                    /     \
                   v       v
                  H         K
                   \       /
                    \     /
                     \   /
                      v v
                       G
\end{verbatim}
    
We've just gone around in a loop!  But that's okay, because we've
learned something en route.

To tersely summarize what we've learned, let's use the fact that a 
groupoid is equivalent to a group precisely when it's "connected":
that is, all its objects are isomorphic.  Furthermore, a functor between 
connected groupoids is equivalent to an inclusion of groups precisely 
when it's "faithful": one-to-one on each homset.  So, when I said that:

<div align = "center">
          A SPAN OF GROUPOIDS EQUIPPED WITH CERTAIN EXTRA STUFF IS  <br/>
                      THE SAME AS A DOUBLE COSET.
</div>

what I really meant was:

<div align = "center">
               A SPAN OF CONNECTED GROUPOIDS FAITHFULLY OVER G<br/>
                      IS THE SAME AS A DOUBLE COSET.                      
</div>

If that's too terse, let me elaborate for you: a "span of connected 
groupoids faithfully over G" is a commutative diamond


\begin{verbatim}

                       C
                      / \
                     /   \
                    /     \
                   v       v
                  A         B
                   \       /
                    \     /
                     \   /
                      v v
                       G
\end{verbatim}
    
where A,B,C are connected groupoids and the arrows are faithful
functors.

This sounds complicated, but it's mainly because we're trying to toss
in extra conditions to make our concepts match the old-fashioned "double
coset" notion.  Here's a simpler, more general fact: 

<div align = "center">
                A SPAN OF GROUPOIDS FAITHFULLY OVER G <br/>
                  IS THE SAME AS A SPAN OF G-SETS.
</div>

where a "G-set" is a set on which G acts.  This is the natural partner 
of the slogan I explained last Week, though not in this language:

<div align = "center">
                   A GROUPOID FAITHFULLY OVER G <br/>
                     IS THE SAME AS A G-SET.
</div>

Things get even simpler if we drop the "faithfulness" assumption, and
simply work with groupoids over G, and spans of these.   This takes
us out of the traditional realm of group actions on sets, and into the
21st century!  And that's where we want to go.   

Indeed, for the last couple weeks I've just been trying to lay out the
historical context for the Tale of Groupoidification, so experts can
see how the stuff to come relates to stuff that's already known.  In
some ways things will get simpler when I stop doing this and march
ahead.  But, I'll often be tempted to talk about group actions on
sets, and double cosets, and other traditional gadgets... so I feel
obliged to set the stage.
       
Okay - here's the answer to the puzzle.  Close your eyes if you want
to think about it more.

An optimal strategy is for you and your friend to each look at your
own coin, and then guess that the other coin landed the other way:
heads if yours was tails, and tails if yours was heads.  With this
strategy, the chance you're both correct is 1/2.

Or, you can both guess that the other coin landed the \emph{same} way.
This works just as well.

The point is: you and your friend can do twice as well at this game if you each
use the result of your own coin toss to guess the result of the other's
coin toss! 

It seems paradoxical that using this random and completely uncorrelated 
piece of information - the result of your own coin toss - helps you 
guess what your friend's coin will do, and vice versa.

But of course it \emph{doesn't}.  You each still have just a 1/2 chance of
guessing the other's coin toss correctly.  What the trick accomplishes
is correlating your guesses, so you both guess right or both guess wrong 
together.  This improves the chance of winning from 1/2 \times  1/2 (the 
product of two independent probabilities) to 1/2.

By the way, the translation of the passage by Einstein is due to
Michael Friedman, a philosopher at Stanford; he used it in his talk
at this conference.  There's a lot more to say about talks at this 
conference.  Let's see if I get around to it.

Also by the way: if you fix a collection of n G-sets, there's always a
Boolean algebra of n-ary invariant relations.  Only the case n = 2 is
related to double cosets, but everything else I said generalizes
easily to higher n using "n-legged spans" of groupoids: an
obvious generalization of the 2-legged spans I've been discussing so
far.  In Boolean algebra people often use the term "atom" to
stand for an element that can't be written as "P or Q" unless exactly
one of P or Q is tautologously false.


\par\noindent\rule{\textwidth}{0.4pt}
<em>Although I am a typical loner in daily life, my consciousness
of belonging to the invisible community of those who strive for
truth, beauty and justice has preserved me from feeling isolated.</em> -
Albert Einstein

\par\noindent\rule{\textwidth}{0.4pt}

% </A>
% </A>
% </A>
