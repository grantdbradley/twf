
% </A>
% </A>
% </A>
\week{September 14, 1995}

Let me continue the tale of "ADE classifications".  Last week
I described an "ABDEFGHI classification" of all finite reflection
groups - that is, finite symmetry groups of Euclidean space,
every element of which is a product of reflections.  Now we'll build
on that to get other related classifications.

So, recall:

Every element of a finite reflection group is a product of reflections
through certain special vectors, which people call "roots".  These
roots are all at angles \pi /n from each other, where n > 1 is an
integer.  To describe the group, we draw a diagram with one dot for
each root.  If two roots are perpendicular we don't draw a line
between them, but otherwise, if they are at an angle \pi /n from each
other, we draw a line and label it with the integer n.  Actually, the
integer n = 3 comes up so often that we don't bother labelling the
line in this case.

Now, not all of these diagrams correspond to finite reflection 
groups.  The following ones, together with disjoint unions of them, 
are all the possibilities.


\begin{verbatim}

A_{n}, which has n dots like this:

o---o---o---o

B_{n}, which has n dots, where n > 1:

          4
o---o---o---o

D_{n}, which has n dots, where n > 3:

              o
             /
o---o---o---o
             \
              o

E_{6}, E_{7}, and E_{8}:

      o               o                   o
      |               |                   |
o--o--o--o--o   o--o--o--o--o--o    o--o--o--o--o--o---o


F_{4}:                   G_{2}:               H_{3} and H_{4}:

      4                6                5                5
o---o---o---o        o---o            o---o---o        o---o---o---o


I_{m}, where m = 5 or m > 6:

  m
o---o

\end{verbatim}
    

Recall that I_{m} is the symmetry group of the of regular m-gon, while
others of these are the symmetry groups of Platonic solids, and still
others are symmetry groups of regular polytopes in n-dimensional
space.  For example, the symmetry group of the dodecahedron is H_{3}, 
while that of its 4-dimensional relative is H_{4}.  

Now you may know that there are no perfect crystals in the shape of a
regular dodecahedron.  However, iron pyrite comes close.  In his wonderful
book:

1) Hermann Weyl, Symmetry, Princeton University Press, Princeton, New
Jersey, 1989.

Weyl suggests that this is how people discovered the regular dodecahedron:

\begin{quote}
    ...the discovery of the last two [Platonic solids] is certainly
    one of the most beautiful and singular discoveries made in the
    whole history of mathematics.  With a fair amount of certainty, it
    can be traced to the colonial Greeks in southern Italy.  The
    suggestion has been made that they abstracted the regular
    dodecahedron from the crystals of pyrite, a sulfurous mineral
    abundant in Sicily.
\end{quote}

Thus while iron pyrite is nothing but "fool's gold" to the miner, it
may have done a good deed by fooling the Greeks into discovering the
regular dodecahedron.  Could this be why the ratio of the diagonal
to the side of a regular pentagon, (\sqrt 5 + 1)/2, is called the golden
ratio?  Or am I just getting carried away?  One is tempted to call the
shape of pyrite crystals the "fool's dodecahedron," but in fact it's
called a "pyritohedron".  (All this information on pyrite, as well as
the puns, I owe to Michael Weiss.)

More recently, I think people have discovered "quasicrystals" (related to
Penrose tiles) having true dodecahedral symmetry.  But no perfectly
repetitive crystals form dodecahedra!  And the reason is that there is no
lattice having H_{3} as its symmetries.

Recall that we get a "lattice" by taking n linearly independent vectors
in n-dimensional Euclidean space and forming all linear combinations
with integer coefficients.  If someone hands us a finite reflection group,
we can look for a lattice having it as symmetries.  If one exists, 
we say the group satisfies the "crystallographic condition".  The only
ones that do are

A_{n}, B_{n}, D_{n}, E_{6}, E_{7}, 
E_{8}, F_{4}, and G_{2}

(and those corresponding to disjoint unions of these diagrams).
In other words, the symmetry groups of the pentagon (I_{5}), the
heptagon and so on (I_{m} with m > 6), and the dodecahedron and its
4-dimensional relative (H_{3} and H_{4}) are ruled out.  

Now let us turn to the theory of Lie groups.  Lie groups are the most
important "continuous" (as opposed to discrete) symmetry
groups.  Examples include the real line (with addition as the group
operation), the circle (with addition mod 2\pi ), and the groups SO(n)
and SU(n) discussed in "<A HREF = "week61.html">week61</A>".
These groups are incredibly important in both physics and mathematics.
Thus it is wonderful, and charmingly ironic, that the same diagrams
that classify the oh-so-discrete finite reflection groups also
classify some of the most beautiful of Lie groups: the
"simple" Lie groups.  It turns out that the simple Lie
groups correspond to the diagrams of forms A,B,D,E,F, and G.  Oh yes,
and C.  I have to tell you what happend to C.

There is a vast amount known about semisimple Lie groups, and everyone
really serious about mathematics winds up needing to learn some of this
stuff.  I took courses on Lie groups and their Lie algebras in grad
school, but it was only later that I really came to appreciate the
beauty of the simple Lie groups.  Back then I found it mystifying
because the work involved in the classification was so algebraic, 
and I preferred the more geometrical aspects of Lie groups.  Part of the
reason is that the treatment I learned emphasized the Lie algebras and
downplayed the groups.  A nice treatment that emphasizes the groups is:

2) John Frank Adams, Lectures on Lie groups, Benjamin, New York, 1969.

So what's the basic idea?  Let me summarize two semesters of grad
school, and tell you the basic stuff about Lie groups and the
classification of simple Lie groups.  Forgive me if it's a bit rushed,
sketchy, and even mildly inaccurate: hopefully the main ideas will shine
through the murk this way.

A Lie group is a group that's also a manifold, for which the group
operations (multiplication and taking inverses) are smooth functions.
This lets you form the tangent space to any point in the group, and
the tangent space at the identity plays a special role.  It's called
the Lie algebra of the group.  If we have any element x in the Lie
algebra, we can exponentiate it to get an element exp(x) in the group,
and we can keep track of the noncommutativity of the group by forming
the "bracket" of elements x and y in the Lie algebra:

[x,y] = (d/dt)(d/ds) exp(sx) exp(ty) exp(-sx) exp(-ty) 

where s and t are real numbers, and we evaluate the derivative at s,t
= 0.  Note that [x,y] = 0 if the group is commutative.  This bracket
operation satisfies some axioms, and algebraists call anything a Lie
algebra that satisfies those axioms.  For example, you could take n x
n matrices and let [x,y] = xy - yx.

Now a Lie algebra is called "semisimple" if for any z, there
are x and y with z = [x,y].  This is sort of the opposite of an
abelian, or commutative, Lie algebra, where [x,y] = 0 for all x and y.
It turns out that we can take direct sums of Lie algebras by defining
operations componentwise, and it turns out that if you have a
\emph{compact} Lie group, its Lie algebra is always the direct sum
of a semsimple Lie algebra and an abelian one.  The abelian ones are
pretty trivial, so all the hard works lies in understanding the
semisimple ones.  Any semisimple one is the direct sum of a bunch of
semisimple ones that aren't sums of anything else, and these basic
building blocks are called the "simple" ones.  They are like
the prime numbers of Lie algebra theory.  Unlike the prime numbers,
though, we can completely classify all of them!

Now how does one classify the simple Lie algebras?  Basically, it goes
like this.  We'll assume our simple Lie algebra is the Lie algebra of a
compact Lie group G - it turns out that they all are.  Now, sitting
inside G there is a maximal commutative subgroup T that's a torus: a
product of a bunch of circles.  Let Lie(T) stand for the Lie algebra of
this torus T.  Now, sitting inside Lie(T) there is a lattice, consisting
of all elements x with exp(x) = 1.  This is how lattices sneak into the
picture!

Moreover, for some elements g in G, if we "conjugate" T by
g, that is, form the set of all elements gtg^{-1} where t is
in T, we get T back.  In other words, these elements of g act as
symmetries of the torus T.  Now, if something acts as symmetries of
something else, it also acts as symmetries of everything naturally
cooked up from that something else.  (Roughly speaking,
"naturally" means "without dependence on arbitrary
choices.)  For this reason, these special elements of G also act as
symmetries of Lie(T) and of the lattice sitting inside Lie(T).  So we
have a lattice together with a group of symmetries, which by the way
is called the Weyl group of G.  Now the cool part is that the Weyl
group is actually a finite reflection group, so it must correspond to
one of the diagrams in the list above!  Even better, it turns out that
the Lie algebra of G is determined by the lattice together with its
Weyl group.

The upshot is that to classify semisimple Lie algebras, all we need is
the classification of finite reflection groups satisfying the
crystallographic condition - which we've done already using diagrams
- together with a classification of lattices having such finite
reflection groups as symmetries.  It turns out that the operation of
taking direct sums of semisimple Lie algebras corresponds to taking
disjoint unions of diagrams, so to get the "building blocks" - the
\emph{simple} Lie algebras - we only need to worry about the diagrams we've
drawn above, not disjoint unions of them.  Now it turns out that for
every type except B, there is (up to isomorphism) only \emph{one} lattice
having that group of symmetries, but for B there are two.  Recall the
diagram B_{n} looks like:


\begin{verbatim}

          4
o---o---o---o
\end{verbatim}
    

with n dots.  And recall that the dots correspond to
"roots", which in the present context are vectors in Lie(T).
Now it turns out that whenever we have a finite reflection group
satisfying the crystallographic condition, we can get a lattice having
it as symmetries by taking integer linear combinations of the roots,
but \emph{not} necessarily roots that are unit vectors; the lengths
of the roots matter.  In all cases except B, there is basically just
one way to get the lengths right, but for B there are two.  We can
keep track of the root lengths with some extra markings on our
diagrams, and then we get two diagrams, which we call B_{n}
and C_{n}.  One of them has the root at the right of the
diagram being longer, and one has the root right next to it being
longer.  This makes no difference when n = 2, since then we just have


\begin{verbatim}

  4
o---o
\end{verbatim}
    

which is perfectly symmetrical.  So folks usually consider C_{n}
only for n > 2, to avoid double counting.  

In other words, all the simple Lie algebras are of the form:

<UL>
<LI>
A_{n}, n > 0
<LI>
B_{n}, n > 1
<LI>
C_{n}, n > 2
<LI>
D_{n}, n > 3
<LI>
E_{6}, E_{7}, E_{8}
<LI>
F_{4}
<LI>
G_{2}


% parser failed at source line 316
