
% </A>
% </A>
% </A>
\week{December 16, 1996 }

Lots of cool papers have been appearing which I've been neglecting
in my attempts to write expository stuff about string theory, lattices,
category theory, and all that.  It's time to start catching up!

Let me start with the following book:

1) J. Scott Carter, Daniel E. Flath and Masahico Saito,
"The Classical and Quantum 6j-Symbols", Princeton University
Press, Princeton, 1995.  ISBN 0-691-02730-7.

Ever since Jones discovered the Jones polynomial invariant of knots,
an amazing story has been unfolding about the relation between algebra
and 3-dimensional topology.  Some key players in this story are the 
"quantum groups": certain noncommutative algebras analogous to the 
commutative algebras of functions on groups.   In fact, not merely
are they analogous, they depend on a parameter, usually called
Planck's constant or \hbar , and in the classical limit where \hbar  \to  0
they actually reduce to algebras of functions on familiar groups.  The 
simplest case is "quantum SU(2)", which reduces in the classical
limit to the group SU(2) of 2x2 unitary matrices with determinant 1.
Ironically, it's good old "classical SU(2)" that governs the quantum 
mechanical theory of angular momentum.   Quantum SU(2) was first 
discovered by people working on physics in 2-dimensional spacetime,
where when you quantize certain systems you also need to quantize their
group of symmetries!  

Nowadays, mathematicians find it simpler to work with the closely 
related "quantum SL(2)", a quantization of the the group SL(2) 
of all 2x2 complex matrices with determinant 1.  The above book is 
largely about quantum SL(2) and its applications to topology.


 All quantum groups give rise to invariants of knots, links, and
tangles.  They also give rise to 3-dimensional topological quantum
field theories of "Turaev-Viro type".  This is a kind of
quantum field theory you can define on a 3-dimensional spacetime that
you've triangulated, i.e., chopped up into tetrahedra.  One of the
main things you want to compute in a quantum field theory is the
"partition function", and we say the Turaev-Viro theories
are "topological" because you get the same answer for the
partition function no matter how you triangulate the 3-dimensional
manifold corresponding to your spacetime: the partition function only
depends on the topology of the manifold.  The SU(2) Turaev-Viro
theory, the first one to be discovered, is also one of the most
interesting because, modulo a few subtle points, this theory is just
quantum gravity in 3 dimensions (see "<A HREF =
"week16.html">week16</A>").  The basic idea, though, is that you
compute the partition function by summing over all ways of labelling
the edges of your tetrahedra by "spins" j = 0, 1/2, 1, 3/2,
etc..  Ponzano and Regge had tried to set up 3-dimensional quantum
gravity this way previously, but there were problems getting the sum
to converge.  The neat thing about the quantum group is that you only
sum over spins less than some fixed spin depending on the value of
\hbar .  Since the sums are finite, they automatically converge.


 It turns out that in these Turaev-Viro theories you are not
actually taking advantage of all the structure of the quantum group.
Using the extra structure, you can also use quantum groups to define
certain \emph{4-dimensional} topological quantum field theories, those of
"Crane-Yetter-Broda" type.  Here you triangulate a
4-dimensional manifold and, in the SU(2) case, you label both the 2d
faces the 3d tetrahedra with spins.  Actually, lots of people think
the Crane-Yetter-Broda theories are boring, because they look sort of
boring if you only examine their implications for 4-dimensional
topology.  However, they become interesting when you realize that,
like all topological quantum field theories defined using
triangulations, they are "extended topological quantum field
theories".  Roughly speaking this means that they have
implications for all dimensions below the dimension they live in.


 In particular, the Crane-Yetter-Broda theories spawn 3-dimensional
topological quantum field theories of
"Chern-Simons-Reshetikhin-Turaev" type, and most people
agree that \emph{these} are interesting.  I like to emphasize, however,
that a deep understanding of these 3-dimensional progeny requires an
understanding of their seemingly innocuous 4-dimensional ancestors.
Also, there are a lot of interesting relationships between the SU(2)
Crane-Yetter-Broda model and quantum gravity in 4 dimensions, which we
are just beginning to understand.  See "<A HREF =
"week56.html">week56</A>" for a bit about this.

If you haven't yet joined the fun, Carter, Saito, and Flath's book is a 
great place to start learning about the marvelous interplay between 
algebra, topology, and physics in 3 and 4 dimensions.  Needless to say, 
it doesn't cover all the ground I've sketched above.  Instead, it focuses 
on a rather specific and concrete aspect: the 6j symbols.  This should
make it especially handy for beginners who aren't familiar with category
theory, path integrals, and all that jazz.  


 What are the 6j symbols, anyway?  Here I need to get a wee bit
more technical.  The "classical" 6j symbols are important in
the representation theory of plain old classical SU(2), while the
"quantum" ones are analogous gadgets applicable to quantum
SU(2).  In either case the idea is the same.  SU(2), classical or
quantum, has different representations corresponding to different
spins j = 0, 1/2, 1, 3/2, etc..  (If you don't know what I mean by
this, try "<A HREF = "week5.html">week5</A>".)  If we take
three representations j1, j2, and j3, we can tensor them either like
this:

(j1 tensor j2) tensor j3

or like this

j1 tensor (j2 tensor j3)

The tensor product is associative, but that doesn't mean that the
above two representations are \emph{equal}.  They are only \emph{isomorphic}.
This \emph{isomorphism} can be thought of as just a big fat matrix, and
the entries in this matrix are a bunch of numbers, the 6j symbols.  

Turaev and Viro used the quantum 6j symbols to define the original 
Turaev-Viro model.   It goes like this: first you chop your 3-dimensional
manifold up into tetrahedra, and then you consider all possible ways
of labelling the edges with spins.  Each tetrahedron gets labelled with
6 spins since it has 6 edges, and from these spins we can compute a 
number: the 6j symbol.  Then we multiply all these together, one for each 
tetrahedron, and finally we sum over labellings to get the partition function.
Marvelously, the identities satisfied by the 6j symbols are precisely what's 
needed to make the result independent of the triangulation!  See "<A HREF = "week38.html">week38</A>" for
an explanation of this seeming miracle: it's actually no miracle at all.

2) E. Guadagnini, L. Pilo, Three-manifold invariants and their relation 
with the fundamental group, 22 pages in LaTeX available as <A HREF = "http://xxx.lanl.gov/ps/hep-th/9612090">hep-th/9612090</A>.

Fans of topological field theory may like this one, though I must
admit I haven't gotten around to doing more than reading the abstract
yet.  In this paper the authors give evidence for the conjecture
that among 3-manifolds M for which the Chern-Simons invariant CS(M) is
nonzero, the absolute value |CS(M)| only depends on the fundamental
group of M.  Chern-Simons theory depends on a choice of group; they 
prove the conjecture for certain manifolds ("lens spaces") when
the group is SU(2), and give numerical evidence when the gauge group
is SU(3).  

What's interesting about this to me is that |CS(M)|^2 is just the
Turaev-Viro theory partition function, so this conjecture is saying
that the Turaev-Viro theories discussed above have a tendency to notice
only the fundamental group. 

3) Michael Reisenberger and Carlo Rovelli, "Sum over surfaces" form
of loop quantum gravity, preprint available as <A HREF = "http://xxx.lanl.gov/ps/gr-qc/9612035">gr-qc/9612035</A>.

This wonderful paper should really push forwards our understanding
of the loop representation of quantum gravity.  I talked a little
bit about the basic idea in "<A HREF = "week86.html">week86</A>".  In the loop representation,
a state of quantum gravity at a given moment is represented by a bunch
of knotted loops or "spin networks" in space.  What's the spacetime
picture?  Well, if you have a surface in spacetime and look at it
at one moment of time, it typically looks like a bunch of loops... so
maybe the spacetime picture of quantum gravity is that spacetime is
packed with 2-dimensional surfaces, all tangled up.  Interestingly,
this is also very reminiscent of the picture of quantum gravity in 
string theory! 

I've been working on this sort of idea ever since I wrote a paper
suggesting that the loop representation and string theory might be
two faces of the same ideas (see "<A HREF = "week18.html">week18</A>").  Since then, most of the 
time I've been trying to understand how these ideas relate to the 
Crane-Yetter-Broda theories, and trying to set up the necessary \emph{algebra} 
(n-category theory) to deal nicely with surfaces in 4-dimensional spacetime.

But there are many other angles from which one can attack this problem,
and one of the best is to start directly from Einstein's equations
for general relativity, try to quantize them using the path-integral
approach, and see how the path integral can be written as a sum over
surfaces.  Reisenberger has already begun work on this in the context of
"simplicial quantum gravity" - where you chop spacetime up into 
the 4-dimensional analog of tetrahedra.   But during the Vienna workshop
on canonical quantum gravity this summer, we talked about a different,
still more direct approach (see "<A HREF = "week89.html">week89</A>").  The idea is to copy standard
quantum field theory, write the propagator describing time evolution
as a time-ordered exponential, and interpret the terms in the resulting
sum as surfaces in spacetime.  It's all very analogous to traditional
Feynman diagrams, where you write the propagator as a sum over diagrams,
but now the "Feynman diagrams" are 2-dimensional surfaces.   (Again, this
is reminiscent of string theory - but with many important differences.)

There is much more to say, but I think I'll leave it at that.... 
Over in the world of n-categories there is also some very interesting 
stuff happening, which I will discuss more next week.  I'm almost done
writing a paper with James Dolan on the definition of n-categories, but 
in the meantime some other folks have been coming up with other definitions of
n-categories, so we will soon be in the position to compare definitions and
see how similar or different they are, and start erecting the formalism
needed to deal with all these topological quantum field theories and
"sums over surfaces" in a really elegant way!  Everything looks like its
fitting together.  At least, that's my momentary optimistic feeling.  
Perhaps it's just the fact that classes are over that is making me so happy.  
Yes, it's probably just that.

<HR>

% </A>
% </A>
% </A>


% parser failed at source line 261
