
% </A>
% </A>
% </A>
\week{August 6, 1996 }

Let me continue my reportage of what happened at the Mathematical
Problems of Quantum Gravity workshop in Vienna.  I will only write
about quantum gravity aspects here.  I had an interesting conversation
with Bertram Kostant in which he explained to me the deep inner
secrets of the exceptional Lie group E8.  However, my writeup of that
has grown to the point where I will save it for some other week.

By the way, my course on n-category theory is not over!  I'm merely
taking a break from it, and will return to it after this workshop.

So...

Wednesday, July 10th - Jerzy Lewandowski gave a talk on the "Spectrum
of the Area Operator".  As I've mentioned a few times before, one of
the exciting things about the loop representation of quantum gravity
is that the spectrum of the area operator associated to any surface is
discrete.  In other words, area is quantized! 

Let me describe the area operator a bit more precisely.  Before I tell
you what the area operator is, I have to tell you what it operates on.
Remember from "<A HREF = "week43.html">week43</A>" that there are various Hilbert spaces floating
around in the canonical quantization of gravity.  First there is the
"kinematical state space".  In the old-fashioned metric approach to
quantum gravity, known as geometrodynamics, this was supposed to be
L^2(Met), where Met is the space of Riemannian metrics on space.  (We
take as space some 3-manifold S, which for simplicity we assume is
compact).  The problem was that nobody knew how to rigorously define
this Hilbert space L^2(Met).  In the "new variables" approach to quantum
gravity, the kinematical state space is taken instead to be L^2(A),
where A is the space of connections on space on some trivial SU(2)
bundle over S.  This \emph{can} be defined rigorously.

Here's the idea, roughly.  Fix any graph g, with finitely many edges
and vertices, embedded in S.   Let A_g, the space of connections
on that graph, be SU(2)^n where n is the number of edges in e.  Thus
a connection on a graph tells us how to parallel transport things along
each edge of that graph --- an idea familiar from lattice gauge theory.
L^2(A_g) is well-defined because SU(2) has a nice measure on it, namely
Haar measure, so A_g gets a nice measure on it as well.  

Now if one graph g is contained in another graph h, the space L^2(A_g)
is contained in the space L^2(A_h) in an obvious way.  So we can form
the union of all the Hilbert spaces L^2(A_g) and get a big Hilbert
space L^2(A).  Mathematicians would say that L^2(A) is the "projective
limit" of the Hilbert spaces L^2(A_g), but let's not worry about that.

So that's how we get the space of "kinematical states" in the loop
representation of quantum gravity.  The space of physical states is
then obtained by imposing constraints: the Gauss law constraint (i.e.,
gauge-invariance), the diffeomorphism constraint (i.e., invariance
under diffeomorphisms of S) and the Hamiltonian constraint (i.e.,
invariance under time evolution).  States in the physical state space
are supposed to only contain information that's invariant under all
coordinate transformations and gauge transformations --- the really
physical information.

I explained these constraints to some extent in "<A HREF = "week43.html">week43</A>", and I don't
really want to worry about them here.  But let me just mention that
the Gauss law constraint is easy to impose in a mathematically
rigorous way.  The diffeomorphism constraint is harder but still
possible, and the Hamiltonian constraint is the big thorny question
plaguing quantum gravity --- see "<A HREF = "week85.html">week85</A>" for some recent progress on
this.  The area operators I'll be talking about are self-adjoint
operators on the space of kinematical states, L^2(A), and are
a preliminary version of some related operators one hopes eventually
to get on the physical state space, after much struggle and sweat.

To define an operator on L^2(A) it's enough to define it on L^2(A_g)
for every graph g and then check that these definitions fit together
consistently to give an operator on the big space L^2(A).  So let's
take a graph g and a surface s in space.  The area operator we're
after is supposed to be the quantum analog of the usual classical
formula for the area of s.  The usual classical area is a function of
the metric on space; similarly, the quantum area is an operator on the
space L^2(A).  

The area operator only cares about the points where the graph
intersects the surface.  We assume that there are only finitely many
points where it does so, apart from points where the edges are tangent
to the surface.  (To make this assumption reasonable, we need to
assume, e.g., that the space S has a real-analytic structure and the
surface and graph are analytic --- an annoying technicality that I
have been seeking to eliminate.)
 
The area operator is built using three operators on L^2(SU(2)) called
J_1, J_2, and J_3, the self-adjoint operators corresponding to the 3
generators of SU(2) --- which often show up in physics as the three
components of angular momentum!  Alternatively, we can think of all
three together as one vector-valued operator J, the "angular momentum
operator".  Note that L^2(A_g) is just the tensor product of one copy
of the Hilbert space L^2(SU(2)) for each edge of our graph g.  Thus
for any edge e we get an angular momentum operator J(e) that acts on
the copy of L^2(SU(2)) corresponding to the edge e in question,
leaving the other copies alone.

This, then, is how we define the operator on L^2(A_g) corresponding to
the area of the surface s.  Pick an orientation for the surface s.
For any point where the graph g intersects s, let J(in) denote the sum
of the angular momentum operators of all edges intersecting s at the
point in question and pointing "inwards" relative to our chosen
orientation.  Similarly, let J(out) be the sum of the angular momentum
operators of edges intersecting s at the point in question and
pointing "outwards".  (Note: edges tangent to the surface contribute
neither to J(in) nor J(out).)  Now sum up, over all points where the
graph intersects the surface, the following quantity:

sqrt[(J(in) - J(out)) . ((J(in) - J(out))]

where the dot denotes the obvious sort of dot product of vector-valued
operators.  Multiply by half the Planck length squared and you've got the
area operator!  

This very beautiful and simple formula was derived by Ashtekar and
Lewandowski, but the first people to try to quantize the area operator
were Rovelli and Smolin; see

1) Discreteness of area and volume in quantum gravity, by Carlo Rovelli
and Lee Smolin, 36 pages in LaTeX format, 13 figures uuencoded,
available as <A HREF = "http://xxx.lanl.gov/abs/gr-qc/9411005">gr-qc/9411005</A>.

Abhay Ashtekar and Jerzy Lewandowski, Quantum theory of geometry I:
area operators, 31 pages in LaTeX format, to appear in the Classical
and Quantum Gravity special issue dedicated to Andrzej Trautman,
preprint available as <A HREF = "http://xxx.lanl.gov/abs/gr-qc/9602046">gr-qc/9602046</A>.

In his talk Jerzy showed how to work the spectrum of the area operator
(which is discrete) and showed how it could depend on whether the
surface s cuts space into 2 parts or not.

Later that day, Mike Reisenberger, Matthias Blau, Carlo Rovelli and I
talked about the relation between string theory and the loop
representation of quantum gravity.  

Mike has been working on a very interesting "state sum model" for
quantum gravity; that is, a discretized model in which spacetime is
made of 4-simplices (the 4d version of tetrahedra), fields are thought
of ways of labelling the faces, edges and so on by spins, elements of
SU(2) and the like, and the path integral is replaced by a sum over
these labellings.  This works out quite nicely for quantum gravity in
3 dimensions --- see "<A HREF = "week16.html">week16</A>" --- but it's much more challenging in
4 dimensions.  

One nice feature of these state sum models for quantum gravity is that
they may be reinterpreted as sums over "worldsheets" --- surfaces
mapped into spacetime.  Since the spacetime is discrete, so are these
surfaces --- they're made of lots of triangles --- but apart from
that, having a path integral that's a sum over worldsheets is
pleasantly reminscent of string theory.  Indeed, once upon a time I
proposed that the loop representation of quantum gravity and string
theory were two aspects of some theory still waiting to be fully
understood:

2) John Baez, Strings, loops, knots, and gauge fields, in "Knots and
Quantum Gravity", ed. J. Baez, Oxford U. Press, Oxford, 1994, preprint
available in LaTeX form as <A HREF = "http://xxx.lanl.gov/abs/hep-th/9309067">hep-th/9309067</A>, 34 pages.   

The problem was getting a concrete way to relate the Lagrangian for
the string theory to the Lagrangian for gravity (or whatever gauge
theory one started with).  Iwasaki tackled this problem was tackled in
3d quantum gravity using state sum models: 

3) Junichi Iwasaki, A reformulation of the Ponzano-Regge quantum
gravity model in terms of surfaces, University of Pittsburgh, 11 pages
in LaTeX format available as <A HREF = "http://xxx.lanl.gov/abs/gr-qc/9410010">gr-qc/9410010</A>.

Later, Reisenberger extended this approach to deal with certain 4d
theories which are simpler than quantum gravity, like BF theory:

4) Michael Reisenberger, Worldsheet formulations of gauge theories and
Gravity, University of Utrecht preprint, 1994, available as
<A HREF = "http://xxx.lanl.gov/abs/gr-qc/9412035">gr-qc/9412035</A>.

In all of these theories, one computes the action for the worldsheets
by summing something over places where they intersect.  In other
words, they "interact" at intersections.  

But the really exciting thing would be to do something like this for
Mike's new state sum model for 4d quantum gravity.  And the real
challenge would be to relate this --- if possible! --- to conventional
string theory.  In a coffeeshop I suggested that one might do this by
using the usual formula for the action in (bosonic) string theory.
This is simply the \emph{area} of the string worldsheet with respect to
some background metric.  The loop representation of quantum gravity
doesn't make reference to any background metric; the closest
approximation to a classical metric is a "weave" state in which space
is tightly packed with lots of loops or spin networks.  From the 4d
point of view, we'd expect this to correspond to a spacetime packed
with lots of worldsheets.  Now, given the relation between area and
intersection number in the loop representation (see above!), one might
expect the area of a given worldsheet to be roughly proportional to
the number of its intersections with the other worldsheets in this
"weave".  But this is what one would expect in any theory where the
worldsheets interact at intersections.  So, one could hope that Mike's
state sum model would be approximately equivalent to a string theory
of the sort string theorists study.

There are lots of obvious problems with this idea, but it led to an
interesting conversation, and I am still not convinced that it is
crazy.  

Thursday, July 11th - Jorge Pullin spoke on skein relations and
the Hamiltonian constraint in lattice quantum gravity.  His idea
was that the Hamiltonian constraint contains a "topological factor"
which serves as a skein relation on loop states.  

Friday, July 12th - Abhay Ashtekar gave a talk on "Noncommutativity
of Area Operators".  This explained how the rather shocking fact that
the area operators for two intersecting surfaces needn't commute actually 
has a perfect analog in classical general relativity.  

Mike Reisenberger spoke on "Euclidean Simplicial GR".  This presented
the details of his state sum model.  Since he hasn't published this
yet, and since I am getting a bit tired out, I won't describe it here.

Monday, July 15th - Renate Loll gave a talk on the volume and area
operators in lattice gravity.  I wrote a bit about her work on the
volume operator in "<A HREF = "week55.html">week55</A>", and more can be found in:
  
5) Renate Loll, The volume operator in discretized quantum gravity,
preprint available as <A HREF = "http://xxx.lanl.gov/abs/gr-qc/9506014">gr-qc/9506014</A>, 15 pages.  

Renate Loll, Spectrum of the volume operator in quantum gravity, 
preprint available as <A HREF = "http://xxx.lanl.gov/abs/gr-qc/9511030">gr-qc/9511030</A>, 14 pages. 

Also, Jerzy Lewandowski spoke on his work with Ashtekar on
the volume operator in the continuum theory:

6) Jerzy Lewandowski, Volume and quantizations, preprint available
as <A HREF = "http://xxx.lanl.gov/abs/gr-qc/9602035">gr-qc/9602035</A>, 8 pages.  

Abhay Ashtekar and Jerzy Lewandowski, Quantum theory of geometry II: 
volume operators, manuscript in preparation.

The volume operator is more tricky than the area operator, and various
proposed formulas for it do not agree.  This is summarized quite
clearly in Jerzy's paper.

In fact, I have already left Vienna by now.  I was too busy there
to keep up with This Week's Finds, but my life is a bit calmer now
and I will try to finish these reports soon.


\par\noindent\rule{\textwidth}{0.4pt}

% </A>
% </A>
% </A>
