
% </A>
% </A>
% </A>
\week{December 24, 2005 }


Happy holidays!  I'll start with some gift suggestions for people
who put off their Christmas shopping a bit too long, before moving 
on to this week's astronomy pictures and then some mathematical 
physics: minimal surfaces.

Back in 2000 I listed some gift ideas in 
"<A HREF = "week162.html">week162</A>".  I decided to do 
it again this year.  After all, where else can you read about quantum 
gravity, nonabelian cohomology, higher categories... and also get 
helpful shopping tips?  But, I put off writing this Week's Finds a
bit too late.  Oh well.

I just saw this book in a local store, and it's \emph{great}:

1) Robert Dinwiddie, Philip Eales, David Hughes, Ian Nicholson, Ian Ridpath, 
Giles Sparrow, Pam Spence, Carole Stott, Kevin Tildsley, and Martin Rees, 
Universe, DK Publishing, New York, 2005.

If you like the astronomy pictures you've seen here lately, you'll love
this book, because it's \emph{full} 
of them - all as part of a well-organized, 
clearly written, information-packed but nontechnical introduction to 
astronomy.  It starts with the Solar System and sails out through the 
Oort Cloud to the Milky Way to the Local Group to the Virgo Supercluster
... and all the way out and back to the Big Bang!

The only thing this book seems to lack - though I could have missed it -
is a 3d map showing the relative scales of our Solar System, Galaxy, and 
so on.  I recommended a wall chart like this 
back in "<A HREF = "week162.html">week162</A>", and my 
friend Danny Stevenson just bought me one.  I'll probably put it up 
near my office in the math department... gotta keep the kids thinking big!

You don't really need to buy a chart like this.  You can just look at 
this website:

2) Richard Powell, An Atlas of the Universe, 
<A HREF = "http://www.anzwers.org/free/universe/">
http://www.anzwers.org/free/universe/</A>

It has nine maps, starting with the stars within 12.5 light years and
zooming out repeatedly by factors of 10 until it reaches the limits of 
the observable universe, roughly 14 billion light years away.  Or more 
precisely, 14 billion years ago!  

(The farther we look, the older things we see, since light takes time to
travel.  The most distant thing we see is light released when hot gas 
from the Big Bang cooled down just enough to let light through!  If we 
calculate how far this gas would be \emph{now}, 
thanks to the expansion of the 
universe, we get a figure of roughly 78 billion light years.  But of course 
we can't see what that gas looks like \emph{now} 
unless we wait a lot longer.  
It's a bit confusing until you think about it for a while.)

For example, here are the clusters of galaxies 
within 100 million lightyears of us: 

<DIV ALIGN = CENTER>
<A HREF = "http://www.anzwers.org/free/universe/virgo.html">
<IMG SRC = "Virgo_supercluster.gif">
% </A>
</DIV>

The biggest of these is the Virgo cluster, which I discussed
in "<A HREF = "week224.html">week224</A>".  This contains
about 2000 galaxies.  The second biggest is the Fornax cluster.
The whole agglomeration shown here is called the Virgo Supercluster.
Superclusters are among the biggest structures in the universe.

This atlas is fun to browse when you're at your computer.
But, if someone you know wants to contemplate the universe in a more 
relaxing way, try getting them one of these:

3) Bathsheba Grossman, Crystal model of a typical 100-megaparsec cube
of the universe, <A HREF = "http://www.bathsheba.com/crystal/largescale/">
http://www.bathsheba.com/crystal/largescale/</A>

Crystal model of the Milky Way, 
<A HREF = "http://www.bathsheba.com/crystal/galaxy/">http://www.bathsheba.com/crystal/galaxy/</A>

I found out about these from David Scharffenberg, who owns the Riverside
Computer Center nearby - a cool little shop that's decorated with archaic 
technology ranging from a mammoth slide rule to a gizmo that computes 
square roots using air pressure.  He gave me the 100-megaparsec cube as 
a present, and it's great!  It's lit up from below, and it shows the 
filaments, sheets and superclusters of galaxies that reign supreme 
at this distance scale.  100 megaparsecs is about 300 million light years, 
so this view is a bit bigger than the previous picture:

<DIV ALIGN = CENTER>
<A HREF = "http://www.bathsheba.com/crystal/largescale/">
<IMG SRC = "largescale_movie2.gif">
% </A>
</DIV>

David says Grossman's model of 
the Milky Way is also nice: it takes into account the latest 
research, which shows our galaxy is a "barred" spiral!  
You can see the bar in the middle here:

<DIV ALIGN = CENTER>
<A HREF = "http://www.spitzer.caltech.edu/Media/mediaimages/sig/sig05-010.shtml">
<IMG SRC = "milky_way_bar.jpg">
% </A>
</DIV>

4) R. Hurt, NASA/JPL-Caltech, Milky Way Bar, 
<A HREF = 
"http://www.spitzer.caltech.edu/Media/mediaimages/sig/sig05-010.shtml">
http://www.spitzer.caltech.edu/Media/mediaimages/sig/sig05-010.shtml</A>

If you really have money to burn, Grossman has also made nice sculptures
of mathematical objects like the 24-cell, the 600-cell and Schoen's
gyroid - a triply periodic minimal surface that chops 3-space into two parts:

<DIV ALIGN = CENTER>
<A HREF = "http://www.bathsheba.com/math/gyroid/gyroid_hex.jpg">
<IMG SRC = "gyroid_hex.jpg">
% </A>
</DIV>

5) Bathsheba Grossman, Mathematical models, <A HREF = "http://www.bathsheba.com/math/">http://www.bathsheba.com/math/</A>

However, the great thing about the web is that lots of beautiful stuff
is free - like these \emph{pictures} of the gyroid:

<DIV ALIGN = CENTER>
<A HREF = "http://www.bathsheba.com/math/gyroid/gyroid3q.jpg">
<IMG SRC = "gyroid3q.jpg">
% </A>
</DIV>


I explained the 24-cell and 600-cell in 
"<A HREF = "week155.html">week155</A>".  So, let me explain 
the gyroid - then I need to start cooking up a Christmas eve dinner!

A "minimal surface" is a surface in ordinary 3d space that 
can't reduce
its area by changing shape slightly.  You can create a minimal surface 
by building a wire frame and then creating a soap film on it.  As long 
as the soap film doesn't actually enclose any air, it will try to minimize
its area - so it will end up being a minimal surface.  

If you make a minimal surface this way, it will have edges along the wire
frame.  A minimal surface without edges is called "complete".  
For a long time,
the only known complete minimal surfaces that didn't intersect
themselves were the plane, the catenoid, and the helicoid.   You get a 
"catenoid" by taking an infinitely long chain and letting it hang to 
form a curve called a "catenary"; then you use this curve to 
form a surface of revolution, which is the catenoid:

6) Eric Weisstein, Catenoid, from Mathworld, 
<A HREF = "http://mathworld.wolfram.com/Catenoid.html">http://mathworld.wolfram.com/Catenoid.html</A>

In cylindrical coordinates the catenoid is given by the 
equation

r = c cosh(z/c) 

for your favorite constant c.

A "helicoid" is like a spiral staircase; in cylindrical 
coordinates it's given by the equation

z = c \theta 

for some constant c.   You can see a helicoid here - and see how it 
can continuously deform into a catenoid:

7) Eric Weisstein, Helicoid, from Mathworld,
<A HREF = "http://mathworld.wolfram.com/Helicoid.html">http://mathworld.wolfram.com/Helicoid.html</A>

In 1987 a fellow named Hoffman discovered a bunch more complete 
non-self-intersecting minimal surfaces with the help of a computer:

8) D. Hoffman, The computer-aided discovery of new embedded minimal 
surfaces, Mathematical Intelligencer 9 (1987), 8-21.

Since then people have gotten good at inventing minimal surfaces.  
You can see a bunch here:

9) GRAPE (Graphics Programming Environment), Surface overview,
<A HREF = "http://www-sfb256.iam.uni-bonn.de/grape/EXAMPLES/AMANDUS/bmandus.html">http://www-sfb256.iam.uni-bonn.de/grape/EXAMPLES/AMANDUS/bmandus.html</A>

10) GANG (Geometry Analysis Numerics Graphics), Gallery of minimal
surfaces, <A HREF = "http://www.gang.umass.edu/gallery/min/">http://www.gang.umass.edu/gallery/min/</A>
 
As you can see, people who work on mininal surfaces like goofy acronyms.
If you look at the pictures, you can also see that a minimal surface 
needs to be locally saddle-shaped.  More precisely, it has "zero mean 
curvature": at any point, if it curves one way along one principal
axis of curvature, it has to curve an equal and opposite amount along 
the perpendicular axis.  Supposedly this was proved by Euler.  

If we write this requirement as an equation, we get a second-order nonlinear
differential equation called "Lagrange's equation" - a special case 
of the Euler-Lagrange equation we get from any problem in the variational
calculus.  So, finding new minimal surfaces amounts to finding new 
solutions of this equation.  Soap films solve this equation automatically,
but only with the help of a wire frame; it's a lot more work to find
minimal surfaces that are complete.

For the theoretical physicist, minimal surfaces also go by another
name: \emph{strings!}  The "worldsheet" of
a bosonic string is just a 2-dimensional
surface in spacetime.  The equation governing the string's motion just
says that the area of this surface can't be reduced by wiggling it 
slightly.  In other words, it's just Lagrange's equation.
There's a big difference between string theory 
and the theory of minimal surfaces, though: in string theory we need 
to take quantum mechanics into account!  (Another big difference is
that spacetime is a Lorentzian rather than Riemannian manifold, unless
we do a trick called "Wick rotation".)

So, bosonic string theory is about the quantum version of soap films -
and "D-branes" serve as the wire frames.  But if this reminds you of "spin 
foams", I should warn you: there are a few big differences.  The main 
thing is that spin foams are background-free: they don't live in spacetime, 
they 
\emph{are} spacetime.  So, it doesn't make any obvious sense for them to 
minimize area, though Smolin has suggested it might make an \emph{unobvious}
kind of sense.  All the fun must happen when the "bubbles" of a spin 
foam meet along their edges... but we don't really know how this should work, 
to create a foam with the right consistency at large scales.  

Anyway....

There are a lot of minimal surfaces that have periodic symmetry in
3 directions, like a crystal lattice.  You can learn about them here:

11) Elke Koch, 3-periodic minimal surfaces, 
<A HREF = "http://staff-www.uni-marburg.de/~kochelke/minsurfs.htm">http://staff-www.uni-marburg.de/~kochelke/minsurfs.htm</A>

In fact, they have interesting relations to crystallography: 

12) Elke Koch and Werner Fischer, Mathematical crystallography
<A HREF = "http://www.staff.uni-marburg.de/~kochelke/mathcryst.htm#minsurf">http://www.staff.uni-marburg.de/~kochelke/mathcryst.htm#minsurf</A>

I guess people can figure out which of the 230 crystal symmetry groups 
(or "space groups") can arise as symmetries of triply 
periodic minimal 
surfaces, and use this to help classify these rascals.  A cool mixture 
of group theory and differential geometry!  I don't get the impression 
that they have completed the classification, though.  

Anyway, Schoen's "gyroid" is one of these triply periodic minimal 
surfaces.  Schoen discovered it before the computer revolution kicked 
in.  He was working for NASA, and his idea was to use it for building
ultra-light, super-strong structures:

13) A. H. Schoen, Infinite periodic minimal surfaces without 
selfintersections, NASA Tech. Note No. D-5541, Washington, DC, 1970.

You can learn more about the gyroid here:

14) Eric Weisstein, Gyroid, from Mathworld,
<A HREF = "http://mathworld.wolfram.com/Gyroid.html">http://mathworld.wolfram.com/Gyroid.html</A>

Apparently it's the only triply periodic non-self-intersecting 
minimal surface with "triple junctions".  I'm not quite 
sure what that means mathematically, but I can see them in the picture!

I said that soap films weren't good at creating \emph{complete} minimal
surfaces.  But actually, people have seen at least approximate gyroids
in nature, made from soap-like films:

15) P. Garstecki and R. Holyst, Scattering patterns of self-assembled 
gyroid cubic phases in amphiphilic systems, J. Chem. Phys. 115 (2001),
1095-1099.

An "amphiphilic" molecule is one that's attracted by water at one 
end and repelled by water at the other.  For example, the stuff in soap.
Mixed with water and oil, such 
molecules form membranes, and really complicated patterns can emerge,
verging on the biological.  Sometimes the membranes make a gyroid
pattern, with oil on one side and water on other!  It's a great example
of how any sufficiently beautiful mathematical pattern tends to show up
in nature somewhere... as Plato hinted in his theory of "forms".

People have fun simulating these "ternary amphiphilic fluids" 
on computers:

16) Nelido Gonzalez-Segredo and Peter V. Coveney, Coarsening dynamics of 
ternary amphiphilic fluids and the self-assembly of the gyroid and 
sponge mesophases: lattice-Boltzmann simulations, available as 
<A HREF = "http://arxiv.org/abs/cond-mat/0311002">cond-mat/0311002</A>.

17) Pittsburgh Supercomputing Center, Ketchup on the grid with joysticks,
<A HREF = "http://www.psc.edu/science/2004/teragyroid/">http://www.psc.edu/science/2004/teragyroid/</A>

The second site above describes the "TeraGyroid Project", in which
people used 17 teraflops of computing power at 6 different facilities 
to simulate the gyroidal phase of oil/water/amphiphile mixtures and 
study how "defects" move around in what's otherwise a regular 
pattern.  
The reference to ketchup comes from some supposed relationship between 
these ternary amphiphilic fluids and how ketchup gets stuck in
the bottle.  I'm not sure ketchup actually \emph{is} a ternary amphiphilic
fluid, though!

Hmm.  I just noticed a pattern to the websites I've been referring
to: first one about a "Milky Way bar", then one about 
a "GRAPE", and
now one about ketchup!  I think it's time to cook that dinner.



\par\noindent\rule{\textwidth}{0.4pt}
<em>
Daydreaming admiring being <br>
Quietly, open the world <br>
I hear the time of the starry sky <br>
Turning over at midnight</em> - Massive Attack

\par\noindent\rule{\textwidth}{0.4pt}

% </A>
% </A>
% </A>
