
% </A>
% </A>
% </A>
\week{July 12, 1995}

A few weeks ago I went to the IVth Porto Meeting on Knot Theory and
Physics, to which I had been kindly invited by Jose Mourao.  Quite 
a few of the (rather few) believers in the relevance of n-categories to physics
were there.  I spoke on higher-dimensional algebra and topological quantum
field theory, and also a bit on spin networks.  Louis Crane spoke on his ideas, 
especially the idea of getting 4-dimensional TQFTs out of state-sum
models.  And John Barrett spoke on 

1)  John Barrett, Quantum gravity as topological quantum field theory, to 
appear in the November 1995 special issue of Jour. Math. Physics, also 
available as <A HREF = "http://xxx.lanl.gov/abs/gr-qc/9506070">gr-qc/9506070</A>.

This is a nice introduction to the concepts of topological quantum field
theory (TQFT) that doesn't get bogged down in the (still substantial) 
technicalities.  In particular, it pays more emphasis than usual to the physical
interpretation of TQFTs, and how this meshes with more traditional issues
in the interpretation of quantum mechanics.  One of the main things
I got out of the conference, in fact, was a sense that there is a budding field
along these lines, just crying out to be developed.  As Barrett notes, Atiyah's
axioms for a TQFT can really be seen as coming from combining 

a) The rules of quantum mechanics for composing amplitudes

and 

b) Functoriality, or the correct behavior under diffeomorphisms of manifolds.

Indeed, he convincingly recovers the TQFT axioms from these two principles.
And of course these two principles could be roughly called "basic quantum
mechanics" and "general covariance"... lending credence to the idea that 
whatever the theory of quantum gravity turns out to be, it should be 
something closely related to a TQFT.  (I should emphasize, though, 
that this question is one of the big puzzles in the subject.) 

The richness inherent in b) makes the business of erecting a formalism
to interpret topological quantum field theory much more interesting than the
(by now) rather stale discussions that only treat a), or "basic quantum 
mechanics".  In particular, in a TQFT, every way of combining 
manifolds - spaces or spacetimes - yields a corresponding rule for 
composing amplitudes.  For example, if we have two spacetimes that look 
like


\begin{verbatim}

 O
| |
| |
| |
| |
 O
\end{verbatim}
    

(that's supposed to look like a pipe!) and 


\begin{verbatim}

 O       O
 \ \   / /
  \ \ / /
   \   /
    | |
     O
\end{verbatim}
    

- that is, a cylinder and a "trinion" (or upside-down pair of pants) - 
we can combine them either "horizontally" like this:


\begin{verbatim}

 O       O      O
 \ \   / /     | |
  \ \ / /      | |
   \   /       | |
    | |        | | 
     O          O
\end{verbatim}
    

or "vertically" like this:


\begin{verbatim}

 O       O
 \ \   / /
  \ \ / /
   \   /
    | |
    | |
    | |
    | |
    | |
     O
\end{verbatim}
    

Corresponding to each spacetime we have a "time evolution operator" -
a linear operator that describes how states going in one end pop out the
other, "evolved in time".  And corresponding to horizontal and vertical 
composition of spacetimes we have two ways to compose operators: 
horizontal composition usually being called "tensor product", and 
vertical composition being called simply "composition".  These two ways 
satisfy some compatibility conditions, as well.

Now if one has read a bit about n-categories and/or "extended" topological
quantum field theories, one already knows that this is just the 
tip of the iceberg.  If we allow ourselves to cut spacetimes into 
smaller bits - e.g., pieces with "corners", such as tetrahedra or their 
higher-dimensional kin - one gets more
possible ways of composing operators, and more compatibility conditions.  These
become algebraically rather sophisticated, but luckily, there is a huge 
amount of evidence that existing TQFTs extend to more 
sophisticated structures of this sort, through a miraculous 
harmony between algebra and topology.  

This leads to some interesting new concepts when it comes to the physical
interpretation of extended TQFTs.   As Crane described in his talk
(see also his papers listed in "<A HREF = "week2.html">week2</A>", "<A HREF = "week23.html">week23</A>" and "<A HREF = "week56.html">week56</A>"), in a 
4-dimensional extended TQFT one expects the following sort of thing.  
If we think of an "observer" as a 3-manifold with boundary - 
imagine a person being the 3-manifold and his skin being the 
boundary, if one likes - the extended TQFT should assign to his boundary a 
"Hilbert category" or "2-Hilbert space".   This is the categorical analog of a
Hilbert space.  In other words, just as a Hilbert space is a \emph{set} 
in which 
you can \emph{sum} things and \emph{multiply} them 
by \emph{complex numbers}, and 
get \emph{complex numbers} by taking \emph{inner products} of things, a 
2-Hilbert space is an analogous structure in which every term 
surrounded by asterisks is replaced by its analog
one step up the categorical ladder.  This means:


\begin{verbatim}

set             \to  category
sum             \to  direct sum
multiply        \to  tensor
complex numbers \to  vector spaces
inner products  \to  homs
\end{verbatim}
    

There's a good chance that you know the analogy between numbers and vector
spaces: just as you can add numbers and multiply them, you can take direct sums
and tensor products of vector spaces, and many of the same rules still 
apply (in a somewhat more sophisticated form, because laws that were 
equations are now isomorphisms).  A little less familiar is the 
analogy between inner products and "homs".  Given two vectors v and w in a 
Hilbert space you can take the inner product <v,w> and get a number; 
similarly, given two (finite-dimensional) Hilbert 
spaces V and W you can form hom(V,W) - that is, the set of all linear 
maps from V to W - and get a Hilbert space.  The same thing works in any 
"2-Hilbert space".  

The most basic example of a 2-Hilbert space would be Hilb, the category of 
finite-dimensional Hilbert spaces, but also Reps(G), the category of finite-
dimensional unitary representations of a finite group.  (Similar remarks hold 
for quantum groups at root of unity.)  Just as the inner product is linear 
in one argument and conjugate-linear in the other, "hom" behaves nicely 
under direct sums in each argument, but each argument behaves a bit 
differently under tensor product, so one can say it's "linear" in 
one and "conjugate-linear" in the other.  
      
So anyway, just as in a 4d TQFT a 3-manifold M determines a Hilbert space Z(M),
and a 4-manifold N with boundary equal to M determines a vector Z(N) in 
Z(M), something similar happens in an extended TQFT.  (For experts, here
I'm really talking about "unitary" TQFTs and extended TQFTs - these are
the physically sensible ones.)   Namely, a "skin of observation" or 2-manifold
S determines a 2-Hilbert space Z(S), and an "observer" or 3-manifold M with
boundary equal to S determines an object in Z(S).  Now, given two observers
M and M' with the same "skin" - for example, the observer "you" and the 
observer "everything in the world except you" - one gets two 
objects Z(M) and Z(M') in Z(S), so one can form the "inner product" 
hom(Z(M),Z(M')), which is a Hilbert space.  This is \emph{your} Hilbert 
space for describing states of \emph{everything in the world except you}.
Note that we are using the term "observer" here in a 
somewhat whimsical sense; in particular, every region of space counts as an 
observer in this game, so we can flip things around and form the inner product 
hom(Z(M'),Z(M)), which is the Hilbert space that <em>everything in the 


% parser failed at source line 219
