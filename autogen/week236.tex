
% </A>
% </A>
% </A>
\week{July 26, 2006 }

 
This week I'd like to catch you up on some papers about 
categorification and quantum mechanics.  

But first, since it's summer vacation, I'd like to take you on
a little road trip - to infinity.  And then, for fun, a little
detective story about the history of the icosahedron.

Cantor invented two kinds of infinities: cardinals and ordinals.  
Cardinals are more familiar.  They say how big sets are.  Two sets 
can be put into 1-1 correspondence iff they have the same number of 
elements - where this kind of "number" is a cardinal.  

But today I want to talk about ordinals.  Ordinals say how big 
"well-ordered" sets are.  A set is well-ordered if it's linearly 
ordered and every nonempty subset has a smallest element.  

For example, the empty set

{}

is well-ordered in a trivial sort of way, and the corresponding 
ordinal is called 

0.

Similarly, any set with just one element, like this:

{0}

is well-ordered in a trivial sort of way, and the corresponding 
ordinal is called

1.

Similarly, any set with two elements, like this:

{0,1}

becomes well-ordered as soon as we decree which element is bigger; 
the obvious choice is to say 0 < 1.  The corresponding ordinal is 
called 

2.  

Similarly, any set with three elements, like this:

{0,1,2}

becomes well-ordered as soon as we linearly order it; the obvious
choice here is to say 0 < 1 < 2.  The corresponding ordinal is called

3.  

Perhaps you're getting the pattern - you've probably seen these 
particular ordinals before, maybe sometime in grade school.  
They're called finite ordinals, or "natural numbers".

But there's a cute trick they probably didn't teach you then: we can
\emph{define} each ordinal to \emph{be} the set of all ordinals
less than it:


\begin{verbatim}

0 = {}            (since no ordinal is less than 0) 
1 = {0}           (since only 0 is less than 1)  
2 = {0,1}         (since 0 and 1 are less than 2)
3 = {0,1,2}       (since 0, 1 and 2 are less than 3)
\end{verbatim}
    
and so on.   It's nice because now each ordinal \emph{is} a
well-ordered set of the size that ordinal stands for.
And, we can define one ordinal to be "less than or equal" to
another precisely when its a subset of the other.

Now, what comes after all the finite ordinals?  Well,
the set of all finite ordinals is itself well-ordered:

{0,1,2,3,...}

So, there's an ordinal corresponding to this - and it's the first
\emph{infinite} ordinal.  It's usually called omega (\omega ).
Using the cute trick I mentioned, we can actually define

\omega  = {0,1,2,3,...}

Now, what comes after this?  Well, it turns out there's a
well-ordered set

{0,1,2,3,...,\omega }

containing the finite ordinals together with \omega , with the
obvious notion of "less than": \omega  is bigger than the rest.  
Corresponding to this set there's an ordinal called

\omega +1

As usual, we can simply define

\omega +1 = {0,1,2,3,...,\omega }

(At this point you could be confused if you know about cardinals,
so let me throw in a word of reassurance.  The sets \omega  and
\omega +1 have the same "cardinality", but they're different as 
ordinals, since you can't find a 1-1 and onto function between 
them that \emph{preserves the ordering}.  This is easy to see, since
\omega +1 has a biggest element while \omega  does not.)

Now, what comes next?  Well, not surprisingly, it's

\omega +2 = {0,1,2,3,...,\omega ,\omega +1}
 
Then comes 

\omega +3, \omega +4, \omega +5,...

and so on.  You get the idea.

What next?  

Well, the ordinal after all these is called \omega +\omega .  People
often call it "\omega  times 2" or "\omega 2" for
short.  So,

\omega 2 = {0,1,2,3,...,\omega ,\omega +1,\omega +2,\omega +3,....}

What next?  Well, then comes 

\omega 2 + 1, \omega 2 + 2,...

and so on.  But you probably have the hang of this already, so 
we can skip right ahead to \omega 3.

In fact, you're probably ready to skip right ahead to \omega 4,
and \omega 5, and so on.

In fact, I bet now you're ready to skip all the way to "\omega 
times \omega ", or \omega ^{2} for short:

\omega ^{2} = 
{0,1,2...\omega ,\omega +1,\omega +2,...,\omega 2,\omega 2+1,\omega 2+2,...}

It would be fun to have a book with \omega  pages, each page half 
as thick as the previous page.  You can tell a nice long story 
with an \omega -sized book.  But it would be even more fun to have 
an encyclopedia with \omega  volumes, each being an \omega -sized book,
each half as thick as the previous volume.  Then you have \omega ^{2} 
pages - and it can still fit in one bookshelf!

<DIV ALIGN = CENTER>
<IMG SRC = "omega_squared.png">
</DIV>

What comes next?  Well, we have 

\omega ^{2}+1, \omega ^{2}+2, ...

and so on, and after all these come 

\omega ^{2}+\omega , \omega ^{2}+\omega +1, \omega ^{2}+\omega +2, ...

and so on - and eventually 

\omega ^{2} + \omega ^{2} = \omega ^{2} 2

and then a bunch more, and then

\omega ^{2} 3

and then a bunch more, and then

\omega ^{2} 4
 
and then a bunch more, and more, and eventually

\omega ^{2} \omega  = \omega ^{3}.

You can probably imagine a bookcase containing \omega  encyclopedias,
each with \omega  volumes, each with \omega  pages, for a total of
\omega ^{3} pages.

I'm skipping more and more steps to keep you from getting bored. 
I know you have plenty to do and can't spend an \emph{infinite} amount 
of time reading This Week's Finds, even if the subject is infinity.  

So, if you don't mind me just mentioning some of the high points,
there are guys like \omega ^{4} and \omega ^{5} and so
on, and after all these comes

\omega ^{\omega }.

And then what?

Well, then comes \omega ^{\omega } + 1, and so on, but I'm sure
that's boring by now.  And then come ordinals like 

\omega ^{\omega } 2,..., \omega ^{\omega } 3, ..., \omega ^{\omega } 4, ...

leading up to 

\omega ^{\omega } \omega  = \omega ^{\omega  + 1}

Then eventually come ordinals like

\omega ^{\omega } \omega ^{2} , ..., \omega ^{\omega } \omega ^{3}, ..., \omega ^{\omega } \omega ^{4}, ...

and so on, leading up to:

\omega ^{\omega } \omega ^{\omega } = \omega ^{\omega  + \omega } = \omega ^{\omega  2}

This actually reminds me of something that happened driving across 
South Dakota one summer with a friend of mine.  We were in college,
so we had the summer off, so we drive across the country.  We drove
across South Dakota all the way from the eastern border to the west
on Interstate 90.  

This state is huge - about 600 kilometers across, and most of it is 
really flat, so the drive was really boring.  We kept seeing signs 
for a bunch of tourist attractions on the western edge of the state, 
like the Badlands and Mt. Rushmore - a mountain that they carved 
to look like faces of presidents, just to give people some reason to keep
driving.  

Anyway, I'll tell you the rest of the story later - I see some more 
ordinals coming up:

\omega ^{\omega 3},... \omega ^{\omega 4},... \omega ^{\omega 5},...

We're really whizzing along now just to keep from getting bored - just
like my friend and I did in South Dakota.  You might fondly imagine
that we had fun trading stories and jokes, like they do in road
movies.  But we were driving all the way from Princeton to my friend
Chip's cabin in California.  By the time we got to South Dakota, we
were all out of stories and jokes.

Hey, look!  It's

\omega ^{\omega  \omega } = \omega ^{\omega <sup>2}</sup>

That was cool.  Then comes 



% parser failed at source line 349
