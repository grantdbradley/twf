
% </A>
% </A>
% </A>
\week{December 19, 2009 }

This week: a fascinating history of categorical logic, and more about
rational homotopy theory.  But first, guess what this is a picture of:

<div align = "center">
<img border = "2" src = "014379_0925.jpg">
</div>

If you give up, go to the bottom of this article.

Next, here's an incredibly readable introduction to the revolution
that happened in logic starting in the 1960s:

1) Jean-Pierre Marquis and Gonzalo Reyes, The history of categorical
logic, 1963-1977.  Available at
<a href = "https://www.webdepot.umontreal.ca/Usagers/marquisj/MonDepotPublic/HistofCatLog.pdf">https://www.webdepot.umontreal.ca/Usagers/marquisj/MonDepotPublic/HistofCatLog.pdf</a>

It's a meaty but still bite-sized 116 pages.  It starts with the
definitions of categories, functors, and adjoint functors.  But it
really takes off in 1963 with Bill Lawvere's thesis, which
revolutionized universal algebra using category theory.  It then moves
on through Lawvere and Tierney's introduction of the modern concept of
topos, and it ends in 1977, when Makkai and Reyes published their book
on categorical logic, and Johnstone published his book on topos theory.
The world has never been the same since!

One great thing about this paper is that it discusses the history in a
blow-by-blow way, including conferences and unpublished but
influential writings.  It also gives a great summary of the key ideas
in Lawvere's thesis.  I'll quote it, since everyone should know or at
least <i>have seen</i> these ideas:



% parser failed at source line 112
