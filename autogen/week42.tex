
% </A>
% </A>
% </A>
\week{November 3, 1994}

String theory means different things to different people.  The original
theory of strings -- at least if I've got my history right -- was a
theory of hadrons (particles interacting via the strong force).  The
strong force wasn't understood too well then, but in 1968 Veneziano cleverly
noticed when thumbing through a math book that Euler's beta function 
had a lot of the properties one would expect of the formula for how
hadrons scattered (the so-called S-matrix).  Later, around 1970, Nambu
and Goto noticed that this function would come out naturally if one
thought of hadrons as different vibrational modes of a relativistic
string.  

This theory had problems, and eventually it was supplanted by the
current theory of the strong force, involving quarks and gluons.  The
gluons are another way of talking about the strong force, which is a
gauge field.  The biggest puzzle about this approach to hadrons is, "how
come we don't see quarks?"  This is called the puzzle of confinement.
In the late 1970's, one proposed solution was that as you pulled the
quark and the antiquark in a meson apart, the strong force effectively
formed an elastic "string" with constant tension.  This would mean that
pulling them apart took energy proportional to how far you pulled them
apart.  Past a certain point, the energy would be enough to create a new
quark-antiquark pair and \emph{snap} - the string would split into two new
strings with quark and antiquark on each end.  So here the "string" idea
is revived but as an approximation to a theory of gauge fields.  One can
even try to derive approximate string equations from the equations for
the strong force: the Yang-Mills equations.  In my paper on strings,
loops, knots and gauge fields (see <A HREF = "week18.html">week18</A>), I gave references to some
early papers on the subject:

1) QCD and the string model, by Y. Nambu, Phys. Lett. B80 (1979) 372-376.  

Gauge fields as rings of glue, A. Polyakov, Nucl. Phys. B164 (1979)
171-188.

The quantum dual string wave functional in Yang-Mills theories, by
J. Gervais and A. Neveu, Phys. Lett. B80 (1979), 255-258.

The interaction among dual strings as a manifestation of the gauge
group, by F. Gliozzi and M. Virasoro, Nucl. Phys. B164 (1980), 141-151. 

Loop-space representation and the large-N behavior of the one-plaquette
Kogut-Susskind Hamiltonian, A. Jevicki, Phys. Rev. D22 (1980), 467-471. 

Quantum chromodynamics as dynamics of loops, by Y. Makeenko and A.
Migdal, Nucl. Phys. B188 (1981) 269-316. 

Loop dynamics: asymptotic freedom and quark confinement, by Y. Makeenko
and A. Migdal, Sov. J. Nucl. Phys. 33 (1981) 882-893.

These papers make very interesting reading even today.  Anyone who knows
particle physics will recognize most of these names!  Strings were big
back then.  But then they went out of fashion, because the string models
predicted a massless spin-2 particle --- and there's no such thing in
particle physics.  Later, when people were trying to cook up "theories
of everything" including gravity, this flaw was again seen as a plus,
since the hypothesized "graviton" meets that description.

The modern, more technical subject of string theory is a lot more fancy
than these early papers.  In particular, the recognition that conformal
invariance was a very good thing when studying strings propagating on
fixed background metric (like that of Minkowski space) pushed string
theorists into a careful study of 2-dimensional conformal invariant
quantum field theories.  (Here the 2 dimensions refer to the surface the
string traces out as it moves through spacetime.)  Conformal field
theory then developed a life of its own!  By now it's pretty
intimidating to the outsider.  Mathematicians might find the following
summary handy:

2) Conformal field theory, by Krzysztof Gawedzki, Seminaire Bourbaki,
Asterisque 177-178 (1989), pp. 95-126.

while physicists might try 

3) Introduction to Superstrings, by Michio Kaku, New York, Springer-Verlag,
1988.

String Fields, Conformal Fields, and Topology, by Michio Kaku, New York,
Springer-Verlag, 1991.

Kaku's books are a decent overview but rather sketchy in spots, since
they cover vast amounts of territory.

Then there is another kind of sophisticated modern string theory,
"string field theory", which doesn't assume the strings are moving
around on a spacetime with a background geometry.   This is clearly more
like what one wants to do if one is using strings to explain quantum
gravity.  I don't understand this nearly as well as I'd like to, but the
guru on this subject is Barton Zwiebach, so if one was really gutsy one
would, after a suitable warmup with Kaku, plunge in and read something like

4)  Quantum background independence of closed string field theory, 
by Ashoke Sen and Barton Zwiebach, 60 pages, phyzzx.tex, MIT-CTP-2244,
available as <A HREF = "http://xxx.lanl.gov/abs/hep-th/9311009">hep-th/9311009</A>.

Background independent algebraic structures in closed string field theory,
by Ashoke Sen and Barton Zwiebach, phyzzx.tex, MIT-CTP-2346, available
as <A HREF = "http://xxx.lanl.gov/abs/hep-th/9408053">hep-th/9408053</A>.  

Unfortunately I'm not quite up to it yet....

Then, in a different direction, a bunch of folks from general relativity
pursued some ideas about string and loops to the point of developing the
"loop representation of quantum gravity."  I'm referring to

5) Loop representation for quantum general relativity, by C. Rovelli
and L. Smolin, Nucl. Phys. B331 (1990), 80-152.

though it's important to credit some of the people who kept alive the idea
that one should study gauge fields as being "loops of string", or more
technically, "Wilson loops":

6) Gauge dynamics in the C-representation, by R. Gambini and A. Trias,
Nucl. Phys. B278 (1986) 436-448. 

Now what's frustrating here is that I understand the loop representation
business, but not the "background-free closed string field theory"
business, even though they have the same historical roots and are both
trying to deal with quantum gravity (among other forces) in a way that
assumes that loops are the basic objects.  Alas, the two strands speak
in different languages!  Heavy-duty mathematicians like Getzler,
Kapranov and Stasheff know how to think about closed string fields in
terms of "operads", and that stuff seems like it should be simple enough
to understand, but alas, when I read it I get snowed in detail (so far).

Let me digress to mention what an "operad" is.  An "operad" is basically
a cool way to handle sets equipped with lots of n-ary operations.  These
operations might be "parametrized" in various ways.  The operad
elegantly keeps track of these parametrizations.  So, for each n, an
operad has a set X(n) which we think of as all the n-ary operations.
Think of something in X(n) as a black box that has n "input" tubes and
one "output" tube, or a tree-shaped thing


\begin{verbatim}

       \  |  /
        \ | /
         \|/
          |
          |
          |
\end{verbatim}
    

with n branches and one root (here n = 3).  Then suppose we have a bunch
of these black boxes.  Say we have something in X(n1), something in
X(n2), .... and so on up to something in X(nk).  Thus we've got a pile
of black boxes with a total of n1 + ... + nk input tubes and k output
tubes.  Now if we \emph{also} have a guy in X(k), which has k input tubes, we
can hook up all the output tubes of all the boxes in our pile to the
input tubes of this guy, to get a monstrous machine with n1 + ... + nk
input tubes and one output.   In short, there is an operation from
X(n1) x ... x X(nk) x X(k) to X(n1 + ... + nk).  For example, if we take
the tree up there, which represents something in X(3), and another
thing in X(3), we can hook up their outputs to the inputs of something
in X(2), to get something that looks like


\begin{verbatim}

       \  |  /   \  |  /
        \ | /     \ | /
         \|/       \|/
          |         |
          |         |
          |         |
          \         /
           \       /
            \     / 
             \   /
              \ /
               |
               |
               |
\end{verbatim}
    

which is in X(6).  The closed string field theorists like operads
because there are lots of parametrized ways of gluing together Riemann
surfaces with punctures together.  It's a handy language, apparently...
I am a bit more familiar with operads (though not much) in the context
of homotopy theory, where they can be used to elegantly summarize the
operations one has floating around in an infinite loop space.  \emph{Very}
roughly, an infinite loop space is a space that looks like the space of 
loops of loops of loops of loops... of loops in some topological space,
where you get to make the "dot dot dot" part go on as long as you want!
A beautifully unpretentious and utterly readable book on these spaces,
operads, and much much more, is:

7) Infinite Loop Spaces, by J. F. Adams, Princeton U. Press, Princeton,
NJ, 1978.  

Lest "infinite loop spaces" seem abstruse, I should emphasize that the
book is really a nice tour of a lot of modern homotopy theory.  As he
says, "my object has been a more elementary exposition, which I hope may
convey the basic ideas of the the subject in a way as nearly painless as
I can make it.  In this the Princeton audience encouraged me; the more I
found means to omit the technical details, the more they seemed to like
it."  A lot of the general mathematical machinery he discusses,
especially in the chapter called "Machinery", is really too nice to be
left for only the homotopy theorists!  

Anyway, once you have gotten the hang of operads you can try the work of
a reformed homotopy theorist, Jim Stasheff, on string field theory:

8) Closed string field theory, strong homotopy Lie algebras and the
operad actions of moduli spaces, by Jim Stasheff, available as
<A HREF = "http://xxx.lanl.gov/abs/hep-th/9304061">hep-th/9304061</A>.  

Actually Graeme Segal, another string theory guru, also used to do
homotopy theory.  He's the one who's famous for:

9) Loop groups, by Andrew Pressley and Graeme Segal, Oxford University
Press, Oxford, 1986. 

So it's possible that these guys didn't really quit homotopy theory,
but just figured out how to get physicists interested in it.  Notice all
those loops!  :-)

But where was I... romping through various approaches to string theory,
taking a detour to mention loops, but all the while sneaking up on my
goal, which is to list a few papers that lend evidence to the thesis of
my paper Strings, Loops, Knots and Gauge Fields, namely that a profound
"string/gauge field duality" is at work in many physical models, and
that the loop representation of quantum gravity, and string theory, may
eventually not be seen as so different after all.

Let's see what we've got here:

10) A reformulation of the Ponzano-Regge quantum gravity model in terms of
surfaces, Junichi Iwasaki, University of Pittsburgh, 11 pages in LaTeX format
available as <A HREF = "http://xxx.lanl.gov/abs/gr-qc/9410010">gr-qc/9410010</A>.  

I've discussed the Ponzano-Regge model quite a bit in <A HREF = "week16.html">week16</A> and <A HREF = "week38.html">week38</A>.
It's an approach to quantum gravity that is especially successful in 3
dimensions, and involves chopping spacetime up into simplices.  The
exact partition function, as they say, can be computed using this
combinatorial discrete approximation to the spacetime manifold.  (In
quantum field theory, when you know enough about the partition function
you can compute the expectation values of observables to your heart's
content.)  Anyway, here Iwasaki does the kind of thing I was pointing
towards in my paper, namely, to rewrite the theory, which starts out as
a gauge theory, as a theory of surfaces ("string worldsheets") in
spacetime.  
 
Meanwhile, more work has been done on the same kind of idea for 
good old quantum chromodynamics, though here there \emph{is} a background
geometry, and one approximates the spacetime manifold by a discrete
lattice not because one expects to get the \emph{exact} answers out that way,
but just because it's a decent approximation that makes things a bit
more manageable: 

11) Lattice QCD as a theory of interacting surfaces, by B. Rusakov,
TAUP-2204-94, 12 pages in LaTeX format available as <A HREF = "http://xxx.lanl.gov/abs/hep-th/9410004">hep-th/9410004</A>.

U(N) Gauge Theory and Lattice Strings, by Ivan K. Kostov, 26 pages, 8
figures not included, available by mail upon request, T93-079 (talk at
the Workshop on string theory, gauge theory and quantum gravity, 28-29
April 1993, Trieste, Italy), available as <A HREF = "http://xxx.lanl.gov/abs/hep-th/9308158">hep-th/9308158</A>.  

Also, if there were any gauge theory that deserved to be a string
theory, it's probably Chern-Simons theory, which has so much to do with
knots... and indeed something like this seems to be the case, though
it's all rather subtle and mysterious so far:

12) Chern-Simons-Witten theory as a topological Fermi liquid, by Michael
R. Douglas, Rutgers University preprint RU-94-29, available as
<A HREF = "http://xxx.lanl.gov/abs/hep-th/9403119">hep-th/9403119</A>. 

Frequently, when there is a whole lot of frenetic, sophisticated-sounding
activity around a certain idea, like this relation between strings and
gauge fields, there is a simple truth yearning to be known.  Sometimes
it takes a while!  We'll see.
\par\noindent\rule{\textwidth}{0.4pt}

% </A>
% </A>
% </A>
