
% </A>
% </A>
% </A>
\week{July 25, 2004}

I'm spending the summer in Cambridge, but last week I was in Dublin 
attending "GR17", which is short for the 17th International Conference 
on General Relativity and Gravitation:

1) GR17 homepage, <A HREF = "http://www.dcu.ie/~nolanb/gr17.htm">
http://www.dcu.ie/~nolanb/gr17.htm</A>

This is where Stephen Hawking decided to announce his solution of
the black hole information loss problem.  Hawking is a media superstar 
right up there with Einstein and Michael Jackson, so when reporters 
heard about this, the ensuing hoopla overshadowed everything else in 
the conference.  

As soon I arrived, one of the organizers complained to me that they'd
had to spend 4000 pounds on a public relations firm to control the 
reporters and other riff-raff who would try to attend Hawking's talk.
Indeed, there seemed to be more than the usual number of crackpots 
floating about, though I admit I haven't been to this particular series 
of conferences before - perhaps general relativity attracts such people?  
The public lecture by Penrose on the last day of the conference may have 
helped lure them in.   He spoke on "Fashion, Faith and Fantasy in 
Theoretical Physics", and people by the door sold copies of his brand 
new thousand-page blockbuster:

2) Roger Penrose, The Road To Reality: A Complete Guide to the 
Physical Universe, Jonathan Cape, 2004.

(You may enjoy guessing which popular theories he classified under
the three categories of fashion, faith and fantasy.)  After his talk, 
\emph{all} the questions were actually harangues from people propounding 
idiosyncratic theories of their own, and the question period was drawn 
to an abrupt halt in the middle of one woman's rant about fractal 
cosmology.  But I bumped into the saddest example when I was having a 
chat with some colleagues at a local pub.  A fellow with long curly grey 
locks and round horn-rimmed glasses sat down beside me.  I'd seen him 
around the conference, so I said hello.  He asked me if I'd like to hear 
about his theory; at this point my internal alarm bells started ringing.   
I told him I was busy, but said I'd take a look at his manuscript later.

It turned out to describe an idea I'd never even dreamt of before: 
a heliocentric cosmology in which the planets move along circular orbits 
with epicycles a la Ptolemy!  And his evidence comes from a neolithic 
Irish tomb called Newgrange.  This tomb may have been aligned to let 
in the sun on the winter solstice, but some people doubt this, because
it seems the alignment would have been slightly off back in 3200 BC when 
Newgrange was built.  However, it's slightly off only if you work out 
the precession of the equinox using standard astronomy.  If you use his 
theory, it lines up perfectly!   Pretty cute.  The only problem is that 
his paper contains no evidence for this claim.  Instead, it's only a 
short note sketching the idea, followed by lengthy attachments containing 
his correspondence with the Dublin police.  In these, he complained that 
people were trying to block his patent on a refrigerator that produces no 
waste heat.  They were constantly flying airplanes over his house, and 
playing pranks like boiling water in his teakettle when he was away, trying 
to drive him insane.   

Anyway, on Wednesday the 21st the whole situation built to a head when 
Hawking gave his talk in the grand concert hall of the Royal Dublin Society.  
As we had been warned, the PR firm checked our badges at the door.  Reporters
with press badges were also allowed in, so the aisles were soon lined with 
cameras and recording equipment.  I got there half an hour early to get a 
good seat, and while I was waiting, Jenny Hogan from the New Scientist 
asked if she could interview me for my reaction afterwards.   In short, 
a thoroughly atypical physics talk!  

But you shouldn't imagine the mood as one of breathless anticipation.  At 
least for the physicists present, a better description would be something 
like "skeptical curiosity".  None of them seemed to believe that Hawking 
could suddenly shed new light on a problem that has been attacked from 
many angles for several decades.  One reason is that Hawking's best work 
was done almost 30 years ago.  A string theorist I know said that thanks 
to work relating anti-deSitter space and conformal field theory - the 
so-called "AdS-CFT" hypothesis - string theorists had become convinced 
that no information is lost by black holes.   Thus, Hawking had been 
feeling strong pressure to fall in line and renounce his previous position, 
namely that information \emph{is} lost.  A talk announcing this would come 
as no big surprise.  

After a while Kip Thorne, John Preskill, Petros Florides and Hawking's
grad student Christophe Galfard came on stage.  Then, amid a burst of 
flashbulbs, Hawking's wheelchair gradually made its way down the aisle 
and up a ramp, attended by a nurse - possibly his wife, I don't know.  
He had been recently sick with pneumonia.

Once Hawking was on stage, the conference organizer Petros Florides made 
an introduction, joking that while physicists believe no information can
travel faster than light, this seems to have been contradicted by the 
speed with which the announcement of Hawking's talk spread around the globe.  
Then he recalled the famous bet that Preskill made with Hawking and
Thorne.  In case you don't know, John Preskill is a leader in quantum
computation at Caltech.   Kip Thorne is an expert on relativity, also 
at Caltech, one of the authors of the famous textbook "Gravitation", 
and now playing a key role in the LIGO project to detect gravitational 
waves.  

The <A HREF = "dublin/index.html#hawking">bet</A> went like this:

\begin{quote}
     Whereas Stephen Hawking and Kip Thorne firmly believe that 
     information swallowed by a black hole is forever hidden from 
     the outside universe, and can never be revealed even as the 
     black hole evaporates and completely disappears,

     And whereas John Preskill firmly believes that a mechanism 
     for the information to be released by the evaporating black 
     hole must and will be found in the correct theory of quantum 
     gravity,

     Therefore Preskill offers, and Hawking/Thorne accept, a wager that:

     When an initial pure quantum state undergoes gravitational collapse 
     to form a black hole, the final state at the end of black hole 
     evaporation will always be a pure quantum state.

     The loser(s) will reward the winner(s) with an encyclopedia of the 
     winner's choice, from which information can be recovered at will.

     Stephen W. Hawking, Kip S. Thorne, John P. Preskill<br>
     Pasadena, California, 6 February 1997 
\end{quote}

It's signed by Thorne and Preskill, with a thumbprint of Hawking's.

After a bit of joking around and an explanation of how the question
session would work, Hawking began his talk.  Since it's fairly short 
and not too easy to summarize, I think I'll just quote the whole 
transcript which I believe Sean Carroll got from the New York Times 
science reporter Dennis Overbye.  I've made a few small corrections.  

There were also some slides, but you're not missing a lot by not seeing
them.  The talk was not easy to understand, so unless quantum gravity is
your specialty you may feel like just skimming it to get the flavor, and
then reading my attempt at a summary.  

The talk began with Hawking's trademark introduction, uttered as usual 
in his computer-generated voice:

\begin{quote}
     Can you hear me?

     I want to report that I think I have solved a major problem in
     theoretical physics, that has been around since I discovered that
     black holes radiate thermally, thirty years ago. The question is, 
     is information lost in black hole evaporation?  If it is, the 
     evolution is not unitary, and pure quantum states, decay into 
     mixed states.

     I'm grateful to my graduate student Christophe Galfard for help in
     preparing this talk.

     The black hole information paradox started in 1967, when Werner
     Israel showed that the Schwarzschild metric, was the only static
     vacuum black hole solution. This was then generalized to the no hair
     theorem: the only stationary rotating black hole solutions of the
     Einstein-Maxwell equations are the Kerr-Newman metrics. The no hair
     theorem implied that all information about the collapsing body was
     lost from the outside region apart from three conserved quantities:
     the mass, the angular momentum, and the electric charge.

     This loss of information wasn't a problem in the classical theory. A
     classical black hole would last for ever, and the information could 
     be thought of as preserved inside it, but just not very accessible. 
     However, the situation changed when I discovered that quantum effects 
     would cause a black hole to radiate at a steady rate.  At least in 
     the approximation I was using, the radiation from the black hole would 
     be completely thermal, and would carry no information.  So what would 
     happen to all that information locked inside a black hole, that 
     evaporated away, and disappeared completely?  It seemed the only way 
     the information could come out would be if the radiation was not exactly 
     thermal, but had subtle correlations.  No one has found a mechanism 
     to produce correlations, but most physicists believe one must exist.  
     If information were lost in black holes, pure quantum states would 
     decay into mixed states, and quantum gravity wouldn't be unitary.

     I first raised the question of information loss in '75, and the
     argument continued for years, without any resolution either
     way.  Finally, it was claimed that the issue was settled in favour 
     of conservation of information, by AdS/CFT. AdS/CFT is a conjectured
     duality between supergravity in anti-deSitter space and a conformal
     field theory on the boundary of anti-deSitter space at infinity. 
     Since the conformal field theory is manifestly unitary, the argument 
     is that supergravity must be information preserving.  Any information 
     that falls in a black hole in anti-deSitter space, must come out 
     again.  But it still wasn't clear how information could get out of 
     a black hole.  It is this question I will address.

     Black hole formation and evaporation can be thought of as a
     scattering process. One sends in particles and radiation from
     infinity, and measures what comes back out to infinity.  All
     measurements are made at infinity, where fields are weak, and one
     never probes the strong field region in the middle.  So one can't 
     be sure a black hole forms, no matter how certain it might be in
     classical theory.  I shall show that this possibility allows
     information to be preserved and to be returned to infinity.

     I adopt the Euclidean approach, the only sane way to do quantum
     gravity non-perturbatively.  [He grinned at this point.]   In 
     this, the time evolution of an initial state is given by a path 
     integral over all positive definite metrics that go between two 
     surfaces that are a distance T apart at infinity.  One then Wick 
     rotates the time interval, T, to the Lorentzian.

     The path integral is taken over metrics of all possible topologies
     that fit in between the surfaces.  There is the trivial topology: the
     initial surface cross the time interval.  Then there are the nontrivial 
     topologies: all the other possible topologies.  The trivial topology 
     can be foliated by a family of surfaces of constant time.  The
     path integral over all metrics with trivial topology, can be treated
     canonically by time slicing. In other words, the time evolution
     (including gravity) will be generated by a Hamiltonian.  This will 
     give a unitary mapping from the initial surface to the final.

     The nontrivial topologies cannot be foliated by a family of
     surfaces of constant time. There will be a fixed point in any time
     evolution vector field on a nontrivial topology.  A fixed point in 
     the Euclidean regime corresponds to a horizon in the Lorentzian. 
     A small change in the state on the initial surface would propagate as 
     a linear wave on the background of each metric in the path integral. 
     If the background contained a horizon, the wave would fall through it,
     and would decay exponentially at late time outside the horizon.  For
     example, correlation functions decay exponentially in black hole
     metrics.  This means the path integral over all topologically
     nontrivial metrics will be independent of the state on the initial
     surface.  It will not add to the amplitude to go from initial state to
     final that comes from the path integral over all topologically
     trivial metrics. So the mapping from initial to final states, given 
     by the path integral over all metrics, will be unitary.  

     One might question the use in this argument, of the concept of a 
     quantum state for the gravitational field on an initial or final 
     spacelike surface.  This would be a functional of the geometries of 
     spacelike surfaces, which is not something that can be measured in 
     weak fields near infinity.  One can measure the weak gravitational 
     fields on a timelike tube around the system, but the caps at top and 
     bottom, go through the interior of the system, where the fields may 
     be strong.

     One way of getting rid of the difficulties of caps would be to join
     the final surface back to the initial surface, and integrate over all
     spatial geometries of the join.  If this was an identification under a
     Lorentzian time interval, T, at infinity, it would introduce closed
     timelike curves. But if the interval at infinity is the Euclidean
     distance, beta, the path integral gives the partition function for
     gravity at temperature 1/\beta .

     The partition function of a system is the trace over all states,
     weighted with e^{-\beta  H}.  One can then integrate \beta  along
     a contour parallel to the imaginary axis with the factor e^{-\beta  E<subL0</sub>}.
     This projects out the states with energy E_{0}. In a gravitational 
     collapse and evaporation, one is interested in states of
     definite energy, rather than states of definite temperature.

     There is an infrared problem with this idea for asymptotically flat
     space.  The Euclidean path integral with period \beta  is the partition
     function for space at temperature 1/beta. The partition function 
     is infinite because the volume of space is infinite.  This infrared 
     problem can be solved by a small negative cosmological constant.  
     It will not affect the evaporation of a small black hole, but it will 
     change infinity to anti-deSitter space, and make the thermal partition 
     function finite.

     The boundary at infinity is then a torus, S^{1} cross S^{2}.  The trivial
     topology, periodically identified anti-deSitter space, fills in the 
     torus, but so also do nontrivial topologies, the best known of which 
     is Schwarzschild anti-deSitter.  Providing that the temperature is 
     small compared to the Hawking-Page temperature, the path integral 
     over all topologically trivial metrics represents self-gravitating 
     radiation in asymptotically anti-deSitter space.   The path integral 
     over all metrics of Schwarzschild AdS topology represents a black 
     hole and thermal radiation in asymptotically anti-deSitter.

     The boundary at infinity has topology S^{1} cross S^{2}.  The simplest
     topology that fits inside that boundary is the trivial topology, 
     S^{1} cross D^{3}, the three-disk. The next simplest topology, and 
     the first nontrivial topology, is S^{2} cross D^{2}.  
This is the 
     topology of the Schwarzschild anti-deSitter metric.  There are 
     other possible topologies that fit inside the boundary, but these 
     two are the important cases: topologically trivial metrics and 
     the black hole.  The black hole is eternal.  It cannot become 
     topologically trivial at late times.

     In view of this, one can understand why information is preserved in
     topologically trivial metrics, but exponentially decays in 
     topologically non trivial metrics.  A final state of empty space
     without a black hole would be topologically trivial, and be foliated
     by surfaces of constant time.  These would form a 3-cycle modulo
     the boundary at infinity.  Any global symmetry would lead to
     conserved global charges on that 3-cycle.  These would prevent
     correlation functions from decaying exponentially in topologically
     trivial metrics.  Indeed, one can regard the unitary Hamiltonian
     evolution of a topologically trivial metric as the conservation of
     information through a 3-cycle.

     On the other hand, a nontrivial topology, like a black hole, will 
     not have a final 3-cycle.  It will not therefore have any conserved
     quantity that will prevent correlation functions from exponentially
     decaying.  One is thus led to the remarkable result that late time
     amplitudes of the path integral over a topologically non trivial
     metric, are independent of the initial state. This was noticed by
     Maldacena in the case of asymptotically anti-deSitter in 3d, and
     interpreted as implying that information is lost in the BTZ black hole
     metric.  Maldacena was able to show that topologically trivial metrics
     have correlation functions that do not decay, and have amplitudes of
     the right order to be compatible with a unitary evolution.  Maldacena
     did not realize, however that it follows from a canonical treatment
     that the evolution of a topologically trivial metric, will be unitary.

     So in the end, everyone was right, in a way.  Information is lost 
     in topologically nontrivial metrics, like the eternal black hole. 
     On the other hand, information is preserved in topologically trivial
     metrics. The confusion and paradox arose because people thought
     classically, in terms of a single topology for spacetime.  It was
     either R^{4}, 
or a black hole.  But the Feynman sum over histories allows
     it to be both at once.  One can not tell which topology contributed the
     observation, any more than one can tell which slit the electron went
     through, in the two slits experiment.  All that observation at infinity
     can determine is that there is a unitary mapping from initial states
     to final, and that information is not lost.

     My work with Hartle showed the radiation could be thought of as
     tunnelling out from inside the black hole.  It was therefore not
     unreasonable to suppose that it could carry information out of the
     black hole.  This explains how a black hole can form, and then give
     out the information about what is inside it, while remaining
     topologically trivial.  There is no baby universe branching off, as 
     I once thought.  The information remains firmly in our universe. 
     I'm sorry to disappoint science fiction fans, but if information is
     preserved, there is no possibility of using black holes to travel to
     other universes. If you jump into a black hole, your mass-energy will
     be returned to our universe, but in a mangled form, which contains the
     information about what you were like, but in an unrecognisable state.

     There is a problem describing what happens, because strictly speaking
     the only observables in quantum gravity are the values of the field
     at infinity.  One cannot define the field at some point in the middle,
     because there is quantum uncertainty in where the measurement is
     done.  However, in cases in which there are a large number, N, of 
     light matter fields, coupled to gravity, one can neglect the 
     gravitational fluctuations, because they are only one among N 
     quantum loops. One can then do the path integral over all matter 
     fields, in a given metric, to obtain the effective action, which 
     will be a functional of the metric.

     One can add the classical Einstein-Hilbert action of the metric to
     this quantum effective action of the matter fields.  If one integrated
     this combined action over all metrics, one would obtain the full
     quantum theory. However, the semiclassical approximation is to
     represent the integral over metrics by its saddle point.  This will
     obey the Einstein equations, where the source is the expectation value
     of the energy momentum tensor, of the matter fields in their vacuum
     state.

     The only way to calculate the effective action of the matter fields,
     used to be perturbation theory. This is not likely to work in the case
     of gravitational collapse. However, fortunately we now have a
     non-perturbative method in AdS/CFT.  The Maldacena conjecture says
     that the effective action of a CFT on a background metric is equal to
     the supergravity effective action of anti-deSitter space with that
     background metric at infinity. In the large N limit, the supergravity
     effective action is just the classical action. Thus the calculation
     of the quantum effective action of the matter fields, is equivalent to
     solving the classical Einstein equations.

     The action of an anti-deSitter-like space with a boundary at
     infinity would be infinite, so one has to regularize.  One 
     introduces subtractions that depend only on the metric of the boundary.
     The first counter-term is proportional to the volume of the boundary. 
     The second counter-term is proportional to the Einstein-Hilbert action
     of the boundary.  There is a third counter-term, but it is not 
     covariantly defined.  One now adds the Einstein-Hilbert action of 
     the boundary and looks for a saddle point of the total action.  
     This will involve solving the coupled four- and five-dimensional 
     Einstein equations. It will probably have to be done numerically.

     In this talk, I have argued that quantum gravity is unitary, and
     information is preserved in black hole formation and evaporation. 
     I assume the evolution is given by a Euclidean path integral over
     metrics of all topologies. The integral over topologically trivial
     metrics can be done by dividing the time interval into thin slices
     and using a linear interpolation to the metric in each slice.  The
     integral over each slice will be unitary, and so the whole path
     integral will be unitary.

     On the other hand, the path integral over topologically nontrivial
     metrics, will lose information, and will be asymptotically independent
     of its initial conditions. Thus the total path integral will be
     unitary, and quantum mechanics is safe.

     It is great to solve a problem that has been troubling me for nearly
     thirty years, even though the answer is less exciting than the
     alternative I suggested.  This result is not all negative however,
     because it indicates that a black hole evaporates, while remaining
     topologically trivial.  However, the large N solution is likely to 
     be a black hole that shrinks to zero.  This is what I suggested in 1975.

     In 1997, Kip Thorne and I bet John Preskill that information was
     lost in black holes.  The loser or losers of the bet are to provide
     the winner or winners with an encyclopaedia of their own choice, from
     which information can be recovered with ease.  I'm now ready to concede
     the bet, but Kip Thorne isn't convinced just yet.  I will give John
     Preskill the encyclopaedia he has requested.  John is all-American, so
     naturally he wants an encyclopaedia of baseball.  I had great difficulty 
     in finding one over here, so I offered him an encyclopaedia of cricket, 
     as an alternative, but John wouldn't be persuaded of the superiority 
     of cricket.  Fortunately, my assistant, Andrew Dunn, persuaded the 
     publishers Sportclassic Books to fly a copy of "Total Baseball: The 
     Ultimate Baseball Encyclopedia" to Dublin.  I will give John the 
     encyclopaedia now.  If Kip agrees to concede the bet later, he
     can pay me back.

\end{quote}
At this point the encyclopedia was brought on stage and given
to John Preskill, who <A HREF = "dublin/index.html#preskill">waved it over his 
head in a parody of athletic triumph</A>.  The order of events is 
a bit fuzzy in my
mind, but sometime around then he said "I always hoped that when 
Stephen conceded, there would be a witness - this really exceeds 
my expectations."

After this, Kip Thorne ran a question and answer period, saying that
he would alternate between questions from conference participants,
which Hawking's grad student would answer, and questions from the 
press, which Hawking would answer - after Thorne checked Hawking's 
facial expressions to see whether he felt they were worth answering.  

First, a correspondent from the BBC asked Stephen Hawking what the 
significance of this result was for "life, the universe and 
everything".
(Here I'm using John Preskill's humorous paraphrase.)  Hawking agreed
to answer this, and while he began laboriously composing a reply using 
the computer system on his wheelchair, his grad student Christophe 
Galfard fielded three questions from experts: Bill Unruh, Gary Horowitz 
and Robb Mann.  I didn't find the replies terribly illuminating, except 
that when asked if information would be lost if we kept feeding the black 
hole matter to keep it from evaporating away, Galfard said "yes".  
Everyone
afterwards commented on what a tough job it would be for a student to 
field questions in front of about 800 physicists and the international 
press.  

At this point Kip Thorne checked to see if Hawking was done composing
his reply.   He was not.  To fill time, Thorne explained why he hadn't
yet conceded the bet, saying "This looks to me, on the face of it, to
be a lovely argument.  But I haven't seen all the details."  He took
this opportunity to tell the reporters a bit about how science was done:
we don't just listen to Hawking and take his word for everything, we have
to go off and check things ourselves. 
He told a nice story about how when Hawking first showed
that black holes radiate, everyone with their own approach to quantum
field theory on curved spacetime needed to redo this calculation their
own way to be convinced - with Yakov Zeldovich, who'd gotten the game
started by showing that energy could be extracted from \emph{rotating} black
holes in the form of radiation, being one of the very last to agree!
Preskill chimed in, saying "I'll be honest - I didn't understand the
talk", and that he too would need to see more details.  

After a bit more of this sort of thing, Hawking was ready to answer
the BBC reporter's question.  His answer was surprisingly short, and it
went something like this (I can't find an exact quote): "This result
shows that everything in the universe is governed by the laws of 
physics."  A suitably grandiose answer for a grandiose question!
One can imagine better explanations of unitarity, but not quicker ones.

At this point Kip Thorne solicited more questions from the press but
said they should confine themselves to yes-or-no questions, so Hawking
could answer them more efficiently. 
Jenny Hogan got the first question, asking what Hawking would do now
that he'd solved this problem.  Kip Thorne pointed out that this was
not a yes-or-no question.  Hogan replied that it shouldn't take long
to reply; Thorne was doubtful, but in the midst of the ensuing conversation
Hawking shot off an unexpectedly rapid response: "I don't know."  
Everyone
laughed, and at this point the public question period was called to a close,
though reporters were allowed to stay and pester Hawking some more.

At the time Hawking's talk seemed very cryptic to me, but in the process
of editing the above transcript it's become a lot clearer, so I'll try
to give a quick explanation.  

I should start by saying that the jargon used in this talk, while 
doubtless obscure to most people, is actually quite standard and not 
very difficult to anyone who has spent some time studying the Euclidean 
path integral approach to quantum gravity.  The problem is not the 
jargon so much as the lack of detail, which requires some imagination 
to fill in.  When I first heard the talk, this lack of detail had me
completely stumped.  But now it makes a little more sense....

He's studying the process of creating a black hole and letting it
evaporate away.   He's imagining studying this in the usual style
of particle physics, as a "scattering experiment", where we throw in
a bunch of particles and see what comes out.  Here we throw in a bunch
of particles, let them form a black hole, let the black hole evaporate
away, and examine the particles (typically photons for the most part) 
that shoot out.  

The rules of the game in a "scattering experiment" are that we can 
only talk about what's going on "at infinity", meaning very far 
from where the black hole forms - or more precisely, where it may
or may not form!  

The advantage of this is that physics at infinity can be described 
without the full machinery of quantum gravity: we don't have to worry 
about quantum fluctuations of the geometry of spacetime messing up 
our ability to say where things are.  The disadvantage is that we 
can't actually say for sure whether or not a black hole formed.  At
least this \emph{seems} like a "disadvantage" 
at first - but a better term 
for it might be a "subtlety", since it's crucial for resolving the 
puzzle:

\begin{quote}
     Black hole formation and evaporation can be thought of as a
     scattering process. One sends in particles and radiation from
     infinity, and measures what comes back out to infinity.  All
     measurements are made at infinity, where fields are weak, and one
     never probes the strong field region in the middle.  So one can't 
     be sure a black hole forms, no matter how certain it might be in
     classical theory.  I shall show that this possibility allows
     information to be preserved and to be returned to infinity.
\end{quote}

Now, the way Hawking likes to calculate things in this sort of 
problem is using a "Euclidean path integral".  This is a rather
controversial approach - hence his grin when he said it's the 
"only sane way" to do these calculations - but let's not worry about
that.   Suffice it to say that we replace the time variable "t"
in all our calculations by "it", do a bunch of calculations, and 
then replace "it" by "t" again at the end.   
This trick is called
"Wick rotation".  In the middle of this process, we hope all our 
formulas involving the geometry of 4d \emph{spacetime} have magically 
become formulas involving the geometry of 4d \emph{space}.   The answers 
to physical questions are then expressed as integrals over all 
geometries of 4d space that satisfy some conditions depending on 
the problem we're studying.   This integral over geometries also
includes a sum over topologies.  

That's what Hawking means by this:

\begin{quote}
     I adopt the Euclidean approach, the only sane way to do quantum
     gravity non-perturbatively.  In this, the time evolution of an 
     initial state is given by a path integral over all positive 
     definite metrics that go between two surfaces that are a distance 
     T apart at infinity.  One then Wick rotates the time interval, T, 
     to the Lorentzian.  The path integral is taken over metrics of 
     all possible topologies that fit in between the surfaces.  
\end{quote}

Unfortunately, nobody knows how to define these integrals.  However,
physicists like Hawking are usually content to compute them in a
"semiclassical approximation".  This means integrating not over all
geometries, but only those that are close to some solution of the 
classical equations of general relativity.  We can then do a clever 
approximation to get a closed-form answer.

(Nota bene: here I'm talking about the equations of general relativity
on 4d \emph{space}, not 4d spacetime.  That's because we're in the middle
of this Wick rotation trick.)

Actually, I'm oversimplifying a bit.  We don't get "the answer" to
our physics question this way: we get one answer for each solution 
of the equations of general relativity that we deem relevant to the 
problem at hand.  To finish the job, we should add up all these partial
answers to get the total answer.  But in practice this last step is 
always too hard: there are too many topologies, and too many classical
solutions, to keep track of them all.

So what do we do?  We just add up a few of the answers, cross our 
fingers, and hope for the best!  If this procedure offends you, go 
do something easy like math.

In the problem at hand here, Hawking focuses on two classical solutions,
or more precisely two classes of them.  One describes a spacetime with no 
black hole, the other describes a spacetime with a black hole which lasts
forever.  Each one gives a contribution to the semiclassical approximation 
of the integral over all geometries.  To get answers to physical questions, 
he needs to sum over \emph{both}.   
In principle he should sum over infinitely 
many others, too, but nobody knows how, so he's probably hoping the crux 
of the problem can be understood by considering just these two.   

He says that if you just do the integral over geometries near the
classical solution where there's no black hole, you'll find - 
unsurprisingly - that no information is lost as time passes.

He also says that if you do the integral over geometries near the
classical solution where there is a black hole, you'll find -
surprisingly - that the answer is \emph{zero} for a lot of questions 
you can measure the answers to far from the black hole.  In physics 
jargon, this is because a bunch of "correlation functions decay 
exponentially".

So, when you add up both answers to see if information is lost in the
real problem, where you can't be sure if there's a black hole or not,
you get the same answer as if there were no black hole!  

\begin{quote}
     So in the end, everyone was right, in a way.  Information is lost 
     in topologically nontrivial metrics, like the eternal black hole. 
     On the other hand, information is preserved in topologically trivial
     metrics. The confusion and paradox arose because people thought
     classically, in terms of a single topology for spacetime.  It was
     either R^{4}, or a black hole.  But the Feynman sum over histories allows
     it to be both at once.  One can not tell which topology contributed the
     observation, any more than one can tell which slit the electron went
     through, in the two slits experiment.  All that observation at infinity
     can determine is that there is a unitary mapping from initial states
     to final, and that information is not lost.
\end{quote}

The mysterious part is why the geometries near the classical solution 
where there's a black hole don't contribute at all to information loss, 
even though they do contribute to other important things, like the
Hawking radiation.  Here I'd need to see an actual calculation.  Hawking
gives a nice hand-wavy topological argument, but that's not enough for
me.   

Since this issue is long enough already and I want to get it out soon,
I won't talk about other things that happened at this conference - nor
will I talk about the conference on n-categories earlier this summer!
I just want to say a few elementary things about the topology lurking 
in Hawking's talk... since some mathematicians may enjoy it. 

As he points out, the answers to a bunch of questions diverge unless 
we put our black hole in a box of finite size.  A convenient way
to do this is to introduce a small negative cosmological constant,
which changes our default picture of spacetime from Minkowski spacetime,
which is topologically R^{4}, to anti-deSitter spacetime, which is 
topologically R x D^{3} after we add the "boundary at infinity".  
Here R is time and the 3-disk D^{3} is space.  This is a Lorentzian 
manifold with boundary, but when we do Wick rotation we get a Riemannian
manifold with boundary having the same topology.  

However, when we are doing Euclidean path integrals at nonzero 
temperature, we should replace the time line R here by a circle 
whose radius is the reciprocal of that temperature.  (Take my word 
for it!)  So now our Riemannian manifold with boundary is S^{1} x 
D^{3}.  
This is what Hawking uses to handle the geometries without a black
hole.  The boundary of this manifold is S^{1} x S^{2}.  
But there's 
another obvious manifold with this boundary, namely D^{2} x 
S</sup>2</sup>.  And 
this corresponds to the geometries with a black hole!   This is cute
because we see it all the time in surgery theory.  In fact I commented
on Hawking's use of this idea a long time ago, in 
"<A HREF = "week67.html">week67</A>".

In his talk, Hawking points out that S^{1} x D^{3} 
has a nontrivial 3-cycle 
in it if we use relative cohomology and work relative to the boundary
S^{1} x S^{2}.  But, D^{2} x S^{2} 
does not.  When spacetime is n-dimensional, 
conservation laws usually come from integrating closed (n-1)-forms over 
cycles that correspond to "space", 
so we get interesting conservation laws 
when there are nontrivial (n-1)-cycles.  Here Hawking is using this to
argue for conservation of information when there's no black hole - namely
for S^{1} x D^{3} - but not when there is, namely for 
D^{2} x S^{2}.  

All this is fine and dandy; the hard part is to see why the case when there 
\emph{is} a black hole doesn't screw things up!  This is where his allusions 
to "exponentially decaying correlation functions come in" - 
and this is 
where I'd like to see more details.  I guess a good place to start is
Maldacena's papers on the black hole in 3d spacetime - the so-called 
Banados-Teitelboim-Zanelli or "BTZ" black hole.  
This is a baby version 
of the problem, one dimension down from the real thing, where everything 
should get much simpler.  For a bunch about the BTZ black hole, try:

3) Maximo Banados, Marc Henneaux, Claudio Teitelboim, and Jorge Zanelli,
Geometry of the 2+1 black hole, 
Phys. Rev. D48 (1993) 1506-1525, also 
available as 
<A HREF = "http://www.arXiv.org/abs/gr-qc/9302012">gr-qc/9302012</A>.

The relevant paper by Maldacena seems to be:

4) Juan Maldacena, Eternal Black Holes in AdS,
JHEP 0304 (2003) 021, also 
available as 
<A HREF = "http://arxiv.org/abs/hep-th/0106112">hep-th/0106122</A>. 

You can also see a talk he gave on this
at the Institute for Theoretical Physics at U. C. Santa Barbara:

5) Juan Maldacena, Eternal Black Holes in AdS, 
<A HREF = "http://online.itp.ucsb.edu/online/mtheory_c01/maldacena/">
http://online.itp.ucsb.edu/online/mtheory_c01/maldacena/</A>.

By the way, here are some photos of the conference...

6) John Baez, Dublin, <A HREF = "http://math.ucr.edu/home/baez/dublin/">
http://math.ucr.edu/home/baez/dublin/</A>

... and also 
photos of the <A HREF = "http://math.ucr.edu/home/baez/dublin/index.html#hamilton">plaque</A> on the bridge where Hamilton carved his defining relations for 
the quaternions!
 
\par\noindent\rule{\textwidth}{0.4pt}
\textbf{Addendum:} My friend Ted Bunn filled a gap in my 
understanding of the history of astronomy.  I had written:


\begin{verbatim}

It turned out to describe an idea I'd never even dreamt of before: 
a heliocentric cosmology in which the planets move along circular orbits 
with epicycles a la Ptolemy!  
\end{verbatim}
    

to which he replied:


$$

There is nothing new under (or orbiting) the Sun.  This idea is
originally due to Copernicus.  Thomas Kuhn's book "The Copernican
Revolution" has a nice discussion.
$$
    

In retrospect it's obvious that \emph{someone} had to try this idea
before Kepler came up with elliptical heliocentric orbits. In fact,
Kepler tried ellipses only because the epicycle theory didn't work well 
for Mars.

\par\noindent\rule{\textwidth}{0.4pt}
<em>I'm not that good at math, but I do know that the universe
is formed with mathematical principles whether I understand them
or not, and I was going to let that guide me.</em> - Bob Dylan,
Chronicles (vol. 1)

\par\noindent\rule{\textwidth}{0.4pt}

% </A>
% </A>
% </A>
