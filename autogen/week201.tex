
% </A>
% </A>
% </A>
\week{January 10, 2004 }



Lately James Dolan and I have been studying number theory.  I used to 
\emph{hate} this subject: it seemed like a massive waste of time.  
Newspapers, 
magazines and even lots of math books seem to celebrate the idea of people 
slaving away for centuries on puzzles whose only virtue is that they're 
easy to state but hard to solve.  For example: are any odd numbers the 
sum of all their divisors?  Are there infinitely many pairs of primes 
that differ by 2?  Is every even number bigger than 2 a sum of two primes? 
Are there any positive integer solutions to 

x^{n} + y^{n} = z^{n}

for n > 2?  My response to all these was: WHO CARES?!  

Sure, it's noble to seek knowledge for its own sake.  But working on a 
math problem just because it's \emph{hard} is like trying to drill a hole in 
a concrete wall with your nose, just to prove you can!   If you succeed,
I'll be impressed - but I'll still wonder why you didn't put all that
energy into something more interesting.  

Now my attitude has changed, because I'm beginning to see that behind 
these silly hard problems there lurks an actual \emph{theory}, full of deep 
ideas and interesting links to other branches of mathematics, including 
mathematical physics.  It just so happens that now and then this theory
happens to crack another hard nut.  

I'd known for a while that something like this must be true: after all, 
when Andrew Wiles proved Fermat's Last Theorem, even the newspapers 
admitted this was just a spinoff of something more important, namely 
a special case of the Taniyama-Shimura Conjecture.  They said this had 
something to do with elliptic curves and modular forms, which are very 
nice geometrical things that show up all over in complex analysis and 
string theory.  Unfortunately, the actual statement of this conjecture 
seemed impenetrable - it didn't resonate with things I understood.  

In fact, the Taniyama-Shimura Conjecture is part of a big \emph{network} of 
problems that are more interesting but harder to explain than the flashy 
ones I listed above: problems like the Extended Riemann Hypothesis, the 
Weil Conjecture (now solved), the Birch-Swinnerton-Dyer Conjecture,
and bigger projects like the Langlands Program and developing the theory 
of "motives".  And these problems rest on top 
of a solid foundation of 
beautiful stuff that's already known, like Galois theory and class field 
theory, and stuff about modular forms and L-functions.

As I'm gradually beginning to understand little bits of these things, 
I'm getting really excited about number theory... so I'm dying to 
\emph{explain}
some of it!  But where to start?  I have to start with something basic that
underlies all the fancy stuff.  Hmm, I think I'll start with Galois theory.

As you may have heard, Galois invented group theory in the process of
showing you can't solve the quintic equation

$$
ax^{5} + bx^{4} + cx^{3} + dx^{2} + ex + f = 0
$$
    
by radicals.  In other words, he showed you can't solve this equation 
by means of some souped-up version of the quadratic formula that just 
involves taking the coefficients a,b,c,d,e,f and adding, subtracting, 
multiplying, dividing and taking nth roots.
  
The basic idea is something like this.  In general, a quintic equation 
has 5 solutions - and there's no "best one", so your formula has 
got to 
be a formula for all five.  And there's a puzzle: how do you give one 
formula for five things?  

Well, think about the quadratic formula!   It has that "plus 
or minus" 
in it, which comes from taking a square root.  So, it's really a formula 
for \emph{both} solutions of the quadratic equation.  If there were a formula 
for the quintic that worked like this, we'd have to get all 5 solutions 
from different choices of nth roots in this formula.

Galois showed this can't happen.  And the way he did it used \emph{symmetry!}
Roughly speaking, he showed that the general quintic equation is completely 
symmetrical under permuting all 5 solutions, and that this symmetry group - 
the group of permutations of 5 things - can't be built up from the symmetry
groups that arise when you take nth roots.  

The moral is this: you can't solve a problem if the answer has some 
symmetry, and your method of solution doesn't let you write down a correct
answer that has this symmetry!   

An old example of this principle is the medieval puzzle called "Buridan's
Ass".  Placed equidistant between two equally good piles of hay, this
donkey starves to death because it can't make up its mind which alternative
is best.  The problem has a symmetry, but the donkey can't go to \emph{both}
bales of hay, so the only symmetrical thing he can do is stand there.  

Buridan's ass would also get stuck if you asked it for \emph{the} solution
to the quadratic equation.  Galois proof of the unsolvability of the 
quintic by radicals is just a more sophisticated variation on this theme.  
(Of course, you \emph{can} solve the quintic if you strengthen your methods.)

A closely related idea is "Curie's principle", named after Marie's 
husband Pierre.  This says that if your problem has a symmetry and 
it is a unique solution, the solution must be symmetrical. 

For example, if some physical system has rotation symmetry and it has a
unique equilibrium state, this state must be rotationally invariant.

Now, in the case of a ferromagnet below its "Curie temperature", the 
equilibrium state is \emph{not} rotationally invariant: the little magnetized
electrons line up in some specific direction!  But this doesn't 
contradict Curie's principle, since there's not a unique equilibrium 
state - there are lots, since the electrons can line up in any direction.  

Physicists use the term "spontaneous symmetry breaking" when any 
\emph{one}
solution of a symmetric problem is not symmetrical, but the whole set 
of them is.  This is precisely what happens with the quintic, or even 
the quadratic equation.

While these general ideas about symmetry apply to problems of all sorts,
their application to number theory kicks in when we apply them to \emph{fields}.  
A "field" is a gadget where you can add, subtract, multiply and divide
by anything nonzero, and a bunch of familiar laws of arithmetic hold, 
which I won't bore you with here.  The three most famous fields are the 
rational numbers Q, the real numbers R, and the complex numbers C.  
However, there are lots of other interesting fields.  

Number theorists are especially fond of algebraic number fields.  An 
"algebraic number" is a solution to a polynomial equation whose coefficients 
are rational numbers.  You get an "algebraic number field" by taking the 
field of rational numbers, throwing in finitely many algebraic numbers, and 
then adding, subtracting, multiplying and dividing them to get more numbers 
until you've got a field.  

For example, we could take the rationals, throw in the square root of 2, 
and get a field consisting of all numbers of the form

\begin{verbatim}
a + b sqrt(2)
\end{verbatim}
    
where a and b are rational.  Notice: if we add, multiply, subtract or
divide two numbers like this, we get another number of this form.
So this is really a field - and it's called Q(sqrt(2)), since we use 
round parentheses to denote the result of taking a field and "extending" 
it by throwing in some extra numbers.

More generally, we could throw in the square root of any integer n, 
and get an algebraic number field called Q(sqrt(n)), consisting of all
numbers

\begin{verbatim}
a + b sqrt(n)
\end{verbatim}
    
where a and b are rational.  If sqrt(n) is rational then this field is
just Q, which is boring.  Otherwise, we call it a "quadratic number field".

Even more generally, we could take the rationals and throw in a
solution of any quadratic equation with rational coefficients.  But 
it's easy to see that this doesn't give anything beyond fields like
Q(sqrt(n)).  And that's the real reason we call these the "quadratic 
number fields".

There are also "cubic number fields", and "quartic number fields",
and "quintic number fields", and so on.  And others, too, where we
throw in solutions to a whole bunch of polynomial equations!

Now, it turns out you can answer lots of classic but rather goofy-sounding 
number theory puzzles like "which integers are a sum of two squares?" 
by converting them into questions about algebraic number fields.  
And the good part is, the resulting questions are connected to all sorts 
of other topics in math - they're not just glorified mental gymnastics!  
So, from a modern viewpoint, a bunch of classic number theory puzzles are 
secretly just tricks to get certain kinds of people interested in algebraic 
number fields.  

But right now I \emph{don't} want to explain how we can use algebraic number 
fields to solve classic but goofy-sounding number theory puzzles.
In fact, I want to downplay the whole puzzle aspect of number theory.

Instead, I hope you're reeling with horror at thought of this vast 
complicated wilderness of fields containing Q but contained in C.
First there's a huge infinite thicket of algebraic number fields...
and then, there's an ever scarier jungle of fields that contain 
transcendental numbers like \pi  and e!  I won't even talk about \emph{that}
jungle, it's so dark and scary.  Physicists usually zip straight past 
this whole wilderness and work with C.  

But in fact, if you stop and carefully examine all the algebraic number 
fields and how they sit inside each other, you'll find some incredibly 
beautiful patterns.  And these patterns are turning out to be related to 
Feynman diagrams, topological quantum field theory, and so on...

However, before we can talk about all that, we need to understand the
basic tool for analyzing how one field fits inside another: Galois theory!

A function from a field to itself that preserves addition, subtraction,
multiplication and division is called an "automorphism".  It's just
a \emph{symmetry} of the field.  But now, suppose we have a field K which 
contains some smaller field k.  Then we define the "Galois group of K 
over k" to be the group of all automorphisms of K that act as the 
identity on k.  We call this group

\begin{verbatim}
Gal(K/k)
\end{verbatim}
    
for short.

The classic example, familiar to all physicists, is the Galois group of 
the complex numbers, C, over the real numbers, R.  This group has two 
elements: the identity transformation, which leaves everything alone, and 
complex conjugation, which switches i and -i.  Since the only group with 2
elements is Z/2, we have

\begin{verbatim}
Gal(C/R) = Z/2
\end{verbatim}
    
Where does complex conjugation come from?  It comes from the fact that
we get C from R by throwing in a solution of the quadratic equation 

$$
x^{2} = -1
$$
    
We say C is a "quadratic extension of R".  But as soon as we throw in one 
solution of this equation, we inevitably throw in another, namely its 
negative - and there's no way to tell which is which.  And complex 
conjugation is the symmetry that switches them!  

Note: we know that i and -i are different, but we can't tell which is 
which!  This sounds a bit odd at first.  It's a bit hard to explain
precisely in ordinary language, which is part of why Galois had to invent 
group theory.  But it's fun to try to explain it in plain English... 
so let me try.  The complex numbers have two solutions to

$$
x^{2} = -1
$$
    
By convention, one of them is called "i", and the other is called
"-i".  Having made this convention, there's never any problem telling
them apart.  But we could reverse our convention and nothing would
go wrong.   For example, if the ghost of Galois wafted into your office
one moonless night and wrote "-i" in all your math and physics books 
wherever there had been "i", everything in these books would still be true!

Here's another way to think about it.  Suppose we meet some 
extraterrestrials and find that they too have developed the complex 
numbers by taking the real numbers and adjoining a square root of -1, 
only they call it "@".  Then there would be no way for us to tell if 
their "@" was our "i" or our "-i".  All we can do is choose an arbitrary
convention as to which is which.

Of course, if they put their "@" in the lower halfplane when 
drawing the 
complex plane, we might feel like calling it "-i"... but here we are 
secretly making use of a convention for matching their complex plane with 
ours, and the \emph{other} convention would work equally well!  If they drew 
their real line \emph{vertically} in the complex plane, it would be more 
obvious that we need a convention to match their complex plane with ours, 
and that there are two conventions for doing this, both perfectly 
self-consistent.  

If you've studied enough physics, this extraterrestrial scenario 
should remind you of those thought experiments where you're trying to
explain to some alien civilization the difference between left and
right... by means of radio, say, where you're \emph{not} allowed to refer
to specific objects you both know - so it's cheating to say "imagine 
you're on Earth looking at the Big Dipper and the handle is pointing 
down; then Arcturus is to the right."  

If the laws of physics didn't distinguish between left and right, 
you couldn't explain the difference between left and right without 
"cheating" like this, so the laws of physics would have a symmetry 
group with two elements: the identity and the transformation that 
switches left and right.  As it turns out, the laws of physics \emph{do}
distinguish between left and right - see "<A HREF = "week73.html">week73</A>" for more on that.  
But that's another story.  My point here is that the Galois group of C 
over R is a similar sort of thing, but built into the very fabric of 
mathematics!  And that's why complex conjugation is so important.  

I could tell you a nice long story about how complex conjugation is
related to "charge conjugation" (switching matter and antimatter) and 
also "time reversal" (switching past and future).  But I won't!

Here's another example of a Galois group that physicists should like.  
Let C(z) be the field of rational functions in one complex variable z -
in other words, functions like

\begin{verbatim}
f(z) = P(z)/Q(z)
\end{verbatim}
    
where P and Q are polynomials in z with complex coefficients.  You can 
add, subtract, multiply and divide rational functions and get other 
rational functions, so they form a field.  And they contain C as a 
subfield, because we can think of any complex number as a \emph{constant}
function.  So, we can ask about the Galois group of C(z) over C.
What's it like?

It's the Lorentz group!

To see this, it's best to think of rational functions as functions not
on the complex plane but on the "Riemann sphere" - the complex plane 
together with one extra point, the "point at infinity".  The only 
conformal transformations of the Riemann sphere are "fractional linear
transformations":

\begin{verbatim}
       az + b
T(z) = ------
       cz + d
\end{verbatim}
    
So, the only symmetries of the field of rational functions that 
act as the identity on constant functions are those coming from 
fractional transformations, like this:

$$
f |-> fT                        where fT(z) = f(T(z)) 
$$
    
If you don't follow my reasoning here, don't worry - the details aren't 
hard to fill in, but they'd be distracting here.  

The last step is to check that the group of fractional linear 
transformations is the same as the Lorentz group.  You can do this 
algebraically, but you can also do it geometrically by thinking of the 
Riemann sphere as the "heavenly sphere": that imaginary sphere the stars 
look like they're sitting on.  The key step is to check this remarkable fact: 
if you shoot past the earth near the speed of light, the constellations will
look distorted by a Lorentz transformation - but if you draw lines connecting
the stars, all the \emph{angles} between these lines 
will remain the same; only 
their \emph{lengths} will get messed up!  

Moreover, it's obvious that if you rotate your head, both angles and lengths
on the heavenly sphere are preserved.  So, any rotation or Lorentz boost 
gives an angle-preserving transformation of the heavenly sphere - that is, 
a conformal transformation!  And this must be a fractional linear 
transformation.

Summarizing, the Galois group of C(z) over C is the Lorentz group, or
more precisely, its connected component, SO_{0}(3,1):

\begin{verbatim}
Gal(C(z)/C) = SO<sub>0</sub>(3,1).
\end{verbatim}
    
We've talked about the Galois group of C(z) over C and the Galois group
of C over R.  What about the Galois group of C(z) over R?   Unsurprisingly,
this is the group of transformations of the Riemann sphere generated by 
fractional linear transformations \emph{and} 
complex conjugation.  And physically,
this corresponds to taking the connected component of the Lorentz group
and throwing in \emph{time reversal!}  
So you see, complex conjugation is related
to time reversal.  But I promised not to go into that....

I've been talking about Galois groups that physicists should like, but
you're probably wondering where the number theory went!  Well, it's 
all part of the same big story.  In number theory we're especially
interested in Galois groups like

\begin{verbatim}
Gal(K/k)
\end{verbatim}
    
where K is some algebraic number field and k is some subfield of K.
For starters, consider this example:

\begin{verbatim}
Gal(Q(sqrt(n))/Q)
\end{verbatim}
    
where sqrt(n) is irrational.  I've already hinted at what this group is!  
Q(sqrt(n)) has sqrt(n) in it, so it also has -sqrt(n) in it, and there's 
an automorphism that switches these two while leaving all the rational 
numbers alone, namely

$$
a + b sqrt(n) |-> a - b sqrt(n)                  (a,b in Q)
$$
    
So, we have:

\begin{verbatim}
Gal(Q(sqrt(n)))/Q) = Z/2
\end{verbatim}
    
just like the Galois group of C over R.  

To get some bigger Galois groups, let's take Q and throw in a "primitive 
nth root of unity".  Hmm, I may need to explain what that means.  There 
are n different nth roots of 1 - but unlike the two square roots of -1, 
these are not all created equal!  Only some are "primitive".

For example, among the 4th roots of unity we have 1 and -1, which are 
actually square roots of unity, and i and -i, which aren't.  A "primitive 
nth root of unity" is an nth root of 1 that's not an kth root for any
k < n.  If you take all the powers of any primitive nth root of unity,
you get \emph{all} the nth roots of unity.  So, if we take some primitive nth 
root of unity, call it

$$
1^{1/n}
$$
    
for lack of a better name, and extend the rationals by this number, 
we get a field

$$
Q(1^{1/n})
$$
    
which contains all the nth roots of unity.  Since the nth roots of unity
are evenly distributed around the unit circle, this sort of field is called
a "cyclotomic field", for the Greek word for "circle cutting".  In fact,
one can apply Galois theory to this field to figure out which regular
n-gons one can construct with a ruler and compass!

But what's the Galois group

$$
Gal(Q(1^{1/n})/Q)
$$
    
like?   Any symmetry in this group must map 1^{1/n} 
to some root of unity, 
say 1^{m/n} - and once you know which one, you completely know the 
symmetry.   But actually, this symmetry must map 1^{1/n} 
to some \emph{primitive}
root of unity, so m has to be relatively prime to n.  Apart from that, 
though, anything goes - so the size of 

$$
Gal(Q(1^{1/n})/Q)
$$
    
is just the number of guys m less than n that are relatively prime to n.  And 
if you think about it, these numbers relatively prime to n are just the 
same as elements of Z/n that have multiplicative inverses!  So if you think
some more, you'll see that

$$
Gal(Q(1^{1/n})/Q) = (Z/n)*
$$
    
where (Z/n)* is the "multiplicative group" of Z/n - that is, the 
elements of Z/n that have multiplicative inverses, made into a group 
via multiplication!  

This group can be big, but it's still abelian.  Can we get some nonabelian
Galois groups from algebraic number fields?

Sure!  Let's say you take some polynomial equation with rational coefficients, 
take \emph{all} its solutions, throw them into the rationals - and keep 
adding, 
subtracting, multiplying and dividing until you get some field K.  This K 
is called the "splitting field" of your polynomial.  

But here's the interesting thing: if you pick your polynomial equation at 
random, the chances are really good that it has n different solutions if 
the polynomial is of degree n, and that \emph{any} permutation of these 
solutions comes from a unique symmetry of the field K.  In other words:
barring some coincidence, all roots are created equal!  So in general we 
have

\begin{verbatim}
Gal(K/Q) = S<sub>n</sub>
\end{verbatim}
    
where S_{n} is the group of all permutations of n things.

Sometimes of course the Galois group will be smaller, since our polynomial
could have repeated roots or, more subtly, algebraic relations between
roots - as in the cyclotomic case we just looked at.  

But, we can already start to see how to prove the unsolvability of the 
general quintic!  Pick some random 5th-degree polynomial, let K be its
splitting field, and note

\begin{verbatim}
Gal(K/Q) = S<sub>5</sub>
\end{verbatim}
    
Then, show that if we build up an algebraic number field by starting
with Q and repeatedly throwing in nth roots of numbers we've already got, 
we just can't get S_{5} as its Galois group over the rationals!  We've 
already seen this in the case where we throw in a square root of n, or 
an nth root of 1.  The general case is a bit more work.  But instead of 
giving the details, I'll just mention a good textbook on Galois theory for 
beginners:

1) Ian Stewart, Galois Theory, 3rd edition, Chapman and Hall, New York, 
2004.

Ian Stewart is famous as a popularizer of mathematics, and it shows
here - he has nice discussions of the history of the famous problems
solved by Galois theory, and a nice demystification of the Galois'
famous duel.  But, this is a real math textbook - so you can really 
learn Galois theory from it!  Make sure to get the 3rd edition, since
it has more examples than the earlier ones.

Having given Ian Stewart the dirty work of explaining Galois theory in
the usual way, let me say some things that few people admit in a first
course on the subject.  

So far, we've looked at examples of a field k contained in some bigger
field K, and worked out the group Gal(K/k) consisting of all automorphisms 
of K that fix everything in k.  

But here's the big secret: this has NOTHING TO DO WITH FIELDS!  It works
for ANY sort of mathematical gadget!  If you've got a little gadget k 
sitting in a big gadget K, you get a "Galois group" Gal(K/k) consisting of 
symmetries of the big gadget that fix everything in the little one. 

But now here's the cool part, which is also very general.   Any subgroup 
of Gal(K/k) gives a gadget containing k and contained in K: namely, the 
gadget consisting of all the elements of K that are fixed by everything 
in this subgroup.

And conversely, any gadget containing k and contained in K gives a 
subgroup of Gal(K/k): namely, the group consisting of all the symmetries 
of K that fix every element of this gadget.  

This was Galois' biggest idea: we call this a GALOIS CORRESPONDENCE.
It lets us use \emph{group theory} to classify gadgets contained in one 
and containing another.  He applied it to fields, but it turns out 
to be useful much more generally.  

Now, it would be great if the Galois corresondence were always a perfect 
1-1 correspondence between subgroups of Gal(K/k) and gadgets containing
k and contained in K.  But, it ain't true.  It ain't even true when we're 
talking about fields!  

However, that needn't stop us.  For example, we can restrict ourselves 
to cases when it \emph{is} true.  
And this is where the Fundamental Theorem of 
Galois Theory comes in!  It's easiest to state this theorem when k and K 
are algebraic number fields, so that's what I'll do.  In this case, there's
a 1-1 correspondence between subgroups of Gal(K/k) and extensions of k 
contained in K if:

\begin{quote}
i) K is a &quot;finite&quot; extension of k.  In other words, K is a finite-dimensional
vector space over k.

ii) K is a &quot;normal&quot; extension of k.  In other words, if a 
polynomial with coefficients in k can't be factored at all in k, 
but has one root in K, then all its roots are in K.
\end{quote}
    

For general fields we also need another condition, namely that K be a 
"separable" extension of k.  But this is automatic for algebraic number
fields, so let's not worry about it.  

At this point, if we had time, we could work out a bunch of Galois groups
and see a bunch of patterns.  Using these, we could see why you can't
solve the general quintic using radicals, why you can't trisect the angle 
or double the cube using ruler-and-compass constructions, and why you can
draw a regular pentagon using ruler and compass, but not a regular heptagon.
Basically, to prove something is impossible, you just show that some number
can't possibly lie in some particular algebraic number field, because it's 
the root of a polynomial whose splitting field has a Galois group that's 
"fancier" than the Galois group of that algebraic number field.  

For example, ruler-and-compass constructions produce distances that lie in 
"iterated quadratic extensions" of the rationals - meaning that you just 
keep throwing in square roots of stuff you've got.  Doubling the cube 
requires getting your hands on the cube root of 2.  But the Galois group 
of the splitting field of 

$$
x^{3} = 2
$$
    
has size divisible by 3, while an iterated quadratic extension has a Galois
group whose size is a power of 2.  Using the Galois correspondence, we see
there's no way to stuff the former field into the latter.

But you can read about this in any good book on Galois theory, so I'd rather
dive right into that thicket I was hinting at earlier: the field of ALL 
algebraic numbers!  The roots of any polynomial with coefficients in this 
field again lie in this field, so we say this field is "algebraically 
closed".  And since it's the smallest algebraically closed field containing 
Q, it's called the "algebraic closure of Q", or Q^{-} for short - that is, Q 
with a bar over it.  (I can't quite draw it here.)

This field Q^{-} is huge.  In particular, it's an infinite-dimensional vector 
space over Q.  So, condition i) in the Fundamental Theorem of Galois Theory 
doesn't hold.  But that's no disaster: when this happens, we just need to 
put a topology on the group Gal(K/k) and set up the Galois correspondence using
\emph{closed} subgroups of Gal(K/k).  Using this trick, every algebraic number field
corresponds to some closed subgroup of Gal(Q^{-}/Q).

So, for people studying algebraic number fields,

$$
                           Gal(Q^{-}/Q)
$$
    
is like the holy grail.  It's the symmetry group of the algebraic numbers, 
and the key to how all algebraic number fields sit inside each other!
But alas, this group is devilishly complicated.  In fact, it has literally 
driven men mad.  One of my grad students knows someone who had a breakdown 
and went to the mental hospital while trying to understand this group!

(There may have been other reasons for his breakdown, too, but as readers
of E. T. Bell's book "Men in Mathematics" know, the facts should never get
in the way of a good anecdote.)  

If Gal(Q^{-}/Q) were just an infinitely tangled thicket, it wouldn't be so 
tantalizing.  But there are things we can understand about it!  To describe
these, I'll have to turn up the math level a notch...

First of all, an extension K of a field k is called "abelian" if Gal(K/k) 
is an abelian group.  Abelian extensions of algebraic number fields can be
understood using something called class field theory.  In particular, the 
Kronecker-Weber theorem says that every finite abelian extension of Q is 
contained in a cyclotomic field.  So, they all sit inside a field called
Q^{cyc}, which is gotten by taking the rationals and throwing in \emph{all} nth 
roots of unity for \emph{all} n.  Since

$$
Gal(Q(1^{1/n})/Q) = (Z/n)*
$$
    
we know from Galois theory that Gal(Q^{cyc}/Q) must be a big group containing 
all the groups (Z/n)* as closed subgroups.  It's easy to see that (Z/n)* is
a quotient group of (Z/m)* if m is divisible by n; this lets us take the 
"inverse limit" of all the groups (Z/m)* - and that's Gal(Q^{cyc}/Q).  This 
inverse limit is also the multiplicative group of the ring Z^{^}, the inverse 
limit of all the rings Z/n.  Z^{^} is also called the "profinite completion of 
the integers", and I urge you to play around with it if you never have!  
It's a cute gadget.

In short:

Gal(Q^{cyc}/Q) = Z^{^}*

and if we stay inside Q^{cyc}, we're in a zone where the pattern of algebraic
number fields can be understood.  This stuff was worked out by people like
Weber, Kronecker, Hilbert and Takagi, with the final keystone, the Artin
reciprocity theorem, laid in place by Emil Artin in 1927.  In a certain 
sense Q^{cyc} is to Q^{-} as homology theory is to homotopy theory: it's all 
about \emph{abelian} Galois groups, so it's manageable.  

People now use Q^{cyc} as a kind of base camp for further expeditions into
the depths of Q^{-}.  In particular, since 

$$
Q is contained in Q^{cyc} and Q^{cyc} is contained in Q^{-}
$$
    
we get an exact sequence of Galois groups:

$$
1 \to  Gal(Q^{-}/Q^{cyc}) \to  Gal(Q^{-}/Q) \to  Gal(Q^{cyc}/Q) \to  1
$$
    
So, to understand Gal(Q^{-}/Q) we need to understand Gal(Q^{cyc}/Q),
Gal(Q^{-}/Q^{cyc}) and how they fit together!  The last two steps are not 
so easy.  Shafarevich has conjectured that Gal(Q^{-}/Q^{cyc}) is the 
profinite completion of a free group, say F^{^}.  This would give

$$
1 \to  F^{^} \to  Gal(Q^{-}/Q) \to  Z^{^}* \to  1
$$
    
but I have no idea how much evidence there is for Shafarevich's conjecture,
or how much people know or guess about this exact sequence.

More recently, Deligne has turned attention to a certain "motivic" version
of Gal(Q^{-}/Q), which is a proalgebraic group scheme.  This sort of group
has a \emph{Lie algebra}, which makes it more tractable.  And there are a 
bunch
of fascinating conjectures about this Lie algebra is related to the Riemann
zeta function at odd numbers, Connes and Kreimer's work on Feynman diagrams,
Drinfeld's work on the Grothendieck-Teichmueller group, and more!  

I really want to understand this stuff better - right now, it's a complete
muddle in my mind.  When I do, I will report back to you.  For now, though,
let me give you some references.  

For two very nice but very different introductions to algebraic number 
fields, try these:

2) H. P. F. Swinnerton-Dyer, A Brief Guide to Algebraic Number Theory,
Cambridge U. Press, Cambridge 2001.

3) Juergen Neukirch, Algebraic Number Theory, trans. Norbert Schappacher,
Springer, Berlin, 1986. 

Both assume you know some Galois theory or at least can fake it.
Neukirch's book is good for the all-important analogy between Galois 
groups and fundamental groups, which I haven't even touched upon here!
Swinnerton-Dyer's book has the virtue of brevity, so you can see the
forest for the trees.  Both have a friendly, slightly chatty style that
I like. 

For Shafarevich's conjecture, try this:

4) K. Iwasawa, On solvable extensions of algebraic number fields,
Ann. Math. 58 (1953) 548-572.

For Deligne's motivic analogue, try this:

5) Pierre Deligne, Le groupe fondamental de la droite projective
moins trois points, in Galois Groups over Q, MSRI Publications 16 (1989),
79-313.

This stuff has a lot of relationships to 3d topological quantum field
theory, braided monoidal categories, and the like... and it all goes
back to the Grothendieck-Teichmueller group.  To learn about this group 
try this book, and especially this article in it:

6) Leila Schneps, The Grothendieck-Teichmuller group: a survey,
in The Grothendieck Theory of Dessins D'Enfants, London Math. Society
Notes 200, Cambridge U. Press, Cambridge 1994, pp. 183-204.

To hear and watch some online lectures on this material, try:

7) Leila Schneps, The Grothendieck-Teichmuller group and fundamental 
groups of moduli spaces, MSRI lecture available at
<A HREF = "http://www.msri.org/publications/ln/msri/1999/vonneumann/schneps/1/">http://www.msri.org/publications/ln/msri/1999/vonneumann/schneps/1/</A>

Grothendieck-Teichmuller group and Hopf algebras,
MSRI lecture available at 
<A HREF = "http://www.msri.org/publications/ln/msri/1999/vonneumann/schneps/2/">http://www.msri.org/publications/ln/msri/1999/vonneumann/schneps/2/</A>

For a quick romp through many mindblowing ideas which touches on this
material near the end:

8) Pierre Cartier, A mad day's work: from Grothendieck to Connes 
and Kontsevich - the evolution of concepts of space and symmetry, 
Bulletin of the AMS, 38 (2001), 389 - 408.  Also available at 
<A HREF = "http://www.ams.org/joursearch/servlet/DoSearch?sendit=Search&f1=msc&v1=&co1=and&f2=title&v2=&co2=and&f3=anywhere&v3=&co3=and&f4=author&v4=cartier&jrnl=all&onejrnl=&startmo=00&startyr=0000&endmo=00&endyr=0000&format=standard&sperpage=30&cperpage=50&ssort=d&cso">http://www.ams.org/joursearch/index.html</A>

For even more mindblowing ideas along these lines:

9) Jack Morava, The motivic Thom isomorphism, talk at the Newton Institute,
December 2002, also available at <A HREF = "http://www.arXiv.org/abs/math.AT/0306151">math.AT/0306151</A>.

\par\noindent\rule{\textwidth}{0.4pt}
\textbf{Addendum:} I received the following email from Avinoam Mann,
which corrects some mistakes I made:

\begin{quote}
Dear John,
It's very nice that you've come to appreciate the beauties of number
theory, and I enjoyed reading your description of Galois theory, but
I hope that you would not mind if I ask you not to help spread some common
misunderstandings about it. First, it was not Galois who proved the
impossibility of solving the quintic by radicals. This was attempted
first by Ruffini, I think in 1799, and the proof by Abel, about ten years
befors Galois, was the one that the mathematical community accepted.
While I often teach Galois theory (e.g. next semester), I never studied
Ruffini's and Abel's work in detail. What Galois did was to give a
criterion checking for an arbitrary equation whether it is soluble by
radicals or not.
Another point: there is no need for Galois theory to prove that
duplication of the cube and trisection of an angle cannot be done by
ruler and compass. Since ruler and compass constructions are equivalent
to solving a series of quadratics, they can lead only to fields F of
dimension 2<sup>n</sup>, for some n, over the rationals. But 
the two problems that
I mentioned lead to extensions of dimension 3. All this is very
elementary. Similar considerations lead to necessary conditions for the
constructibility of regular polygons, but proving these conditions
sufficient does require more theory (unless, I guess, you provide
directly the relevant system of quadratics; I think that is what Gauss
did - his proof also preceded Galois). Squaring the circle is, of course,
a different matter. Here we need the transcendence of &pi;.
Best wishes from wet Jerusalem,
Avinoam Mann
\end{quote}
    

It's true that Abel and Ruffini beat Galois when it came to the quintic; 
the details of this history are covered pretty well by Ian Stewart's book, 
I think.
And, it's quite true that one doesn't need of Galois theory to solve
a bunch of these problems: for example, to show one can't duplicate
the cube, we just need to see that Q(2^{1/3}) has dimension 3
as a vector space over Q, while quadratic extensions have dimension 
2^{n}.   My use of the Galois correspondence to express this in
terms of the size of certain Galois groups was overkill!  The real point of
Galois theory is that it provides a unified framework for tackling a wide
range of problems.  


\par\noindent\rule{\textwidth}{0.4pt}
<em>
Paris, 1 June - A deplorable duel yesterday has deprived the exact 
sciences of a young man who gave the highest expectations, but whose 
celebrated precosity was lately overshadowed by his political activities.  
The young Evariste Galois... was fighting with one of his old friends,
a young man like himself, like himself a member of the Society of
Friends of the People, and who was known to have figured equally in
a political trial.  It is said that love was the cause of the combat.
The pistol was the chosen weapon of the adversaries, but because of
their old friendship they could not bear to look at one another and 
left their decision to blind fate.</em> - Le Precursor, June 4, 1832

<HR>

% </A>
% </A>
% </A>


% parser failed at source line 934
