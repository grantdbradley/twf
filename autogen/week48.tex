
% </A>
% </A>
% </A>
\week{February 26, 1995}
 
There are a few things I've bumped into that I feel I should let folks know 
about, so here's a special issue from Munich, where the Weissbier
is very good.  (And not at all white, but that's another subject.) 

One of the most exciting aspects of mathematics over the last few
years --- in my utterly biased opinion --- has been how topological
quantum field theories have revealed the existence of deep connections
between 3-dimensional topology, complex analysis, and algebra, particularly
the algebra of quantum groups.

The most interesting topological quantum field theory in this game is
Chern-Simons theory.  This a field theory in 3 dimensions, and the
reason it's called "topological" is that you don't need any metric
or other geometrical structure on your 3d spacetime manifold for this 
theory to make sense.  Thus it admits \emph{all} coordinate transformations 
(or more precisely, all diffeomorphisms) as symmetries.  In particular, this
means that the quantity folks like to compute whenever they see a
quantum field theory --- the partition function, which you get by
doing a path integral a la Feynman --- is an invariant of 
3-dimensional manifold you happen to have taken as "spacetime".  

Now computing path integrals is often a very dubious and tricky
business.  They are integrals over the space of all possible histories
of the classical fields corresponding to your quantum field theory.
This is a big fat infinite-dimensional space, of the sort to which ordinary
integration theory doesn't really apply.  If you aren't very careful,
path integrals often give infinite answers.  So one very nice thing is
that, suitably interpreted, the path integrals in Chern-Simons theory
actually make rigorous sense!

A key advance here was Atiyah's axioms for topological quantum field
theories (or TQFTs).  These axioms formalize exactly what properties
you'd hope path integrals would have in the case of a
diffeomorphism-invariant theory.  If, by no matter what devilish
tricks, one can get a theory that satisfies these axioms, it's in many
ways just as good if one had figured out how to make sense of the path
integrals by honest labor, because all the manipulations one would normally
want to do are then permitted.  The marvelous thing about Chern-Simons
theory is that one can show the TQFT axioms hold starting from some 
beautiful algebraic structures called quantum groups.  Corresponding to
every "semisimple Lie group" --- examples being the groups SU(n) of
unitary complex nxn matrices with determinant 1 --- there is a quantum
group, which is not really a group, but a so-called "quasitriangular Hopf 
algebra."  Amusingly these quantum groups really depend on Planck's constant 
\hbar , and reduce to the ordinary groups in the "classical limit" \hbar  \to  0!

Now, where these quantum groups come from has always been a bit of a
puzzle.  They can be rigorously shown to exist, that's for sure.
There are also many good algebraic reasons why they "should" exist.
But it is still a bit remarkable that they have the exactly the
properties needed to get Chern-Simons theory to be a TQFT.  So people
have tried in many ways to turn the tables on them, and get \emph{them}
from the \emph{path integrals}.  Lots of these approaches have succeeded,
but most of them involve a few subtleties here and there, so
mathematicians, who only feel happy when everything is *blindingly
obvious* (to them, that is, not you), have continued to seek the most
beautiful, elegant way of getting at them.

1) Quantum groups from path integrals, by Daniel Freed, preprint,
41 pages in AMSTeX 2.1 format available as <A HREF = "http://xxx.lanl.gov/abs/q-alg/9501025">q-alg/9501025</A>.

This is a nice expository treatment of the work of Free and Quinn 
on topological quantum field theories, particularly Chern-Simons theory
with finite gauge group.  In this case, the path integral reduces to 
a finite sum and one really can get the quantum group from the path 
integral very beautifully.  But there are some differences in this
case of finite gauge group.  For example, the resulting "finite quantum 
group" does \emph{not} depend on \hbar ; it's just the "quantum double" of the
group algebra of the group.  For more on what this has to do with marvelous
algebraic things like n-categories, the reader should check out the paper 
by Freed cited in <A HREF = "week12.html">week12</A>, which has subsequently been published:

2) Higher algebraic structures and quantization, by Daniel Freed,
Comm. Math. Phys. 159 (1994), 343-398.

and also

3) Chern-Simons theory with finite gauge group, by Daniel Freed and
Frank Quinn, Comm. Math. Phys. 156 (1993), 435-472.


Now in addition to the path-integral approach to quantum field theory there 
is another, the so-called Hamiltonian approach, which is very much like
the approach people usually learn in a first course on quantum mechanics:
if you know the wavefunction of your system at t = 0, the Hamiltonian tells
you how it evolves in time from then on.  Now this has special subtleties
in diffeomorphism-invariant theories.  When there is no unique best 
coordinate system, there's no unique best notion of "time evolution".
This leads to the so-called "problem of time", very important in 
quantum gravity, but rather easier to deal with in toy models like
Chern-Simons theory.  

Now if we take a 3-dimensional spacetime and look at the t = 0
slice, we will with some luck get a 2-dimensional manifold, such
as a sphere, torus, or more general n-holed doughnut.  This is where
the complex analysis comes in, because the complex plane is 2-dimensional,
and we can cover these surfaces with coordinate systems that look
locally like the complex plane, making them into "Riemann surfaces"
upon which we can merrily proceed with complex analysis.  In particular,
the phase space of classical Chern-Simons theory is something of
which complex analysts have long been enamored, namely the "moduli
space of flat bundles" over our Riemann surface.  (Let me reassure
physicists that this "flat" business is merely a weird way of saying
that in Chern-Simons theory the analog of the magnetic field vanishes.)

Now starting from the description of the classical phase space for
Chern-Simons theory one should be able to get ahold of the quantum
theory by some "quantization" business just as one does in elementary
quantum mechanics, where the "classical phase space" is the space
of p's and q's, and to quantize one merely decrees that these no
longer commute:


$$

                        pq - qp = -i \hbar .
$$
    

So one should be able to get ahold of quantum groups this way too:
starting with the "moduli space of flat bundles" and "quantizing" it.
I had long why nobody did this in the way which seemed most tempting
and plausible to me.  To lapse into jargon for a bit here, since the
quantum group is obtained from the group by deformation quantization,
where the Poisson structure of the group itself is described by some
"classical r-matrices", and since Chern-Simons theory is in some sense
obtained by quantizing the moduli space, where the Poisson structure
was explicitly described by Goldman, it seemed natural to me that
somehow the classical r-matrices should be \emph{derivable} from Goldman's
formulas.  But after a few wimpy tries at figuring it out, with
not much success, I gave up.  Luckily it turns out someone else 
succeeded nicely, as I found out in a talk by Alekseev here at the
Mathematisches Institut of the Universitaet Muenchen:

4) Poisson structures on moduli of flat connections on Riemann
surfaces and r-matrices, V. V. Fock and A. A. Rosly, preprint ITEP 72-92,
June 1992, Moscow.

They figured out a beautiful formula relating the classical r-matrices
and the Poisson structure on moduli space.  Using this, Alekseev, 
Grosse, and Schomerus were able to get at quantum groups quite
directly from deformation quantization of moduli space, though
there are a few important points left to nail down:

5) Combinatorial quantization of the Hamiltonian Chern-Simons theory,
I \text{\&}  II, by Yu. Alekseev, H. Grosse, and V. Schomerus, <A HREF = "http://xxx.lanl.gov/abs/hep-th/9403066">hep-th/9403066</A>
and <A HREF = "http://xxx.lanl.gov/abs/hep-th/9408097">hep-th/9408097</A>.  

In fact Schomerus told me about this while I was in Cambridge Mass.
over Christmas, but somehow I didn't pick up on the coolest thing about it, 
that it gets the quantum groups quite naturally from Chern-Simons theory,
via the relation between the Poisson brackets on moduli space and
the classical r-matrices.

Let's see.  This is getting technical so I'll give in and get
technical.  One other paper that I found out about while here studies
rather similar issues, but from the viewpoint of geometric
quantization rather than deformation quantization.  Axelrod, Della Pietra and
Witten did some very fundamental work on geometric quantization of
Chern-Simons theory:

6) Geometric quantization of Chern-Simons gauge theory, S. Axelrod,
S. Della Pietra and E. Witten, Jour. Diff. Geom. 33 (1991), 787-902.

but this only treated the case Riemann surfaces, not the Riemann
surfaces with punctures that you need to think about when sticking
\emph{knots} in your 3-manifold.  Martin Schottenloher, one of the professors,
here, gave me a nice review of the work of one of his students on the
case with punctures:

7) Metaplectic quantization of the moduli space of flat and parabolic
bundles (after Peter Scheinhost), in Public. I. R. M. A. Strasbourg,
45 (1993), 43-70.

This uses a lot of complex geometry that's beyond my ken, but one
very exciting remark penetrated my thick skull, namely that you
really need to take into account the so-called "metaplectic correction", 
as one usually does in geometric quantization, and that in the case
of no punctures, this has the sole effect of accomplishing the
mysterious "level shift" (k \to  k + N) that pervades SU(N) Chern-Simons
theory.  (Of course, I bet this must be what's going on for other gauge
groups too.)  Also, when there are punctures, you apparently really need
the metaplectic correction to get the right answers from geometric
quantization; the ad hoc level shift ain't enough.

Well, I have some other things, but this issue seems more or less
devoted to Chern-Simons theory so far, and only Chern-Simons freaks
are likely to have read this far, so I'll put off the rest for later.
\par\noindent\rule{\textwidth}{0.4pt}

% </A>
% </A>
% </A>
