
% </A>
% </A>
% </A>
\week{February 16, 2003 }




As Bush prepares to bomb Hussein's weapons of mass destruction into
oblivion, along with an unknown number of Iraqis, I've been reading
reading biographies of nuclear physicists, concentrating on that
exciting and terrifying age when they discovered quantum mechanics, 
the atomic nucleus, the neutron, and fission.  I started with this book
about Lise Meitner:

1) Ruth Sime, Lise Meitner: A Life in Physics, University of California
Press, 1997.

Meitner's life was a fascinating but difficult one.  The Austrian
government did not open the universities to women until 1901, when she
was 23.  They had only opened high schools to women in 1899, but
luckily her father had hired a tutor to prepare her for the university
before it opened, so she was ready to enter as soon as they let her in.
She decided to work on physics thanks in part to the enthralling
lectures and friendly encouragment of Ludwig Boltzmann.  After getting
her doctorate in 1906, she went to Berlin to work with Max Planck.  At
first she found his lectures dry and a bit disappointing compared to
Boltzmann's, but she soon saw his ideas were every bit as exciting, and
came to respect him immensely.

In Berlin she also began collaborating with Otto Hahn, a young chemist
who was working on radioactivity.  Since women were not allowed in the
chemistry institute, supposedly because their hair might catch fire,
she had to perform her experiments in the basement for two years until
this policy was ended.  Even then, she did not receive any pay at all
until 1911!  But gradually her official status improved, and by 1926 she
became the first woman physics professor in Germany.

Meitner was one of those rare physicists gifted both in theory and
experiment; her physics expertise meshed well with the analytical
chemistry skills of Hahn, and as a team they identified at least nine
new radioisotopes.  The most famous of these was the element
protactinium, which they discovered and named in 1918.  This was the
long-sought "mother of actinium" in this decay series:


\begin{verbatim}

92 uranium 235    
     |
     | &alpha; decay: 7.1 x 10^8 years    
     |
     v
90 thorium 231 
     |
     | &beta; decay: 25.5 hours             
     |                                  
     v
91 protactinium 231                       
     |                                
     | &alpha; decay: 3.25 x 10^4 years 
     |
     v
89 actinium 227
     |
     | &beta; decay: 21.8 years
     |
     v
90 thorium 227
     |
     | &alpha; decay: 18.2 days
     |
     v
88 radium 223 
     |
     | &alpha; decay: 11.4 days
     |
     v
88 radon 219
     |
     | &alpha; decay: 4 seconds 
     |
     v
84 polonium 215
     |
     | &alpha; decay: 1.78 milliseconds
     |
     v
82 lead 211
     |
     | &beta; decay: 36.1 minutes
     |
     v
83 bismuth 211 
     |
     | &alpha; decay: 2.13 minutes
     |
     v
81 tellurium 207
     |
     | &beta; decay: 4.77 minutes
     |
     v
82 lead 207 

\end{verbatim}
    
As you can see by staring at the numbers, in "&alpha; decay" a
nucleus emits a helium nucleus or "&alpha; particle" - 2 protons
and 2 neutrons - so its atomic number goes down by 2 and its atomic mass
goes down by 4.  In "&beta; decay" a neutron decays into a
proton and emits a neutrino and an electron, or "&beta;
particle", so its atomic number goes up by 1 and its mass stays
almost the same.

But to understand Meitner's work in context, you have to realize that
these facts only became clear through painstaking work and brilliant
leaps of intuition.  Much of the work was done by her team in Berlin,
Pierre and Marie Curie in France, Ernest Rutherford's group in
Manchester and later Cambridge, and eventually Enrico Fermi's group in
Rome.


At first people thought electrons were bound in a nondescript jelly of
positive charge - Thomson's "plum pudding" atom.  Even when
Rutherford, Geiger and Marsden shot &alpha; particles at atoms in 1909 and
learned from how they bounced back that the positive charge was
concentrated in a small "nucleus", there remained the puzzle
of what this nucleus was.


In 1914 Rutherford referred to the hydrogen nucleus as a "positive
electron".  In 1920 he coined the term "proton".  But the
real problem was that nobody knew about neutrons!  Instead, people
guessed that the nucleus consisted of protons and "nuclear
electrons", which made its charge differ from the atomic mass.  Of
course, it was completely mysterious why these nuclear electrons should
act different from the others: as Bohr put it, they showed a
"remarkable passivity".  They didn't even have any spin
angular momentum!  But on the other hand, they certainly seemed to exist
- since sometimes they would shoot out in the form of &beta; radiation!

To solve this puzzle one needed to postulate a neutral particle as heavy
as a proton and invent a theory of &beta; decay in which this particle could
decay into a proton while emitting an electron.  But there was an
additional complication: unlike &alpha; radiation, which had a definite
energy, &beta; radiation had a continuous spectrum of energies.  Meitner
didn't believe this at first, but eventually her careful experiments
forced her and everyone else to admit it was true.  The energy
bookkeeping just didn't add up properly.

This led to a crisis in nuclear physics around 1929.  Bohr decided that
the only way out was a failure of conservation of energy!  Pauli thought
of a slightly less radical way out: in &beta; decay, maybe some of the
energy is carried off by yet another neutral particle, this time one of
low mass.  Two mysterious unseen neutral particles was a lot to stomach!
In 1931 Fermi called the big one the "neutron" and the little
one the "neutrino".  In 1932 Chadwick realized that you could
create beams of neutrons by hitting beryllium with &alpha; particles.  The
neutrino was only seen much later, in the 1950s.

(I hope people remember this story when they scoff at the notion that
"dark matter" makes up most of the universe: even if something
is hard to see, it might still exist.)


As a physicist, Ruth Sime is good at conveying in her book not only 
the excitement but also the technical details of Meitner's detective
work.  At first, most of this work involved studying three different
decay series.  The one I drew above is called the "actinium
series": starting with uranium 235, it hopscotches around the
period table until it lands stably on lead 207.  Since both &alpha; and
&beta; decay conserve the atomic mass modulo 4, all elements in the
actinium series have atomic mass equal to 3 mod 4.  Similarly,
elements in the "uranium series" starting with uranium 238
have atomic mass equal to 2 mod 4.  Elements in the "thorium
series" starting with thorium 232 have atomic mass equal to 0
mod 4.


These decay series bring back nostalgic memories for me, since as a kid
I learned about them, and a lot of other stuff, from my dad's old CRC
Handbook of Chemistry and Physics.  This was small compared to more
recent editions.  But it was squat, almost thick as it was tall, bound
in red, with yellowing pages, and it contained more math than they do
these days - I think I learned trigonometry from that thing!  I believe
it was the 1947 edition, which makes sense, since my father studied
chemistry on the GI bill after serving as a soldier in World War II.
The radioisotopes still had their quaint old names, like
"mesothorium", "radiothorium", "brevium",
and "thoron".  Alas, my mother eventually threw this handbook
away in one of her housecleaning purges.

Anyway, all three of these decay series are best visualized using
2-dimensional pictures:

2) Argonne National Laboratory, Natural decay series,
<A HREF = "http://www.ead.anl.gov/pub/doc/NaturalDecaySeries.pdf">http://www.ead.anl.gov/pub/doc/NaturalDecaySeries.pdf</A>

But I know what the mathematicians out there are wondering: what about
the atomic mass equal to 1 mod 4?

This is the "neptunium series".  It is somewhat less important
than the rest, since it involves elements that are less common in
nature.

Hmm.  I don't know about you, but when I hear an answer like that, I
just want to ask more questions!  WHY are the elements in the neptunium
series less common?  Because they're less stable: none has a halflife
exceeding 10 million years except for bismuth 209, the stable endpoint
of this series.  WHY are they less stable?  Maybe this has something to
do with the 1 mod 4, but I'm not enough of a nuclear physicist to know.
Thanks to Pauli exclusion, elements are more stable when they have
either an even number of neutrons, an even number of protons, or better
yet both.  In general I guess this makes elements with atomic mass 0
mod 4 the most stable, followed by those with atomic mass 2 mod 4.
But why is 1 mod 4 less happy than 3 mod 4?  Dunno.

Back to Meitner:

When Hitler gained power over Germany in 1933, her life became
increasingly tough, especially because she was a Jew.  In May of that
year, Nazi students at her university set fire to books by undesirable
writers such as Mann, Kafka, and Einstein.  By September, she received 
a letter saying she was dismissed from her professorship.  Nonetheless,
she continued to do research.  

In 1934, Fermi started trying to produce "transuranics" -
elements above uranium - by firing neutron beams at uranium.  Meitner
got excited about this and began doing the same with Hahn and another
chemist, Fritz Strassman.  They seemed to be succeeding, but the results
were strange: the new elements seemed to decay in many different ways!
Their chemical properties were curiously variable as well.  And the more
experiments the team did, the stranger their results got.

No doubt this is part of why Meitner took so long to flee Germany.
Another reason was her difficulty in finding a job.  For a while she was
protected somewhat by her Austrian citizenship, but that ended in 1938
when Hitler annexed Austria.  After many difficulties, she found an
academic position in Stockholm and managed to sneak out of Germany using
a no-longer-valid Austria passport.


She was now 60.  She had been the head of a laboratory in Berlin,
constantly discussing physics with all the top scientists.  Now she was
in a country where she couldn't speak the language.  She was given a
small room to use a lab, but essentially no equipment, and no
assistants.  She started making her own equipment.  Hahn continued work
with Strassman in Berlin, and Meitner attempted to collaborate from
afar, but Hahn stopped citing her contributions, for fear of the Nazis
and their hatred of "decadent Jewish scence".  Meitner
complained about this to him.  He accused her of being unsympathetic to
\emph{his} plight.  It's no surprise that she wrote to him:

\begin{quote}
 "Perhaps you cannot fully appreciate how unhappy it makes me to 
 realize that you always think I am unfair and embittered, and that 
 you also say so to other people.  If you think it over, it cannot 
 be difficult to understand what it means to me that I have none of 
 my scientific equipment.  For me that is much harder than everything 
 else.  But I am really not embittered - it is just that I see no real 
 purpose in my life at the moment and I am very lonely...."

\end{quote}

What \emph{is} a surprise is that this is when she made her greatest
discovery.  She couldn't bear spending the Christmas of 1938 alone, so
she visited a friend in a small seaside village, and so did her nephew
Otto Frisch, who was also an excellent physicist.  They began talking
about physics.  According to letters from Hahn and Strassman, one of the
"transuranics" was acting a lot like barium.  Talking over the
problem, Meitner and Frisch realized what was going on: the neutrons
were making uranium nuclei \emph{split} into a variety of much lighter
elements!

In short: fission.


I won't bother telling the story of all that happened next: their
calculations and experiments confirming this guess, the development of
the atomic bomb, which Meitner refused to participate in, how Meitner
was nonetheless hailed as the "Jewish mother of the bomb" when
she came to America in 1946, and how Hahn alone got the Nobel prize for
fission, also in 1946.  It's particularly irksome how Hahn seemed to
claim all the credit for himself in his later years.  But history has
dealt him a bit of poetic justice.  Element 105 was tentatively called
"hahnium" by a team of scientists at Berkeley who produced it,
but later, the International Union of Pure and Applied Chemistry decreed
that it be called "dubnium" - after Dubna, where a Russian
team also made this element.  To prevent confusion, no other element can
now be called "hahnium".  But element 109 is called
"meitnerium".

It's a fascinating story.  But it's just one of many fascinating stories
from this age, all of which interweave.  After reading about Meitner, I
started reading these other books:

3) Emilio Segre, Enrico Fermi: Physicist, U. of Chicago Press, Chicago,
1970.

4) Abraham Pais, Niels Bohr's Times: in Physics, Philosophy and Polity,
Oxford U. Press, Oxford, 1991.

5) The Neutron and the Bomb: a Biography of Sir James Chadwick, Oxford
U. Press, Oxford, 1997.

Taken together, they provide a pretty good view of that age in physics.
There are also, of course, lots of books focusing on the Manhattan 
Project.   For a website on Meitner, try:

6) Lise Meitner online, <A HREF = "http://www.users.bigpond.com/Sinclair/fission/LiseMeitner.html">http://www.users.bigpond.com/Sinclair/fission/LiseMeitner.html</A>

For better and worse, fundamental physics is much less dramatic now.  We
are not in a time when developments in basic physics rush towards
earth-shattering new technologies.  Instead we are stuck pondering hard
questions... like quantum gravity.


In "<A HREF = "week189.html">week189</A>", I mentioned some
new ideas about the "quantum of area", and how Dreyer has made
some progress reconciling loop quantum gravity with Hod's argument that
the smallest possible nonzero area is 4 ln 3 times the square of the
Planck length.  You may recall that Hod's work relied on some numerical
computations: they gave the answer 4 ln 3 up to six significant figures,
but nobody knew what the next decimal place would bring!

Since then, a lot has happened.  Most importantly, Lubos Motl has shown
(not rigorously, but convincingly) that the agreement is indeed exact:

7) Lubos Motl, An analytical computation of asymptotic Schwarzschild
quasinormal frequencies, available at <A HREF = "http://xxx.lanl.gov/abs/gr-qc/0212096">gr-qc/0212096</A>.

Alejandro Corichi has tried to explain why Dreyer's work using SO(3)
loop quantum gravity is consistent with the existence of spin-1/2
particles:

8) Alejandro Corichi, On quasinormal modes, black hole entropy, and
quantum geometry, available at <A HREF = "http://xxx.lanl.gov/abs/gr-qc/0212126">gr-qc/0212126</A>.

Personally I must admit I'm not convinced yet.  

Motl and Neitzke have investigated what happens with black holes in
higher dimensions:

9) Lubos Motl and Andrew Neitzke, Asymptotic black hole quasinormal 
frequencies, available at <A HREF = "http://xxx.lanl.gov/abs/hep-th/0301173">hep-th/0301173</A>.

Also, Hod has generalized his work to rotating black holes:

10) Shahar Hod, Kerr black hole quasinormal frequencies, 
available at <A HREF = "http://xxx.lanl.gov/abs/gr-qc/0301122">gr-qc/0301122</A>.

I won't explain any of these new developments here, since I've written
two articles explaining them - a less technical one and a more technical
one - and you can get both on my webpage:

11) John Baez, The quantum of area?, Nature 421 (Feb. 13 2003), 702-703.
   
John Baez, Quantization of area: the plot thickens, to appear in 
Spring 2003 edition of Matters of Gravity at <A HREF = "http://www.phys.lsu.edu/mog/">http://www.phys.lsu.edu/mog/</A>
Both also available at <A HREF = "http://math.ucr.edu/home/baez/area.html">http://math.ucr.edu/home/baez/area.html</A>

Anyway, it's fascinating, and puzzling, and frustrating subject!

Now for some math.  I've been talking about operads a little
bit lately, and now I want to connect them to Jordan algebras.  

People often say: to understand Lie algebras, start with an associative
algebra and see what you can do just with the operation


\begin{verbatim}

[X,Y] = XY - YX 
\end{verbatim}
    
What identities must this always satisfy, regardless of the associative 
algebra you started with?  It turns out that all the identities are 
consequences of just two:


\begin{verbatim}

[X,Y] = -[Y,X]                                    ANTISYMMMETRY

[X,[Y,Z]] = [[X,Y],Z] + [Y,[X,Z]]                 JACOBI IDENTITY
\end{verbatim}
    
together with the fact that the bracket is linear in each slot.
Thus we make these identities into the definition of a Lie algebra.  

People also say: to understand Jordan algebras, start with an 
associative algebra and see what you can do with just 1 and the 
operation


\begin{verbatim}

X o Y = XY + YX 
\end{verbatim}
    
This looks very similar; the only difference is a sign!  
But it's
harder to find all the identities this operation must satisfy.
Actually, if you don't mind, I think I'll switch to the more commonly
used normalization

\begin{verbatim}

X o Y = (XY + YX)/2
\end{verbatim}
    

Two of the identities are obvious:


\begin{verbatim}

1 o X = X                                         UNIT LAW

X o Y = Y o X                                     COMMUTATIVITY 
\end{verbatim}
    
The next one is less obvious:


\begin{verbatim}

X o ((X o X) o Y) = (X o X) o (X o Y)             JORDAN IDENTITY
\end{verbatim}
    
At this point, Pascual Jordan quit looking for more and made these 
his definition of what we now call a "Jordan algebra":

12) Pascual Jordan, Ueber eine Klasse nichtassociativer hyperkomplexer 
Algebren, Nachr. Ges. Wiss. Goettingen (1932), 569-575.   

He wrote this paper while pondering the foundations of quantum theory,
since bounded self-adjoint operators on a Hilbert space represent
observables, and they're closed under the product ab + ba.

Later, with Eugene Wigner and John von Neumann, he classified all
finite-dimensional Jordan algebras that are "formally real", meaning
that a sum of terms of the form X o X is zero only if each one is zero.
This condition is reasonable in quantum mechanics, because observables
like X o X are "positive".  It also leads to a nice classification,
which I described in "<A HREF = "week162.html">week162</A>".

Interestingly, one of these formally real Jordan algebras doesn't
sit inside an associative algebra: the "exceptional Jordan algebra", 
which consists of all 3x3 hermitian matrices with octonion entries.

This algebra has lots of nice properties, and it plays a mysterious 
role in string theory and some other physics theories.  This is the 
main reason I'm interested in Jordan algebras, but I've said plenty 
about this already; now I want to focus on something else.

Namely: did Jordan find all the identities?  

More precisely: if we set X o Y = (XY + YX)/2, can all the identities 
satisfied by this operation in every associative algebra be derived 
from the above 3 and the fact that this operation is linear in each slot?

This was an open question until 1966, when Charles M. Glennie found 
the answer is \emph{no}.


It's a bit like Tarski's "high school algebra problem", where
Tarksi asked if all the identities involving addition, multiplication
and exponentiation which hold for the positive the natural numbers
follow from the ones we all learned in high school.  Here too the answer
is \emph{no} - see "<A HREF = "week172.html">week172</A>"
for details.  That really shocked me when I heard about it!  Glennie's
result is less shocking, because Jordan algebras are less
familiar... and the Jordan identity is already pretty weird, so maybe we
should expect other weird identities.

It's easiest to state Glennie's identity with the help of
the "Jordan triple product"


\begin{verbatim}

{X,Y,Z} = (X o Y) o Z + (Y o Z) o X - (Z o X) o Y
\end{verbatim}
    
Here it is:


\begin{verbatim}

     2{{Y, {X,Z,X},Y}, Z, XoY} - {Y, {X,{Z,XoY,Z},X}, Y}
    -2{XoY, Z, {X,{Y,Z,Y},X}} + {X, {Y,{Z,XoY,Z},Y}, X} = 0 
\end{verbatim}
    
Blecch!  It makes you wonder how Glennie found this, and why.

I don't know the full story, I know but Glennie was a Ph.D. student of
Nathan Jacobson, a famous algebraist and expert on Jordan algebras.  I'm
sure that goes a long way to explain it.  He published his result here:

13) C. M. Glennie, Some identities valid in special Jordan
algebras but not in all Jordan algebras, Pacific J. Math. 16
(1966), 47-59.

Was this identity the only extra one?

Well, I'm afraid the title of the paper gives that away: in addition to
the above identity of degree 8, Glennie also found another.  In fact
there turn out to be \emph{infinitely} many identities that can't be derived
from the previous ones using the Jordan algebra operations.

As far as I can tell, the full story was discovered only in the 1980s.
Let me quote something by Murray Bremner.  It will make more sense if
you know that the identities we're after are called "s-identities",
since they hold in "special" Jordan algebras: those coming from
associative algebras.  Here goes:

\begin{quote}
   Efim Zelmanov won the Fields Medal at the International Congress of
   Mathematicians in Zurich in 1994 for his work on the Burnside Problem
   in group theory. Before that he had solved some of the most important
   open problems in the theory of Jordan algebras.  In particular he
   proved that Glennie's identity generates all s-identities in the
   following sense: if G is the T-ideal generated by the Glennie
   identity in the free Jordan algebra FJ(X) on the set X (where X has
   at least 3 elements), then the ideal S(X) of all s-identities is
   quasi-invertible modulo G (and its homogeneous components are nil
   modulo G) [....] Roughly speaking, this means that all other
   s-identities can be obtained by substituting into the Glennie
   identity, generating an ideal, extracting n-th roots, and summing up.

\end{quote}
This is a bit technical, but basically it means you need to expand your
arsenal of tricks a bit before Glennie's identity gives all the rest.
The details can be found in Theorem 6.7 here:

14) Kevin McCrimmon, Zelmanov's prime theorem for quadratic Jordan 
algebras, Jour. Alg. 76 (1982), 297-326. 

and I got the above quote from a talk by Bremner:

15) Murray Bremner, Using linear algebra to discover the defining
identities for Lie and Jordan algebras, available at 
<A HREF = "http://web.archive.org/web/20030324024322/http://math.usask.ca/~bremner/research/colloquia/calgarynew.pdf">http://web.archive.org/web/20030324024322/http://math.usask.ca/~bremner/research/colloquia/calgarynew.pdf</a>

Now, the Jordan triple product


\begin{verbatim}

{X,Y,Z} = (X o Y) o Z + (Y o Z) o X - (Z o X) o Y
\end{verbatim}
    
may at first glance seem almost as bizarre as Glennie's identity,
but it's not!  To understand this, it helps to think about "operads".

I defined operads last week.  Very roughly, these are gadgets 
that for each n have a set O(n) of abstract n-ary operations:



\begin{verbatim}

                  \    |    /
                   \   |   / 
                    \  |  /
                     -----
                    |     | 
                     -----
                       |
                       |
\end{verbatim}
    
together with ways to compose them, like this:



\begin{verbatim}

         \    /     \  |  /       | 
          \  /       \ | /        |
          -----      -----      -----
         |     |    |     |    |     |
          -----      -----      -----
              \        |        /  
               \       |       /
                \      |      /
                 \     |     /
                  \    |    / 
                   \   |   /  
                    \  |  /
                     -----
                    |     | 
                     -----
                       |
                       |
\end{verbatim}
    
Given an operad O, an "O-algebra" is, again very roughly, a set S on
which each element of O(n) is represented as an actual n-ary operation:
that is, a function from S^{n} to S.

Now, all of this also works if we replace the sets by vector spaces,
functions by linear operators, and the Cartesian product by the tensor
product.  We then have "linear operads", whose algebras are vector
spaces equipped with multilinear operation.

For example, there's linear operad called Commutative, whose algebras
are precisely commutative algebras.  Get it?  O-algebras with O = 
Commutative are commutative algebras!  This is the sort of joke that 
has stuffy old professors rolling on the floor with laughter.

There's also a linear operad called Associative whose algebras are 
precisely associative algebras, and a linear operad Lie whose elements 
are Lie algebras, and a linear operad Jordan whose elements are Jordan 
algebras.  

This last fact seems to be my own personal observation, made in
discussion with James Dolan.  The Lie operad is well-known, but I've
never heard of anyone talk about the Jordan operad!  What follows is
some related stuff that we came up with:

The operads Lie is the suboperad of Associative generated by the
binary operation


\begin{verbatim}

[X,Y] = XY - YX 
\end{verbatim}
    
Similarly, we can try to get the operad Jordan by taking the suboperad 
of Associative generated by the binary operation


\begin{verbatim}

X o Y = XY + YX
\end{verbatim}
    
and the nullary operation 


\begin{verbatim}

1
\end{verbatim}
    
However, there's a problem: the operations in this suboperad will
satisfy not just the identities for a Jordan algebra, but also the
"s-identities" that hold when you have a Jordan algebra that came 
from an associative algebra.  So, this suboperad should be called
"SpecialJordan".  To get Jordan, we have to \emph{throw out} 
all the
s-identities.  But mathematically, unlike the process of putting
in extra relations, it's a bit irksome to describe the process of
"throwing out" relations. 

This makes Jordan algebras seem like just a defective version of special
Jordan algebras.  However, there are other things which are really
\emph{good} about Jordan algebras... so I still think there should be some
nice way to characterize the operad Jordan.

For that matter, I think there's a nicer way to characterize the operad
SpecialJordan!  Here's a little conjecture.  The operad Associative has
an automorphism


$$

R: Associative \to  Associative
$$
    

which "reverses" any operation.  For example, if we take the
operation sending (W,X,Y,Z) to the product YWXZ, and hit it with R, we
get the operation sending (W,X,Y,Z) to ZXWY.  Now, the fixed points of
an operad automorphism always form a suboperad.  So, the fixed points of
R form a suboperad of Associative... and I conjecture that this is
SpecialJordan.

In other words, summarizing a bit crudely: I think the Jordan algebra
operations are just the associative algebra operations that are
"palindromes" - their own reverses.

Let's check and see.  The nullary operation


\begin{verbatim}

1
\end{verbatim}
    
is a palindrome and it's a Jordan algebra operation.  The unary
operation


\begin{verbatim}

X
\end{verbatim}
    
is a palindrome and it's a Jordan algebra operation: as I mentioned last
week, this "identity operation" is in \emph{every} 
operad, by definition.
The binary operation


\begin{verbatim}

XY + XX
\end{verbatim}
    
is a palindrome, and it's just the Jordan product!  So far so good.  
But what about


\begin{verbatim}

XYX ?
\end{verbatim}
    
Well... this ain't even an operation in Associative, because it's not
linear in each argument!  Ha!  I was just testing you.  But it's not a
complete hoax: to get something sensible, we can take XYX and pull a
trick called "polarization": replace X by X+Z, then replace it
by X-Z, and then subtract the two to get something linear in X, Y, and
Z:


\begin{verbatim}

(X+Z)Y(X+Z) - (X-Z)Y(X-Z) = 2(XYZ + ZYX)
\end{verbatim}
    
This is a ternary operation in Associative that's a palindrome.
But is it a Jordan algebra operation?  

\emph{Yes}, by the following identity:


\begin{verbatim}

(XYZ + ZYX)/2 = (X o Y) o Z + (Y o Z) o X - (Z o X) o Y    
\end{verbatim}
    
In fact, this is just the "Jordan triple product" I was
talking about earlier:


\begin{verbatim}

{X,Y,Z} = (XYZ + ZYX)/2 

        = (X o Y) o Z + (Y o Z) o X - (Z o X) o Y    
\end{verbatim}
    
So, the Jordan triple product is not as insane as it looks: it 
shows up naturally when we try to express all palindrome operations 
in terms of the Jordan product!

I leave it to the energetic reader to continue checking this conjecture.

If I had more energy myself, I would now bring Jordan triple systems and
Lie triple systems into the game, discuss their relation to geometry and
physics, and other nice things.  But I'm too tired!  So, I'll just leave
off by mentioning that Bremner has invented a q-deformed version of the
octonions:

16) Murray Bremner, Quantum octonions, Communications in Algebra 
27 (1999), 2809-2831, also available at 
<A HREF = "http://math.usask.ca/~bremner/research/publications/qo.pdf">http://math.usask.ca/~bremner/research/publications/qo.pdf</A>

However, he did it using the representation theory of quantum sl(2).
These folks define a \emph{different} q-deformation of the octonions using
the representation theory of quantum so(8):

17) Georgia Benkart, Jose M. Pirez-Izquierdo, A quantum octonion 
algebra, Trans. Amer. Math. Soc. 352 (2000), 935-968, also available at 
<A HREF = "http://www.arXiv.org/abs/math.QA/9801141">math.QA/9801141</A>.

I find that a bit more tempting, since the ordinary octonions arise from
triality: the outer automorphism relating the three 8-dimensional irreps
of so(8).  I don't know how (or whether) these quantum octonions are
related to the 7-dimensional representation of quantum G2, which could
be called the "quantum imaginary octonions" and was studied by Greg
Kuperberg:

18) Greg Kuperberg, The quantum G_{2} link invariant, Internat. 
J. Math. 5 (1994) 61-85, also available with some missing diagrams at
<A HREF = "http://www.arXiv.org/abs/math.QA/9201302">math.QA/9201302</A>.

(I thank Sean Case, Alejandro Corichi, Rob Johnson and Bruce Smith
for helping me correct some errors in this Week's Finds.)  
\par\noindent\rule{\textwidth}{0.4pt}
<em>"I believe all young people think about how they would like their
lives to develop; when I did so, I always arrived at the conclusion
that life need not be easy, provided only that it is not empty. 
That life has not always been easy - the first and second World
Wars and their consequences saw to that - while for the fact that it has
indeed been full, I have to thank the wonderful developments of
physics during my lifetime and the great and lovable personalities
with whom my work in physics brought me contact."</em> - Lise Meitner


\par\noindent\rule{\textwidth}{0.4pt}

% </A>
% </A>
% </A>
