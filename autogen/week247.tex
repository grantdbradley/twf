
% </A>
% </A>
% </A>
\week{March 23, 2007 }

Symmetry has fascinated us throughout the ages.  Greek settlers 
in Sicily may have seen irregular 12-sided crystals of pyrite in Sicily
and dreamt up the regular dodecahedron simply because it was more 
beautiful, more symmetrical. 

<div align = "center">
<a href = "http://www.minerals.net/mineral/sulfides/pyrite/pyrite.htm">
<img src = "dodecahedron/pyrite.jpg">
% </a>
</div>

The Alhambra, a Moorish palace in Granada
built around 1300, has tile patterns with at least 13 of the 17 possible 
symmetry groups:

1) Branko Gr&uuml;nbaum, What symmetry groups are present in the 
Alhambra?, Notices of the AMS, 53 (2006),
670-673.  Also available at <a href = "http://www.ams.org/notices/200606/comm-grunbaum.pdf">http://www.ams.org/notices/200606/comm-grunbaum.pdf</a>

You can see some of these patterns here:

2) Moresque tiles, <a href = "http://www.spsu.edu/math/tile/grammar/moor.htm">http://www.spsu.edu/math/tile/grammar/moor.htm</a>

Recently, Peter Lu and Paul Steinhardt discovered that Islamic tile 
designs also include "quasicrystals".  A perfectly repetitive tiling
can't have 5-fold symmetry.  Nor can a 3-dimensional crystal: that's why
the dodecahedra formed by pyrite aren't regular.  But by using 
patterns that never quite repeat, the Islamic artists
achieved \emph{approximate} 5-fold symmetry:

3) Peter J. Lu and Paul J. Steinhardt, Decagonal and quasi-crystalline
tilings in medieval Islamic architecture, Science 315 (2007),
1106-1110.  

Here's an example from the <a href = "http://archnet.org/library/sites/one-site.tcl?site_id=7544">I'timad al-Daula mausoleum</a> in the Indian city of
Agra, built by Islamic conquerors in 1622 - together with a more 
mathematical version constructed by Lu and Steinhardt:

<div align = "center">
<a href = "http://www.physics.harvard.edu/~plu/publications/Science_315_1106_2007_SOM.pdf">
<img width = "400" style = "border:none;" src = "quasicrystal_I'timid_al-Daula.jpg">
% </a>
</div>

Here's another, from the 
<a href = "http://archnet.org/library/sites/one-site.tcl?site_id=8380">Darb-i Imam shrine</a>
in Isfahan, Iran, also built 
in the 1600s:

<div align = "center">
<a href = "http://www.physics.harvard.edu/~plu/publications/Science_315_1106_2007_SOM.pdf">
<img width = "400" src = "quasicrystal_Darb-i_Imam.jpg">
% </a>
</div>

This came as a big surprise, since everyone had \emph{thought} that the 
math behind quasicrystals was first discovered by Penrose around 1974, 
then seen in nature by Shechtman, Blech, Gratias and Cahn in 1983.
It goes to show that the appeal of symmetry, even in its subtler forms,
is very old!  It also goes to show that you can make big discoveries
just by looking carefully at what's in front of you.


For more on quasicrystals, try this:

4) Steven Webber, Quasicrystals, <a href = "http://www.jcrystal.com/steffenweber/">http://www.jcrystal.com/steffenweber/</a>

Of course, the appeal of symmetry didn't end with ancient Greeks or
medieval Islamic monarchs.  It also seems to have gotten ahold of John
Fry, chief executive of Fry's Electronics - a chain of retail shops
whose motto is "Your best buys are always at Fry's".  In 1994 he 
set up something called the American Institute of Mathematics.  The
headquarters was in a Fry's store in Palo Alto - not very romantic.  
But last year, this institute announced plans to move to a full-scale 
replica of the Alhambra!

<div align = "center">
<a href = "http://aimath.org/about/morganhill.html">
<img "width = 500" src = "aim_alhambra.jpg">
% </a>
</div>

5) Associated Press, Silicon valley will get Alhambra-like castle,
August 18, 2006.  Available at <a href = "http://www.msnbc.msn.com/id/14412387/">http://www.jcrystal.com/steffenweber/</a>
 
And this week, the institute flexed its mighty PR muscles and coaxed 
reporters from the New York Times, BBC, Le Monde, Scientific American, 
Science News, and so on to write about a highly esoteric advance in 
our understanding of symmetry - a gargantuan calculation involving the 
Lie group E_{8}:

6) American Institute of Mathematics, Mathematicians map E_{8}, 
<a href = "http://aimath.org/E8">http://aimath.org/E8</a>

The calculation is indeed huge.  The \emph{answer} takes up 60 gigabytes of
data: the equivalent of 45 days of music in MP3 format.  If this 
information were written out on paper, it would cover Manhattan!

But what's the calculation \emph{about?} It almost seems a good
explanation of that would \emph{also} cover Manhattan.  I took a
stab at it here:

7) John Baez, News about E_{8}, 
<a href = "http://golem.ph.utexas.edu/category/2007/03/news_about_e8.html">http://golem.ph.utexas.edu/category/2007/03/news_about_e8.html</a>

but I only got as far as sketching a description of E_{8} and some gadgets
called R-polynomials.  Then come Kazhdan-Lusztig polynomials, and 
Kazhdan-Lusztig-Vogan polynomials....  For more details, follow
the links, especially to the page written by Jeffrey Adams, who led
the project.

In weeks to come, I'll say more about some topics tangentially related
to this calculation - especially flag varieties, representation theory
and the Weil conjectures.  I may even talk about Kazhdan-Lusztig polynomials!

For starters, though, let's just look at some pretty pictures by John Stembridge
that hint at the majesty of E_{8}.  Then I'll sketch the real subject 
of Weeks to come: symmetry, geometry, and "groupoidification".

To warm up to E_{8}, let's first take a look at D_{4}, D_{5}, E_{6}, and E_{7}.  

In "<A HREF = "week91.html">week91</A>" I spoke about the D_{4}
lattice.  To get this, first take a bunch of equal-sized spheres in 4
dimensions.  Stack them in a hypercubical pattern, so their centers
lie at the points with integer coordinates.  A bit surprisingly,
there's a lot of room left over - enough to fit in another copy of
this whole pattern: a bunch of spheres whose centers lie at the points
with \emph{half-integer} coordinates!

If you stick in these extra spheres, you get the densest known packing
of spheres in 4 dimensions.  Their centers form the "D_{4}
lattice".  It's an easy exercise to check that each sphere
touches 24 others.  The centers of these 24 are the vertices of a
marvelous shape called the "24-cell" - one of the six
4-dimensional Platonic solids.  It looks like this:

<div align = "center">
<a href = "octonions/conway_smith">
<img style = "border:none;" src = "octonions/conway_smith/24_cell.jpg">
% </a>
</div>

8) John Baez, picture of 24-cell, in a review of On Quaternions
and Octonions: Their Geometry, Arithmetic and Symmetry, by John H.
Conway and Derek A. Smith, available at 
<a href = "http://math.ucr.edu/home/baez/octonions/conway_smith/">
http://math.ucr.edu/home/baez/octonions/conway_smith/</a>

Here I'm using a severe form of perspective to project 4 dimensions down 
to 2.  The coordinate axes are drawn as dashed lines; the solid lines are 
the edges of the 24-cell.

How about in 5 dimensions?  Here the densest known packing of spheres
uses the "D_{5} lattice".  This is a lot like the D_{4}
lattice... but only if you think about it the right way.

Imagine a 4-dimensional checkerboard with "squares" - really
hypercubes! - alternately colored red and black.  Put a dot in the
middle of each black square.  Voila!  You get a rescaled version of
the D_{4} lattice.  It's not instantly obvious that this matches my
previous description, but it's true.

If you do the same thing with a 5-dimensional checkerboard, you get 
the "D_{5} lattice", by definition.  This gives the densest known 
packing of spheres in 5 dimensions.  In this packing, each sphere
has 40 nearest neighbors.  The centers of these nearest neighbors 
are the vertices of a solid that looks like this:

<div align = "center">
<a href = "http://www.math.lsa.umich.edu/~jrs/">
<img style = "border:none;" src = "d5_stembridge.jpg">
% </a>
</div>


9) John Stembridge, D_{5} root system, available at
<a href = "http://www.math.lsa.umich.edu/~jrs/data/coxplanes/">http://www.math.lsa.umich.edu/~jrs/data/coxplanes/</a>

If you do the same thing with a 6-dimensional checkerboard, you get
the "D_{6} lattice"... and so on.  

However, in 8 dimensions something cool happens.  If you pack spheres
in the pattern of the D_{8} lattice, there's enough room left
to stick in an extra copy of this whole pattern!  The result is called
the "E_{8} lattice".  It's twice as dense as the
D_{8} lattice.

If you then take a well-chosen 7-dimensional slice through the origin
of the E_{8} lattice, you get the E_{7} lattice.  And
if you take a well-chosen 6-dimensional slice of this, you get the
E_{6} lattice.  For precise details on what I mean by
"well-chosen", see "<A HREF =
"week65.html">week65</A>".

E_{6} and E_{7} give denser packings of spheres than
D_{6} and D_{7}.  In fact, they give the densest known packings
of spheres in 6 and 7 dimensions!

In the E_{6} lattice, each sphere has 72 nearest neighbors.  They form
the vertices of a solid that looks like this:

<div align = "center">
<a href = "http://www.math.lsa.umich.edu/~jrs/">
<img style = "border:none;" src = "e6_stembridge.jpg">
% </a>
</div>

10) John Stembridge, E_{6} root system, available at
<a href = "http://www.math.lsa.umich.edu/~jrs/data/coxplanes/">http://www.math.lsa.umich.edu/~jrs/data/coxplanes/</a>

In the E_{7} lattice, each sphere has 126 nearest neighbors.  They form
the vertices of a solid like this:

<div align = "center">
<a href = "http://www.math.lsa.umich.edu/~jrs/">
<img style = "border:none;" src = "e7_stembridge.jpg">
% </a>
</div>

11) John Stembridge, E_{7} root system, available at
<a href = "http://www.math.lsa.umich.edu/~jrs/data/coxplanes/">http://www.math.lsa.umich.edu/~jrs/data/coxplanes/</a>


In the E_{8} lattice, each sphere has 240 nearest neighbors.  They form
the vertices of a solid like this:

<div align = "center">
<a href = "http://www.math.lsa.umich.edu/~jrs/">
<img style = "border:none;" src = "e8_stembridge_small.jpg">
% </a>
</div>

12) John Stembridge, E_{8} root system, available at
<a href = "http://www.math.lsa.umich.edu/~jrs/data/coxplanes/">http://www.math.lsa.umich.edu/~jrs/data/coxplanes/</a>


Faithful readers will know I've discussed these lattices often before.
For how they give rise to Lie groups, see "<A HREF =
"week63.html">week63</A>".  For more about "ADE
classifications", see "<A HREF =
"week64.html">week64</A>" and "<A HREF =
"week230.html">week230</A>".  I haven't really added much this
time, except Stembridge's nice pictures.  I'm really just trying to
get you in the mood for a big adventure involving all these ideas: the
Tale of Groupoidification!

If we let this story lead us where it wants to go, we'll meet 
all sorts of famous and fascinating creatures, such as:

<ul>
<li>
 Coxeter groups, buildings, and the quantization of logic  
</li>
<li>
 Hecke algebras and Hecke operators 
</li>
<li>
 categorified quantum groups and Khovanov homology 
</li>
<li>
 Kleinian singularities and the McKay correspondence 
</li>
<li>
 quiver representations and Hall algebras 
</li>
<li>
 intersection cohomology, perverse sheaves and Kazhdan-Lusztig theory 
</li>
</ul>
\end{quote}

However, the charm of the tale is how many of these ideas are unified
and made simpler thanks to a big, simple idea: groupoidification.

So, what's groupoidification?  It's a method of exposing the combinatorial 
underpinnings of linear algebra - the hard bones of set theory underlying
the flexibility of the continuum.

Linear algebra is all about vector spaces and linear maps.  One of the 
lessons that gets drummed into you when you study this subject is that 
it's good to avoid picking bases for your vector spaces until you need 
them.  It's good to keep the freedom to do coordinate transformations... 
and not just keep it in reserve, but keep it \emph{manifest!}

As Hermann Weyl wrote, "The introduction of a coordinate system
to geometry is an act of violence".
 
This is a deep truth, which hits many physicists when they study special 
and general relativity.  However, as Niels Bohr quipped, a deep truth is one 
whose opposite is also a deep truth.  There are some situations where
a vector space comes equipped with a god-given basis.  Then it's foolish
not to pay attention to this fact!

The most obvious example is when our vector space has been \emph{defined}
to consist of formal linear combinations of the elements of some set.  
Then this set is our basis.  

This often happens when we use linear algebra to study combinatorics.

But if sets give vector spaces, what gives linear operators?  Your
first guess might be \emph{functions}.  And indeed, functions between sets
do give linear operators between their vector spaces.  For example, 
suppose we have a function

f: {livecat, deadcat} \to  {livecat, deadcat}

which "makes sure the cat is dead":

f(livecat) = deadcat

f(deadcat) = deadcat

Then, we can extend f to a linear operator defined on formal 
linear combinations of cats: 

F(a livecat + b deadcat) = a deadcat + b deadcat

Written as a matrix in the {livecat, deadcat} basis, this looks like


\begin{verbatim}

 0  0

 1  1 
\end{verbatim}
    
(The relation to quantum mechanics here is just a vague hint of
themes to come.  I've deliberately picked an example where the linear
operator is \emph{not} unitary.)

So, we get some linear operators from functions... but not all!  
We only get operators whose matrices have exactly a single 1 in 
each column, the rest of the entries being 0.  That's because a
function f: X \to  Y sends each element of X to a single element of Y.

This is very limiting.  We can do better if we get operators from
\emph{relations} between sets.  In a relation between sets X and Y,
an element of X can be related to any number of elements of Y, and
vice versa.  For example, let the relation

R: {1,2,3,4} &mdash;/\to  {1,2,3,4}

be "is a divisor of".  Then 1 is a divisor of everybody, 2 is a 
divisor of itself and 4, 3 is only a divisor of itself, and 4 is
only a divisor of itself.  We can encode this in a matrix:


\begin{verbatim}

 1 0 0 0
 1 1 0 0
 1 0 1 0
 1 1 0 1
\end{verbatim}
    

where 1 means "is a divisor of" and 0 means "is not a
divisor of".

We can get any matrix of 0's and 1's this way.  Relations are really
just matrices of truth values.  We're thinking of them as matrices of 
numbers.  Unfortunately we're still far from getting \emph{all} matrices
of numbers!  

We can do better if we get matrices from \emph{spans} of sets.  A span of
sets, written S: X &mdash;/\to  Y, is just a set S equipped with functions to
X and Y.  We can draw it like this:


\begin{verbatim}

                     S
                    / \
                   /   \
                 F/     \G
                 /       \
                v         v 
               X           Y
\end{verbatim}
    
This is my wretched ASCII attempt to draw two arrows coming down from
the set S to the sets X and Y.  It's supposed to look like a bridge - 
hence the term "span".  

Spans of sets are like relations, but where you can be related to 
someone more than once!

For example, X could be the set of Frenchman and Y could be the set of
Englishwomen.  S could be the set of Russians.  As you know, every
Russian has exactly one favorite Frenchman and one favorite
Englishwoman.  So, F could be the function "your favorite
Frenchman", and G could be "your favorite Englishwoman".

Then, given a Frenchman x and an Englishwoman y, they're related by 
the Russian s whenever s has x as their favorite Frenchman and y as 
their favorite Englishwoman:

F(s) = x  and  G(s) = y.   

Some pairs (x,y) will be related by no Russians, others will be related 
by one, and others will be related by more than one!  I bet the pair

(x,y) = (G&eacute;rard Depardieu, Emma Thompson)

is related by at least 57 Russians.

This idea lets us turn spans of sets into matrices of natural numbers.
Given a span of finite sets:


\begin{verbatim}

                     S
                    / \
                   /   \
                 F/     \G
                 /       \
                v         v 
               X           Y
\end{verbatim}
    
we get an X \times  Y matrix whose (x,y) entry is the number of Russians - 
I mean elements s of S - such that 

F(s) = x  and  G(s) = y.   

We can get any finite-sized matrix of natural numbers this way.  

Even better, there's a way to "compose" spans that nicely matches the
usual way of multiplying matrices.  You can figure this out yourself if
you solve this puzzle:

\begin{quote}
 Let X be the set of people on Earth.  Let T be the X \times  X matrix
 corresponding to the relation "is the father of".  Why does
 the matrix T^{2} correspond to the relation "is the paternal
grandfather of"?  Let S correspond to the relation "is a friend
 of".  Why doesn't the matrix S^{2} correspond to the
 relation "is a friend of a friend of"?  What span does this
 matrix correspond to?  \end{quote}

To go further, we need to consider spans, not of sets, but of groupoids!

I'll say more about this later - I suspect you're getting tired.
But for now, briefly: a groupoid is a category with inverses.  Any
group gives an example, but groupoids are more general - they're the
modern way of thinking about symmetry.

There's a way to define the cardinality of a finite groupoid:

12) John Baez and James Dolan, From finite sets to Feynman diagrams,
in Mathematics Unlimited - 2001 and Beyond, vol. 1, eds. Bjorn Engquist 
and Wilfried Schmid, Springer, Berlin, 2001, pp. 29-50.  Also available
as <A HREF = "http://xxx.lanl.gov/abs/math.QA/0004133">math.QA/0004133</A>.

And, this can equal any nonnegative \emph{rational} number!  This lets us
generalize what we've done from finite sets to finite groupoids, and
get rational numbers into the game.

A span of groupoids is a diagram


\begin{verbatim}

                     S
                    / \
                   /   \
                 F/     \G
                 /       \
                v         v 
               X           Y
\end{verbatim}
    
where X, Y, S are groupoids and F, G are functors.  If all the groupoids
are finite, we can turn this span into a finite-sized matrix of nonnegative 
rational numbers, by copying what we did for spans of finite sets.  

There's also a way of composing spans of groupoids, which corresponds 
to multiplying matrices.  However, there's a trick involved in getting
this to work - I'll have to explain this later.  For details, try:

13) Jeffrey Morton, Categorified algebra and quantum mechanics, 
Theory and Application of Categories 16 (2006), 785-854.  Available
at 
<a href = "http://www.emis.de/journals/TAC/volumes/16/29/16-29abs.html">http://www.emis.de/journals/TAC/volumes/16/29/16-29abs.html</a>; 
also available as <A HREF = "http://xxx.lanl.gov/abs/math.QA/0601458">math.QA/0601458</A>.

14) Simon Byrne, On Groupoids and Stuff, honors thesis, Macquarie
University, 2005, available at 
<a href = "http://www.maths.mq.edu.au/~street/ByrneHons.pdf">
http://www.maths.mq.edu.au/~street/ByrneHons.pdf</a> and
<a href = "http://math.ucr.edu/home/baez/qg-spring2004/ByrneHons.pdf">http://math.ucr.edu/home/baez/qg-spring2004/ByrneHons.pdf</a>


Anyway: the idea of "groupoidification" is that in many cases where 
mathematicians think they're playing around with linear operators
between vector spaces, they're \emph{actually} playing around with spans of
groupoids!  

This is especially true in math related to simple Lie groups, their
Lie algebras, quantum groups and the like.  While people usually study
these gadgets using linear algebra, there's a lot of combinatorics 
involved - and where combinatorics and symmetry show up, one invariably
finds groupoids.

As the name suggests, groupoidification is akin to categorification.
But, it's a bit different.  In categorification, we try to boost up
mathematical ideas this way:

\begin{quote}
sets \to  categories <br>
functions \to  functors <br>
\end{quote}

In groupoidification, we try this:

\begin{quote}
vector spaces \to  groupoids <br>
linear operators \to  spans of groupoids<br>
\end{quote}

Actually, it's "decategorification" and
"degroupoidification" that are systematic processes.  These
processes lose information, so there's no systematic way to reverse
them.  But, as I explained in "<A HREF =
"week99.html">week99</A>", it's still fun to try!  If we succeed,
we discover an extra layer of structure beneath the math we thought we
understood... and this usually makes that math \emph{clearer} and 
\emph{less technical}, because we're not seeing it through a blurry,
information-losing lens.

Okay, that's enough for now.  On a completely different note, here's
a book on "structural realism" and quantum mechanics:

15) Dean Rickles, Steven French, and Juha Saatsi, The Structural
Foundations of Quantum Gravity, Oxford University Press, Oxford, 
2006.  Containing:

\begin{quote}
Dean Rickles and Steven French, Quantum gravity meets structuralism: 
interweaving relations in the foundations of physics.  Also available at 
<a href = "http://fds.oup.com/www.oup.co.uk/pdf/0-19-926969-6.pdf">http://fds.oup.com/www.oup.co.uk/pdf/0-19-926969-6.pdf</a>

Tian Yu Cao, Structural realism and quantum gravity.

John Stachel, Structure, individuality, and quantum gravity.
Also available as <A HREF = "http://xxx.lanl.gov/abs/gr-qc/0507078">gr-qc/0507078</A>.

Oliver Pooley, Points, particles, and structural realism.  Also
available at <a href = "http://philsci-archive.pitt.edu/archive/00002939/">http://philsci-archive.pitt.edu/archive/00002939/</a>

Mauro Dorato and Massimo Pauri, Holism and structuralism in 
classical and quantum general relativity.  Also available at 
<a href = "http://philsci-archive.pitt.edu/archive/00001606/">http://philsci-archive.pitt.edu/archive/00001606/</a>

Dean Rickles, Time and structure in canonical gravity.  Also
available at <a href = "http://philsci-archive.pitt.edu/archive/00001845/">http://philsci-archive.pitt.edu/archive/00001845/</a>

Lee Smolin, The case for background independence.  Also available
as <A HREF = "http://xxx.lanl.gov/abs/hep-th/0507235">hep-th/0507235</A>.

John Baez, Quantum quandaries: A category-theoretic perspective.
Also available at <a href = "http://math.ucr.edu/home/baez/quantum/">http://math.ucr.edu/home/baez/quantum/</a> and as
<A HREF = "http://xxx.lanl.gov/abs/quant-ph/0404040">quant-ph/0404040</A>.
\end{quote}

Very loosely speaking - I ain't no philosopher - structural realism is
the idea that what's "real" about mathematics, or the
abstractions in physical theories, are not individual entities but the
structures, or patterns, they form.  So, instead of asking tired
questions like "What is the number 2, really?" or "Do
points of spacetime really exist?", we should ask more global
questions about the roles that structures like "natural
numbers" or "spacetime" play in math and physics.  It's
a bit like how in category theory, we can only understand an object in
the context of the category it inhabits.

Finally, here's a puzzle for lattice and Lie group fans.  The dots 
in Stembridge's pictures are the shortest nonzero vectors in the D_{5}, 
E_{6}, E_{7}, and E_{8} lattices - or in technical terms, the "roots".  Of
course, only for ADE Dynkin diagrams are the roots all of equal length -
but those are the kind we have here.  Anyway: in the D_{5} case, only 32 
of the 40 roots are visible.  The other 8 are hidden in back somewhere.  
Where are they?

I asked John Stembridge about this and he gave a useful clue.  His
planar pictures show projections of the roots into what he calls the 
"Coxeter plane".  

Recall from "<A HREF = "week62.html">week62</A>" that the "Coxeter group" associated to a Dynkin 
diagram acts as rotation/reflection symmetries of the roots; it's 
generated by reflections through the roots.  There's a basis of roots 
called "simple roots", one for each dot in our Dynkin diagram, and 
the product of reflections through all these simple roots is called 
the "Coxeter element" of our Coxeter group - it's well-defined up to 
conjugation.  The "Coxeter plane" is the canonical plane on which the
Coxeter element acts as a rotation.

A rotation by how much?  The order of the Coxeter element is called
the "Coxeter number" and denoted h, so the Coxeter element
acts on the Coxeter plane as a rotation of 2\pi /h.  The Coxeter number
is important for other reasons, too!  Here's how it goes:


$$

Coxeter group   Coxeter number
   A_{n}               n+1
   B_{n}               2n
   C_{n}               2n
   D_{n}               2n-2
   E_{6}               12
   E_{7}               18 
   E_{8}               30
   F_{4}               12
   G_{2}               6
$$
    
For D_{5} the Coxeter number is 8, which accounts for the
8-fold symmetry of Stembridge's picture in that case.  The
E_{8} picture has 30-fold symmetry!  My D_{4} picture
has 8-fold symmetry, so I must not have been projecting down to the
Coxeter plane.

Anyway, this stuff should help answer my puzzle.  I don't know the answer,
though.


\par\noindent\rule{\textwidth}{0.4pt}
\textbf{Addenda:} I thank David Corfield and James Dolan for catching
mistakes.  Tony Smith found a nice picture created by G&uuml;nter
Ziegler of the D_{4} root system (that is, the 24-cell) 
viewed from the Coxeter plane.  The D_{4} root system is
4-dimensional, but it's been drawn with a bit of 3d perspective.  
The 6-fold symmetry is evident:

<div align = "center">
<img width = "500" src = "24-cell.jpg">
</div>

16) G&uuml;nter M. Ziegler, picture of 24-cell,
<a href = "http://www.math.tu-berlin.de/~ziegler/24-cell.jpeg">
http://www.math.tu-berlin.de/~ziegler/24-cell.jpeg</a>

For more discussion, go to the <a href = "http://golem.ph.utexas.edu/category/2007/03/this_weeks_finds_in_mathematic_8.html">\emph{n}-Category 
Caf&eacute;</a>.

\par\noindent\rule{\textwidth}{0.4pt}
<em>The true spirit of delight, the exaltation, the sense of being more 
than Man, which is the touchstone of the highest excellence, is to 
be found in mathematics as surely as poetry.</em> - Bertrand Russell

\par\noindent\rule{\textwidth}{0.4pt}

% </A>
% </A>
% </A>
