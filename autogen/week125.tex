
% </A>
% </A>
% </A>
\week{November 3, 1998 }

Last week I promised to explain some mysterious connections between
elliptic curves, string theory, and the number 24.  I claimed that
it all boils down to the fact that there are two especially symmetric
lattices in the plane, namely the square lattice:



\begin{verbatim}


                    *     *     *     *


                    *     *     *     *


                    *     *     *     *
         
\end{verbatim}
    
with 4-fold symmetry, and the hexagonal lattice:



\begin{verbatim}


                    *       *      *      *


                        *       *      *                


                    *       *      *      *

\end{verbatim}
    

with 6-fold symmetry.  Now it's time for me to start backing up those
claims.  

First I need to talk a bit about lattices and SL(2,Z).  As I explained
in "<A HREF = "week66.html">week66</A>", a lattice in the complex plane consists of all points that
are integer linear combinations of two complex numbers, 
say \omega _{1} and
\omega _{2}.  However, we can change these numbers without 
changing the lattice by letting


$$

                 \omega '_{1} = a \omega _{1} + b \omega _{2}

                 \omega '_{2} = c \omega _{1} + d \omega _{2}

$$
    
where 

\begin{verbatim}

                          a   b 

                          c   d
\end{verbatim}
    
is a 2\times 2 invertible matrix of integers whose inverse again consists of
integers.  Usually it's good to require
that our transformation preserve the handedness of the basis (\omega _{1},
\omega _{2}), which means that this matrix should have determinant 1.
Such matrices form a group called SL(2,Z).  In the context of elliptic
curves it's also called the "modular group".

Now associated to the square lattice is a special element of SL(2,Z)
that corresponds to a 90 degree rotation.   Everyone calls it S:


\begin{verbatim}

                         0  -1
                   S =
                         1   0
\end{verbatim}
    
Associated to the hexagonal lattice is a special element of SL(2,Z) 
that corresponds to a 60 degree rotation.   Everyone calls it ST:


\begin{verbatim}

                         0  -1
                  ST =
                         1   1
\end{verbatim}
    
(See, there's a matrix they already call T, and ST is the product of S
and that one.)  Now, you may complain that the matrix ST doesn't look
like a rotation, but you have to be careful!   What I mean is, if you
take the hexagonal lattice and pick a basis for it like this:


$$

                     \omega _{2}      
                *       *      *      *

                                 \omega _{1}
                    *      0*      *                


                *       *      *      *

$$
    
then in \emph{this} basis the matrix ST represents a 60 degree rotation.

So far this is pretty straightforward, but now come some surprises. 
First, it turns out that SL(2,Z) is \emph{generated} by S and ST.  In other
words, every 2\times 2 integer matrix with determinant 1 can be written as a
product of a bunch of copies of S, ST, and their inverses.  Second, all
the relations satisfied by S and ST follow from these obvious ones:

                             S^{4} = 1
                          (ST)^{6} = 1
together with

                          S^{2} = (ST)^{3}
which holds because both sides describe a 180 degree rotation.

Right away this implies that SL(2,Z) has a certain inherent "12-ness" to
it.  Let me explain.  SL(2,Z) is a nonabelian group - this is how
someone with a Ph.D. says that matrix multiplication doesn't commute -
but suppose we abelianize it by imposing extra relations \emph{forcing}
commutativity.  Then we get a group generated by S and ST, satisfying
the above relations together with an extra one saying that S and ST
commute.  This is the group Z/12, which has 12 elements!

This "12-ness" has a lot to do with the magic properties of the number
24 in string theory.  But to see how this "12-ness" affects string
theory, we need to talk about elliptic curves a bit more.   It will take
forever unless I raise the mathematical sophistication level a little. 
So....

We can define an elliptic curve to be a torus C/L formed by taking the 
complex plane C and modding out by a lattice L.   Since C is an abelian
group and L is a subgroup, this torus is an abelian group, but in the
theory of elliptic curves we consider it not just as a group but also as
a complex manifold.  Thus two elliptic curves C/L and C/L' are
considered isomorphic if there is a complex-analytic function from one
to the other that's also an isomorphism of groups.  This happens
precisely when there is a nonzero number z such that zL = L', or in
other words, whenever L' is a rotated and/or dilated version of L.   

There's a wonderful space called the "moduli space" of elliptic
curves: each point on it corresponds to an isomorphism class of
elliptic curves.  In physics, we think of each point in it as
describing the geometry of a torus-shaped string worldsheet.  Thus in
the path-integral approach to string theory we need to integrate over
this space, together with a bunch of other moduli spaces corresponding
to string worldsheets with different topologies.  All these moduli
spaces are important and interesting, but the moduli space of elliptic
curves is a nice simple example when you're first trying to learn this
stuff.  What does this space look like?

Well, suppose we have an elliptic curve C/L.  We can take our lattice
L and describe it in terms of a right-handed basis (\omega _{1}, \omega _{2}).
For the purposes of classifying the describing the elliptic curve up
to isomorphism, it doesn't matter if we multiply these basis elements
by some number z, so all that really matters is the ratio

$$

                       \tau  = \omega _{1}/\omega _{2}.
$$
    
Since our basis was righthanded, \tau  lives in the upper halfplane, 
which people like to call H.  

Okay, so now we have described our elliptic curve in terms of a
complex number \tau  lying in H.  But the problem is, we could have
chosen a different right-handed basis for our lattice L and gotten a
different number \tau .  We've got to think about that.  Luckily, we've
already seen how we can change bases without changing the lattice: we
just apply a matrix in SL(2,Z), getting a new basis


$$

                 \omega '_{1} = a \omega _{1} + b \omega _{2}

                 \omega '_{2} = c \omega _{1} + d \omega _{2}
$$
    
This has the effect of changing \tau  to 

$$

                \tau ' = (a \tau  + b)/(c \tau  + d).
$$
    
If you don't see why, figure it out - you've gotta understand this to
understand elliptic curves!    

Anyway, two numbers \tau  and \tau ' describe isomorphic elliptic curves
if and only if they differ by the above sort of transformation.  So
we've figured out the moduli space of elliptic curves: it's the
quotient space H/SL(2,Z), where SL(2,Z) acts on H as above!

Now, the quotient space H/SL(2,Z) is not a smooth manifold, because
while the upper halfplane H is a manifold and the group SL(2,Z) is
discrete, the action of SL(2,Z) on H is not free: i.e., certain points
in H don't move when you hit them with certain elements of SL(2,Z).

If you don't see why this causes trouble, think about a simpler
example, like the group G = Z/n acting as rotations of the complex
plane, C.  Most points in the plane move when you rotate them, but the
origin doesn't.  The quotient space C/G is a cone with its tip
corresponding to the origin.  It's smooth everywhere except the tip,
where it has a "conical singularity".  The moral of the story is that
when we mod out a manifold by a group of symmetries, we get a space
with singularities corresponding to especially symmetrical points in
the original manifold.

So we expect that H/SL(2,Z) has singularities corresponding to points in
H corresponding to especially symmetrical lattices.   These, of course,
are our friends the square and hexagonal lattices!

But let's be a bit more careful.  First of all, \emph{nothing} 
in H moves when 
you hit it with the matrix -1.  But that's no big deal: we can just
replace the group SL(2,Z) by 


\begin{verbatim}

                        PSL(2,Z) = SL(2,Z)/{&plusmn;1}
\end{verbatim}
    
Since -1 doesn't move \emph{any} points of H, the action of SL(2,Z) on H
gives an action of PSL(2,Z), and the moduli space of elliptic curves is
H/PSL(2,Z).  

Now most points in H aren't preserved by any element of PSL(2,Z).  
However, certain points are!  The point 


$$

                            \tau  = i
$$
    
corresponding to the square lattice, is preserved by S and all its
powers.  And the point


$$

                         \tau  = exp(2\pi i/3)
$$
    
corresponding to the hexagonal lattice, is preserved by ST and all its
powers.   These give rise to two conical singularities in the moduli
space of elliptic curves.  Away from these points, the moduli space is
smooth.  

Lest you get the wrong impression, I should hasten to reassure you that
the moduli space is not all that complicated: it looks almost like the
complex plane!  There's a famous one-to-one and onto function from the
moduli space to the complex plane: it's called the "modular function"
and denoted by j.  So the moduli space is \emph{topologically} just like the 
complex plane; the only difference is that it fails to be \emph{smooth} at
two points, where there are conical singularities.  

This may seem a bit hard to visualize, but it's actually not too hard.   
Here's one way.  Start with the region in the upper halfplane outside 
the unit circle and between the vertical lines x = -1/2 and x = 1/2.  
It looks sort of like this:


\begin{verbatim}

                   .....................
                   .....................
                   .....................
                   .....................
                   .....................   
                   .....................
                   .....................               
                   .....................         
                   ..........A..........
                   .....           .....
                   ...               ...
                   .                   .
                   B                   B'

\end{verbatim}
    
Then glue the vertical line starting at B to the one starting at B',
and glue the arc AB to the arc AB'.  We get a space that's smooth
everywhere except at the points A and B = B', where there are conical
singularities.  The total angle around the point A is just 180 degrees
- half what it would be if the moduli space were smooth there.  The
total angle around B is just 120 degrees - one third what it would be
if the moduli space were smooth there.

The reason this works is that the region shown above is a "fundamental
domain" for the action of PSL(2,Z) on H.  In other words, every
elliptic curve is isomorphic to one where the parameter \tau  lies in
this region.  The point A is where \tau  = i, and the point B is where
\tau  = exp(2\pi i/3).

Now let's see where the "12-ness" comes into this picture.  Minhyong
Kim explained this to me in a very nice way, but to tell you what he
said, I'll have to turn up the level of mathematical sophistication
another notch.  (Needless to say, all the errors will be mine.)

So, I'll assume you know what a "complex line bundle" is - this is
just another name for a 1-dimensional complex vector bundle.  Locally
a section of a complex line bundle looks a lot like a complex-valued
function, but this isn't true globally unless your line bundle is
trivial.  If you aren't careful, sometimes you may \emph{think} you have a
function defined on a space, only to discover later that it's actually
a section of a line bundle.  This sort of thing happens all the time
in physics.  In string theory, when you're doing path integrals on
moduli space, you have to make sure that what you're integrating is
really a function!  So it's important to understand all the line bundles
on moduli space.

Now, given any sort of space, we can form the set of all isomorphism
classes of line bundles over this space.  This is actually an abelian
group, since when we tensor two line bundles we get another line
bundle, and when you tensor any line bundle with its dual, you get the
trivial line bundle, which plays the role of the multiplicative
identity for tensor products.  This group is called the "Picard 
group" of your space.

What's the Picard group of the moduli space of elliptic curves?  Well,
when I said "any sort of space" I was hinting that there are all sorts
of spaces - topological spaces, smooth manifolds, algebraic varieties,
and so on - each one of which comes with its own particular notion of
line bundle.  Thus, before studying the Picard group of moduli space
we need to decide what context we're going to work in!  As a mere
\emph{topological space}, we've seen that the moduli space of elliptic
curves is indistinguishable from the plane, and every \emph{topological}
line bundle over the plane is trivial, so in \emph{this} context the Picard
group is the trivial group - boring!

But the moduli space is actually much more than a mere topological
space.  It's not a smooth manifold, but it's awfully close: it's the
quotient of the smooth manifold H by the discrete group SL(2,Z), and
its singularities are pretty mild in nature.  

Somehow we should take advantage of this when defining the Picard
group of the moduli space.  One way to do so involves the theory of
"stacks".  Without getting into the details of this theory, let me 
just vaguely sketch what it does for us here.  For a much more careful
treatment, with more of an algebraic geometry flavor, try:

1) David Mumford, Picard groups of moduli problems, in Arithmetical
Algebraic Geometry, ed. O. F. G. Schilling, Harper and Row, New York, 
1965.

Suppose a discrete group G acts on a smooth manifold X.  A
"G-equivariant" line bundle on X is a line bundle equipped with an
action of G that gets along with the action of G on X.  If G acts
freely on X, a line bundle on X/G is the same as a G-equivariant line
bundle on X.  This isn't true when the action of G on X isn't free.
But we can still go ahead and \emph{define} the Picard group of X/G to be
the group of isomorphism classes of G-equivariant line bundles on X.
Of course we should say something to let people know that we're using
this funny definition.  In our example, people call it the Picard
group of the moduli \emph{stack} of elliptic curves.

So what's this group, anyway?  

Well, it turns out that you can get any SL(2,Z)-equivariant line
bundle on H, up to isomorphism, by taking the trivial line bundle on H
and using a 1-dimensional representation of SL(2,Z) to say how it acts
on the fiber.  So we just need to understand 1-dimensional
representations of SL(2,Z).  The set of isomorphism classes of these
forms a group under tensor product, and this is the group we're after.

Well, a 1-dimensional representation of a group always factors through
the abelianization of that group.  We saw the abelianization of
SL(2,Z) was Z/12.  But everyone knows that the group of 1-dimensional
representations of Z/n is again Z/n - this is called Pontryagin duality.
So: the Picard group of the moduli stack of elliptic curves is Z/12. 

So we see again an inherent "12-ness" built into the theory of
elliptic curves!  You may be wondering how this makes the number 24 so
important in string theory.  In particular, where does that extra
factor of 2 come from?  I'll say a little more about this next Week.
I may or may not manage to tie together the loose ends!

You may also be wondering about "stacks".  In this you're not alone.
There's an amusing passage about stacks in the following book:

2) Joe Harris and Ian Morrison, Moduli of Curves, Springer-Verlag, New
York, 1998.

They write:

\begin{quote}
     "Of course, here I'm working with the moduli stack rather than 
     with the moduli space.  For those of you who aren't familiar with
     stacks, don't worry: basically, all it means is that I'm allowed to
     pretend that the moduli space is smooth and that there's a universal
     family over it."

     Who hasn't heard these words, or their equivalent, spoken in a
     talk?  And who hasn't fantasized about grabbing the speaker by
     the lapels and shaking him until he says what - exactly - he means
     by them?  But perhaps you're now thinking that all that is in the
     past, and that at long last you're going to learn what a stack is
     and what they do.

     Fat chance.  

\end{quote}
Actually Mumford's paper cited above gives a nice introduction to
the theory of stacks without mentioning the dreaded word "stack".  
Alternatively, you can wait and read this book when it comes out:

3) K. Behrend, L. Fantechi, W. Fulton, L. Goettsche and A. Kresch,
An Introduction to Stacks, in preparation.

But let me just briefly say a bit about stacks and the moduli stack of
elliptic curves in particular.  A stack is a weak sheaf of categories.
For this to make sense you must already know what a sheaf is!  In the
simplest case, a sheaf over a topological space, the sheaf S gives you
a set S(U) for each open set U, and gives you a function S(U,V): S(U)
\to  S(V) whenever the open set U is contained in the open set V.  These
functions must satisfy some laws.  The notion of "stack" is just a
categorification of this idea.  That is, a stack S over a topological
space gives you a \emph{category} S(U) for each open set U, and gives you a
\emph{functor} S(U,V): S(U) \to  S(V).  These functors satisfy the same laws
as before, but \emph{only up to specified natural isomorphism}.  And these
natural isomorphisms must in turn satisfy some new laws of their own,
so-called coherence laws.

In the case at hand there's a stack over the moduli space of elliptic
curves.  For any open set U in the moduli space, an object of S(U) is a
family of elliptic curves over U, such that each elliptic curve in the
family sits over the point in moduli space corresponding to its
isomorphism class.  Similarly, a morphism in S(U) is a family of
isomorphisms of elliptic curves.  This allows us to keep track of the
fact that some elliptic curves have more automorphisms than others!  And
it takes care of the funny stuff that happens at the singular points in
the moduli space.

By the way, this watered-down summary leaves out a lot of the
algebraic geometry that you usually see when people talk about stacks.

Finally, one more thing - it looks like Kreimer and company are making
great progress on understanding renormalization in a truly elegant way. 

4) D. J. Broadhurst and D. Kreimer, Renormalization automated by Hopf algebra,
preprint available as 
<A HREF = "http://xxx.lanl.gov/abs/hep-th/9810087">hep-th/9810087</A>.  


Let me quote the abstract:

\begin{quote}
It was recently shown that the renormalization of quantum field theory
is organized by the Hopf algebra of decorated rooted trees, whose
coproduct identifies the divergences requiring subtraction and whose
antipode achieves this. We automate this process in a few lines of
recursive symbolic code, which deliver a finite renormalized expression
for any Feynman diagram. We thus verify a representation of the operator
product expansion, which generalizes Chen's lemma for iterated
integrals. The subset of diagrams whose forest structure entails a
unique primitive subdivergence provides a representation of the Hopf
algebra H_{R} of undecorated rooted trees. Our undecorated Hopf algebra
program is designed to process the 24,213,878 BPHZ contributions to the
renormalization of 7,813 diagrams, with up to 12 loops. We consider 10
models, each in 9 renormalization schemes. The two simplest models
reveal a notable feature of the subalgebra of Connes and Moscovici,
corresponding to the commutative part of the Hopf algebra H_{T} of the
diffeomorphism group: it assigns to Feynman diagrams those weights which
remove \zeta  values from the counterterms of the minimal subtraction
scheme. We devise a fast algorithm for these weights, whose squares are
summed with a permutation factor, to give rational counterterms.
\end{quote}



 \par\noindent\rule{\textwidth}{0.4pt}

% </A>
% </A>
% </A>
