<html><head><title>week219</title></head>

% </A>
% </A>
% </A>

I'm about to head to Sydney and Canberra to help celebrate the 60th birthday of 
Ross Street ... the world's best n-category theorist!

1) Categories in Algebra, Geometry and Mathematical Physics, conference
in honor of the 60th birthday of Ross Street, 
   <A HREF = "http://streetfest.maths.mq.edu.au/">http://streetfest.maths.mq.edu.au/
% </A>

Lots of people will be talking about the interface of higher-dimensional
algebra and physics.  Some will be talking about higher
gauge theory, which generalizes ordinary gauge theory from point particles
to strings, loops,
or higher-dimensional "branes" by replacing groups with n-groups.

Ezra Getzler will be speaking on how to get an n-group from a Lie n-algebra, 
and Mikhail Kapranov will be speaking on higher-dimensional holonomies.  
Alissa Crans, Danny Stevenson and I will also be talking about this stuff. 

There will eventually be a conference proceedings, but for now you can see
the abstracts of people's talks on the website.  You can see my talks here:

2) John Baez, Higher gauge theory, 
<A HREF = "http://math.ucr.edu/home/baez/street/">http://math.ucr.edu/home/baez/street/
% </A>

Anyway, before I take off, here's a roundup of stuff I've been meaning to 
mention: the centennial of Einstein's "annus mirabilis", 
the Pioneer anomaly, 
silicon photonics, a company that plans to sell quantum computers, and some 
relationships between Klein's quartic curve, the Fano plane, and special relativity 
over the integers mod 7.  I want to leave you lots of stuff to think about.  :-)  

Okay... let's start with Einstein's "annus mirabilis".

1905 was indeed a miraculous year for Albert Einstein.  He published four 
earth-shaking papers, all in the same journal:

3) Albert Einstein, On a heuristic viewpoint concerning the production and 
   transformation of light, Annalen der Physik 17 (1905), 132-148.  Available at 
   <A HREF = "http://dbserv.ihep.su/~elan/src/einstein05/eng.pdf">http://dbserv.ihep.su/~elan/src/einstein05/eng.pdf</A> 

   On the movement of small particles suspended in stationary liquids required by 
   the molecular-kinetic theory of heat, Annalen der Physik 17 (1905), 549-560.
   Available at 
   <A HREF = "http://lorentz.phl.jhu.edu/AnnusMirabilis/AeReserveArticles/eins_brownian.pdf">http://lorentz.phl.jhu.edu/AnnusMirabilis/AeReserveArticles/eins_brownian.pdf</A>

   On the electrodynamics of moving bodies, Annalen der Physik 17 (1905), 891-921.  
   Available at 
   <A HREF = "http://dbserv.ihep.su/~elan/src/einstein05b/eng.pdf"> http://dbserv.ihep.su/~elan/src/einstein05b/eng.pdf</A>

   Does the inertia of a body depend upon its energy content?, 
   Annalen der Physik 18 (1905), 639-641.  Available at 
   <A HREF = "http://dbserv.ihep.su/~elan/src/einstein05c/eng.pdf">
   http://dbserv.ihep.su/~elan/src/einstein05c/eng.pdf</A>

In the first of these papers, Einstein explained the photoelectric effect by 
assuming that light consisted of particles each carrying an energy equal to 
Planck's constant times its frequency.  This was an important step towards 
quantum mechanics - a theory he would later fight against.  

In the second, he showed that Brownian motion was explained by the existence 
of atoms.  His calculations were later used to measure Boltzmann's constant.

In the third, Einstein derived the formula for Lorentz transformations from 
two simple assumptions: only relative motion can be detected, and every 
unaccelerated observer measures light to have the same speed.

In the fourth, he derived a relation between mass, momentum, and energy, including 
as a special case the famous formula E = mc^{2} relating the mass of a body to its 
energy at rest.

The most impressive thing about these papers is how simple and readable they are: 
they go from clearly stated assumptions to world-shaking conclusions with clear 
logic and only a little math.  

Which makes one wonder: will there ever be another Einstein?  Can one person ever
again make such revolutions in physics?  Or are all the remaining discoveries yet
to be made in physics too complicated?  Why has fundamental physics been stuck 
ever since the completion of the Standard Model?  There are lots of glorious  
theories, but none of them have gotten any experimental confirmation.  There have
also been some big empirical discoveries - dark matter, dark energy, neutrino
oscillations, and more - but these weren't predicted, and our theories still 
don't shed much light on them.  

So, are the times ripe for a good new idea... a really smart person who will
fit the jigsaw puzzle together?  Or are there still too many missing pieces?

Lee Smolin has some interesting thoughts about this: 

4) Lee Smolin, <A HREF = "no-new-einstein.pdf">Why no "new 
Einsteins"?</A>, Physics Today, June 2005, 56-57.

I think the mention of Einstein is mainly just a trick to get people to read the 
article.  He focuses on institutional pressures that push physicists to conform: 
to follow "research programs"
in big teams rather than strike out on their own.

This is certainly a big problem.  But I'm not sure it's the only problem.  

Fundamental physics is in a tough situation, partly a victim of its own success.
Pure thought may not be enough.  To make real progress, it helps to have lots of 
experimental results that don't fit the current theories.  Not just a few 
numbers here or there, like neutrino masses.   We want \emph{piles} of unexplained 
data! - enough to draw lots of graphs.   Then we could cook up theories that fit 
the data.

Barring this, we need to assemble all the mysteries we can get our hands on,
and see if they fit together somehow.  

Luckily, right now several spacecraft are leaving the the Solar System... 
and they're leaving us with the best of gifts: a mystery!

Pioneer 10 was launched in 1972.  It was the first spacecraft to travel through 
the asteroid belt, and the first to take closeups photos of Jupiter, and study the
intense magnetic field of this planet.  I remember getting excited about this when 
I was a kid. 

Pioneer 10 was also the first spacecraft to explore the outer solar system.
In 1983 it passed the orbit of Neptune.  On February 7th, 2003, Pioneer 10's 
radioactive power source weakened to the point where its radio signals became 
too feeble to detect.  When this happened, it was 80 astronomical units from the 
Sun: in other words, 80 times as far from the Sun as we are, or about twice as far 
out as Pluto. 

But this was just the beginning of its cold dark journey.  It will continue to coast 
through deep space in the general direction of the red giant Aldebaran, 68 light 
years from us.  A light year is about 63,000 astronomical units, so this journey 
will take a while: over 2 million years!  We'll probably be long gone by then, for 
better or worse.

Pioneer 11 was launched in 1973.  It followed its sister ship to Jupiter, then
swung past Saturn, and then sailed out into the night, studying the solar wind
as it went.  On September 30, 1995 its power source became too weak to run 
any more experiments, and NASA stopped monitoring it.   It was 45 astronomical 
units from the Sun, moving out at 2 AU per year.  

We are no longer in the beam of its radio signal.  Its antenna cannot be rotated.  
It is heading towards Aquila - the Eagle - and it will pass one of the stars in this 
constellation in 4 million years.

Here's a picture of what these spacecraft are doing:

5) NASA, Pioneer path, 
   <A HREF = 
"http://spaceprojects.arc.nasa.gov/Space_Projects/pioneer/path.html">
   http://spaceprojects.arc.nasa.gov/Space_Projects/pioneer/path.html</A>
<br>

<A HREF = "http://spaceprojects.arc.nasa.gov/Space_Projects/pioneer/path.html">
<div align = "center">
<IMG SRC = pioneer.gif>
</div>
% </A>

<br>
See how they'll eventually pass through a bubble called the "termination 
shock"?  
That's where the solar wind drops below supersonic speed as it crashes into 
the gas of the Milky Way.  You can see a pattern vaguely reminscient of the
wave formed by a boat sailing through the sea - that's because we're moving
through the Milky Way.  

There are also two other spacecraft in this picture: Voyager 1 and 2!  These craft 
are still transmitting, and Voyager 1 is now farther than Pioneer 10.   It crossed 
the termination shock in December 2004, zipping along at 3.6 AU per year.  It's 
sending back information about this region of space - called the 
"heliosheath" -  
and its batteries should be good until 2020.

But what's the mystery?

Well, the Pioneer missions yielded the most precise information we have about 
navigation in deep space.  However, analysis of their radio tracking data indicates 
a small unexplained acceleration towards the Sun!  The magnitude of this acceleration
is roughly 10^{-9} meters per second per second.   It's called the 
"Pioneer anomaly".   

This anomaly has also been seen in the Ulysses spacecraft, and possibly also in 
the Galileo spacecraft, though the data is much more noisy, since these were 
Jupiter probes, hence much closer to the Sun, where there is a lot more pressure 
from solar radiation.  The Voyagers are not set up to provide good data on this
issue, since they are able to rotate themselves and this messes things up.  The 
Viking mission to Mars did \emph{not} detect the Pioneer anomaly - and it would have, 
had an acceleration of this magnitude been present, because its radio tracking was 
incredibly accurate - good to about 12 meters.

Many physicists and astronomers have tried to explain the Pioneer anomaly using 
conventional physics, but so far nobody seems to have succeeded.  Radiation
pressure from the Sun, the solar wind, the back-reaction from the radio emissions:
all these point the wrong way!  Other explanations also seem to fail, like 
gravity from the Kuiper belt, small amounts of gas venting from the spacecraft, 
and thermal radiation from the craft, 

As conventional explanations seemed to fail, people started trying to explain the 
anomaly using new physics - for example, modified theories of gravity, or dark 
matter.  But it's hard to get these explanations to work, either.  For example, 
explaining the Pioneer anomaly by the gravitational attraction of dark matter 
would require more than .0003 solar masses of dark matter within 50 AU units of 
the Sun.  But this is in conflict with our highly successful calculations of 
planetary orbits - even a millionth of a solar mass of dark matter in this region 
would be enough to throw those off!

So, we may have an interesting clue on our hands.

For more information on the Pioneer anomaly, see these sources and the many 
references therein:

6) Wikipedia, Pioneer anomaly, <A HREF = "http://en.wikipedia.org/wiki/Pioneer_anomaly">
http://en.wikipedia.org/wiki/Pioneer_anomaly</A> 

7) Chris P. Duif, Pioneer anomaly - literature and links, 
   <A HREF = "http://www.space-time.info/pioneer/pioanomlit.html">
http://www.space-time.info/pioneer/pioanomlit.html</A>

I got interested in the Pioneer anomaly when I read this recent proposal
for a mission to study it:

8) The Pioneer Collaboration, A mission to explore the Pioneer anomaly,
   available as
   <A HREF = "http://arxiv.org/abs/gr-qc/0506139">gr-qc/0506139</A>.

Finally, here's a fun book on the Pioneer missions:

9) Mark Wolverton, The Depths of Space: The Story of the Pioneer Planetary 
   Probes, Joseph Henry Press, 2004.  Available at 
   <A HREF = "http://www.nap.edu/books/0309090504/html/">
   http://www.nap.edu/books/0309090504/html/
% </A>

Next: silicon lasers.  I don't want to say much about these, just that Intel
thinks they could be the next big thing.  You've probably heard about 
"Moore's Law", 
how the number of components in an integrated circuit doubles every 2 
years.   Or more vaguely: how computers keeping getting more powerful, really
fast!  And you've probably heard that this exponential growth is in danger of 
running into a brick wall someday.  One of the problems is that copper wire 
carries too little data, too slowly.  

For a long time some people have touted "photonics" 
as the way around this: 
transmitting information as pulses of light, rather than clumps of electrons.
One problem is that substances that emit such pulses tend to be expensive, 
like gallium arsenide and indium phosphide.  But now they've figured out to 
make silicon function as a laser!  Researchers at Intel have created a silicon 
laser that emits a continuous beam of light.  They've also developed a modulator 
that chops the beam into 10 billion pulses per second - a 10 gigahertz signal.  

For details, try this:

10) Intel, Silicon photonics, <A HREF = 
"http://www.intel.com/technology/silicon/sp/">
http://www.intel.com/technology/silicon/sp/</A>


11) Robert Service, Intel's breakthrough, Technology Review, July 2005, 62-65.  
    Also available at <A HREF = 
"http://www.technologyreview.com/articles/05/07/issue/feature_intel.asp">
http://www.technologyreview.com/articles/05/07/issue/feature_intel.asp</A>

While we're talking about high tech, how about quantum computers?  Most of
them don't work very well - yet, say the optimists.  Interaction with the 
environment ruins the coherence of the quantum state.  But there's a company
called D-Wave Systems that aims to build a different breed of quantum computer - 
one that doesn't mind a fair amount of noise.   It's an analog chip made of lots
of superconductors, which is supposed to quantum tunnel to a lowest-energy state,
The idea is that you can get it to solve lots of minimization problems this way,
like the travelling salesman problem.  

I don't see why it'll work better than other analogue computers, or methods 
like simulated annealing.  It could get stuck when there are lots of 
"almost-minima"
to sift through... just 
like glass gets stuck when it tries to find \emph{its} lowest
energy state.  Physicists call this 
"frustration": the poor glass in your window
is "frustrated", 
trying to crystallize but unable to decide how to do it.  In
principle it could do it by quantum tunnelling, but in practice that takes forever.
Why will D-Wave Systems' computer be better?

By the way, they admit their computer \emph{can't} do cool stuff like rapidly factor 
large numbers via Shor's algorithm, the way full-fledged quantum computers should.  
True devotees of quantum computation probably wouldn't even call it a quantum 
computer!  But I'll be happy if it works.

They've certain managed to convince some investors.  Here's some more info:

12) Erika Jonietz, Quantum calculation, Technology Review, July 2005, 24-25.
    Also available at
    <A HREF = "http://www.technologyreview.com/articles/05/07/issue/forward_quantum.asp">
http://www.technologyreview.com/articles/05/07/issue/forward_quantum.asp</A>

But enough practical stuff!  Now let me say a bit more about the Klein quartic.  

I wrote about this gadget in 
"<A HREF = "week214.html">week214</A>" and 
"<A HREF = "week215.html">week215</A>".   
Simply put, this is a "Platonic surface": 
a 3-holed torus tiled by regular heptagons, with 3 
heptagons meeting at each vertex.  

It's perfectly symmetrical, but you can't stuff it into 3-dimensional Euclidean 
space without warping it.  So, its charms are a bit more esoteric than those of,
say, a dodecahedron.

Of course, this is the kind of challenge that some people just can't resist.  
Mike Stay and Gerard Westendorp bravely tried making paper models of it:

13) Mike Stay, Klein quartic, <A HREF = 
"http://math.ucr.edu/~mike/klein/">
http://math.ucr.edu/~mike/klein/
% </A>

14) Gerard Westendorp, Geometry, <A HREF = 
"http://www.xs4all.nl/~westy31/Geometry/Geometry.html">
http://www.xs4all.nl/~westy31/Geometry/Geometry.html</A>

Mike wisely stopped just short of the final step, which would create a nasty 
crumpled mess.  Gerard succeeded in completing the task by switching to \emph{pastry}
instead of paper.  Check it out!

There might be a small niche market for Klein quartic birthday cakes... but 
computer graphics are probably better if you just want to visualize this surface
instead of actually eat it.  About 10 or 12 years ago, Joe Christy made the 
following pictures using a program called Geomview, which makes virtual 3d objects:

15) Joe Christy, Klein quartic pictures: <br>
    <A HREF = "http://math.ucr.edu/home/baez/pentacontihexahedron.jpg">http://math.ucr.edu/home/baez/pentacontihexahedron.jpg</A><br>
    <A HREF = "http://math.ucr.edu/home/baez/pentacontihexahedron2.jpg">http://math.ucr.edu/home/baez/pentacontihexahedron2.jpg</A><br>
    <A HREF = "http://math.ucr.edu/home/baez/pentacontihexahedron3.jpg">http://math.ucr.edu/home/baez/pentacontihexahedron3.jpg</A>
<br>
<div align="center">


<img src="pentacontihexahedron.jpg" height="200" width="200">


<img src="pentacontihexahedron2.jpg" height="200" width="200">


<img src="pentacontihexahedron3.jpg" height="200" width="200">


</div>
It took about a day on the fastest Linux machine they had at the time.  The 
advantage of Geomview is that once the virtual object is done, you can 
quickly create different views.  These pictures show the dual version of 
Klein's quartic, which has 7 regular triangles meeting at each quarter.  
The funky name "pentacontihexahedron" 
refers to the fact that there are 56
of these triangles.

While I'm at it, I can't resist showing you another beautiful picture: 

16) Joe Christy, Fano plane on Roman surface, 
    <A HREF = 
"http://math.ucr.edu/home/baez/roman.jpg">http://math.ucr.edu/home/baez/roman.jpg</A>
<br>
</p><div align="center">
<img src = "roman.jpg">
</div>
<br>
</p>


The blue thing is called the "Roman surface" 
because it was discovered 
by the mathematician Jakob Steiner while he was visiting Rome.  It's a 
self-intersecting immersion of the projective plane in 3d space.  On it, 
the 7 lines of the Fano plane are visible in red, with four of them drawn 
as circles.  

Mathematically, one nice thing about this picture is that it exhibits the 
tetrahedral symmetry of the Fano plane!  

You can see all these pictures and much more on my Klein quartic website:

17) John Baez, Klein's quartic curve, <A HREF = 
"http://math.ucr.edu/home/baez/klein.html">
http://math.ucr.edu/home/baez/klein.html</A>

You may think I'm digressing, but the relation between Klein's quartic curve and 
the Fano plane underlies what I want to talk about today.  Greg Egan and I 
realized that this relation is just part of a bigger picture involving special 
relativity in 3-dimensional spacetime... over the integers mod 7.

Huh?

Well, these days so-called physicists have no shame studying physics in all sorts 
of dimensions, but they usually confine themselves to building their spacetimes 
out of the real numbers.

That makes sense if they're trying to claim some relevance to real-world physics,
however slight.   But mathematically, there's no reason not to try other number 
systems, like finite fields, just for the fun of it.  

And this sheds new light on the Klein quartic.  Why?  Because the symmetries of 
the Klein quartic and the Fano plane also act as \emph{Lorentz transformations} in 3 
dimensional spacetime if you work using the integers mod 7.  This lets us see the 
Klein quartic and Fano plane as being closely related to special relativity in this 
funny context.

Let's see how this goes.

First, recall that a "field" 
is a number system where you can add, subtract, 
multiply and divide to your hearts content, with all the basic laws holding that 
hold for real numbers.  A "finite field"
is one with finitely many elements, like
the integers mod any prime number p.  This example is called Z/p, or 
F_{p} if you 
really want to emphasize that you're thinking of it as a field.

So, let's do some 3d special relativity with Z/7, and see what it has to say
about the Klein quartic.  

First, some basic stuff about finite fields.

The concepts of "positive" 
and "negative" make sense in any finite field!  
Say a nonzero element of the field is 
"positive" 
if it's of the form x^{2} 
and "negative" 
otherwise.  (Number theorists call the positive elements 
"quadratic residues", just to intimidate outsiders.) 

Then multiplication works nicely: 

<ul>
<li>
  if you multiply two positive elements you get a positive one;
</li>
<li>
 if you multiply two negative elements you get a positive one;
</li>
<li>
 if you multiply a positive and a negative element you get a negative one.
</li>
</ul>

There are finite fields whose cardinality is any prime power, but if we focus on 
those whose cardinality is a prime, namely the fields Z/p, there are three 
possibilities: the good, the bad, and the ugly. 

<ul>
<li>
 GOOD: If the field is Z/p for p = 4n+3 then -1 is negative, <br>
 so we
 can switch the "sign" of a number by multiplying it by -1.  
</li>
<li>
 BAD: If the field is Z/p for p = 4n+1 then -1 is positive, so we
 can't do this.
</li>
</p>
</ul>

In both these cases there are as many positive as negative elements.
Then there's

<ul>
<li>
 UGLY: If the field is Z/2 then every element is positive.
</li>
</ul>

Luckily, p = 7 is good.   But beware: addition doesn't get along with positivity 
very well.  In fields like Z/p, \emph{every} element is a sum of positive elements.

Next, let's ponder the peculiarities of special relativity over a finite field.

We can define Minkowski spacetime of any dimension over any field F: 
it's just F^{n+1} with the quadratic form

x^{2} = x_{0}^{2} - x_{1}^{2} - ... - x_{n}^{2}

We define O(n,1) to be the group of transformations of F^{n+1} that preserve 
the above quadratic form, and define the "Lorentz group" 
SO(n,1) to be
the subgroup consisting of transformations that also have determinant 1.

As usual, we say that a vector x in Minkowski spacetime is:
<ul>
<li>
   timelike if  x^{2} > 0
</li>
<li>
   lightlike if x^{2} = 0
</li>
<li>
   spacelike if x^{2} < 0</li>
</li>
</ul>


We define a "ray" 
to be a line through the origin.  We say a ray is
timelike, spacelike or lightlike if any vector on it - hence all! - 
is of that type.

The lightlike rays are usually called "light rays", 
both because it sounds cool 
(we went into physics because we liked things like X-rays and rayguns) and 
because it's accurate.  The light rays going through a given point - the origin -
are precisely like this.   

Next, let's ponder the peculiarities of 3-dimensional spacetime.

For any field F, 2\times 2 matrices with determinant 1 act as Lorentz transformations
of 3d Minkowski spacetime.   I touched upon this idea when discussing Trautman's 
"Pythagorean spinors" 
in 
"<A HREF = "week196.html">week196</A>".   Here's how it works:

We can think of 3d Minkowski spacetime as consisting of all 2\times 2 matrices
that are equal to their own transpose:


$$

    ( x_{0} + x_{1}       x_{2}   )
x = (                   )
    (    x_{2}      x_{0} - x_{1} )
$$
    
since the determinant of such a matrix is just 

$$

x^{2} = x_{0}^{2} - x_{1}^{2} - x_{2}^{2}
$$
    
In this picture, the group SL(2,F) consisting of 2\times 2 matrices with determinant 1
acts as Lorentz transformations:

g: x |\to  gxg*

where g* is the transpose of g.   So, we get a homomorphism from SL(2,F)
to the 3d Lorentz group:

SL(2,F) \to  SO(2,1)

This is two-to-one, since it sends both 1 and -1 to the identity Lorentz 
transformation.   People typically cure this by defining

PSL(2,F) = SL(2,F)/{+-1}

We then get a one-to-one homomorphism

PSL(2,F) \to  SO(2,1)

Alas, this homomorphism is not onto: it's only "half-onto".  
In the traditional
case where F is the real numbers, its range is just one of the two connected 
components of SO(2,1).  In the case we're interested in here, where F = Z/7, 
the group PSL(2,Z/7) has 168 elements but SO(2,1) has twice as many. 

Next, let's bring the hyperbolic plane into the game!

Special relativity in 3 dimensions is closely related to the hyperbolic
plane.  The reason is that the set of timelike rays in (n+1)-dimensional Minkowski
spacetime forms a copy of hyperbolic n-space: physicists call this the
"mass hyperboloid".   
So, for n = 2, we get the hyperbolic plane.  This is 
most familiar for special relativity based on the real numbers, but the same 
idea applies to other fields.

So, let's make some definitions:

<ul>
<li>
  the hyperbolic plane H+ is the set of timelike rays in 3d Minkowski spacetime
</li>
<li>
  the heavenly circle L is the set of light rays in 3d Minkowski spacetime
</li>
<li>
  the hyperbolic cylinder H- is the set of spacelike rays in 3d Minkowski spacetime
</li>
</ul>

You may think I'm being silly to call the set of light rays "the heavenly
circle", 
but in 4-dimensional spacetime the analogous thing is often called 
"the heavenly sphere", and we're studying things one dimension down.

Why "heavenly sphere"?  
Well, when you look at the stars at night, they seem to 
be lying on a sphere.  That's the heavenly sphere: the set of light rays 
entering 
your eye - the set of directions you can look!  

One dimension down, flatlanders get to enjoy the "heavenly circle".  Mathematicians 
call this the projective line, since it's a line with one extra point added on.

Now for something fun: points on the hyperbolic plane give lines on the 
hyperbolic cylinder and vice versa!  

This is basically by definition of "line".  We define a "line" in H+ to consist 
of all points that are orthogonal to a given point in H-, and vice versa.
Note a point in either of these spaces is really a ray in Minkowski spacetime, 
but it makes sense to say that two rays are orthogonal.  


This definition isn't arbitrary: it reduces to a standard notion of "line" in
the hyperbolic plane - namely a geodesic - when our field is the real numbers.


Finally, let's dive into the case we're really interested in: 
3-dimensional Minkowski spacetime over F = Z/7.


In this case the positive numbers are 1,2,4, and the negative numbers
are 3,5,6.  

It turns out that:
<ul>
<li>
 The hyperbolic plane over Z/7, namely H+, has size 21.  

</li>
<li>

 The heavenly circle over Z/7, namely L, has size 8.

</li>
<li>

 The hyperbolic cylinder over Z/7, namely H-, has size 28.   
</li>
</ul>

So, H+ is a nice finite version of the hyperbolic plane with 21 points and 
28 lines!  A little calculation shows there are 3 points on each line and 
4 lines through each point.

We know that PSL(2,Z/7) acts on everything in sight here: H+, H-, and L.  
It also acts on the Fano plane and Klein's quartic curve.  So, we can try to 
match up various features of 3d special relativity with features in the Fano 
plane or Klein's quartic curve!

Greg Egan found the following correspondence:

\begin{verbatim}

HYPERBOLIC PLANE OVER Z/7                      FANO PLANE    

7 triads                                       7 points  
7 antitriads                                   7 lines       
21 points                                      21 flags     
28 lines                                       28 apartments 
\end{verbatim}
    
As usual with these correspondences, simple things in the Fano plane correspond 
to subtle things in the hyperbolic plane, and simple things in the hyperbolic 
plane correspond to subtle things in the Fano plane.

First of all, points and lines in the Fano plane correspond to "triads" and 
"antitriads" in H+.

Huh?  

Well, for starters, a "triad" or "antitriad" is an unordered triple of orthogonal 
points in H+.   

Huh?  

Well, remember that a point in H+ is a timelike ray in 3d Minkowski spacetime.
You can't have three timelike rays that are orthogonal in ordinary special 
relativity, but you can over Z/7, because the sum of postive numbers can be zero, 
or even negative.  For example, these three vectors are timelike and orthogonal:

(1,0,0), &nbsp; (0,4,2),  &nbsp; (0,-2,4)

We call the corresponding triple of rays a 
"triad", and we get a total of 7 triads 
by applying elements of PSL(2,Z/7) to it.  But, there are triples of orthogonal 
timelike rays that aren't among these 7.  For example, we get one from these three 
vectors:

(1,0,0),  &nbsp; (0,2,4), &nbsp; (0,-4,2)

We call the corresponding triple of rays an "antitriad", 
and we get 7  
antitriads by applying elements of PSL(2,Z/7).

Each line in the Fano plane contains 3 points, and each point lies on 3
lines.  This incidence relation can also be seen in terms of triads and
antitriads: each triad has nonempty intersection with 3 antitriads, and
each antitriad has nonempty intersection with 3 triads!

We can also go backwards: points and lines in the hyperbolic plane correspond
to "flags" and "apartments" in the Fano plane.  

Huh?

Flags and apartments are standard concepts in the 
theory of "buildings"
which I began to explain in "<A HREF = "week186.html">week186</A>".  
But, I don't want or need to explain
this general theory here.  In the Fano plane, a "flag" 
consists of a point
lying on a line:

\begin{verbatim}
                  
                    --------x---------
\end{verbatim}
    
An "apartment" 
consists of 3 distinct points lying on 3 distinct lines like
this:

\begin{verbatim}

                       \           /
                        \         /
                   ------x-------x-----
                          \     /
                           \   /
                            \ /
                             x
                            / \
                           /   \
\end{verbatim}
    
Each apartment in the Fano plane contains 3 flags, and each flag is contained
in 4 apartments.  This incidence can also been seen in terms of points and lines
in the hyperbolic plane: each line contains 3 points, and each point lies on 4
lines!

So, there's an interesting but complicated relation between hyperbolic geometry
over Z/7 and the Fano plane.  How does the Klein quartic curve fit in?   There's
more to this side of the story than I've managed to absorb, so I'll just say a 
few words - probably more than you want to hear.  For more detail, try my Klein
quartic curve webpage.

There are 48 nonzero lightlike vectors in 3d Minkowksi spacetime, but if you
take one of them and apply elements of PSL(2,Z/7) to it, you get an orbit 
consisting of only 24.  These 24 guys correspond to the 24 heptagons in the 
heptagonal tiling of the Klein quartic curve!  In other words, PSL(2,Z/7)
acts in precisely the same way.   

You may ask what the point of all this stuff is, and the answer is - I'm not
sure yet, except that it's fun!  Apparently the coincidence

PSL(2,Z/7) = PSL(3,Z/2)

is the only coincidence among classical groups over finite fields, not
counting the ones we already know over the real numbers.   So, it's got to be
good for something!  And, I haven't even begun to exploit the fact that
the Klein quartic curve is a quotient space of the real hyperbolic plane:
this has got to be related to the hyperbolic plane over Z/7.  So, I think
something interesting should emerge, though I'm not sure what.
                            


</p>\par\noindent\rule{\textwidth}{0.4pt}
\textbf{Addendum:} Regarding the Pioneer anomaly, 
the cosmologist Ned Wright writes:

\begin{quote}
John, 
</p>
I don't think the obvious possibilities have been ruled out.  In this I
disagree with Anderson but that's science at the fringes.
</p>
I have a Web page on this at:<br>
<A HREF = "http://www.astro.ucla.edu/~wright/PioneerAA.html">
http://www.astro.ucla.edu/~wright/PioneerAA.html</A>
</p>
\end{quote}

Just for the record, that's:

17) Ned Wright, Pioneer anomalous acceleration,
<A HREF = "http://www.astro.ucla.edu/~wright/PioneerAA.html">
http://www.astro.ucla.edu/~wright/PioneerAA.html</A>

For a contrasting viewpoint see:

18) Slava G. Turyshev, Michael Martin Nieto, and John D. Anderson,
The Pioneer anomaly and its implications, available as 
<A HREF = "http://arxiv.org/abs/gr-qc/0510081">gr-qc/0510081</A>.


</p>
\par\noindent\rule{\textwidth}{0.4pt}
<em>Space isn't remote at all. It's only an hour's drive away if your 
car could go straight upwards.</em> - Sir Fred Hoyle
\par\noindent\rule{\textwidth}{0.4pt}

% </A>
% </A>
% </A>
