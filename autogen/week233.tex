
% </A>
% </A>
% </A>
\week{May 20, 2006 }

On Tuesday I'm supposed to talk with Lee Smolin about an idea he's
been working on with Fotini Markopoulou and Sundance Bilson-Thompson.
This idea relates the elementary particles in one generation of the 
Standard Model to certain 3-strand framed braids:

1) Sundance O. Bilson-Thompson, A topological model of composite
preons, available as <A HREF =
"http://arxiv.org/abs/hep-ph/0503213">hep-ph/0503213</A>.

2) Sundance O. Bilson-Thompson, Fotini Markopoulou, and Lee Smolin, 
Quantum gravity and the Standard Model, <A HREF = "http://xxx.lanl.gov/abs/hep-th/0603022">hep-th/0603022</A>.
 
It's a very speculative idea: they've found some interesting relations, 
but nobody knows if these are coincidental or not.  

Luckily, one of my hobbies is collecting mysterious relationships between
basic mathematical objects and trying to figure out what's going on.
So, I already happen to know a bunch of weird facts about 3-strand braids.  
I figure I'll tell Smolin about this stuff.  But if you don't mind, I'll 
practice on you!  

So, today I'll try to tell a story connecting the 3-strand braid group, 
the trefoil knot, rational tangles, the groups SL(2,Z) and PSL(2,Z), 
conformal field theory, and Monstrous Moonshine.  

I've talked about some of these things before, but now I'll introduce
some new puzzle pieces, which come from two places:

3) Imre Tuba and Hans Wenzl, Representations of the braid group B_{3}
and of SL(2,Z), available as <A HREF =
"http://arxiv.org/abs/math.RT/9912013">math.RT/9912013</A>.
 
4) Terry Gannon, The algebraic meaning of genus-zero, available as
<A HREF = "http://arxiv.org/abs/math.NT/0512248">math.NT/0512248</A>.

You could call it "a tale of two groups".

On the one hand, the 3-strand braid group has generators 


\begin{verbatim}

     |   |   |
      \ /    |
A  =   /     |
      / \    |
     |   |   |
\end{verbatim}
    
and


\begin{verbatim}

     |   |   |   
     |    \ /    
B =  |     /    
     |    / \  
     |   |   |
\end{verbatim}
    
and the only relation is

ABA = BAB

otherwise known as the "third Reidemeister move":


\begin{verbatim}

|   |  |         |  |   | 
 \ /   |         |   \ /
  /    |         |    / 
 / \   |         |   / \
|   \ /           \ /   |  
|    /      =      /    | 
|   / \           / \   | 
 \ /   |         |   \ / 
  /    |         |    /
 / \   |         |   / \  
\end{verbatim}
    
On other hand, the group SL(2,Z) consists of 2\times 2 integer matrices with
determinant 1.  It's important in number theory, complex analysis,
string theory and other branches of pure mathematics.  I've described
some of its charms in "<A HREF =
"week125.html">week125</A>", "<A HREF =
"week229.html">week229</A>" and elsewhere.

These groups look pretty different at first.  But, there's a 
homomorphism from B_{3} onto SL(2,Z)!  It goes like this:


$$

        1   1
A |->  
        0   1


        1   0
B |->  
       -1   1
$$
    

Both these matrices describe "shears" in the plane.  
You may enjoy drawing these shears and visualizing the equation 
ABA = BAB in these terms.  I did.

I would like to understand this better... and here are some clues.

The center of B_{3} is generated by the element (AB)^{3}.  
This element corresponds to a "full twist".  In other words, it's
the braid you get by hanging 3 strings from the ceiling, grabbing 
them all with one hand at the bottom, and giving them a full 360-degree 
twist:


\begin{verbatim}

|   |  |  
 \ /   | 
  /    |       A
 / \   |
|   \ /    
|    /         B 
|   / \   
 \ /   | 
  /    |       A
 / \   |
|   \ /    
|    /         B 
|   / \   
 \ /   | 
  /    |       A
 / \   |
|   \ /    
|    /         B 
|   / \   
|  |   |
\end{verbatim}
    
This full twist gets sent to -1 in SL(2,Z):


$$

             -1   0 
(AB)^{3} |->  
              0  -1
$$
    
So, the double twist gets sent to the identity:


$$

             1   0 
(AB)^{6} |->  
             0   1
$$
    
In fact, Tuba and Wenzl say the double twist \emph{generates} the
kernel of our homomorphism from B_{3} to SL(2,Z).  So, SL(2,Z)
is isomorphic to the group of 3-strand braids modulo double twists!

This reminds me of spinors... since you have to twist an electron 
around \emph{twice} to get its wavefunction back to where it started.   
And indeed, SL(2,Z) is a subgroup of SL(2,C), which is the double 
cover of the Lorentz group.  So, 3-strand braids indeed act on the 
state space of a spin-1/2 particle, with double twists acting 
trivially!

(For more on this, check out Trautman's work on "Pythagorean
spinors" in "<A HREF = "week196.html">week196</A>".
There's also a version where we use integers mod 7, described in
"<A HREF = "week219.html">week219</A>".)

If instead we take 3-strand braids modulo full twists, we get the 
so-called "modular group":

PSL(2,Z) = SL(2,Z)/{+-1}

Now, SL(2,Z) is famous for being the "mapping class group"
of the torus - that is, the group of orientation-preserving
diffeomorphisms, modulo diffeomorphisms connected to the identity.
Similary, PSL(2,Z) is famous for acting on the rational numbers
together with a point at infinity by means of fractional linear
transformations:


$$

         az + b
z |->   -------
         cz + d
$$
    

where a,b,c,d are integers and ad-bc = 1.  The group PSL(2,Z) also
acts on certain 2-strand tangles called "rational tangles".
In "<A HREF = "week229.html">week229</A>", I told a nice
story I heard from Michael Hutchings, explaining how these three facts
fit together in a neat package.
 

But now let's combine those facts with the stuff I just said!  Since
PSL(2,Z) acts on rational tangles, and there's a homomorphism from
B_{3} to PSL(2,Z), 3-strand braids must act on rational
tangles.  How does that go?

There's an obvious guess, or two, or three, or four, but let's just
work it out.

I just said that the 3-strand braid A gets mapped to this shear:


$$

        1   1
A |->  
        0   1
$$
    
In "<A HREF = "week229.html">week229</A>" I said what this
shear does to a rational tangle.  It gives it a 180 degree twist at
the bottom, like this:


$$

  |   |                |   |
  |   |                |   |
  |   |                |   |
 -------              -------
 |  T  |   |---->     |  T  |        
 -------              -------
  |   |                 \ /
  |   |                  / 
  |   |                 / \
$$
    
Next, Tuba and Wenzl point out that


$$

                  0   1
ABA = BAB |-> 
                 -1   0
$$
    

which is a quarter turn.  From "<A HREF =
"week229.html">week229</A>" you can see how this quarter turn
acts on a rational tangle:


$$

  |   |                       |     | 
  |   |              ____     |     |
  |   |             /     \   |     |
 -------           |     -------    |
 |  T  |   |---->  |     |  T  |    |    
 -------           |     -------    |
  |   |            |      |   \____/
  |   |            |      |    
  |   |            |      |  
$$
    
It gives it a quarter turn!      

From these facts, we can figure out what the braid B does to a 
rational tangle.  So, let me do the calculation.  

Scribble, scribble, curse and scribble.... Eureka!

Since we know what A does, and what ABA does, we can figure out 
what B must do.  But, to make the answer look cute, I needed a 
sneaky fact about rational tangles, which is that A \emph{also} acts 
like this:


$$

  |   |                 \ / 
  |   |                  /
  |   |                 / \ 
 -------              -------
 |  T  |   |---->     |  T  |        
 -------              -------
  |   |                |   |
  |   |                |   | 
  |   |                |   |
$$
    
This is proved in Goldman and Kauffman's paper cited in "<A HREF = "week228.html">week228</A>".
With the help of this, I can show B acts like this:


$$

  |   |              |          |
  |   |              |   ___    |
  |   |              |  /   \   |
 -------             | /    -------   
 |  T  |   |---->     \     |  T  |  
 -------             / \    ------- 
  |   |             |   \___/   | 
  |   |             |           | 
  |   |             |           |
$$
    
And this is \emph{great!}  It means our action of 3-strand braids on 
rational tangles is really easy to describe.  Just take your tangle
and let the upper left strand dangle down:


\begin{verbatim}

           |
   ____    |               
  /    \   |              
 |    -------           
 |    |  T  |   
 |    -------  
 |     |   |  
 |     |   | 
 |     |   |
\end{verbatim}
    
To let a 3-strand braid act on this, just attach it to the bottom of
the picture!   

(That's why there were \emph{four} obvious guesses about this would work:
one can easily imagine four variations on this trick, depending on 
which strand is the "odd man out" - here it's the upper right.  It's
just a matter of convention which we use, but my conventions give this.)

In fact, even the group of 4-strand braids acts on rational tangles in 
an obvious way, but the 3-strand braid group is enough for now.

Let me summarize.  
The 3-strand braid group B_{3} acts on rational tangles
in an obvious way.  The subgroup that acts trivially is precisely the 
center of B_{3}, generated by the full twist.  
Using stuff from "<A HREF = "week229.html">week229</A>", 
it follows that the quotient of B_{3} by its center acts 
on the projectivized 
rational homology of the torus.  We thus get a topological explanation 
of why B_{3} mod its center is PSL(2,Z).

But there's more.

For starters, the 3-strand braid group is also the fundamental group of 
S^{3} minus the trefoil knot!

And, S^{3} minus the trefoil knot is secretly the same as
SL(2,R)/SL(2,Z)!

In fact, Terry Gannon writes that the 3-strand braid group can be
regarded as "the universal central extension of the modular
group, and the universal symmetry of its modular forms".  I'm not
completely sure what that means, but here's \emph{part} of what it
means.

Just as PSL(2,C) is the Lorentz group in 4d spacetime, PSL(2,R) is the 
Lorentz group in 3d spacetime.  This group has a double cover SL(2,R), 
which shows up when you study spinors.  But, it also has a universal 
cover, which shows up when you study anyons.  The universal cover has
infinitely many sheets.   And up in this universal cover, sitting over 
the subgroup SL(2,Z), we get... the 3-strand braid group!

In math jargon, we have this commutative diagram where the
rows are short exact sequences:



$$

    1 ----> Z -----> B_{3} -------> SL(2,Z) ----> 1

            |        |             |          
            |        |             |         
            v        v             v        

    1 ----> Z ---> SL(2,R)^{~} ---> SL(2,R) ----> 1
$$
    
 
Here SL(2,R)^{~} is the universal cover of SL(2,R).
Since \pi _{1}(SL(2,R)) = Z, this is a Z-fold cover.
You can describe this cover explicitly using the Maslov index, 
which is a formula that actually computes an integer for any loop 
in SL(2,R), or indeed any symplectic group.  

But anyway, fiddling around with this diagram and the long exact
sequence of homotopy groups for a fibration, you can show that indeed:

\pi _{1}(SL(2,R)/SL(2,Z)) = B_{3}.

This also follows from the fact that 
SL(2,R)/SL(2,Z) looks like S^{3} minus a trefoil.

Gannon believes that number theorists should think about all this stuff,
since he thinks it's lurking behind that weird network of ideas called
Monstrous Moonshine (see "<A HREF = "week66.html">week66</A>").

And here's the basic reason why.  I'll try to get this right....

Any rational conformal field theory has a "chiral algebra" A which 
acts
on the left-moving states.  Mathematicians call this sort of thing a
"vertex operator algebra".  A representation of this on some vector
space V is a space of states for the circle in some "sector" of our
theory.  Let's pick some state v in V.   Then we can define a 
"one-point function" where we take a Riemann surface with little 
disk cut out and insert this state on the boundary.  This is a number,
essentially the amplitude for a string in the give state to evolve like 
this Riemann surface says. 

In fact, instead of chopping out a little disk it's nice to just
remove a point - a "puncture", they call it.  But, we get an ambiguous
answer unless we pick coordinates at this point, or in the lingo of
complex analysis, a choice of "uniforming parameter".  Then our
one-point function becomes a function on the moduli space of Riemann
surfaces equipped with a puncture and a choice of uniformizing parameter.

If we didn't have this uniformizing parameter to worry about, we'd
just have the moduli space of tori equipped with a marked point, 
which is nothing but the usual moduli space of elliptic curves,

H/PSL(2,Z)

where H is the complex upper halfplane.  Then our one-point function
would have nice transformation properties under PSL(2,Z).  

But, with this uniformizing parameter to worry about, our one-point 
function only has nice transformation properties under B_{3}.  This 
is somehow supposed to be related to how B_{3} is the "universal
central extension" of PSL(2,Z): in conformal field theory, all sorts
of naive symmetries hold only up to a phase, so you have to replace
various groups by central extensions thereof... and here that's what's 
happening to PSL(2,Z)!

That last paragraph was pretty vague.  If I'm going to understand this
better, either someone has to help me or I've got to read something
like this:

5) Yongchang Zhu, Modular invariance of characters of vertex operator algebras,
J. Amer. Math. Soc 9 (1996), 237-302.  Also available at
<A HREF = "http://www.ams.org/jams/1996-9-01/S0894-0347-96-00182-8/home.html">http://www.ams.org/jams/1996-9-01/S0894-0347-96-00182-8/home.html</A>


But I shouldn't need any conformal field theory to see how the moduli
space of punctured tori with uniformizing parameter is related to the
3-strand braid group!  I bet this moduli space is X/B_{3} for
some space X, or something like that.  There's something simple at the
bottom of all this, I'm sure.

\par\noindent\rule{\textwidth}{0.4pt}
<B>Addenda:</B>  Another relation between the trefoil and
the punctured torus: the trefoil has genus 1, meaning that
it bounds a torus minus a disc embedded in R^{3}.
You can see this in the lecture "Genus and knot sum" in 
this course on knot theory:

6) Brian Sanderson, The knot theory MA3F2 page, 
<A HREF = ""http://www.maths.warwick.ac.uk/~bjs/MA3F2-page.html">http://www.maths.warwick.ac.uk/~bjs/MA3F2-page.html</A>

This course also has material on rational tangles.

The fact that B_{3} is a central extension of
PSL(2,Z) by Z, and the quantum-mechanical interpretation of
a central extension in terms of phases, plays an important role
here:


7) R. Voituriez, Random walks on the braid group B_{3} 
and magnetic translations in hyperbolic geometry, Nucl. Phys. B621 
(2002), 675-688.  Also available as 
<A HREF = "http://arxiv.org/abs/math-ph/0103008">http://arxiv.org/abs/math-ph/0103008</A>.

Among other things, he points out that the homomorphism
B_{3} \to  SL(2,Z) described above is the "Burau
representation" of B_{3} evaluated at t = 1.
In general, the Burau representation of B_{3} is given 
by:


$$

        t   1
A |->  
        0   1


        1   0
B |->  
       -t   t
$$
    
(Conventions differ, and this may not be the best, but it's
the one he uses.)  The Burau representation can also be used
to define a knot invariant called the Alexander polynomial.  
I believe that with some work, one can use this to explain
why Conway could calculate the rational number associated to a
rational tangle in terms of the ratio of Alexander polynomials of two
links associated to it, called its "numerator" and
"denominator".  In fact he computed this ratio of
polynomials and then evaluate it at a special value of t - 
presumably the same special value we're seeing here (modulo
differences in convention).

Another issue: I wrote

\begin{quote}
For starters, the 3-strand braid group is also the fundamental group of 
S^{3} minus the trefoil knot!

And, S^{3} minus the trefoil knot is secretly the same as
SL(2,R)/SL(2,Z)!
\end{quote}

The first one is pretty easy to see; you start with the "Wirtinger 
presentation" of the fundamental group of S^{3} minus a 
trefoil, and show by a fun little calculation that this isomorphic to the 
braid group on 3 strands.  A more conceptual proof would be very nice, though.
(See "<a href = "week261.html">week261</a>" for such a proof -
and much more on all this stuff.)

What about the second one?  Why is  S^{3} minus the trefoil knot 
diffeomorphic
to SL(2,R)/SL(2,Z)?   Terry Gannon says so in his paper above, but doesn't
say why.  Some people asked about this, and eventually some people found
some explanations.  First of all, there's a proof on page 84 of this book:

8) John Milnor, Introduction to Algebraic K-theory, Annals of Math.
Studies 72, Princeton U. Press, Princeton, New Jersey, 1971.

Milnor credits it to Quillen.  Joe Christy summarizes it below.  I
can't tell if this proof is essentially the same as another sketched
below by Swiatoslaw Gal, which exhibits a diffeomorphism using
functions called the Eisenstein series g_{2} and
g_{3}.  They are probably quite similar arguments.

Joe Christy writes:

\begin{quote}
  I wouldn't be surprised if this was known to Seifert in the 30's,
  though I can't lay my hands on Seifert \text{\&}  Threfall at the moment to
  check. Likewise for Hirzebruch, Brieskorn, Pham & Milnor in the 60's in
  relation to singularities of complex hypersurfaces and exotic spheres.
  When I was learning topology in the 80's it was considered a warm up
  case of Thurston's Geometrization Program - the trefoil knot complement
  has PSL(2,R) geometric structure.

  In any case, peruse Milnor's Annals of Math Studies for concrete
  references. There is a (typically) elegant proof on p.84 of
  "Introduction to Algebraic K-theory" [study 72], which Milnor credits 
  to Quillen. It contains the missing piece of John's argument:
  introducing the Weierstrass P-function and remarking that the
  differential equation that it satisfies gives the diffeomorphism to
  S^{3}-trefoil as the boundary of the pair (discriminant of diff-eq, 
  C^{2} = (P,P')-space).

  This point of view grows out of some observations of Zariski, fleshed
  out in "Singular Points of Complex Hypersurfaces" [study 61]. The
  geometric viewpoint is made explicit in the paper "On the Brieskorn
  Manifolds M(p,q,r)" in "Knots, Groups, and 3-manifolds" [study 84].

  It is also related to the intermediate case between the classical
  Platonic solids and John's favorite Platonic surface - the 
  <A HREF = "klein.html">Klein quartic</A>.  By way of a hint, look to
  relate the trefoil, qua torus knot, the seven-vertex triangulation of
  the torus, and the dual hexagonal tiling of a (flat) Clifford torus in 
  S^{3}.
  Joe
\end{quote}

Swiatoslaw Gal writes:

\begin{quote}
  In fact the isomorphism is a part of the modular theory:

  Looking for
  f: GL(2,R)/SL(2,Z) \to  C^{2} - {x^{2}=y^{3}}

  (there is an obvious action of R+ on both sides:
 M |\to  tM for M in  GL(2,R),<br>
  x |\to  t^{6} x, <br> y |\to  t^{4} y,
   and the quotient is what we want).

  GL(2,R)/SL(2,Z) is a space of lattices in C.
  Such a lattice L has classical invariants

  g_{2}(L) =  60 sum_{z in L'} z^{-4}, 

  and

   g_{3}(L) = 140 sum_{z in L'} z^{-6},
  
  where L'=L-{0}

  The modular theory asserts that:

  1. For every pain (g_{2},g_{3}) there exist a
     lattice L such that g_{2}(L) = g_{2} and
     g_{3}(L) = g_{3} provided that
     g_{2}^{3} is not equal to
     27g_{3}^{2}.
  2. Such a lattice is unique.
  Best,<br>
  S. R. Gal
\end{quote} 

The quantity g_{2}^{3} - 27
g_{3}^{2} is called the "discriminant" of the
lattice L, and vanishes as the lattice squashes down to being
degenerate, i.e. a discrete subgroup of C with one rather than two
generators.

\par\noindent\rule{\textwidth}{0.4pt}
% </A>
% </A>
% </A>
