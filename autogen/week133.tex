
% </A>
% </A>
% </A>
\week{April 23, 1999 }


I'd like to start with a long quote from a paper by Ashtekar:

1) Abhay Ashtekar, Quantum Mechanics of Geometry, preprint available 
as <A HREF = "http://xxx.lanl.gov/abs/gr-qc/9901023">gr-qc/9901023</A>.

\begin{quote}
During his Goettingen inaugural address in 1854, Riemann suggested that
the geometry of space may be more than just a fiducial, mathematical
entity serving as a passive stage for physical phenomena, and may in
fact have direct physical meaning in its own right. General relativity
provided a brilliant confirmation of this vision: curvature of space now
encodes the physical gravitational field.  This shift is profound.  To
bring out the contrast, let me recall the situation in Newtonian
physics.  There, space forms an inert arena on which the dynamics of
physical systems - such as the solar system - unfolds.  It is like a
stage, an unchanging backdrop for all of physics.  In general
relativity, by contrast, the situation is very different.  Einstein's
equations tell us that matter curves space.  Geometry is no longer
immune to change.  It reacts to matter.  It is dynamical.  It has
"physical degrees of freedom" in its own right.  In general relativity,
the stage disappears and joins the troupe of actors!  Geometry is a
physical entity, very much like matter.

Now, the physics of this century has shown us that matter has
constituents and the 3-dimensional objects we perceive as solids are in
fact made of atoms.  The continuum description of matter is an
approximation which succeeds brilliantly in the macroscopic regime but
fails hopelessly at the atomic scale.  It is therefore natural to ask:
Is the same true of geometry?  If so, what is the analog of the `atomic
scale?'  We know that a quantum theory of geometry should contain three
fundamental constants of Nature, c, G, \hbar , the speed of light,
Newton's gravitational constant and Planck's constant.  Now, as Planck
pointed out in his celebrated paper that marks the beginning of quantum
mechanics, there is a unique combination, 

L = sqrt(\hbar  G/c^3), 

of these constants which has dimension of length.  (L ~ 10^{-33} cm.) 
It is now called the Planck length.  Experience has taught us that the
presence of a distinguished scale in a physical theory often marks a
potential transition; physics below the scale can be very different from
that above the scale.  Now, all of our well-tested physics occurs at
length scales much bigger than L. In this regime, the continuum
picture works well.  A key question then is: Will it break down at the
Planck length?  Does geometry have constituents at this scale?  If so,
what are its atoms?  Its elementary excitations?  Is the space-time
continuum only a `coarse-grained' approximation?  Is geometry quantized? 
If so, what is the nature of its quanta?

To probe such issues, it is natural to look for hints in the
procedures that have been successful in describing matter.  Let us
begin by asking what we mean by quantization of physical quantities.
Take a simple example - the hydrogen atom.  In this case, the answer is
clear: while the basic observables - energy and angular momentum -
take on a continuous range of values classically, in quantum mechanics
their eigenvalues are discrete; they are quantized.  So, we can ask if
the same is true of geometry.  Classical geometrical quantities such as
lengths, areas and volumes can take on continuous values on the phase
space of general relativity.  Are the eigenvalues of corresponding
quantum operators discrete?  If so, we would say that geometry is
quantized and the precise eigenvalues and eigenvectors of geometric
operators would reveal its detailed microscopic properties.

Thus, it is rather easy to pose the basic questions in a precise
fashion.  Indeed, they could have been formulated soon after the advent
of quantum mechanics.  Answering them, on the other hand, has proved to
be surprisingly difficult.  The main reason, I believe, is the
inadequacy of standard techniques.  More precisely, to examine the
microscopic structure of geometry, we must treat Einstein gravity
quantum mechanically, i.e., construct at least the basics of a quantum
theory of the gravitational field.  Now, in the traditional approaches
to quantum field theory, one \emph{begins} with a continuum, background
geometry.  To probe the nature of quantum geometry, on the other hand,
we should \emph{not} begin by assuming the validity of this picture.  We must
let quantum gravity decide whether this picture is adequate; the theory
itself should lead us to the correct microscopic model of geometry.

With this general philosophy, in this article I will summarize the
picture of quantum geometry that has emerged from a specific approach to
quantum gravity.  This approach is non-perturbative.  In perturbative
approaches, one generally begins by assuming that space-time geometry is
flat and incorporates gravity - and hence curvature - step by step by
adding up small corrections.  Discreteness is then hard to unravel.

[Footnote: The situation can be illustrated by a harmonic oscillator:
While the exact energy levels of the oscillator are discrete, it would
be very difficult to "see" this discreteness if one began with a free
particle whose energy levels are continuous and then tried to
incorporate the effects of the oscillator potential step by step via
perturbation theory.]   

In the non-perturbative approach, by contrast, there is no background
metric at all.  All we have is a bare manifold to start with.  All
fields - matter as well as gravity/geometry - are treated as dynamical
from the beginning. Consequently, the description can not refer to a
background metric.  Technically this means that the full diffeomorphism
group of the manifold is respected; the theory is generally covariant.

As we will see, this fact leads one to Hilbert spaces of quantum states
which are quite different from the familiar Fock spaces of particle
physics.  Now gravitons - the three dimensional wavy undulations on a
flat metric - do not represent fundamental excitations. Rather, the
fundamental excitations are \emph{one} dimensional.  Microscopically, geometry
is rather like a polymer.  Recall that, although polymers are
intrinsically one dimensional, when densely packed in suitable
configurations they can exhibit properties of a three dimensional
system.  Similarly, the familiar continuum picture of geometry arises as
an approximation: one can regard the fundamental excitations as `quantum
threads' with which one can `weave' continuum geometries.  That is, the
continuum picture arises upon coarse-graining of the semi-classical
`weave states'.  Gravitons are no longer the fundamental mediators of the
gravitational interaction.  They now arise only as approximate notions. 
They represent perturbations of weave states and mediate the
gravitational force only in the semi-classical approximation.  Because
the non-perturbative states are polymer-like, geometrical observables
turn out to have discrete spectra.  They provide a rather detailed
picture of quantum geometry from which physical predictions can be made.

The article is divided into two parts.  In the first, I will indicate
how one can reformulate general relativity so that it resembles gauge
theories.  This formulation provides the starting point for the quantum
theory.  In particular, the one-dimensional excitations of geometry
arise as the analogs of "Wilson loops" which are 
themselves analogs of
the line integrals exp(i integral A.dl) of electromagnetism.  In the
second part, I will indicate how this description leads us to a quantum
theory of geometry.  I will focus on area operators and show how the
detailed information about the eigenvalues of these operators has
interesting physical consequences, e.g., to the process of Hawking
evaporation of black holes.
\end{quote}

I feel like quoting more, but I'll resist.  It's a nice semi-technical
introduction to loop quantum gravity - a very good place to start if you
know some math and physics but are just getting started on the quantum
gravity business.

Next, here are some papers by younger folks working on loop quantum
gravity:

2) Fotini Markopoulou, The internal description of a causal set: What
the universe looks like from the inside, preprint available as
<A HREF = "http://xxx.lanl.gov/abs/gr-qc/9811053">gr-qc/9811053</A>.  

Fotini Markopoulou, Quantum causal histories, preprint available as
<A HREF = "http://xxx.lanl.gov/abs/hep-th/9904009">hep-th/9904009</A>.   

Fotini Markopoulou is perhaps the first person to take the issue of
causality really seriously in loop quantum gravity.  In her earlier work
with Lee Smolin (see "<A HREF = "week99.html">week99</A>" and "<A HREF = "week114.html">week114</A>") she proposed a way to equip an 
evolving spin network (or what I'd call a spin foam) with a partial order on 
its vertices, representing a causal structure.  In these papers she is
further developing these ideas.  The first one uses topos theory!  It's
good to see brave young physicists who aren't scared of using a little
category theory here and there to make their ideas precise.  Personally 
I feel confused about causality in loop quantum gravity - I think we'll
have to muck around and try different things before we find out what 
works.  But Markopoulou's work is the main reason I'm even \emph{daring} to 
think about these issues....


 3) Seth A. Major, Embedded graph invariants in Chern-Simons theory,
preprint available as <A HREF =
"http://xxx.lanl.gov/abs/hep-th/9810071">hep-th/9810071</A>.


In This Week's Finds I've already mentioned Seth Major has worked with
Lee Smolin on q-deformed spin networks in quantum gravity (see "<A
HREF = "week72.html">week72</A>").  There is a fair amount of
evidence, though as yet no firm proof, that q-deforming your spin
networks corresponds to introducing a nonzero cosmological constant.
The main technical problem with q-deformed spin networks is that they
require a "framing" of the underlying graph.  Here Major
tackles that problem....

And now for something completely different, arising from a thread on
sci.physics.research started by Garrett Lisi.  What's the gauge group
of the Standard Model?  Everyone will tell you it's U(1) x SU(2) x
SU(3), but as Marc Bellon pointed out, this is perhaps not the most
accurate answer.   Let me explain why and figure out a better answer.  

Every particle in the Standard Model transforms according to some
representation of U(1) x SU(2) x SU(3), but some elements of this
group act trivially on all these representations.  Thus we can find
a smaller group which can equally well be used as the gauge group 
of the Standard Model: the quotient of U(1) x SU(2) x SU(3) 
by the subgroup of elements that act trivially.

Let's figure out this subgroup!  To do so we need to go through all
the particles and figure out which elements of U(1) x SU(2) x SU(3) 
act trivially on all of them.  

Start with the gauge bosons.  In any gauge theory, the gauge bosons
transform in the adjoint representation, so the elements of the gauge
group that act trivially are precisely those in the \emph{center} of the 
group.  U(1) is abelian so its center is all of U(1).  Elements of SU(n)
that lie in the center must be diagonal.  The n x n diagonal unitary 
matrices with determinant 1 are all of the form exp(2 \pi  i / n), 
and these form a subgroup isomorphic to Z/n.  It follows that the
center of U(1) x SU(2) x SU(3) is U(1) x Z/2 x Z/3.   

Next let's look at the other particles.  If you forget how these work,
see "<A HREF = "week119.html">week119</A>".  For the fermions, it suffices to look at those of the
first generation, since the other two generations transform exactly
the same way.  First of all, we have the left-handed electron and
neutrino:

(\nu _L, e_L)                                

These form a 2-dimensional representation.  This representation is the
tensor product of the irreducible rep of U(1) with hypercharge -1, the
isospin-1/2 rep of SU(2), and the trivial rep of SU(3).   

A word about notation!  People usually describe irreducible reps of U(1)
by integers.  For historical reasons, hypercharge comes in integral
multiples of 1/3.  Thus to get the appropriate integer we need to 
multiply the hypercharge by 3.  Also, the group SU(2) here is
associated, not to spin in the sense of angular momentum, but to
something called "weak isospin".  That's why we say "isopin-1/2 rep"
above.  Mathematically, though, this is just the usual spin-1/2
representation of SU(2).   

Next we have the left-handed up and down quarks, which come in 3
colors each:

(u_L, u_L, u_L, d_L, d_L, d_L)              

This 6-dimensional representation is the tensor product of the
irreducible rep of U(1) with hypercharge 1/3, the isospin-1/2
rep of SU(2), and the fundamental rep of SU(3).   

That's all the left-handed fermions.  Note that they all transform
transform according to the isospin-1/2 rep of SU(2) - we  call them
"isospin doublets".  The right-handed fermions all transform according
to the isospin-0 rep of SU(2) - they're "isospin singlets".  First we
have the right-handed electron:

e_R                                         

This is the tensor product of the irreducible rep of U(1) with
hypercharge -2, the isospin-0 rep of SU(2), and the trivial rep of
SU(3).  Then there are the right-handed up quarks:

(u_R, u_R, u_R)                             

which form the tensor product of the irreducible rep of U(1) with
hypercharge 4/3, the isospin-0 rep of SU(2), and the fundamental rep of
SU(3).  And then there are the right-handed down quarks:

(d_R, d_R, d_R)                             

which form the tensor product of the irreducible rep of U(1)
with hypercharge 2/3, the isospin-0 rep of SU(2), and the 
3-dimensional fundamental rep of SU(3).  

Finally, besides the fermions, there is the - so far unseen - Higgs
boson:

(H_+, H_0)

This transforms according to the tensor product of the irreducible
rep of U(1) with hypercharge 1, the isospin-1/2 rep of SU(2), and
the 1-dimensional trivial rep of SU(3).  

Okay, let's see which elements of U(1) x Z/2 x Z/3 act trivially on all
these representations!  Note first that the generator of Z/2 acts as
multiplication by 1 on the isospin singlets and -1 on the isospin
doublets.  Similarly, the generator of Z/3 acts as multiplication by 
1 on the leptons and exp(2 \pi  i / 3) on the quarks.  Thus everything
in Z/2 x Z/3 acts as multiplication by some sixth root of unity.  So
to find elements of U(1) x Z/2 x Z/3 that act trivially, we only need
to consider guys in U(1) that are sixth roots of unity.  

To see what's going on, we make a little table using the information
I've described:


$$


         ACTION OF            ACTION OF             ACTION OF
       exp(\pi  i / 3)            -1               exp(2 \pi  i / 3)
          IN U(1)             IN SU(2)               IN SU(3)

e_L         -1                  -1                     1
\nu _L        -1                  -1                     1
u_L     exp(\pi  i / 3)           -1               exp(2 \pi  i / 3)
d_L     exp(\pi  i / 3)           -1               exp(2 \pi  i / 3)

e_R          1                   1                     1
u_R     exp(4 \pi  i / 3)          1               exp(2 \pi  i / 3)
d_R     exp(4 \pi  i / 3)          1               exp(2 \pi  i / 3)


 H          -1                  -1                     1

$$
    
And we look for patterns!  

See any?

The most important one for our purposes is that if we multiply all three
numbers in each row, we get 1. 

This means that the element (exp(\pi  i / 3), -1, exp(2 \pi  i / 3)) in U(1)
x SU(2) x SU(3) acts trivially on all particles.  This element generates
a subgroup isomorphic to Z/6.  If you think a bit harder you'll see
there are no \emph{other} patterns that would make any \emph{more} elements of
U(1) x SU(2) x SU(3) act trivially.  And if you think about the relation
between charge and hypercharge, you'll see this pattern has a lot to do
with the fact that quark charges in multiples of 1/3, while leptons have
integral charge.  There's more to it than that, though....

Anyway, the "true" gauge group of the Standard Model - i.e., the
smallest possible one - is not U(1) x SU(2) x SU(3), but the quotient of
this by the particular Z/6 subgroup we've just found.  Let's call 
this group G.  

There are two reasons why this might be important.  First, Marc Bellon
pointed out a nice way to think about G: it's the subgroup of U(2) x U(3) 
consisting of elements (g,h) with

(det g)(det h) = 1.

If we embed U(2) x U(3) in U(5) in the obvious way, then this subgroup G
actually lies in SU(5), thanks to the above equation.  And this is what
people do in the SU(5) grand unified theory.  They don't actually stuff
all of U(1) x SU(2) x SU(3) into SU(5), just the group G!  For more
details, see "<A HREF = "week119.html">week119</A>".  Better
yet, try this book that Brett McInnes recommended to me:

4) Lochlainn O'Raifeartaigh, Group structure of gauge theories, 
Cambridge University Press, Cambridge, 1986.

Second, this magical group G has a nice action on a 7-dimensional 
manifold which we can use as the fiber for a 11-dimensional Kaluza-Klein 
theory that mimics the Standard Model in the low-energy limit.  The way 
to get this manifold is to take S^3 x S^5 sitting inside C^2 x C^3 and 
mod out by the action of U(1) as multiplication by phases.  The group 
G acts on C^2 x C^3 in an obvious way, and using this it's easy to see 
that it acts on (C^2 x C^3)/U(1).  

I'm not sure where to read more about this, but you might try:

5) Edward Witten, Search for a realistic Kaluza-Klein theory, 
Nucl. Phys. B186 (1981), 412-428.

Edward Witten, Fermion quantum numbers in Kaluza-Klein theory, 
Shelter Island II, Proceedings: Quantum Field Theory and the 
Fundamental Problems of Physics, ed. T. Appelquist et al, 
MIT Press, 1985, pp. 227-277.


6) Thomas Appelquist, Alan Chodos and Peter G.O. Freund, editors,
Modern Kaluza-Klein Theories, Addison-Wesley, Menlo Park, California,
1987.


 \par\noindent\rule{\textwidth}{0.4pt}

% </A>
% </A>
% </A>
