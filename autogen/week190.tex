
% </A>
% </A>
% </A>
\week{December 26, 2002 }


Fall quarter was very busy for me.  Next quarter I'll be on
sabbatical.  I just graded all my final exams, turned in the 
grade reports, cleaned up my house for the people who will 
be renting it, and left town.  Now I'm in Hong Kong, away 
from all my usual duties, and I have time to catch up on 
various things.  

For example: in November I went to the annual meeting of the
Philosophy of Science Association, which was held in Milwaukee.
I've never gotten around to talking about this yet.  I spoke
in a session on Structuralist Approaches to Quantum Gravity,
organized by Steven French.  "Structuralism" means a lot of things,
but as far as I can tell, in the philosophy of physics, it's 
an attempt to understand how terms gain their meaning as part 
of a physical theory, and the subtle sense in which they can retain 
some of their meaning as theories evolve.  

I ran into this problem in a very practical way when a mathematician 
once asked me to "define an electron".   I was reduced to incoherent 
sputtering: physics ain't math!  The electron in Bohr's model of the 
hydrogen atom is a very different thing than the electron in Dirac's 
hydrogen atom, which in turn is very different from the electron in 
QED...  but still, there's something "the same" about them, even 
apart from the fact that they're all attempts to model the "same
thing out there in the real world".  How does this work, exactly?  
It's way too complicated for me, but you can try reading this and 
see if it helps:

1) Heinz-Juergen Schmidt, Structuralism in physics, The Stanford 
Encyclopedia of Philosophy (Winter 2002 Edition), ed. Edward N. Zalta, 
<A HREF = "http://plato.stanford.edu/entries/physics-structuralism/">http://plato.stanford.edu/entries/physics-structuralism/</A>  

To be honest, what I really like about structuralism is that it 
makes philosophers think a lot about things like "mappings between 
theories", which gets them interested in category theory, which 
in turn gets them interested in other good stuff: background-free 
physics, n-categories, and so on.   It can't be all bad if it does
that!

I liked this conference because I met quite a few philosophers
who are well-versed in the technical aspects of physics and busy
thinking about interesting things.  Perhaps the most obvious
example is John Earman, who gave a big plenary talk on spontaneous 
symmetry breaking.  (A reference to this paper can be found at the end of this 
article.)  I'm really fond of a paper Earman wrote with his 
student Gordon Belot on the problem of time in quantum gravity:

2) John Earman and Gordon Belot, Pre-Socratic quantum gravity, 
in Physics Meets Philosophy at the Planck Scale, eds. Chris 
Callender and Nick Huggett, Cambridge U. Press, Cambridge, 2001.

and also this paper on the C*-algebraic approach to quantum
field theory on curved spacetime:

3) Aristidis Arageorgis, John Earman, and Laura Ruetsche,
Weyling the time away: the non-unitary implementability 
of quantum field dynamics on curved spacetime, in Studies
in the History and Philosophy of Modern Physics, in press.

Laura Ruetsche is another student of Earman.  At the conference,
she gave a nice talk about what it means when a quantum theory 
formulated in terms of C*-algebras has many inequivalent Hilbert 
space representations. 

Here's another paper by the Earman gang:

4) Gordon Belot, John Earman and Laura Ruetsche, The Hawking 
information loss paradox: the anatomy of a controversy, 
British Journal for the Philosophy of Science, 50 (1999), 189-230.

I haven't read it yet, but I heard John Earman talk about the subject 
in Vancouver in 1999 and he made a lot of sense.  He emphasized that
talk of a "paradox" is overblown: there's no reason information 
\emph{needs}
to be conserved in the process of Hawking radiation.  Most physicists
wish it were, though.

Another philosopher I enjoyed speaking to was Alisa Bokulich of Boston
University.  She mentioned some fascinating things about how one can 
calculate the spectrum of the helium atom in terms of the dynamics of 
the \emph{classical} three-body problem.  This is precisely where the "old 
quantum mechanics" of Bohr and Sommerfeld gave up - their ideas only 
worked for completely integrable systems, where all the orbits are
periodic.   The three-body problem is not completely integrable: it
exhibits chaos, so there are lots of nonperiodic orbits, with periodic 
orbits densely woven among them.

But the old quantum mechanice has recently experienced a kind of 
renaissance thanks to work on quantum chaos.  Apparently now
people can compute the energy levels of the \emph{quantum} version of 
a chaotic system in terms of a sum over the periodic orbits of the 
corresponding \emph{classical} system!  You use something called the 
"Gutzwiller trace formula", and maybe some other stuff...

I don't understand this, but I want to - especially because thanks
to this "trace formula" business, there are tantalizing connections 
to the Riemann hypothesis!  People like Michael Berry have hinted 
that maybe someone could solve this famous open problem if they
found a chaotic dynamical system with orbits having periods related
to the prime numbers in the right way... or \emph{something} like that; 
by now I'm just babbling half-forgotten second-hand gossip.  

Anyway, Bokulich gave me some references that I plan to read.  First, 
a thorough historical review of the subject:

5) G. Tanner, K. Richter and J. Rost, The theory of two-electron 
atoms: between ground state and complete fragmentation, Reviews of 
Modern Physics 72 (2000), 497-544.

Then, a classic paper extolling the forgotten virtues of the
old quantum theory:

6) J. Leopold and I. Percival, The semiclassical two-electron 
atom and the old quantum theory, Jour. Phys. B13 (1980) 1037-1047.

Next, a paper containing the first successful semiclassical 
quantization of helium:

7) G. Ezra, K. Richter, G. Tanner, and D. Wintgen, Semiclassical 
cycle expansion for the helium atom, Journal of Physics B 24 (1991), 
L413-L420.  

If you don't know anything about the old quantum mechanics,
here's a good place to start - it begins with a long
explanation and then has translations of original papers:
8) D. ter haar, The Old Quantum Theory, Pergamon Press, London, 
1967.

And finally, a here's a modern online book on semiclassical
methods and quantum chaos, in the process of construction:

9) Predrag Cvitanovic, Roberto Artuso, Per Dahlqvist, Ronnie
Mainieri, Gregor Tanner, Gabor Vattay, Niall Whelan and Andreas
Wirzba, Chaos: Classical and Quantum, 
<A HREF = "http://www.nbi.dk/ChaosBook/">http://www.nbi.dk/ChaosBook/</A>
 
Cvitanovic is really big on these online books: he's almost done 
writing one about diagrammatic methods in group representation 
theory.  I should talk about this soon, because it contains some
exciting new insights on the exceptional groups.   But I'm not really 
ready yet, so for now I'll just throw you the reference:

10) Predrag Cvitanovic, Group Theory, <A HREF = "http://www.nbi.dk/GroupTheory/">http://www.nbi.dk/GroupTheory/</A>

Instead, let me talk some more about structure types and their generating
functions.  I described these concepts in "<A HREF = "week185.html">week185</A>", but I didn't give
many examples, which is a real pity.  I want to make up for that
omission now.  

First remember the basic idea.  A "structure type" is any sort of
structure that we can put on a finite set.  Given any structure type F,
we let F_{k} be the \emph{set} of ways we can put this structure on a k-element
set, and let |F_{k}| be the \emph{number} of ways we can do it.  We define the 
"generating function" |F| to be the formal power series


$$

               |F_{k}|  
|F|(x) = sum  -----  x^{k}
                k!
$$
    
Nice operations on generating functions come from nice operations on
structure types, so we use the same notation for both.  

For example: given structure types G and H, we define the structure 
type G+H by saying an G+H-structure on the set S consists of either a
G-structure on S or an H-structure on S.  This definition gives:


\begin{verbatim}

|G+H| = |G| + |H| 
\end{verbatim}
    
Or: we define the structure type GH by saying an GH-structure on S
consists of a way of chopping S into two disjoint subsets and putting 
a G-structure on the first subset and an H-structure on the second.  
If we make this definition, we get:


\begin{verbatim}

|GH| = |G| |H|
\end{verbatim}
    
Or: we can define a structure GoH by saying an GoH-structure on S
consists of a way of partitioning S into disjoint parts, putting 
a G-structure on the set of parts, and putting an H-structure on 
each part.  Then we get:


\begin{verbatim}

|GoH| = |G| o |H|
\end{verbatim}
    
where on the right "o" means that we're composing the generating
functions |G| and |H|.  Here we have to be a bit careful: the
composite of formal power series is not always a well-defined 
formal power series, so the above equation only works when the
right-hand side makes sense.

It's easy and highly instructive to check all the claims I just made. 
But let's see what cool stuff we can do with them!   

First, consider the structure of "being a totally ordered n-element
set".  There are \emph{no} 
ways to put this structure on a k-element set if
k is different from n, and there are n! ways to put it on an n-element 
set.  So the generating function of this structure type is just


$$

x^{n}
$$
    
If we call this structure type X^{n}, we get this cute equation:


$$

|X^{n}| = x^{n}
$$
    
Next, suppose G is the structure "being a totally ordered set".  This 
is the same as being a totally ordered 0-element set or being a totally
ordered 1-element set or being a totally ordered 2-element set or... 
you get the idea.  So, we have


$$

     G = 1 + X + X^{2} + ...
$$
    
and thus


$$

|G|(x) = 1 + x + x^{2} + ...
 
           1
       = -----
         1 - x
$$
    
Next, suppose H is the structure "being a totally ordered set with 1 
or 2 elements".  This has the generating function


$$

|H|(x) = x + x^{2}
$$
    
Now let's consider the structure type


\begin{verbatim}

F = GoH
\end{verbatim}
    
To put a structure of this type on a set, we partition the set, order the
parts, and give each part the structure of being a totally ordered set
with 1 or 2 elements.  This sounds a bit weird!  But if you think about
it, it means:

\begin{quote}
      "To put an F-structure on a set, order it and then partition
       it into parts of size 1 or 2."
\end{quote}
And we can count the ways of doing this by using this generating function:


$$

|F|(x)  = |G| o |H| (x)

              1
        = ----------
          1 - x - x^{2}

        = 1 + (x + x^{2}) + (x + x^{2})^{2} + (x + x^{2})^{3} + (x + x^{2})^{4} + ...

        = 1 + x + 2x^{2} + 3x^{3} + 5x^{4} + ...
$$
    
Hey!  Fibonacci numbers!  It looks like the kth coefficient of
this generating function is just the kth Fibonacci number!

Now, remember that generating functions have a factorial built into
them:


$$

             |F_{k}|  
|F|(x) =  &sum; -----  x^{k}
              k!
$$
    
So apparently in this example |F_{k}| is k! times the kth Fibonacci
number.  Of course, k! is the number of ways to order a k-element set.
So apparently the kth Fibonacci number is just the number of ways to
chop a k-element set into parts of size 1 or 2.  

But how can be \emph{sure} we're getting the Fibonacci numbers as coefficients?  
Well, after the checking the first couple of coefficients, we just need
to make sure that each coefficient in our generating function is the sum
of the previous two.  And that follows straight from this equation:


$$

          1              x               x^{2}
     ----------  =  -----------  +  -----------  + 1
     1 - x - x^{2}      1 - x - x^{2}      1 - x - x^{2}
$$
    
Even better, the above equation comes from an isomorphism between
structure types:


$$

          F      =       X F      +     X^{2} F     + 1
$$
    
Since X^{n} is the structure "being a totally ordered n-element set",
this isomorphism says:

\begin{quote}
      "To put an F-structure on a set S, either remove one element 
      from S and put an F-structure on the rest of S, or remove two 
      elements, order them, and put an F-structure on the rest of S,
      or check to see if S is the empty set - in which case it has
      exactly one F-structure, by definition."
\end{quote}
This recursive definition of F is a categorified version of the 
recursive definition of the Fibonacci numbers.  It gives perhaps the
most direct way to see that the number of ways of chopping an n-element 
set into parts of size 1 or 2 is equal to the nth Fibonacci number.   
It's pretty simple, and we might have discovered it without structure 
types - but we can get this sort of thing \emph{systematically} if we use
structure types.  

We also get other spinoffs.  For example, the pole of this function


$$

     1
 ---------- 
 1 - x - x^{2}
$$
    
that's closest to zero occurs at the reciprocal of the golden ratio:


\begin{verbatim}

1/G = 0.6180339...
\end{verbatim}
    
So, by a theorem of Hadamard, the nth coefficient of the corresponding
series


$$

1 + x + 2x^{2} + 3x^{3} + 5x^{4} + 8x^{5} + ...
$$
    
must grow roughly like G^{n}.  In other words, the Fibonacci numbers grow
roughly like powers of the golden ratio.  Now, this should not be news
to any true lover of mathematics!  And you can get far more precise
information along these lines without much more work.  But I'm just
trying to make a general point: in combinatorics, we can estimate how
fast the number of ways of doing something grows by studying poles of
generating functions.

For example, suppose you wanted to know approximately how many ways
there are to take a million dollars and break it down into 1, 5, and
10 dollar bills.  The generating function that solves this problem is


$$

           1       1         1
         ------  ------   -------
         1 - x   1 - x^{5}   1 - x^{10}
$$
    
I'll let you do the rest.  

Here's another classic example.  The number of binary trees with n leaves 
is called (annoyingly) the (n-1)st Catalan number.  There is a structure
type T where a T-structure on a set is a way of making it into the
leaves of a binary tree.  For example, here's a T-structure on the set
{a,b,c,d}:


\begin{verbatim}

              b   d    c   a
               \   \  /   /
                \   \/   /
                 \   \  /           
                  \   \/
                   \  /
                    \/
\end{verbatim}
    
The number of T-structures on an n-element set is n! times the
(n-1)st Catalan number, thanks to the different orderings.

To put a T-structure on a set, we either check to see that it has one
element, in which case there's a single T-structure, or chop it into
two parts and put a T-structure on each part.  This means that


$$

T = X + T^{2}
$$
    
and thus


$$

|T| = x + |T|^{2}
$$
    
so 


$$

|T|(x) = (1 - sqrt(1 - 4x))/2 

       =  x + x^{2} + 2x^{3} + 5x^{4} + 14x^{5} + 42x^{6} + ....
$$
    
so, for example, there are 42 binary trees with 6 leaves.  In fact, I did
this calculation already in "<A HREF = "week144.html">week144</A>", but I didn't explain it in terms
of structure types.  You can learn more about Catalan numbers there.

If you think this stuff is fun, ponder T(1).  This
corresponds naturally to the set of all trees. 
What's the cardinality of this set?  Well, the sensible answer
is to sum the series:

\begin{verbatim}

|T|(1) = 1 + 1 + 2 + 5 + 14 + 42 + ....
\end{verbatim}
    
In other words, infinity!  But if we were feeling
quite relaxed about everything, we might use the other formula
for |T|(x) and guess


$$

|T|(1) = (1 - sqrt(-3))/2

       = exp(-i \pi /6)
$$
    
This is pretty odd: it's a complex number!   The problem is, we're
outside the radius of convergence of the power series.
However, this answer is not \emph{completely} crazy: we can use
it to guess things that would be hard to guess otherwise!
For example, this number is a sixth root of unity, so if we
raise it to the seventh power, we get the same
number back again:

$$

|T|(1)^{7} = |T|(1)
$$
    

Categorifying this fact, Lawvere guessed there was indeed a nice
isomorphism 


$$

T(1)^{7} = T(1)
$$
    

In other words: one can take this weird calculation and use it
to construct a one-to-one correspondence between trees and 
7-tuples of trees!  For a good treatment see this paper by Blass:

11) Andreas Blass, Seven trees in one, Jour. Pure Appl. Alg. 103 (1995), 
1-21.  Also available at <A HREF = "http://www.math.lsa.umich.edu/~ablass/cat.html">http://www.math.lsa.umich.edu/~ablass/cat.html</A>
Recently, Leinster and Fiore have proved a very general theorem 
on how to reason rigorously with complex-valued "cardinalities":
12) Marcelo Fiore and Tom Leinster, Objects of categories as
complex numbers, available as <A HREF = "http://www.arxiv.org/abs/math.CT/0212377">math.CT/0212377</A>.

This explains the curious result of Lawvere and Blass, and
should be a good clue when it comes to a favorite puzzle of
mine: how can we categorify the complex numbers?

There's much more to say: I should discuss all this using more 
category theory, say how it's related to "operads", and so on...
but I'm sitting in a coffee shop and I shouldn't keep hogging
this computer, so I'll quit now.   Happy Boxing Day!

\par\noindent\rule{\textwidth}{0.4pt}
\textbf{Addendum:} Gordon McCabe sent me an email with some useful
extra references.  The second paper here is the talk John Earman 
gave at the Philosophy of Science Association meeting described above.

\begin{quote}
John,
Pleased to see Philosophy of Science making an appearance in the latest
'This Week's Finds'!
I noticed that there were no http references to the papers by Earman and
Belot that you allude to. Philosophers do seem to have been very slow to
catch on to this business of Internet preprints, but there is a growing
archive of electronic preprints hosted by the University of Pittsburgh at
<A HREF = "http://philsci-archive.pitt.edu/">http://philsci-archive.pitt.edu/</A>
    You can find a number of papers here by
Earman and Belot, which you might want to add as http references to 'This
Week's Finds':
Earman, John (2001) Gauge Matters.
<A HREF = "http://philsci-archive.pitt.edu/documents/disk0/00/00/00/70/index.html">http://philsci-archive.pitt.edu/documents/disk0/00/00/00/70/index.html</A>
Earman, John (2002) Laws, Symmetry, and Symmetry Breaking; Invariance,
Conservation Principles, and Objectivity.
<A HREF = "http://philsci-archive.pitt.edu/documents/disk0/00/00/08/78/index.html">http://philsci-archive.pitt.edu/documents/disk0/00/00/08/78/index.html</A>
Belot, Gordon (2002) Symmetry and Gauge Freedom.
<A HREF = "http://philsci-archive.pitt.edu/documents/disk0/00/00/05/27/index.html">http://philsci-archive.pitt.edu/documents/disk0/00/00/05/27/index.html</A>

Regards,
Gordon McCabe
\end{quote}



Also, someone noticed something funny about the following: 

\begin{verbatim}

>      "To put an F-structure on a set, order it and then partition
>       it into parts of size 1 or 2."
>
>And we can count the ways of doing this by using this generating function:
>
>|F|(x)  = |G| o |H| (x)
>
>                1
>        = ----------- 
>           1 - x - x^2
>
>        = 1 + (x + x^2) + (x + x^2)^2 + (x + x^2)^3 + (x + x^2)^4 + ...
>
>        = 1 + x + 2x^2 + 3x^3 + 5x^4 + ...
>
>Hey!  Fibonacci numbers!  It looks like the kth coefficient of
>this generating function is just the kth Fibonacci number!
>
>Now, remember that generating functions have a factorial built into
>them:
>
>              |F_k|  
>|F|(x) = sum  -----  x^k
>                k!
>
>So apparently in this example |F_k| is k! times the kth Fibonacci
>number.  Of course, k! is the number of ways to order a k-element set.
>So apparently the kth Fibonacci number is just the number of ways to
>chop a k-element set into parts of size 1 or 2.  
\end{verbatim}
    

What's funny is how the choice of orderings introduces a factor of k! 
whose only purpose in life is to cancel the 1/k! in the definition of 
"generating function".  
This guy knew that besides the generating functions I was discussing - 
sometimes called "exponential generating functions" - there are some 
other generating functions - sometimes called "ordinary generating 
functions" - whose definition doesn't have that 1/k! in it.   If I'd 
used those, I wouldn't have needed to play this cancellation game!
I knew that already, but I didn't want to confuse people by
introducing two flavors of generating function. 
But now that the subject has come up, I might as well say
something about it. 
The way I like to think about it, structure types are really
functors
F: C \to  Set
where C is the category of finite sets and bijections.  But we
also have "structure types on ordered sets" (don't know a good name for them) 
F: D \to  Set
where D is the category of linearly ordered finite sets and 
order-preserving bijections.  The exponential generating function
applies to structure types, and is defined as above.  The ordinary 
generating function applies to structure types on ordered sets,
and is defined by
|F|(x) = sum  |F_{k}|  x^{k}
It has many of the same nice properties as the exponential
generating function, as long as we careful to adapt everything
to the category D.   You can read all about this in the book
by Bergeron et al cited in <A HREF = "week185.html">"week185"</A>.
I claim that it's best to always insist on this viewpoint:
exponential generating functions for structure types, 
ordinary generating functions for structure types on ordered sets.
However, if you want to have fun (i.e. get confused) you can
convert structure types into structure types on ordered sets,
or vice versa, before you take the generating function!
After all, there is a forgetful functor from D to C.   This induces
a functor from hom(C,Set) to hom(D,Set): given a structure on a
set S, we automatically get a structure on any linearly ordered set 
we obtain by slapping an ordering on S.   Furthermore, \emph{this} functor 
has an adjoint - in fact, both right and left adjoints.  
In short, there are three ways to hop back and forth between structure 
types and structure types on ordered sets, which allow you to get
very confused about which you are working with at any given moment. 
To add to the fun (i.e. confusion), there are some formulas relating 
the exponential generating functions of the former to the ordinary 
generating functions of the latter.  I was implicitly using one of 
these above.  So, if you want to become deconfused, you 
should figure out these formulas.
And if you want to do it in an elegant way: 
Both "structure types" and "ordered structure types" 
form 2-rigs - 
i.e. categories with + and x, satisfying some obvious ring-ish 
axioms up to isomorphism, but without additive inverses.
Let's call these 2-rigs hom(C,Set) and hom(D,Set).  
If we decategorify a 2-rig we get a rig, so there are rigs I'll call 
|hom(C,Set)| and |hom(D,Set)|.  Elements of the first are just isomorphism
classes of structure types; elements of the second are isomorphism
classes of ordered structure types; in both cases the + and x 
operations are hopefully obvious.
Now, the exponential generating function is best thought of as
a rig homomorphism
egf: |hom(C,Set)| \to  N{{x}}
where N{{x}} is the rig of formal power series where the
coefficient of the nth term is a natural number divided by n!,
while the ordinary generating function is best thought of as
a rig homomorphism
ogf: |hom(D,Set)| \to  N[[x]]
The relations between exponential and ordinary generating 
functions are really relations between the rigs |hom(C,Set)| 
and |hom(D,Set)|.  And these, in turn, are \emph{really} relations 
between the 2-rigs hom(C,Set) and hom(D,Set).  
I've already said that there is a functor 
hom(C,Set) \to  hom(D,Set)
and two going the other way.  The question is, which of these
functors are 2-rig homomorphisms?  I.e., which get along with
+ and x?  These are the ones where there will be \emph{very} nice
relations between generating functions - namely, relations 
that get along with + and x.
I leave this as a little puzzle, partially because I am
too lazy to work out the answer and explain it nicely.
But for category mavens, here's an extra hint.  To see if
these functors between hom(C,Set) and hom(D,Set) are 2-rig
homomorphisms, we need to see whether they preserve + (colimits)
and x (the monoidal structure).  
Preserving colimits is a very general question.
Given any functor from D to C we always 
get three functors going between the categories 
hom(C,Set) and hom(D,Set), and the question is: which of 
these preserve colimits?
Preserving the monoidal structure is a slightly less general 
(but still bloody frigging general!) question.  The point is that
C and D are monoidal categories and hom(C,Set) and hom(D,Set)
get their multiplication from that, via a trick called "Day
convolution", which is just a categorified version of ordinary 
convolution of functions.  (By now I'm at Macquarie University
in Australia, and Brian Day's office is right across the hall,
so I had to say this.)  
So, here the question is: when you have a \emph{monoidal} functor from 
D to C, as we do here, which of the three functors between hom(C,Set) 
and hom(D,Set) are monoidal with respect to Day convolution?
As usual, I learned most of this category theory stuff from 
James Dolan, so any errors in the above are his fault, not mine.  

\par\noindent\rule{\textwidth}{0.4pt}
% </A>
% </A>
% </A>
