
% </A>
% </A>
% </A>
\week{November 26, 1995}


Probably many of the mathematicians reading this know about the Newton
Institute in Cambridge, a mathematics institute run by Sir Michael
Atiyah.  It's a cozy little building, in a quiet neighborhood a certain
distance from the center of town, which one can reach by taking a nice
walk or bike ride over the bridge near Trinity College, across Grange
Road, and down Clarkson Road.  Inside it's one big space, with stairways
slightly reminiscent of a certain picture by Escher, with a nice little
library on the first floor, tea and coffee on the 3rd floor, blackboards
in the bathrooms... everything a mathematician could want.  This is
where Wiles first announced his proof of Fermat's last theorem, and they
sell T-shirts there commemorating that fact, which are unfortunately too
small to contain the proof itself... as they do not refrain from
pointing out.  

I just got back from a conference there on New Connections between
Mathematics and Computer Science.  It was organized by Jeremy
Gunawardena, who was eager to expose computer scientists and
mathematicians to a wide gamut of new interactions between the two
subjects.  I spoke about n-categories in logic, topology and physics.
Since I don't know anything about computer science, when I first got the
invitation I thought it was a mistake: a wrong email address or
something!  But Gunawardena assured me otherwise.  I assumed the idea
was that n-categories, being so abstract, must have \emph{some} application
to just about \emph{everything}, even computer science.  Luckily, some other
speakers at the conference gave some very nice applications of
n-category theory to computer science, so now I know they really exist.




Unfortunately I had to miss the beginning of the conference, and
therefore missed some interesting talks of a geometrical nature by
Smale, Gromov, Shub and others.  Let me say a bit about some of the
talks I did catch.  You can find a list of all the speakers and abstracts 
of their talks at
1) Basic Research Institute in the Mathematical Sciences, 
New Connections web page, <A HREF = "http://www-uk.hpl.hp.com/brims/">
http://www-uk.hpl.hp.com/brims/</A>

Richard Jozsa gave an interesting talk on quantum computers, in part
outlining Peter Shor's work (see "<A HREF = "week34.html">week34</A>") 
on efficient factoring via
quantum computation, but also presenting some new results on
"counterfactual quantum computation".  It turns out that - in
principle - in some cases you can get a quantum computer to help you
answer a question, even without running it, just as long as you COULD
HAVE run it!  (I should add that in practice a lot of things make this
quite impractical.)  This is a new twist on the Elitzur-Vaidman bomb-testing
paradox about how if you have a bunch of bombs and half of them are
duds, and the only way you can test a bomb is by lighting the fuse and
seeing if it goes off, you can still get a bomb you're sure will work,
if you use quantum mechanics.  The trick involves getting a fuse that's
so sensitive that even one photon will make the bomb go off, and then
setting up a beam-splitter, and using the bomb to measure which path
the photon followed, before recombining the beams.  Check out:

2) A. C. Elitzur and L. Vaidman, Quantum mechanical interaction-free
measurements, Foundations of Phys. 23 (1993), 987-997.

Graeme Mitchison and Richard Jozsa, Counterfactual quantum computation,
Proc. Roy. Soc. Lond. A457 (2001) 1175-1194.  Also available as
<A HREF = "http://arxiv.org/abs/quant-ph/9907007">quant-ph/9907007</A>.

Jean-Yves Girard gave an overview of linear logic.  Linear logic is a
new version of logic that he invented, which has some new operations
besides the good old ones like "and", "or", and "not".  For example,
there are things like "par" (written as an upside-down ampersand), "!"
(usually pronounced "bang") and "?".  Ever since I started going to
conferences on category theory and computer science I have been hearing
a lot about it, and I keep trying to get people to explain these weird
new logical operations to me.  Unfortunately, I keep getting very
different answers, so it has remained rather mysterious to me, even
though it seems like a lot of fun (see "<A HREF = "week40.html">week40</A>").  Thus I was eager to
hear it from the horse's mouth.

Indeed, Girard gave a fascinating talk on it which almost made me feel I
understood it.  I think the big thing I'd been missing was a good
appreciation of topics in proof theory like "cut elimination".  He noted
that this subject usually appears to be all about the precise
manipulation of formulas according to purely syntactic rules: "Very
bureaucratic" he joked, "one parenthesis missing and you've had it!"
(For full effect, one must imagine this being said in a French accent by
someone stylishly dressed entirely in black.)  He wanted to get a more
\emph{geometrical} way to think about proofs, but to do this it turned out to
be important to refine ordinary logic in certain ways.... leading to
linear logic.  However, I still don't feel up to explaining it, so let
me turn you to:

3) Jean-Yves Girard, Linear logic, Theoretical Computer Science 50,
1-102, 1987.

Jean-Yves Girard, Y. Lafont and P. Taylor, Proofs and Types, Cambridge
Tracts in Theoretical Computer Science 7, Cambridge U. Press, 1989.
Also available at <A HREF = "http://www.cs.man.ac.uk/~pt/stable/Proofs+Types.html">http://www.cs.man.ac.uk/~pt/stable/Proofs+Types.html</A>

Eric Goubault and Vaughan Pratt talked, in somewhat different ways,
about a formalism for treating concurrency using "higher-dimensional
automata".  The basic idea is simple: say we have two jobs to do, one of
which gets us from some starting-point A to some result B, and the other
of which gets us from A' to B'.  We can represent each task by an arrow,
as follows:


\begin{verbatim}

A ----> B


A' ---> B'
\end{verbatim}
    

We can think of this arrow as a "morphism", that is, a completely abstract
sort of operation taking A to B.  Or, we can think of it more concretely
as an interval of time, where our computer is doing something at each
moment.  Alternatively, we can think of it more discretely as a sequence
of steps, starting with A and winding up with B.  

If we now consider doing both these tasks concurrently, we can represent the
situation by a square:


\begin{verbatim}

AA' -----> BA'
 |          |
 |          |
 |          |
 |          |
 v          v
AB' -----> BB'
\end{verbatim}
    

Going first across and then down corresponds to completing one task
before starting the other, while going first down and then across
corresponds to doing the other one first.  However, we can also imagine
various roughly diagonal paths through the square, corresponding to
doing both tasks at the same time.  We might go horizontally for a
while, then vertically, then diagonally, and so on.  Of course, if the
two tasks were not completely independent - for example, if some steps
of one could only occur after some steps of the other were finished -
we would have some constraints on what paths from AA' to BB' were
allowed.  The idea is then to model these constaints as "holes" in the
square, forbidden regions where the path cannot go.  There may then be
several "essentially distinct" ways of getting from AA' to BB', that is,
classes of paths that cannot be deformed into each other.  

To anyone who knows homotopy theory, this will seem very familiar,
homotopy theory being all about spaces with holes in them, and how those
holes prevent you from continuously deforming one path into another.
Goubault's title, "Scheduling problems and homotopy theory", emphasized
the relationships.  But there are also some big differences.  Unlike
homotopy theory, here the paths are typically required to be
"monotonic": they can't double back and go backwards in time.  And, as I
mentioned, the tasks can be thought of more abstractly than as paths in
some space.  So we are really talking about 2-categories here: they give
a general framework for studying situations with "dots" or "objects",
"arrows between dots" or "morphisms", and "arrows between arrows between
dots" or "2-morphisms".  Similarly, when we study concurrency with more
than 2 tasks at a time we can think of it in terms of n-categories.

By the way, since I don't know much about parallel processing, I'm not
sure how much the above formalism actually helps the "working man".
Probably not much, yet.  I get the impression, however, that parallel
processing is a complicated problem, and that people are busily dreaming
up new formalisms for talking about it, hoping they will eventually
be useful for inventing and analyzing parallel programming languages.

Some references for this are:

4) Eric Goubault, Schedulers as abstract interpretations of higher-dimensional
automata, in Proc. PEPM '95 (La Jolla), ACM Press, 1995.
Also available at <A HREF = "http://www.di.ens.fr/%7Egoubault/GOUBAULTpapers.html">http://www.di.ens.fr/%7Egoubault/GOUBAULTpapers.html</A>

Eric Goubault and Thomas Jensen, Homology of higher-dimensional automata,
in Proc. CONCUR '92 (New York), Lecture Notes in Computer Science 630, Springer,
1992.   Also available at <A HREF = "http://www.di.ens.fr/%7Egoubault/GOUBAULTpapers.html">http://www.di.ens.fr/%7Egoubault/GOUBAULTpapers.html</A>

5) Vaughan Pratt, Time and information in sequential and concurrent
computation, in Proc. Theory and Practice of Parallel Programming, Sendai,
Japan, 1994.

Yves Lafont also gave a talk with strong connections to n-category
theory.  Recall that a monoid is a set with an associative product
having a unit element.  One way to describe a monoid is by giving a
presentation with "generators", say

a, b, c, d,

and "relations", say 

ab = a, da = ac.

We get a monoid out of this in an obvious sort of way, namely by taking
all strings built from the generators a,b,c, and d, and then identifying
two strings if you can get from one to the other by repeated use of the
relations.  In math jargon, we form the free monoid on the generators
and then mod out by the relations.  

Suppose our monoid is finitely presented, that is, there are finitely
many generators and finitely many relations.  How can we tell whether
two elements of it are equal?  For example, does

dacb = acc 

in the above monoid?  Well, if the two are equal, we will always
eventually find that out by an exhaustive search, applying the relations
mechanicallly in all possible ways.  But if they are not, we may never
find out!  (For the above example, the answer appears at the end of this
article in case anyone wants to puzzle over it.  Personally, I can't
stand this sort of puzzle.)  In fact, there is no general algorithm for
solving this "word problem for monoids", and in fact one can even write
down a \emph{specific} finitely presented monoid for which no algorithm
works.  

However, sometimes things are nice.  Suppose you write the relations
as "rewrite rules", that go only one way:

ab \to  a
da \to  ac

Then if you have an equation you are trying to check, you can try to
repeatedly apply the rewrite rules to each side, reducing it to "normal
form", and see if the normal forms are equal.  This will only work,
however, if some good things happen!  First of all, your rewrite rules
had better terminate: it had better be that you can only apply them
finitely many times to a given string.  This happens to be true for
the above pair of rewrite rules, because both rules decrease the number
of b's and c's.  Second of all, your rewrite rules had better be
confluent: it had better be that if I use the rules one way until I
can't go any further, and you use them some other way, that we both wind
up with the same thing!  If both these hold, then we can reduce any
string to a unique normal form by applying the rules until we can't do
it any more.  

Unfortunately, the rules above aren't confluent; if we start with 
the word  dab , you can apply the rules like this

dab \to  acb 

while I apply them like this

dab \to  da \to  ac

and we both terminate, but at different answers.  We could try to cure
this by adding a new rule to our list, 

acb \to  ac.

This is certainly a valid rule, which cures the problem at hand...
but if we foolishly keep adding new rules to our list this way we may
only succeed in getting confluence and termination when we have an
\emph{infinite} list of rules: 

ab \to  a 

da \to  ac 

acb \to  ac 

accb \to  acc 

acccb \to  accc 

accccb \to  acccc... 


and so on.  I leave you to check that this is really terminating and
confluent.  Because it is, and because it's a very predictable list of
rules, we can use it to write a computer program in this case to solve
the word problem for the monoid at hand.  But in fact, if we had been
cleverer, we could have invented a \emph{finite} list of rules that was
terminating and confluent:

ab \to  a 

ac \to  da

Lafont's went on to describe some work by Squier:

6) Craig C. Squier, Word problems and a homological finiteness condition
for monoids, Jour. Pure Appl. Algebra 49 (1987), 201-217.

Craig C. Squier, A finiteness condition for rewriting systems, revision by F.
Otto and Y. Kobayashi, to appear in Theoretical Computer Science.

Craig C. Squier and F. Otto, The word problem for finitely presented
monoids and finite canonical rewriting systems, in Rewriting Techniques 
and Applications, ed. J. P. Jouannuad, Lecture Notes in Computer Science 
256, Springer, Berlin, 1987, 74-82.

which gave general conditions which must hold for there to be a finite
terminating and confluent set of rewrite rules for a monoid.  The nice
thing is that this relies heavily on ideas from n-category theory.
Note: we started with a monoid in which the relations are \emph{equations},
but we then started thinking of the relations as rewrite rules or
\emph{morphisms}, so what we really have is a monoidal category.  We then
started worrying about "confluences", or equations between these
morphisms.  This is typical of "categorification", in which equations
are replaced by morphisms, which we then want to satisfy new equations
(see "<A HREF = "week38.html">week38</A>").

For the experts, let me say exactly how it all goes.  Given any monoid M,
we can cook up a topological space called its "classifying space" KM,
as follows. We can think of KM as a simplicial complex.  We start by
sticking in one 0-simplex, which we can visualize as a dot like this:

\begin{verbatim}

O
\end{verbatim}
    
Then we stick in one 1-simplex for each element of the monoid, which we
can visualize as an arrow going from the dot to itself.  Unrolled a bit,
it looks like this: 

\begin{verbatim}

O---a---O
\end{verbatim}
    
Really we should draw an arrow going from left to right, but soon
things will get too messy if I do that, so I won't.  Then, whenever we
have ab = c in the monoid, we stick in a 2-simplex, which we can
visualize as a triangle like this:

\begin{verbatim}

      O
     / \
    a   b
   /     \
  O---c---O
\end{verbatim}
    
Then, whenever we have abc = d in the monoid, we stick in a 3-simplex,
which we can visualize as a tetrahedron like this

% </A>
% </A>
% </A>
