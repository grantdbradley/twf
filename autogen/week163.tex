
% </A>
% </A>
% </A>
\week{December 31, 2000 }


If you think numbers start with the number 1, you probably think the
millennium is ending now.  I think it ended last year... but either 
way, now is a good time to read this book:

1) Georges Ifrah, The Universal History of Numbers from Prehistory to
the Invention of the Computer, Wiley, New York, 2000.

On the invention of zero:

\begin{quote}
     Most peoples throughout history failed to discover the rule of
     position, which was discovered in fact only four times in the 
     history of the world.  (The rule of position is the principle in
     which a 9, let's say, has a different magnitude depending on 
     whether it comes in first, second, third... position in a numerical 
     expression.)  The first discovery of this essential tool of  
     mathematics was made in Babylon in the second millennium BCE.  
     It was then rediscovered by the Chinese arithmeticians at around
     the start of the Common Era.  In the third to fifth centuries CE,
     Mayan astronomers reinvented it, and in the fifth century CE it
     was rediscovered for the last time, in India.

     Obviously, no civilization outside of these four ever felt the 
     need to invent zero; but as soon as the rule of position became
     the basis for a numbering system, a zero was needed.  All the same,
     only three of the four (the Babylonians, the Mayans, and the 
     Indians) managed to develop this final abstraction of number;
     the Chinese only acquired it through Indian influences.  However,
     the Babylonian and Mayan zeroes  were not conceived of as numbers,
     and only the Indian zero had roughly the same potential as the 
     one we use nowadays.  That is because it is indeed the Indian zero,
     transmitted to us through the Arabs together with the number-symbols
     that we call Arabic numerals and which are in reality Indian numerals,
     with their appearance altered somwhat by time, use and travel.

\end{quote}
Among other things, this book has wonderful charts showing the
development of each numeral.  You can see, for example, how the
primitive numeral

\begin{verbatim}

                          ____
                          ____
                          ____

\end{verbatim}
    
slowly evolved to our modern "3".  Hmm - how come this doesn't feel
like progress?

Now, I usually keep my eyes firmly focused on the beauties of nature,
but once in a millennium I feel the need to engage in some politics.  
So....

In "<A HREF = "week155.html">week155</A>" I talked a lot about polyhedra and their 4-dimensional
generalizations, and I referred to Eric Weisstein's online math
encyclopedia since it had lots of nice pictures.  Now this website 
has been closed down, thanks to a lawsuit by the people at CRC Press:

2) Frequently asked questions about the MathWorld case, <a href = "http://mathworld.wolfram.com/docs/faq.html">http://mathworld.wolfram.com/docs/faq.html</a>

Weisstein published a print version of his encyclopedia with CRC press,
but now they claim to own the rights to the online version as well.  So
I urge you all to remember this: <em>when dealing with publishers, never 
sign away the electronic rights on your work unless you're willing to 
accept the consequences!</em>

For example, suppose you write a math or physics paper and put it on the
preprint archive, and then publish it in a journal.  They'll probably
send you a little form to sign where you hand over the rights to this
work - including the electronic rights.  If you're like most people,
you'll sign this form without reading it.  This means that if they feel
like it, they can now sue you to make you take your paper off the
preprint archive!  Journals don't do this yet, but as they continue
becoming obsolete and keep fighting ever more desperately for their
lives, there's no telling what they'll do.  Corporations everywhere are
taking an increasingly aggressive line on intellectual property rights -
as the case of Weisstein shows.

So what can you do?  Simple: don't agree to it.  When you get this form,
cross out any sentences you refuse to agree to, put your initials by
these deletions, and sign the thing - indicating that you agree to the
\emph{other} stuff!  Keep a copy.  If they complain, ask them how much these
electronic rights are worth.

Basically, I think it's time for academics to take more responsibility
about keeping their work easily accessible.  

There are lots of things you can do.  One of the easiest is to stop
refereeing for ridiculously expensive journals.  Journal prices bear
little relation to the quality of service they provide.  For example,
the Elsevier-published journal "Nuclear Physics B" costs $12,596 per
year for libraries, or $6,000 for a personal subscription.  The
comparable journal "Advances in Theoretical and Mathematical Physics"
costs $300 for libraries or $80 for a personal subscription - and
access to the electronic version is free.  So when Nuclear Physics B
asks me to referee manuscripts, I now say "Sorry, I'll wait until your
prices go down."  

In fact, I no longer referee articles for any journals published by
Elsevier, Kluwer, or Gordon \text{\&}  Breach.  If you've looked at their prices,
you'll know why.  G\text{\&} B has even taken legal action against the American
Institute of Physics, the American Physical Society, and the American
Mathematical Society for publishing information about journal prices!

3) Gordon and Breach et al v. AIP and APS, brief of amici curiae of the
American Library Association, Association of Research Libraries and
the Special Library Association, <a href = "http://www.arl.org/scomm/gb/amici.html">http://www.arl.org/scomm/gb/amici.html</a>

4) AIP/APS prevail in suit by Gordon and Breach, G\text{\&} B to appeal,
<a href = "http://www.arl.org/newsltr/194/gb.html">http://www.arl.org/newsltr/194/gb.html</a>

Of course, the ultimate solution is to support the math and physics
preprint archives, and figure out ways to decouple the refereeing 
process from the distribution process.

Okay, enough politics.  I was thinking about 4-dimensional polytopes,
and Eric Weisstein's now-defunct website... but what got me going in the
first place was this:

5) John Stilwell, The story of the 120-cell, AMS Notices 48 (January 2001),
17-24.

The 120-cell is a marvelous 4-dimensional shape with 120 regular
dodecahedra as faces.  I talked about it in "<A HREF =
"week155.html">week155</A>", but this article is full of additional
interesting information.  For example, Henri Poincare once conjectured
that every compact 3-manifold with the same homology groups as a
3-sphere must \emph{be} a 3-sphere.  He later proved himself wrong by
finding a counterexample: the "Poincare homology 3-sphere".
This is obtained by identifying the opposite faces of the dodecahedron
in the simplest possible way.  What I hadn't known is that the
fundamental group of this space is the "binary icosahedral
group", I.  This is the 120-element subgroup of SU(2) consisting of
all elements that map to rotational symmetries of the icosahedron under
the two-to-one map from SU(2) to SO(3).  Now SU(2) is none other than
the 3-sphere... so it follows that SU(2)/I is the Poincare homology
3-sphere!

When cosmologists study the possility that universe is finite in size,
they usually assume that space is a 3-sphere.  In this scenario, 
barring sneaky tricks, it's likely that the universe would recollapse
before light could get all the way around the universe.   But there's no
strong reason to favor this topology.  Some people have checked to see
whether space is a 3-dimensional torus.  In such a universe, light might
wrap all the way around - so you might see the same bright quasars by
looking in various different directions!  People have looked for this
effect but not seen it.   This doesn't rule out a torus-shaped universe,
but it puts a limit on how small it could be.  

In fact, some physicists have even considered the possibility that space
is a Poincare homology 3-sphere!  Can light go all the way around in
this case?  I don't know.  If so, we might see bright quasars in a
pretty dodecahedral pattern.  

Amusingly, Plato hinted at something resembling this in his "Timaeus":

6) Plato, Timaeus, translated by B. Jowett, in The Collected Dialogues,
Princeton U. Press, Princeton, 1969 (see line 55c).

This dialog is one the first attempts at doing mathematical physics.  
In it, the Socrates character guesses that the four elements earth, 
air, water and fire are made of atoms shaped like four of the five 
Platonic solids: cubes, octahedra, icosahedra and tetrahedra, respectively.   
Why?  Well, fire obviously feels hot because of those pointy little 
tetrahedra poking you!  Water is liquid because of those round little 
icosahedra rolling around.  Earth is solid because of those little cubes 
packing together so neatly.  And air... well, ahem... we'll get back to 
you on that one.

\emph{But what about the dodecahedron?}  On this topic, Plato makes only the
following cryptic remark: "There was yet a fifth combination which God
used in the delineation of the universe with figures of animals."  

Huh???  I think this is a feeble attempt to connect the 12 sides of the
dodecahedron to the 12 signs of the zodiac.  After all, lots of the
signs of the zodiac are animals.  The word "zodiac" comes from the
Greek phrase "zodiakos kuklos", or "circle of carved figures" - where
"zodiakos" or "carved figure" is really the diminutive of "zoion",
meaning "animal".  There may even be a connection between the
dodecahedron and the "quintessence": the fifth element, of which the
heavenly bodies were supposedly made.  I know, this is all pretty weird,
but there seems to be some tantalizingly murky connection between the
dodecahedron and the heavens in Greek cosmology.... so it would be cool
if space turned out to be a Poincare homology 3-sphere.  But of course,
there's no reason to believe it is.

Okay, enough goofing around.  Now let me talk a bit about the
exceptional Jordan algebra and the octonionic projective plane.  I'll
basically pick up where I left off in "<A HREF = "week162.html">week162</A>" - but you might want to
reread "<A HREF = "week61.html">week61</A>", "<A HREF = "week106.html">week106</A>" and "<A HREF = "week145.html">week145</A>" to prepare yourself for the
weirdness to come.   Also, keep in mind the following three facts about
the number 3, which fit together in a spooky sort of synergy that makes
all the magic happen:

i) An element of h_{3}(O) is a 3x3 hermitian matrix with octonionic
entries, and thus consists of 3 octonions and 3 real numbers:


\begin{verbatim}

                   ( a    z*   y* )
                   ( z    b    x  )          a,b,c in R,  x,y,z in O
                   ( y    x*   c  )            
\end{verbatim}
    
ii) The octonions arise naturally from "triality": the
relation between the three 8-dimensional irreps of Spin(8), i.e. the
vector representation V_{8}, the right-handed spinor
representation S_{8}^{+}, and the left-handed spinor
representation S_{8}^{-}.

iii) The associative law (xy)z = x(yz) involves 3 variables.

Let's see how it goes.


First, if we take the 3 octonions in our element of h_{3}(O) and
identify them with elements of the three 8-dimensional irreps of
Spin(8), we get


$$

h_{3}(O) = R^{3}  +  V_{8}  +  S_{8}^{+}  +  S_{8}^{-} 
$$
    

A little calculation then reveals a wonderful fact: while superficially
the Jordan product in h_{3}(O) is built using the structure of O
as a normed division algebra, it can actually be defined using just the
natural map


$$

t: V_{8}  x  S_{8}^{+}  x  S_{8}^{-} \to  R
$$
    

and the inner products on these 3 spaces.  It follows that any element
of Spin(8) gives an automorphism of h_{3}(O).  Indeed, Spin(8)
becomes a subgroup of Aut(h_{3}(O)).

So the exceptional Jordan algebra has a lot to do with geometry in 8
dimensions - that's not surprising.  What's surprising is that it also
has a lot to do with geometry in 9 dimensions!  When we restrict the
spinor and vector representations of Spin(9) to the subgroup Spin(8),
they split as follows:


$$

S_{9} = S_{8}^{+}  +  S_{8}^{-}

V_{9} = R  +  V_{8}
$$
    
This gives an isomorphism


$$

h_{3}(O) = R^{2}  +  V_{9}  +  S_{9}
$$
    
and in fact the product in h_{3}(O) can be described in terms of
natural maps involving scalars, vectors and spinors in 9 dimensions.  It
follows that Spin(9) is also a subgroup of Aut(h_{3}(O)).

This does not exhaust all the symmetries of h_{3}(O), since
there are other automorphisms coming from the permutation group on 3
letters, which acts on (a,b,c) in R^{3} and (x,y,z) in
O^{3} in an obvious way.  Also, any matrix g in the orthogonal
group O(3) acts by conjugation as an automorphism of h_{3}(O);
since the entries of g are real, there is no problem with
nonassociativity here.  The group Spin(9) is 36-dimensional, but the
full automorphism group h_{3}(O) is 52-dimensional.  In fact, it
is the exceptional Lie group F_{4}!

However, we can already do something interesting with the automorphisms
we have: we can use them to diagonalize any element of h_{3}(O).   To
see this, first note that the rotation group, and thus Spin(9), acts
transitively on the unit sphere in the vector representation V_{9}.  
This means we can use an automorphism in our Spin(9) subgroup to bring 
any element of h_{3}(O) to the form


\begin{verbatim}

                     ( a    z*   y* )
                     ( z    b    x  )         
                     ( y    x*   c  )           
\end{verbatim}
    
where x is \emph{real}.  The next step is to apply an automorphism that makes
y and z real while leaving x alone.  To do this, note that the subgroup
of Spin(9) fixing any nonzero vector in V_{9} is isomorphic to
Spin(8).  When we restrict the representation S_{9} to this
subgroup it splits as S_{8}^{+} +
S_{8}^{-}, and with some work one can show that Spin(8)
acts on S_{8}^{+} + S_{8}^{-} =
O^{2} in such a way that any element (y,z) in O^{2} can
be carried to an element with both components real.  The final step is
to take our element of h_{3}(O) with all real entries and use an
automorphism to diagonalize it.  We can do this by conjugating it with a
suitable matrix in O(3).

To understand the octonionic projective plane, we need to understand
projections in h_{3}(O).  Here is where our ability to
diagonalize matrices in h_{3}(O) via automorphisms comes in
handy.  Up to automorphism, every projection in h_{3}(O) looks
like one of these four guys:


$$

       ( 0    0    0 )        
       ( 0    0    0 ) = p_{0} 
       ( 0    0    0 )


       ( 1    0    0 )        
       ( 0    0    0 ) = p_{1} 
       ( 0    0    0 ) 


       ( 1    0    0 )        
       ( 0    1    0 ) = p_{2} 
       ( 0    0    0 )


       ( 1    0    0 )       
       ( 0    1    0 ) = p_{3} 
       ( 0    0    1 )
$$
    

Now, the trace of a matrix in h_{3}(O) is invariant under
automorphisms, because we can define it using only the Jordan algebra
structure:


$$

tr(a) = (1/3) tr(L_{a})    
$$
    
where L_{a} is left multiplication by a.  It follows that the trace
of any projection in h_{3}(O) is 0, 1, 2, or 3.  

Remember from "<A HREF = "week162.html">week162</A>" that the
"dimension" of a projection p in a formally real Jordan
algebra is the largest number d such that there's a chain of projections


$$

         p_{0} < p_{1} < ... < p_{d} = p  
$$
    
In favorable cases, like the exceptional Jordan algebra, the dimension-1
projections become the points of a projective plane, while the
dimension-2 projections become the lines.  But what's a practical way to
compute the dimension of a projection?  Well, in h_{3}(O) the
dimension equals the trace.

Why?  

Well, clearly the dimension is less than or equal to the trace, since p
< q implies tr(p) < tr(q), and the trace goes up by integer steps.
But on the other hand, the trace is less than or equal to the dimension.
To see this it suffices to consider the four projections shown above,
since both trace and dimension are invariant under automorphisms.  Since
p_{0} < p_{1} < p_{2} < p_{3}, it is clear that for these projections
the trace is indeed less than or equal to the rank.

So: the points of the octonionic projective plane are the projections
with trace 1 in h_{3}(O), while the lines are projections with
trace 2.  A brutal calculation in Reese Harvey's book:

7) F. Reese Harvey, Spinors and Calibrations, Academic Press, Boston, 1990.

reveals that any projection with trace 1 has the form


$$

p = |\psi > <\psi | = ( xx*   xy*    xz* )
                  ( yx*   yy*    yz* )
                  ( zx*   zy*    zz* )
$$
    
where 


$$

|\psi > = (x,y,z) 
$$
    
is a unit vector in O^{3} for which (xy)z = x(yz).  This is
supposed to remind you of stuff about spinors and the heavenly sphere in
"<A HREF = "week162.html">week162</A>".

On the other hand, any projection with trace 2 is of the form 1 - p
where p has trace 1.  This sets up a one-to-one correspondence between
points and lines in the octonionic projective plane.  If we use this
correspondence to think of both as trace-1 projections, the point p
lies on the line q if and only if p < 1 - q.  Of course, p < 1 - q
iff q < 1 - p.   The symmetry of this relation means the octonionic
projective plane is self-dual!  This is also true of the real, complex
and quaternionic projective planes.  In all cases, the operation that
switches points and lines corresponds in quantum logic to "negation".

Let's use OP^{2} to stand for the set of points in the
octonionic projective plane.  Given any nonzero element (x,y,z) in
O^{3} with (xy)z = x(yz), we can normalize it and then use the
above formula to obtain a point of OP^{2}, which we call
[(x,y,z)].  We can make OP^{2} into a smooth manifold by
covering it with three coordinate charts: one containing all points of
the form [(x,y,1)], one containing all points of the form [(x,1,z)], and
one containing all points of the form [(1,y,z)].  Checking that this
works is a simple calculation.  The only interesting part is to make
sure that whenever the associative law might appear necessary, we can
either use the weaker equations


\begin{verbatim}

(xx)y = x(xy) 
(xy)x = x(yx)
(yx)x = y(xx)
\end{verbatim}
    
which still hold for the octonions, or else the fact that only triples
with (xy)z = x(yz) give points [(x,y,z)] in OP^{2}.  

Clearly the manifold OP^{2} is 16-dimensional.  The lines in
OP^{2} are copies of OP^{1}, and thus 8-spheres.  It is
also good to work out the space of lines going through any point.  Here
we can use self-duality: since the space of all points lying on any
given line is a copy of OP^{1}, so is the space of all lines on
which a given point lies!  So the space of lines through a point is also
an 8-sphere.  Everything is very pretty.


If we give OP^{1} the nicest possible metric, its isometry group
is F_{4}: just the automorphism group of the exceptional Jordan
algebra.  However, the group of "collineations" - i.e.,
line-preserving transformations - is a form of the 78-dimensional
exceptional Lie group E_{6}.  From stuff explained last week,
the subgroup of collineations that map a point p to itself and also map
the line 1 - p to itself is isomorphic to Spin(9,1).  This gives a nice
embedding of Spin(9,1) in this form of E_{6}.  So the octonionic
projective plane is also related to 10-dimensional \emph{spacetime} 
geometry.

I hope I've got that last part right.... ultimately, this is supposed to
explain why various different theories of physics formulated in 10d
spacetime wind up being related to the exceptional Lie groups!  But I'm
afraid that so far, I'm just struggling to understand the basic geometry.

Happy New Year!



 \par\noindent\rule{\textwidth}{0.4pt}

% </A>
% </A>
% </A>
