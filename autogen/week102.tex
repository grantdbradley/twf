
% </A>
% </A>
% </A>
\week{April 21, 1997}

In "<A HREF = "week101.html">week101</A>" I claimed to have
figured out the real reason for the importance of the number 24 in
string theory.  Now I'm not so sure - some pieces of the puzzle that
I thought would fit together don't seem to be fitting.  Maybe if I
explain what I know so far, some experts will hand me some of the
missing pieces, or tell me where the ones I have go.

Most of the puzzle pieces came from a talk at a conference on homotopy
theory that I went to:

1) Ulrike Tillmann, The moduli space of Riemann surfaces - a homotopy
theory approach, talk at Northwestern University Algebraic Topology
Conference, March 27, 1997

However, some conversations with Andre Joyal during this conference
really helped turn my attention towards what might be going on here.

Let's start by recalling some stuff about homotopy groups of spheres.
There are often lots of topologically different ways of wrapping an
m-dimensional sphere around a k-dimensional sphere.  For example, if
m = k = 1, we're talking about the ways of wrapping a circle around a
circle.  These are classified by an integer called the "winding number".
We can make this concrete by thinking of the circle as the unit circle
in the complex plane.  Take your favorite integer and call it n.  Then
the function 

f(z) = z^{n}

maps the unit circle (the complex numbers with |z| = 1) to itself.  If n
is positive, this function wraps the unit circle around itself n times
in the counterclockwise direction.  If n is negative, the circle gets
wrapped around in the other direction.  If n is zero, f(z) = 1, so we
have a constant function - no "wrapping around" at all!  

It turns out that any continuous function from the circle to itself can
be continuously deformed to exactly one of these functions f(z) = z^{n}.
Homotopy theory is all about such continuous deformations.  In the
jargon of homotopy theory, we say two functions from some space to some
other space are "homotopic" if we can continuously deform the first
function to the second.  Another way of putting it is that the two
functions lie in the same "homotopy class".  Speaking of jargon, real
topologists never say "continuous function": instead, they say "map".
So, using this jargon: we know the homotopy class of a map from the
circle to itself if we know its winding number.

Now: what happens if we go to higher dimensions?  What are all the
homotopy classes of maps from the m-dimensional sphere to the
k-dimensional sphere?  Spheres are pretty simple spaces, so one might at
first guess there is some simple answer to this question for all m and
k.

Unfortunately, it's far from simple.  In fact, nobody knows the answer
for all m and k!  People \emph{do} know the answer for zillions of particular
values of m and k.  But there is no simple pattern to it: instead, there
is an incredibly complicated and beautiful weave of subtle patterns,
which we have not gotten to the bottom of... and perhaps never will.

To get a little feel for this, let's bring in some standard notation:
folks use \pi _{m}(X) to denote the set of homotopy classes of maps from an
m-dimensional sphere to the space X.  When m > 0, this set is actually a
group, called the "mth homotopy group" of X.  These groups are of major
importance in algebraic topology.

So, what we are talking about is \pi _{m}(S^{k}): the set of all homotopy
classes of ways of wrapping an m-sphere around an k-sphere.  I already 
implicitly said that 

\pi _{1}(S^{1}) = Z

where Z stands for the integers, since the winding number is an
integer.  The same thing happens if we go up a dimension:

\pi _{2}(S^{2}) = Z   

In other words: you can wrap a 2-sphere (an ordinary sphere) n times around
itself for any integer n.  How?  Well, say we use spherical coordinates
and describe a point on the sphere using its angle \phi  from the north
pole, together with the angle \theta  saying how far east it is from
Greenwich.  Then the map

f(\phi ,\theta ) = (\phi , n\theta ) 

does the job.  Any map from S^{2} to itself is homotopic to exactly
one of these.

The same basic idea works in any higher dimension, too:

\pi _{k}(S^{k}) = Z                   for any k \ge  1

In other words, there is always an integer n that plays the role of the
"winding number" of a map from the k-sphere to itself - though only
uncouth physicists call it the "winding number"; mathematicians call
it the "degree".

So far, so good.  Now, what about mapping a sphere to another sphere of
\emph{higher} dimension?  This is nice and simple:

\pi _{m}(S^{k}) = {0}                whenever m < k

The {0} there is just a standard way to write a set with only one
element, which we call "zero".  So what we mean is that there's only
\emph{one} homotopy class of ways to map a sphere to a sphere of higher
dimension.  There is always enough "room" to wiggle around one map until
it looks like another.

What about mapping a sphere to another sphere of \emph{lower} dimension?
Here is where the trouble starts! - or the fun, depending on your
attitude towards complexity.   For example, there is only one homotopy
class of maps from a 2-sphere to a circle:

\pi _{2}(S^{1}) = {0}

There is just no way a 2-sphere can get interestingly "stuck" on the
"hole" of the circle.  This may seem obvious.  But it's not 
really quite as obvious at it seems, because if we move up one dimension, 
we have:

\pi _{3}(S^{2}) = Z

This came as a big shock when Heinz Hopf first discovered it in the
1930's; before then, people had no idea how sneaky homotopy groups were!

There is a beautiful way to compute an integer called the "Hopf 
invariant"
that keeps track of the homotopy class of a map from the 3-sphere to the
2-sphere.  There are lots of nice ways to compute it, but alas, I only
have time to briefly sketch one!  Suppose that the map f: S^{3} \to S^{2} is
smooth (otherwise we can always smooth it up).  Then most points p in
S^{2} have the property that the points x in S^{3} with f(x) = p form a
"link": a bunch of knots in S^{3}.  If we take two different points in S^{2}
with this property, we get two links.  From these two links we can
compute an integer called the "linking number":
for example, we can just
draw these two links and count the times one crosses over or under the 
other (with appropriate plus or minus signs for each crossing).  This number
turns out not to depend on how we picked the two points!  Moreover, it
only depends on the homotopy class of f.  It's called the Hopf invariant
of f.

Moving up one dimension, it turns out that

\pi _{4}(S^{3}) = Z/2

Here Z/2 is the group with two elements, usually written 0 and 1,
with addition mod 2.  Why only two homotopy classes of maps from
S^{4} to S^{3}?  Well, you can compute something like the Hopf invariant
for these maps, exactly as we did before, but the thing is, links
in 4 dimensions are easy to unlink.  You can unlink something like


\begin{verbatim}

\   /
 \ /
  \
 / \
/   \
\   /
 \ /
  \
 / \
/   \

\end{verbatim}
    
and make it look like


\begin{verbatim}

|    |
|    |
|    |
|    |
|    |
|    |
|    |
|    |
|    |
|    |
|    |

\end{verbatim}
    
so the linking number in 4 dimensions is only defined mod 2.  Thus
the "Hopf invariant" is only defined mod 2.  

The exact same thing happens in higher dimensions, too, so in fact we
have:

\pi _{k+1}(S^{k}) = Z/2       for any k >= 3

This illustrates an important general fact: when the dimensions get high
enough, there's more room to wiggle things around, and as we keep
jacking up the dimension, homotopy groups simplify a bit and settle down
after a while.  This is the idea behind "stable homotopy theory".

Let's look at some more examples.  We have

\pi _{3}(S^{1}) = {0}

\pi _{4}(S^{2}) = Z/2

\pi _{5}(S^{3}) = Z/2

\pi _{6}(S^{4}) = Z/2

and so on:

\pi _{k+2}(S^{k}) = Z/2       for any k \ge  2

Sadly, I do \emph{not} understand why this is true.  How do you wrap a
4-sphere around a 2-sphere in an interesting way?  Dunno.  

(Thanks to Dan Christensen, an answer appears at the end of this post.)

Plunging on undeterred, we have:

\pi _{4}(S^{1}) = {0}

\pi _{5}(S^{2}) = Z/2

\pi _{6}(S^{3}) = Z/12

\pi _{7}(S^{4}) = Z + Z/12

\pi _{8}(S^{5}) = Z/24

\pi _{9}(S^{6}) = Z/24

and so on:

\pi _{k+3}(S^{k}) = Z/24     for any k \ge  5.

Here is where the magic number 24 comes in!  What the above says
is that if k is large enough, there are exactly 24 different homotopy
class of maps from an (k+3)-sphere to an k-sphere!  

Now I should explain what this has to do with string theory.  But first
I should say more about the homotopy groups of spheres.  There are some
simple patterns worth knowing about.  First,

\pi _{m}(S^{1}) = {0}         for any m \ge  2.

Second, there is a nice formula for when the homotopy groups settle
down as we jack up the dimension:

\pi _{k+n}(S^{k})   is independent of k as long as k \ge  n+2

The homotopy groups can stabilize sooner, as we saw for n = 2, but never
later, and often they stabilize right at k = n+2.  There is a simple
reason for this.  We saw that \pi _{k+1}(S^{k}) stabilized at k = 3 because
it's easy to unlink links in 4 or more dimensions.  Similarly,
\pi _{k+n}(S^{k}) must stabilize by the time k = n+2, because it's easy to
untie knotted n-dimensional surfaces in 2n+2 or more dimensions!

For more on stable homotopy groups of spheres, try:

2) Douglas C. Ravenel, Complex cobordism and stable homotopy groups of
spheres, Academic Press, Orlando, 1986. 
Douglas C. Ravenel, Nilpotence and periodicity in stable homotopy
theory, Princeton University Press, Princeton, 1992.  

Ravenel also spoke at this conference and is a real expert on stable
homotopy groups of spheres.  Unfortunately his talk was too high-powered
for me.  The 2nd book above is a bit more forgiving to the amateur,
but the first one has lots of nice tables of stable homotopy groups
of spheres.   

The relationship between homotopy groups of spheres and higher-
dimensional knot theory is a wonderful thing.  James Dolan and I are
learning a lot about n-categories by pondering it.  When I spoke to
him at the conference at Northwestern, it became clear that Andre
Joyal had also thought about it very deeply.  Joyal is famous for his
work relating category theory, combinatorics and topology, and his way
of thinking about the homotopy groups of spheres reflects these
interests.  He said a very fascinating thing; he said "really we
should call the sphere spectrum the 'true integers'".  I
would like to explain this... but here things get a bit technical, and
I am afraid they will get a lot more technical when I get around to
the string theory stuff.

What's the "sphere spectrum"?  Well, roughly it's just the list
of spheres S^{0}, S^{1}, S^{2}, ..., but the word "spectrum" refers to the
way all these spaces are all related, all aspects of one big thing.

Here's a nice way to think of it.  Start with the integers.  Normally
we think of these as just a set, or actually a group, since we can add
them.  But if we avoid the sin of mistaking isomorphism for equality
we can think of them as a category.  

I already began to explain this in my parable about the shepherd in
"<A HREF = "week99.html">week99</A>".  The shepherd started with the category of finite sets and
"decategorified" it to obtain the set of natural numbers, associating to
each finite set a natural number, its number of elements.  Taking
disjoint unions of sets corresponds to addition, the empty set
corresponds to zero, and so on.

Okay.  What are the \emph{integers} the decategorification of?

Well, we can imagine finite sets that have both "positive" and
"negative" elements.  The "number of elements" of such a set will be the
number of positive elements minus the number of negative elements.  This
is a bit weird if we're talking about sheep, but perhaps not so weird if
we talk about positrons and electrons, which can annihilate each other.
(In "<A HREF = "week92.html">week92</A>" I explain what I'm hinting at here: the relation between
antiparticles and adjunctions.)

Topologists prefer to speak of "positively and negatively oriented
points".  We can draw a set of positively and negatively oriented points
like this:


\begin{verbatim}

                   -      +    +     +    +      -    -

\end{verbatim}
    
We can add them by setting them side by side.  But how do the positively
and negatively oriented points cancel?  Well, remember, we're trying to
get a category!  If finite lists of positively and negatively oriented
points are our objects, what are our morphisms?  How about tangles, like this:


\begin{verbatim}

                   -      +    +     +    +      -    -
                    \    /     |     |     \    /     |
                     \  /      |     |      \  /      |
                      \/       |      \      \/      /
                               |       \            /
                               |        \          /
                               |         \        /
                               |          \      /    
                      /\       |           \    /
                     /  \      |            \  /
                    /    \     |             \/
                   /      \    |
                  +        -   +

\end{verbatim}
    
These let us cancel or create positive and negative points in pairs.
Voila!  The categorified integers!  Just as the integers form a monoid
under addition, these form a monoidal category (see "<A HREF = "week89.html">week89</A>" for these
concepts).  The monoidal structure here is disjoint union, which we can
denote with a plus sign if we like.  Similarly, we can write the empty 
set as 0.  
Above it looks like I'm drawing a 1-dimensional tangle in 2-dimensional
space.  To understand the "commutativity" of the categorified integers
we should work with 1-dimensional tangles in higher-dimensional space.
If we consider them in 3-dimensional space, we have room to switch
things around:

\begin{verbatim}

             +       +
              \     /
               \   /
                \ /
                 /
                / \
               /   \
              /     \
             +       +
\end{verbatim}
    
This gets us commutativity, as I explained in "week100".  Technically
speaking, we get a "braided" monoidal category.  However, there are two
different ways to switch things around; for example, in addition to the
above way there is 

\begin{verbatim}


             +       +
              \     /
               \   /
                \ /
                 \
                / \
               /   \
              /     \
             +       +
 
\end{verbatim}
    
To get rid of this problem (if you consider it a problem) we can work
with 1-dimensional tangles in 4-dimensional space, where we can deform
the first way of switching things to the second.  We get a "symmetric"
monoidal category.  Working in higher dimensions doesn't change
anything: things have stabilized.

If we impose the extra condition that the morphisms


\begin{verbatim}

                   /\
                  /  \
                 /    \
                /      \
               +        -

\end{verbatim}
    
and 

\begin{verbatim}

               +        -
                \      /
                 \    /
                  \  / 
                   \/
                    
\end{verbatim}
    
are inverses, as are 


\begin{verbatim}

                   /\
                  /  \
                 /    \
                /      \
               -        +

\end{verbatim}
    
and

\begin{verbatim}

               -        +
                \      /
                 \    /
                  \  /
                   \/

\end{verbatim}
    
then all morphisms become invertible, so we have not just a monoidal
category but a monoidal groupoid - a groupoid being a category with all
morphisms invertible (see "<A HREF = "week74.html">week74</A>").  In fact, not only are morphisms
invertible, so are all objects!  By this I mean that every object x
has an object -x such that x + -x and -x + x are isomorphic to 0.
For example, if x is the positively oriented point, -x is the negatively
oriented point, and vice versa.  So we have not just a monoidal groupoid
but a "groupal groupoid".  (I have adopted this charming terminology
from James Dolan.)

Very nice.  We seem to have avoided the sin of decategorification, and
are no longer treating the integers as a mere \emph{set} (or group, or
commutative group).  We are treating them as a \emph{category} (or groupal 
groupoid, or braided groupal groupoid, or symmetric groupal groupoid).

On the other hand, it's a bit odd to say that 


\begin{verbatim}

                   /\
                  /  \
                 /    \
                /      \
               +        -

\end{verbatim}
    
and

\begin{verbatim}

               +        -
                \      /
                 \    /
                  \  /
                   \/

\end{verbatim}
    
are inverses.  This amounts to saying that the morphism:


\begin{verbatim}

                   /\
                  /  \
                 /    \
                /      \
                \      /
                +\    /-
                  \  /
                   \/

\end{verbatim}
    
is equal to the identity morphism from 0 to 0, which corresponds to the
empty picture:

\begin{verbatim}










\end{verbatim}
    
Hmm.  They sure don't \emph{look} equal.  We must be doing something wrong.

What are we doing wrong?  We're committing the sin of
decategorification: treating isomorphisms as equations!  We should treat
the integers not as a mere category, but as a 2-category!  See "<A HREF = "week80.html">week80</A>"
for the precise definition of this concept; for now, it's enough to say
that a 2-category has things called 2-morphisms going between morphisms.
If we treat the integers as a 2-category, we can say there is a
2-morphism going from


\begin{verbatim}

                   /\
                  /  \
                 /    \
                /      \
                \      /
                +\    /-
                  \  /
                   \/

\end{verbatim}
    
to the identity morphism.  This 2-morphism has a nice geometrical
description in terms of a 2-dimensional surface: the surface in
3d space that's traced out as the above picture shrinks down to the 
empty picture.  It's hard to draw, but let me try:


\begin{verbatim}

                   /\
                  /  \           /\
                 /    \         /  \       /\
                /      \   =>  /    \  => /  \  => /\  =>
                \      /       \    /     \  /    +\/-
                +\    /-       +\  /-     +\/-
                  \  /           \/
                   \/


\end{verbatim}
    
Okay, say we do this: treat the integers as a 2-category.  We again are
faced with a question: do we make all the 2-morphisms invertible?  If we
do, we get a "2-groupoid", or actually a "groupal 2-groupoid".  But
again, to do so amounts to committing the sin of decategorification.  To
avoid this sin, we should tread the integers as a 3-category.  Etc, etc!

You may have noted how worlds of ever higher-dimensional topology are
automatically unfolding from our attempt to avoid the sin of
decategorification.  This is what's so neat about n-categories.  I
haven't told you how it all works, but let me summarize what's actually
happening here.  Normally we treat the integers as the free group on
one generator, or else the free commutative group on one generator.  
But groups and commutative groups are just the tip of the iceberg!
The following picture is similar to that in "<A HREF = "week74.html">week74</A>":


\begin{verbatim}

                 k-tuply groupal n-groupoids



              n = 0           n = 1             n = 2




k = 0         sets          groupoids         2-groupoids
     



k = 1        groups          groupal            groupal
                            groupoids         2-groupoids



k = 2       commutative      braided            braided
             groups          groupal            groupal
                            groupoids         2-groupoids



k = 3          `'           symmetric            weakly
                             groupal           involutory
                            groupoids            groupal 
                                               2-groupoids



k = 4          `'              `'               strongly 
                                               involutory
                                                groupal
                                              2-groupoids



k = 5          `'              `'                 `'


\end{verbatim}
    
What are all these things?  Well, an n-groupoid is an n-category with
all morphisms invertible, at least up to equivalence.  An (k+n)-groupoid
with only trivial j-morphisms for j < k can be seen as a special sort of
n-groupoid, which we call a "k-tuply groupal n-groupoid".  

Part of Joyal's point was that we should really think of the integers as
the "free k-tuply monoidal n-groupoid on one object".  Here the idea is
not to fix n and k once and for all - this would only prevent us from
understanding the subtleties that show up when we increase n and k!
Instead, we should think of them as variable, or perhaps consider the
limit as they become large.  

The other part of his point was that there's a correspondence between
n-groupoids and the information left in topological spaces when we
ignore all homotopy groups above dimension n - so-called "homotopy
n-types".  Using this correspondence, the "free k-tuply monoidal
n-groupoid on one object" corresponds to the homotopy (k+n)-type of the
k-sphere.  Moreover, if we keep jacking up k, this stabilizes when k \ge 
n+2.  Actually, as the dittos in the above chart suggest, it's a quite
general fact that the notion of k-tuply monoidal n-groupoid stabilizes
for k \ge  n+2.

Yet another point is that the pictures above explain the relation
between higher-dimensional knot theory and the homotopy groups of
spheres in a very vivid, direct way.

Okay.  What about string theory and the magic number 24?  Well, notice
that the pictures above started looking a bit like strings!  Hmm....

Here's the idea, as far as I understand it.  Presumably all but the
hardy have stopped reading this article by now, so I will pull out all
the stops.  The string worldsheet is a Riemann surface so we'll need
some stuff about Riemann surfaces from "<A HREF = "week28.html">week28</A>".  Let M(g,n) be the
moduli space of Riemann surfaces with genus g and n punctures, and let
G(g,n) be the corresponding mapping class group.  Since M(g,n) is the
quotient of Teichmueller space by G(g,n) and Teichmueller space is
contractible, we have 

M(g,n) = BG(g,n)

where "B" means "classifying space".  There's a
natural inclusion

G(g,n) \to  G(g+1,n)

defined by sewing an torus with two punctures onto your genus-g surface
with n punctures, which increases the genus by 1.  Let's define
G(\infty ,n) to be direct limit as g \to  \infty , and let

M(\infty ,n) = BG(\infty ,n).  

Now it turns out M(\infty ,1) has a kind of product on it.  
The reason is that there are products

M(g,1) x M(h,1) \to  M(g+h,1)

given sewing two surfaces together with a 3-punctured sphere.
Using this product we can define the group completion 

M(\infty ,1)+

and the result Tillmann stated which got me so excited was
that 

\pi _{3}(M(\infty ,1)+) = Z/24 + H

for some unknown group H.  Since this is really a result about the
mapping class groups of surfaces, it \emph{must} have something to do with
how conformal field theories always give projective representations of
these mapping class groups, with the "phase ambiguity" being
of the form exp(2 \pi  c i / 24), where c is the "central charge".  
No?  I just don't quite see why.  Maybe someone
will enlighten me.

Anyway, the way she proved this definitely ties right into the stuff
about stable homotopy groups of spheres.  She used explicit maps between
the third stable homotopy group of spheres

\pi _{k+3}(S^{k}) = Z/24       for k >= 5

and \pi _{3}(M(\infty ,1)+)!  And the way she got the map from the latter
to the former amounts to working with pictures I was drawing above.  
To put it more precisely, in

3) Higher-dimensional algebra and topological quantum field theory, by
John Baez and James Dolan, Jour. Math. Phys. 36 (1995), 6073-6105.

we argue that framed n-dimensional surfaces embedded in
(n+k)-dimensions should be described by the free k-tuply monoidal
n-category with duals on one object.  This should map down to the free
k-tuply groupal n-groupoid on one object, by the usual yoga of
"freeness".  Taking n = 3 and k sufficiently large, we
should obtain a homomorphism from the mapping class group of any
Riemann surface to the third stable homotopy group of spheres!
Presumably the idea is that in the limit of large genus this
homomorphism is onto!

Of course, Tillmann doesn't prove her result using the still-nascent
formalism of n-categories, but I think it will eventually be possible.
(Also, my rough argument applies to Riemann surfaces with no punctures,
while she considers those with one puncture, but various things she said
make me think this might not be such a big deal.)  The real puzzle is
this: what does

\pi _{3}(M(\infty ,n)+)

have to do with central extensions of G(g,n) for finite g?  If I could figure
this out I'd be very happy.  


\par\noindent\rule{\textwidth}{0.4pt}

\textbf{Addendum:} Dan Christensen answered a puzzle above.  Here's how
to get a nontrivial element of
\pi _{4}(S^{2}).
Take the map f: S^{3} \to  S^{2} generating 
\pi _{3}(S^{2}) 
and compose it with the map g: S^{4} \to  S^{3} generating 
\pi _{4}(S^{3}) 
(which, by the way, is obtained from f by "suspension") to obtain the
desired map from S^{4} to S^{2}.  This is an instance of a very general trick:
composing elements of homotopy groups of spheres to get new ones!





\par\noindent\rule{\textwidth}{0.4pt}
<em>
Think of one and minus one.  Together they add up to zero, nothing,
nada, niente, right?  Picture them together, then picture them
separating, peeling part.... Now you have something, you have two
somethings, where you once had nothing.</em> - John Updike, Roger's Version

\begin{verbatim}
                   /\
                  /  \
                 /    \
                /      \
               +        -
\end{verbatim}
    

\par\noindent\rule{\textwidth}{0.4pt}

% </A>
% </A>
% </A>
