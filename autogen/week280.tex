
% </A>
% </A>
% </A>
\week{September 27, 2009 }


I have a lot to talk about, since I just got back from this quantum
gravity summer school in Corfu:

1) 2nd School and workshop on quantum gravity and quantum geometry,
September 13-20, 2009, organized by John Barrett, Harald Grosse,
Larisa Jonke and George Zoupanos.  Information at
<a href = "http://www.physics.ntua.gr/corfu2009/qg.html">http://www.physics.ntua.gr/corfu2009/qg.html</a>

I felt a bit like Rip Van Winkle, the character who fell asleep for 20
years and woke to find everything changed.  I gave up working on
quantum gravity about 4 years ago because so many problems seemed
intractable.  Now a lot of these have been solved, or at least seen
some progress.  It was great!

The school featured courses by Ashtekar, Barrett, Rivasseau, Rovelli
and myself - we each gave 5 hours of lectures.  All these guys were
friends whom I was very happy to see again - except for Rivasseau, whom
I'd never met.  But it was great to meet him, since he works on
mathematically rigorous quantum field theory, the topic I tried to do
my PhD on.  He had some amazing things to say about the combinatorics
of trees and the problem of summing divergent series.  Sadly, right
now I only have time to summarize the courses by Ashtekar and Rovelli.

Abhay Ashtekar gave a review of recent work on loop quantum cosmology.
Starting with the work of Martin Bojowald around 2001, there's been a
lot of interest in the possibility that loop quantum gravity could
eliminate the singularity at the Big Bang.  The big problem is that
dynamics in loop quantum gravity is not understood.  There are lots of
choices for how it might work, and nobody knows which is right, or if
there even <i>is</i> a right one.  Luckily, by imposing the symmetry
conditions appropriate to a homogeneous isotropic cosmology one can
narrow down the problem of seeking a reasonable dynamics to something
much more manageable.  Instead of infinitely many degrees of freedom,
there are just a few.

Unfortunately, there are still lots of choices involved in guessing a
reasonable theory of quantum cosmology inspired by loop quantum
gravity.  Checking their implications involves both computer
calculations and conceptual head-scratching.  So, it took about 7
years to find a satisfactory candidate.  It might not be right, but at
least it's self-consistent and elegant.  For a nice survey, see this:

2) Abhay Ashtekar, Loop quantum cosmology: an overview, 
Gen. Rel. Grav. 41 (2009), 707-741.  Also available as 
<a href = "http://arxiv.org/abs/0812.0177">arXiv:0812.0177</a>.

Section IVa sketches the history of the subject, but it's best to
read the previous stuff first.

Anyway, what are the results?  What does the currently popular theory
of loop quantum cosmology imply?

In a nutshell: if you follow the history of the Universe back in time,
it looks almost exactly like what ordinary cosmology predicts until
the density reaches about 1/100 of the Planck density.

What's the Planck density, you ask?

Well, you can cook up units of mass, length and time by saying that
Planck's constant and the speed of light and Newton's gravitational
constant are all 1 in these units.  Unsurprisingly, these units are
called:

<UL>
<LI>
the Planck mass: 2.2\times 10^{-8} kilograms
<LI>
the Planck length: 1.6\times 10^{-35} meters
<LI>
the Planck time: 5.4\times 10^{-44} seconds
</UL>

The "Planck density" is one Planck mass per cubic Planck
length - or in ordinary units, about 5\times 10^{96} kilograms
per cubic meter.  That's incredibly dense!  It's what you'd get by
compressing 10^{23} solar masses into the volume of an atomic
nucleus.


 According to loop quantum cosmology, at around 1/100 the Planck
density quantum gravity effects come in, creating a force that
prevents the universe from shrinking further as we march backwards in
time.  And at about .4 times the Planck density, there's a
"bounce".  Going further back in time, we see the Universe
expand again!  Indeed, the Universe is symmetrical in time around the
moment of maximum compression.

So, the Big Bang is replaced by a Big Bounce.  

<div align = "center">
<a href = "http://gravity.psu.edu/outreach/articles/bigbounce.pdf">
<img border = "none" src = "http://math.ucr.edu/home/baez/big_bounce_new_scientist.jpg" alt = ""/>
</div>
% </a>

The above picture comes from this popular account:

3) Anil Ananthaswamy, From Big Bang to Big Bounce, New Scientist,
December 13, 2008, pp. 32-35.  Also available at
<a href = "http://gravity.psu.edu/outreach/articles/bigbounce.pdf">http://gravity.psu.edu/outreach/articles/bigbounce.pdf</a>

Interestingly, in this model quantum effects don't create much
dispersion as the Universe passes through a big bounce.  In other
words: if the Universe's wavefunction is sharply peaked around a
certain classical geometry, it remains so through the big bounce.  It
doesn't "smear out" too much.

By the way, I would be very happy if anyone working on loop quantum
cosmology could give me a intuitive physical explanation for the force
that prevents the singularity.  Mathematically I know that it arises
from a kind of discretization.  Instead of talking about the curvature
of spacetime at infinitesimally small scales, we can only measure
curvature by carrying a particle around a finite-sized loop.  This has
little effect when spacetime is only slightly curved, but when the
curvature is big it makes a big difference.  This causes an effect
very much like a force that prevents the Universe from crushing down
to nothing as we follow it backwards in time.

So far, so good.  But a good physical intuition would explain the
<i>sign</i> of this effective force.  Why does it prevent the
singularity instead of, say, making it worse?

For some more details, try this treatment which resembles the course
Ashtekar taught:

4) Abhay Ashtekar, An introduction to loop quantum gravity through
cosmology, Nuovo Cimento 122B (2007), 135-155.  Also available as
<a href = "http://arXiv.org/abs/gr-qc/0702030">arXiv:gr-qc/0702030</A>.

Anyway, the real fun will start when people systematically compute the
behavior of inhomogeneous perturbations in loop quantum cosmology
model.  After all, the little ripples in the microwave background
radiation are the first interesting thing we see in the Universe.

A lot of work on cosmology studies these inhomogeneities by
calculating backwards to a hypothesized "inflationary epoch"
about 10^{-35} seconds after the Big Bang - or Big Bounce, if
that's your theory.  Quantum gravity effects are likely to become
important only at much earlier times, since the Planck time is about
10^{-43} seconds.  Here I'm using "much earlier" in
a funny logarithmic sense.  But that's actually appropriate here.  The
inflationary epoch comes about 100 million Planck times after the Big
Bounce.  According to Ashtekar's calculations, by then quantum gravity
corrections only affect the rate of expansion of the universe by about
one part in 100 thousand.  So, it's not clear that loop quantum
gravity calculations are going to have anything interesting to say
about anything we can see today.  But who knows?

Next, Carlo Rovelli.  His class was an introduction to spin foam
models, which are an attempt to pin down a specific dynamics for loop
quantum gravity.  Here I'm going to get more technical, because this
material is closer to my heart.  If you need a warmup, try
"<a href = "week109.html">week109</A>"-"<a href = "week113.html">week113</A>" for the basics.

I worked on spin foams for about 5 years.  I love them because they
offer the hope of building spacetime from abstract algebra - higher
category theory, in fact.  But I gave up because a lot of puzzle
pieces just didn't seem to be fitting together.  Back then, the best
candidate for a spin foam model of gravity was the Barrett-Crane
model.  But there were three big problems:

\begin{quote}
A) The Barrett-Crane model used spin networks of a different kind from
the usual ones in loop quantum gravity.  Instead of spin networks with
edges labelled by unitary representations of SU(2) (the double cover
of the rotation group), it used unitary representations of SL(2,C)
(the double cover of the Lorentz group).  This is because it's all
about <i>spacetime</i>, while loop quantum gravity focuses on
<i>space</i>.  And instead of using spin networks with vertices
labelled by arbitrary intertwiners, it only used a special intertwiner
called the "Barrett-Crane intertwiner".

B) While loop quantum gravity in its modern formulation includes the
Immirzi parameter - a dimensionless constant that sets the scale of
area quantization - the Barrett-Crane model did not.  If the currently
accepted calculations are right, we need to choose a special and
rather peculiar value of the Immirzi parameter if we want loop quantum
gravity to get the right answer for the entropy of black holes.  So,
along with problem A), this makes it even harder to connect the
Barrett-Crane model to loop quantum gravity.

C) At first people hoped for various clues linking the Barrett-Crane
model to general relativity.  For example, we hoped that the
asymptotic value of the amplitude for a large 4-simplex in the
Barrett-Crane model was nicely related to the action for general
relativity.  But this turned out to be false: in the Barrett-Crane
model, the amplitude for a large 4-simplex is dominated by certain
degenerate geometries where the 4-simplex is squashed down to 3
dimensions.  See "<a href = "week128.html">week128</A>",
"<a href = "week168.html">week168</A>", "<a href =
"week170.html">week170</A>" and "<a href =
"week198.html">week198</A>" for the story here.  Carlo Rovelli
raised our hopes again in a more sophisticated way: he tried to
compute the propagator for gravitons starting from the Barrett-Crane
model.  For a beautiful and physically very sensible reason, the
degenerate geometries don't dominate this calculation.  Rovelli got
some promising results for certain components of the graviton
propagator, and left a student to work out the rest of the
components... but it didn't work!
\end{quote}

It seems all these problems have been solved now.  There's a new model
sometimes called the EPRL model, after Engle, Pereira, Rovelli, and
Livine, although other people were involved as well - I'll list some
papers later.

The basic idea of the EPRL model is to start with the Holst 
Lagrangian for general relativity.  In 1995, Soren Holst came
up with a nice Lagrangian for gravity:

5) Soren Holst, Barbero's Hamiltonian derived from a generalized
Hilbert-Palatini action, Phys. Rev. D53 (1996), 5966-5969.  Also
available as <a href =
"http://arXiv.org/abs/arXiv:gr-qc/9511026">arXiv:gr-qc/9511026</A>.

It looks like this:

tr(e ^ e ^ *F) + (1/\gamma ) tr(e ^ e ^ F)

I'll explain this in detail later, because there was a student who
twice asked about the math behind this Lagrangian, and Rovelli and I
brushed the question off by saying "it's just like Palatini
Lagrangian".  I feel guilty, so someone find that student and
tell him to read my explanation below.

But that gets a bit technical, so for now let me say: "it's just
like the Palatini Lagrangian".  Namely, the first term is the
usual Palatini Lagrangian for gravity.  The second term involves the
Immirzi parameter, \gamma .  The second term doesn't affect the
classical equations of motion, because its variation is a total
derivative.  But it does affect the quantum theory!

If we triangulate spacetime and carry out a spin foam quantization of
this theory - which is a bit like doing lattice gauge theory - we can
show (in a rough-and-ready physicist's way) that the partition
function of the quantum theory is computed as a sum over spin foams
where the spin foams are labelled by certain special representations
of SL(2,C).

Physicists don't learn the unitary representations of the Lorentz
group in school the way they do for the Poincare group.  But the
unitary representations of the Lorentz group - or its double cover
SL(2,C) - are very nice.  Except for the trivial representation
they're all infinite-dimensional, which is a bit scary at first...
but there's a bunch called the "principal series" indexed by
a spin j = 0,1/2,1,3/2,... and a nonnegative real number I'll call k.
Very roughly speaking the spin j has to do with rotations, while k is
an analogous quantity related to boosts.  If you want more details,
the only online explanation I can find is this:

6) Wikipedia, Representation theory of the Lorentz group,
<a href = "http://en.wikipedia.org/wiki/Representation_theory_of_the_Lorentz_group">http://en.wikipedia.org/wiki/Representation_theory_of_the_Lorentz_group</a>

It may be better to read some of the many books cited there.

Anyway, the special representations of SL(2,C) that show up in the
EPRL model are those with 

k = \gamma  j

This is beautiful because there's one for each spin.  So, the category
of these representations and their direct sums is equivalent to the
category of finite-dimensional unitary representations of SU(2)!

This is how the EPRL model gets around problem A) listed above.  Spin
networks in this new model are nicely compatible with spin networks in
loop quantum gravity, because you can think of their edges
<i>either</i> as labelled by special representations of SL(2,C),
<i>or</i> as labelled by arbitrary representations of SU(2).  The
first is the "spacetime" or Lagrangian viewpoint, the second
is the "space" or Hamiltonian viewpoint.

This is also the key to how the EPRL model gets around problem B).  The
Immirzi parameter is built into the model in a very natural way.  As a
result, the quantization of area and volume in this model is
compatible with that in loop quantum gravity.

I don't think I'll describe the rest of the model, which consists of a
formula for computing the amplitude for a 4-simplex with edges
labelled by spins.  But it's this formula that solves problem C).  The
EPRL model gets the graviton propagator right!

Of course there are even bigger tests still ahead for this spin foam
model.  We need to see if it reduces to general relativity in the
classical limit.  In other words, we need to get Einstein's equations
out of it.  And we need to see if it reduces to the usual perturbative
theory of quantum gravity in some other limit.  In other words, we
need to compute, not just graviton propagators (which describe the
probability of a lone graviton zipping from here to there on the
background of Minkowski spacetime), but graviton scattering amplitudes
(which describe the probability of various outcomes when two or more
gravitons collide).  

Both these tasks are both computationally and <i>conceptually</i>
difficult.  In other words, it's not just hard to do the calculations:
it's hard to know what calculations to do!  When I said "in some
limit" and "in some other limit", I know what limits
these are in a physical sense, but not how to describe them using spin
foams.  Actually we seem closer to understanding graviton scattering
amplitudes, thanks to the work of Rovelli.  But it seems miraculous
and strange that we can compute graviton propagators (much less
scattering amplitudes) using very simple spin foams, as Rovelli and
his collaborators have done.  Every time I meet him, I ask Rovelli
what's going on here: how we can describe the behavior of a graviton
in terms of just a few 4-simplices of spacetime.

So, the road is still long, steep, and fraught with danger.  But 
three problems that had everyone completely stumped have now been
solved in one elegant blow.

Though there's much more to say, it's dinnertime now.  So, let me list
some references and then explain the differential geometry behind the
Holst Lagrangian, just to make up for not explaining it to that
student.

The original "EPR model" was introduced here - but this
treated Riemannian rather than Lorentzian metrics, and only in the
special case where the second term in the Holst action was left out -
so, no Immirzi parameter:

7) Jonathan Engle, Roberto Pereira and Carlo Rovelli, Flipped spinfoam
vertex and loop gravity, Nucl. Phys. B798 (2008), 251-290.  Also
available as <a href = "http://arxiv.org/abs/0708.1236">arXiv:0708.1236</a>.

This is a very nice paper which describes a lot of geometry that
I haven't had time to mention.  However, the full-fledged model
appeared later:

8) Jonathan Engle, Etera Livine, Roberto Pereira, and Carlo Rovelli,
LQG vertex with finite Immirzi parameter, Nucl. Phys.  B799 (2008),
136-149.  Also available as <a href =
"http://arxiv.org/abs/0711.0146">arXiv:0711.0146</a>.

But there are also other people whose work deserves credit!  For
example, my friends Laurent Freidel and Kirill Krasnov:

9) Laurent Freidel and Kirill Krasnov, A new spin foam model for 4d
gravity, Class. Quant. Grav. 25 (2008), 125018.  Also available as
<a href = "http://arxiv.org/abs/0708.1595">arXiv:0708.1595</a>.

This paper gives a bit of the history, which I don't know very well,
since I wasn't paying attention.  Kirill visited me once and tried to
get me interested in his new spin foam model, but I wasn't in the
mood.  Now that everything is nicely polished, of course I like it
more.

There are probably lots of other important papers that I'm leaving
out, but let me turn to a few papers that discuss graviton
propagators.

Here's the paper where Rovelli's student found problems with
getting the graviton propagator from the Barrett-Crane model:

10) Emanuele Alesci and Carlo Rovelli, The complete LQG propagator:
I. Difficulties with the Barrett-Crane vertex.  Phys. Rev. D76 (2007),
104012.  Also available as <a href = "http://arxiv.org/abs/0708.0883">arXiv:0708.0883</a>.

Then Rovelli and collaborators found numerical evidence that the EPR
model seemed to be working better:

11) Elena Magliaro, Claudio Perini and Carlo Rovelli, Numerical
indications on the semiclassical limit of the flipped vertex,
Class. Quant. Grav. 25 (2008), 095009.  Also available as
<a href = "http://arxiv.org/abs/0710.5034">arXiv:0710.5034</a>.

Then Alesci and Rovelli wrote a paper using the new model:

12) Emanuele Alesci and Carlo Rovelli, The complete LQG propagator:
II. Asymptotic behavior of the vertex, Phys. Rev. D77 (2008), 044024.
Also available as <a href = "http://arxiv.org/abs/0711.1284">arXiv:0711.1284</a>.

and then Alesci and Rovelli wrote another paper in their
series, with Eugenio Bianchi:

13) Emanuele Alesci, Eugenio Bianchi, Carlo Rovelli, LQG propagator:
III. The new vertex, available as <a href =
"http://arxiv.org/abs/0812.5018">arXiv:0812.5018</a>.

This paper used the work of John Barrett and collaborators, who
analyzed the asymptotics of the amplitude for a 4-simplex in the new
model:

14) John W. Barrett, R. J. Dowdall, Winston J. Fairbairn, Henrique
Gomes and Frank Hellmann, Asymptotic analysis of the EPRL four-simplex
amplitude, available as <a href = "http://arxiv.org/abs/0902.1170">arXiv:0902.1170</a>.

For a nice treatment of spin foams that generalizes the new model 
to spin foams that don't come from triangulations of spacetime, try:

15) Wojciech Kaminski, Marcin Kisielowski, Jerzy Lewandowski,
Spin-foams for all loop quantum gravity, available as 
<a href = "http://arxiv.org/abs/0909.0939">arXiv:0909.0939</a>.

Everything Lewandowski does is very precise, so if you're a
mathematician you might actually want to start here.
     
There are also lots of other papers in this general line of work.  I
apologize to everyone whose work I didn't cite - like Dan
Christensen and Igor Khavkine, for example!

Anyway, I'm excited about this new work and I hope to write another
paper or two about spin foam models.  I came up with two fun ideas
during the Corfu summer school, which I'd like to work on.

But let me conclude by explaining the Holst Lagrangian for gravity.
I explained the Palatini Lagrangian and a whole bunch of other
Lagrangians for gravity back in "<a href =
"week176.html">week176</A>".  But maybe you were just a kid back
then... or maybe you weren't paying adequate attention!  So, let me
repeat my explanation in slightly different words.

Assume spacetime is an orientable smooth 4-manifold M.  Pick a vector
bundle T that's isomorphic to the tangent bundle TM.  Physicists don't
have a name for this bundle, but they call any of its fibers the
"internal space".  I call it the "fake tangent bundle".

We then equip T with a Lorentzian metric and orientation.  This lets
us describe a Lorentzian metric on M using a vector bundle map

e: TM \to  T  

This map has lots of names: the "cotetrad", the
"soldering form", or the "coframe field".
Whatever we call it, we can use it to pull the metric on T back to the
tangent bundle. If e is an isomorphism, this gives a Lorentzian metric
on M.  If it's not, we get something like a metric, but with
degenerate directions.  For now let's only consider the case where e
is an isomorphism.

The cotetrad is one of the two basic fields used to define the Holst
action.  The other is a metric-compatible connection on T.  This
connection is usually denoted A and called a "Lorentz
connection".  Its curvature is denoted F.

Now, what does the Holst Lagrangian

tr(e ^ e ^ *F) + (1/\gamma ) tr(e ^ e ^ F)

actually mean?  

First of all, the curvature F is an End(T)-valued 2-form, but using
the metric on T we get an isomorphism between T and its dual, so we
can also think of the curvature as a 2-form taking values in T tensor
T. However, if we do this, the fact that A is metric-compatible means
that F is skew-symmetric: it takes values in the second exterior power
of T, \Lambda ^{2}(T).

Since T has a metric and orientation, we can define a Hodge star
operator on the exterior algebra \Lambda (T) just as we normally do for
differential forms on a manifold with metric and orientation.  We call
this the "internal" Hodge star operator.  Using this we can define *F,
which is again a 2-form taking values in \Lambda ^{2}(T).

Next, note that the cotetrad e can be thought of as a T-valued 1-form.
This allows us to define the wedge product e ^ e as a
\Lambda ^{2}(T)-valued 2-form.  This is the same sort of
gadget as the curvature F and its internal Hodge dual *F!  So, we can
take the wedge product of the differential form parts of e ^ e and *F
while using the metric on T to pair together their
\Lambda ^{2}(T) parts to get a number.  The overall result is
a plain old 4-form, which we call tr(e ^ e ^ *F).  This is the
Palatini Lagrangian!

If you work out the equations of motion coming from this Lagrangian,
they say A that pulls back via e to a <i>torsion-free</i>
metric-compatible connection on the tangent bundle.  This is just the
Levi-Civita connection!  It follows that F pulls back to the curvature
of the Levi-Civita connection.  This is just the Riemann tensor!
Finally, it turns out that tr(e ^ e ^ *F) is just the Ricci scalar
curvature times the volume form on M, so we were doing general
relativity all along!

We define tr(e ^ e ^ F) the same sort of way, and throwing this term
into the action doesn't affect the classical equations of motion.
It's very much like Yang-Mills theory, where you can take the usual
action

tr(F ^ *F)

and throw in a "theta term", proportional to
"second Chern class"

tr(F ^ F)

without changing the classical equations of motion.  But it does
affect the quantum theory!  

For a more detailed treatment of the Holst action including 
a cosmological constant term proportional to

tr(e ^ e ^ e ^ e)

and three topological terms corresponding to the Pontryagin 
class, the Euler class and the Nieh-Yan class, see:

16) Danilo Jimenez Rezende and Alejandro Perez, 
4d Lorentzian Holst action with topological terms, 
available as <a href = "http://arxiv.org/abs/0902.3416">arXiv:0902.3416</a>.

\par\noindent\rule{\textwidth}{0.4pt}

\textbf{Addenda:} I thank Dan Christensen for catching an error
in my quick history of the graviton propagator calculations, and
Derek Wise for catching some other errors.  
Urs Schreiber asked a question about how the singularity gets
avoided in loop quantum cosmology:

\begin{quote}

My impression is the following, I'd be grateful to hear what is wrong
about it.

What happens is a special case of this: start with a differential
equation whose solutions diverge at the origin. Replace it with a
difference equation on a subset of points that does not include the
origin. This will have solutions not showing the singularity.
\end{quote}

I replied:

\begin{quote}

This may have been true for some early models - there have been a lot
of models between 2001 and now.  It's definitely not true for the
currently popular models that Ashtekar was explaining.

It's true that a differential equation is getting replaced by a
difference equation.  But the singularity is not being avoided simply
by "stepping over it".  That would be a cheap trick.  In the models
under study now, there really is an effective "force" that prevents
the singularity.  And it arises from the mechanism I sketched:

\begin{quote}
Instead of talking about the curvature of spacetime at infinitesimally
small scales, we can only measure curvature by carrying a particle
around a finite-sized loop. This has little effect when spacetime is
only slightly curved, but when the curvature is big it makes a big
difference. This causes an effect very much like a force that prevents
the Universe from crushing down to nothing as we follow it backwards
in time.
\end{quote}

A bit more precisely - but still quite roughly: if the curvature is c,
in loop quantum gravity we don't have a quantum operator corresponding
to c.  Instead we have something more like exp(ic).  So we use
something like sin(c)=(exp(ic)-exp(-ic))/2i as a substitute for c in
certain equations.  When the curvature is small these are almost the
same.  When it gets big, they're different.  This gives an effective
force that prevents the singularity.  This force starts getting big
when the density is about 1/100 of the Planck density.  When the
density hits .41 times the Planck density, it's enough to create a
"bounce".

(The math is actually a bit more complicated than what I said, but
it's similar in spirit.)

In fact, a bunch of calculations have shown that the quantum dynamics
are quite nicely approximated by an "effective Friedmann equation" - a
differential equation like the usual classical one that describes the
Big Bang, but with an extra term.

And, the solutions of the difference equation are very close to the
solutions of this differential equation.

\end{quote}

Then Urs asked for more detail, and I admitted that the references
given above might not satisfy him:

\begin{quote}
... you might prefer <a href =
"http://arxiv.org/abs/0706.1057">Effective equations for
isotropic quantum cosmology including matter</a> by Bojowald,
Hernandez and Skirzewski.

In Section 2, they start with the Friedmann equation for a homogeneous
isotropic cosmology with k=0 and a massless scalar field \phi .  In
Section 2.1 they "deparametrize" this equation, eliminating the time
variable by using \phi  as a clock field, or "internal time" - a standard
way of dealing with the problem of time in quantum cosmology.  In
Section 2.2 they review the standard Wheeler-DeWitt quantization of
the resulting theory.

Then, in Section 2.3, they switch to loop quantization.  They don't
give a vast amount of detail on why one should do this, but they do
say what is being done.  Briefly, instead of the variable c = da/d\tau 
(the rate of expansion of the universe, closely related to spacetime
curvature in this model) we switch to using exp(ic) and exp(-ic).  In
a deeper treatment this comes from using holonomies instead of the
curvature at a point.  This is what produces the effective force that
prevents the singularity.

They don't derive difference equations in this paper; instead, they
just derive effective corrections to the usual Wheeler-DeWitt
equation.  Calculations have shown that solutions of the resulting
differential equations closely match those of the difference equations
that a more thorough approach yields.  So, as far as physical
intuition goes, perhaps the most important thing is to see where these
effective corrections to the Wheeler-DeWitt equation come from, and
why they prevent the singularity.
\end{quote}

For more discussion visit the <a href =
"http://golem.ph.utexas.edu/category/2009/09/this_weeks_finds_in_mathematic_41.html">\emph{n}-Category
Caf&eacute;</a>.

\par\noindent\rule{\textwidth}{0.4pt}

<i>Many fine physicists have burned away their lives grappling with
the problem of quantum gravity.</i> - <a href =
"http://arxiv.org/abs/0907.4238">R. P. Woodard</a>

\par\noindent\rule{\textwidth}{0.4pt}

% </A>
% </A>
% </A>
