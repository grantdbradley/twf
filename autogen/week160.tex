
% </A>
% </A>
% </A>
\week{November 20, 2000 }



Anyone who grew up on science fiction in the 1960s probably read a 
bunch about adventures on strange planets, and dreamt of our future 
in space.  At least I did.  Asimov, Clarke, Heinlein... they helped
get me interested in science, but they also painted a romantic vision
of human destiny.  Only later did it become clear that \emph{for now}, the
real adventures will come from the microscopic realm: from applications
of integrated circuits, biotechnology, nanotechnology, and the like.  
When you're trying to have lots of fun in a hurry, the speed limit is 
the speed of light - and this makes interstellar travel a drag.

Nonetheless, when you imbibe a romantic dream in childhood, it can
be hard to shake it it as an adult.  So I still like to read about
strange planets, even if know rationally that I can have more fun at
home.

So - let me start by talking about a world where it might rain methane!

1) Ralph D. Lorenz, The weather on Titan, Science 290 (October 20, 
2000), 467-468.

Caitlin A. Griffith, Joseph L. Hall and Thomas R. Geballe, Detection
of daily clouds on Titan, Science 290 (October 20, 2000), 509-513.

Titan is the largest moon of Saturn, and it's the only moon in our
solar system with a significant atmosphere.  Its atmosphere is mostly
nitrogen, with a surface pressure 1.5 times that of the air pressure 
here on Earths' surface.  However, there is also a fair amount of 
methane, and even some ethane.  At the surface of Titan, it's cold 
enough 
for these compounds to liquefy.  People have even seen what look like
pitch-black oceans of hydrocarbon compounds, hundreds of kilometers in 
size!  

However, 14 kilometers or more from the surface, it gets cold enough
for methane to freeze.  And the news is that recently Caitlin Griffith 
et al have spotted things that look like methane clouds.  Compared to
Earth, which is usually 30 percent covered with clouds, the cloud cover 
on Titan seems spotty.  There's not really enough methane for lots of
clouds.  But there may be rain!   The drops would be larger than terrestrial
raindrops, and fall slowly in the gravity of Titan, which is like that
of our moon.    Since the near-surface atmosphere usually has a relative 
humidity of at most 60%, the drops would tend to evaporate before hitting
the ground.  (I've seen a similar thing in New Mexico.)  However, in
a big rainstorm the evaporation of the first drops might elevate the
humidity to the point where later drops could reach the surface.  So
there might even be erosion on the surface of Titan.  With any luck,
the Cassini spacecraft will arrive at Saturn in 2004 and make about 40
flybys of Titan in the following 4 years, getting a good look at this stuff.

Now for a crazy speculation of my own.  Once upon a time James Lovelock
argued that you could tell there was life on earth simply by noting that 
the atmosphere contains lots of oxygen, despite the fact that oxygen is
highly reactive.  This means the atmosphere is far from equilibrium.  
Yet the percentage of oxygen in the atmosphere has remained fairly constant
for long periods of time!  So presumably there must be some homeostatic 
mechanism at work to keep it constant.  Only life - he argued - could 
be responsible!  Conversely, Lovelock guessed there is not life on Mars, 
because its atmosphere \emph{is} in equilibrium.   

Now, the methane in Titan's atmosphere is dissociated by sunlight, and
this process is irreversible, since the resulting hydrogen flies off
into space.  At the rate this happens, the entire methane content of the
atmosphere would be destroyed in only 10 million years if it were not
renewed somehow.  In the first article cited above, the author writes:
"For the methane we see today not to be a bizarre fluke, it must be
continuously resupplied from a surface reservoir or by cryovolcanism
(that is, volcanism where the molten `rock' is just ice)." And this
made me wonder: where is Lovelock when we need him?  Maybe \emph{life} is
responsible for this out-of-equilibrium condition.

Or maybe not.  After all, it really could be something else.

Next: a world where it might rain diamonds!

2) Richard A. Kerr, Neptune may crush methane into diamonds, Science
286 (October 1, 1999), 25. 

Laura Robin Benedetti, Jeffrey H. Nguyen, Wendell A. Caldwell, Hongjian 
Liu,  Michael Kruger, and Raymond Jeanloz,  Dissociation of CH_{4} 
at high 
pressures and temperatures: diamond formation in giant planet interiors?,
Science 286 (October 1, 1999), 100-102. 

The atmosphere of Neptune is believed to contain lots of methane when you 
go 4000 kilometers or more beneath the cloud tops.  And Neptune ain't
no measly moon: it's a gas giant, so the atmospheric pressure becomes 
enormous as you go further in.  Recently, people have been compressing
methane under ridiculously high pressures, using techniques too fiendish
to describe here.  At sufficiently high pressures, it releases hydrogen 
and turns into diamond crystals! - together with lots of other crud,
like ethane and acetylene.  This could happen in Neptune at a depth of 
about 7000 kilometers below the cloud tops, where the pressure reaches 
500,000 times that of the Earth's atmosphere.   So in fact, there could 
be a steady rain of diamond crystals on Neptune!

By the way, all these Science articles are available for free online
here:

3) Science Magazine, <a href = "http://www.sciencemag.org/search.dtl">
http://www.sciencemag.org/search.dtl</a>

I also want to say a bit about spin foams.  Papers continue to come out
on this subject:

4) Alejandro Perez and Carlo Rovelli, A spin foam model without bubble 
divergences, available as <A HREF = "http://xxx.lanl.gov/abs/gr-qc/0006107">gr-qc/0006107</A>.


A while ago, De Pietri, Freidel, Krasnov and Rovelli showed how to get
the Barrett-Crane model for Riemannian quantum gravity from a quantum
field theory on a product of 4 copies of SO(4) - see "<A HREF =
"week140.html">week140</A>".  This was based on earlier work by
Boulatov and Ooguri, who did a similar thing for BF theory.  The basic
idea is to cook up a quantum field theory on a product of copies of Lie
group, with a nice Lagrangian that encodes how simplices can stick
together to form a spacetime.  If you do a Feynman diagram expansion of
this quantum field theory, the Feynman diagrams can be identified with
spin foams, and the sum over Feynman diagrams becomes a sum over spin
foams.

The sum over spin foams may diverge; this paper attempts to control 
those divergences.  It makes some precise mathematical conjectures 
about the convergence of certain sums - mathematicians who like
analysis and representation theory should get to work on these!

5) Alejandro Perez and Carlo Rovelli, Spin foam model for Lorentzian
general relativity, available as <A HREF =
"http://xxx.lanl.gov/abs/gr-qc/0009021">gr-qc/0009021</A>.

This paper modifies the De Pietri-Freidel-Krasnov-Rovelli 
construction to get the \emph{Lorentzian} Barrett-Crane model from 
quantum field theory on a product of 4 copies of SO(3,1).

6) Alejandro Perez and Carlo Rovelli, 3+1 spinfoam model of quantum
gravity with spacelike and timelike components, available as <A HREF =
"http://xxx.lanl.gov/abs/gr-qc/0011037">gr-qc/0011037</A>.

In the original Lorentzian Barrett-Crane model, spacetime is made
of 4-simplices whose triangular faces are space/timelike - in
other words, like little bits of the xt plane in Minkowski spacetime.
This model also allows 4-simplices whose triangular faces are 
space/spacelike - in other words, like little bits of the xy plane.
This amounts to using a different class of irreducible unitary 
representations of the Lorentz group to label the triangles. 

7) Daniele Oriti and Ruth M. Williams, Gluing 4-simplices: a derivation
of the Barrett-Crane spin foam model for Euclidean quantum gravity,
available as <A HREF =
"http://xxx.lanl.gov/abs/gr-qc/0010031">gr-qc/0010031</A>.

This gives an alternate derivation of the Riemannian Barrett-Crane spin 
foam model starting from the Lagrangian for Riemannian general relativity.
This is good because it gives some more intuition for the relation between
classical general relativity and the spin foam approach to quantum gravity.

Finally, if you're hopelessly confused about spin foams and other
approaches to quantum gravity, you might enjoy the following little
history of quantum gravity.  It explains how many different approaches
were tried, leading up to the research directions that people pursue
now:

8) Carlo Rovelli, Notes for a brief history of quantum gravity,
presented at the 9th Marcel Grossmann Meeting in Rome, July 2000.
Available as <A HREF =
"http://xxx.lanl.gov/abs/gr-qc/0006061">gr-qc/0006061</A>.










<p> <hr>

% </A>
% </A>
% </A>


% parser failed at source line 257
