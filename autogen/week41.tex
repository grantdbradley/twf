
% </A>
% </A>
% </A>
\week{ctober 17, 1994}

In the beginning of September I went to a conference at the Center for
Gravitational Physics and Geometry at Penn State.  This is the center
run by Abhay Ashtekar, and it has Jorge Pullin and Lee Smolin as
faculty, and Roger Penrose as a part-time visitor --- so it's a
great place to visit if you're interested in quantum gravity.  There
are a lot of good postdocs and such there, too.  I've been too busy
to say much so far about what happened at this conference, but I'd like
to now.  

One talk I enjoyed a lot was Steve Carlip's, on the entropy of black
holes.  This has subsequently come out as a preprint, available
electronically:

1)  The Statistical Mechanics of the (2+1)-Dimensional Black Hole, by
Steve Carlip, 12 pages available as <A HREF = "http://xxx.lanl.gov/abs/gr-qc/9409052">gr-qc/9409052</A>.  

It's well-known by know that in certain situations it makes sense to
speak of the "entropy" of a black hole, but the real meaning of this
entropy is still mysterious.  In particular, since the entropy of a
black hole is (often, but not always) proportional to the area of its
event horizon, it would be very satisfying if the entropy corresponded
somehow to degrees of freedom that "lived at the event horizon".  Steve
Carlip has done a pretty credible calculation along these lines (though
not without various subtle difficulties) in the case of a black hole in
3-dimensional spacetime.

I should say a little bit about gravity in 3 dimensions and why people
are interested in it.  3-dimensional gravity is drastically simpler
than 4-dimensional gravity, since in 3 dimensions the vacuum
Einstein's equations say the spacetime metric is \emph{flat}, at least if the
cosmological constant vanishes.  Thus there can be no gravitational radiation
(and in quantum theory no "gravitons"), and the metric produced by a
static point mass is not like the Schwarschild metric, instead, on 
space it is just like that of a cone.  Things are a bit different if the
cosmological constant is nonzero; in particular, there are black-hole
type solutions.  But there is still no gravitational radiation.

Basically, people are interested in 3-dimensional quantum gravity
because it's simple enough that one can compute something and hope it
sheds some light on the 4-dimensional world we live in.  For some issues
this appears to be the case: primarily, conceptual issues having to do
with theories in which there is no "background metric".  Unfortunately,
there are SEVERAL DIFFERENT WAYS to set up 3-dimensional quantum
gravity, corresponding to different approaches people have to
4-dimensional quantum gravity.  For this, check out Carlip's paper "Six
ways to quantize (2+1)-dimensional gravity," mentioned in "<A HREF = "week16.html">week16</A>".
However, I think the "best" way to quantize gravity in 3 dimensions is
the way involving Chern-Simons theory, because this way is the most
closely related to Ashtekar's approach to quantizing gravity in 4
dimensions, hence it sheds the most light on the things I'm interested
in --- and I also think it's the most beautiful.  In this approach, you
can compute a lot of things, and basically what Carlip has done is to
show that associated to the event horizon there are degrees of freedom
which should give entropy proportional to its area.

I suppose I can't say how he does it much more clearly than he says it,
so I'll quote the introduction, taking the liberty of turning some of
his LaTeX into English.  If you get scared by the "Virasoro operator
L_0" below, never fear --- in this context, it just amounts to the
angular momentum operator, which generates rotations about the origin.
So:

\par\noindent\rule{\textwidth}{0.4pt}
The basic argument is quite simple.  Begin by considering general
relativity on a manifold M with boundary.  We ordinarily split the
metric into true physical excitations and ``pure gauge'' degrees of
freedom that can be removed by diffeomorphisms of M.  But the presence
of a boundary alters the gauge invariance of general relativity: the
infinitesimal transformations [...] must now be restricted to those
generated by vector fields [...] with no component normal to the
boundary, that is, true diffeomorphisms that preserve the boundary of M.
As a consequence, some degrees of freedom that would naively be viewed
as ``pure gauge'' become dynamical, introducing new degrees of freedom
associated with the boundary.

Now, the event horizon of a black hole is not a true boundary, although
the black hole complementarity approach of Susskind et al. suggests
that it might be appropriately treated as such.  Regardless of one's
view of that program, however, it is clear that in order to ask quantum
mechanical questions about the behavior of black holes, one must put in
``boundary conditions'' that ensure that a black hole is present.  This
means requiring the existence of a hypersurface with particular metric
properties---say, those of an apparent horizon.

The simplest way to do quantum mechanics in the presence of such a
surface is to quantize fields separately on each side, imposing the
appropriate correlations as boundary conditions.  In a path integral
approach, for instance, one can integrate over fields on each side,
equate the boundary values, and finally integrate over those boundary
values compatible with the existence of a black hole.  But this process
again introduces boundary terms that restrict the gauge invariance of
the theory, leading once more to the appearance of new degrees of
freedom at the horizon that would otherwise be treated as unphysical.

My suggestion is that black hole entropy is determined by counting these
would-be gauge degrees of freedom.  The resulting picture is similar to
Maggiore's membrane model of the black hole horizon, but with a
particular derivation and interpretation of the ``membrane'' degrees of
freedom.

The analysis of this phenomenon is fairly simple in 2+1 dimensions.  It
is well known that (2+1)-dimensional gravity can be written as a
Chern-Simons theory, and it is also a standard result that a
Chern-Simons theory on a manifold with boundary induces a dynamical
Wess-Zumino-Witten (WZW) theory on the boundary.  In the presence of a
cosmological constant Lambda = -1/L^2 appropriate for the
(2+1)-dimensional black hole, one obtains a slightly modified SO(2,1) x
SO(2,1) WZW model, with coupling constant

k = L sqrt(2)/8G 

This model is not completely understood, but in the large k --- i.e.,
small \Lambda  --- limit, it may be approximated by a theory of six
independent bosonic oscillators.  I show below that the Virasoro
operator L_0 for this theory takes the form

L_0 ~ N - (r/4G)^2,

where N is a number operator and r is the horizon radius.  It is a
standard result of string theory that the number of states of such a
system behaves asymptotically as

n(N) ~ exp(\pi  sqrt 4 N)

If we demand that L_0 vanish --- physically, requiring states to
be independent of the choice of origin of the angular coordinate at the
horizon --- we thus obtain

log n(r) ~ (2 \pi  r)/4G ,

precisely the right expression for the entropy of the (2+1)-dimensional
black hole.  
\par\noindent\rule{\textwidth}{0.4pt}


Also, Carlo Rovelli spoke about describing the dynamics of quantum
gravity coupled to a scalar field in terms of "spin network" states.  I think
this was based on work he did in collaboration with Lee Smolin, and I
don't think it's out yet.  I'm just about to finish up a little paper on
spin network states myself, since they seem like very useful things in
quantum gravity.  The simplest sort of spin network is just a
trivalent graph (i.e., 3 edges adjacent to each vertex) with edges
labelled by "spins" 0,1/2,1,3/2,..., and satisfying the "triangle
inequality" at each vertex:

$$
         j1 + j2 <= j3,      j2 + j3 <= j1,      j3 + j1 <= j2,
$$
    

where j1, j2, j3 are the spins labelling the edges adjacent to the given
vertex.  Really, the spins should be thought of as irreducible
representations of SU(2), and the triangle inequalities is necessary
for the representation j3 to appear as a summand in the tensor
product of the representations j1 and j2.  (If the last sentence was
meaningless to you, reading "<A HREF = "week5.html">week5</A>" will help a little, though probably
not quite enough.)  

Penrose introduced spin networks as part of a purely combinatorial
approach to spacetime in the paper:

2) Angular momentum; an approach to combinatorial space
time, by Roger Penrose, in "Quantum Theory and Beyond," ed. T. Bastin,
Cambridge University Press, Cambridge, 1971.

It is somehow satisfying, therefore, to see that spin networks arise
naturally as a convenient description of states in the loop
representation of quantum gravity, which STARTS mainly with Einstein's
equations and the principles of quantum mechanics.  Certainly there is a
lot more we need to learn about them.... One place worth reading about
them is:

3) Conformal field theory, spin geometry, and quantum gravity, by Louis
Crane, Phys. Lett. B259 (1991), 243-248.

I will be coming out with a paper on them next week if I get my act
together, and I may say a bit more about them in future "weeks".  

Rovelli also mentioned an interesting paper he wrote about the problem of
time in quantum gravity with the operator-algebra/ noncommutative-
geometry guru Alain Connes:

4) Von Neumann algebra automorphisms and time-thermodynamics relation in
general covariant quantum theories, by A. Connes and C. Rovelli, 25
pages in LaTex format available as <A HREF = "http://xxx.lanl.gov/abs/gr-qc/9406019">gr-qc/9406019</A>.  

The problem of time in quantum gravity is a bit tricky to describe,
since it takes different guises in different approaches to quantum
gravity, but I have attempted to give a rough introduction to it in
"<A HREF = "week11.html">week11</A>" and "<A HREF = "week27.html">week27</A>".  One way to get a feeling for it is to realize
that anything you are used to doing with Hamiltonians in quantum
mechanics or quantum field theory, you CAN'T do in quantum gravity, at
least not in any simple way, because there is no Hamiltonian in general
relativity, but only a ``Hamiltonian constraint'' --- which in quantum
gravity becomes the Wheeler-DeWitt equation

$$
                         H \Psi  = 0.
$$
    

Now, people know there is a mystical relationship between time and
temperature that might be written

\begin{verbatim}
                         it = 1/kT
\end{verbatim}
    

where t is time, T is temperature, and k is Boltzmann's constant.  This
equation is a bit of an exaggeration!  But the point is that in quantum
theory, when there is a Hamiltonian H around one evolves states using the
operator 

\begin{verbatim}
                         exp(-itH)
\end{verbatim}
    

while the Gibbs state, that is, the equilibrium state at temperature T,
is given by the density matrix

\begin{verbatim}
                         exp(-H/kT).
\end{verbatim}
    

It is this fact that relates statistical mechanics and quantum field
theory so closely.  

Now, in quantum gravity things aren't so simple, since there isn't a 
Hamiltonian (just a Hamiltonian constraint).  However, people \emph{do} know
that there are all sorts of funny relationships between statistical
mechanics and quantum gravity.  For example, an accelerating observer in
Minkowski space will see the vacuum as a heat bath with temperature
proportional to her acceleration, so in curved spacetime, where there
are no truly inertial frames, there really is no well-defined notion of
a vacuum; in some vague sense, all there are is "thermal" states.
This fact is also somehow related to Hawking radiation, and to the
notion of black hole entropy... but really, there is a lot that nobody
understands about all these connections!  

In any event, Rovelli was prompted to use thermodynamics to DEFINE time
in quantum gravity as follows.  Given a mixed state with density matrix
D, \emph{find} some operator H such that D is the Gibbs state exp(-H/kT).
In lots of cases this isn't hard; it basically amounts to 

\begin{verbatim}
                        H = -kT ln D
\end{verbatim}
    

Of course, H will depend on T, but this really is just saying that
fixing your units of temperature fixes your units of time!

Operator theorists have pondered this notion very carefully for a long
time and generalized it into something called the Tomita-Takesaki
theorem, which Connes and Rovelli explain.  This gives a very general
way to cook up a Hamiltonian (hence a notion of time evolution) from a
state of a quantum system!  For example, one can use this trick to start
with a Robertson-Walker universe full of blackbody radiation, and
recover a notion of "time".  This is very intriguing, and it may
represent some real progress in understanding the deep relations between
time, thermodynamics, and gravity.  There are, of course, lots of
problems and puzzles to deal with.  

Another intriguing talk at the conference was given by Viqar Husain, on
the subject of the following paper:

5) The affine symmetry of self-dual gravity, by Viqar Husain, 17 pages
in ReVTeX format available as <A HREF = "http://xxx.lanl.gov/abs/hep-th/9410072">hep-th/9410072</A>.  

Let me simply quote the abstract, since I don't feel I really understand
the essence of this business well enough to say anything useful yet:

\par\noindent\rule{\textwidth}{0.4pt}
Self-dual gravity may be reformulated as the two dimensional chiral
model with the group of area preserving diffeomorphisms as its gauge
group.  Using this formulation, it is shown that self-dual gravity
contains an infinite dimensional hidden symmetry algebra, which is the
Affine (Kac-Moody) algebra associated with the Lie algebra of area
preserving diffeomorphisms.  This result provides an observable algebra
and a solution generating technique for self-dual gravity.
\par\noindent\rule{\textwidth}{0.4pt}

A couple more things before I wrap this up....  First, in case any
mathematicians out there are wondering what this "knots and quantum
gravity" business is all about, here's something I wrote to review the
subject: 

6) Knots and Quantum Gravity: Progress and Prospects, John Baez, 22
pages in LaTeX format, available as <A HREF = "http://xxx.lanl.gov/abs/gr-qc/9410018">gr-qc/9410018</A>.

My abstract:

\par\noindent\rule{\textwidth}{0.4pt}
Recent work on the loop representation of quantum gravity has revealed
previously unsuspected connections between knot theory and quantum
gravity, or more generally, 3-dimensional topology and 4-dimensional
generally covariant physics.  We review how some of these relationships
arise from a `ladder of field theories' including quantum gravity and BF
theory in 4 dimensions, Chern-Simons theory in 3 dimensions, and the G/G
gauged WZW model in 2 dimensions.  We also describe the relation between
link (or multiloop) invariants and generalized measures on the space of
connections.  In addition, we pose some research problems and describe
some new results, including a proof (due to Sawin) that the Chern-Simons
path integral is not given by a generalized measure.
\par\noindent\rule{\textwidth}{0.4pt}

Finally, let me draw people's attention to "Matters of Gravity", the 
newsletter Jorge Pullin puts together at considerable effort, to keep
people informed about general relativity and the like, experimental and
theoretical:

7) "Matters of Gravity", a newsletter for the gravity community, Number
4, edited by Jorge Pullin, 24 pages in Plain TeX, available as
<A HREF = "http://xxx.lanl.gov/abs/gr-qc/9409004">gr-qc/9409004</A>, or from WWW by http://vishnu.nirvana.phys.psu.edu/ 

Here's the table of contents of this issue:

\begin{verbatim}
Editorial.
Gravity News:
 Report on the APS topical group in gravitation, Beverly Berger.
Research briefs:
 Gravitational microlensing and the search for dark matter, Bohdan Paczynski.
 Laboratory gravity: the G mystery, Riley Newman.
 LIGO project update, Stan Whitcomb.
Conference Reports
 PASCOS '94, Peter Saulson.
 The Vienna Meeting, P. Aichelburg, R. Beig.
 The Pitt binary black hole grand challenge meeting, Jeff Winicour.
 International symposium on experimental gravitation at Pakistan, 
Munawar Karim.
 10th Pacific coast gravity meeting, Jim Isenberg.
\end{verbatim}
    
<HR>

% </A>
% </A>
% </A>


% parser failed at source line 436
