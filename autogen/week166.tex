
% </A>
% </A>
% </A>
\week{March 27, 2001}

Do you know this number?

2.685452001065306445309714835481795693820382293994462953051152....

They say that mathematics is not really about numbers, and 
they're right.  But sometimes it's fun to play around with 
the darn things!  

Given any positive number you can work out its continued 
fraction expansion, like this:


\begin{verbatim}

sqrt(2) =  1 +   1
               ---------
                2 +  1
                   -------- 
                    2 +  1
                      -------
                       2 +  1
                          ------
                           2 + 1
                              ----  
                               2 + 1
                                  ----
                                     .
                                       .
                                         .

\end{verbatim}
    
But normally it won't look so pretty!  A number is rational if
and only if the continued fraction stops after finitely many steps.  
If its continued fraction expansion eventually repeats, like this:


\begin{verbatim}

sqrt(3) =  1 +   1
               ---------
                1 +  1
                  -------- 
                   2 +   1
                      -------
                       1 +  1
                          ------
                           2 + 1
                              ----  
                               1 + 1
                                  ----
                                     .
                                       .
                                         .
\end{verbatim}
    
then it satisfies a quadratic equation with integer coefficients.  
So the continued fraction expansion of e can't ever repeat... but
it's cute nonetheless:


\begin{verbatim}

  e  =   2  +    1
               ---------
                1 +  1
                  --------
                   2 +   1
                      -------
                       1 +  1
                          ------
                           1 + 1
                              ------
                               4 + 1
                                  ------
                                   1 + 1
                                       ------
                                        1 + 1
                                           ---- 
                                            6 + 1 
                                               ----
                                                 .
                                                    .
                                                       .
\end{verbatim}
    
It continues on predictably after that initial hiccup.  The number \pi ,
on the other hand, gives a random-looking mess.  This is a hint that
\pi  is number-theoretically more complicated than e, which is also
apparent when you compare the proofs that e and \pi  are transcendental
- the proof for e is much easier.

Pondering all this, it's natural to ask about the "average" behavior
of the continued fraction expansion of a number.   What's the average
behavior of the series a_{1}, 
a_{2}, a_{3}, .... that we get this way:


\begin{verbatim}

       x =  a_{1} +   1
                ---------
                a_{2} +  1
                    -------- 
                    a_{3} +  1
                         -------
                          a_{4} +  1
                                ----
                                   .
                                     .
                                       .
                                         ?
\end{verbatim}
    
It turns out that if we take the geometric mean of the first n terms
and then let n approach \infty , the mean almost always converges to
"Khinchin's constant" - the number at the beginning of this article!
Here by "almost always" I mean that the set of exceptions has measure
zero.  One can prove this using some ideas from ergodic theory.



Now, there is much more to say about continued fraction expansions,
but my real goal is simply to point out that there are lots of
interesting constants in mathematics besides \pi , e, the golden
ratio, and Euler's number.  Where can you read about them?  Here:

1) Steven Finch, MathSoft Constants,
<a href = "http://pauillac.inria.fr/algo/bsolve/constant/constant.html">http://pauillac.inria.fr/algo/bsolve/constant/constant.html</a>

This is a great place to learn about Khinchin's constant,
Feigenbaum's number, Madelung's constant, Artin's constant, 
Grothendieck's constant, and many other fun numbers!   

Speaking of fun websites, here's another:

2) The Mathematics Genealogy Project, <a href = "http://hcoonce.math.mankato.msus.edu/">http://hcoonce.math.mankato.msus.edu/</a>

My advisor's advisor's advisor's advisor's advisor's advisor's
advisor's advisor was Gauss.  If you think I'm showing off, you're
right!  But I couldn't have done it without this website, and if
you're a mathematician, there's a good chance you use it to track 
down \emph{your} academic lineage.  And if you can't, you can at least 
add your information to the database.

Before Demian Cho showed me this site, I'd gotten stuck 3 generations
back in my attempts to discover my academic ancestors.  Now I can go
back 11 generations.  I know it's annoying, but I'm gonna tell you the
whole story:

My advisor was Irving Segal, the guy who helped prove the
Gelfand-Naimark-Segal theorem.  This is a basic result about
C*-algebras, a kind of gadget he invented to formalize the notion of
an "algebra of observables" in quantum theory.  The GNS theorem
implies that every C*-algebra sits inside the algebra of all bounded
operators on some Hilbert space, so it's a kind of justification for
using Hilbert spaces in quantum physics.  But even better, it gives a
procedure for representing a C*-algebra as operators on a Hilbert
space starting from a "state" on the C*-algebra.  The upshot is that
while Hilbert spaces are important, the right Hilbert space to use can
depend on the state of the system you're studying.  At first people 
thought Segal was nuts for saying this, but by now it's well-accepted.

Segal also did work on quantum field theory, nonlinear partial
differential equations, and other topics at the borderline between
physics and functional analysis.  His students include Isadore Singer
and Bertram Kostant, whose work on geometric quantization generalized
Segal's ideas on the "Bargmann-Segal representation".  I worked with
Segal because I liked analysis and wanted to understand quantum field
theory in a rigorous way.  

Segal's advisor was Einar Hille, the guy who helped prove the
Hille-Yosida theorem.  Hille did a lot of work on integral and
differential equations, but later he became interested in functional
analysis: the study of infinite-dimensional vector spaces equipped
with nice topologies, such as Hilbert spaces, Banach spaces and the
like.  At the time, he was rather special in his emphasis on applying
these abstract ideas to concrete problems.  In his book "Methods in
Classical and Functional Analysis," he wrote:

\begin{quote}
     If the book has a thesis, it is that a functional analyst is an
     analyst, first and foremost, and not a degenerate species of a
     topologist.  His problems come from analysis and his results 
     should throw light on analysis....
\end{quote}
The Hille-Yosida theorem shows how to write a large class of
one-parameter semigroups of linear operators on Banach spaces in the
form exp(-tH).  These so-called "contraction semigroups" naturally
come from the heat equation and its relatives.  Segal was fond of this
idea, and he generalized it to semigroups of nonlinear operators,
which arise naturally from \emph{nonlinear} partial differential equations.
He used this idea to prove global existence of solutions for various
nonlinear classical field theories.

Hille's advisor was Marcel Riesz, the guy who didn't prove the Riesz
representation theorem.  Marcel's brother Frigyes was the guy who did
that.  Marcel worked on functional analysis, partial differential
equations, and mathematical physics - even Clifford algebras and
spinors!  

The advisor of Marcel Riesz was Lipot Fejer, the guy who discovered
the Fejer kernel.  This shows up when you sum Fourier series.  If you
just naively sum the Fourier series of a continuous function on the
circle, it may not converge uniformly.  However, if you use a trick
called Cesaro summation, which amounts to averaging the partial sums,
you get uniform convergence.  The average of the first n partial sums
of the Fourier series of your function is equal to its convolution
with the Fejer kernel.  Fejer also worked on conformal mappings.  His
students included Paul Erdos and Gabor Szego.

Fejer's advisor was Karl Herman Amandus Schwarz, the guy who helped
prove the Cauchy-Schwarz inequality.  That's a wonderful inequality
which everyone should know!  But Schwarz also worked on minimal
surfaces and complex analysis: for example, conformal mappings from
polyhedra into the sphere, and also the Dirichlet problem.  Don't mix
him up with Laurent Schwartz, the guy who invented distributions.

(Actually, Lipot Fejer's name was originally Leopold Weiss.  He changed
it to seem more Hungarian.  This was a common practice at the time in 
Hungary, but when he did it, his advisor Schwarz stopped speaking to him!)

Schwarz's advisor was Karl Weierstrass, the guy who proved the Weierstrass 
theorem.  This theorem says that every continuous real-valued function
on the unit interval is a uniform limit of polynomials.  Weierstrass
also has a function named after him: the Weierstrass elliptic function,
which I explained in "<A HREF = "week13.html">week13</A>".  But his real claim to fame is how he
made analysis more rigorous!  For example, he discovered the importance
of uniform convergence, and found a continuous function with no derivative 
at any point.   Besides Schwarz, his students include Frobenius, Killing,
and Kowalevsky.

Now, Weierstrass doesn't have an advisor listed in the mathematics
genealogy.  However, by using this website full of mathematician's
biographies, I can go back further:

3) John J. O'Connor and Edmund F. Robertson, The MacTutor History of
Mathematics Archive, 
<a href = "http://www-groups.dcs.st-andrews.ac.uk/~history/index.html">http://www-groups.dcs.st-andrews.ac.uk/~history/index.html</a>

According to this, Weierstrass had an erratic career as a student: his
father tried to make him study finance instead of math, so he spent
his undergraduate years fencing and drinking.  He learned a lot of
math on his own, and got really interested in elliptic functions from
the work of Abel and Jacobi.  I can't tell if he ever had an official
dissertation advisor.  However, in 1839 he went to the Academy at
Muenster to study under Christoph Gudermann, who worked on elliptic 
functions and spherical geometry.  Gudermann strongly encouraged 
Weierstrass in his mathematical studies.  Weierstrass asked for a
question on elliptic functions, and wound up writing a paper which
Gudermann assessed "... of equal rank with the discoverers who were
crowned with glory."  (When Weierstrass heard this, he commended
Gudermann's generosity, since he had strongly criticized 
Gudermann's methods.)
Given all this, and the fact that Weierstrass seems 
to have had no \emph{other} mentor, I'll declare Gudermann to be his 
advisor, de facto even if not officially.

But who was Gudermann?  He's the guy they named the "gudermannian" 
after!  That's this function:

gd(u) = 2 arctan(exp(u)) - \pi /2.

Now, if you're wondering why such a silly function deserves a name,
you should work out its inverse function:

gd^{-1}(x) = ln(sec(x) + tan(x)).

And if you don't recognize \emph{this}, you probably haven't taught
freshman calculus lately!  It's the integral of sec(x), which is one
of the hardest of the basic integrals you teach in that kind of
course.  But it's not just hard, it's historically important: a point
at latitude gd(u) has distance u from the equator in a Mercator
projection map.  If you think about it a while, this is precisely
what's needed to make the projection be a conformal transformation -
that is, angle-preserving.  And that's just what you want if you're
sailing a ship in a constant direction according to a compass and you
want to know where you'll wind up.

If you don't see how this works, try:

4) Wikipedia, <a href = "http://en.wikipedia.org/wiki/Mercator_projection">http://en.wikipedia.org/wiki/Mercator_projection</a>

Gudermann's advisor was Carl Friedrich Gauss, the guy they named
practically \emph{everything} after!  Poor Gudermann, who was content to mess
around with special functions and spherical geometry, seems to have been
one of Gauss' worst students.  But that's not so bad, since three of the
other four were Bessel, Dedekind and Riemann.   

Gauss' advisor was Johann Pfaff, the guy they named the "Pfaffian"
after.  If the matrix A is skew-symmetric, we can write

det(A) = Pf(A)^{2}

where Pf(A) is also a polynomial in the entries of A.  Pfaffians now
show up in the study of fermionic wavefunctions.  Pfaff worked on
various things, including the integrability of partial differential
equations, where the concept of a "Pfaffian system" is important.
Unfortunately I've never gotten around to understanding these.

Pfaff's advisor was Abraham Kaestner.  I'd never heard of him before
now.  He wrote a 4-volume history of mathematics, but his most important
work was on axiomatic geometry.  His interest in the parallel postulate
indirectly got Gauss, Bolyai and Lobachevsky interested in that topic:
we've already seen that he taught Gauss' advisor, but he also taught
Bolyai's father, as well as Lobachevsky's teacher, one J. M. C. Bartels.
In fact, Kaestner was still teaching when Gauss went to school, but
Gauss didn't go to Kaestner's courses, because he found them too
elementary.  Gauss said of him, "He is the best poet among
mathematicians and the best mathematician among poets".  Perhaps 
this faint praise refers to Kaestner's knack for aphorisms.    


At this point I got stuck until my student Miguel Carrion-Alvarez
helped out.  It appears that Kaestner's advisor was one Christian
A. Hausen.  He's the guy they named the Hausen crater after - a lunar
crater located at 65.5 S, 88.4 W.  He did his thesis on theology in
1713, but became a professor of mathematics in Leipzig.  He worked on
electrostatics, but made no memorable discoveries.


 At this point the trail disappears into mist.  For some
conjectures, see this page:

5) Anthony M. Jacobi, Academic Family Tree,
<a href = "http://www.staff.uiuc.edu/%7Ea-jacobi/tree.html">http://www.staff.uiuc.edu/%7Ea-jacobi/tree.html</a>


It's interesting how the same themes keep popping up in this
genealogy.  For example, Weierstrass invented uniform convergence and
proved that the limit of a uniformly convergent series of continuous
functions is continuous.  The Fejer kernel shows up when you're trying
to write functions on the circle as a uniformly convergent sum of
complex exponentials.  Segal's C*-algebras generalize the notion of
uniform convergence to operator algebras.  I guess these things just
go from generation to generation....

A little while ago John McKay visited me and told me about all sorts of
wonderful things: relations between subgroups of the Monster group,
exceptional Lie groups, and modular forms... a presentation of the
Monster group with 2 generators, a way to build the Leech lattice from 3
copies of the E8 lattice... a way to get ahold of the Monster group
starting with a diagram with 26 nodes....

Unfortunately, I'm having trouble finding references for some of these
things!  It's possible that the last two items are really these:


6) Robert L. Griess, Pieces of eight: semiselfdual lattices and a new
foundation for the theory of Conway and Mathieu groups. Adv. Math. 148
(1999), 75-104.  

7) John H. Conway, Christopher S. Simons, 26 implies the Bimonster,
Jour. Algebra 235 (2001), 805-814.

Anyway, I need to read about this stuff.


Speaking of exceptionology: in "<A HREF =
"week163.html">week163</A>" I explained how Spin(9) sits inside the
Lie group F4, thanks to the fact that Spin(9) is the automorphism group
of Jordan algebra of 2x2 hermitian octonionic matrices, and F4 is the
automorphism group of the Jordan algebra of 3x3 hermitian matrices.  But
in fact, since there are different ways to think of 2x2 matrices as
special 3x3 matrices, there are actually 3 equally good ways to stuff
Spin(9) in F4.  Since I'd been hoping this might be important in
particle physics, it was nice to discover that Pierre Ramond, a real
expert on this stuff, has had similar thoughts.  In fact he's written
two papers on this!  Let me just quote the abstracts:

8) Pierre Ramond, Boson-fermion confusion: the string path to
supersymmetry, available at <A HREF = "http://xxx.lanl.gov/abs/hep-th/0102012">hep-th/0102012</A>.

Reminiscences on the string origins of supersymmetry are followed by a
discussion of the importance of confusing bosons with fermions in
building superstring theories in 9+1 dimensions. In eleven dimensions,
the kinship between bosons and fermions is more subtle, and may
involve the exceptional group F4.

9) T. Pengpan and Pierre Ramond, M(ysterious) patterns in SO(9),
Phys. Rep. 315 (1999) 137-152, also available as <A HREF = "http://xxx.lanl.gov/abs/hep-th/9808190">hep-th/9808190</A>.

The light-cone little group, SO(9), classifies the massless degrees of
freedom of eleven-dimensional supergravity, with a triplet of
representations. We observe that this triplet generalizes to four-fold
infinite families with the quantum numbers of massless higher spin
states. Their mathematical structure stems from the three equivalent
ways of embedding SO(9) into the exceptional group F4.


\par\noindent\rule{\textwidth}{0.4pt}
<em>
"This is why we are here," said Teacher, "to be good
and kind to other people."

Pippi stood on her head on the horse's back and waved her legs in the
air.  "Heigh-ho," said she, "then why are the other
people here?" </em> 

Astrid Lingren, \emph{Pippi Goes on Board}

\par\noindent\rule{\textwidth}{0.4pt}

% </A>
% </A>
% </A>
