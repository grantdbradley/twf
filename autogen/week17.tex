
% </A>
% </A>
% </A>
\week{June 13, 1993}

This'll be the last "This Week's Finds" for a few weeks, as I am going up to
disappear until July.  I've gotten some requests for introductory
material on gauge theory, knot theory, general relativity, TQFTs and
such recently, so I just made a list
of <A HREF = "books.html">some of my favorite books</A> on this kind of 
thing - with an emphasis on the readable ones.  

Also, just as a little plug here, a graduate student here at UCR (Javier
Muniain) and I are turning my course notes from this year into a book
called "Gauge Fields, Knots and Gravity," meant to be an elementary
introduction to these subjects.  This will eventually be published by
World Scientific if all goes well.  It will gently remind the reader
about manifolds, differential forms, Lagrangians, etc., develop a little
gauge theory, knot theory, and general relativity, and at the very end
it'll get to the relationship between knot theory and quantum gravity -
at which point one could read more serious stuff on the subject.   

A while back Lee Rudolph asked my opinion of the following article:

1) ``Theoretical Mathematics'': Toward a cultural synthesis of
mathematics and theoretical physics, by Arthur Jaffe and Frank Quinn, to
appear in the July 1993 Bulletin of the AMS (apparently available by
gopher at e-math.ams.com, but don't ask me how since I couldn't get it
that way).

People who are seriously into mathematical physics will know that with
string theory the interaction between mathematicians and physicists,
especially mathematicians who haven't traditionally been close to
physics (e.g. algebraic geometers), has strengthened steadily for the
last 10 years or so.  Physicists are coming up with lots of exciting
mathematical "results" - often NOT rigorously proved! - and
mathematicians are getting very interested.  Let me quote the abstract:

\begin{quote}
Is speculative mathematics dangerous?  Recent interactions between
physics and mathematics pose the question with some force:  traditional
mathematical norms discourage speculation; but it is the fabric of
theoretical physics.  In practice there can be benefits, but there can also
be unpleasant and destructive consequences.  Serious caution is required,
and the issue should be considered before, rather than after obvious
damage occurs.  With the hazards carefully in mind, we propose a
framework that should allow a healthy and a positive role for speculation.
\end{quote}

Replies have been solicited, so there may be a debate on this timely
subject in the AMS Bulletin.  This subject has a great potential for
flame wars - or, as Greeks and academics refer to them, "polemics."  Luckily
Jaffe and Quinn take a rather careful and balanced tone.   I think
anyone interested in the culture of mathematics and physics should take
a look at this.

Now for two books:

2) New Scientific Applications of Geometry and Topology, ed. DeWitt L.
Sumner, Proc. Symp. Appl. Math. 45, published by the AMS.

This volume has a variety of introductory papers on applications of knot
theory; the titles are roughly "Evolution of DNA topology," "Geometry
and topology of DNA and DNA-protein interactions," "Knot theory and
DNA," "Topology of polymers," "Knots and Chemistry," and "Knots and
physics."  

3) Temperley-Lieb Recoupling Theory and Invariants of 3-Manifolds, by
Louis Kauffman and Sostenes Lins, to be published by Princeton U. Press.

This is an elegant exposition of the 3-manifold invariants obtained from the
quantum group SU_q(2) - or in other words, from Chern-Simons theory.  In
part this is a polishing of existing work, but it also contains some
interesting new ideas.

And now for some papers:

4) 12j-symbols and four-dimensional quantum gravity, by M. Carfora, M.
Martellini (martellini@milano.infn.it), and A. Marzuoli, Dipartimento di
Fisica, Universita di Roma "La Sapienza" preprint.

This is an attempt to do for 4d quantum gravity what Regge, Ponzano and
company so nicely did for 3d quantum gravity (see "<A HREF = "week16.html">week16</A>") - describe
it using triangulated manifolds and angular momentum theory.

5) Selected topics in quantum groups, by Y. S. Soibelman
(soibel@math.harvard.edu), Lectures for the European School of Group
Theory, Harvard University preprint.

This is a nice review of Soibelman's work on quantum groups, quantum
spheres, and other aspects of "quantum geometry."

6) Braids and movies, by J. Scott Carter (carter@mathstat.usouthal.edu)
and Masahico Saito, preprint. 

Just as every knot or link is given as the closure of a braid - for
example, the trefoil knot 


\begin{verbatim}

 ________________________
/    _______________     \ 
\   /               \     |
 \ /                 |    |
  /                  |    |    
 / \                 |    |
/   \                |    |
\   /                |    |
 \ /                 |    |
  /                  |    |    
 / \                 |    |
/   \                |    |
\   /                |    |
 \ /                 |    |
  /                  |    |    
 / \                 |    |
/   \________________/    |
\_________________________/
\end{verbatim}
    

is the closure of


\begin{verbatim}

\   /       
 \ /        
  /         
 / \        
/   \       
\   /       
 \ /        
  /         
 / \        
/   \       
\   /       
 \ /        
  /         
 / \        
/   \ ,
\end{verbatim}
    

every "2-knot" or "2-link" - that is, a surface embedded in R^4, is the
closure of a "2-braid".  Just as there are "Markov moves" that say when
two links come from the same braid, there are moves for 2-braids - as
discussed here.

7) Combinatorial Invariants from Four Dimensional Lattice Models: II, by
Danny Birmingham and Mark Rakowski, preprint available in LaTeX form as
<A HREF = "http://xxx.lanl.gov/abs/hep-th/9305022">hep-th/9305022</A>.

The previous paper obtains some invariants of 4-manifolds by
triangulating them and doing a kind of "state sum" much like those I
described in "<A HREF = "week16.html">week16</A>".  This paper shows those invariants are trivial -
at least for compact manifolds, where one just gets the answer "1".
This seems to be happening a lot lately.

8)  A note on the four-dimensional Kirby calculus, by Boguslaw Broda,
preprint, 5 pages in TeX available as <A HREF = "http://xxx.lanl.gov/abs/hep-th/9305101">hep-th/9305101</A>.

An earlier attempt by Broda to construct 4-manifold invariants along
similar lines was discussed here in "<A HREF = "week9.html">week9</A>" and "<A HREF = "week10.html">week10</A>" - the upshot
being that the invariant was trivial.  This is a new attempt and Broda
has told me that the arguments for the earlier invariant being trivial
do not apply.  Here's hoping!

9)  Solutions to the Wheeler DeWitt Constraint of Canonical Gravity Coupled to
Scalar Matter Fields, by H.-J. Matschull, preprint, 7 pages in LaTeX
available as <A HREF = "http://xxx.lanl.gov/abs/gr-qc/9305025">gr-qc/9305025</A>.

One very important technical issue in the loop representation of quantum
gravity is how to introduce matter fields into the picture.  Let quote:

It is shown that the Wheeler DeWitt constraint of canonical gravity
coupled to Klein Gordon scalar fields and expressed in terms
of Ashtekar's variables admits formal solutions which are parametrized
by loops in the three dimensional hypersurface and which are
extensions of  the well known Wilson loop solutions found by
Jacobson, Rovelli and Smolin.

10) Hilbert space of wormholes, by Luis J. Garay, preprint, 23 pages (2
figures available upon request from garay@cc.csic.es) available in REVTEX as
<A HREF = "http://xxx.lanl.gov/abs/gr-qc/9306002">gr-qc/9306002</A>.   
 
I think I'll just quote the abstract on this one:


\begin{verbatim}

Wormhole boundary conditions for the Wheeler--DeWitt equation
can be  derived from the path integral formulation. It is
proposed that the wormhole wave function must be square
integrable in the maximal analytic extension of minisuperspace.
Quantum wormholes can be invested with a Hilbert space
structure, the inner product being naturally induced by the
minisuperspace metric, in which the Wheeler--DeWitt operator is
essentially self--adjoint. This provides us  with a kind of
probabilistic interpretation.  In particular, giant wormholes
will give extremely small contributions to any wormhole state.
We also study the whole spectrum of the Wheeler--DeWitt
operator and its role in the calculation of Green's functions
and effective low energy interactions.
\end{verbatim}
    

11) Chern-Simons theory as topological closed string,
Vipul Periwal, preprint, 7 pages available as <A HREF = "http://xxx.lanl.gov/abs/hep-th/9305115">hep-th/9305115</A>.  

Lately people have been getting interested in gauge theories that can be
interpreted as closed string field theories.  I mentioned one recent
paper along these lines in "<A HREF = "week15.html">week15</A>," which considers Yang-Mills in 2
dimensions.  (This was not the first paper to do so, I should
emphasize.)  A while back Witten wrote a paper on Chern-Simons gauge
theory in 3 dimensions as a background-free open string field theory,
but I was unable to understand it.  This paper seems conceptually
simpler, although it uses some serious mathematics.  I think I might be
able to understand it.  It starts:


\begin{verbatim}

The perturbative expansion of any quantum field theory (qft) with fields
transforming in the adjoint representation of SU(N) is
a topological expansion in surfaces, with N^{-2} playing the role
of a handle-counting parameter.  For N large, one hopes that the
dynamics of the qft is approximated by the sum (albeit largely
intractable) of all planar diagrams.  The topological classification of
diagrams has nothing a priori to do with approximating the dynamics with
a theory of strings evolving in spacetime.

Gross (see also refs...) has shown recently that the large N expansion
does actually provide a way of associating a theory of strings in QCD.
Maps of two-dimensional string worldsheets into two-dimensional
spacetimes are necessarily somewhat constricted.  What one would like is
a qft with fields transforming in the adjoint representation in d > 2,
which is at the same time exactly solvable.  One could then, in
principle, attempt to associate a theory of strings with such a qft by
exhibiting a `sum over connected surfaces' interpretation for the free
energy of the qft. There is no guaranty that such an association will exist.
\end{verbatim}
    

The author argues that Chern-Simons theory is a "rara avis among QFTs"
for which such an association exists.  He takes the free energy
for SU(N) Chern-Simons theory on S^3, does a large-N expansion on it, and shows
that the coefficient of the N^{2-2g} term is the (virtual) Euler
characteristics of the moduli space of surfaces with g handles.  I wish
I understood this better at a very pedestrian level!   E.g., is there
some string-theoretic reason why one might expect the free energy to be
of this form? Anyway, then he considers T^3, and gets something
related to surfaces with a single puncture in them.   
\par\noindent\rule{\textwidth}{0.4pt}

% </A>
% </A>
% </A>
