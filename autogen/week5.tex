
% </A>
% </A>
% </A>
\week{ebruary 13, 1993}

I think I'll start out this week's list of finds with an elementary
introduction to Lie algebras, so that people who aren't "experts" can
get the drift of what these are about.  Then I'll gradually pick up
speed...

1) Vyjayanathi Chari and Alexander Premet, Indecomposable restricted
representations of quantum sl_{2}, University of California at
Riverside preprint.

Vyjayanathi is our resident expert on quantum groups, and Sasha, who's
visiting, is an expert on Lie algebras in characteristic p.  They have
been talking endlessly across the hall from me and now I see that it has
born fruit.  This is a pretty technical paper and I am afraid I'll never
really understand it, but I can see why it's important, so I'll try to 
explain that!  

Let me start with the prehistory, which is the sort of thing everyone
should learn.  Recall what a Lie algebra is... a vector space
with a "bracket" operation such that the bracket [x,y] of any two vectors
x and y is again a vector, and such that the following hold:


\begin{verbatim}

a) skew-symmetry:  [x,y] = -[y,x].
b) bilinearity:    [x,ay] = a[x,y],  
                   [x,y+z] = [x,y] + [x,z].  (a is any number) 
c) Jacobi identity: [x,[y,z]] + [y,[z,x]] + [z,[x,y]] = 0.
\end{verbatim}
    

These conditions, especially the third, may look sort of weird if you
are not used to them, but the examples make it all clear.  The easiest
example of a Lie algebra is gl(n,C), which just means all n\times n complex
matrices with the bracket defined as the "commutator":


\begin{verbatim}

 [x,y] = xy - yx.  
\end{verbatim}
    

Then straightforward calculations show a)-c) hold... so these conditions
are really encoding the essence of the commutator.  

Now recall that the trace of a matrix, the sum of its diagonal entries,
satisfies tr(xy) = tr(yx).  So the trace of any commutator is zero, and
if we let sl(n,C) denote the matrices with zero trace, we see that it's a
sub-Lie algebra of gl(n,C) - that is, if x and y are in sl(n) so is [x,y],
so we can think of sl(n,C) as a Lie algebra in its own right.  Going from
sl(n,C) to gl(n,C) is essentially a trick for booting out the identity
matrix, which commutes with everything else (hence has vanishing
commutators).  The identity matrix is the only one with this property,
so it's sort of weird, and it simplifies things to get rid of it here.  

The simplest of the sl(n,C)'s is the Lie algebra sl(2,C), affectionately
known simply as sl(2), which is a 3-dimensional Lie algebra with a
basis given by matrices people call E, F, and H for mysterious reasons:


\begin{verbatim}

E = 0 1        F = 0 0        H = 1  0 
    0 0            1 0            0 -1
\end{verbatim}
    

You will never be an expert on Lie algebras until you know by heart that


\begin{verbatim}

[H,E] = 2E,    [H,F] = -2F,     [E,F]  = H.
\end{verbatim}
    

Typically that's the sort of remark I make before screwing up by a
factor of two or something, so you'd better check!  This is a
cute little multiplication table... but very important, since sl(2) is
the primordial Lie algebra from which the whole theory of "simple" Lie
algebras unfolds.   

Physicists are probably more familiar with a different basis of sl(2),
the Pauli matrices:


$$

\sigma _{1} = 0 1       \sigma _{2} = 0 -i       \sigma _{3} = 1  0 
_{&nbsp;}    1 0    _{&nbsp;}       i  0  _{&nbsp;}         0 -1
$$
    

For purposes of Lie algebra theory it's actually better to divide each of
these matrices by i and call the resulting matrices I, J, and K,
respectively.  We then have


$$

IJ = -JI = K,	JK = -KJ = I,    KI = -IK = J,   I^{2} = J^{2} = K^{2} = -1
$$
    

which is just the multiplication table of the quaternions!  From the
point of view of Lie algebras, though, all that matters is


\begin{verbatim}

[I,J] = 2K,    [J,K] = 2I,    [K,I] = 2J.
\end{verbatim}
    

Given the relation of these things and cross products, it should be no
surprise that the Pauli matrices have a lot to with angular momentum
around the x, y, and z axes in quantum mechanics.   

If we take all \emph{real} linear combinations of E,F,H we get a Lie algebra
over the \emph{real} numbers called sl(2,R), and if we take all real linear
combinations of I,J,K we get a Lie algebra over the reals called
su(2).  These two Lie algebras are two different "real forms" of sl(2).  

Now, people know just about everything about sl(2) that they might want
to.  Well, there's always something more, but I'm certainly personally
satisfied!  I recall when as an impressionable student I saw a book by
Serge Lang titled simply "SL(2,R)," big and fat and scary inside.  I
knew what SL(2,R) was, but not how one could think of a whole book's
worth of things to write about it!  A whole book on 2\times 2 matrices??  

Part of how one gets so much to say about a puny little Lie algebra
like sl(2) is by talking about its representations.  What's a
representation?  Well, first you have to temporarily shelve the idea
that sl(2) consists of 2\times 2 matrices, and think of it more
abstractly simply as a 3-dimensional vector space with basis E,F,H,
equipped with a Lie algebra structure given by the multiplication
table [H,E] = 2E, [H,F] = -2F, [E,F] = H.  If this is how I'd
originally defined it, it would then be a little \emph{theorem} that
this Lie algebra has a "representation" as 2\times 2 matrices.  And it
would turn out to have other representations too.  For example,
there's a representation as 3\times 3 matrices given by sending E to


\begin{verbatim}

0 1 0
0 0 2
0 0 0, 
\end{verbatim}
    

F to 


\begin{verbatim}

0 0 0 
2 0 0 
0 1 0, 
\end{verbatim}
    

and H to 


\begin{verbatim}

2 0  0 
0 0  0 
0 0 -2
\end{verbatim}
    

In other words, these matrices satisfy the same commutation relations as
E,F, and H do.  

More generally, and more precisely, we say an n-dimensional
representation of a Lie algebra L (over the complex numbers) is a linear
function R from L to n\times n matrices such that


\begin{verbatim}

R([x,y]) = [R(x),R(y)]
\end{verbatim}
    

for all x,y in L.  Note that on the left the brackets are the brackets
in L, while on the right they denote the commutator of n\times n matrices.  

One good way to understand the essence of a Lie algebra is to figure out
what representations it has.  And in quantum physics, Lie algebra
representations are where it's at: the symmetries of the world are
typically Lie groups, each Lie group has a corresponding Lie algebra,
the states of a quantum system are unit vectors in a Hilbert space, and
if the system has a certain Lie group of symmetries there will be a
representation of the Lie algebra on the Hilbert space.  As any particle
physicist can tell you, you can learn a lot just by knowing which
representation of your symmetry group a given particle has.

So the name of the game is classifying Lie algebra representations...
and many tomes have been written on this by now.  To keep things from
becoming too much of a mess it's crucial to make two observations.
First, there's an easy way to get new representations by taking the
"direct sum" of old ones: the sum of an n-dimensional representation and
an m-dimensional one is an (n+m)-dimensional one, for example.  Another
way, not so easy, to get new representations from an old one is to look for
"subrepresentations" of the given representation.  In particular, a
direct sum of two representations has them as subrepresentations.  
(I won't define "direct sum" and "subrepresentation" here... hopefully
those who don't know will be tempted to look it up.)

So rather than classifying \emph{all} representations, it's good to start by
classifying "irreducible" representations - those that have no
suprepresentations (other than themselves and the trivial 0-dimensional
representation).  This is sort of like finding prime numbers... they are
"building blocks" in representation theory.  But things are a little bit
messier, alas.  We say a representation is "completely reducible" if it
is a direct sum of irreducible representations.  Unfortunately, not all
representations need be completely reducible!  

Let's consider the representations of sl(2,C).  (The more sophisticated
reader should note that I am implicitly only considering finite-
dimensional complex representations!)   Here life is as nice as could
be: all representations are completely reducible, and
there is just one irreducible n-dimensional representation for each
n, with the 2-dimensional and 3-dimensional representations as above.  
(By the way, I really mean that there is only one irreducible
n-dimensional representation up to a certain equivalence relation!) 
Physicists - who more often work with the real form su(2) - call these
the spin-0, spin-1/2, spin-1, etc. representations.  The "spin" of a
particle is, in mathematical terms, just the thing that tells you which
representation of su(2) it corresponds to!

Now let me jump up several levels of sophistication.  In the last few
years people have realized that Lie groups are just a special case of
something called "quantum groups"... nobody talks about "quantum Lie
algebras" but that's essentially a historical accident: quantum groups
are NOT groups, they're a generalization of them, and they DON'T have Lie
algebras, but they have a generalization of them - so-called
quantized enveloping algebras.  

Quantum groups can be formed from simple Lie algebras, and they depend
on a parameter q, a nonzero complex parameter.  This parameter - q is
for quantum, naturally - can be thought of as

<div align = "center">
e^{\hbar }
</div>
The exponential of Planck's constant!  When we set \hbar  = 0 we get q
= 1, and we get back to the "classical case" of plain old-fashioned
Lie algebras and groups.  Every representation of a quantum group
gives an invariant of links (actually even tangles), and these link
invariants are functions of q.  If we take the nth derivative of one
of these invariants with respect to \hbar  and evaluate it at \hbar  =
0 we get a "Vassiliev invariant of degree n" (see "<A HREF =
"week3.html">week3</A>" for the definition).  Better than that, when q
is a root of unity each quantum group gives us a 3-dimensional
"topological quantum field theory," or TQFT known as Chern-Simons
theory.  In particular, we get an invariant of compact oriented
3-manifolds.  So there is a hefty bunch of mathematical payoffs from
quantum groups.  And there are good reasons to think of them as the
right generalization of groups for dealing with symmetries in the
physics of 2 and 3 dimensions.  If string theory \emph{or} the loop
variables approach to quantum gravity have any truth to them, quantum
groups play a sneaky role in honest 4-dimensional physics too.

In particular, there is a quantum version of sl(2) called
sl_{q}(2).  When q = 1 we essentially have the good old sl(2).
Chari and Premet have just worked out a lot of the representation
theory of sl_{q}(2).  First of all, it's been known for some
time that as long as q is not a root of unity - that is, as long as we
don't have 

q^{n} = 1 

for some integer n - the story is almost like that
for ordinary sl(2).  Namely, there is one irreducible representation
of each dimension, and all representations are completely reducible.
This fails at roots of unity - which turns out to be the reason why
one can cook up TQFTs in this case.  It turns out that if q is an nth
root of unity one can still define representations of dimension
0,1,2,3, etc., more or less just like the classical case, but only
those of dimension < n are irreducible.  There are, in fact,
exactly n irreducible representations, and the fact that there are
only finitely many is what makes all sorts of neat things happen.  The
k-dimensional representations with k \ge  n are not completely
reducible.  And, besides the representations that are analogous to the
classical case, there are a bunch more.  They have not been completely
classified - they are, according to Chari, a mess!  But she and Premet
have classified a large batch of the "indecomposable" ones, that is,
the ones that aren't direct sums of other ones.  I guess I'll leave it
at that.
 
2) David Kazhdan and Iakov Soibelman, Representations of the quantized
function algebras, 2-categories and Zamolodchikov tetrahedra
equations, Harvard University preprint.

In this terse paper, Kazhdan and Soibelman construct a braided
monoidal 2-category using quantum groups at roots of unity.  As I've
said a few times, people expect braided monoidal 2-categories to play
a role in generally covariant 4d physics analogous to what braided
monoidal categories do in 3d physics.  In particular, one might hope
to get invariants of 4-dimensional manifolds, or of surfaces embedded
in 4-manifolds, this way.  (See last week's post for a little bit
about the details.)  I don't feel I understand this construction well
enough yet to want to say much about it, but it is clearly related to
the construction of a braided monoidal 2-category from the category of
quantum group representations given by Crane and Frenkel (see "<A HREF
= "week2.html">week2</A>").

3) Adrian Ocneanu, A note on simplicial dimension shifting, 
available in AMSLaTeX as <A HREF =
"http://arxiv.org/abs/hep-th/9302028">arXiv:hep-th/9302028</A>.

Ouch!  This paper claims to show that the very charming 4d TQFT
constructed by Crane and Yetter in "A categorical construction of 4d
topological quantum field theories" (<A HREF =
"http://arxiv.org/abs/hep-th/9301062">arXiv:hep-th/9301062</A>) is
trivial!  In particular, he says the resulting invariant of compact
oriented 4-manifolds is identically equal to 1.  If so, it's back to
the drawing board.  Crane and Yetter took the 3d TQFT coming from
sl_{q}(2) at roots of unity and then used a clever trick to
get 3-manifolds from a simplicial decomposition of a 4-manifold to get
a 4d TQFT.  Ocneanu claims this trick, which he calls "simplicial
dimension shifting," only gives trivial 4-manifold invariants.

I am not yet in a position to pass judgement on this, since both
Crane/Yetter and Ocneanu are rather sketchy in key places.  If indeed
Ocneanu is right, I think people are going to have to get serious about
facing up to the need for 2-categorical thinking in 4-dimensional
generally covariant physics.  I had asked Crane, a big proponent of
2-categories, why they played no role in his 4d TQFT, and he said that
indeed he felt like the kid who took apart a watch, put it back
together, and found it still worked even though there was a piece left
over.  So maybe the watch didn't really work without that extra piece
after all.  In late March I will go to the Conference on Quantum Topology
thrown by Crane and Yetter (at Kansas State U. at Manhattan), and I'm
sure everyone will try to thrash this stuff out.  

4) Abhay Ashtekar and Jerzy Lewandowski, Representation theory of
analytic holonomy C*-algebras, available as <a href = "http://arxiv.org/abs/gr-qc/9311010">arXiv:gr-qc/9311010</a>.

This paper is a follow-up of the paper

5) Abhay Ashtekar and Chris Isham, Representations of the holonomy
algebras of gravity and non-Abelian gauge theories, Journal of
Classical and Quantum Gravity 9 (1992), 1069-1100.  Also available
as <a href =
"http://arxiv.org/abs/hep-th/9202053">arXiv:hep-th/9202053</a>.

and sort of complements another,

6) John Baez, Link invariants, holonomy algebras and functional
integration, available as <A HREF =
"http://arxiv.org/hep-th/9301063">arXiv:hep-th/9301063</A>.

The idea here is to provide a firm mathematical foundation for the
loop variables representation of gauge theories, particularly quantum
gravity.  Ashtekar and Lewandowski consider an algebra of
gauge-invariant observables on the space of su(2) connections on any
real-analytic manifold, namely that generated by piecewise analytic
Wilson loops.  This is the sort of thing meant by a "holonomy
algebra".  They manage to construct an explicit
diffeomorphism-invariant state on this algebra.  They also relate this
algebra to a similar algebra for sl(2) connections - the latter being
what really comes up in quantum gravity.  And they do a number of
other interesting things, all quite rigorously.  My paper dealt
instead with an algebra generated by "regularized" or "smeared" Wilson
loops, and showed that there was a 1-1 map from
diffeomorphism-invariant states on this algebra to invariants of
framed links - thus showing that the loop variables picture, in which
states are given by link invariants, doesn't really lose any of the
physics present in traditional approaches to gauge theories.  I am
busy at work trying to combine Ashtekar and Lewandowski's ideas with
my own and push this program further - my own personal goal being to
make the Chern-Simons path integral rigorous - it being one of those
mysterious "measures on the space of all connections mod gauge
transformations" that physicists like, which unfortunately aren't
really measures, but some kind of generalization thereof.  What it
\emph{should} be is a state (or continuous linear functional) on
some kind of holonomy algebra.  

\par\noindent\rule{\textwidth}{0.4pt}
 <!-- BEGIN FOOTER --> &#169;
1993 John Baez<br> baez@math.removethis.ucr.andthis.edu <br>
<TABLE WIDTH = 100%> <TR>
<TD WIDTH=10%>
<A HREF = "week4.html">
   <img border = none; src="lastweek.png"></A>
<TD WIDTH=80%>
<CENTER>
<A HREF="README.html">
  <img border = none; src="home.png"><br>
% </A>
<A HREF="http://math.ucr.edu/home/baez/TWF.html">
    <img border = none; src="contents.png">
% </A>
</CENTER>
<TD WIDTH=10%>
<A HREF = "week6.html">
  <img border = none; src="nextweek.png">
% </A>
</TABLE><!-- END FOOTER -->

