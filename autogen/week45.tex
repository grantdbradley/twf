
% </A>
% </A>
% </A>
\week{November 12, 1994}

This Week's Finds 45 - Donaldson Theory Update

In the previous edition of "This Week's Finds" I mentioned a burst of
recent work on Donaldson theory. I provocatively titled it "The End of
Donaldson Theory?", since the rumors I was hearing tended to be phrased
in such terms.  But I hope I made it clear at the conclusion of the
article that this recent work should lead to a lot of \emph{new} results in
4-dimensional topology!  An example is Kronheimer and Mrowka's proof of
the Thom conjecture.  

Many thanks to my network of spies for obtaining a preprint of the
following paper:


1) The genus of embedded surfaces in the projective plane, by 
P. B. Kronheimer and T. S. Mrowka, 10 pages.


Let me simply quote the beginning of the paper:

\par\noindent\rule{\textwidth}{0.4pt}
"The genus of a smooth algebraic curve of degree d in CP^2 is given by
the formula g = (d-1)(d-2)/2.  A conjecture sometimes attributed to Thom
states that the genus of the algebraic curve is a lower bound for the
genus of any smooth 2-manifold representing the same homology class. The
conjecture has previously been proved for d <= 4 and for d =6, and less
sharp lower bounds for the genus are known for all degrees [references
omitted].  In this note we confirm the conjecture.
 
Theorem 1.  Let S be an oriented 2-manifold smoothly embedded in CP^2 so
as to represent the same homology class as an algebraic curve of degree
d.  Then the genus g of S satisfies g >= (d-1)(d-2)/2.

Very recently, Seiberg and Witten [references below] introduced new
invariants of 4-manifolds, closely related to Donaldson's polynomial
invariants [reference omitted], but in many respects much simpler to
work with.  The new techniques have led to more elementary proofs of
many theorems in the area.  Given the monopole equation and the
vanishing theorem which holds when the scalar curvature is positive
(something which was pointed out by Witten), the rest of the argument
presented here is not hard to come by.  A slightly different proof of
the Theorem, based on the same techniques, has been found by Morgan,
Szabo and Taubes."
\par\noindent\rule{\textwidth}{0.4pt}

The reference to Donaldson's polynomial invariants appears in "<A HREF = "week44.html">week44</A>".
The references to the new Seiberg-Witten invariants are:


2) Monopoles and four-manifolds, by Edward Witten, in preparation.  
 
Electric-magnetic duality, monopole condensation, and confinement in N=2
supersymmetric Yang-Mills theory, by Edward Witten and Nathan Seiberg,
45 pages, available as <A HREF = "http://xxx.lanl.gov/abs/hep-th/9407087">hep-th/9407087</A>.

Monopoles, duality and chiral symmetry breaking in N=2 supersymmetric
QCD, by Edward Witten and Nathan Seiberg, 89 pages, available as
<A HREF = "http://xxx.lanl.gov/abs/hep-th/9408099">hep-th/9408099</A>.  


Differential geometers attempting to read the second two papers will
find that they contain no instance of the term "Donaldson theory", and
they may be frustrated to find that these are very much \emph{physics} papers.
They concern the ground states of supersymmetric Yang-Mills theory in 4
dimensions with gauge group SU(2).  The "ground states" of a field
theory are its least-energy states, which represent candidates for the
physical vacuum.  In certain theories there is not a unique ground
state, but instead a "moduli space" of ground states.  Seiberg and
Witten study these moduli spaces of ground states in both the classical
and quantum versions of SU(2) supersymmetric Yang-Mills theory in 4
dimensions.  They also consider the theory coupled to spinor fields,
which they call "quarks", using the analogy of the theory to quantum
chromodynamics, aka "QCD".

I haven't had time to go through their papers, since this isn't my main
focus of interest.  Perhaps the most useful thing I can do at this point
is to use Kronheimer and Mrowka's clear description of their moduli
space (which is presumably closely related to Seiberg and Witten's
moduli spaces) to simplify and fill in the holes of what I wrote in
"<A HREF = "week44.html">week44</A>".  I will aim my exposition to mathematicians, but make some
elementary digressions on physics to spice things up.

We start with a compact oriented Riemannian 4-manifold X, and assume we
are given Spin-c structure on X.  Recall the meaning of this.  First,
the orthonormal frame bundle of X has structure group SO(4), and a spin
structure would be a double cover of this which is a principal bundle
with structure group given by the double cover of SO(4), namely SU(2) x
SU(2).  Thus we get two principal bundles with structure group SU(2),
the left-handed and right-handed spin bundles.  Using the fundamental
representation of SU(2), we obtain two vector bundles called the bundles
of left-handed and right-handed spinors.  This "handedness" or
"chirality" phenomenon for spinors is of great importance in physics,
since neutrinos are left-handed spinors --- meaning, in down-to-earth
terms, that they always spin clockwise relative to their direction of
motion.  The fact that the laws of nature lack chiral symmetry came as
quite a shock when it was first discovered, and part of Seiberg and
Witten's motivation in their second paper is to study mechanisms for
"spontaneous breaking" of chiral symmetry.  This means simply that while
the theory has chiral symmetry, its ground states need not.

A Spin-c structure is a bit more subtle, but it allows us to define
bundles of left-handed and right-handed spinors as U(2) bundles,
which Kronheimer and Mrowka denote by W+ and W-.  The determinant
bundle L of W+ is a line bundle on X.  The first big ingredient of the
theory is a hermitian connection A on L.  In physics lingo this is the
vector potential of a U(1) gauge field.  This gives a Dirac operator D_A
mapping sections of W+ to sections of W-.  The connection A has
curvature F, and the self-dual part F+ of F can be identified with a
section of sl(W+).  (This is just a global version of the isomorphism
between the self-dual part of \Lambda ^2 C^4 and sl(2,C).)

The second big ingredient of the theory is a section \Psi  of W+, i.e. a 
left-handed spinor field.  There is a way to pair two sections of 
W+ to get a section of sl(W+), which we write as \sigma (.,.) and which is 
conjugate-linear in the first argument and linear in the second.
This is a global version of the similar pairing 


$$

                     \sigma (.,.): C^2 x C^2 \to  sl(2,C)
$$
    

where \sigma (v,w) given by taking the traceless part of the 2x2 matrix 
v* tensor w.  Here v* is the element of the dual of C^2 coming from 
v via the inner product on C^2.

To get the magical moduli space, we consider solutions (A,\psi ) of


$$

                       D_A \psi  = 0
                       F+ = i \sigma (\psi ,\psi ).
$$
    

Here we are thinking of F+ as a section of sl(W+).  These are pretty
reasonable equations for some sort of massless left-handed spinor field
coupled to a U(1) gauge field.  Let M be the space of solutions modulo
gauge transformations.  Kronheimer and Mrowka show the "moduli space" M
is compact.   

One can also perturb the equations above as follows.  If we have
any self-dual 2-form \delta  on X we can consider


$$

                       D_A \psi  = 0
                       F+ + i \delta  = i \sigma (\psi ,\psi ).
$$
    

and get a moduli space M(\delta ).  This will still be compact if \delta  is
nice (here I gloss over issues of analysis).  

Now, if X has an almost complex structure, Kronheimer and Mrowka show
that one can pick a Spin-c structure for X such that, for "good" metrics
and generic small \delta , M(\delta ) is a compact 0-dimensional manifold.
Using this fact and some geometrical yoga, it follows that the number n of
points in M(\delta ), counted mod 2, is independent of (such) \delta .
(This is essentially a glorified version of the fact that, when you look
at the multiple images of an object in a warped mirror and slowly bend
the mirror, the images generically appear or disappear in pairs.)
Moreover, if the self-dual Betti number b+ of X is > 1, the space of
good metrics is path-connected, and n mod 2 is independent of the choice
of good metric.  Kronheimer and Mrowka call this a "simple mod 2 version
of the invariants of Seiberg and Witten".  It is one ingredient of their
proof of the Thom conjecture.  
\par\noindent\rule{\textwidth}{0.4pt}

% </A>
% </A>
% </A>
