
% </A>
% </A>
% </A>
\week{November 14, 2005}


This week I'd like to talk about two aspects of higher gauge theory:
p-form electromagnetism and nonabelian cohomology.  Lurking behind both
of these is the mathematics of n-categories, but I'll do my best to hide
that until the end, to build up the suspense.  

But first, some cool pictures.  Astronomy is booming these days, and it's
a great way to see beautiful complexity emerging from simple laws in this 
wonderful universe of ours.  So, I'd like the freedom to occasionally start 
This Week's Finds with some pictures from the skies.  Think of it as an 
appetizer before the main course.  Sometimes I'll explicitly relate these 
pictures to math and physics; other times not.

Here's Saturn's moon Hyperion, photographed up close by the Cassini probe:

<BR>
<DIV ALIGN = CENTER>
<A HREF = "http://saturn.jpl.nasa.gov/multimedia/images/images.cfm?subCategoryID=29">
<IMG SRC = hyperion.jpg>
% </A>
</DIV>
<BR>

1) Cassini-Huyghens Mission, Hyperion: Odd World, 
<A HREF = "http://saturn.jpl.nasa.gov/multimedia/images/image-details.cfm?imageID=1762">
http://saturn.jpl.nasa.gov/multimedia/images/image-details.cfm?imageID=1762</A>

It seems to be a huge pile of rubble loosely held together by gravity and
heavily cratered by meteor bombardments.  

Hyperion is interesting because it's the only known moon that tumbles 
chaotically on a short time scale, thanks to its eccentric shape and 
gravitational interactions with Saturn and Titan.  

This leads to some interesting math.  We can think of Hyperion's angular 
momentum vector as a point on a sphere.  If we started out knowing
this point lay inside some small disk, time evolution would warp this disk 
into an ever more complicated region as time passed.  This region would 
always have the same area, thanks to the wonders of symplectic geometry.   
But it would sprout ever more complicated tendrils, with its perimeter 
growing by a factor of e about every 100 days or so!  

That's chaos for you.

Indeed, only quantum mechanics would stop the intricacy from growing forever,
by blurring it out.  After about 37 years, the area of a typical tendril 
would equal Planck's constant.  At this point, classical mechanics would
no longer be accurate.  You'd really need to describe Hyperion's spin 
state using quantum theory: for example, a holomorphic section of some 
line bundle on the sphere.

Well... at least you would if it weren't for decoherence caused by the 
interaction of Hyperion with its environment, for example solar radiation!
For an explanation of how this changes the story, try:

2) Michael Berry, Chaos and the semiclassical limit of quantum mechanics
(is the moon there when somebody looks?), in Quantum Mechanics: Scientific
Perspectives on Divine Action, CTNS Publications, Vatican Observatory, 2001.
Also available at 
<A HREF = "http://www.phy.bris.ac.uk/people/berry_mv/the_papers/berry337.pdf">http://www.phy.bris.ac.uk/people/berry_mv/the_papers/berry337.pdf</A>

Here's another great picture:

<BR>
<DIV ALIGN = CENTER>
<A HREF = "http://heritage.stsci.edu/2004/27/index.html">
<IMG SRC = cats-eye_nebula.jpg>
% </A>
</DIV>
<BR>
3) The Hubble Heritage Project, Cat's Eye Nebula - NGC 6543, 
<A HREF = "http://heritage.stsci.edu/2004/27/index.html">http://heritage.stsci.edu/2004/27/index.html</A>

This is a star about the size of the Sun, nearing the end of its life, 
emitting pulses of gas and dust.  Astronomers call such a thing a 
"planetary nebula", though it has nothing to do with planets.  It's in our 
galaxy, about 3000 light years from us.  When it's done shedding its outer
layers, all that's left of this star will be a white dwarf.

Our own Sun will become a planetary nebula in about 6.9 billion years,
after two separate stages of being a red giant - one as it runs out
hydrogen, and one as it runs out of helium.  When the helium is all gone,
the Sun will start to pulsate every 100,000 years, ejecting more and more 
mass in each pulse, eventually throwing off all but a hot inner core made
of heavier elements.  The astronomer Bruce Balick has written eloquently 
on what this will mean for the Earth:

\begin{quote}
    Here on Earth, we'll feel the wind of the ejected gases sweeping 
    past, slowly at first (a mere 5 miles per second!), and then 
    picking up speed as the spasms continue - eventually to reach 
    1000 miles per second!!  The remnant Sun will rise as a dot of 
    intense light, no larger than Venus, more brilliant than 100 
    present Suns, and an intensely hot blue-white color hotter than 
    any welder's torch.  Light from the fiendish blue "pinprick" 
    will braise the Earth and tear apart its surface molecules and 
    atoms.  A new but very thin "atmosphere" of free electrons 
    will form as the Earth's surface turns to dust.
\end{quote}

So, don't keep procrastinating - enjoy life now!

For other pictures of planetary nebulae, try Balick's webpage:

4) Bruce Balick, Hubble Space Telescope images of planetary nebulae,
<A HREF = "http://www.astro.washington.edu/balick/WFPC2/index.html">
http://www.astro.washington.edu/balick/WFPC2/index.html</A>

For a timeline of the universe, including the future life of our
Sun, try:

5) John Baez, A brief history of the universe,
<A HREF = "http://math.ucr.edu/home/baez/timeline.html">
http://math.ucr.edu/home/baez/timeline.html</A>

Now... on to p-form electromagnetism!

In ordinary electromagnetism, the secret star of the show turns out
to be not the electromagnetic field but the "vector potential", A. 
At least locally, we can think of this as a 1-form.  A 1-form is just
a gadget that you can integrate along a path and get a number.  In the
case of the vector potential, this number describes the change in 
phase that a charged particle acquires as it moves along this path.  

The 1-form A gives rise to a 2-form F called the 
"electromagnetic field".
A 2-form is a gadget you can integrate over a surface and get a 
number.  Here's how we get F from A.  Suppose we move a charged particle 
around a loop that's the boundary of some surface.   Then the integral 
of F over this surface is defined to be the integral of A around the loop!  
We summarize this by saying that F is the "exterior derivative" of A,
and writing

F = dA.

F is called the electromagnetic field because... that's what it is!
It contains both the electric and magnetic fields in a single neat
package.  In 4d spacetime, the magnetic field describes the change in 
a phase of a charged particle that loops around a surface in the 
xy, yz or zx planes.  The electric field describes the change in phase
of a charged particle that loops around a surface in the xt, yt or zt 
planes.  

If you don't know this stuff, you're missing some of the best fun
life has to offer.  For an easy introduction with lots of gorgeous 
pictures, see:

6) Derek Wise, Electricity, magnetism and hypercubes, available at
<A HREF = "http://math.ucr.edu/~derek/talks/050916bw.pdf">
http://math.ucr.edu/~derek/talks/050916bw.pdf</A>

The idea of p-form electromagnetism is to replace point particles
by strings or higher-dimensional membranes.  To see how this goes,
it's enough to look at 2-form electromagnetism.

In 2-form electromagnetism, the star of the show is a 2-form, A. 
As already mentioned, a 2-form is a gadget you can integrate over a 
surface and get a number.  In 2-form electromagnetism, this number 
describes the change in phase that a charged string acquires as it 
moves along, tracing out a surface in spacetime.  

The 2-form A gives rise to a 3-form, F.  A 3-form is a gadget 
you can integrate over a 3-dimensional region and get a number.  
Suppose we move a charged string and let it trace out a surface 
that's the boundary of some 3-dimensional region.   Then the integral 
of F over this region is defined to be the integral of A over the 
surface!  Again we write this as: 

F = dA.

So, we're just adding one to the dimensions of things.  This makes it 
easy to keep on going.  In fact, for any integer p, we can write down 
a generalization of Maxwell's equations.

It goes like this.  We start with a p-form A.  We define a (p+1)-form

F = dA

This automatically implies some of Maxwell's equations:

dF = 0

but the nontrivial Maxwell equations say that

*d*F = J

where * is the Hodge star operator and J is a p-form called the "current",
which is produced by charged matter.  

What does this mean, physically?  The idea is that we have charged matter 
consisting of (p-1)-dimensional membranes.  These trace out p-dimensional 
surfaces in spacetime as time passes.  The current J is a p-form that's 
concentrated on these surfaces.  The current affects the A field in a 
manner governed by Maxwell's equations.  Conversely, the A field affects 
the motion of the membranes.  Classically, we just integrate the A field 
over the surface traced out by a membrane and add the result to the 
\emph{action} for the membrane.  In the path integral approach to quantum 
mechanics, this number gives a change in phase, as already mentioned.

Maxwell's equations and their p-form generalization make sense when 
spacetime is any Lorentzian manifold.  However, to get a theory 
where initial data determine a unique global solution, we want our 
spacetime to be "globally hyperbolic", which means that it has a 
"Cauchy surface": roughly, a spacelike surface that any sufficiently
long timelike curve hits precisely once.  To get a good \emph{quantum} 
theory of p-form electromagnetism with a Hilbert space of states on which 
time evolution acts as unitary operators, we need more: our spacetime 
should be "stationary", meaning that it has time 
translation symmetry.  
Otherwise there's no way to define energy and the vacuum state - which 
is defined to be the least-energy state.

My student Miguel Carrion-Alvarez tackled an important special case 
in his thesis, namely "static" globally hyperbolic spacetimes: 
 
7) Miguel Carrion-Alvarez, Loop quantization versus Fock quantization 
of p-form electromagnetism on static spacetimes, available as 
<A HREF = "http://arxiv.org/abs/math-ph/0412032">math-ph/0412032</A>.   

There's a lot of interesting analysis involved, especially when space
(the Cauchy surface) is noncompact.  When it's compact, we can use 
"Hodge's theorem" to relate its deRham cohomology to its topology, 
and this turns out to be crucial for understanding p-form electromagnetism - 
especially issues like the p-form Bohm-Aharonov effect.  When it's 
noncompact we need something called "twisted L^{2} 
cohomology" instead, 
and Miguel proved a generalization of Hodge's theorem for this.  

With the analysis under control, Miguel was able to set up a very 
beautiful approach to  "loop quantum electromagnetism" and its p-form 
generalization.  Here the idea is to write Maxwell's equations in terms 
of the integrals of A around all possible loops in space - or more 
generally, over all p-dimensional surfaces.  People interested in loop
quantum gravity should like this.

As you can guess, either from seeing all the "d" operators or seeing all
the buzzwords I'm throwing around, p-form electromagnetism is really just
cohomology incarnated as physics!  My student Derek Wise made this very
precise for a version of the theory where spacetime is \emph{discrete} -
so-called "lattice p-form electromagnetism":

8) Derek Wise, Lattice p-form electromagnetism and chain field
theory, available as <A HREF = "http://xxx.lanl.gov/abs/gr-qc/0510033">gr-qc/0510033</A>.  Version with better graphics
and related material at 
<A HREF = "http://math.ucr.edu/~derek/pform/index.html">
http://math.ucr.edu/~derek/pform/index.html</A>

In this paper, he shows lattice p-form electromagnetism is a "chain
field theory": something like a topological quantum field theory, but 
where what matters is not spacetime itself so much as the cochain 
complex of differential forms \emph{on} spacetime, equipped with just enough
extra geometrical structure to write down the p-form version of Maxwell's
equations.

Both Miguel's thesis and Derek's papers are great if you want to learn 
lots of math and physics.  I seem to attract students who enjoy explaining 
things.  

Speaking of which....

Next I want to explain some stuff Danny Stevenson told me at a mall in the 
little town of Cabazon while we were recovering from a hike in the desert 
followed by pancakes at the Wheel Inn - a roadside restaurant famous for 
its <A HREF = "http://www.bigwaste.com/photos/ca/cabazon_dinos/">enormous 
statues of dinosaurs</A>.

Danny works on gerbes, stacks, and higher gauge theory.  Last year we
wrote a paper with Alissa Crans and Urs Schreiber constructing 2-groups
(categorified groups) from the math of string theory - more precisely, 
from central extensions of loop groups.  Since then I've been spending a 
lot of time writing a paper with Urs on higher gauge theory, where we set 
up a theory of parallel transport along surfaces.  2-form electromagnetism
is the simplest case of this theory.  Meanwhile, Danny has been thinking 
about connections on 2-vector bundles and their relation to the cohomology 
of Lie 2-algebras. 

This has led him to generalize Schreier theory in some interesting ways.  
So, let me tell you about Schreier theory! 

Schreier theory is a way to classify short exact sequences of groups.
I'll say what I mean by that in a minute... but what makes Schreier theory
special is that avoids some simplifying assumptions you might have seen 
if you've studied short exact sequences before.

Normally people water down their short exact sequences by assuming some 
of the groups in question are \emph{abelian}.  
This lets them use "cohomology 
theory" to do the classification.
See "<A HREF = "week210.html">week210</A>" 
for a nice book that takes this approach.

This standard approach is great - I'm not knocking it - but Schreier theory 
is more general: it's really a branch of "nonabelian cohomology theory".   
It's not all that hard to explain, either.  So, I'll explain it and then 
talk about various simplifying assumptions people make.

The goal of Schreier theory is to classify short exact sequences of groups:

1 \to  F \to  E \to  B \to  1

for a given choice of F and B.  "Exact" means that the arrows stand 
for homomorphisms and the image of each arrow is the kernel of the next.  
Here this just means that F is a normal subgroup of E and B is the quotient 
group E/F.  Such a short exact sequence is also called an "extension of B 
by F", since E is bigger than B and contains F.  The simplest choice is to 
let E be the direct sum of F and B.  Usually there are other more interesting
extensions as well.

To classify these, the trick is to use the analogy between group theory 
and topology. 
 
As I explained in "<A HREF = "week213.html">week213</A>", you can think of a group as a watered-down 
version of a connected space with a chosen point.  The reason is that
given such a space, we get a group consisting of homotopy classes of 
loops based at the chosen point.  This is called the "fundamental group"
of our space.  There's a lot more information in our space than this group.  
But pretty much anything you can do for groups, you can do for such spaces. 
It's usually harder, but it's completely analogous!

In particular, classifying short exact sequences is a lot like 
classifying "fibrations":

1 \to  F \to  E \to  B \to  1

where now the letters stand for connected spaces with a chosen point, and 
the arrows stand for continuous maps.   If you're a physicist or geometer 
you may prefer fiber bundles to "fibrations" - but luckily, they're so
similar we can ignore the difference in a vague discussion like this.
The idea is basically just that E maps onto B, and sitting over each point 
of B we have a copy of F.  We call B the "base space", E the "total space" 
and F the "fiber".

If we want to classify such fibrations we can consider carrying the fiber
F around a loop in B and see how it twists around.  For example, if all our
spaces are smooth manifolds, we can pick a connection on the total space
E and see what parallel transport around a loop in the base space B does 
to points in the fiber F.  This gives a kind of homomorphism

\Omega B \to  Aut(F)

sending loops in B to invertible maps from F to itself.  And, the cool
thing is: this homomorphism lets us classify the fibration!

Here I say "kind of homomorphism" since 
\Omega B, the space of loops in B 
based at the chosen point, is only "kind of" a topological group: the 
group laws only hold up to homotopy.  But let's not worry about this 
technicality  - especially since I'm being vague about all sorts of other 
equally important issues!

The reason I can get away with not worrying about these issues is that 
I'm trying to explain a very robust powerful principle - one that can 
easily survive a dose of vagueness that would kill a lesser idea.  Namely, 
if B is a connected space with a chosen basepoint,
<DIV ALIGN = CENTER>
               FIBRATIONS OVER THE BASE SPACE B WITH FIBER F  <br>
                          ARE "THE SAME" AS <br>
         HOMOMORPHISMS SENDING LOOPS IN B TO AUTOMORPHISMS OF F.
</DIV>
This could be called "the basic principle of Galois theory", for reasons
explained in "<A HREF = "week213.html">week213</A>".  There I explained the special case where the 
fiber is discrete.  Then our fibration called a "covering space", and 
the basic principle of Galois theory boils down to this:

<DIV ALIGN = CENTER>
                    COVERING SPACES OVER B WITH FIBER F <br>
                           ARE "THE SAME" AS <br>
    HOMOMORPHISMS FROM THE FUNDAMENTAL GROUP OF B TO AUTOMORPHISMS OF F.
</DIV>
Okay.  Now let's use the same principle to classify extensions of a group 
B by a group F:

1 \to  F \to  E \to  B \to  1

The group B here acts like "loops in the base".  But what acts like 
"automorphisms of the fiber"?  

You might guess it's the group of automorphisms of F.  But, it's 
actually the \emph{2-group} of automorphisms of F!

A 2-group is a categorified version of a group where all the usual group
laws hold up to natural isomorphism.  They play a role in higher gauge 
theory like that of groups in ordinary gauge theory.  In higher gauge 
theory, parallel transport along a path is described by an \emph{object} in 
a 2-group, while parallel transport along a path-of-paths is described 
by a \emph{morphism}.  
In 2-form electromagnetism we use a very simple "abelian" 
2-group, which has one object and either the real line or the circle as
morphism.  But there are other more interesting "nonabelian" examples.

If you want to learn more about 2-form electromagnetism from this 
perspective, try "<A HREF = "week210.html">week210</A>".  For 2-groups in general, try this paper:

9) John Baez and Aaron Lauda, Higher-dimensional algebra V: 2-groups, 
Theory and Applications of Categories 12 (2004), 423-491. Available online at 
<A HREF = 
"http://www.tac.mta.ca/tac/volumes/12/14/12-14abs.html">
http://www.tac.mta.ca/tac/volumes/12/14/12-14abs.html</A>
or as <A HREF = "http://arxiv.org/abs/math.QA/0307200">math.QA/0307200</A>.

Anyway: it turns out that any group F gives a 2-group AUT(F) where the
objects are automorphisms of F and the morphisms are "conjugations" - 
elements of F acting to conjugate one automorphism and yield another.  
And, extensions

1 \to  F \to  E \to  B \to  1

are classified by homomorphisms

B \to  AUT(F)

where we think of B as a 2-group with only identity morphisms.  More
precisely:

<DIV ALIGN = CENTER>
                EXTENSIONS OF THE GROUP B BY THE GROUP F <br>
                            ARE "THE SAME" AS <br>
               HOMOMORPHISMS FROM B TO THE 2-GROUP AUT(F)
</DIV>
It's fun to work out the details, but it's probably not a good use of 
our time together grinding through them here.   So, I'll just sketch
how it works.

Starting with our extension

&nbsp; &nbsp; &nbsp; &nbsp; &nbsp; &nbsp; &nbsp; 
 i    
&nbsp; &nbsp; &nbsp; 
 p  <br>
1 \to  F \to  E \to  B \to  1
we pick a "section"

&nbsp; &nbsp; &nbsp; &nbsp; &nbsp; &nbsp; &nbsp; 
&nbsp; &nbsp; &nbsp; &nbsp
                s  <br>
&nbsp; &nbsp; &nbsp; &nbsp; &nbsp; &nbsp; &nbsp; &nbsp; 
            E &larr; B

meaning a function with 

p(s(b)) = b

for all b in B.  We can find a section because p is onto.  However,
the section usually \emph{isn't} a homomorphism.

Given the section s, we get a function 

&alpha;: B \to  Aut(F)

where Aut(F) is the group of automorphisms of F.  Here's how:

&alpha;(b) f = s(b) f s(b)^{-1}

However, usually &alpha; \emph{isn't} a homomorphism.  

So far this seems a bit sad: functions between groups want
to be homomorphisms.
But, we can measure how much s fails to be a homomorphism 
using the function

g: B^{2} \to  F

defined by
g(b,b') = s(bb') s(b')^{-1} s(b)^{-1}
Note that g = 1 iff s is a homomorphism.  

We can then use this
function g to save &alpha;.  The sad thing about &alpha; is that
it's not a homomorphism... but the good thing is, it's a homomorphism
\emph{up to conjugation by g!}  In other words:

&alpha;(bb') f = g(b,b') [ &alpha;(b) &alpha;(b') f ] g(b,b')^{-1}

Taken together, &alpha; and g satisfy some equations ("cocycle
conditions") which say precisely that they form
a homomorphism from B to the 2-group AUT(F).  Conversely, 
any such homomorphism gives an extension of B by F.  

In fact, isomorphism classes of extensions of B by F correspond
in a 1-1 way with isomorphism classes of homorphisms from B to AUT(F).
So, we've classified these extensions!
 
In fact, something even better is true!  
It's evil to "decategorify" by 
taking isomorphism classes as we did in the previous paragraph.  To avoid 
this, we can form a groupoid whose objects are extensions of B by F, and a 
groupoid whose objects are homomorphisms B \to  AUT(F).  I'm pretty sure
that if you form these groupoids in the obvious way, they're equivalent.  
And that's what this slogan really means:

<DIV ALIGN = CENTER>
                EXTENSIONS OF THE GROUP B BY THE GROUP F <br>
                            ARE "THE SAME" AS <br>
               HOMOMORPHISMS FROM B TO THE 2-GROUP AUT(F)
</DIV>

Next, let me say how Schreier theory reduces to more familiar ideas in two 
special cases.

People have thought a lot about the special case where F is abelian and 
lies in the center of E.  These are called "central extensions".  
This is just the case where &alpha; = 1.  The set 
of isomorphism classes of central extensions is called H^{2}(B,F) - 
the "second cohomology" of B with coefficients in F.

People have also thought about "abelian extensions".  
That's an even more special case where all three groups are abelian.  
The set of isomorphism classes of such extensions is called Ext(B,F).  

Since we don't make any simplifying assumptions like this in Schreier 
theory, it's part of a subject called "nonabelian cohomology".  
It was 
actually worked out by Dedecker in the 1960's, based on much earlier 
work by Schreier:

10) O. Schreier, Ueber die Erweiterung von Gruppen I, Monatschefte fur
Mathematik and Physick 34 (1926), 165-180.  Ueber die Erweiterung von 
Gruppen II, Abh. Math. Sem. Hamburg 4 (1926), 321-346.

11) P. Dedecker, Les foncteuers Ext_{\Pi }, 
H^{2}_{\Pi } and 
H^{2}_{\Pi } 
non abeliens,
C. R. Acad. Sci. Paris 258 (1964), 4891-4895.

More recently, Schreier theory was pushed one step up the categorical 
ladder by Larry Breen.  As far as I can tell, he essentially classified 
the extensions of a 2-group B by a 2-group F in terms of homomorphisms 
B \to  AUT(F), where AUT(F) is the \emph{3-group} of automorphisms of F:

12) Lawrence Breen, Theorie de Schreier superieure, Ann. Sci. Ecole Norm. 
Sup. 25 (1992), 465-514.  Also available at
<A HREF = "http://www.numdam.org/numdam-bin/feuilleter?id=ASENS_1992_4_25_5">
http://www.numdam.org/numdam-bin/feuilleter?id=ASENS_1992_4_25_5</A>

We can keep pushing Schreier theory upwards like this, but we can also 
expand it "sideways" by replacing groups with groupoids.  
You should have 
been annoyed by how I kept assuming my topological spaces were connected 
and equipped with a specified point.   I did this to make them analogous 
to groups.  For example, it's for only spaces like this that the fundamental 
group is powerful enough to classify covering spaces.  For more general
spaces, we must use the fundamental \emph{groupoid}. 
And, we can set up a Schreier theory for extensions of groupoids:

13) V. Blanco, M. Bullejos and E. Faro, Categorical non abelian cohomology,
and the Schreier theory of groupoids, available as 
<A HREF = "http://arxiv.org/abs/math.CT/0410202">math.CT/0410202</A>.

In fact, these authors note that Grothendieck did something similar
back in 1971: he classified \emph{all} groupoids fibered over a groupoid
B in terms of weak 2-functors from B to Gpd, which is the 2-groupoid of 
groupoids!  The point here is that Gpd contains AUT(F) for any fixed 
groupoid F:

14) Alexander Grothendieck, 
Rev&ecirc;tements &Eacute;tales et Groupe Fondamental (SGA1),
chapter VI: Cat&eacute;gories fibr&eacute;es et descente,
Lecture Notes in Mathematics 224, Springer, Berlin, 1971.
Also available as <A HREF = "http://arxiv.org/abs/math.AG/0206203">
math.AG/0206203</A>.

Having extended the idea "sideways" like this, one can then continue 
marching "upwards".  I don't know how much work has been done on this,
but the slogan should be something like this:

<DIV ALIGN = CENTER>
                  n-GROUPOIDS FIBERED OVER AN n-GROUPOID B <br>
                             ARE "THE SAME" AS <br>
            WEAK (n+1)-FUNCTORS FROM B TO THE (n+1)-GROUPOID nGpd
</DIV>

Grothendieck also studied this kind of thing with categories replacing
groupoids, so there should also be an n-category version, I think... 
but it's more delicate to define "fibrations" for categories than
for groupoids, so I'm a bit scared to state a slogan suitable for 
n-categories.

However, I'm not scared to go from n-groupoids to \omega -groupoids, which
are basically the same as spaces.  In terms of spaces, the slogan goes 
like this:

<DIV ALIGN = CENTER>
                        SPACES FIBERED OVER THE SPACE B <br>
                               ARE "THE SAME" AS <br>
                    MAPS FROM B TO THE SPACE OF ALL SPACES
</DIV>

This is how James Dolan taught it to me.  Most mortals are scared of "the 
space of all spaces" - both for fear of Russell's paradox, and because we 
really need a \emph{space} 
of all spaces, not just a mere set of them.  To avoid 
these terrors, you can water down Jim's slogan by choosing a specific space 
F to be the fiber:

<DIV ALIGN = CENTER>
                  FIBRATIONS WITH FIBER F OVER THE SPACE B <br>
                             ARE "THE SAME" AS <br>
                MAPS FROM B TO THE CLASSIFYING SPACE OF AUT(F)
</DIV>
where AUT(F) is the topological group of homotopy self-equivalences of F.
The fearsome "space of all spaces" is then the disjoint union of the 
classifying spaces of all these topological groups AUT(F).  It's too large
to be a space unless you pass to a larger universe of sets, but otherwise 
it's perfectly fine.  Grothendieck invented the notion of a "Grothendieck 
universe" for precisely this purpose:

15) Wikipedia, Grothendieck universe, 
<A HREF = "http://en.wikipedia.org/wiki/Grothendieck_universe">
http://en.wikipedia.org/wiki/Grothendieck_universe</A>

\par\noindent\rule{\textwidth}{0.4pt}
<B>Addendum:</B> 
I'd like to thank Leo Alonso for pointing out that Grothendieck's
famous "SGA1" is now available on the arXiv in TeX form,
courtesy of the Soci&eacute;t&eacute; Math&eacute;matique de France.  
SGA2 is also available on the arXiv, and more are coming.  

Here are further addenda thanks to Kevin Buzzard, Toby Bartels, 
David Corfield, Peter May, Jim Stasheff, and Ronnie Brown.

First, an email I sent in reply to Kevin Buzzard, who was curious about 
how we classify extensions of the group B by the group F using homomorphisms 
from B to the 2-group AUT(F).  In particular, he wanted to know the definition 
of "2-group" and "homomorphism between 2-groups", and how AUT(F) is defined:

In particular, he wanted to know 
the definition of "2-group" and "homomorphism between
2-groups", and how AUT(F) is defined:

\begin{quote}
Dear Kevin -

You write:


\begin{verbatim}

  I'm just checking some of the details (extensions of groups are morphisms 
  of 2-groups) and I find that you've not given me quite enough information 
  to do it (in the sense that I'm not knowledgeable enough about standard 
  facts about 2-groups to fill in some of the details that you allude to).
\end{verbatim}
    
 
Sorry.   I'm glad you care enough to want to know this stuff:
2-groups and homomorphisms between them are defined in loving detail 
in 
<A HREF = "http://arxiv.org/abs/math.QA/0307200">my 
paper with Aaron Lauda</A>, but I'll answer your questions here 
and append this to "<A HREF = "week223.html">week223</A>" 
to help out anyone else who cares.


$$

  Say we're in the "classical" case where F is an abelian group, B is a
  group, and we're classifying extensions 1\to F\to E\to B\to 1 where F lies in the
  centre of E. We know the answer here: such E's are classified by H^{2}(B,F)
  which, for me, means 2-cocycles over 2-coboundaries. Recall that
  a 2-cocycle is g:B^{2}\to F satisfying g(a,bc)+g(b,c)=g(a,b)+g(ab,c).
$$
    
 
Right.


$$

  You want the answer to be homomorphisms B \to  AUT(F). You don't \emph{quite} give
  the definitions of these things. 
$$
    

True.  Let me start by saying what a 2-group is, and then how the group 
B becomes a 2-group, and then how AUT(F) is defined.

A group is just a category with one object and with all morphisms invertible.
Slick!  But, we usually prefer to think of a group as a set: the morphisms 
of our category get called "elements" 
of this set.  This set then has a multiplication function


m: G \times  G \to  G


and an identity element

1 &isin; G

satisfying the associative and unit laws, and such that every element
has an inverse.  

All this categorifies painlessly:

A 2-group is just a 2-category with one object and with all morphisms and 
2-morphisms invertible.  Slick!  But, we sometimes prefer to think of a 
2-group as a category: the morphisms of our 2-category get called 
"objects" of this category, and the 2-morphisms get called 
"morphisms".  This category
then has a multiplication functor

m: G \times  G \to  G

and an identity object

1 &isin; G

satisfying the associative and unit laws, and such that every object
and morphism has an inverse.  

In short: a 2-group is just like a group, but with the word 
"category"
replacing the word "set", the word "object" 
replacing the word "element", and so on!

Now, how does a group become a 2-group?  Simple: we take the \emph{elements}
of our group and make them the \emph{objects} of our 2-group; then we say the
only morphisms of our 2-group are identity morphisms.  The 2-group 
multiplication m: G \times  G \to  G
comes from the multiplication in our group, and so on.


$$

  Let's stick to F abelian. You think of AUT(F) as being the 2-category with
  objects Aut(F), 1-morphisms Hom(\phi ,\psi )=F if \phi =\psi  and empty otherwise
  (because F is abelian). What are the 2-morphisms? Is Hom(\phi ,\psi ) supposed
  to be a category with objects F?
$$
    

Alas, you're one dimension down: thought of as a 2-category, we want our
2-group AUT(F) to have one object, the usual group Aut(F) as morphisms, 
and conjugations between these as 2-morphisms.  

Here's how we get this.  Think of our group F as a category.  Then 
let AUT(F) has F as its only object, invertible functors 

a: F \to  F

as its morphisms, and natural isomorphisms between these as its 2-morphisms.

That's very slick.  But let me say this in a different way using the
other viewpoint, where we think of a 2-group as a "category with 
multiplication and inverses".  Given a group F, our 2-group AUT(F) 
will be the category where the objects are automorphisms

a: F \to  F

and where a morphism f: a \to  a' is an element f of F that conjugates 
a to give a':

f a(g) f^{-1} = a'(g)   for all g &isin; F

This viewpoint requires some extra work to check that AUT(F) is a 
2-group.  The 2-category viewpoint is actually much more efficient.


$$

  Now what is a homomorphism B \to  AUT(F)?
$$
    

Here's where the subtlety comes in: weakening!  So far we haven't
weakened anything: all the equations in the definition of a group
became equations in the definition of a 2-group.  We're really just
dealing with "strict" 2-groups here.  
But we need to \emph{weaken} the
definition of homomorphism, replacing some equations by isomorphisms,
to get things to work out well now.

If we think of B and AUT(F) as 2-categories, a homomorphism B \to  AUT(F)
is just a weak 2-functor.  Slick!  But, you may not enjoy this definition 
as much as I do.

So, let's think of B and AUT(F) as "categories with multiplication and
inverses".  Then a homomorphism 

&alpha;: B \to  AUT(F) 

is a functor that preserves multiplication of objects <em>up to a 
specified isomorphism</em>, which must satisfy some laws of its own.  

Quite roughly, this means that given two objects b and b' of B,
we don't have an equation

&alpha;(bb') = &alpha;(b) &alpha;(b')

Instead, we have an isomorphism

g(b,b'): &alpha;(bb') \to  &alpha;(b) &alpha;(b')   

This needs to satisfy some equations.  I can tell you these if you
want, but for starters you can check that in our application, this 
g(b,b') thing will be the 2-cocycle familiar from group cohomology!

And, the laws g must satisfy say precisely that g is a 2-cocycle.

(Indeed, for a \emph{central} extension &alpha; = 1, so all we really need
to think about is this 2-cocycle g.  Schreier theory goes on to consider 
more general extensions, where &alpha; &ne; 1.)


\begin{verbatim}

  I'm sure I could just go and read Breen's book, but these questions
  are so trivial that I'm sure you can instantly answer them, and you
  also get the confirmation that there are still people out there
  reading TWF.
\end{verbatim}
    

That's worth a lot!  If you ever want to learn more about 2-groups 
and homomorphisms between them, I think 
<A HREF = "http://arxiv.org/abs/math.QA/0307200">my 
paper with Aaron Lauda</A>
could be easier than 
<A HREF = "http://www.numdam.org/numdam-bin/feuilleter?id=ASENS_1992_4_25_5">
Breen's opus</A>.  Breen's opus concerns "higher
Schreier" theory - classifying extensions of 2-groups with the help
of 3-groups!

Best,<br>
jb
\end{quote}

It's important to note that it's \emph{isomorphism classes} of extensions 
that correspond to \emph{isomorphism classes} of homomorphisms B \to  
Aut(F).  
For this, one needs to know what a "2-isomorphism" 
between homomorphisms 
of 2-groups is.  Again, this is explained in my paper with Lauda.  It's 
a special case of a weak natural isomorphism between weak 2-functors 
between 2-categories, but we say what this means in terms that working
mathematicians can understand. 

Also, Toby Bartels had some comments on the dinosaurs of Cabazon
and size issues in category theory:

\begin{quote}
John wrote in part:

\begin{verbatim}
  Next I want to explain some stuff Danny Stevenson told me at a mall in
  the little town of Cabazon while we were recovering from a hike in the
  desert followed by pancakes at the Wheel Inn - a roadside restaurant
  famous for its enormous statues of dinosaurs.
\end{verbatim}
    

Did you see <A HREF = "http://www.latimes.com/news/local/la-me-dinosaurs27aug27,0,3988775,full.story">the creationist sign by the dinosaurs?</A>

\begin{verbatim}
                    SPACES FIBERED OVER THE SPACE B
                           ARE "THE SAME" AS
                 MAPS FROM B TO THE SPACE OF ALL SPACES

  The fearsome "space of all spaces" is then the disjoint union of
  the classifying spaces of all these topological groups AUT(F). It's
  too large to be a space unless you pass to a larger universe of sets,
  but otherwise it's perfectly fine.
\end{verbatim}
    

So if you want your slogan to treat size issues carefully:

<div align = center>
SPACES FIBERED OVER THE SPACE B<br>
  ARE "THE SAME" AS<br>
MAPS FROM B TO THE SPACE OF ALL SMALL SPACES
</div>

But you were secretly doing this all along!
After all, when you wrote:

\begin{verbatim}
                 n-GROUPOIDS FIBERED OVER AN n-GROUPOID B
                          ARE "THE SAME" AS
            WEAK (n+1)-FUNCTORS FROM B TO THE (n+1)-GROUPOID nGpd
\end{verbatim}
    

you simply used "nGpd" to abbreviate "OF ALL 
SMALL n-GROUPOIDS".
So there are really no new size issues at the \omega  level.

(Exercise for those that like this sort of thing:
Do we need to state that B and F are small?)


-- Toby
\end{quote}


Thanks for clearing this up.  I prefer not to distract people with
size issues, so I didn't mention them until "the space of all
spaces" walked in the door, at which point I figured alarm bells 
would start ringing for lots of people.  But, it was already hiding
in the "the (n+1)-groupoid of all n-groupoids".  I prefer to use a
new Grothendieck universe for each level of the n-categorical 
hierarchy, to justify such expressions.  I guess \omega -categories
then require an \omega -th universe. 

Yes, I saw that silly sign in front of the dinosaurs, though I didn't 
understand its full meaning until now.  It wasn't there the last time 
I visited.  Apparently the new owners have decided to enlist this nice 
roadside attraction as part of the propaganda campaign for creationism. 
They actually believe Adam and Eve walked with dinosaurs in Eden - as 
one biologist put it, "they think The Flintstones is a documentary".

It's sad how just as the magnificent history of the universe is becoming
vividly clear, some want to truncate it to a pitifully human scale - 
and then claim \emph{that} was God's work.

Next, David Corfield had some questions about the "space of all spaces",
which I answered in this email:

\begin{quote}
 Dear David - 

 You wrote:


\begin{verbatim}

  Hi,
  
         SPACES FIBERED OVER THE SPACE B 
               ARE "THE SAME" AS 
    MAPS FROM B TO THE SPACE OF ALL SMALL SPACES 
  
  Is there another handle on this, other than
  
       OMEGA-GROUPOIDS FIBERED OVER THE OMEGA-GROUPOID B 
                    ARE "THE SAME" AS 
    WEAK OMEGA-FUNCTORS FROM B TO THE OMEGA-GROUPOID OF ALL
                  SMALL OMEGA-GROUPOIDS  ?
  
  Presumably B must be small, and the spaces fibered over it.
\end{verbatim}
    

 It suffices for the fibers to be small, so if you want a really
nitpicky motto:

<DIV ALIGN = CENTER>
    SMALL \omega -GROUPOIDS FIBERED OVER THE \omega -GROUPOID B <br>
                    ARE "THE SAME" AS <br>
    WEAK \omega -FUNCTORS FROM B TO THE \omega -GROUPOID OF ALL <br>
              SMALL \omega -GROUPOIDS <br> 
</DIV>


\begin{verbatim}

  Do you and Jim have other intuitions about THE SPACE OF ALL
  SMALL SPACES? 
\end{verbatim}
    
 One can describe it in a completely precise and rigorous way.
 It's the disjoint union over all homotopy types of small
 spaces F of the classifying spaces B(Aut(F)).  Here Aut(F) is the 
 topological group of homotopy self-equivalences of F.
 
 Note: the "largeness" of this space is solely due to it being a 
 disjoint union of a proper class of connected components.  When 
 we map any small space to it, the map can only hit a set's worth
 of components.  So, it's not really scary.

 And, if we map a connected small space X to it, we get a map 

 X \to  B(Aut(F)) 

 for some F, which is just what you need to get an F-bundle over X.


\begin{verbatim}

  Like, is it one of your FREE SUCH-AND-SUCHES?
\end{verbatim}
    
 I don't know a description like that offhand, since "free" suggests
 a left universal property, and the space of all (small) spaces mainly 
 has a right universal property, which describes maps \emph{into} it.

 Namely: maps from X into the space of all spaces are "the same as"
 fibrations over X.

I could make this completely precise, but it's probably not worth bothering
here; one just needs suitable equivalence relations.

\end{quote}
Next, Peter May wrote: 

\begin{quote}
 In his posting today, John Baez advertised the slogan:

<DIV ALIGN = CENTER>
              FIBRATIONS OVER THE BASE SPACE B WITH FIBER F <br>
                           ARE "THE SAME" AS <br>
          HOMOMORPHISMS SENDING LOOPS IN B TO AUTOMORPHISMS OF F.
</DIV>

 He hedged it with a "dose of vagueness", but in fact I proved a
 completely precise and general version of exactly this result in:

16) Peter May, Classifying spaces and fibrations, 
Memoirs AMS 155, Jan. 1975.  

 Using
 Moore loops on B, LB, one has a topological monoid, and one also has the
 topological monoid Aut(F) of homotopy equivalences of F.  
 A "transport"
 is a homomorphism of topological monoids from LB to Aut(F).  Allowing F to
 vary by a homotopy equivalence, one can define an equivalence relation on
 transports such that the equivalence classes are in natural bijective
 correspondence with the equivalence classes of "fibrations over the base
 space B with fiber F".  One can generalize the context by allowing fibers
 in some nice category and prove the same result.  See opus cit, Theorem
 14.2, page 83.  That was over 30 years ago, so naturally I wasn't thinking
 about categorification, but I would imagine that the methods categorify.
\end{quote}

Here and below I've taken the liberty of numbering the references to
papers, so it's easier to find them in my table of contents for This
Week's Finds.  

Jim Stasheff wrote:

\begin{quote}
 Even more ancient:

 
17) James Stasheff, 
 Parallel transport in fibre spaces, Bol. Soc. Mat. Mexicana (1968), 68-86.

 Unfortunately that's a hard paper to get a hold of.

 Somewhat related is:

18) James Stasheff, 
 Associated fibre spaces, Michigan Math. Journal 15 (1968), 457-470.

 and at the survey level:

19) James Stasheff, 
 H-spaces and classifying spaces, I-IV, AMS Proc. Symp. Pure Math.
 22 (1971), 247-272.

 Of course, as you might expect, I describe things in terms of
 A_{\infty }-morphisms from the space of loops into Aut(F) of homotopy 
 equivalences of F.

 Now that some of us are comfortable with A_{\infty }-cats,
 categorification should proceed, perhaps with some technical details.

 jim
\end{quote}

Ronnie Brown wrote:

\begin{quote}
 John Baez gave an interesting account of nonabelian cohomology and
 extension theory.  Here are a few more references with which I have been
 involved, all using crossed complexes and free crossed resolutions:

20) 
 Ronald Brown and P. J. Higgins, Crossed complexes and non-abelian
 extensions, Category theory proceedings, Gummersbach,
 1981, (ed. K.H. Kamps et al) Lecture Notes in Math. 962 
 Springer, Berlin, 1982, pp. 39-50.

 This generalises classical Schreier theory to extensions of groupoids.

21)
Ronald Brown and O. Mucuk, 
Covering groups of non-connected 
 topological groups revisited, Math. Proc. Camb. Phil. 
 Soc., 115 (1994) 97-110.  Also available as 
 <A HREF = "http://arxiv.org/abs/math.AT/0009021">math.AT/0009021</A>.

 This relates the theory of covering topological groups of non connected
 topological groups to the classical theory of extensions and
 obstructions to a Q-kernel with an invariant in H3.  It uses the
 properties of the internal hom in crossed complexes CRS(F,C), and 
 exact sequences derived from a fibration C \to  D and the induced 
 fibration on CRS(F, -).

22) 
Ronald Brown and Timothy Porter,
On the Schreier theory of non-abelian
 extensions: generalisations and computations, Proceedings
 Royal Irish Academy 96A (1996), 213-227.
Also available at
<A HREF = "http://www.informatics.bangor.ac.uk/public/math/research/ftp/algtop/rb/nonabex4.ps.gz">http://www.informatics.bangor.ac.uk/public/math/research/ftp/algtop/rb/nonabex4.ps.gz</A>
 This establishes a generalisation of the Schreier theory in two ways 
 (but only for groups). One is using coefficients in a crossed module,
 following Dedecker's key ideas, as in the references in John's account.
 Second it shows how to compute with such extensions

 A \to  E \to  G

 in terms of presentations of the group G.  This involves identities
 among relations for the presentation, as shown originally by Turing in

23) 
 Alan Turing, The extensions of a group, Compositio Mathematica 5 (1938), 
 357-367.

 The advantage of this method is that one can actually do sums, even when
 the group G may be infinite.  The example given by us is G = trefoil group
 with two generators x,y and relation x^{3}=y^{2}. 
 This presentation has no
 identities among relations, and so the calculation is especially simple.
 Equivalence of extensions is described in terms of homotopies of
 morphisms of crossed complexes, and this relates the ideas to other
 forms of homological or homotopical algebra.

 An advantage of this approach is the ability to calculate some small
 free crossed resolutions of some groups: this is one reason for using
 crossed complexes.  Note that a convenient monoidal closed structure on
 the category of crossed complexes has been explicitly written down, and
 this allows convenient calculation and representations of homotopies,
 using the `unit interval' groupid, as a crossed complex.

 One of the problems I have with the globular approach is the difficulty
 of writing down homotopies, and higher homotopies.  For example, Ilhan
 Icen and I found it difficult to rewrite in terms of group-groupoids the
 well known Whitehead theory of automorphisms of crossed modules,
 explained for the crossed modules of groupoids case in:

24) 
Ronald Brown and Ilhan Icen, 
Homotopies and automorphisms of crossed modules
 of groupoids, Applied Categorical Structures, 11 (2003) 185-206.
 Also available as <A HREF = "http://arxiv.org/abs/math.CT/0008117">math.CT/0008117</A>.

 It looks as if it would be better expressed in terms of automorphisms of
 2-groupoids: good marks to anyone who writes it down in that way!

 One knows such homotopies of globular \infty  groupoids exist because
 globular \infty -groupoids are equivalent to crossed complexes:

25) 
Ronald Brown and P. J. Higgins, 
The equivalence of \infty -groupoids and
 crossed complexes, Cah. Top. Geom. Diff. 22 (1981) 371-386.

 (This paper contains an early definition of a (strict, globular)
 \infty  category.)

 This raises a question: what is the crossed complex associated to the
 free globular groupoid on one generator of dimension n?  I have a
 round-about sketch proof, using also cubical theory, and a van Kampen
 theorem, that it \emph{is} the fundamental crossed complex of the n-globe.
 Does anyone have a purely algebraic proof?
 
 The idea of singular nonabelian cohomology of a space X with
 coefficients in a crossed complex C is given in:

26) Ronald Brown and 
P. J. Higgins, The classifying space of a crossed complex,
 Math. Proc. Camb. Phil. Soc. 110 (1991) 95-120.

 This cohomology is given by [\Pi SX, C], homotopy classes of maps from
 the fundamental crossed complex of the singular complex of X, to C.
 There is also a Cech version (current work with Jim Glazebrook and Tim
 Porter).

 An interesting problem is to classify extensions of crossed complexes!

 There is an interesting account of extensions of principal bundles and
 transitive Lie groupoids by Androulidakis, developing work of Mackenzie,
 at:

 27) Iakovos Androulidakis, Classification of extensions of principal 
 bundles and transitive Lie groupoids with prescribed kernel and 
 cokernel, J. Math. Phys. 45 (2004), 3095-4012.  Also available as
 <A HREF = "http://arxiv.org/abs/math.DG/0402007"</A>math.DG/0402007</A>. 

 (not using crossed complexes).

 Ronnie Brown <br>
 <A HREF = "http://www.bangor.ac.uk/r.brown">http:www.bangor.ac.uk/r.brown</A> 
\end{quote}

Finally, here's my reply to a bemused comment by Jim Stasheff:

\begin{quote}
 Jim Stasheff wrote:
 

$$

   John and anyone else who cares to weigh in, here are some comments from 
   the purely topological or rather homotopy theory side:
 
   For both bundles and fibrations (e.g. over a paracompact base), your 
   last slogan is the oldest:
 
                  FIBRATIONS WITH FIBER F OVER THE SPACE B
                             ARE "THE SAME" AS
                 MAPS FROM B TO THE CLASSIFYING SPACE OF AUT(F)
 
   "the same as" referring to homotopy classes.
$$
    

 It's certainly old, but I mentioned another that may be older:

<DIV ALIGN = CENTER>
                    COVERING SPACES OF B WITH FIBER F <br>
                           ARE "THE SAME" AS <br>
    HOMOMORPHISMS FROM THE FUNDAMENTAL GROUP OF B TO AUTOMORPHISMS OF F
</DIV>

 although one usually sees this special case (which I didn't bother
 to mention):

<DIV ALIGN = CENTER>
                 CONNECTED COVERING SPACES OF B WITH FIBER F <br>
                            ARE "THE SAME" AS <br>
            TRANSITIVE ACTIONS OF THE FUNDAMENTAL GROUP OF B ON F
</DIV>

 which is usually disguised as follows:

<DIV ALIGN = CENTER>
                 CONNECTED COVERING SPACES OF B <br>
                           ARE "THE SAME" AS <br>
               SUBGROUPS OF THE FUNDAMENTAL GROUP OF B 
</DIV>

 Anyway, I wasn't trying to present things in historical order.  
 I was trying present them roughly in order of increasing
 "dimension", starting with extensions of groups, then going up to 
 2-groups, then expanding out to groupoids, then going up to n-groupoids,
 and finally \omega -groupoids... which are the same as homotopy types!

 And here, as usual, the n-category theorists meet up with the 
 topologists - and find that the topologists have already done everything 
 there is to do with \omega -groupoids ... but usually by thinking of 
 them of them as \emph{spaces}, rather than \omega -groupoids!  

 It's sort of like climbing a mountain, surmounting steep cliffs with
 the help of ropes and other equipment, and then finding a Holiday Inn 
 on top and realizing there was a 4-lane highway going up the other side.  
 
 So, as usual, the main point of calling homotopy types 
 "\omega -groupoids"
 instead of "spaces" 
 is not to reinvent topology, but to see how ideas
 from topology generalize to n-category theory.  Think of spaces as 
 \omega -groupoids, but use those as a springboard for \omega -categories -
 or at least n-categories, perhaps just for low values of n if one is 
 feeling tired.

 In the case at hand, the \omega -groupoidal slogan: 

<DIV ALIGN = CENTER>
               FIBRATIONS OF \omega -GROUPOIDS WITH FIBER F AND BASE B <br>
                               ARE "THE SAME" AS <br>
                       WEAK \omega -FUNCTORS FROM B TO AUT(F)
</DIV>
 is just a reformulation of:

<DIV ALIGN = CENTER>
                  FIBRATIONS WITH FIBER F OVER THE SPACE B <br>
                             ARE "THE SAME" AS <br>
                 MAPS FROM B TO THE CLASSIFYING SPACE OF AUT(F)
</DIV>
 but it suggests a grandiose generalization:

<DIV ALIGN = CENTER>
               FIBRATIONS OF \omega -CATEGORIES WITH BASE B<br>
                            ARE "THE SAME" AS <br>
   WEAK \omega -FUNCTORS FROM B^{op} TO THE \omega -CATEGORY OF \omega -CATEGORIES!
</DIV>

 I guess we can thank Grothendieck for making precise and proving a
 version of this with \omega  replaced by n = 1:

<DIV ALIGN = CENTER>
                      FIBRATIONS OF CATEGORIES WITH BASE B <br>
                               ARE "THE SAME" AS <br>
            WEAK 2-FUNCTORS FROM B^{op} TO THE 2-CATEGORY OF CATEGORIES.
</DIV>

 More recently people have been thinking about the n = 2 case, especially
 Claudio Hermida:

 28) Claudio Hermida, Descent on 2-fibrations and strongly 2-regular 
 2-categories, Applied Categorical Structures, 12 (2004), 427-459.
 Also available at <A HREF = "http://maggie.cs.queensu.ca/chermida/papers/2-descent.pdf">http://maggie.cs.queensu.ca/chermida/papers/2-descent.pdf</A>

 He states something that hints at this:

<DIV ALIGN = CENTER>
                      FIBRATIONS OF 2-CATEGORIES WITH BASE B <br>
                               ARE "THE SAME" AS <br>
         WEAK 3-FUNCTORS FROM B^{op} TO THE WEAK 3-CATEGORY OF 2-CATEGORIES.
</DIV>
 Here I'm using B^{op} to mean B with the directions of 
 both 1-morphisms and 2-morphisms reversed.  
 Hermida follows tradition and calls this B^{coop} - 
 "op" for reversing 
 1-morphisms and "co" for reversing 2-morphisms.  
 But, it looks like we'll 
 be needing to reverse all kinds of morphisms in n-category case, so we'll 
 need a short name for that.

 Best,<br>
 jb
\end{quote}

\par\noindent\rule{\textwidth}{0.4pt}
<em>"Hah, what a fantastic night," Gunn said.
"Arcturus is absolutely steady."  He leaned back, put
his elbows on the rail of the lift, and looked up at the sky.
His glasses glinted faintly in the starlight.  "Astronomy
is not terribly important," he said.  He fell silent for a 
moment.  "Although it is one of the more important things
we do as a species," he said.  He did not see any contradiction


% parser failed at source line 1565
