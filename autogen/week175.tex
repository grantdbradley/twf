
% </A>
% </A>
% </A>
\week{December 29, 2001 }


I spent this Christmas in Greenwich, England.  Over repeated visits 
to England I have discovered many fascinating things of which many
Americans are unaware.  For example: while in traffic one must drive on
the left side of the road, in escalators one must stand on the right. 
You flip switches down to turn on lights.   Camels and zebras have
escaped from the Royal Zoo and mated, and their hybrids roam the English
countryside.  On the roadside you will occasionally see signs for
"humped zebra crossings".  Also, the Royal Observatory in Greenwich
fires a powerful green laser each night to mark the Prime Meridian -
zero degrees longitude.

Four of the last five sentences are true.  In particular, you really
\emph{can} see a green laser beam shining due north from the Royal 
Observatory, across the Thames, past the Citigroup Building and out 
into the night.  And speaking of longitude, the day before Christmas  
I visited this observatory and had a wonderful time learning how John
Harrison solved the longitude problem.  

The longitude problem?  Ah, how soon we forget!  It's pretty easy to
tell your latitude by looking at the sun or the stars.  However, it's
pretty hard to tell your longitude, unless you have a clock that keeps
good time.  After all, if you know what time it is in a fixed place,
like Greenwich, you can figure out how far east or west you've gone by
comparing the time you see the sun rise to the time it would rise there.
Unfortunately, until the late 1700's, pendulum clocks didn't work well
at sea, due to the rocking waves.  This was a real problem!  Ships would
lose track of their longitude, go astray, and sometimes even run aground, 
killing hundreds of sailors.  

Since England was a big maritime power, in 1714 they set up the Board of
Longitude, which offered a prize of 20,000 pounds to anyone who could
solve this problem.  Newton and Halley favored a solution which involved
measuring the angle between the moon and nearby stars and then consulting 
a bunch of tables.  This was a complicated system that could only work
with the help of an accurate star atlas and a detailed understanding of
the motion of the moon.  Newton set to work on the necessary calculations.  
John Flamsteed was made the royal astronomer of England, and he set to work 
on the star atlas.  He moved into the Royal Observatory, and stayed up each 
night making observations with the help of his wife.

However, before this "lunar distance method" came online, the watchmaker 
John Harrison invented the first of a series of ingenious clocks that worked
well despite rocking waves and fluctuations of temperature.  All these
can still be seen at the Royal Observatory - they're very beautiful!
In the process, Harrison developed a whole bunch of cool technology like 
ball bearings and the bimetallic strip used in thermostats. 

Alas, the Board refused to pay up even when Harrison built a clock that
was accurate to within .06 seconds a day, which was certainly good
enough.  Finally King George III persuaded the board to give him the
prize - but by then he was an old man.  Luckily, I get the feeling
Harrison was really more interested in building clocks than winning the
prize money.  He loved his work... one of the keys to a happy life.

Here's a book that tells his story in more detail:

1) Dava Sobel, Longitude, Fourth Estate Ltd., London, 1996.

I found it in the gift shop of the Observatory.  It's a fun read, but
for the technical reader it's frustratingly vague on the technical
details of how Harrisons' clocks actually work.  

I also bought this book there:

2) E. G. Richards, Mapping Time: The Calendar and its History, Oxford
U. Press, Oxford, 1998.

Since it's almost New Year's Day, let me tell you a bit what I
learned about calendars!

Mathematical physics has deep roots in astronomy, which may have been
the first exact science.  Thanks to astrology, the ancient theocratic
states put a lot of resources into precisely tracking and predicting the
motion of the sun, moon and planets.  For example, by 700 BC the
Babylonians had measured the length of the year to be 365.24579 days,
with an error of only .00344 days.  Two hundred years later, they had
measured the length of the month to be 29.53014 days - an error of only
2.6 seconds.

If there were 360 days in a year, 30 days in a month, and 12 months in a
year, the ancients would have been happy, since they loved numbers with
lots of divisors.  But alas, there aren't!  These whole numbers come
tantalizingly close, but not close enough, so the need for accurate
calendars, balanced by the desire for simplicity, kept pushing the
development of mathematics and astronomy forward.  

There are also lots of complications I haven't mentioned.   I've been
talking about the "mean solar day", the "mean synodic month" and the
"tropical year", but in fact the length of the day and month vary
substantially due to the tilt of the earth's axis, the tilt of the
moon's orbit, and other effects - so actually there are several
different definitions of day, month and year.  This was enough to keep
the astronomer-priests in business for centuries.  For more on the
physics of it all, try:

3) John Baez, The wobbling of the earth and other curiosities,
<A HREF = "http://math.ucr.edu/home/baez/wobble.html">http://math.ucr.edu/home/baez/wobble.html</A>

Unfortunately, the Romans, whose calendar we inherit, were real
goofballs when it came to calendrics.  Their system was run by a body of
"pontifices" headed by the Pontifex Maximus.  In 450 BC these guys
adopted a calendar in which odd-numbered years had 12 months and 355
days, while even-numbered years had 13 months and alternated between 377
and 378 days.  The extra month, called Mercedonius, was stuck smack in
the middle of February.  Even worse, this system gave an average of 366
and 1/4 days per year - one too many - so it kept drifting out of kilter
with the seasons.  The pontifices were authorized to fix things on an ad
hoc basis as needed, but power corrupts, so they started taking bribes
to suddenly advance or postpone the start of the year.  

As a result, by the time Julius Caesar became dictator, the calendar was
three months in advance of the seasons!  After consulting with the
Alexandrian astronomer Sosigenes, he decided to institute reforms.  To
straighten things out, the year 46 BC was made 445 days long.   This 
was known as the Last Year of Confusion.  It featured an extra long
Mercedonius as well as two extra months after December, called Undecimber 
and Duodecimber.  

The new so-called "Julian calendar" featured 12 months and 365 days,
with an extra day in February every fourth year.  The months alternated
nicely between 31 and 30 days, except for February, which only had 30 on
leap years.  Unfortunately, Caesar was assassinated in 44 BC before this
system fully took hold.   The pontifices ineptly interpreted his orders
and stuck in an extra day every \emph{third} year.  This didn't get fixed
until 9 BC, Augustus stopped this practice and decreed that the next 3
leap years be skipped to make up for the extra ones the pontifices had
inserted.  

From then on, things went more smoothly, except for a lot of name-grabbing.  
When Julius Caesar was assassinated, the Senate took the month of Quintilis 
and renamed it "Iulius" in his honor, giving us July.  Augustus followed
suit, naming the month of Sextilis after himself - giving us August. 
More annoyingly, he stole the last day from February and stuck it on his
own month to make it 31 days long, and did some extra reshuffling so the
months next to his had only 30 - giving us our current messy setup.

The Senate offered to name a month after the next emperor, Tiberius, but
he modestly declined.  The next one, Caligula, was not so modest: he
renamed June after his father Germanicus.  Then Claudius renamed May
after himself, and Nero grabbed April.  Later, Domitian took October and
Antonius took September.  The vile Commodus tried to rename all twelve
months, but that didn't stick.  Then Tacitus snatched September away
from Antonius... but luckily, all these later developments have been
forgotten!

This is only a tiny fraction of the fascinating lore in Richards' book. 
Ever wonder why there are 7 days in a week?  That's pretty easy: they're
named after the 7 planets - in the old sense of "planets", meaning
heavenly bodies visible by eye that don't move with the stars.  But
here's a harder puzzle!  Why are the 7 planets are listed in this order?


\begin{verbatim}

Sun     (Sunday - Dies Solis)
Moon    (Monday - Dies Lunae)
Mars    (Tuesday - Dies Martis)
Mercury (Wednesday - Dies Mercurii)
Jupiter (Thursday - Dies Iovis)
Venus   (Friday - Dies Veneris)
Saturn  (Saturday - Dies Saturnis)
\end{verbatim}
    
There's actually a nice explanation.  However, I won't give it away here.
Can you guess it?

Since ancient science was closely tied to numerology, I can't resist
mentioning some fun facts relating the calendar and the deck of cards. 
As you probably know, playing cards come in 4 suits of 13 cards each,
for a total of 52.  52 is also the number of weeks in a year.  The 4
suites correspond to the 4 seasons, so there are 13 weeks in each
season, just as there are 13 cards in each suite.  

Even better, if we add up the face values of all the cards in the deck,
counting an ace as 1, a deuce as 2, and so on up to 13, we get

(1 + 2 + 3 + 4 + 5 + 6 + 7 + 8 + 9 + 10 + 11 + 12 + 13) x 4 = 364,

which is one less than the number of days in a year!  The remaining day
corresponds to the "joker", a card which does not belong to any suite. 

Many calendars contain "epagomenal days" not included in any month.  
For example, the Egyptians had 5 epagomenal days, leaving 360 which they
could split up neatly into 12 months.  In a system with one epagomenal
day - the "joker" - the remaining 364 days can be divided not only as

(30 + 30 + 31) x 4,

which allows for two 30-day calendar months and one 31-day calendar 
month per season, but also as 

13 x 28

which allows for 13 anomalistic months of 28 days each - where an
"anomalistic month" is the time it takes for the moon to come round to
its perigee, where it's as close to the earth as possible.  

Putting it all together, we see that the number 364 factors as 

13 x 4 x 7,

which corresponds to 13 months, each containing 4 weeks, each containing
7 days - or alternatively to 4 seasons, each containing 13 weeks, each
containing 7 days - or to 4 suites, each containing 13 cards, with an
average face value of 7.  

Cute, eh?  I'm not sure how much of this stuff is coincidence and how
much was planned out by the mysterious mystics who invented playing
cards.   Of course we can't take these whole numbers too seriously - 
for example, the anomalistic month is actually 27.55455 days long, not
28.  However, a 364-day year \emph{is} mentioned in the the Book of Enoch, 
a pseudepigrapical Hebrew text which was found, among other places, in the 
Dead Sea Scrolls.  In fact, a year of this length was used in Iceland as 
late as 1940.  The idea of having one epagomenal day and dividing each 
season into months with 30, 30 and 31 days has also been favored by many
advocates of calendar reform.  

Of course, numerology should always be left to competent mathematicians
who don't actually believe in it.

Here's another nice book:

4) Alain Connes, Andre Lichnerowicz and Marcel Paul Schutzenberger,
A Triangle of Thoughts, AMS, Providence, 2000.  

This consists of polished-up transcripts of dialogues (or should I 
say trialogues?) among these mathematicians.  I wish more good scientists 
would write this sort of thing; it's much less strenuous to learn stuff 
by listening to people talk than by reading textbooks!  It's true that
textbooks are necessary when you want to master the details, but for 
the all-important "big picture", conversations can be much better.  

This book focuses on mathematical logic and physics, with a strong touch 
of philosophy... but it wanders all over the map in a pleasant way - from 
Bernoulli numbers to game theory!  The conversation is dominated by 
Connes, whose name appears on the title in bigger letters than the other 
two authors, perhaps because they others are now dead.

There is only one mistake in this book that I would like to complain 
about.  Following Roger Penrose, Connes takes quasicrystals as evidence 
for some mysterious uncomputability in the laws of nature.  The idea 
is that since there's no algorithm for deciding when a patch of Penrose 
tiles can be extended to a tiling of the whole plane, nature must do 
something uncomputable to produce quasicrystals of this symmetry.  The 
flaw in this reasoning seems obvious: when nature gets stuck, it feels 
free to insert a \emph{defect} in the quasicrystal.  Quasicrystals do not 
need to be perfect to produce the characteristic diffraction patterns by 
which we recognize them.  

But that's a minor nitpick: the book is wonderful!  Read it!

In case you don't know: Alain Connes is a Fields medalist, who won the
prize mainly for two things: his work on Von Neumann algebras, and his
work on noncommutative geometry.  Now I'll talk a bit about von Neumann
algebras, since you'll need to understand a bit about them to follow the
rest of my description of the paper by Michael Mueger that I have
been slowly explaining throughout "<A HREF = "week173.html">week173</A>" and "<A HREF = "week174.html">week174</A>".  


So: what's a von Neumann algebra?  Before I get technical and you all
leave, I should just say that von Neumann designed these algebras to be
good "algebras of observables" in quantum theory.  The
simplest example consists of all n x n complex matrices: these become an
algebra if you add and multiply them the usual way.  So, the subject of
von Neumann algebras is really just a grand generalization of the theory
of matrix multiplication.

But enough beating around the bush!  For starters, a von Neumann algebra
is a *-algebra of bounded operators on some Hilbert space of countable
dimension - that is, a bunch of bounded operators closed under addition,
multiplication, scalar multiplication, and taking adjoints: that's the *
business.  However, to be a von Neumann algebra, our *-algebra needs one
extra property!  This extra property is cleverly chosen so that we can
apply functions to observables and get new observables, which is
something we do all the time in physics.  

More precisely, given any self-adjoint operator A in our von Neumann
algebra and any measurable function f: R \to  R, we want there to be a
self-adjoint operator f(A) that again lies in our von Neumann algebra.
To make sure this works, we need our von Neumann algebra to be "closed"
in a certain sense.   The nice thing is that we can state this closure
property either algebraically or topologically.  

In the algebraic approach, we define the "commutant" of a bunch of
operators to be the set of operators that commute with all of them.  
We then say a von Neumann algebra is a *-algebra of operators that's 
the commutant of its commutant.  

In the topological approach, we say a bunch of operators T_{i}
converges "weakly" to an operator T if their expectation
values converge to that of T in every state, that is,


$$

<\psi , T_{i} \psi >  \to   <\psi , T \psi >
$$
    
for all unit vectors \psi  in the Hilbert space.  We then say a von 
Neumann algebra is an *-algebra of operators that is closed in the 
weak topology.  

It's a nontrivial theorem that these two definitions agree!

While classifying all *-algebras of operators is an utterly hopeless
task, classifying von Neumann algebras is almost within reach - close
enough to be tantalizing, anyway.  Every von Neumann algebra can be
built from so-called "simple" ones as a direct sum, or more generally a
"direct integral", which is a kind of continuous version of a direct
sum.  As usual in algebra, the "simple" von Neumann algebras are defined
to be those without any nontrivial ideals.  This turns out to be
equivalent to saying that only scalar multiples of the identity commute
with everything in the von Neumann algebra.   

People call simple von Neumann algebras "factors" for short.  Anyway, 
the point is that we just need to classify the factors: the process 
of sticking these together to get the other von Neumann algebras is
not tricky.


The first step in classifying factors was done by von Neumann and
Murray, who divided them into types I, II, and III.  This classification
involves the concept of a "trace", which is a generalization
of the usual trace of a matrix.

Here's the definition of a trace on a von Neumann algebra.  First, we
say an element of a von Neumann algebra is "nonnegative" if
it's of the form xx* for some element x.  The nonnegative elements form
a "cone": they are closed under addition and under
multiplication by nonnegative scalars.  Let P be the cone of nonnegative
elements.  Then a "trace" is a function


$$

tr: P \to  [0, +\infty ]
$$
    
which is linear in the obvious sense and satisfies

tr(xy) = tr(yx)

whenever both xy and yx are nonnegative.

Note: we allow the trace to be infinite, since the interesting von
Neumann algebras are infinite-dimensional.   This is why we define 
the trace only on nonnegative elements; otherwise we get "\infty  minus
\infty " problems.  The same thing shows up in the measure theory,
where we start by integrating nonnegative functions, possibly getting
the answer +\infty , and worry later about other functions.

Indeed, a trace very much like an integral, so we're really studying a
noncommutative version of the theory of integration.  On the other hand,
in the matrix case, the trace of a projection operator is just the
dimension of the space it's the projection onto.  We can define a
"projection" in any von Neumann algebra to be an operator with
p* = p and p^{2} = p.  If we study the trace of such a thing,
we're studying a \emph{generalization of the concept of dimension}.  
It turns out this can be infinite, or even nonintegral!

We say a factor is type I if it admits a nonzero trace for
which the trace of a projection lies in the set {0,1,2,...,+\infty }.
We say it's type I_{n} if we can normalize the trace
so we get the values {0,1,...,n}.  Otherwise, we say it's type
I_{\infty }, and we can normalize the trace to get all the 
values {0,1,2,...,+\infty }.

It turn out that every type I_{n} factor is isomorphic to the
algebra of n x n matrices.  Also, every type I_{\infty } factor
is isomorphic to the algebra of all bounded operators on a Hilbert space
of countably infinite dimension.

Type I factors are the algebras of observables that we learn to love in
quantum mechanics.  So, the real achievement of von Neumann was to begin
exploring the other factors, which turned out to be important in quantum
field theory.

We say a factor is type II_{1} if it admits a trace
whose values on projections are all the numbers in the unit interval
[0,1].  We say it is type II_{\infty } if it admits
a trace whose value on projections is everything in [0,+\infty ].

Playing with type II factors amounts to letting dimension be a
continuous rather than discrete parameter!  


Weird as this seems, it's easy to construct a type II_{1}
factor.  Start with the algebra of 1 x 1 matrices, and stuff it into the
algebra of 2 x 2 matrices as follows:


\begin{verbatim}

      ( x  0 ) 
x |-> (      )
      ( 0  x )
\end{verbatim}
    
This doubles the trace, so define a new trace on the algebra of 2 x 2
matrices which is half the usual one.  Now keep doing this, doubling the
dimension each time, using the above formula to define a map from the
2^{n} x 2^{n} matrices into the 2^{n+1} x
2^{n+1} matrices, and normalizing the trace on each of these
matrix algebras so that all the maps are trace-preserving.  Then take
the \emph{union} of all these algebras... and finally, with a little work,
complete this and get a von Neumann algebra!

One can show this von Neumann algebra is a factor.  It's pretty
obvious that the trace of a projection can be any fraction in the
interval [0,1] whose denominator is a power of two.  But actually, 
\emph{any} number from 0 to 1 is the trace of some projection in this
algebra - so we've got our paws on a type II_{1} factor.

This isn't the only II_{1} factor, but it's the only one that
contains a sequence of finite-dimensional von Neumann algebras whose
union is dense in the weak topology.  A von Neumann algebra like that is
called "hyperfinite", so this guy is called "the
hyperfinite II_{1} factor".

It may sound like something out of bad science fiction, but the
hyperfinite II_{1} factor shows up all over the place in physics!

First of all, the algebra of 2^{n} x 2^{n} matrices is a
Clifford algebra, so the hyperfinite II_{1} factor is a kind of
infinite-dimensional Clifford algebra.  But the Clifford algebra of
2^{n} x 2^{n} matrices is secretly just another name for
the algebra generated by creation and annihilation operators on the
fermionic Fock space over C^{2n}.  Pondering this a bit, you can show
that the hyperfinite II_{1} factor is the smallest von Neumann
algebra containing the creation and annihilation operators on a
fermionic Fock space of countably infinite dimension.

In less technical lingo - I'm afraid I'm starting to assume you know
quantum field theory! - the hyperfinite II_{1} factor is the
right algebra of observables for a free quantum field theory with only
fermions.  For bosons, you want the type I_{\infty } factor.

There is more than one type II_{\infty } factor, but again
there is only one that is hyperfinite.  You can get this by tensoring
the type I_{\infty } factor and the hyperfinite II_{1}
factor.  Physically, this means that the hyperfinite
II_{\infty } factor is the right algebra of observables for a
free quantum field theory with both bosons and fermions.

The most mysterious factors are those of type III.  These can be simply
defined as "none of the above"!  Equivalently, they are factors for 
which any nonzero trace takes values in {0,\infty }.  In a type III
factor, all projections other than 0 have infinite trace.  In other
words, the trace is a useless concept for these guys.  

As far as I'm concerned, the easiest way to construct a type III factor
uses physics.  Now, I said that free quantum field theories had
different kinds of type I or type II factors as their algebras of
observables.  This is true if you consider the algebra of \emph{all}
observables. However, if you consider a free quantum field theory on
(say) Minkowski spacetime, and look only at the observables that you can
cook from the field operators on some bounded open set, you get a
subalgebra of observables which turns out to be a type III factor!  

In fact, this isn't just true for free field theories.  According to a
theorem of axiomatic quantum field theory, pretty much all the usual
field theories on Minkowski spacetime have type III factors as their
algebras of "local observables" - observables that can be measured in
a bounded open set.  

Okay, so much for the crash course on von Neumann algebras!  Next time
I'll hook this up to Mueger's work on 2-categories.

In the meantime, here are some references on von Neumann algebras in
case you want to dig deeper.  For the math, try these:

5) Masamichi Takesaki, Theory of Operator Algebras I, Springer, 
Berlin, 1979.

6) Richard V. Kadison and John Ringrose, Fundamentals of the
Theory of Operator Algebras, 4 volumes, Academic Press, New York,
1983-1992.

7) Shoichiro Sakai, C*-algebras and W*-algebras, Springer, Berlin,
1971.

A W*-algebra is basically just a von Neumann algebra, but defined
"intrinsically", in a way that doesn't refer to a particular
representation as operators on a Hilbert space. 

For applications to physics, try these:

8) Gerard G. Emch, Algebraic Methods in Statistical Mechanics and Quantum
Field Theory, Wiley-Interscience, New York, 1972.

9) Rudolf Haag, Local Quantum Physics: Fields, Particles, Algebras,
Springer, Berlin, 1992.

10) Ola Bratelli and Derek W. Robinson, Operator Algebras and Quantum
Statistical Mechanics, 2 volumes, Springer, Berlin, 1987-1997.
\par\noindent\rule{\textwidth}{0.4pt}

Postscript:
For more about the measurement of time, 
Theo Buehler recommends this lecture:
11) John B. Conway, <a href = "http://www.math.utk.edu/~conway/Time.html">http://www.math.utk.edu/~conway/Time.html</A>

For technical information on John Harrison's clocks, Nigel Seeley
recommends this book, which also has a bunch of nice pictures:
12) William J. H. Andrewes, editor, The Quest for Longitude: The Proceedings
of the Longitude Symposium, Harvard University, Cambridge, Massachusetts,
November 4-6, 1993.  Harvard University Collection of Historical Scientific
Instruments, Cambridge Massachusetts, 1996.
Nigel Seeley and Julian Gilbey also recommend the following book on
calendrics:
13) Edward M. Reingold, and Nachum Dershowitz, Calendrical Calculations:
The Millennium Edition, Oxford U. Press, Oxford, 1997.
268 pages.

Finally, here's a correction and the answer to the puzzle I gave
above:


\begin{verbatim}

Derek Wise wrote: 

>JB wrote:

>> ....[Augustus] stole the last day from February and stuck it on his
>> own month to make it 31 days long, and did some extra reshuffling so the
>> months next to his had only 30 - giving us our current messy setup.

>In the modern calendar, July has 31 days and is adjacent to August.
\end{verbatim}
    

Yeah - I only remembered that a few days ago, after writing that
issue of This Week's Finds.  As a kid I refused to remember how
many days were in each month, since it seemed hopelessly arbitrary
and ugly - an all-too-human invention, rather than something 
intrinsic to the universe.  Also, I was never fond of the mnemonic

\begin{verbatim}

      Thirty days hath September
      All the rest I don't remember ....
\end{verbatim}
    
mainly because so many months end in "-ember" that this mnemonic
would need a mnemonic of its own for me to recall it.  It was only
much later that I learned the "knuckles and spaces" method for
keeping track of this information.  For some reason I tried this
a few days ago, and then I said "Hey!  There's a month with 31 days 
next to August!  What gives?"  I meant to look up the facts in 
Richards' book \emph{Mapping Time}, but I forgot.  Thanks for reminding
me!

Anyway, here's the deal: the calendar reform of Julius Caesar
gave the months these numbers of days:


\begin{verbatim}

Januarius  31
Februarius 29/30
Martius    31
Aprilis    30
Maius      31
Iunius     30
Iulius     31
Sextilis   30
September  31
October    30
November   31
December   30
\end{verbatim}
    
A nice systematic alternation, though you might why \emph{February} gets
picked on; this is because the earlier Roman calendar had a short
February, and a month called Mercedonius stuck in the middle of
February now and then. 

Augustus screwed it up as follows:


\begin{verbatim}

Januarius  31
Februarius 29/30
Martius    31
Aprilis    30
Maius      31
Iunius     30
Iulius     31
Augustus   31
September  30
October    31
November   30
December   31
\end{verbatim}
    
In short: he took the month of Sextilis, renamed it after himself, 
gave it an extra day, and switched the alternating pattern of 30 and
31 after that month.  

By the way, Richard Bullock gave the "right" answer to my puzzle 
about why the 7 planets are listed in the order they are as names 
of days of the week.  By this I mean he gives the same answer that 
Richards does in \emph{Mapping Time}.  Astrologers like to list the 
planets in order of decreasing orbital period, counting the sun 
as having period 365 days, and the moon as period 29 days:


\begin{verbatim}

Saturn    (29 years)
Jupiter   (12 years)
Mars      (687 days)
Sun       (365 days)
Venus     (224 days)
Mercury   (88 days)
Moon      (29.5 days)
\end{verbatim}
    

For the purposes of astrology they wanted to assign a planet to
each hour of each day of the week.  They did this in a reasonable
way: they assigned Saturn to the first hour of the first day,
Jupiter to the second hour of the first day, and so on, cycling
through the list of planets over and over, until each of the 
7 x 24 = 168 hours was assigned a planet.   Each day was then
named after the first hour in that day.   Since 24 mod 7 equals
3, this amounts to taking the above list and reading every third
planet in it (mod 7), getting:

\begin{verbatim}
 
Saturn  (Saturday)
Sun     (Sunday)
Moon    (Monday)
Mars    (Tuesday)
Mercury (Wednesday)
Jupiter (Thursday)
Venus   (Friday)
\end{verbatim}
    
I don't think anyone is \emph{sure} that this is how the days got the
names they did; the earliest reference for this scheme is the 
Roman historian Dion Cassius (AD 150-235), who came long after the
days were named.  However, Dion says the scheme goes back to Egypt.
In the \emph{Moralia} of Plutarch (AD 46-120) there was an essay entitled
"Why are the days named after the planets reckoned in a different order
from the actual order?"  Unfortunately this essay has been lost and
only the title is known.  
To bring the subject back to physics: we should see all these attempts
to bring order to time as part of a gradual process of developing
ever more precise and logical coordinate systems for the spacetime
manifold we call our universe.  We may laugh at how the Roman pontifices
took bribes to start the year a day early; our descendants may laugh
at how we add or subtract leap seconds from Coordinated Universal 
Time (UTC) to keep it in step with the irregular rotation of that lumpy 
ball of rock we call Earth (or more precisely, the time system called
UT2, based on the Earth's rotation).  How precise will we get?  Will 
we someday be worrying about leap attoseconds?  Leap Planck times?






 \par\noindent\rule{\textwidth}{0.4pt}

% </A>
% </A>
% </A>
