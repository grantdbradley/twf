
% </A>
% </A>
% </A>
\week{November 26, 1996 }

Last week I talked about asymptotic freedom - how the "strong" force 
gets weak at high energies.   Basically, I was trying to describe an 
aspect of "renormalization" without getting too technical about it.  
By coincidence, I recently got my hands on a book I'd been meaning 
to read for quite a while:

1) Laurie M. Brown, ed., "Renormalization: From Lorentz
to Landau (and Beyond)", Springer-Verlag, New York, 1993.

It's a nice survey of how attitudes to renormalization have changed 
over the years.  It's probably the most fun to read if you know some 
quantum field theory, but it's not terribly technical, and it includes 
a "Tutorial on infinities in QED", by Robert Mills, that might serve 
as an introduction to renormalization for folks who've never studied 
it.

Okay, on to some new stuff....  

It's a bit funny how one of the most curious features of bosonic 
string theory in 26 dimensions was anticipated by the number theorist 
Edouard Lucas in 1875.  I assume this is the same Lucas who is famous 
for the Lucas numbers: 1,3,4,7,11,18,..., each one being the sum of 
the previous two, after starting off with 1 and 3.  They are not 
quite as wonderful as the Fibonacci numbers, but in a study of pine 
cones it was found that while \emph{most} cones have consecutive Fibonacci 
numbers of spirals going around clockwise and counterclockwise, a 
small minority of deviant cones use Lucas numbers instead.  

Anyway, Lucas must have liked playing around with numbers, because 
in one publication he challenged his readers to prove that: "A square 
pyramid of cannon balls contains a square number of cannon balls only
when it has 24 cannon balls along its base".  In other words, the 
only integer solution of


$$

           1^{2} + 2^{2} + ... + n^{2} = m^{2},
$$
    
is the solution n = 24, not counting silly solutions like n = 0 and
n = 1.

It seems that Lucas didn't have a proof of this; the first proof is
due to G. N. Watson in 1918, using elliptic functions.   Apparently 
an elementary proof appears in the following ridiculously overpriced book:

2) W. S. Anglin, "The Queen of Mathematics: An Introduction to
Number Theory", Kluwer, Dordrecht, 1995.

For more historical details, see the review in 

3) Jet Wimp, Eight recent mathematical books, Math. Intelligencer
18 (1996), 72-79.

Unfortunately, I haven't seen these proofs of Lucas' claim, so I don`t
know why it's true.  I do know a little about its relation to 
string theory, so I'll talk about that.  

There are two main flavors of string theory, "bosonic" and
"supersymmetric".  The first is, true to its name, just the quantized,
special-relativistic theory of little loops made of some abstract 
string stuff that has a certain tension - the "string tension".  
Classically this theory would make sense in any dimension, but 
quantum-mechanically, for reasons that I want to explain \emph{someday} 
but not now, this theory works best in 26 dimensions.   Different 
modes of vibration of the string correspond to different particles, 
but the theory is called "bosonic" because these particles are all 
bosons.  That's no good for a realistic theory of physics, because
the real world has lots of fermions, too.  (For a bit about 
bosons and fermions in particle physics, see "<A HREF = "week93.html">week93</A>".)  

For a more realistic theory people use "supersymmetric" string
theory.  The idea here is to let the string be a bit more abstract:
it vibrates in "superspace", which has in addition to the usual
coordinates some extra "fermionic" coordinates.  I don't want to
get too technical here, but the basic idea is that while the usual
coordinates commute as usual:


$$

x_{i} x_{j} = x_{j} x_{i}
$$
    
the fermionic coordinates "anticommute"


$$

y_{i} _{j} = - y_{j} y_{i}
$$
    
while the bosonic coordinates commute with fermionic ones:


$$

x_{i} y_{j} = y_{j} x_{i }
$$
    
If you've studied bosons and fermions this will be sort of
familiar; all the differences between them arise from the difference 
between commuting and anticommuting variables.  For a little glimpse
of this subject try:

4) John Baez, Spin and the harmonic oscillator, 
<A HREF = "http://math.ucr.edu/home/baez/harmonic.html">http://math.ucr.edu/home/baez/harmonic.html</A>

As it so happens, supersymmetric string theory - often abbreviated
to "superstring theory" - works best in 10 dimensions.  There are 
five main versions of superstring theory, which I described in 
"<A HREF = "week74.html">week74</A>".  The type I string theory involves open strings - little
segments rather than loops.  The type IIA and type IIB theories
involve closed strings, that is, loops.   But the most popular sort
of superstring theories are the "heterotic strings".  A nice 
introduction to these, written by one of their discoverers, is:

5) David J. Gross, The heterotic string, in "Workshop on Unified 
String Theories", eds. M. Green and D. Gross, World Scientific, 
Singapore, 1986, pp. 357-399.  

These theories involve closed strings, but the odd thing about 
them, which accounts for the name "heterotic", is that vibrations 
of the string going around one way are supersymmetric and act as 
if they were in 10 dimensions, while the vibrations going around
the other way are bosonic and act as if they were in 26 dimensions!

To get this string with a split personality to make sense, people 
cleverly think of the 26 dimensional spacetime for the bosonic part 
as a 10-dimensional spacetime times a little 16-dimensional curled-up 
space, or "compact manifold".  To get the theory to work, it seems that
this compact manifold needs to be flat, which means it has to be a 
torus - a 16-dimensional torus.  We can think of any such torus as
16-dimensional Euclidean space "modulo a lattice".  Remember, a lattice
in Euclidean space is something that looks sort of like this:


\begin{verbatim}

                  x
                      x
               x          x
                   x          x 
            x          x
                x          x
         x          x
             x          x
      x          x   
          x          x
              x 
                  x


\end{verbatim}
    
Mathematically, it's just a discrete subset L of R^{n} (n-dimensional
Euclidean space, with its usual coordinates) with the property that 
if x and y lie in L, so does jx + ky for all integers j and k.  When 
we form n-dimensional Euclidean space "modulo a lattice", we decree 
two points x and y to be the same if x - y is in L.   For example, 
all the points labelled x in the figure above count as the same 
when we "mod out by the lattice"... so in this case, we get a 
2-dimensional torus.  

For more on 2-dimensional tori and their relation to complex analysis,
you can read "<A HREF = "week13.html">week13</A>."  Here we are going to be macho and plunge right
into talking about lattices and tori in arbitrary dimensions.  

To get our 26-dimensional string theory to work out nicely when we
curl up 16-dimensional space to a 16-dimensional torus, it turns
out that we need the lattice L that we're modding out by to have some
nice properties.  First of all, it needs to be an "integral" lattice,
meaning that for any vectors x and y in L the dot product x.y must
be an integer.  This is no big deal - there are gadzillions of
integral lattices.  In fact, sometimes when people say "lattice" they 
really mean "integral lattice".  What's more of a big deal is that 
L must be "even", that is, for any x in L the inner product x.x is 
even.  This implies that L is integral, by the identity


\begin{verbatim}

(x + y).(x + y) = x.x + 2x.y + y.y
\end{verbatim}
    
But what's really a big deal is that L must also be "unimodular".
There are different ways to define this concept.  Perhaps the easiest 
to grok is that the volume of each lattice cell - e.g., each 
parallelogram in the picture above - is 1.  Another way to say it 
is this.  Take any basis of L, that is, a bunch of vectors in L
such that any vector in L can be uniquely expressed as an integer
linear combination of these vectors.  Then make a matrix with the 
components of these vectors as rows.  Then take its determinant.  
That should equal plus or minus 1.  Still another way to say it
is this.  We can define the "dual" of L, say L*, to be all the 
vectors x such that x.y is an integer for all y in L.  An integer
lattice is one that's contained in its dual, but L is unimodular if 
and only if L = L*.   So people also call unimodular lattices 
"self-dual".  It's a fun little exercise in linear algebra to show 
that all these definitions are equivalent.  

Why does L have to be an even unimodular lattice?  Well, one
can begin to understand this a litle by thinking about what a closed 
string vibrating in a torus is like.   If you've ever studied the 
quantum mechanics of a particle on a torus (e.g. a circle!) you may 
know that its momentum is quantized, and must be an element of L*.  So 
the momentum of the center of mass of the string lies in L*.  

On the other hand, the string can also wrap around the torus in 
various topologically different ways.  Since two points in Euclidean 
space correspond to the same point in the torus if they differ by a 
vector in L, if we imagine the string as living up in Euclidean space,
and trace our finger all around it, we don't necesarily come back to 
the same point in Euclidean space: the same point \emph{plus} any vector
in L will do.  So the way the string wraps around the torus is 
described by a vector in L.  If you've heard of the "winding number",
this is just a generalization of that.  

So both L and L* are really important here (which has to do with
the fashionable subject of "string duality"), and a bunch 
more work shows that they \emph{both} need to be even, which implies 
that L is even and unimodular.

Now something cool happens: even unimodular lattices are only 
possible in certain dimensions - namely, dimensions divisible by 8.  
So we luck out, since we're in dimension 16.  

In dimension 8 there is only \emph{one} even unimodular lattice (up to 
isometry), namely the wonderful lattice E8!   The easiest way to think 
about this lattice is as follows.  Say you are packing spheres in n 
dimensions in a checkerboard lattice - in other words, you color
the cubes of an n-dimensional checkerboard alternately red and black,
and you put spheres centered at the center of every red cube, using
the biggest spheres that will fit.  There are some little hole left 
over where you could put smaller spheres if you wanted.  And as you 
go up to higher dimensions, these little holes gets bigger!  By the 
time you get up to dimension 8, there's enough room to put another 
sphere OF THE SAME SIZE AS THE REST in each hole!   If you do that, 
you get the lattice E8.  (I explained this and a bunch of other 
lattices in "<A HREF = "week65.html">week65</A>", so more info take a look at that.)


In dimension 16 there are only \emph{two} even unimodular lattices.
One is E8 + E8.  A vector in this is just a pair of vectors in E8.  The
other is called D16+, which we get the same way as we got E8: we take a
checkerboard lattice in 16 dimensions and stick in extra spheres in all
the holes.  More mathematically, to get E8 or D16+, we take all vectors
in R^{8} or R^{16}, respectively, whose coordinates are
either \emph{all} integers or \emph{all} half-integers, for which
the coordinates add up to an even integer.  (A "half-integer"
is an integer plus 1/2.)

So E8 + E8 and D16+ give us the two kinds of heterotic string 
theory!  They are often called the E8 + E8 and SO(32) heterotic
theories.  

In "<A HREF = "week63.html">week63</A>" and "<A HREF = "week64.html">week64</A>" I explained a bit about lattices and Lie
groups, and if you know about that stuff, I can explain why the
second sort of string theory is called "SO(32)".  Any compact
Lie group has a maximal torus, which we can think of as some Euclidean
space modulo a lattice.  There's a group called E8, described in
"<A HREF = "week90.html">week90</A>", which gives us the E8 lattice this way, and the product
of two copies of this group gives us E8 + E8.  On the other hand, we
can also get a lattice this way from the group SO(32) of rotations in 
32 dimensions, and after a little finagling this
gives us the D16+ lattice (technically, we need to use the lattice
generated by the weights of the adjoint representation and one of the
spinor representations, according to Gross).  
In any event, it turns out that these
two versions of heterotic string theory act, at low energies, like
gauge field theories with gauge group E8 x E8 and SO(32), respectively!
People seem especially optimistic that the E8 x E8 theory is 
relevant to real-world particle physics; see for example:

6) Edward Witten, Unification in ten dimensions, in "Workshop on 
Unified String Theories", eds. M. Green and D. Gross, World Scientific,
Singapore, 1986, pp. 438-456.

Edward Witten, Topological tools in ten dimensional physics, with
an appendix by R. E. Stong, in "Workshop on Unified String Theories", 
eds. M. Green and D. Gross, World Scientific, Singapore, 1986, pp. 
400-437.

The first paper listed here is about particle physics; I mention 
the second here just because E8 fans should enjoy it - it discusses
the classification of bundles with E8 as gauge group.

Anyway, what does all this have to do with Lucas and his stack
of cannon balls?

Well, in dimension 24, there are \emph{24} even unimodular lattices,
which were classified by Niemeier.  A few of these are obvious, like 
E8 + E8 + E8 and E8 + D16+, but the coolest one is the "Leech 
lattice", which is the only one having no vectors of length 2.  
This is related to a whole WORLD of bizarre and perversely fascinating
mathematics, like the "Monster group", the largest sporadic
 finite simple 
group - and also to string theory.  I said a bit about this stuff
in "<A HREF = "week66.html">week66</A>", and I will say more in the future, but for now let
me just describe how to get the Leech lattice.

First of all, let's think about Lorentzian lattices, that is,
lattices in Minkowski spacetime instead of Euclidean space.  
The difference is just that now the dot product is defined by


$$

(x_{1},...,x_{n}) . (y_{1},...,y_{n}) = - x_{1} y_{1} + x_{2} y_{2} + ... + x_{n} y_{n}
$$
    
with the first coordinate representing time.  It turns out that
the only even unimodular Lorentzian lattices occur in dimensions
of the form 8k + 2.  There is only \emph{one} in each of those dimensions,
and it is very easy to describe: it consists of all vectors whose
coordinates are either all integers or all half-integers, and whose
coordinates add up to an even number.  

Note that the dimensions of this form: 2, 10, 18, 26, etc., are 
precisely the dimensions I said were specially important in "<A HREF = "week93.html">week93</A>" 
for some \emph{other} string-theoretic reason.  Is this a "coincidence"?
Well, all I can say is that I don't understand it.

Anyway, the 10-dimensional even unimodular Lorentzian lattice
is pretty neat and has attracted some attention in string theory:

7) Reinhold W. Gebert and Hermann Nicolai, E10 for beginners,
preprint available as <A HREF = "http://xxx.lanl.gov/abs/hep-th/9411188">hep-th/9411188</A> 

but the 26-dimensional one is even more neat.  In particular,
thanks to the cannonball trick of Lucas, the vector


\begin{verbatim}

               v = (70,0,1,2,3,4,...,24)
\end{verbatim}
    
is "lightlike".  In other words, 


\begin{verbatim}

                    v.v = 0 
\end{verbatim}
    
What this implies is that if we let T be the set of all integer
multiples of v, and let S be the set of all vectors x in our lattice
with x.v = 0, then T is contained in S, and S/T is a 24-dimensional
lattice - the Leech lattice!

Now \emph{that} has all sorts of ramifications that I'm just barely 
beginning to understand.   For one, it means that if we do bosonic 
string theory in 26 dimensions on R^{26} modulo the 26-dimensional even 
unimodular lattice, we get a theory having lots of symmetries related 
to those of the Leech lattice.  In some sense this is a "maximally 
symmetric" approach to 26-dimensional bosonic string theory:

8) Gregory Moore, Finite in all directions, preprint available
as <A HREF = "http://xxx.lanl.gov/abs/hep-th/9305139">hep-th/9305139</A>.
 
Indeed, the Monster group is lurking around as a symmetry group here!
For a physics-flavored introduction to that aspect, try:

9) Reinhold W. Gebert, Introduction to vertex algebras,
Borcherds algebras, and the Monster Lie algebra, preprint 
available as <A HREF = "http://xxx.lanl.gov/abs/hep-th/9308151">hep-th/9308151</A>

and for a detailed mathematical tour see:

10) Igor Frenkel, James Lepowsky, and Arne Meurman, "Vertex Operator 
Algebras and the Monster," Academic Press, 1988.  

Also try the very readable review articles by Richard Borcherds, who
came up with a lot of this business:

11) Richard Borcherds, Automorphic forms and Lie algebras.

Richard Borcherds, Sporadic groups and string theory.   

These and other papers available at 
<A HREF =
"http://www.pmms.cam.ac.uk/Staff/R.E.Borcherds.html">http://www.pmms.cam.ac.uk/Staff/R.E.Borcherds.html</A>;
click on the personal home page.

Well, there is a lot more to say, but I need to go home and pack
for my Thanksgiving travels.   Let me conclude by answering a natural
followup question: how many even unimodular lattices are there in 
32 dimensions?  Well, one can show that there are AT LEAST 80 MILLION!

Some of you may have wondered what's
happened to the "tale of n-categories".  I haven't forgotten that!
In fact, earlier this fall I finished writing a big fat paper on 
2-Hilbert spaces (which are to Hilbert spaces as categories are to
sets), and since then I have been struggling to finish another big
fat paper with James Dolan, on the general definition of "weak 
n-categories".  I want to talk about this sort of thing, and other
progress on n-categories and physics, but I've been so busy \emph{working}
on it that I haven't had time to \emph{chat} about it on This Week's Finds.
Maybe soon I'll find time.
\par\noindent\rule{\textwidth}{0.4pt}
\textbf{Addenda:} Robin Chapman pointed out that Anglin's proof also
appears in the American Mathematical Monthly, February 1990, pp. 120-124,
and that another elementary proof has subsequently appeared in the Journal 
of Number Theory.  David Morrison pointed out in email that since
the sum

$$

1^{2} + 2^{2} + ... + n^{2} = m^{2},
$$
    

is n(n+1)(2n+1)/6, this problem can be solved by finding all the
rational points (n,m) on the elliptic curve


$$

(1/3) n^{3} + (1/2) n^{2} + (1/6) n = m^{2}
$$
    

which is the sort of thing folks know how to do.
                                                    
Also, here's something 
Michael Thayer wrote on one of the newsgroups, and my reply:

\begin{verbatim}

>John Baez wrote:

>> In particular, thanks to the cannonball trick of Lucas, the vector
>>
>>                v = (70,0,1,2,3,4,...,24)
>>
>> is "lightlike".  In other words,
>>
>>                     v.v = 0

>I don't see what is so significant about the vector v.  For instance,
>the 10 dimensional vector (3,1,1,1,1,1,1,1,1,1) is also light like, and
>you make no big deal about that.  Is there some reason why the ascending
>values in v are important?
\end{verbatim}
    
Yikes!  Thanks for catching that massive hole in the exposition.
You're right that there's no shortage of lightlike vectors in
the even unimodular Lorentzian lattices of other dimensions 8n+2; there
are also lots of other lightlike vectors in the 26-dimensional one.
Any one of these gives us a lattice in 8n-dimensional Euclidean space.
In fact, we can get all 24 even unimodular lattices in 24-dimensional
Euclidean space by suitable choices of lightlike vector.   The lightlike
vector you wrote down happens to give us the E8 lattice in 8 dimensions.
So what's so special about I wrote, which gives the Leech lattice?  Of
course the Leech lattice is itself special, but what does this
have to do with the nicely ascending values of the components of v?
Alas, I don't know the real answer.  I'm not an expert on this
stuff; I'm just explaining it in order to try to learn it.  Let me
just say what I know, which all comes from Chap. 27 of Conway and
Sloane's book "Sphere Packings, Lattices, and Groups".
If we have a lattice, we say a vector r in it is a "root" if
the reflection through r is a symmetry of the lattice.   Corresponding
to each root is a hyperplane consisting of all vectors perpendicular
to that root.  These chop space into a bunch of "fundamental regions".
If we pick a fundamental region, the roots corresponding to the
hyperplanes that form the walls of this region are called "fundamental
roots".  The nice thing about the fundamental roots is that the
reflection through any root is a product of reflections through these
fundamental roots.
[For more stuff on reflection groups and lattices see "week62" and
the following weeks.]
In 1983 John Conway published a paper where he showed various
amazing things; this is now Chapter 27 of the above book.  First,
he shows that the fundamental roots of the even unimodular Lorentzian
lattices in dimensions 10, 18, and 26 are the vectors r with r.r = 2
and r.v = -1, where the "Weyl vector" v is
(28,0,1,2,3,4,5,6,7,8)
(46,0,1,2,3,......,16)
and
(70,0,1,2,3,......,70)
respectively.
They all have this nice ascending form but only in 26 dimensions
is the Weyl vector lightlike!
Howerver, Conway doesn't seem to explain \emph{why} the Weyl vectors have
this ascending form.  So I'm afraid I really don't understand how
all the pieces fit together.  All I can say is that for some reason
the Weyl vectors have this ascending form, and the fact that the Weyl
vector is also lightlike makes a lot of magic happen in 26 dimensions.
For example, it turns out that in 26 dimensions there are *infinitely
many* fundamental roots, unlike in the two lower dimensional cases.
Just to add mystery upon mystery, Conway notes that in higher dimensions
there is no vector v for which all the fundamental roots r have
r.v equal to some constant.  So the pattern above does not continue.
I find this stuff fascinating, but it would drive me nuts to try
to work on it.  It's as if God had a day off and was seeing how many
strange features he could build into mathematics without actually
making it inconsistent.

\par\noindent\rule{\textwidth}{0.4pt}
\textbf{Yet another addendum (August 2001):} now, with the rise of interest
in 11-dimensional physics, there is even a paper on E11:

12) P. West, E_{11} and M-theory, available as <A HREF = "http://xxx.lanl.gov/abs/hep-th/0104081">hep-th/0104081</A>.
\par\noindent\rule{\textwidth}{0.4pt}

% </A>
% </A>
% </A>
