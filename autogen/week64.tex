
% </A>
% </A>
% </A>
\week{September 23, 1995 }

I have been talking about different "ADE classifications".  This time
I'll start by continuing the theme of last Week, namely simple Lie
algebras, and then move on to discuss affine Lie algebras and quantum
groups.  These are algebraic structures that describe the symmetries
appearing in quantum field theory in 2 and 3 dimensions.  They are very
important in string theory and topological quantum field theory, and
it's largely physics that has gotten people interested in them.

Remember, we began by classifying finite reflection groups.  A finite
reflection group is simply a finite group of linear transformations of
R^{n}, every element of which is a product of reflections.  Finite
reflection groups are in 1-1 correspondence with the following "Coxeter
diagrams", together with disjoint unions of such diagrams:


\begin{verbatim}

A_{n}, which has n dots like this:

o---o---o---o

B_{n}, which has n dots, where n > 1:

          4
o---o---o---o

D_{n}, which has n dots, where n > 3:

              o
             /
o---o---o---o
             \
              o

E_{6}, E_{7}, and E_{8}:

      o               o                   o
      |               |                   |
o--o--o--o--o   o--o--o--o--o--o    o--o--o--o--o--o---o


F_{4}:                   G_{2}:               H_{3} and H_{4}:

      4                6                5                5
o---o---o---o        o---o            o---o---o        o---o---o---o


I_{m}, where m = 5 or m > 6:

  m
o---o
\end{verbatim}
    

Not all of these finite reflection groups satisfy the "crystallographic
condition", namely that they act as symmetries of some lattice.  The
ones that do are of types A,B,D,E,F, and G, and disjoint unions thereof
--- but I'm going to stop reminding you about disjoint unions all the
time!

Now, if we have a finite reflection group that's the symmetries of some
lattice, we can take the dimension of the lattice to be the number of
dots in the Coxeter diagram.  In fact, the dots correspond to a basis of
the lattice, and the lines between them (and their numberings) keep
track of the angles between the basis vectors.  These basis vectors are
called "roots".  To describe the lattice completely, in principle we
need to know the lengths of the roots as well as the angles between
them.  But it turns out that except for type B, there is just one choice
of lengths that works (up to overall scale).  For type B there are two
choices, which people call B_{n} and C_{n}, respectively.  People keep
track of the lengths with a "Dynkin diagram" like this:


\begin{verbatim}

B_{n} has n dots, where n > 1:       C_{n} has n dots, where n > 2:

          4                                  4
o---o---o->-o                      o---o---o-<-o
\end{verbatim}
    

The arrow points to the shorter root; for B_{n} all the roots
except the last one are sqrt(2) times as long as the last one, while
for C_{n} all the roots except the last one are 1 / \sqrt 2 as
long.  (In fact, the lattices corresponding to B_{n} and
C_{n} are "dual", in the hopefully obvious sense.)  The only
reason why we require n > 2 for C_{n} is that B_{2} is
basically the same as C_{2}!

Now last Week, I also sketched how the Lie algebras of the compact
simple Lie groups were \emph{also} classified by the same Dynkin diagrams
of types A, B, C, D, E, F, and G.  These were real Lie algebras; we
can also switch viewpoint and work with complex Lie algebras if we
like, in which case we simply say we're studying the complex simple
Lie algebras, as opposed to their "compact real forms".

Unfortunately, I didn't say much about what these Lie algebras actually
are!  Basically, they go like this:

A_{n} - The Lie algebra A_{n} is just sl_{n+1}(C),
the (n+1) \times  (n+1) complex matrices with vanishing trace, which form a
Lie algebra with the usual bracket [x,y] = xy - yx.  The compact real
form of sl_{n}(C) is su_{n}, and the corresponding
compact Lie group is SU(n), the n \times  n unitary matrices with
determinant 1.  The symmetry group of the electroweak force is U(1) x
SU(2), where U(1) is the 1 \times  1 unitary matrices.  The symmetry group
of the strong force is SU(3).  The study of A_{n} is thus a
big deal in particle physics.  People have also considered "grand
unified theories" with symmetry groups like SU(5).

B_{n} - The Lie algebra B_{n} is
so_{2n+1}(C), the (2n+1) \times  (2n+1) skew- symmetric complex
matrices with vanishing trace.  The compact real form of
so_{n}(C) is so_{n}, and the corresponding compact Lie
group is SO(n), the n \times  n real orthogonal matrices with determinant 1,
that is, the rotation group in Euclidean n-space.  For some basic cool
facts about SO(n), check out "<A HREF = "week61.html">week61</A>".

C_{n} - The Lie algebra C_{n} is sp_{n}(C),
the 2n \times  2n complex matrices of the form


\begin{verbatim}

     A  B
     C  D
\end{verbatim}
    

where B and C are symmetric, and D is minus the transpose of A.  The
compact real form of sp_{n}(C) is sp_{n}, and the
corresponding compact Lie group is called Sp(n).  This is the group of
n \times  n quaternionic matrices which preserve the usual inner product on
the space H^{n} of n-tuples of quaternions.

D_{n} - The Lie algebra D_{n} is so_{2n}(C),
the 2n \times  2n skew-symmetric complex matrices with vanishing trace.  See
B_{n} above for more about this.  It may seem weird that SO(n)
acts so differently depending on whether n is even or odd, but it's
true: for example, there are "left-handed" and "right-handed" spinors
in even dimensions, but not in odd dimensions.  Some clues as to why
are given in "<A HREF = "week61.html">week61</A>".

Those are the "classical" Lie algebras, and they are things that are
pretty easy to reinvent for yourself, and to get interested in for all
sorts of reasons.  As you can see, they are all about "rotations" in 
real, complex, and quaternionic vector spaces.  

The remaining ones are called "exceptional", and they are much more
mysterious.  They were only discovered when people figured out the
classification of simple Lie algebras.  As it turns out, they tend to be
related to the octonions!  Some other week I will say more about them,
but for now, let me just say:

F_{4} - This is a 52-dimensional Lie algebra.  Its smallest
representation is 26-dimensional, consisting of the traceless
3\times 3 hermitian matrices over the octonions, on which it preserves
a trilinear form.  

G_{2} - This is a 14-dimensional Lie algebra, and the
compact Lie group corresponding to its compact real form is also often
called G_{2}.  This group is just the group of symmetries
(automorphisms) of the octonions!  In fact, the smallest
representation of this Lie algebra is 7-dimensional, corresponding to
the purely imaginary octonions.

E_{6} - This is a 78-dimensional Lie algebra.  Its smallest
representation is 27-dimensional, consisting of all the 3x3 hermitian
matrices over the octonions this time, on which it preserves the
anticommutator.  

E_{7} - This is a 133-dimensional Lie algebra.  Its smallest 
representation is 56-dimensional, on which it preserves a tetralinear
form. 

E_{8} - This is a 248-dimensional Lie algebra, the biggest
of the exceptional Lie algebras.  Its smallest representation is
248-dimensional, the so-called "adjoint" representation, in which it
acts on itself.  Thus in some vague sense, the simplest way to
understand the Lie group corresponding to E_{8} is as the
symmetries of itself!  (Thanks go to Dick Gross for this charming
information; I think he said all the other exceptional Lie algebras
have representations smaller than themselves, but I forget the sizes.)
In "<A HREF = "week20.html">week20</A>" I described a way to get its
root lattice (the 8-dimensional lattice spanned by the roots) by
playing around with the icosahedron and the quaternions.

People have studied simple Lie algebras a lot this century, basically
studied the hell out of them, and in fact some people were getting a
teeny bit sick of it recently, when along came some new physics that put
a lot of new life into the subject.  A lot of this new physics is
related to string theory and quantum gravity.  Some of this physics is
"conformal field theory", the study of quantum fields in 2 dimensional
spacetime that are invariant under all conformal (angle-preserving)
transformations.  This is important in string theory because the string
worldsheet is 2-dimensional.  Some other hunks of this physics go by the
name of "topological quantum field theory", which is the study of
quantum fields, usually in 3 dimensions so far, that are invariant under
\emph{all} transformations (or more precisely, all diffeomorphisms).

Every simple Lie algebra gives rise to an "affine Lie algebra" and
a "quantum group".   The symmetries of conformal field theories are
closely related to affine Lie algebras, and the symmetries of
topological quantum field theories are quantum groups.  I won't
tell you what affine Lie algebras and quantum groups ARE, since it
would take quite a while.  Instead I'll refer you to a good
good introduction to this stuff:  

1) Juergen Fuchs, Affine Lie Algebras and Quantum Groups, Cambridge
Monographs on Mathematical Physics, Cambridge U. Press, Cambridge 1992.

Let me whiz through his table of contents and very roughly sketch what
it's all about.

<H5> 1.  Semisimple Lie algebras </H5>

This is a nice summary of the theory of semisimple Lie algebras
(remember, those are just direct sums of simple Lie algebras) and their
representations.  Especially if you are a physicist, a slick summary
like this might be a better way to start learning the subject than a big
fat textbook on the subject.  He covers the Dynkin diagram stuff and
much, much more.  

<H5> 2.  Affine Lie algebras </H5>

This starts by describing Kac-Moody algebras, which are certain
\emph{infinite-dimensional} analogs of the simple Lie algebras.  Fuchs
concentrates on a special class of these, the affine Lie algebras, and
describes the classification of these using Dynkin diagrams.  The main
nice thing about the affine Lie algebras is that their corresponding
infinite-dimensional Lie groups are very nice: they are almost "loop
groups".  If we have a Lie group G, the loop group LG is just the set of
all smooth functions from the circle to G, which we make into a group by
pointwise multiplication.  If you're a physicist, this is obviously
relevant to string theory, because at each time, a string is just a
circle (or bunch of circles), and if you are doing gauge theory on 
the string, with symmetry group G, the gauge group is then just the loop
group LG.  So you'd expect the representation theory of loop groups
and their Lie algebras to be really important.  

You'd \emph{almost} be right, but there is a slight catch.  In quantum
theory, what counts are the "projective" representations of a group,
that is, representations that satisfy the rule g(h(v)) = (gh)(v) 
\emph{up to a phase}.  (This is because "phases are unobservable in quantum
theory" - one of those mottoes that needs to be carefully interpreted
to be correct.)  The projective representations of the loop group LG
correspond to the honest-to-goodness representations of a "central
extensions" of LG, a slightly fancier group than LG itself.  And the
Lie algebra of \emph{this} group is called an affine Lie algebra.

So, people who like gauge theory and string theory need to know a
lot about affine Lie algebras and their representations, and that's
what this chapter covers.  A real heavy-duty string theorist will
need to know more about Kac-Moody algebras, so if you're planning
on becoming one of those, you'd better also try:

2) Victor Kac, Infinite Dimensional Lie Algebras, 3rd ed.,
Cambridge University Press, Cambridge, 1990.

You'll also need to know more about loop groups, so try:

3) Loop groups, by Andrew Pressley and Graeme Segal, Oxford University
Press, Oxford, 1986. 

<H5> 3.  WZW theories </H5>

Well, I just said that physicists liked affine Lie algebras because they
were the symmetries of conformal field theories that were also gauge
theories.  Guess what: a Wess-Zumino-Witten, or WZW, theory, is a
conformal field theory that's also a gauge theory!  You can think of it
as the natural generalization of the wave equation in 2 dimension (which
is conformally invariant, btw) from the case of real-valued fields, to
general G-valued fields, where G is our favorite Lie group.  

<H5> 4.  Quantum groups </H5>

When you quantize a WZW theory whose symmetry group G is some simple
Lie group, something funny happens.  In a sense, the group itself
also gets quantized!  In other words, the algebraic structure of
the group, or its Lie algebra, gets "deformed" in a way that depends on 
the parameter \hbar  (Planck's constant).  I have muttered much about
quantum groups on This Week's Finds, especially concerning their
relevance to topological quantum field theory, and I will not try to
explain them any better here!  Eventually I will discuss a bunch of
books that have come out on quantum groups, and I hope to give
a mini-introduction to the subject in the process.

<H5> 5.  Duality, fusion rules, and modular invariance </H5>

The previous chapter described quantum groups as abstract algebraic
structures, showing how you can get one from any simple Lie algebra.
Here Fuchs really shows how you get them from quantizing a WZW theory.
WZW theories are invariant under conformal transformations, and
quantum groups inherit lots of cool properties from this fact.  For
example, suppose you form a torus by taking the complex plane
and identifying two points if they differ by any number of the form
n z_{1} + m z_{2}, where z_{1} and z_{2} 
are fixed complex numbers and n, m are arbitrary integers.  For example, 
we might identify all these points:



\begin{verbatim}

                x      x      x      x      x



                x      x      x      x      x



                x      x      x      x      x

\end{verbatim}
    

The resulting torus is a "Riemann surface" and it has lots of
transformations, called "modular transformations".  The group of
modular transformations is the discrete group SL(2,Z) of 2\times 2 integer
matrices with determinant 1; I leave it as an easy exercise to guess
how these give transformations of the torus.  (This is an example of a
"mapping class group" as discussed in "<A HREF =
"week28.html">week28</A>".)  In any event, the way the the WZW theory
transforms under modular transformations translates into some cool
properties of the corresponding quantum group, which Fuchs discusses.
That's roughly what "modular invariance" means.

Similarly, "fusion rules" have to do with the thrice-punctured sphere,
or "trinion", which is another Riemann surface.  And "duality" has
to do with the sphere with four punctures, which can be viewed as
built up from trinions in either of two "dual" ways:


\begin{verbatim}

    0\          /0
     \\________//
      )________(
     //        \\
    0/          \0
\end{verbatim}
    

or


\begin{verbatim}

      O    O
      \\  //
       \\//
        ||                 
        ||
       //\\
      //  \\
      O    O
\end{verbatim}
    

This is one of the reasons string theory was first discovered -
we can think of the above pictures as two Feynman diagrams for
interacting strings, and the fact that they are really just distorted
versions of each other gives rise to identities among Feynman diagrams.
Similarly, this fact gives rise to identities satisfied by the fusion
rules of quantum groups.

So - Fuchs' book should make clear how, starting from the austere
beauty of the Dynkin diagrams, we get not only simple Lie groups, but
also a wealth of more complicated structures that people find important
in modern theoretical physics.  

\par\noindent\rule{\textwidth}{0.4pt}
<em>Mathematics, rightly viewed, possesses not only truth, but supreme 
beauty - a beauty cold and austere, like that of sculpture, without appeal 
to any part of our weaker nature, without the gorgeous trappings of painting 
or music, yet sublimely pure, and capable of a stern perfection such as 
only the greatest art can show.</em> - Bertrand Russell.

\par\noindent\rule{\textwidth}{0.4pt}

% </A>
% </A>
% </A>
