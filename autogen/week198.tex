
% </A>
% </A>
% </A>
\week{September 6, 2003 }


I recently got back from a summer spent mostly in Hong Kong.
It was interesting being there.  Since I wasn't there long,
most of my observations are pretty superficial.  For example, 
they have a real commitment to public transportation.  Not 
only is there a wonderful system of subways, ferries, buses 
and green minibuses where you can pay for your ride using a 
cool high-tech "octopus card", the local gangs run their own 
system of \emph{red} minibuses.  These don't run on fixed schedules, 
and they don't take the octopus card, but they seem perfectly
safe, and they go places the others don't.  

Another obvious feature is the casual attitude towards English, 
which is still widely used, but plays second fiddle to Cantonese 
now that the Brits have been kicked out.  Menus feature strange 
items such as "mocked eel" and "mocked shark fin soup", which 
bring to mind the unsettling image of a cook ridiculing hapless
sea creatures before cooking them.  Also, perfectly nice people 
wear T-shirts saying things that wouldn't be wise where I come 
from, like


\begin{verbatim}

                         Lost Pig
\end{verbatim}
    
or 


\begin{verbatim}

                          I SEE 
                           WHY 
                         YOU SUCK
\end{verbatim}
    
On a more serious note, it was interesting to see the effects 
of the July 1st protest against Article 23 - an obnoxious piece of
security legislation that Tung Chee-Hwa was trying to push through.   
About 8% of the entire population went to this demonstration.  It 
stopped or at least delayed passage of the current version of this 
bill, and seems to have invigorated the democracy movement.  Time 
will tell if it leads to good effects or just a crackdown of some 
sort.  The police have placed a large order for tear gas.   

While in Hong Kong, I received a copy of a very interesting book:

1) David Corfield, Towards a Philosophy of Real Mathematics,
Cambridge U. Press, Cambridge, 2003.  More information
and part of the book's introduction available at 
<A HREF = "http://www-users.york.ac.uk/~dc23/Towards.htm">
http://www-users.york.ac.uk/~dc23/Towards.htm</A>

I should admit from the start that I'm completely biased
in favor of this book, because it has a whole chapter 
on one of my favorite subjects: higher-dimensional algebra.
Furthermore, Corfield cites me a lot and says I deserve 
"lavish praise for the breadth and quality of my exposition".  
How could I fail to recommend a book by so wise an author? 

That said, what's really special about this book is that it 
shows a philosopher struggling to grapple with modern mathematics 
as it's actually carried out by its practitioners.  This is what
Corfield means by "real" mathematics.  Too many philosophers of
mathematics seem stuck in the early 20th century, when explicitly 
"foundational" questions - questions of how we can be certain of 
mathematical truths, or what mathematical objects "really are" - 
occupied some the best mathematicians.  These questions are fine 
and dandy, but by now we've all heard plenty about them and not 
enough about other \emph{equally} interesting things.  Alas, too many 
philosophers seem to regard everything since Goedel's theorem as 
a kind of footnote to mathematics, irrelevant to their loftier
concerns (read: too difficult to learn).

Corfield neatly punctures this attitude.  He calls for philosophers 
of mathematics to follow modern philosophers of the natural sciences
and focus more on what practitioners actually do:

\begin{quote}
     [...] to the extent that we wish to emulate Lakatos and 
     represent the discipline of mathematics as the growth of 
     a form of knowledge, we are duty bound to study the means 
     of production throughout its history.  There is sufficient
     variation in these means to warrant the study of contemporary
     forms.  The quaint hand-crafted tools used to probe the 
     Euler conjecture in the early part of the nineteenth century 
     studied by Lakatos in "Proofs and Refutations" have been 
     supplanted by the industrial-scale machinery of algebraic 
     topology developed since the 1930s.  

\end{quote}
He also tries to strip away the "foundationalist filter" that 
blinds people into seeing philosophically interesting mathematics
only in the realms of logic and set theory:

\begin{quote}
     [...] Straight away, from simple inductive considerations, 
     it should strike us as implausible that mathematicians 
     dealing with number, function and space have produced 
     nothing of philosophical significance in the past seventy 
     years in view of their record over the previous three centuries.
     Implausible, that is, unless by some extraordinary event 
     in the history of philosophy a way had been found to \emph{filter},
     so to speak, the findings of mathematicians working in core
     areas, so that even the transformations brought about by
     the development of category theory, which surfaced explicitly
     in 1940s algebraic topology, or the rise of non-commutative
     geometry over the past seventy years, are not deemed to merit
     philosophical attention.  

\end{quote}
To me, it's a breath of fresh air just to see a philosopher
of mathematics \emph{mention} non-commutative geometry.  So often 
they seem to occupy an alternate universe in which mathematics 
stopped about a hundred years ago!  Elsewhere in the book 
we find interesting discussions of Eilenberg-MacLane spaces, 
groupoids, the Ising model, and Monstrous Moonshine.  One
gets the feeling that the author is someone we might meet 
on the internet instead of the coffeehouses of fin-de-siecle 
Vienna, who writes using a word processor instead of
a fountain pen.    

The book consists of chapters on loosely linked subjects,
some of which seem closer to "real mathematics" than others.
The chapters on "Communicating with automated theorem provers" 
and "Automated conjecture formation" are mildly depressing, 
given how poor computers are at spotting or proving truly 
interesting conjectures without lots of help from humans -
at least so far.  True, Corfield describes how in 1996 the 
automated theorem prover EQP was the first to crack the 
Robbins conjecture.  This states that a Boolean algebra is 
the same as a set equipped with an commutative associative 
binary operation "or" together with a unary operation "not" 
for which one mind-numbing axiom holds, namely:


\begin{verbatim}

not(not(p or q) or not(p or not(q)) = p
\end{verbatim}
    
All the rest of Boolean logic is a consequence!   But proving 
this seems more like a virtuoso stunt than the sort of thing 
we working mathematicians do for a living.  This is actually 
part of Corfield's point, but I find it a somewhat odd choice 
of topic, unless perhaps philosophers need to be convinced 
that the business of mathematics is still a mysterious process, 
not yet easily automated.

Apart from the one on higher-dimensional algebra, the chapters
that make me happiest are the ones on "The importance of 
mathematical conceptualisation" and "The role of analogy in
mathematics".   

The first is a marvelous study of the so-called "conceptual 
approach" in mathematics, which emphasizes verbal reasoning using 
broad principles over calculations using symbol manipulation.
Some people are fond of the conceptual approach, while others 
regard it as "too abstract".  Corfield illustrates this split
using the debate over "groupoids versus groups", with the supporters 
of groupoids (including Grothendieck, Brown and Connes) taking 
the conceptual high road, but others preferring to stick with
groups whenever possible.  As a philosopher, Corfield naturally
leans towards the conceptual approach.

The second is all about analogies.  Analogies are incredibly 
important in mathematics.   Some can be made completely precise 
and their content fully captured by a theorem, but the "deep" 
ones, the truly fruitful ones, are precisely those that resist 
complete encapsulation and only yield their secrets a bit at a 
time.  Corfield quotes Andre Weil, who describes the phenomenon 
as only a Frenchman could - even in translation, this sounds like 
something straight out of Proust:

\begin{quote}
     As every mathematician knows, nothing is more fruitful
     than these obscure analogies, these indistinct reflections
     of one theory into another, these furtive caresses, these 
     inexplicable disagreements; also nothing gives the researcher 
     greater pleasure.
\end{quote}
I actually doubt that \emph{every} mathematician gets so turned on by 
analogies, but many of the "architects" of mathematics do, and 
Weil was one.  Corfield examines various cases of analogy and 
studies how they work: they serve not only to discover and 
prove results but also to \emph{justify} them - that is, explain why 
they are interesting.  He also examines the amount of freedom 
one has in pushing forwards an analogy.  This is a nice concrete
way to ponder the old question of how much of math is a free
human creation and how much is a matter of "cutting along the 
grain" imposed by the subject matter.  

The analogy he considers in most detail is a famous one between 
number fields and function fields, going back at least to Dedekind 
and Kummer.  By a "number field", we mean something like the set 
of all numbers 


\begin{verbatim}

a + b sqrt(-5)
\end{verbatim}
    
with a,b rational.  This is closed under addition, subtraction,
multiplication, and division by anything nonzero, and the usual
laws hold for these operations, so it forms a "field".  By a
"function field", we mean something like the set of all rational
functions in one complex variable:


\begin{verbatim}

P(z)/Q(z)                    
\end{verbatim}
    
with P,Q polynomials.  This is again a field under the usual
operations of addition, subtraction, multiplication and division.

Sitting inside a number field we always have something called the 
"algebraic integers", which in the above example are the numbers


\begin{verbatim}

a + b sqrt(-5)
\end{verbatim}
    
with a,b integers.  These are closed under addition, subtraction,
multiplication but not division so they form a "commutative ring".
Similarly, sitting inside our function field we have the "algebraic 
functions", which in the above example are the polynomials


\begin{verbatim}

P(z)
\end{verbatim}
    
This is again a commutative ring.  

So, an analogy exists.  But the cool part is that there's a 
good generalization of "prime numbers" in the algebraic integers
of any number field, invented by Kummer and called "prime ideals"... 
and prime ideals in the algebraic functions of a function field 
have a nice \emph{geometrical} interpretation!  In the example given
above, they correspond to points in the complex plane!

The analogy between number fields and function fields has been 
pushed to yield all sorts of important results in number theory 
and algebraic geometry.  In Weil's hands it led to the theory of 
adeles and the Weil conjectures.  These in turn led to etale 
cohomology, Grothendieck's work on topoi, and much more.  And 
the underlying analogy is still far from exhausted!  But if we 
ever get it completely nailed down, then (in the words of Weil):

\begin{quote}
     The day dawns when the illusion vanishes; intuition
     turns to certitude; the twin theories reveal their 
     common source before disappearing; as the \emph{Gita} teaches 
     us, knowledge and indifference are attained at the same 
     moment.  Metaphysics has become mathematics, ready to
     form the material for a treatise whose icy beauty no
     longer has the power to move us.
\end{quote}
Or something like that.

Anyway, I hope this book shows philosophers that modern mathematics
poses many interesting questions apart from the old "foundational"
ones.  These questions can only be tackled after taking time to 
learn the relevant math... but what could be more fun than that?!  
I also hope this book shows mathematicians that having a well-
informed and clever philosopher around makes math into a more 
lively and self-aware discipline.

(The same is true of physics, of course.  I listed a few good 
philosophers of physics in "<A HREF = "week190.html">week190</A>".)  

Someday I'd like to say more about the analogy between number
fields and function fields, because I'm starting to study 
this stuff with James Dolan... but it will take a while 
before I know enough to say anything interesting.  So instead, 
let me say what's going on with spin foam models of quantum gravity.  

I've already talked about these 
in "<A HREF = "week113.html">week113</A>",
"<A HREF = "week120.html">week120</A>", 
"<A HREF = "week128.html">week128</A>" 
and "<A HREF = "week168.html">week168</A>".  
The idea is to calculate the amplitude for spacetime 
to have any particular geometry.  An amplitude is just a complex 
number, sort of the quantum version of a probability.  If you know 
how to calculate an amplitude for each spacetime, you can try to 
compute the expectation value of any observable by averaging its 
value over all possible geometries of spacetime, weighted by their 
amplitudes.   When you do this to answer questions about physics at 
large distances scales, the amplitudes should almost cancel except 
for spacetimes that come close to satisfying the equations of general 
relativity.  This is how quantum gravity should reduce to classical
gravity at distance scales much larger than the Planck length.

But in a spin foam model, a spacetime geometry is not described 
by putting a metric on a manifold, as in general relativity.  
Instead, it's described in a somewhat more "discrete" manner.  
Only at distances substantially larger than the Planck length 
should it resemble a metric on a manifold.

How do you describe a spacetime geometry in a spin foam model?
Well, first you take some 4-dimensional manifold representing 
spacetime and chop it into "4-simplices".  A "4-simplex" is 
just the 4-dimensional analogue of a tetrahedron: it has 5 
tetrahedral faces, 10 triangles, 10 edges and 5 vertices.  
Then, you label all the triangles in these 4-simplices by numbers.  
These describe the \emph{areas} of the triangles.  Here the details 
depend on which spin foam model you're using.  In the Riemannian 
Barrett-Crane model, you label the triangles by spins j = 0, 1/2, 
1, 3/2....  But in the Lorentzian Barrett-Crane model, which 
should be closer to the real world, you label them by arbitrary 
positive real numbers.   Either way, a spacetime chopped up into 
4-simplices labelled with numbers is called a "spin foam".

To compute an amplitude for one of these spin foams, you first use 
the labellings on the triangles and follow certain specific formulas
to calculate a complex number for each 4-simplex, each tetrahedron, 
and each triangle.  Then you multiply all these numbers together 
to get the amplitude!  

In "<A HREF = "week170.html">week170</A>", I mentioned some mysterious news about the Barrett-Crane 
model.  At the time - this was back in August of 2001 - my collaborators
Dan Christensen and Greg Egan were using a supercomputer to calculate 
the amplitudes for lots of spin foams.  The hard part was calculating 
the numbers for 4-simplices, which are called the "10j symbols" since 
they depend on the labels of the 10 triangles.  They had come up with 
an efficient algorithm to compute these 10j symbols, at least in the 
Riemannian case.  And using this, they found that the 10j symbols were 
\emph{not} coming out as an approximate calculation by Barrett and Williams 
had predicted!

Barrett and Williams had done a "stationary phase approximation" to
argue that in the limit of a very large 4-simplex, the 10j symbols
were asymptotically equal to something you'd predict from general
relativity.  This seemed like a hint that the Barrett-Crane model 
really did reduce to general relativity at large distance scales, 
as desired.

However, things actually work out quite differently!  By now
the asymptotics of the 10j symbols are well understood, and they're
\emph{not} given by the stationary phase approximation.  If you want to 
see the details, read these papers:

2) John C. Baez, J. Daniel Christensen and Greg Egan,
Asymptotics of 10j symbols, Class. Quant. Grav. 19 (2002) 6489-6513.  
Also available as <A HREF = "http://xxx.lanl.gov/abs/gr-qc/0208010">gr-qc/0208010</A>.

3) John W. Barrett and Christopher M. Steele, Asymptotics of 
relativistic spin networks, Class. Quant. Grav. 20 (2003) 1341-1362.
Also available as <A HREF = "http://xxx.lanl.gov/abs/gr-qc/0209023">gr-qc/0209023</A>.

4) Laurent Freidel and David Louapre, Asymptotics of 6j and 10j 
symbols, Class. Quant. Grav. 20 (2003) 1267-1294.  Also available as 
<A HREF = "http://xxx.lanl.gov/abs/hep-th/0209134">hep-th/0209134</A>.

The physical meaning of this fact is still quite mysterious.  I could 
tell you everyone's guesses, but I'm not sure it's worthwhile.  Next
spring, Carlo Rovelli, Laurent Freidel and David Louapre are having 
a conference on loop quantum gravity and spin foams in Marseille.  
Maybe after that people will understand what's going on well enough
for me to try to explain it!

I'd like to wrap up with a few small comments about last Week.
There I said a bit about a 24-element group called the "binary 
tetrahedral group", a 24-element group called SL(2,Z/3), and 
the vertices of a regular polytope in 4 dimensions called the 
"24-cell".  The most important fact is that these are all the 
same thing!  And I've learned a bit more about this thing from here:
 
5) Robert Coquereaux, On the finite dimensional quantum group
H = M_{3} + (M_{2|1}(\Lambda ^{2}))_{0}, 
available as <A HREF = "http://xxx.lanl.gov/abs/hep-th/9610114">hep-th/9610114</A> and at 
<A HREF = "http://www.cpt.univ-mrs.fr/~coque/articles_html/SU2qba/SU2qba.html"> http://www.cpt.univ-mrs.fr/~coque/articles_html/SU2qba/SU2qba.html</A>

Just to review: let's start with the group consisting of all the 
ways you can rotate a regular tetrahedron and get it looking the 
same again.  You can achieve any even permutation of the 4 vertices 
using such a rotation, so this group is the 12-element group A_{4} 
consisting of all even permutations of 4 things - see "<A HREF = "week155.html">week155</A>".  
But it's also a subgroup of the rotation group SO(3).   So,
its inverse image under the double cover

SU(2) \to  SO(3)

has 24 elements.  This is called the "binary tetrahedral group".

As usual, the algebra of complex functions on this finite group
is a Hopf algebra.  But the cool thing is, this Hopf algebra is
closely related to the quantum group U_{q}(sl(2)) when q is a third
root of unity - a quantum group used in Connes' work on particle
physics because of its relation to the Standard Model gauge group!
In short: the plot thickens.

I'm not really ready to describe this web of ideas in detail,
so I'll just paraphrase the abstract of Coquereaux's paper and 
urge you to either read this paper or look at his website:

\begin{quote}
     We describe a few properties of the non-semisimple associative 
     algebra H = M_{3} + (M_{2|1}(\Lambda ^{2}))_{0}, where \Lambda ^{2} is 
     the Grassmann algebra with two generators.  We show that H 
     is not only a finite dimensional algebra but also a 
     (non-cocommutative) Hopf algebra, hence a "finite quantum 
     group". By selecting a system of explicit generators, we 
     show how it is related with the quantum enveloping algebra 
     of U_{q}(sl(2)) when the parameter q is a cubic root of unity.  
     We describe its indecomposable projective representations as 
     well as the irreducible ones. We also comment about the relation
     between this object and the theory of modular representations 
     of the group SL(2,Z/3), i.e. the binary tetrahedral group.  
     Finally, we briefly discuss its relation with the Lorentz group
     and, as already suggested by A. Connes, make a few comments 
     about the possible use of this algebra in a modification of 
     the Standard Model of particle physics (the unitary group of
     the semi-simple algebra associated with H is U(3) x U(2) x U(1)). 
\end{quote}

\par\noindent\rule{\textwidth}{0.4pt}
\textbf{Addenda:} I got some interesting feedback from Martin
Krieger and Noam Elkies.   

 Martin Krieger writes:

\begin{quote}
In the interchange on Corfield's book, and John Baez's discussion of it,
there is a reference to Weil's Rosetta Stone analogy. The quotations come
from a charming and deep letter Weil wrote in 1940 to his sister, Simone,
from Bonne Nouvelle prison. In my book <em>Doing Mathematics: Convention,
Subject, Calculation, Analogy</em> (World Scientific, 2003) that long letter is
translated into English (see Appendix D, pp. 293-305). I also happen to have
a discussion of the analogy, in chapter 5 (pp. 189-230), in connection with
the Langlands program and with results in statistical mechanics of the Ising
model.
Martin Krieger<BR>
University of Southern California<BR>
Los Angeles CA 90089-0626<BR>
\end{quote}

You can now find Krieger's translation of this letter 
online, as long as you register with the American
Mathematical Society (it's free):

6) Martin H. Krieger, A 1940 letter of Andre Weil on analogy
in mathematics, AMS Notices 52 (March 2005), 334-341.  Available
at <A HREF = "http://www.ams.org/notices/200503/200503-toc.html">http://www.ams.org/notices/200503/200503-toc.html</A>


Noam Elkies writes:

\begin{quote}
Hello again,

You write:

$$

[...]

>I'd like to wrap up with a few small comments about last Week.
>There I said a bit about a 24-element group called the "binary 
>tetrahedral group", a 24-element group called SL(2,Z/3), and 
>the vertices of a regular polytope in 4 dimensions called the 
>"24-cell".  The most important fact is that these are all the 
>same thing!  And I've learned a bit more about this thing from 
>here:

[...]
$$
    

Here's yet another way to see this: the 24-cell is the subgroup
of the unit quaternions (a.k.a. SU(2)) consisting of the elements
of norm 1 in the Hurwitz quaternions - the ring of quaternions
obtained from the Z-span of {1,i,j,k} by plugging up the holes
at (1+i+j+k)/2 and its <1,i,j,k> translates.  Call this ring A.
Then this group maps injectively to A/3A, because for any g,g'
in the group |g-g'| is at most 2 so g-g' is not in 3A unless g=g'.
But for any odd prime p the (Z/pZ)-algebra A/pA is isomorphic
with the algebra of 2*2 matrices with entries in Z/pZ,
with the quaternion norm identified with the determinant.
So our 24-element group injects into SL_{2}(Z/3Z) - which is
barely large enough to accommodate it.  So the injection
must be an isomorphism.

Continuing a bit longer in this vein: this 24-element group
then injects into SL_{2}(Z/pZ) for any odd prime p, but this injection
is not an isomorphism once p>3.  For instance, when p=5 the image
has index 5 - which, however, does give us a map from SL_{2}(Z/5Z)
to the symmetric group of order 5, using the action of SL_{2}(Z/5Z)
by conjugation on the 5 conjugates of the 24-element group.
This turns out to be one way to see the isomorphism of PSL_{2}(Z/5Z)
with the alternating group A_{5}.

Likewise the octahedral and icosahedral groups S_{4} 
and A_{5}
can be found in PSL_{2}(Z/7Z) and PSL_{2}(Z/11Z), which gives
the permutation representations of those two groups on
7 and 11 letters respectively; and A_{5} is also an index-6 subgroup
of PSL_{2}(F_{9}), which yields the identification of 
that group with A_{6}.

NDE
\end{quote}

\par\noindent\rule{\textwidth}{0.4pt}
<em>The enrapturing discoveries of our field systematically conceal,
like footprints erased in the sand, the analogical train of thought
that is the authentic life of mathematics</em> - Gian-Carlo Rota
\par\noindent\rule{\textwidth}{0.4pt}

% </A>
% </A>
% </A>
