
% </A>
% </A>
% </A>
\week{ugust 10, 1994}

Mainly this week I have various bits of news to report from the
7th Marcel Grossman Meeting on general relativity.   It was big
and had lots of talks.  Bekenstein gave a nice review talk on
entropy/area relations for black holes, and Strominger gave a 
talk in which he proposed a solution to the information loss
puzzle for black holes.  (Recall that if one believes, as most people
seem to believe, that black holes radiate away all their mass in the
form of completely random Hawking radiation, then there's a question
about where the information has gone that you threw into the black hole
in the form of, say, old issues of Phys. Rev. Lett..  Some people
think the information goes into a new "baby universe" formed at the
heart of the black hole - see "<A HREF = "week31.html">week31</A>" for more.  The information
would still, of course, be gone from \emph{our} point of view in this
picture.  Strominger proposed a set up in which one had a quantum theory
of gravity with annihilation and creation operators for baby universes,
and proposed that the universe (the "metauniverse"?) was in a coherent
state, that is, an eigenstate of the annihilation operator for baby
universes.  This would apparently allow handle the problem, though right
now I can't remember the details.)  There were also lots of talks on the
interferometric detection of gravitational radiation, other general
relativity experiments, cosmology, etc..  But I'll just try to describe
two talks in some detail here.

1) L. Lindblom, Superfluid hydrodynamics and the stability of
rotating neutron stars, talk at MG7 meeting, Monday July 5, 1994,
Stanford University.

Being fond of knots, tangles, and such, I have always liked
knowing that in superfluids, vorticity (the curl of the  velocity
vector field) tends to be confined in "flux tubes", each
containing an angular momentum that's an integral multiple of 
Planck's constant, and that similarly, in type II
superconductors, magnetic fields are confined to magnetic flux
tubes.  And I was even more happy to find out that the cores of 
neutron stars are expected to be made of neutronium that is
\emph{both} superfluid \emph{and} superconductive, and contain lots of flux
tubes of both types.  In this talk, which was really about a
derivation of detailed equations of state for neutron stars,
Lindblom began by saying that the maximum rotation rate of a
rotating neutron star is due to some sort of "gravitational
radiation instability due to internal fluid dissipation".  I
didn't quite understand the details of that, which weren't
explained, but it motivated him to study the viscosity in
neutron star cores, which are superfluid if they are cool enough
(less than a billion degrees Kelvin).  There are some protons and
electrons mixed in with the neutrons in the core, and both the
protons and neutrons go superfluid, but the electrons form a
normal fluid.  That means that there are actually \emph{two} kinds of
superfluid vortices - proton and neutron - in addition to the
magnetic vortices.   These vortices mainly line up along the axis
of rotation, and their density is about 10^6 per square
centimeter.  Rather curiously, since the proton, neutron, and
electron fluids are coupled due to &beta; decay (and the reverse
process), even the neutron vortices have electric currents
associated to them and generate magnetic fields.  This means that
the electrons scatter off the neutron vortex cores as well as the
proton vortex cores, which is one of the mechanisms that yields
viscosity.

2) Abhay Ashtekar, Mathematical developments in quantum general
relativity, a sampler, talk at MG7 meeting, Tuesday July 6, 1994,
Stanford University.

This talk, in addition to reviewing what's been done so far on 
the "loop representation" of quantum gravity, presented two new
developments that I found quite exciting, so I'd like to sketch
what they are.  The details will appear in future papers by
Ashtekar and collaborators.   

The two developments Ashtekar presented concerned mathematically
rigorous treatments of the "reality conditions" in his approach to
quantum gravity, and the "loop states" used by Rovelli and Smolin.
First let me try to describe the issue of "reality
conditions".  As I described in "<A HREF = "week7.html">week7</A>", one trick that's important
in the loop representation is to use the "new variables" for
general relativity introduced by Ashtekar (though Sen and
Plebanski already had worked with similar ideas).  In the older
Palatini approach to general relativity, the idea was to view
general relativity as something like a gauge theory with gauge
group given by the Lorentz group, SO(3,1).   But to do this one
actually uses two different fields: a "frame field", also called
a "tetrad", "vierbein" or "soldering form" depending on who
you're talking to, and the gauge field itself, usually called a
"Lorentz connection" or "SO(3,1) connection".  Technically, the
frame field is an isomorphism between the tangent bundle of
spacetime and some other bundle having a fixed metric of
signature +---, usually called the "internal space", and the
Lorentz connection is a metric-preserving connection on the
internal space.   

The "new variables" trick is to use the fact that SO(3,1) has as
a double cover the group SL(2,C) of two-by-two complex matrices
with determinant one.  (For people who've read previous posts of mine, I
should add that the Lie algebra of SL(2,C) is called sl(2,C) and is
the same as the complexification of the Lie algebra so(3), which
allows one to introduce the new variables in a different but equivalent
way, as I did in "<A HREF = "week7.html">week7</A>".)  Ignoring topological niceties for now, this
lets one reformulate \emph{complex} general relativity (that is, general
relativity where the metric can be complex-valued) in terms of a
\emph{complex-valued} frame field and an SL(2,C) connection that is
just the Lorentz connection in disguise.  The latter is called
either the "Sen connection", the "Ashtekar connection", or the
"chiral spin connection" depending on who you're talking to.  The
advantage of this shows up when one tries to canonically quantize
the theory in terms of initial data.   (For a bit on this, try
"<A HREF = "week11.html">week11</A>".)  Here we assume our 4-dimensional spacetime can be split up
into "space" and "time", so that space is a 3-dimensional
manifold, and we take as our canonically conjugate fields the
restriction of the chiral spin connection to space, call it A,
and something like the restriction of the complex frame field to a
complex frame field E on space.  (Restricting the complex frame field to
one on space is a wee bit subtle, especially because one doesn't really
want a frame field or "triad field", but really a "densitized
cotriad field" - but let's not worry about this here.   I
explain this in terms even a mathematician can understand in my
paper "string.tex", available by ftp along with all my "week"
files as described below.)  The point is, first, that the A and E
fields are mathematically very analogous to the vector potential
and electric field in electromagnetism - or really in SL(2,C)
Yang-Mills theory - and secondly, that if you compute their
Poisson brackets, you really do see that they're canonically
conjugate.  Third and best of all, the constraint equations in
general relativity can be written down very simply in terms of A
and E.   Recall that in general relativity, 6 of Einstein's 10 equations
act as \emph{constraints} that the metric and its time derivative must
satisfy at t = 0 in order to get a solution at later times.  
In quantum gravity, these constraints are a big technical problem one
has to deal with, and the point of Ashtekar's new variables is precisely
that the constraints simplify in terms of these variables.  (There's
more on these constraints in "<A HREF = "week11.html">week11</A>".)

The price one has paid, however, is that one now seems to be
talking about \emph{complex-valued} general relativity, which isn't
what one had started out being interested in.  One needs to get
back to reality, as it were - and this is the problem of the
so-called "reality conditions".  One approach is to write down extra
constraints on the E field that say that it comes from a \emph{real}
frame field.  These are a little messy.   Ashtekar,
however, has proposed another approach especially suited to
the quantum version of the theory, and in his talk he filled in
some of the crucial details.   

Here, to save time, I will allow myself to become a bit more
technical.  In the quantum version of the theory one expects the
space of wavefunctions to be something like L^2 functions on the
space of connections modulo gauge transformations - actually
this is the "kinematical state space" one gets before writing the
constraints as operators and looking for wavefunctions
annihilated by these constraints.  The problem had always been
that this space of L^2 functions is ill-defined, since there is no "Lebesgue
measure" on the space of connections.  This problem is addressed
(it's premature to say "solved") by developing a theory of
generalized measures on the space of connections and proving the
existence of a canonical generalized measure that deserves the
name "Lebesgue measure" if anything does.   One can then define
L^2 functions and work with them.   For compact gauge groups,
like SU(2), this was done by Ashtekar, Lewandowski and myself;
see e.g. the papers "state.tex" and "conn.tex" available by ftp.  
In the case of SU(2), Wilson loops act as self-adjoint
multiplication operators on the resulting L^2 space.   But in
quantum gravity we really want to use gauge group SL(2,C), which
is not compact, and we want the adjoints of Wilson loop operators
to reflect that fact that the SL(2,C) connection A in quantum
gravity is really equal to \Gamma  + iK, where \Gamma  is the
Levi-Civita connection on space, and K is the extrinsic
curvature.  Both \Gamma  and K are real in the classical theory, so 
the adjoint of the quantum version of A should be \Gamma  - iK, and
this should reflect itself in the adjoints of Wilson loop
operators.   

The trick, it turns out, is to use some work of Hall which
appeared in the Journal of Functional Analysis in 1994 (I don't
have a precise reference on me).  The point is that  SL(2,C) is
the complexification of SU(2), and can also be viewed as the
cotangent bundle of SU(2).  This allows one to copy a trick 
people use for the quantum mechanics of a point particle on R^n -
a trick called the Bargmann-Segal-Fock representation.  
Recall that in the ordinary Schrodinger representation of a
quantum particle on R^n, one takes as the space of states
L^2(R^n).   However, the phase space for a particle in R^n, which
is the cotangent bundle of R^n, can be identified with C^n, and
in the Bargmann representation one takes as the space of states
HL^2(C^n), by which I mean the \emph{holomorphic} functions on C^n
that are in L^2 with respect to a Gaussian measure on C^n.  In
the Bargmann representation for a particle on the line, for
example, the creation operator is represented simply as
multiplication by the complex coordinate z, while the
annihilation operator is d/dz.   Similarly, there is an
isomorphism between L^2(SU(2)) and a certain space HL^2(SL(2,C)).
Using this, one can obtain an isomorphism between the space of
L^2 functions on the space of SU(2) connections modulo gauge
transformations, and the space of holomorphic L^2 functions on
the space of SL(2,C) connections modulo gauge transformations.
Applying this to the loop representation, Ashtekar has found a
very natural way to take into account the fact that the chiral
spin connection A is really \Gamma  + iK, basically analogous to the
fact that  in the Bargmann multiplication by z is really q + ip (well,
up to various factors of sqrt(2), signs and the like).  

Well, that was pretty sketchy and probably not especially
comprehensible to anyone who hasn't already worried about this
issue a lot!  In any event, let me turn to the other good news
Ashtekar reported: the constuction of "loop states".  Briefly put
(I'm getting worn out), he and some collaborators have figured
out how to \emph{rigorously} construct generalized measures on the
space of connections modulo gauge transformations, starting from
invariants of links.  This begins to provide an inverse to the
"loop transform" (which is a construction going the other way).
<HR>

% </A>
% </A>
% </A>


% parser failed at source line 269
