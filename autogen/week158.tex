
% </A>
% </A>
% </A>
\week{October 16, 2000 }

Like lots of mathematicians these days, I'm trying to understand
M-theory.  It's a bit difficult, partially because the theory doesn't
really \emph{exist} yet.  If it existed, it would explain lots of stuff: on
that everyone agrees.  But nobody knows how to formulate M-theory in a
precise way, so you can't open up a paper and stare at "the fundamental
equation of M-theory", or anything like that.  There are some conjectures 
about what M-theory might be like, but no solid agreement.

One thing that \emph{does} exist is 11-dimensional supergravity.  This is 
supposed to be some kind of classical limit of M-theory.  But the good 
thing is, it's a classical field theory with a Lagrangian that you can 
write down and ponder to your heart's content.  So I'm trying to learn a 
bit about this. 
 
Unfortunately, being a mathematician, I like to understand everything 
rather carefully, preferably in a conceptual way that doesn't involve 
big equations with indices dangling all over the place.  This is slowing  
me down, because all the descriptions I've seen make 11-dimensional 
supergravity look sort of ugly, when in fact it should be really pretty. 
The physicists always point out that it's a lot simpler than the 
supergravity theories in lower dimensions.  On that I agree!  But I  
don't find it to be quite as simple as I'd like.   
 
Now, mathematicians always whine like this when they are trying to learn
physics that hasn't been pre-processed by some other mathematician.  So
just to show that I'm not completely making this stuff up, let me show
you the Lagrangian for 11d supergravity, as taken from the famous string
theory text by Green, Schwarz and Witten (see "<A HREF =
"week118.html">week118</A>"):
 

$$

L = - (1/2k^{2}) e R  
    - (1/2) e \psi *_{M} \Gamma ^{MNP} D_{N}[\omega  + \omega ')/2] \psi _{P} 
    - (1/48) e F^{2}
    - (\sqrt 2k/384) e (\psi *_{M} \Gamma ^{MNPQRS} \psi _{S} + 12 \psi *^{N} \Gamma ^{PQ} \psi ^{R}) (F + F')_{NPQR} 
    - (\sqrt 2k/3456) \epsilon ^{M1 ... M11} F_{M1 ... M4} F_{M5 ... M8} A_{M9 ... M11}
$$
    
For comparison, here's the Lagrangian for ordinary gravity: 
 

\begin{verbatim}

L = e R 
\end{verbatim}
    
Here e is the volume form and R is the Ricci scalar curvature.  Of
course, there is a lot of stuff packed into this "R".  General relativity 
didn't look so slick when Einstein first made it up!  But by now,
mathematicians have gnawed away at it for long enough that there's a
nice theory of differential geometry, where after a few months of work
you learn about "R".  And after you've done this work, you realize that
"R" is a very natural concept.  I want to get to this point for the
Lagrangian for 11d supergravity, but I'm not there yet.
 
You'll note that apart from a constant, the Lagrangian for 11d 
supergravity starts out basically like the Lagrangian for ordinary 
gravity.  So \emph{that} part I understand.  It's just the other stuff  
that's the problem. 
 
Modulo some subtleties discussed below, the whole Lagrangian is built 
from just three ingredients, which are the three basic fields in the 
theory: 
 
A) a Lorentzian metric g on the 11-dimensional manifold 
representing spacetime, 
 
B) a field \psi  on this manifold which takes values in the real  
spin-3/2 representation of SO(10,1), 
 
C) a 3-form field A on this manifold. 
 
Physicists call the metric the "graviton".  They call the spin-3/2 field 
the "gravitino" or a "Rarita-Schwinger field".  And they call the 3-form  
a "gauge field", by analogy to the 1-form that appears in electromagnetism. 
Above it's written as "A", to remind us of this analogy, but people often  
use a "C" instead - for reasons I'll explain later. 
 
Let me say a bit more about these three items.  To define a spin-3/2
field on a manifold we need to give the manifold a spin structure.
Locally, we can do this by picking a smoothly varying basis of tangent
vectors.  Such a thing is called a "frame field", but it also has other
names: in 4-dimensional spacetime people call it a "tetrad" or
"vierbein", after the German word for "four legs", but in 11-dimensional
spacetime people call it an "elfbein", after the German word for "eleven
legs".  Anyway, this frame field determines a spin structure, and also
a metric, if we declare the basis to be orthonormal.
 
The metric, in turn, determines the Levi-Civita connection on the 
tangent bundle.  However, in modern Lagrangians for gravity, people 
often treat the frame field and connection as independent variables.   
This amounts to dropping the requirement that the connection be 
torsion-free (while still requiring that it be metric-preserving).  Only 
when you work out the equations of motion from the Lagrangian do you get 
back the equation saying the connection is torsion-free - and even this 
only happens when there are no fields with \emph{spin} around.   In these 
theories, spin creates torsion!  But the torsion doesn't propagate:  it 
just sits there, determined by other fields.  So we are basically just 
repackaging the same data when we work with a frame field and connection 
instead of a metric.   
 
As a slight variant, instead of working with a frame field and 
connection on the tangent bundle, we can work with a frame field and 
"spin connection" - a connection on the spin bundle.  We need to do this 
whenever we have fields with half-integer spin around, as in supergravity.    
 
Okay, so we'll use a frame field and spin connection to describe the
graviton.  What about the gravitino?  I'm less clear about this, but I
guess the idea is that we think of the spin-3/2 representation of the
Lorentz group SO(10,1) as sitting inside the tensor product of the
spin-1 representation and the spin-1/2 representation.  This allows us
to think of the gravitino as a spinor-valued 1-form on spacetime.
That's why people write it as \psi _{N}: 
the subscript indicates that we've
got some sort of 1-form on our hands.  One thing I don't understand is
what, if any, constraints there are on a spinor-valued 1-form to make
it lie in the spin-3/2 representation.
  
What are spinors like in 11-dimensional spacetime?  For this, go back
and reread "<A HREF = "week93.html">week93</A>".  You'll see
that by Bott periodicity, spinors in (n+8)-dimensional spacetime are
just like spinors in n-dimensional spacetime, but tensored with
R^{16}.  So spinors in 11-dimensional spacetime are a lot like
spinors in 3-dimensional spacetime!  In 3 dimensions, the double cover
of the Lorentz group is just SL(2,R), and its spinor representation is
R^{2}.  Actually these are "real" spinors, or what
physicists call "Majorana" spinors.  We could complexify and
get "complex" or "Dirac" spinors - but we won't!
 
Since the space of Majorana spinors in 3d spacetime is R^{2},
the space of Majorana spinors in 11d spacetime is R^{2} tensor
R^{16} = R^{32}.  The gravitino is a 1-form taking
values in this space.
 
Finally, what about the 3-form that appears in 11d supergravity?  Why is
it called a "gauge field"?  Well, if you've made it this far,
you probably know that the 1-form in electromagnetism (the "vector
potential") is perfectly suited for integrating along the worldline
of a charged point particle.  Classically, the resulting number is just
the \emph{action}  In quantum theory, the exponential of the action describes
how the particle's \emph{phase} changes.
 
If we're dealing with strings instead of point particles, we can pull
the same trick using a 2-form, which is the right sort of thing to
integrate over the 2-dimensional worldsheet of a string.  Since people
call the 1-form in electromagnetism A, they naturally took to calling
this 2-form B.  People like to study strings propagating in a background
metric that satisfies the vacuum Einstein equations, but they also study
what happens when you throw in a background B field like this, and add a
term to the string action that's proportional to the integral of B over
the string worldsheet.  It works out nice when the B field satisfies the
obvious analogues of the vacuum Maxwell equations:
 

\begin{verbatim}

dF = 0, d*F = 0 
\end{verbatim}
    
where the "curvature" or "field strength tensor" F
is given by F = dB.
 
Like Maxwell's equations, these equations are
"gauge-invariant", in the sense that we can change B like this
without changing the field strength tensor:
 

$$

B \to  B + dw, 
$$
    
where w is any 1-form.    
 
Similarly, people believe that M-theory involves 2-dimensional membranes
called "2-branes".  A 2-brane traces out a 3-dimensional
"world-volume" in spacetime.  The 3-form field in 11d
supergravity is perfectly suited for integrating over this world-volume!
So we're really dealing with a still higher-dimensional analog of
electromagnetism.  Since we've already talked about a 1-form A that
couples to point particles and a 2-form field B that couples to strings,
it makes sense to call this 3-form C.  Lots of people do that.  But I'll
stick with Green, Schwarz and Witten, and call it A.  I'll write F for
the corresponding field strength (which is 6dA if we use their nutty
normalization).
 
Let's look at that Lagrangian again, and see how much of it we can 
understand now: 

$$

L = - (1/2k^{2}) e R  
    - (1/2) e \psi *_{M} \Gamma ^{MNP} D_{N}[\omega  + \omega ')/2] \psi _{P} 
    - (1/48) e F^{2}
    - (\sqrt 2k/384) e (\psi *_{M} \Gamma ^{MNPQRS} \psi _{S} + 12 \psi *^{N} \Gamma ^{PQ} \psi ^{R}) (F + F')_{NPQR} 
    - (\sqrt 2k/3456) \epsilon ^{M1 ... M11} F_{M1 ... M4} F_{M5 ... M8} A_{M9 ... M11}
$$
    

The number "k" is just a coupling constant.  The quantity
"e" is the volume form cooked up from the frame field.  The
quantity "R" is the Ricci scalar cooked up from the spin
connection.  "\psi _{N}" is the gravitino field, and
physicists write the inner product on spinors as
"\psi *_{N} \psi ^{N}", where the
star should really be a bar: a line drawn over the letter \psi .
"A" is the 3-form field and "F" is the field
strength.  There's also some other weird stuff I haven't explained yet.
 
Note: the first, middle, and last terms in this Lagrangian only involve 
the bosonic fields - not the gravitino.   They have the following meanings: 
 
The first term, the "e R" part, is just the Lagrangian for the 
gravitational field. 
 
The middle term is, up to a constant, just what I'd call "F ^
*F": the Lagrangian for the 3-form analog of Maxwell's equations.
 
The last term is, again up to a constant, just what I'd "F ^ F ^
A".  This is an 11-dimensional analog of the Chern-Simons term F ^
A that you can add on to the electromagnetic Lagrangian in 3d spacetime.
 
The other two terms involve the gravitino.  This is where I start getting 
nervous.  We've got this: 
 

$$

    - (1/2) e \psi *_{M} \Gamma ^{MNP} D_{N}[(\omega  + \omega ')/2] \psi _{P}
 
$$
    
and this: 
 

$$

     - (\sqrt 2k/384) e (\psi *_{M} \Gamma ^{MNPQRS} \psi _{S} + 12 \psi *^{N} \Gamma ^{PQ} \psi ^{R}) (F + F')_{NPQR} 
$$
    
The first one is mainly about how the gravitino propagates in a given 
metric - it's a kind of spin-3/2 analog of the Lagrangian for the Dirac 
equation.  The second one is mainly about the coupling of the gravitino 
to the 3-form field A - it's sort of like the coupling between the electron 
and electromagnetic field in QED.  But there's some funky stuff going on 
here!   
 
The "\Gamma " gadgets are antisymmetrized products of \gamma 
matrices, i.e. Clifford algebra generators.  I don't mind that.  It's
the stuff involving \omega ' and F' that confuses me.  "\omega "
is just a name for the spin connection, so D_{v}[\omega ] would mean
"covariant differentiation with respect to the spin
connection".  But instead of using that, we use D_{v}[(\omega  +
\omega ')/2], where \omega ' is the "supercovariantization" of the
spin connection.  Don't ask me that that means!  I know it amounts to
adding some terms that are quadratic in the gravitino field, and I know
it's required to get the whole Lagrangian to be invariant under a
"supersymmetry transformation", which mixes up the gravitino
field with the graviton and 3-form fields.  But I don't really
understand the geometrical meaning of what's going on, especially
because the supersymmetry only works "on shell" - i.e.,
assuming the equations of motion.  Similarly, I guess F' is some sort of
"supercovariantization" of the field strength tensor - but
again, it seems fairly mysterious.
 
Anyway, we can summarize all this by saying we've got gravity, a
gravitino, and a 3-form gauge field interacting in a manner vaguely
reminiscent of how gravity, the electron and the electromagnetic field
interact in the Einstein-Dirac-Maxwell equations - except that there's a
"four-fermion" term where four gravitinos interact directly.

Stepping back a bit, one is tempted to ask: what exactly is so great 
about this theory?   
 
There are various ways to focus this question a bit.  For example: the 
Lagrangian for ordinary gravity makes sense in a spacetime of any 
dimension.  The 11d supergravity Lagrangian, on the other hand, only 
makes sense in 11 dimensions.   Why is that?   
 
Well, if you ask a physicist, they'll tell you something like this: 
 
\begin{quote}
    Eleven is the maximum spacetime dimension in which one can formulate 
    a consistent supergravity, as was first recognized by Nahm in his 
    classification of supersymmetry algebras.  The easiest way to see this 
    is to start in four dimensions and note that one supersymmetry relates 
    states differing by one half unit of helicity.  If we now make the  
    reasonable assumption that there be no massless particles with spins 
    greater than two, then we can allow up to a maximum of N = 8  
    supersymmetries taking us from the helicity -2 through to helicity +2. 
    Since the minimal supersymmetry generator is a Majorana spinor with 
    four offshell components, this means a total of 32 spinor components. 
    Now in a spacetime with D dimensions and signature (1,D-1), the 
    maximum value of D admitting a 32 component spinor is D = 11. 
\end{quote}

In case you're wondering, this is from the first paragraph of this book: 
 
1) The World in Eleven Dimensions: Supergravity, Supermembranes and 
M-theory, ed. M. J. Duff, Institute of Physics Publishing, Bristol, 
1999. 
 
which is a collection of the most important articles on these topics. 
It's a fun book to carry around - you can really impress people with the 
title.  But if you're a mathematician trying to decipher the above 
passage, it helps to note a few things. 
 
First, this explanation of why 11d supergravity is good boils down to 
saying that it's the biggest, baddest supergravity theory around that 
doesn't give particles of spin greater than two when we compactify the 
extra dimensions in order to get a 4d theory.   
 
Second, why is it "reasonable" to assume that there aren't massless 
particles with spin greater than two?  Because it's physics folklore 
that quantum field theories with such particles are bad, nasty and evil 
- in fact, so evil that nobody even dares explain why!  Well, actually 
there's a paper by Witten in the above book that contains references to 
papers that supposedly explain why particles of spin > 2 are bad.  It's  
an excellent paper, too: 
 
2) Edward Witten, Search for a realistic Kaluza-Klein theory, Nucl. 
Phys. B186 (1981), 412-428. 
 
Maybe someday I'll get up the nerve to read those references. 
 
Third, once we buy into this "spin > 2 bad" idea, the rest
of the argument is largely stuff about spinors and Clifford algebras.
This is easy for mathematicians to learn, at least after a little
physics jargon has been explained.  For example, a "Majorana"
spinor is just a real spinor, and "offshell components" refer
to the components of a field that are independent before you impose the
equations of motion.
 
Fourth, if you're a mathematician wondering what "supersymmetry  
algebras" are, there are places where you can start learning about 
this without needing to know lots of physics: 
 
3) Quantum Fields and Strings: A Course for Mathematicians, 2 volumes,
eds. P. Deligne, P. Etinghof, D. Freed, L. Jeffrey, D. Kazhdan, D. Morrison 
and E. Witten, American Mathematical Society, Providence, Rhode Island,
1999.
 
Unfortunately, this book does not cover supergravity theories. 
 
Fifth, Nahm's classification of supersymmetry algebras looks like the 
sort of thing an algebraist should be able to understand, though I 
haven't yet understood it.  You can find it in Duff's book, or in the 
original paper:  
 
4) W. Nahm, Supersymmetries and their representations, Nucl. Phys. 
B135 (1978), 149-166. 
 
Next I want to mention some wild guesses and speculations about 11d  
supergravity and M-theory.  I'm guessing these theories are somehow 
a cousin of 3d Chern-Simons theory, related in a way that involves 
Bott periodicity.  And I'm guessing that there's something deeply 
octonionic about this theory.  There's probably something wrong about
these guesses, since I can't quite get everything to fall in line.
But there's also probably something right about them.
 
We've seen two clues already: 
 
First, the 11d spinors are related to 3d spinors via Bott periodicity,
which amounts to tensoring with R^{16} - the space of Majorana spinors
in 8d Euclidean space.  Given the relation between octonion, 8d spinors
and Bott periodicity (see "<A HREF = "week61.html">week61</A>"
and "<A HREF = "week105.html">week105</A>"), it's also very
natural to think of these Majorana spinors as pairs of octonions.
 
Second, the Chern-Simons-like term F ^ F ^ A in 11d supergravity is akin
to the 3d Chern-Simons Lagrangian F ^ A.  But this relation is a bit
odd, since a crucial part of it involves switching from a 1-form gauge
field in the 3d case to a 3-form gauge field in the 11d case.  To really
understand this, we first need to understand the geometry of these
generalized "gauge fields".  These higher gauge fields are
really not connections on bundles, but connections on
"n-gerbes", which are categorified analogues of bundles.  I
explained this to some extent in "<A HREF =
"week25.html">week25</A>" and "<A HREF =
"week151.html">week151</A>", but the basic idea is that there's an
analogy like this:
 

\begin{verbatim}

1-forms   connections on bundles     parallel transport of point particles 
2-forms   connections on gerbes      parallel transport of strings 
3-forms   connections on 2-gerbes    parallel transport of 2-branes 
4-forms   connections on 3-gerbes    parallel transport of 3-branes 
   .                 .                             . 
   .                 .                             . 
   .                 .                             . 
 
\end{verbatim}
    
and so on.  Just as connections on bundles naturally give rise to 
Chern classes and the Chern-Simons secondary characteristic classes, 
the same should be true for these higher analogues of connections. 
 
There is also another clue: as I mentioned in "<A HREF =
"week118.html">week118</A>", you can only write down Lagrangians
for supersymmetric membranes in certain dimensions.  There are
supposedly 4 basic cases, which correspond to the 4 normed division
algebras:


\begin{verbatim}

the 2-brane in dimension 4  - real numbers
the 3-brane in dimension 6  - complex numbers
the 5-brane in dimension 10 - quaternions
the 2-brane in dimension 11 - octonions
\end{verbatim}
    
Part of the point is that the in these theories there are 1, 2, 4, 
or 8 dimensions transverse to the worldvolume of the brane in question.
So 2-branes in 11 dimensions, in particular, are inherently "octonionic". 
This seems like a wonderful clue, but so far I don't really understand it.   
The evidence is lurking here: 
 
5) T. Kugo and P. Townsend, Supersymmetry and the division algebras, 
Nucl. Phys. B221 (1983), 357-380. 
 
6) G. Sierra, An application of the theories of Jordan algebras and 
Freudenthal triple systems to particles and strings, Class. Quant.  
Grav. 4 (1987) 227. 
 
7) J. M. Evans, Supersymmetric Yang-Mills theories and division algebras, 
Nucl. Phys. B298 (1988), 92. 
 
8) M. J. Duff, Supermembranes: the first fifteen weeks, Class. Quant. 
Grav. 5 (1988), 189-205. 
 
There are also tantalizing clues scattered through these fascinating 
books: 
 
9) Feza Gursey and Chia-Hsiung Tze, On the Role of Division, Jordan, and 
Related Algebras in Particle Physics, World Scientific, Singapore, 1996. 
 
10) Jaak Lohmus, Eugene Paal and Leo Sorgsepp, Nonassociative Algebras 
in Physics, Hadronic Press, Palm Harbor, Florida, 1994. 
 
However, these books are frustrating to me, because they make some 
interesting claims without providing solid evidence.   
 
Anyway, I'll try to keep gnawing away at this bone until I get to the 
marrow!   Any help would be appreciated.
 

\par\noindent\rule{\textwidth}{0.4pt}
\textbf{Addenda:}
Here is an article that Maxime Bagnoud posted to sci.physics.research,
which answers some of my questions above....

\begin{quote}

John Baez wrote:

\begin{verbatim}

> One thing that <em>does</em> exist is 11-dimensional supergravity.
\end{verbatim}
    

Unfortunately, only at the classical level, presumably. The quantum theory
doesn't seem to exist, neither. It's non-renormalizable, despite the large
amount of SUSY.  We were not sure about this until quite recently, actually 
(2 years ago?) You probably know this, but maybe not all the readers of the
"Finds".


$$

> Okay, so we'll use a frame field and spin connection to describe the
> graviton.  What about the gravitino?  I'm less clear about this, but I
> guess the idea is that we think of the spin-3/2 representation of the
> Lorentz group SO(10,1) as sitting inside the tensor product of the
> spin-1 representation and the spin-1/2 representation.  This allows us
> to think of the gravitino as a spinor-valued 1-form on spacetime.
> That's why people write it as \psi _{N}: the subscript indicates that we've
> got some sort of 1-form on our hands.  One thing I don't understand is
> what, if any, constraints there are on a spinor-valued 1-form to make
> it lie in the spin-3/2 representation.
$$
    

As you guessed, there is a Clebsch-Gordan relationship like:

1 \otimes  1/2 = 3/2 &oplus 1/2 (where \otimes  is tensor product, \oplus  is direct sum)

in fact, out of a general spinor-vector, you can form a linear combination of
its components to get a spin 1/2 spinor by multiplying \psi _M with a 
\Gamma ^M matrix and summing of course on the vector index. The remaining part 
of the representation is irreducible and it's the gravitino.  (You can look 
for example at Polchinski vol. II, page 23).

I guess that was your question.


$$

> Similarly, people believe that M-theory involves 2-dimensional membranes
> called "2-branes".  A 2-brane traces out a 3-dimensional "world-volume"
> in spacetime.  The 3-form field in 11d supergravity is perfectly suited
> for integrating over this world-volume!  So we're really dealing with a
> still higher-dimensional analog of electromagnetism.  Since we've already
> talked about a 1-form A that couples to point particles and a 2-form field
> B that couples to strings, it makes sense to call this 3-form C.  Lots of
> people do that.  But I'll stick with Green, Schwarz and Witten, and call
> it A.  I'll write F for the corresponding field strength (which is 6dA
> if we use their nutty normalization).
>
> Let's look at that Lagrangian again, and see how much of it we can
> understand now:
>
> L = - (1/2k^2) e R
>     - (1/2) e psibar_M \Gamma ^{MNP} D_N[(\omega  + \omega ')/2] \psi _P
>     - (1/48) e F^2
>     - (sqrt(2)k/384) e
>       (psibar_M \Gamma ^{MNPQRS} \psi _S + 12 psibar^N \Gamma ^{PQ} \psi ^R)
>       (F + F')_{NPQR}
>     - (sqrt(2)k/3456) \epsilon ^{M1 ... M11}
>       F_{M1 ... M4} F_{M5 ... M8} A_{M9 ... M11}
>
> The middle term is, up to a constant, just what I'd call "F ^ *F": the
> Lagrangian for the 3-form analog of Maxwell's equations.
$$
    

Now, it's time for me to answer one of your old questions! You seem to be
ready to hear the answer (you see, I never forget...).
Why should there be a 5-form in M-theory?
You nicely have replaced F^2 by F/\*F. Cool! Now, we can go further.
A is a 3-form, so F is a 4-form, then *F is a 11-4=7-form, then it should be
the field strength tensor of some 6-form potential, dA_(6)=*F, But a 6-form is
perfectly suited to be integrated over a 6-dimensional world-volume, i.e. a
5-brane! Here comes the M5-brane into the play.
Of course, in 11D SUGRA, the membrane is the fundamental object and the
M5-brane is a solitonic solution, but in a non-perturbative theory, solitonic
solutions can become fundamental at strong coupling and vice-versa. That's 
why we expect that the M5-brane will play an important role in M-theory.

The other question was what this had to do with the theory of Smolin?

In the BFSS matrix model, there is only one kind of objects, matrix-valued
1-forms (D0-branes).

These have a nice interpretation in terms of M2-branes (that's how modern-day
physicists write membranes...:->) wrapped on the two light-cone coordinates,
but what is the role of M5-branes in this game is unclear.
While in the matrix model proposed by Smolin in hep-th/0002009, there are more
terms involving also a 4-form, which might be related with a wrapped M5-brane.
This raises the hope that this matrix model might be a better try for a
non-perturbative version of M-theory than the usual BFSS one. But this has to
be investigated in more detail, of course; that's more or less what I'm doing
now.


\begin{verbatim}

> Second, why is it "reasonable" to assume that there aren't massless
> particles with spin greater than two?  Because it's physics folklore
> that quantum field theories with such particles are bad, nasty and evil
> - in fact, so evil that nobody even dares explain why!  Well, actually
> there's a paper by Witten in the above book that contains references to
> papers that supposedly explain why particles of spin > 2 are bad.  It's
> an excellent paper, too:
>
> 2) Edward Witten, Search for a realistic Kaluza-Klein theory, Nucl.
> Phys. B186 (1981), 412-428.
\end{verbatim}
    

I'm not a specialist of this, but higher spins involve the representation
theory of W-algebras, which can hardly be described as easy. Of course, 
that's not an argument, but I think that this has prevented many physicists 
from pursuing the matter too far.


\begin{verbatim}

> Unfortunately, this book does not cover supergravity theories.
\end{verbatim}
    

As a matter of fact, there are some books on supergravity in 4D, but no books
covering higher-dimensional supergravity theories with a reasonable amount of
explanations.

Of course, people really able to do this properly are a handful on this
planet, and even for them, this would require an enormous amount of work to
get things consistent all the way with a coherent choice of conventions and
check all the horrible formulas. On the other hand, when you hear their talks,
you usually don't get the feeling that they really want you to understand it,
but rather that they try to hide the truth about SUGRA in a well-hidden
"grimoire", maybe somewhere in Wizard's castle.
I hope some other people can shed more light on the subject, for example on
the supercovariantization of the spin connection (which I don't understand
very deeply, neither), maybe Aaron?

In any case, best regards to everyone,
and thanks John for the "This Week's Finds".

Maxime
\end{quote}

And here is one by Robert Helling:

\begin{quote}
John Baez wrote, concerning 11d supergravity:


\begin{verbatim}

 >I knew that people <em>thought</em> it wasn't renormalizable - that's not 
 >very new - but I didn't know people had become <em>sure</em> about it.
\end{verbatim}
    

Well, it depends a bit on your definition of "non-renormalizable". In a strict
sense, it means that renormalization would require an _infinite_ number
of different counter terms. In order to fix all their coefficients
one would have to do an infinite number of experiments before the
theory becomes predictive. This should be compared to renormalizable
theories that get along with a finite number although their coefficients
have to be adopted a each order of pertubation theory. Better are
superrenormalizable theories that also have a finite number of counter
terms but there coefficients are not changed after some order in
pertubation theory. 

The status of supergravity is as follows (in my understanding): Long ago
(what you refer to as thought) people figured out an additional
term in the action that might appears as counter term and that is 
invariant under all symmetries of the action (well, in 11d not
all symmetries, the full supermultiplet is not known and is expected
to be infinite but with _fixed_ relative coefficients. So there
is still just one parameter). E.g. in 4D, the situation is simpler because
there a superspace formulation is at hand that allows you to write expressions
that are automatically supersymmetric.

What people didn't know was whether this counter term really arises in
loop integrals. But now, in 11D Deser at al have calculated that a
certain combitation of four Riemann tensors appears as a counterterm
(has a non-zero coefficient) at 2 loop order.

This should be compared to Einstein's theory in 4D: There it was known
that a certain combination of two Weyl tensors does not vanish
by Bianchi identities or is topological. Therefore it is a possible
counterterm. 10 years ago, people did a 3 loop calculation (this is
really hard work!) to show that it actually arises. 4D sugra does not 
allow this term and its first possible counter term appears only at the 
next loop order. I know somebody personally that spend the last 10 years
doing this calcualtion and hasn't got very far (luckily he still has a 
job in physics).

But finding one counter term that was not in the classical action
does not show a theory is non-renormalizable (remember this is a
statement about infinitely many counter terms, so it is about
an infinity of orders of pertubation theory). It might just be
that this one term has been in the classical action just with coefficient
(coupling constant) 0 that is renormalized at higher orders. This
behaviour is highly unlikely but a mathematical possibility.

Actually showing a theory to be non-renormalizable is as hard
as showing a theory is renormalizable (not too long ago a 
Nobel prize was awarded for such a proof ;-))

Now for your point: "Is renormalizability a must?". I think it
is very old fashioned to give an affirmative answer to this 
question. A more modern answer would probably be: It's fine
for a theory to be non-renormalizable as long as it is only
an effective theory. Fermi \psi ^4 theory is not renormalizable
and is a nice theory of weak interactions as long as one
stays away from the EW breaking scale.

The appearance of the infinity of counter terms just shows
that there is some understanding of the high energy degrees
of freedom missing. And there will be a more fundamental
theory lurking around that reduces to this effective
theory for small energies.

So for a string theorist, non-renormalizability for sugra
is just fine: It's just the low energy effective theory
of string or M theory. It does not contain all degrees
of freedom, just the light ones. One way of thinking about
this is that string theory is just a fancy way of regulating
sugra. It supplies finite coefficients for the infinity
of possible counter terms. For example, in 10D sugra has
a one loop counterterm of the form R^4. This is just
an infinity in sugra. But in string theory, this has to be
a finite number, and in fact it is. It is

\zeta (3) = sum_n n^(-3).

The same thing is expected for 11D sugra and M-Theory. But
as long as nobody really knows what M-Theory really is this
does not help very much. 

Let me add a personal remark: In hep-th/9905183 we have
tried to do exactly this thing for M(atrix)-Theory, but
as it turned out, there are problems remaining.


\begin{verbatim}

 >>>Unfortunately, this book does not cover supergravity theories.

 >>As a matter of fact, there are some books on supergravity in 4D, 
 >>but no books covering higher-dimensional supergravity theories 
 >>with a reasonable amount of explanations.

 >I've noticed!  It's scandalous! 

 >>Of course, people really able to do this properly are a handful 
 >>on this planet, and even for them, this would require an enormous 
 >>amount of work to get things consistent all the way with a coherent 
 >>choice of conventions and check all the horrible formulas. 
\end{verbatim}
    

I know that at least three of the sugra hot shots of the eighties 
independently started such projects and there are sugra_book.tex files 
of various stages on their hard disks. They all gave up or made it a really 
long term project since they figured out that it would cost them years to 
basically redo all calculations in a coherent formalism.

This is just a horrible mess. Dealing with fermions just increases the pain. 
Doing a calculation twice you never get the same signs. I have already spend 
days figuring out what  + h.c.  in the stony brook textbook on 4D sugra meant 
(actually, it should have read - h.c. since what was computed was a anti-
hermitian quantity). They never stated what their conventions for hermititan 
conjugation are. Does it also reverse the order of differential operators? 
What about index positions (remember, for anticommuting variables 
\psi ^a \phi _a = - \psi _a \phi ^a) and all these kinds of things?

In addition, the old guys that have done many of the calcualtions use very 
strange (aka "convenient") conventions, like

\psi ^{2} = 1/2 \psi ^{a} \psi _{a}


or they raise and lower SL(2,C) not with the \epsilon  tensor, but with i times 
the \epsilon  tensor (relate this to h.c.!) This is just a mess and you always 
get the feeling that you are wasting your time with such things but in the end
your calculations are not even reliable!

This was all 4D, but the horror starts in higher dimensions. There \gamma  matrix
algebra becomes interesting. Again there are N+1 conventions if N people 
work on something and you have to have hunderets of Fierz identities at hand.
I know a grad student that spend months working them out on a computer and 
thought it would be a good service to the community to write a paper like
"Gamma identities and Fierzing in diverse dimensions". This would probably
be like the PhysRep by Slansky and Lie algebra stuff. But his advisor told
him not to do that "This is your capital. Put it in your drawer and lock it.
Be sure, erverybody in the field has such a drawer!"

And this is why there will never be such a text.  But I heard people say that
working out for yourself that 11d sugra is indeed supersymmetric is a good
exercise. I have never done it.

Robert
\end{quote}
 



 \par\noindent\rule{\textwidth}{0.4pt}

% </A>
% </A>
% </A>
