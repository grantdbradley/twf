
% </A>
% </A>
% </A>
\week{March 11, 2000 }


\begin{quote}
     Paper in white the floor of the room, and rule it off in one-
     foot squares.  Down on one's hands and knees, write in the 
     first square a set of equations conceived as able to govern
     the physics of the universe.  Think more overnight.  Next day
     put a better set of equations into square two.  Invite one's most
     respected colleagues to contribue to other squares.  At the
     end of these labors, one has worked oneself out into the 
     doorway.  Stand up, look back on all those equations, some 
     perhaps more hopeful than others, raise one's finger commandingly,
     and give the order "Fly!"  Not one of those equations will put
     on wings, take off, or fly.  Yet the universe "flies".

     Some principle uniquely right and compelling must, when one 
     knows it, be also so compelling that it is clear the universe
     is built, and must be built, in such and such a way, and that
     it could not be otherwise.  But how can one disover that principle?

\end{quote}
John Wheeler was undoubtedly the author of these words, which appear
near the end of Misner, Thorne and Wheeler's textbook "Gravitation",
published in 1972.  Since then, more people than ever before in the
history of the world have tried their best to find this uniquely
compelling principle.  A lot of interesting ideas, but no success yet.

But what if Wheeler was wrong?  What if there is \emph{not} a uniquely
compelling principle or set of equations that governs our universe?  For
example, what if \emph{all} equations govern universes?  In other words, what
if all mathematical structures have just as much "physical
existence" (whatever that means!) as those describing our universe
do?  Many of them will not contain patterns we could call awareness or
intelligence, but some will, and these would be "seen from
within" as "universes" by their inhabitants.  In this
scenario, there's nothing special about \emph{our} universe except that we
happen to be in this one.

In other words: what if there is ultimately no difference between
mathematical possibility and physical existence?  This may seem crazy, but
personally I believe that most alternatives, when carefully pondered,
turn out to be even \emph{more} crazy.

Of course, it's fun to think about such ideas and difficult to get
anywhere with them.  But tonight when I was nosing around the web, I
found out that someone has already developed and published this idea:

1) Max Tegmark, Is the "theory of everything" merely the ultimate
ensemble theory?, Ann. Phys. 270 (1998), 1-51, preprint available
as <A HREF = "http://xxx.lanl.gov/abs/gr-qc/9704009">gr-qc/9704009</A>.

Max Tegmark, Which mathematical structure is isomorphic to the
universe?, <a href = "http://www.hep.upenn.edu/~max/toe.html">http://www.hep.upenn.edu/~max/toe.html</a>

3) Marcus Chown, Anything goes, New Scientist 158 (1998) 26-30, also
available at <a href = "http://www.hep.upenn.edu/~max/toe_press.html">http://www.hep.upenn.edu/~max/toe_press.html</a>

As far as I can tell, most of the time Max Tegmark is a perfectly
respectable physicist at the University of Pennsylvania; he works on 
the cosmic microwave background radiation, the large-scale structure
of the universe (superclusters and the like), and type 1A supernovae.
But he has written a fascinating paper on the above hypothesis, which
I urge you to read.  It's less far-out than you may think.

Okay, now on to quantum gravity.  Jan Ambjorn and Renate Loll have
teamed up to work on discrete models of spacetime geometry, with an
emphasis on the Lorentzian geometry of triangulated manifolds.  Much
more has been done over on the Riemannian side of things, so it's high
time to focus more energy on the physically realistic Lorentzian case.
Of course, if the metric is fixed you can often use a trick called "Wick
rotation" to turn results about quantum field theory on Riemannian
spacetime into results for Lorentzian spacetime.  But it's never been
clear that this works when the geometry of spacetime is a variable - and
quantized, for that matter.  So we need both more work on Wick rotation
in this context and also work that goes straight for the jugular: the
Lorentzian context.

Here are some of their papers:

4) J. Ambjorn, J. Correia, C. Kristjansen, and R. Loll, On the
relation between Euclidean and Lorentzian 2d quantum gravity, 
preprint avilable as <A HREF = "http://xxx.lanl.gov/abs/hep-th/9912267">hep-th/9912267</A>.

J. Ambjorn, J. Jurkiewicz and R. Loll, Lorentzian and Euclidean
quantum gravity - analytical and numerical results, preprint available
as <A HREF = "http://xxx.lanl.gov/abs/hep-th/0001124">hep-th/0001124</A>. 

J. Ambjorn, J. Jurkiewicz and R. Loll, A non-perturbative Lorentzian
path integral for gravity, preprint avilable as <A HREF = "http://xxx.lanl.gov/abs/hep-th/0002050">hep-th/0002050</A>.

The last paper is especially interesting to me, since it tackles the
problem of defining a path integral for 3+1-dimensional Lorentzian
quantum gravity.  They describe a path integral where you first slice 
spacetime like a salami using surfaces of constant time, and then pack 
each slice with simplices having edges with specified lengths - the 
edges being spacelike within each surface, and timelike when they go
from one surface to the next.  They allow the number of simplices in each
slice to be variable.  Actually they focus on the 2+1-dimensional case, 
but they say the 3+1-dimensional case works similarly, and I actually 
trust them enough to believe them about this - especially since nothing 
they do relies on the fact that 2+1-dimensional gravity lacks local 
degrees of freedom.  They can Wick-rotate this picture and get a time
evolution operator that's self-adjoint and positive, just like you'd
expect of an operator of the form exp(-tH).

Speaking of Wick rotations in quantum gravity, here's another paper
to think about:

5) Abhay Ashtekar, Donald Marolf, Jose Mourao and Thomas Thiemann,
Osterwalder-Schrader reconstruction and diffeomorphism invariance,
preprint available as 
<A HREF = "http://xxx.lanl.gov/abs/quant-ph/9904094">quant-ph/9904094</A>.

The Osterwalder-Shrader theorem is the result people use when they
want to \emph{rigorously} justify Wick rotations.  Here these authors 
generalize it so that it applies to a large class of background-free
field theories - perhaps even quantum gravity!  It turns out not to
be hard, once you go about it properly.  Quite a surprise.

I've been working with Ashtekar and Krasnov for a couple of years now 
on computing the entropy of black holes using loop quantum gravity.  I
talked about this in "<A HREF = "week112.html">week112</A>", right after we came out with a short
paper sketching the calculation.  Now we're almost done with the
detailed paper.  In the meantime, Ashtekar has written a couple of
pedagogical accounts explaining the basic idea.  I mentioned one he
wrote with Krasnov in "<A HREF = "week120.html">week120</A>", and here's another:

6) Abhay Ashtekar, Interface of general relativity, quantum physics and
statistical mechanics: some recent developments, to appear in Annalen
der Physik, preprint available as <A HREF = "http://xxx.lanl.gov/abs/gr-qc/9910101">gr-qc/9910101</A>.

Let me just quote the abstract - I can't bear to talk about this any
more until the actual paper is finished:

\begin{quote}
     The arena normally used in black holes thermodynamics was recently
     generalized to incorporate a broad class of physically interesting
     situations.  The key idea is to replace the notion of stationary
     event horizons by that of `isolated horizons.'  Unlike event
     horizons, isolated horizons can be located in a space-time
     quasi-locally. Furthermore, they need not be Killing horizons. In
     particular, a space-time representing a black hole which is itself
     in equilibrium, but whose exterior contains radiation, admits an
     isolated horizon.  In spite of this generality, the zeroth and
     first laws of black hole mechanics extend to isolated horizons.
     Furthermore, by carrying out a systematic, non-perturbative
     quantization, one can explore the quantum geometry of isolated
     horizons and account for their entropy from statistical mechanical
     considerations.  After a general introduction to black hole
     thermodynamics as a whole, these recent developments are briefly
     summarized.
\end{quote}
There have also been a number of papers working out the details
of the classical notion of "isolated horizon" - I've mentioned some 
already, but let me just list them all here:

7) Abhay Ashtekar, Alejandro Corichi, and Kirill Krasnov, 
Isolated horizons: the classical phase space, preprint available
as <A HREF = "http://xxx.lanl.gov/abs/gr-qc/9905089">gr-qc/9905089</A>.  

Abhay Ashtekar, Christopher Beetle, and Stephen Fairhurst, 
Mechanics of isolated horizons, Class. Quant. Grav. 17 (2000) 253-298,
preprint available as <A HREF = "http://xxx.lanl.gov/abs/gr-qc/9907068">gr-qc/9907068</A>.

Abhay Ashtekar and Alejandro Corichi, Laws governing isolated horizons:
inclusion of dilaton couplings, preprint available as <A HREF = "http://xxx.lanl.gov/abs/gr-qc/9910068">gr-qc/9910068</A>.

Jerzy Lewandowski, Space-times admitting isolated horizons, preprint
available as <A HREF = "http://xxx.lanl.gov/abs/gr-qc/9907058">gr-qc/9907058</A>. 

Lewandowski's paper is important because it gets serious about studying
rotating isolated horizons - this makes me feel a lot more optimistic
that we'll eventually be able to extend the entropy calculation to 
rotating black holes (so far it's just done for the nonrotating case).

Okay, now let me turn my attention to spin foams.  Last month, 
Reisenberger and Rovelli came out with a couple of papers that push
forward the general picture of spin foams as Feynman diagrams, 
generalizing the old work of Boulatov and Ooguri, and the newer
work of De Pietri et al.  Again, I'll just quote the abstracts....

8) Michael Reisenberger and Carlo Rovelli, Spin foams as Feynman diagrams,
preprint available as <A HREF = "http://xxx.lanl.gov/abs/gr-qc/0002083">gr-qc/0002083</A>.

\begin{quote}
     It has been recently shown that a certain non-topological spin foam
     model can be obtained from the Feynman expansion of a field theory
     over a group. The field theory defines a natural "sum over
     triangulations", which removes the cutoff on the number of
     degrees of freedom and restores full covariance. The resulting
     formulation is completely background independent: spacetime emerges
     as a Feynman diagram, as it did in the old two-dimensional matrix
     models. We show here that any spin foam model can be obtained from
     a field theory in this manner. We give the explicit form of the
     field theory action for an arbitrary spin foam model. In this way,
     any model can be naturally extended to a sum over triangulations. 
     More precisely, it is extended to a sum over 2-complexes.
\end{quote}
9) Michael Reisenberger and Carlo Rovelli, Spacetime as a Feynman diagram:
the connection formulation, preprint available as <A HREF = "http://xxx.lanl.gov/abs/gr-qc/0002095">gr-qc/0002095</A>.

\begin{quote}
     Spin foam models are the path integral counterparts to loop
     quantized canonical theories. In the last few years several spin
     foam models of gravity have been proposed, most of which live on
     finite simplicial lattice spacetime. The lattice truncates the
     presumably infinite set of gravitational degrees of freedom down to
     a finite set. Models that can accomodate an infinite set of degrees
     of freedom and that are independent of any background simplicial
     structure, or indeed any a priori spacetime topology, can be
     obtained from the lattice models by summing them over all lattice
     spacetimes. Here we show that this sum can be realized as the sum
     over Feynman diagrams of a quantum field theory living on a
     suitable group manifold, with each Feynman diagram defining a
     particular lattice spacetime. We give an explicit formula for the
     action of the field theory corresponding to any given spin foam
     model in a wide class which includes several gravity models. Such a
     field theory was recently found for a particular gravity model [De
     Pietri et al, <A HREF = "http://xxx.lanl.gov/abs/hep-th/9907154">hep-th/9907154</A>]. Our work generalizes this result as
     well as Boulatov's and Ooguri's models of three and four
     dimensional topological field theories, and ultimately the old
     matrix models of two dimensional systems with dynamical topology. A
     first version of our result has appeared in a companion paper
     [<A HREF = "http://xxx.lanl.gov/abs/gr-qc/0002083">gr-qc/0002083</A>]: here we present a new and more detailed derivation
     based on the connection formulation of the spin foam models.
\end{quote}

I'm completely biased, but I think this is the way to go in quantum
gravity... we need to think more about the Lorentzian side of things,
like Barrett and Crane have been doing, but these spin foam models are
so darn simple and elegant I can't help but think there's something
right about them - especially when you see the sum over triangulations
pop out automatically from the Feynman diagram expansion of the relevant
path integral.

There's also been some good work on the relation between canonical
quantum gravity and Vassiliev invariants.  The idea is to use this class
of knot invariants as a basis for a Hilbert space of diffeomorphism-
invariant states - a tempting alternative to the Hilbert space having
spin networks as a basis.  Maybe everything will start making sense 
when we see how these two choices fit together.  But anyway, these
papers tackle the crucial issue of the Hamiltonian constraint using
this Vassiliev approach, and get results startlingly similar to those
obtained by Thiemann using the spin network approach:

10) Cayetano Di Bartolo, Rodolfo Gambini, Jorge Griego, and Jorge
Pullin, Consistent canonical quantization of general relativity in the
space of Vassiliev invariants, preprint available as <A HREF = "http://xxx.lanl.gov/abs/gr-qc/9909063">gr-qc/9909063</A>.

Canonical quantum gravity in the Vassiliev invariants arena:
I. Kinematical structure, preprint available as <A HREF = "http://xxx.lanl.gov/abs/gr-qc/9911009">gr-qc/9911009</A>.

Canonical quantum gravity in the Vassiliev invariants arena: II. 
Constraints, habitats and consistency of the constraint algebra,
preprint available as <A HREF = "http://xxx.lanl.gov/abs/gr-qc/9911010">gr-qc/9911010</A>.

Finally, Martin Bojowald has written a couple of papers applying the
loop approach to quantum cosmology.  The idea is to apply loop
quantization to a "minisuperspace" - a phase space describing only those
solutions of general relativity that have a certain large symmetry
group.

11) Martin Bojowald, Loop Quantum Cosmology I: Kinematics, preprint
available as <A HREF = "http://xxx.lanl.gov/abs/gr-qc/9910103">gr-qc/9910103</A>.

Martin Bojowald, Loop Quantum Cosmology II: Volume Operators, 
<A HREF = "http://xxx.lanl.gov/abs/gr-qc/9910104">gr-qc/9910104</A>.








 \par\noindent\rule{\textwidth}{0.4pt}

% </A>
% </A>
% </A>
