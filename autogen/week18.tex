
% </A>
% </A>
% </A>
\week{eptember 11, 1993}

I will be resuming this series of articles this fall, though perhaps not
at a rate of one "Week" per week, as I'll be pretty busy.  For those of
you who haven't seen this series before, let me explain.  It's meant to
be a guide to some papers, mostly in preprint form, that I have found
interesting.  I should emphasize that it's an utterly personal and
biased selection - if more people did this sort of thing, we might get a
fairer sample, but I'll be unashamed in focussing on my own obsessions,
which these days lean towards quantum gravity, topological quantum field
theories, knot theory, and the like.  

Quite a pile of papers has built up over the summer, but I will start by
describing what I did over my summer vacation:

1) Strings, loops, knots, and gauge fields, by John Baez, preprint
available in LaTeX form as <A HREF = "http://xxx.lanl.gov/abs/hep-th/9309067">hep-th/9309067</A>, 34 pages.

When I tell layfolk that I'm working on the loop representation of
quantum gravity, and try to describe its relation to knot theory, I
usually say that in this approach one thinks of space, not as a
smoothly curved manifold (well, I try not to say "manifold"), but as a
bunch of knots linked up with each other.  If thy have been exposed to
physics popularizations they will usually ask me at this point if I'm
talking about superstring theory.  To which I used to respond, somewhat
annoyed, that no, it was quite different.   Superstring theory, I
explained, is a grandiose "theory of everything" that tries to describe
all known forces and particles, and lots more, too, as being vibrating
loops of string hurling around in 349-dimensional space.  (Well, maybe
just 10, or 26.)  It is a complicated mishmash of all previous failed
approaches to unifying gravity with the other forces: Yang-Mills theory,
Kaluza-Klein models, strings, and supersymmetry.  (The last is a
symmetry principle that postulates for every particle another one, a
mysterious "superpartner," despite the fact that no such superpartners
have been seen.)  And it has made no testable predictions as of yet.
The loop representation of quantum gravity, on the other hand, is a much
more conservative project.  It simply attempts to use some new
mathematics to reconcile two theories which both seem true, but up to
now have been as immiscible as oil and water: quantum field theory, and
general relativity.  If it works, it will still be only the first step
towards uniftying gravity with the other forces.  If the questioner has
the gall to ask if \emph{it} has made any testable predictions, I say that so
far it is essentially a mathematics project.  On the one hand, here are
Einstein's equations; on the other hand, here are the rules of thumb for
"quantizing" some equations.  Is there a consistent and elegant way of
applying those rules to those equations?   People have tried for 40
years or so without real success, but quite possibly they just weren't
being clever enough, since the rules of thumb leave a lot of scope for
creativity.  Then a physicist named Ashtekar came along and reformulated
Einstein's equations using some new variables (usually known by experts
as the "new variables").  This made the equations look much more like
those that describe the other forces in physics.  This led to renewed
hope that Einstein's equations might be consistently quantized after
all.  Then physicists named Rovelli and Smolin , working with Ashtekar,
made yet \emph{another} change of variables, based on the new variables.
Rovelli and Smolin's variables were labelled by loops in space, so they
are called the loop variables.  These loops are quite unlike strings,
since they are merely mathematical artifacts for playing with Einstein's
equations, not actual little \emph{objects} whizzing about.  But using them,
Rovelli and Smolin were able to quantize Einstein's equations and
actually find a lot of solutions!  However, they were making up a lot of
new mathematics as they went along, and, as usual in theoretical
physics, it wasn't 100% rigorous (which, as we know, is like the the
woman who could trace her descent from William the Conqueror with only
two gaps).  So I, as a mathematician, got interested in this and am
trying to help out and see how much of this apparently wonderful
development is for real....

The odd thing is that there are a lot of mathematical connections
between string theory and the loop representation.  Gradually, as time
went on, I became more and more convinced that maybe the layfolk were
right - maybe the loop representation of quantum gravity really WAS
string theory in disguise, or vice versa.  This made a little embarassed
by how much I had been making fun of string theory.  Still, it could be
a very good thing.  On the one hand, the loop representation of quantum
gravity is much more well-motivated from basic physical principles than
string theory - it's not as baroque - a point I still adhere to.  So
maybe one could use it to understand string theory a lot more clearly.
On the other hand, string theory really attempts to explain, not just
gravity, but a whole lot more - so maybe it might help people see what
the loop representation of quantum gravity has to do with the other
forces and particles (if in fact it actually works).  

I decided to write a paper about this, and as I did some research I was
intrigued to find more and more connections between the two approaches,
to the point where it is clear that while they are presently very
distinct, they come from the same root, historically speaking.  

Here's what I wound up saying:

\begin{quote}
The notion of a deep relationship between string theories
and gauge theories is far from new.  String theory first arose as a
model of hadron interactions.  Unfortunately this theory had a number of undesirable features; in particular, it predicted massless spin-2 particles.
It was soon supplanted by quantum chromodynamics (QCD), which models the
strong force by an SU(3) Yang-Mills field.   However, string
models continued to be popular as an approximation of the confining phase
of QCD.  Two quarks in a meson, for example, can be thought of as
connected by a string-like flux tube in which the gauge field is
concentrated, while an excitation of the gauge field alone can be
thought of as a looped flux tube.   This is essentially a modern
reincarnation of Faraday's notion of ``field lines,'' but it can be
formalized using the notion of Wilson loops.  If A denotes a classical
gauge field, or connection, a Wilson loop is simply the trace of the
holonomy of A around a loop in space.  If instead A denotes a quantized
gauge field, the Wilson loop may be reinterpreted as an operator on the
Hilbert space of states, and applying this operator to the vacuum state
one obtains a state in which the Yang-Mills analog of the electric field
flows around the loop.
In the late 1970's, Makeenko and Migdal, Nambu, Polyakov, and others
attempted to derive equations of string dynamics as an approximation to 
the Yang-Mills equation, using Wilson loops.   More recently, D. Gross
and others have been able to <em>exactly</em> reformulate Yang-Mills theory in
2-dimensional spacetime as a string theory by writing an asymptotic
series for the vacuum expectation values of Wilson loops as a sum over
maps from surfaces (the string worldsheet) to spacetime.   This
development raises the hope that other gauge theories might also be
isomorphic to string theories.  For example, recent work by Witten and
Periwal  suggests that Chern-Simons theory in 3 dimensions is also
equivalent to a string theory. 
String theory eventually became popular as a theory of everything
because the massless spin-2 particles it predicted could be interpreted
as the gravitons one obtains by quantizing the spacetime
metric perturbatively about a fixed ``background'' metric.  Since
string theory appears to avoid the renormalization problems in
perturbative quantum gravity, it is a strong candidate for a theory
unifying gravity with the other forces.  
However, while classical general relativity is an elegant
geometrical theory relying on no background structure for its formulation, 
it has proved difficult to describe string theory along these lines.
Typically one begins with a fixed background structure and writes down
a string field theory in terms of this; only afterwards can one
investigate its background independence.  The clarity of a manifestly
background-free approach to string theory would be highly desirable.  
On the other hand, attempts to formulate Yang-Mills theory in terms of
Wilson loops eventually led to a full-fledged ``loop representation'' of
gauge theories, thanks to the work of Gambini, Trias, and
others.   After Ashtekar formulated quantum gravity as a sort of gauge
theory using the ``new variables,''  Rovelli and Smolin were able to use
the loop representation to study quantum gravity nonperturbatively in a
manifestly background-free formalism.   While superficially
quite different from modern string theory, this approach to quantum gravity
has many points of similarity, thanks to its common origin.  In particular,
it uses the device of Wilson loops to construct a space of states
consisting of ``multiloop invariants,'' which assign an amplitude to any
collection of loops in space.  The resemblance of these states to
wavefunctions of a string field theory is striking.  It is natural,
therefore, to ask whether the loop representation of quantum gravity might
be a string theory in disguise - or vice versa.  
The present paper does not attempt a definitive answer to this question.
Rather, we begin by describing a general framework relating gauge theories and 
string theories, and then consider a variety of examples.  Our treatment of
examples is also meant to serve as a review of Yang-Mills theory in 2
dimensions and quantum gravity in 3 and 4 dimensions. 
\end{quote}
    

I should add that the sort of string theory I talk about in this paper
is fairly crude compared to that which afficionados of the subject
usually concern themselves with.  It treats strings only as maps from a
surface (the string worldsheet) into spacetime, and only cares about
such maps up to diffeomorphism, i.e., smooth change of coordinates.  In
most modern string theory the string worldsheet is equipped with more
geometrical structure (a conformal structure) - it looks locally like
the complex plane, so one can talk about holomorphic functions on it and
the like.  This is why string theorists are always muttering about
conformal field theory.  But the sort of string theory that Gross and others
(Taylor, Minahan, and Polychronakos, particularly) have been using to
describe 2d Yang-Mills theory does not require a conformal structure on
the string worldsheet, so it's at least \emph{possible} that more interesting
theories like 4d quantum gravity can be formulated as string theories
without reference to conformal structures.  (Of course, if one
integrates over all conformal structures, that's a way of referring to
conformal structures without actually picking one.)  I guess I'm
rambling on here a bit, but this is really the most mysterious point as
far as I'm concerned.  

One hint of what might be going on is as follows.
And here, I'm afraid, I will be quite technical.  As noted by Witten and
formalized by Moore, Seiberg, and Crane, a rational conformal field
theory gives rise to a particularly beautiful sort of category called a
modular tensor category.  This contains, as it were, the barest essence
of the theory.  Any modular tensor category gives rise in turn to a 3d
topological quantum field theory - examples of which are Chern-Simons
theory and quantum gravity in 3 dimensions.  And Crane and Frenkel have
shown (or perhaps it's fairer to say that if they ever finish their
paper they \emph{will} have shown) that the nicest modular tensor categories
give rise to braided tensor 2-categories, which should, if there be
justice, give 4d topological quantum field theories.  (For more
information on all these wonderful things - which no doubt seem utterly
intimidating to the uninitiated - check out previous "This Week's
Finds.")  Quantum gravity in 4 dimensions is presumably something
roughly of this sort, if it exists.  So what I'm hinting at, in brief,
is that a bunch of category theory may provide the links between
\emph{modern} string theory with its conformal fields and the loop
representation of quantum gravity.  This is not as outre as it may
appear.  The categories being discussed here are really just ways of
talking about \emph{symmetries} (see my stuff on <A HREF = 
"categories.html">categories</A> and <A HREF = "symmetries.html">
symmetries</A> for more on this).  As usual in physics, the clearest 
way to grasp the connection between two seemingly disparate problems is 
often by recognizing that they have the same symmetries.
<HR>

% </A>
% </A>
% </A>


% parser failed at source line 263
