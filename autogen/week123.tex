
% </A>
% </A>
% </A>
\week{September 19, 1998}

\begin{quote}
     It all started out as a joke.  Argument for argument's sake.  Alison
     and her infuriating heresies.  

     "A mathematical theorem," she'd proclaimed, "only becomes true when a 
     physical system tests it out: when the system's behaviour depends in 
     some way on the theorem being \emph{true} or \emph{false}.

     It was June 1994.  We were sitting in a small paved courtyard, 
     having just emerged from the final lecture in a one-semester
     course on the philosophy of mathematics - a bit of light relief
     from the hard grind of the real stuff.  We had fifteen minutes to
     to kill before meeting some friends for lunch.  It was a social
     conversation - verging on mild flirtation - nothing more.  Maybe
     there were demented academics, lurking in dark crypts somewhere,
     who held views on the nature of mathematical truth which they were
     willing to die for.  But were were twenty years old, and we \emph{knew}
     it was all angels on the head of a pin.  

     I said, "Physical systems don't create mathematics.  Nothing
     \emph{creates} mathematics - it's timeless.  All of number theory would
     still be exactly the same, even if the universe contained nothing
     but a single electron."

     Alison snorted.  "Yes, because even \emph{one electron}, 
     plus a space-time to put it in, needs all of quantum mechanics and all of
     general relativity - and all the mathematical infrastructure they
     entail.  One particle floating in a quantum vacuum needs half the
     major results of group theory, functional analysis, differential
     geometry - "

     "OK, OK!  I get the point.  But if that's the case... the events in
     the first picosecond after the Big Bang would have `constructed'
     every last mathematical truth required by \emph{any} physical system,
     all the way to the Big Cruch.  Once you've got the mathematics
     which underpins the Theory of Everything... that's it, that's all
     you ever need.  End of story."

     "But it's not.  To \emph{apply} the Theory of Everything to a particular
     system, you still need all the mathematics for dealing with <em>that
     system</em> - which could include results far beyond the mathematics 
     the TOE itself requires.  I mean, fifteen billion years after the
     Big Bang, someone can still come along and prove, say... Fermat's
     Last Theorem."  Andrew Wiles at Princeton had recently announced
     a proof of the famous conjecture, although his work was still being
     scrutinised by his colleagues, and the final verdict wasn't yet in.
     "Physics never needed \emph{that} before."

     I protested, "What do you mean, `before'?  Fermat's Last Theorem
     never has - and never will - have anything to do with any branch 
     of physics."

     Alison smiled sneakily.  "No \emph{branch}, no.  But only because the 
     class of physical systems whose behaviour depend on it is so 
     ludicrously specific: the brains of mathematicians who are trying
     to validate the Wiles proof."

     "Think about it.  Once you start trying to prove a theorem, then
     even if the mathematics is so `pure' that it has no relevance to
     any other object in the universe... you've just made it relevant
     to \emph{yourself}.  You have to choose \emph{some} physical process to test
     the theorem - whether you use a computer, or a pen and paper... or
     just close your eyes and shuffle \emph{neurotransmitters}.  There's no
     such thing as a proof which doesn't rely on physical events, and
     whether they're inside or outside your skull doesn't make them
     any less real."
\end{quote}
And this is just the beginning... the beginning of Greg Egan's tale of
an inconsistency in the axioms of arithmetic - a "topological defect"
left over in the fabric of mathematics, much like the cosmic strings or
monopoles hypothesized by certain physicists thinking about the early
universe - and the mathematicians who discover it and struggle to
prevent a large corporation from exploiting it for their own nefarious
purposes.  This is the title story of his new collection, "Luminous".

I should also mention his earlier collection of stories, named after a
sophisticated class of mind-altering nanotechnologies, the "axiomatics",
that affect specific beliefs of anyone who uses them:

1) Greg Egan, Axiomatic, Orion Books, 1995.

Greg Egan, Luminous, Orion Books, 1998.

Some of the stories in these volumes concern math and physics, such as
"The Planck Dive", about some far-future explorers who send copies of
themselves into a black hole to study quantum gravity firsthand.  One
nice thing about this story, from a pedant's perspective, is that Egan
actually works out a plausible scenario for meeting the technical
challenges involved - with the help of a little 23rd-century technology.
Another nice thing is the further exploration of a world in which
everyone has long been uploaded to virtual "scapes" and can easily
modify and copy themselves - a world familiar to readers of his novel
"Diaspora" (see "<A HREF = "week115.html">week115</A>").  But what I really like is that it's not
just a hard-science extravaganza; it's a meditation on mortality.  You
can never really know what it's like to cross an event horizon unless 
you do it....

Other stories focus on biotechnology and philosophical problems of
identity.   The latter sort will especially appeal to everyone who
liked this book:

2) Daniel C. Dennett and Douglas R. Hofstadter, The Mind's I: Fantasies
and Reflections on Self and Soul, Bantam Books, 1982.

Among these, one of my favorite is called "Closer".  How close can you
be to someone without actually \emph{being them}?  Would temporarily merging
identities with someone you loved help you understand them better?
Luckily for you penny-pinchers out there, this particular story is
available free at the following website:

3) Greg Egan, Closer, <A HREF = "http://www.eidolon.net/old_site/issue_09/09_closr.htm">http://www.eidolon.net/old_site/issue_09/09_closr.htm</A>

Whoops!  I'm drifting pretty far from mathematical physics, aren't I?
Self-reference has a lot to do with mathematical logic, but....  To
gently drift back, let me point out that Egan has a website in which he
explains special and general relativity in a nice, nontechnical way:

4) Greg Egan, Foundations, 
<A HREF = "http://www.netspace.net.au/~gregegan/FOUNDATIONS/index.html">http://www.netspace.net.au/~gregegan/FOUNDATIONS/index.html</A>

Also, here are some interesting papers:

5) Gordon L. Kane, Experimental evidence for more dimensions reported,
Physics Today, May 1998, 13-16.

Paul M. Grant, Researchers find extraordinarily high temperature
superconductivity in bio-inspired nanopolymer, Physics Today, May
1998, 17-19.

Jack Watrous, Ribosomal robotics approaches critical experiments;
government agencies watch with mixed interest, Physics Today, May
1998, 21-23.

What these papers have in common is that they are all works of science
fiction, not science.  They read superficially like straight science
reporting, but they are actually the winners of Physics Today's "Physics
Tomorrow" essay contest!

For example, Grant writes:

"Little's concept involved replacing the phonons - characterized by the
Debye temperature - with excitons, whose much higher characteristic
energies are on the order of 2 eV, or 23,000 K.  If excitons were to
become the electron-pairing `glue', superconductors with T_c's as high
as 500 K might be possible, even under weak coupling conditions.  Little
even proposed a possible realization of the idea: a structure composed
of a conjugated polymer chain (polyene) dressed with highly polarizable
molecule (aromatics) as side groups.  Simply stated, the polyene chain
would be a normal metal with a single mobile electron per C-H molecular
unit; electrons on separate units would be paired by interacting with
the exciton field on the polarizable side groups."

Actually, I think this part is perfectly true - William A. Little
suggested this way to achieve high-temperature superconductivity back in
the 1960s.  The science fiction part is just the description, later on
in Grant's article, of how Little's dream is actually achieved.

Okay, enough science fiction!  Time for some real science!  Quantum
gravity, that is.  (Stop snickering, you skeptics....)

6) Laurent Freidel and Kirill Krasnov, Spin foam models and the 
classical action principle, preprint available as <A HREF = "http://xxx.lanl.gov/abs/hep-th/9807092">hep-th/9807092</A>.

I described the spin foam approach to quantum gravity in "<A HREF = "week113.html">week113</A>".  But
let me remind you how the basic idea goes.  A good way to get a handle
on this idea is by analogy with Feynman diagrams.  In ordinary quantum
field theory there is a Hilbert space of states called "Fock space".
This space has a basis of states in which there are a specific number of
particles at specific positions.  We can visualize such a state simply
by imagining a bunch of points in space, with labels to say which
particles are which kinds: electrons, quarks, and so on.  One of the
main jobs of quantum field theory is to let us compute the amplitude for
one such state to evolve into another as time passes.  Feynman showed
that we can do it by computing a sum over graphs in spacetime.  These
graphs are called Feynman diagrams, and they represent "histories".  For example,

\begin{verbatim}

\u       e/
 \       /
  \__W__/
  /     \
 /       \
/d      \nu \
\end{verbatim}
    
would represent a history in which an up quark emits a W boson and turns
into a down quark, with the W being absorbed by an electron, turning it
into a neutrino.  Time passes as you march down the page.  Quantum field
theory gives you rules for computing amplitudes for any Feyman diagram.
You sum these amplitudes over all Feynman diagrams starting at one state
and ending at another to get the total amplitude for the given transition
to occur.

Now, where do these rules for computing Feynman diagram amplitudes
come from?  They are not simply postulated.  They come from perturbation
theory.  There is a general abstract formula for computing amplitudes in
quantum field theory, but it's not so easy to use this formula in
concrete calculations, except for certain very simple field theories
called "free theories".  These theories describe particles
that don't interact at all.  They are mathematically tractable but
physically uninteresting.  Uninteresting, that is, \emph{except} as a
starting-point for studying the theories we \emph{are} interested in -
the so-called "interacting theories".

The trick is to think of an interacting theory as containing parameters,
called "coupling constants", which when set to zero make it reduce
to a free theory.  Then we can try to expand the transition amplitudes
we want to know as a Taylor series in these parameters.  As usual,
computing the coefficients of the Taylor series only requires us to to
compute a bunch of derivatives.  And we can compute these derivatives
using the free theory!  Typically, computing the nth derivative of some
transition amplitude gives us a bunch of integrals which correspond to
Feynman diagrams with n vertices.   

By the way, this means you have to take the particles you see in Feynman
diagrams with a grain of salt.  They don't arise purely from the
mathematics of the interacting theory.  They arise when we \emph{approximate}
that theory by a free theory.  This is not an idle point, because we can
take the same interacting theory and approximate it by \emph{different} free
theories.  Depending on what free theory we use, we may say different
things about which particles our interacting theory describes!  In
condensed matter physics, people sometimes use the term
"quasiparticle" to describe a particle that appears in a free
theory that happens to be handy for some problem or other.  For example,
it can be helpful to describe vibrations in a crystal using
"phonons", or waves of tilted electron spins using
"spinons".  Condensed matter theorists rarely worry about
whether these particles "really exist".  The question of
whether they "really exist" is less interesting than the
question of whether the particular free theory they inhabit provides a
good approximation for dealing with a certain problem.  Particle
physicists, too, have increasingly come to recognize that we shouldn't
worry too much about which elementary particles "really
exist".

But I digress!  My point was simply to say that Feynman diagrams arise
from approximating interacting theories by free theories.  The details
are complicated and in most cases nobody has ever succeeded in making
them mathematically rigorous, but I don't want to go into that here.
Instead, I want to turn to spin foams.

Everything I said about Feynman diagrams has an analogy in this approach
to quantum gravity.  The big difference is that ordinary "free
theories" are formulated on a spacetime with a fixed metric -
usually Minkowski spacetime, with its usual flat metric.  Attempts to
approximate quantum gravity by this sort of free theory failed dismally.
Perhaps the fundamental reason is that general relativity doesn't
presume that spacetime has a fixed metric - au contraire, it's a theory
in which the metric is the main variable!

So the idea of Freidel and Krasnov is to approximate quantum graivty
with a very different sort of "free theory", one in which the metric is
a variable.  The theory they use is called "BF theory".  I said a lot
about BF theory in "<A HREF = "week36.html">week36</A>", but here the main point is simply that it's
a topological quantum field theory, or TQFT.  A TQFT is a quantum field
theory that does not presume a fixed metric, but of a very simple sort,
because it has no local degrees of freedom.  I very much like the idea that
a TQFT might serve as a novel sort of "free theory" for the purposes
of studying quantum gravity.  

Everything that Freidel and Krasnov do is reminscent of familiar quantum
field theory, but also very different, because their starting-point is
BF theory rather than a free theory of a traditional sort.  For example,
just as ordinary quantum field theory starts out with Fock space, in the
spin network approach to quantum gravity we start with a nice simple
Hilbert space of states.  But this space has a basis consisting, not of
collections of 0-dimensional particles sitting in space at specified
positions, but of 1-dimensional "spin networks" sitting in space.  (For
more on spin networks, see "<A HREF = "week55.html">week55</A>" and "<A HREF = "week110.html">week110</A>".)  And instead of
using 1-dimensional Feynman diagrams to compute transition amplitudes,
the idea is now to use 2-dimensional gadgets called "spin foams".  The
amplitudes for spin foams are easy to compute in BF theory, because there
are a lot of explicit formulas using the so-called "Kauffman bracket", which
is an easily computable invariant of spin networks.   So then the trick
is to use this technology to compute spin foam amplitudes for quantum
gravity.

Now, I shouldn't give you the wrong impression here.  There are lots of
serious problems and really basic open questions in this work, and the
whole thing could turn out to be fatally flawed somehow.  Nonetheless,
something seems right about it, so I find it very interesting.

Anyway, on to some other papers.  I'm afraid I don't have enough energy
for detailed descriptions, because I'm busy moving into a new house, so
I'll basically just point you at them....

7) Abhay Ashtekar, Alejandro Corichi and Jose A. Zapata,
Quantum theory of geometry III: Non-commutativity of Riemannian
structures, preprint available as <A HREF = "http://xxx.lanl.gov/abs/gr-qc/9806041">gr-qc/9806041</A>.

This is the long-awaited third part of a series giving a mathematically
rigorous formalism for interpreting spin network states as "quantum
3-geometries", that is, quantum states describing the metric on
3-dimensional space together with its extrinsic curvature (as it sits
inside 4-dimensional spacetime).  Here's the abstract:

"The basic framework for a systematic construction of a quantum
theory of Riemannian geometry was introduced recently. The quantum
versions of Riemannian structures - such as triad and area operators -
exhibit a non-commutativity.  At first sight, this feature is
surprising because it implies that the framework does not admit a
triad representation. To better understand this property and to
reconcile it with intuition, we analyze its origin in detail. In
particular, a careful study of the underlying phase space is made and
the feature is traced back to the classical theory; there is no
anomaly associated with quantization.  We also indicate why the
uncertainties associated with this non-commutativity become negligible
in the semi-classical regime."

In case you're wondering, the "triad" field is more or less what 
mathematicians would call a "frame field" or "soldering form" - and
it's the same as the "B" field in BF theory.  It encodes the information
about the metric in Ashtekar's formulation to general relativity.

Moving on to matters n-categorical, we have:

8) Andre Hirschowitz, Carlos Simpson, Descente pour les n-champs 
(Descent for n-stacks), approximately 240 pages, in French, preprint
available as math.AG/9807049.
<A HREF = "http://xxx.lanl.gov/abs/math.AG/9807049">math.AG/9807049</A>.


Apparently this provides a theory of "n-stacks", which are the
n-categorical generalization of sheaves.  Ever since Grothendieck's
600-page letter to Quillen (see "<A HREF =
"week35.html">week35</A>"), this has been the holy grail of
n-category theory.  Unfortunately I haven't mustered sufficient courage
to force my way through 240 pages of French, so I don't really know the
details!

For the following two n-category papers, exploring some themes close
to my heart, I'll just quote the abstracts:

9) Michael Batanin, Computads for finitary monads on globular sets,
preprint available at <A HREF = "http://www.ics.mq.edu.au/~mbatanin/papers.html">http://www.ics.mq.edu.au/~mbatanin/papers.html</A>

"This work arose as a reflection on the foundation of higher
dimensional category theory.  One of the main ingredients of any
proposed definition of weak n-category is the shape of diagrams
(pasting scheme) we accept to be composable. In a globular approach
[due to Batanin] each k-cell has a source and target (k-1)-cell.  In
the opetopic approach of Baez and Dolan and the multitopic approach of
Hermida, Makkai and Power each k-cell has a unique (k-1)-cell as
target and a whole (k-1)-dimensional pasting diagram as source.  In
the theory of strict n-categories both source and target may be a
general pasting diagram.

The globular approach being the simplest one seems too restrictive to
describe the combinatorics of higher dimensional compositions.  Yet, we
argue that this is a false impression. Moreover, we prove that this
approach is a basic one from which the other type of composable
diagrams may be derived.  One theorem proved here asserts that the
category of algebras of a finitary monad on the category of n-globular
sets is \emph{equivalent} to the category of algebras of an appropriate
monad on the special category (of computads) constructed from the data
of the original monad. In the case of the monad derived from the
universal contractible operad this result may be interpreted as the
equivalence of the definitions of weak n-categories (in the sense of
Batanin) based on the `globular' and general pasting diagrams.  It may
be also considered as the first step toward the proof of equivalence
of the different definitions of weak n-category.

We also develop a general theory of computads and investigate some
properties of the category of generalized computads.  It turned out,
that in a good situation this category is a topos (and even a presheaf
topos under some not very restrictive conditions, the property firstly
observed by S. Schanuel and reproved by A. Carboni and P. Johnstone
for 2-computads in the sense of Street)."

10) Tom Leinster, Structures in higher-dimensional category theory,
preprint available at <A HREF = "http://www.dpmms.cam.ac.uk/~leinster">http://www.dpmms.cam.ac.uk/~leinster</A>

"This is an exposition of some of the constructions which have arisen
in higher-dimensional category theory.  We start with a review of the
general theory of operads and multicategories.  Using this we give an
account of Batanin's definition of n-category; we also give an
informal definition in pictures.  Next we discuss Gray-categories and
their place in coherence problems.  Finally, we present various
constructions relevant to the opetopic definitions of n-category.

New material includes a suggestion for a definition of lax cubical
n-category; a characterization of small Gray-categories as the small
substructures of 2-Cat; a conjecture on coherence theorems in higher
dimensions; a construction of the category of trees and, more
generally, of n-pasting diagrams; and an analogue of the Baez-Dolan
slicing process in the general theory of operads."

Okay - now for something completely different.  In "<A HREF = "week122.html">week122</A>" I said how
Kreimer and Connes have teamed up to write a paper relating Hopf
algebras, renormalization, and noncommutative geometry.  Now it's out:

11) Alain Connes and Dirk Kreimer, Hopf algebras, renormalization and 
noncommutative geometry, preprint available as <A HREF = "http://xxx.lanl.gov/abs/hep-th/9808042">hep-th/9808042</A>.

Also, here's an introduction to Kreimer's work:

12) Dirk Kreimer, How useful can knot and number theory be for loop 
calculations?, Talk given at the workshop "Loops and Legs in Gauge 
Theories", preprint available as <A HREF = "http://xxx.lanl.gov/abs/hep-th/9807125">hep-th/9807125</A>.


 Switching over to homotopy theory and its offshoots... when I
visited Dan Christensen at Johns Hopkins this spring, he introduced me
to all the homotopy theorists there, and Jack Morava gave me a paper
which really indicates the extent to which new-fangled "quantum
topology" has interbred with good old- fashioned homotopy theory:

12) Jack Morava, Quantum generalized cohomology, preprint available
as 
<A HREF = "http://xxx.lanl.gov/abs/math.QA/9807058">math.QA/9807058</A>
and <A HREF = "http://hopf.math.purdue.edu/">http://hopf.math.purdue.edu/</A>

Again, I'll just quote the abstract rather than venturing my own
summary:

"We construct a ring structure on complex cobordism tensored with the
rationals, which is related to the usual ring structure as quantum
cohomology is related to ordinary cohomology. The resulting object
defines a generalized two- dimensional topological field theory taking
values in a category of spectra."

Finally, Morava has a student who gave me an interesting paper on 
operads and moduli spaces:

13) Satyan L. Devadoss, Tessellations of moduli spaces and the mosaic operad, 
preprint available as 
<A HREF = "http://xxx.lanl.gov/abs/math.QA/9807010">math.QA/9807010</A>.

"We construct a new (cyclic) operad of `mosaics' defined by polygons
with marked diagonals. Its underlying (aspherical) spaces are the sets
M_{0,n}(R) of real points of the moduli space of punctured
Riemann spheres, which are naturally tiled by Stasheff associahedra. 
We (combinatorially) describe them as iterated blow-ups and show that 
their fundamental groups form an operad with similarities to the operad 
of braid groups."


 \par\noindent\rule{\textwidth}{0.4pt}
\emph{Some things are so serious that one can only jest about them.} -
Niels Bohr.
\par\noindent\rule{\textwidth}{0.4pt}

% </A>
% </A>
% </A>
