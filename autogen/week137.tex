
% </A>
% </A>
% </A>
\week{September 4, 1999 }

Now I'm in Cambridge England, chilling out with the category theorists, 
so it makes sense for me to keep talking about category theory.  I'll 
start with some things people discussed at the conference in Coimbra 
(see last week).  
1) Michael Mueger, Galois theory for braided tensor categories 
and the modular closure, preprint available as <A HREF = "http://xxx.lanl.gov/abs/math.CT/9812040">math.CT/9812040</A>.

A braided monoidal category is simple algebraic gadget that captures
a bit of the essence of 3-dimensionality in its rawest form.  It
has a bunch of "objects" which we can draw a labelled dots like
this:
\begin{verbatim}
                          x
                          .   
\end{verbatim}
    
So far this is just 0-dimensional.  Next, given a bunch of objects 
we get a new object, their "tensor product", which we can draw by setting 
the dots side by side.  So, for example, we can draw x tensor y like this:

\begin{verbatim}
                        x  y
                        .  .
\end{verbatim}
    
This is 1-dimensional.  But in addition we have, for any pair of 
objects x and y, a bunch of "morphisms" f: x \to  y.  We can draw a
morphism from a tensor product of objects to some other tensor
product of objects as a picture like this:
\begin{verbatim}
                       x  y  z
                       |  |  |
                       |  |  |
                       |  |  |
                       -------
                      |   f   |
                       -------
                        |   |   
                        |   |   
                        |   |   
                        u   v   
\end{verbatim}
    
This picture is 2-dimensional.  In addition, we require that for any 
pair of objects x and y there is a "braiding", a special morphism 
from x tensor y to y tensor x.  We draw it like this:
\begin{verbatim}
                        x     y
                        |     |
                         \   /
                          \ /
                           /
                          / \
                         /   \
                        |     |
                        y     x
\end{verbatim}
    
With this crossing of strands, the picture has become 3-dimensional!

We also require that we can "compose" a morphism f: x \to  y and a morphism
g: y \to  z and get a morphism fg: x \to  z.  We draw this by sticking one 
picture on top of each other like this... I'll draw a fancy example where
all the objects in question are themselves tensor products of other objects:
\begin{verbatim}
                       x  y  z
                       |  |  |
                       |  |  |
                       |  |  |
                       -------
                      |   f   |
                       -------
                        |   |
                        |   |
                        |   |  
                       -------
                      |   g   |
                       -------
                       | | | |  
                       | | | |  
                       | | | |  
                       a b c d
\end{verbatim}
    
Finally, we require that the tensor product, braiding and composition
satisfy a bunch of axioms.  I won't write these down because I already
did so in "week121", but the point is that they all make geometrical
sense - or more precisely, topological sense - in terms of the above 
pictures. 

The pictures I've drawn should make you think about knots and tangles 
and circuit diagrams and Feynman diagrams and all sorts of things
like that - and it's true, you can understand all these things very 
elegantly in terms of braided monoidal categories!  Sometimes it's
nice to throw in another rule:
\begin{verbatim}
                  x     y                x     y
                  |     |                |     |
                   \   /                  \   /
                    \ /                    \ /
                     /          =           \
                    / \                    / \
                   /   \                  /   \
                  |     |                |     |
                  y     x                y     x
\end{verbatim}
    
where we cook up the second picture using the inverse of the braiding.
This rule is good when you don't care about the difference between
overcrossings and undercrossings.  If this rule holds we say our 
braided monoidal category is "symmetric".  Topologically, this rule 
makes sense when we study 4-dimensional or higher-dimensional situations,
where we have enough room to untie all knots.  For example, the
traditional theory of Feynman diagrams is based on symmetric monoidal
categories (like the category of representations of the Poincare
group), and it works very smoothly in 4-dimensional spacetime.
But 3-dimensional spacetime is a bit different.  For example,
when we interchange two identical particles, it really makes
a difference whether we do it like this:
\begin{verbatim}
                  x     y               
                  |     |         
                   \   /         
                    \ /         
                     /         
                    / \       
                   /   \     
                  |     |   
                  y     x               
\end{verbatim}
    
or like this:
\begin{verbatim}
                  x     y               
                  |     |         
                   \   /         
                    \ /         
                     \         
                    / \       
                   /   \     
                  |     |   
                  y     x               
\end{verbatim}
    
Thus in 3d spacetime, besides bosons and fermions, we have other sorts
of particles that act differently when we interchange them - sometimes
people call them "anyons", and sometimes people talk about "exotic
statistics".  
Now let me dig into some more technical aspects of the picture.  

Starting with Reshetikhin and Turaev, people have figured out
how to use braided monoidal categories to construct topological 
quantum field theories in 3-dimensional spacetime.  But they can't 
do it starting from any old braided monoidal category, because
quantum field theory has a lot to do with Hilbert spaces.  So usually 
they start from a special sort called a "modular tensor category".  
This is a kind of hybrid of a braided monoidal category and a Hilbert 
space.  
In fact, apart from one technical condition - which is at the
heart of Mueger's work - we can get the definition of a modular
tensor category by taking the definition of "Hilbert space", 
categorifying it once to get the definition of "2-Hilbert 
space", and then throwing in a tensor product and braiding 
that are compatible with this structure.   
It's amazing that by such abstract conceptual methods we come
up with almost precisely what's needed to construct topological
quantum field theories in 3 dimensions!  It's a great illustration
of the power of category theory.  It's almost like getting something 
for nothing!  But I'll resist the temptation to tell you the details, 
since <a href = "week99.html">week99</a> 
explains a bunch of it, and the rest is in here:
2) John Baez, Higher-dimensional algebra II: 2-Hilbert spaces,
Adv. Math. 127 (1997), 125-189.   Also available as 
<A HREF = "http://xxx.lanl.gov/abs/q-alg/9609018">q-alg/9609018</A>.

In this paper I call a 2-Hilbert space with a compatible tensor
product a "2-H*-algebra", and if it also has a compatible braiding,
I call it a "braided 2-H*-algebra".  This terminology is bit
clunky, but for consistency I'll use it again here.   

Okay, great: we \emph{almost} get the definition of modular tensor
category by elegant conceptual methods.  But there is one
niggling but crucial technical condition that remains!  There
are lots of different ways to state this condition, but Mueger
proves they're equivalent to the following very elegant one.
Let's define the "center" of a braided monoidal category to
be the category consisting of all objects x such that

\begin{verbatim}
                  x     y                x     y
                  |     |                |     |
                   \   /                  \   /
                    \ /                    \ /
                     /          =           \
                    / \                    / \
                   /   \                  /   \
                  |     |                |     |
                  y     x                y     x
\end{verbatim}
    
for all y, and all morphisms between such objects.  The center 
of a braided monoidal category is obviously a symmetric monoidal 
category.  The term "center" is supposed to remind you of the 
usual center of a monoid - the elements that commute with all 
the others.  And indeed, both kinds of center are special cases 
of a general construction that pushes you down the columns of 
the "periodic table":
\begin{verbatim}
                   k-tuply monoidal n-categories

              n = 0           n = 1             n = 2


k = 0         sets          categories         2-categories
     

k = 1        monoids         monoidal           monoidal
                            categories        2-categories

k = 2       commutative      braided            braided
             monoids         monoidal           monoidal
                            categories        2-categories 

k = 3         " "           symmetric            weakly
                             monoidal          involutory
                            categories          monoidal
                                              2-categories

k = 4         " "             " "               strongly 
                                               involutory
                                                monoidal
                                              2-categories

k = 5         " "             " "                "  "
\end{verbatim}
    
I described this in "week74" and "week121", so I won't do so 
again.  My point here is really just that lots of this 
3-dimensional stuff is part of a bigger picture that applies 
to all different dimensions.  For more details, including a
description of the center construction, try:
3) John Baez and James Dolan, Categorification, in Higher
Category Theory, eds. Ezra Getzler and Mikhail Kapranov, 
Contemporary Mathematics vol. 230, AMS, Providence, 1998,
pp. 1-36.  Also available at <A HREF = "http://xxx.lanl.gov/abs/math.QA/9802029">math.QA/9802029</A>.
Anyway, Mueger's elegant characterization of a modular tensor
category amounts to this: it's a braided 2-H*-algebra whose 
center is "trivial".  This means that every object in the center
is a direct sum of copies of the object 1 - the unit for the
tensor product.  

Mueger does a lot more in his paper that I won't describe here,
and he also said a lot of interesting things in his talk about
the general concept of center.  For example, the center of a
monoidal category is a braided monoidal category.  In particular,
if you take the center of a 2-H*-algebra you get a braided
2-H*-algebra.  But what if you then take this braided 2-H*-algebra
and look at \emph{its} center?  Well, it turns out to be "trivial" in 
the above sense!

There's a bit of overlap between Mueger's paper and this one:
4) A. Bruguieres, Categories premodulaires, modularisations et 
invariants des varietes de dimension 3, preprint.  
One especially important issue they both touch upon is this:
if you have a braided 2-H*-algebra, is there any way to mess
with it slightly to get a modular tensor category?  The answer
is yes.  Thus we can really get a topological quantum field theory 
from any braided 2-H*-algebra.  But this raises another question:
can we describe this topological quantum field theory directly,
without using the modular tensor category?  The answer is again
yes!  For details see:

5) Stephen Sawin, Jones-Witten invariants for nonsimply-connected
Lie groups and the geometry of the Weyl alcove, preprint available
as <A HREF = "http://xxx.lanl.gov/abs/math.QA/9905010">math.QA/9905010</A>.
This paper uses this machinery to get topological quantum field
theories related to Chern-Simons theory.   People have thought about 
this a lot, ever since Reshetikhin and Turaev, but the really
great thing about this paper is that it handles the case when
the gauge group isn't simply-connected.  This introduces a lot
of subtleties which previous papers touched upon only superficially.
Sawin works it out much more thoroughly by an analysis of subsets
of the Weyl alcove that are closed under tensor product.  It's
very pretty, and reading it is very good exercise if you want to
learn more about representations of quantum groups.
Now, I said that a lot of this is part of a bigger picture that
works in higher dimensions.  However, a lot of this higher-dimensional
stuff remains very mysterious.  Here are two cool papers that make
some progress in unlocking these mysteries:
6) Marco Mackaay, Finite groups, spherical 2-categories, and 4-manifold
invariants, preprint available as 
<A HREF = "http://xxx.lanl.gov/abs/math.QA/9903003">math.QA/9903003</A>.
7) Mikhail Khovanov, A categorification of the Jones polynomial, 
preprint available as <A HREF = "http://xxx.lanl.gov/abs/math.QA/9908171">math.QA/9908171</A>.
Marco Mackaay spoke about his work in Coimbra, and I had grilled
him about it in Lisbon beforehand, so I think I understand it 
pretty well.  Basically what he's doing is categorifying the 
3-dimensional topological quantum field theories studied by Dijkgraaf 
and Witten to get 4-dimensional theories.  It fits in very nicely 
with his earlier work described in "week121".  
People have been trying to categorify the magic of quantum groups
for quite some time now, and Khovanov appears to have made a good
step in that direction by describing the Jones polynomial of a
link as the "graded Euler characteristic" of a chain complex of
graded vector spaces.  Since graded Euler characteristic is a
generalization of the dimension of a vector space, and taking the
dimension is a process of decategorification (i.e., vector spaces
are isomorphic iff they have the same dimension), Khovanov's 
chain complex can be thought of as a categorified version of the
Jones polynomial.  
I would like to understand better the relation between Khovanov's
work and the work of Crane and Frenkel on categorifying quantum
groups (see "week58").  For this, I guess I should read the 
following papers:
8) J. Bernstein, I. Frenkel and M. Khovanov, A categorification
of the Temperley-Lieb algebra and Schur quotients of U(sl_2)
by projective and Zuckerman functors, to appear in Selecta
Mathematica.
9) Mikhail Khovanov, Graphical calculus, canonical bases and
Kazhdan-Lusztig theory, Ph.D. thesis, Yale, 1997.

<p> <hr>

% </A>
% </A>
% </A>


% parser failed at source line 392
