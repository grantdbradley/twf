
% </A>
% </A>
% </A>
\week{June 5, 2000}

 

Last week I talked about some millennium-related books.  This week, some
millennial math problems!  In 1900, at the second International Congress
of Mathematicians, Hilbert posed a famous list of 23 problems.  No one
individual seems to have the guts to repeat that sort of challenge now. 
But the newly-founded Clay Mathematics Institute, based in Cambridge
Massachusetts and run by Arthur Jaffe, has just laid out a nice list of
7 problems:

1) Clay Mathematics Institute, Millennium Prize Problems, 
<a href = "http://www.claymath.org/prizeproblems/index.htm">http://www.claymath.org/prizeproblems/index.htm</a>

There is a 1 million dollar prize for each one!  Unlike most of Hilbert's 
problems, these weren't cooked up specially for the occasion: they have 
already proved their merit by resisting attack for some time.  

Here they are:

<H3>1. P = NP?</H3>


This is the newest problem on the list and the easiest
to explain. An algorithm is "polynomial-time" 
if the time it takes to run
is bounded by some polynomial in the length of the input data. This is
a crude but easily understood condition to decide whether an algorithm
is fast enough to be worth bothering with. A 
"nondeterministic polynomial-time"
algorithm is one that can \emph{check} a purported solution to a problem in
an amount of time bounded by some polynomial in the input data. All algorithms
in P are in NP, but how about the converse? Is P = NP? Stephen Cook posed
this problem in 1971 and it's still open. It seems unlikely to be true
- a good candidate for a counterexample is the problem of factoring integers
- but nobody has \emph{proved} that it's false. This is the most practical 
question
of the lot, because if the answer were "yes", 
there's a chance that one could use this result to quickly crack 
all the current best encryption schemes.

<h3> 2. The Poincar&eacute; conjecture</h3>

Spheres are among the most fundamental topological spaces, but spheres
hold many mysteries. For example: is every 3-dimensional manifold with
the same homotopy type as a 3-sphere actually homeomorphic to a 3-sphere?
Or for short: are homotopy 3-spheres really 3-spheres? Poincar&eacute; 
posed this
puzzle in 1904 shortly after he knocked down an easier conjecture of his
by finding 3-manifolds with the same homology groups as 3-spheres that
weren't really 3-spheres. The higher-dimensional analogues of Poincar&eacute;'s
question have all been settled in the affirmative - Smale, Stallings and
Wallace solved it in dimensions 5 and higher, and Freedman later solved
the subtler 4-dimensional case - but the 3-dimensional case is still unsolved.
This is an excellent illustration of a fact that may seem surprising at
first: many problems in topology are toughest in fairly low dimensions!
The reason is that there's less "maneuvering room". 
The last couple decades
have seen a burst of new ideas in low-dimensional topology - this has been
a theme of This Week's Finds ever since it started - but the 
Poincar&eacute; conjecture
remains uncracked.

<h3>3. The Birch-Swinnerton-Dyer conjecture</h3>
This is a conjecture about elliptic curves, and indirectly, number theory.
For a precise definition of an elliptic curve I'll refer you 
to "<A HREF = "week13.html">week13</A>" 
and "<A HREF = "week125.html">week125</A>", 
but basically, it's a torus-shaped surface described by
an algebraic equation like this: 

y^{2} = x^{3} + ax + b

Any elliptic curve
is naturally an abelian group, and the points on it with rational coordinates
form a finitely generated subgroup. When are there infinitely many such
rational points? In 1965, Birch and Swinnerton-Dyer conjectured a criterion
involving something called the "L-function" 
of the elliptic curve. The
L-function L(s) is an elegant encoding of how many solutions there are
to the above equation modulo p, where p is any prime. The Birch-Swinnerton-Dyer
conjecture says that L(1) = 0 if and only if the elliptic curve has infinitely
many rational points. More generally, it says that the order of the zero
of L(s) at s = 1 equals the rank of the group of rational points on the
elliptic curve (that is, the rank of the free abelian summand of this group.)
A solution to this conjecture would shed a lot of light on Diophantine
equations, one of which goes back to at least the 10th century - namely,
the problem of finding which integers appear as the areas of right triangles
all of whose sides have lengths equal to rational numbers.

<h3> 4. The Hodge conjecture</h3>

This question is about algebraic geometry and topology. A 
"projective nonsingular complex algebraic variety" 
is basically a compact smooth manifold described
by a bunch of homogeneous complex polynomial equations. Such a variety
always has even dimension, say 2n. We can take the DeRham cohomology of
such a variety and break it up into parts H^{p,q} 
labelled by pairs (p,q)
of integers between 0 and n, using the fact that every function is a sum
of a holomorphic and an antiholomorphic part. Sitting inside the DeRham
cohomology is the rational cohomology, The rational guys inside 
H^{p,p} are called "Hodge forms". 
By Poincar&eacute; duality any closed analytic subspace
of our variety defines a Hodge form - this sort of Hodge form is called
an algebraic cycle. The Hodge conjecture, posed in 1950 states: every Hodge
form is a rational linear linear combination of algebraic cycles. It's
saying that we can concretely realize a bunch of cohomology classes using
closed analytic subspaces sitting inside our variety.

<h3> 5. Existence and mass gap for Yang-Mills theory</h3>

One of the great open problems of modern mathematical physics is whether
the Standard Model of particle physics is mathematically consistent. It's
not even known whether "pure" Yang-Mills theory - uncoupled to 
fermions or the Higgs - is a well-defined quantum field theory with reasonable 
properties.  To make this question precise, people have formulated various 
axioms for a quantum field theory, like the so-called "Haag-Kastler 
axioms". The job
of constructive quantum field theory is to mathematically study questions
like whether we can construct Yang-Mills theory in such a way that it satisfies
these axioms. But one really wants to know more: at the very least, existence
of Yang-Mills theory coupled to fermions, together with a "mass 
gap" -
i.e., a nonzero minimum mass for the particles formed as bound states of
the theory (like protons are bound states of quarks).

<H3>6. Existence and smoothness for the Navier-Stokes equations</H3>
The Navier-Stokes equations are a set of partial differential equations
describing the flow of a viscous incompressible fluid. If you start out
with a nice smooth vector field describing the flow of some fluid, it will
often get complicated and twisty as turbulence develops. Nobody knows whether
the solution exists for all time, or whether it develops singularities
and becomes undefined after a while! In fact, numerical evidence hints
at the contrary. So one would like to know whether solutions exist for
all time and remain smooth - or at least find conditions under which this
is the case. Of course, the Navier-Stokes equations are only an approximation
to the actual behavior of fluids, since it idealizes them as a continuum
when they are actually made of molecules. But it's important to understand
whether and how the continuum approximation breaks down as turbulence develops.

<h3>7. The Riemann hypothesis</h3>
For Re(s) > 1 the Riemann zeta function is defined by 

\zeta (s) = 1/1^{s} + 1/2^{s} + 1/3^{s} + ....


But we can extend it by analytic continuation to most
of the complex plane - it has a pole at s = 1.
The zeta function has a bunch of zeros in the "critical strip" where
Re(s) is between 0 and 1. In 1859, Riemann conjectured that all such zeros
have real part equal to 1/2. This conjecture has lots of interesting 
ramifications
for things like the distribution of prime numbers. By now, more than a
billion zeros in the critical strip have been found to have real part 1/2;
it has also been shown that "most" such zeros have this property, but the
Riemann hypothesis remains open.

If you solve one of these conjectures and win a million dollars because
you read about it here on This Week's Finds, please put me in your will.

Okay, now on to some other stuff.

This week was good for me in two ways.  First of all, Ashtekar, Krasnov
and I finally finished a paper on black hole entropy that we've been
struggling away on for over 3 years.  I can't resist talking about this
paper at length, since it's such a relief to be done with it.  Second, 
the guru of n-category theory, Ross Street, visited Riverside and explained 
a bunch of cool stuff to James Dolan and me.  I may talk about this next
time.  

2) Abhay Ashtekar, John Baez and Kirill Krasnov, Quantum geometry of
isolated horizons and black hole entropy, preprint available at 
<a href = "http://xxx.lanl.gov/abs/gr-qc/0005126">gr-qc/0005126</a>
or at <a href = "http://math.ucr.edu/home/baez/black2.ps">http://math.ucr.edu/home/baez/black2.ps</a>

I explained an earlier version of this paper in "<A HREF = "week112.html">week112</A>", but now I
want to give a more technical explanation.  So:

The goal of this paper is to understand the geometry of black holes in a
way that takes quantum effects into account, using the techniques of
loop quantum gravity.  We do not consider the region near the singularity, 
which is poorly understood.  Instead, we focus on the geometry of the
event horizon, since we wish to compute the entropy of a black hole by
counting the microstates of its horizon.  

Perhaps I should say a word about why we want to do this.  As explained
in "<A HREF = "week111.html">week111</A>", Bekenstein and Hawking found a formula relating the
entropy S of a black hole to the area A of its event horizon.  It is
very simple:

                           S = A/4

in units where the speed of light, Newton's constant, Boltzmann's
constant and Planck's constant equal 1.  Now, in quantum statistical
mechanics, the entropy of a system in thermal equilibrium is roughly 
the logarithm of the number N of microstates it can occupy:

                           S = ln N.

This is exactly right when all the microstates have the same energy.
Thus we expect that a black hole of area A has about

                          N = exp(A/4)

microstates.  For a solar-mass black hole, that's about exp(10^76) 
microstates!   Any good theory of quantum gravity must explain what
these microstates are.  Since their number is related to the event
horizon's area, it is natural to guess that they're related to the
geometry of the event horizon.  But how?  

It's clear that everything will work perfectly if each little patch of
the event horizon with area 4 ln(2) has exactly 2 states.  I think
Wheeler was the first to take this seriously enough to propose a toy
model where each such patch stores one bit of information, making the
black hole into something sort of like an enormous hard drive:

3) John Wheeler, It from bit, in Sakharov Memorial Lecture on Physics,
Volume 2, eds. L. Keldysh and V. Feinberg, Nova Science, New York, 1992.

Of course, this idea sounds a bit nutty.  However, the quantum state of
a spinor contains exactly one bit of information, and loop quantum
gravity is based on the theory of spinors, so it's not as crazy as it
might seem....  Still, there are some, ahem, \emph{details} to be worked out! 

So let's work them out.

The first step is to understand the classical mechanics of a black hole
in a way that allows us to apply the techniques of loop quantum gravity.
In other words, we want to describe a classical phase space for our black
hole.   This step was done in a companion paper:

4) Abhay Ashtekar, Alejandro Corichi and Kirill Krasnov, Isolated 
horizons: the classical phase space, Advances in Theoretical and
Mathematical Physics 3 (2000), 418-471.   Preprint available at 
<a href = "http://xxx.lanl.gov/abs/gr-qc/9905089">gr-qc/9905089</a>.

The idea is to consider the region of spacetime outside the black hole
and assume that its boundary is a cylinder of the form R x
S^{2}.  We demand that this boundary is an "isolated
horizon" - crudely speaking, a surface that light cannot escape
from, with no matter or gravitational radiation falling in for the
stretch of time under consideration.  To make this concept precise we
need to impose some boundary conditions on the metric and other fields
at the horizon.  These are most elegantly described using Penrose's
spinor formalism for general relativity, as discussed in "<A HREF
= "week109.html">week109</A>".  With the help of these boundary
conditions, we can start with the usual Lagrangian for general
relativity, turn the crank, and work out a description of the phase
space for an isolated black hole.

If we temporarily ignore the presence of matter, a point in this phase 
space describes the metric and extrinsic curvature of space outside the
black hole at a given moment of time.   Technically, we do this using an
SU(2) connection A together with an su(2)-valued 2-form E.  You can
think of these as analogous to the vector potential and electric field
in electromagnetism.  As usual, they need to satisfy some constraints
coming from Einstein's equations for general relativity.  They also need
to satisfy boundary conditions coming from the definition of an isolated
horizon.  

Since the black hole is shaped like a ball, the boundary conditions hold
on a 2-sphere that I'll call the "horizon 2-sphere".   One thing the
boundary conditions say is that on the horizon 2-sphere, the SU(2)
connection A is completely determined by a U(1) connection, say W.  This
U(1) connection is really important, because it describes the intrinsic
geometry of the horizon 2-sphere.    Here's a good way to think about
it: first you restrict the spacetime metric to the horizon 2-sphere, and
then you work out the Levi-Civita connection of this metric on the
2-sphere.   Finally, since loop quantum gravity is based on the parallel
transport of spinors, you work out the corresponding connection for
spinors on the 2-sphere, which is a U(1) connection.   That's W!  

The boundary conditions also say that on the horizon 2-sphere, the E
field is proportional to the curvature of W.  So on the horizon
2-sphere, \emph{all} the fields are determined by W.  This is even true when
we take the presence of matter into account.  When we quantize, it'll be
the microstates of this field W that give rise to the black hole
entropy.  Since W is just a technical way of describing the shape of the
horizon 2-sphere, this means that the black hole entropy arises from the
many slightly different possible shapes that the horizon can have.

But I'm getting ahead of myself here!  We haven't quantized yet; we're
just talking about the classical phase space for an isolated black hole.

The most unusual feature of this phase space is that its symplectic
structure is a sum of two terms.  First, there is the usual integral
over space at a given time, which makes the E field canonically
conjugate to the A field away from the horizon 2-sphere.  But then there
is a boundary term: an integral over the horizon 2-sphere.   This gives
the geometry of the horizon a life of its own, which ultimately accounts
for the black hole entropy.   Not surprisingly, this boundary term
involves the U(1) connection W.  In fact, this boundary term is just
the symplectic structure for U(1) Chern-Simons theory on the 2-sphere!
It's the simplest thing you can write down:

\omega (\delta  W, \delta  W') = (k/2\pi ) &int; \delta  W ^ \delta  W'

Here \omega  is the U(1) Chern-Simons symplectic structure; we're
evaluating it on two tangent vectors to the space of U(1) connections 
on the 2-sphere, which we call \delta  W and \delta  W'.  These are
the same as 1-forms, so we can wedge them and integrate the result
over the 2-sphere.  The number k is some constant depending on the
area of the black hole... but more about that later!  


I guess this Chern-Simons stuff needs some background to fully
appreciate.  I have been talking about it for a long time here on This
Week's Finds.  The quantum version of Chern-Simons theory is a
3-dimensional quantum field theory that burst into prominence thanks to
Witten's work relating it to the Jones polynomial, which is an invariant
of knots.  At least heuristically, you can calculate the Jones
polynomial by doing a path integral in SU(2) Chern-Simons theory.  It
also turns out that Chern-Simons theory is deeply related to quantum
gravity in 3d spacetime.  For quite a while, various people have hoped
that Chern-Simons theory was important for quantum gravity in 4d
spacetime, too - see for example "<A HREF =
"week56.html">week56</A>" and "<A HREF =
"week57.html">week57</A>".  However, there have been serious
technical problems in most attempts to relate Chern-Simons theory to
physically realistic problems in 4d quantum gravity.  I think we may
finally be straightening out some of these problems!  But the ironic
twist is that we're using U(1) Chern-Simons theory, which is really very
simple compared to the sexier SU(2) version.  For example, U(1)
Chern-Simons theory also gives a knot invariant, but it's basically just
the self-linking number.  And the math of U(1) Chern-Simons theory goes
back to the 1800s - it's really just the mathematics of "theta
functions".

As a historical note, I should add that the really nice derivation of 
the Chern-Simons boundary term in the symplectic structure for isolated
black holes was found in a paper written \emph{after} the one I mentioned 
above:

5) Abhay Ashtekar, Chris Beetle and Steve Fairhurst, Mechanics of
isolated horizons, Class. and Quant. Gravity 17 (2000), 253-298.
Available at 
<a href = "http://xxx.lanl.gov/abs/gr-qc/9907068">gr-qc/9907068</a>.

Originally, everyone thought that to make the action differentiable as a
function of the fields, you had to add a boundary term to the usual
action for general relativity, and that this boundary term was
responsible for the boundary term in the symplectic structure.  This
seemed a bit ad hoc.  Of course, you need to differentiate the action to
get the field equations, so it's perfectly sensible to add an extra term
if that's what you need to make the action differentiable, but still you
wonder: where did the extra term COME FROM?

Luckily, Ashtekar and company eventually realized that while you \emph{can}
add an extra term to the action, you don't really \emph{need} to.  By
cleverly using the "isolated horizon" boundary conditions, you
can show that the usual action for general relativity is already
differentiable without any extra term, and that it yields the
Chern-Simons boundary term in the symplectic structure.

Okay: we've got a phase space for an isolated black hole, and we've
got the symplectic structure on this phase space.  Now what?

Well, now we should quantize this phase space!  It's a bit complicated,
but thanks to the two-part form of the symplectic structure, it
basically breaks up into two separate problems: quantizing the A field
and its canonical conjugate E outside the horizon 2-sphere, and
quantizing the W field on this 2-sphere.  The first problem is basically
just the usual problem of loop quantum gravity - people know a lot about
that.  The second problem is basically just quantizing U(1) Chern-Simons
theory - people know even \emph{more} about that!  But then you have to go
back and put the two pieces together.  For that, it's crucial that on
the horizon, the E field is proportional to the curvature of the 
connection W.    

So: what do quantum states in the resulting theory look like?  I'll
describe a basis of states for you....


Outside the black hole, they are described by spin networks.  I've
discussed these in "<A HREF = "week110.html">week110</A>" and
elsewhere, but let me just recall the basics.  A spin network is a graph
whose edges are labelled by irreducible representations of SU(2), or in
other words spins j = 0, 1/2, 1, and so on.  Their vertices are labelled
as well, but that doesn't concern us much here.  What matters more is
that the spin network edges can puncture the horizon 2-sphere.  And it
turns out that each puncture must be labelled with a number m chosen
from the set

                       {-j, -j+1, .... j-1, j}

These numbers m determine the state of the geometry of the horizon 
2-sphere. 

What do these numbers j and m really MEAN?  Well, they should be vaguely 
familiar if you've studied the quantum mechanics of angular momentum.  
The same math is at work here, but with a rather different interpretation.  
Spin network edges represent quantized flux lines of the gravitational 
E field.  When a spin network edge punctures the horizon 2-sphere, it
contributes \emph{area} to the 2-sphere: a spin-j edge contributes an area 
equal to 

                      8 \pi  \gamma  sqrt(j(j+1))

for some constant \gamma .


But due to the boundary conditions relating the E field to the curvature of 
the connection W, each spin network edge also contributes \emph{curvature} to 
the horizon 2-sphere.  In fact, this 2-sphere is flat except where a spin 
network edge punctures it; at the punctures it has cone-shaped singularities.  
You can form a cone by cutting out a wedge-shaped slice from a piece of 
paper and reattaching the two new edges, and the shape of this cone is 
described by the "deficit angle" - the angle of the wedge 
you removed.  In this black hole business, a puncture labelled by the 
number m gives a conical singularity with a deficit angle equal to 

                         4 \pi  m / k

where k is the same constant appearing in the formula for the Chern-
Simons symplectic structure. 

I guess now it's time to explain these mysterious constants!  First of 
all, \gamma  is an undetermined dimensionless constant, usually called 
the "Immirzi parameter" because it was first discovered by Fernando 
Barbero.   This parameter sets the scale at which area is quantized!  
Of course, the formula for the area contributed by a spin-j edge:

                      8 \pi  \gamma  sqrt(j(j+1))

also has a factor of the Planck area lurking in it, which you can't 
see because I've set c, G, and \hbar  to 1.  That's not surprising.  
What's surprising is the appearance of the Barbero-Immirzi parameter.  
So far, loop quantum gravity cannot predict the value of this parameter 
from first principles.

Secondly, the number k, called the "level" in Chern-Simons theory, is
given by

                      k = A / 4 \pi  \gamma 

Okay, that's all for my quick description of what we get when we quantize
the phase space for an isolated black hole.  I didn't explain how the 
quantization procedure actually \emph{works} - it involves all sorts of fun 
stuff about theta functions and so on.  I just told the final result.

Now for the entropy calculation.  Here we ask the following question: 
"given a black hole whose area is within \epsilon  of A, what is the 
logarithm of the number of microstates compatible with this area?"  
This should be the entropy of the black hole - and it won't depend
much on the number \epsilon , so long as its on the Planck scale.    

To calculate the entropy, first we work out all the ways to label
punctures by spins j so that the total area comes within \epsilon  of A.
For any way to do this, we then count the allowed ways to pick numbers
m describing the intrinsic curvature of the black hole surface.  Then
we sum these up and take the logarithm.  

What's the answer?  Well, I'll do the calculation for you now in a
really sloppy way, just to sketch how it goes.  To get as many ways
to pick the numbers m as possible, we should concentrate on states 
where most of the spins j labelling punctures equal 1/2.  If \emph{all} 
these spins equal 1/2, each puncture contributes an area

          8 \pi  \gamma  sqrt(j(j+1)) =  4 \pi  \gamma  \sqrt 3

to the horizon 2-sphere.  Since the total area is close to A, this 
means that there are about A/(4 \pi  \gamma  \sqrt 3) punctures.  Then
for each puncture we can pick m = -1/2 or m = 1/2.  This gives 

                     N = 2^{ A/4 \pi  \gamma  \sqrt 3}

ways to choose the m values.  If this were \emph{exactly} right, the entropy
of the black hole would be

               S = ln N = (ln 2 /4 \pi  \gamma  \sqrt 3) A


Believe it or not, this crude estimate asymptotically approaches the
correct answer as A approaches infinity.  In other words, when the
black hole is in its maximum-entropy state, the vast majority of the spin
network edges poking through the horizon are spin-1/2 edges.
            
So, what have we seen?  Well, we've seen that the black hole entropy
is (asymptotically!) proportional to the area, just like Bekenstein
and Hawking said.  That's good.  But we don't get the Bekenstein-Hawking
formula
                         S = A/4
because there is an undetermined parameter in our formula - the
Barbero-Immirzi parameter.  That's bad.  However, our answer will match
the Bekenstein-Hawking formula if we take
                   \gamma  = ln 2 / \pi  \sqrt 3
If we do this, we no longer have that annoying undetermined constant
floating around in loop quantum gravity.  In fact, we can say that we've
determined the "quantum of area" - the smallest possible unit
of area.  That's good.  And then it's almost true that in our model,
each little patch of the black hole horizon with area 4 ln(2) contains a
single bit of information - since a spin-1/2 puncture has area 4 ln(2),
and the angle deficit at a puncture labelled with spin 1/2 can take only
2 values, corresponding to m = -1/2 and m = 1/2.  Of course, there are also  
punctures labelled by higher values of j, but the j = 1/2 punctures 
dominate the count of the microstates.

Of course, one might object to this procedure on the following grounds:
"You've been ignoring matter thus far.  What if you include, say, 
electromagnetic fields in the game?  This will change the calculation,
and now you'll probably need a different value of \gamma  to match the
Bekenstein-Hawking result!"  

However, this is not true: we can redo the calculation including 
electromagnetism, and the same \gamma  works.  That's sort of nice.

There are a lot of interesting comparisons between our way of computing
black hole entropy and the ways its done in string theory, and a lot
of other things to say, too but for that, you'll have to read the paper... 
I'm worn out now!

\par\noindent\rule{\textwidth}{0.4pt}
\textbf{Addenda:}
I thank Herman Rubin and Lieven Marchand for some corrections of 
errors I made while describing the Riemann hypothesis and P = NP
conjecture.    I also thank J. Maurice Rojas for pointing out that
Steve Smale was an individual who \emph{did} have the guts to pose a list 
of math problems for the 21st century, back in 1998.  This appears in:

6) Stephen Smale, Mathematical problems for the next century,
Mathematical Intelligencer, 20 (1998), 7-15.  Also available 
in 
<a href = "http://www.cityu.edu.hk/ma/staff/smale/pap104.ps">Postscript</a>
and 
<a href = "http://www.cityu.edu.hk/ma/staff/smale/pap104.pdf">PDF</a> 
as item 104 on Smale's webpage,
<a href = "http://www.cityu.edu.hk/ma/staff/smale/bibliography.html">
http://www.cityu.edu.hk/ma/staff/smale/bibliography.html</a>

I believe this also appears in the book edited by Arnold mentioned
at the beginning of "<a href = "week147.html">week147</a>".




 \par\noindent\rule{\textwidth}{0.4pt}
<em>... for beginners engaging in research, a most difficult
feature to grasp is that of quality - that is, the depth of a
problem.  Sometimes authors work courageously and at length to
arrive at results which they believe to be significant and which
experts believe to be shallow. This can be explained by the analogy
of playing chess.  A master player can dispose of a beginner with
ease no matter how hard the latter tries.  The reason is that, even
though the beginner may have planned a good number of moves ahead, by
playing often the master has met many similar and deeper problems;
he has read standard works on various aspects of the game so that
he can recall many deeply analyzed positions.  This is the same in
mathematical research.  We have to play often with the masters
(that is, try to improve on the results of famous mathematicians);
we must learn the standard works of the game (that is, the "well-known"
results).  If we continue like this our progress becomes inevitable.</em> -
Hua Loo-Keng, Introduction to Number Theory

\par\noindent\rule{\textwidth}{0.4pt}

% </A>
% </A>
% </A>
