
% </A>
% </A>
% </A>
\week{August 19, 1997 }

This summer I've been hanging out in Cambridge Massachusetts, working
on quantum gravity and also having some fun.  Not so long ago I gave a
talk on cellular automata at Boston University, thanks to a kind
invitation from Bruce Boghosian, who is using cellular automata to
model cool stuff like emulsions:

1) Florian W. J. Weig, Peter V. Coveney, and Bruce M. Boghosian, Lattice-
gas simulations of minority-phase domain growth in binary immiscible and
ternary amphiphilic fluid, preprint available as 
<A HREF = "http://xxx.lanl.gov/abs/cond-mat/9705248">cond-mat/9705248</A>.

As you add more and more of an amphiphilic molecule (e.g. soap) to a
binary immiscible fluid (e.g. oil and water), the boundary layer likes
to grow in area.  This is why you wash your hands with soap.  There
are various phases depending on the concentrations of the three
substances - a "spongy" phase, a "droplet phase", and so on - and
it is very hard to figure out what is going on quantitatively using
analytical methods.  

Luckily, one can simulate this stuff using a cellular automaton!
Standard numerical methods for solving the Navier-Stokes equation tend
to outrun cellular automata when it comes to plain old hydrodynamics,
but with these fancy "ternary amphiphilic fluids", cellular automata
really seem to be the most practical way to study things - apart
from experiments, of course.  This is very heartwarming to me, since
like many people I've been fond of cellular automata ever after
learning of John Conway's game of Life, and I've always hoped they
could serve some practical purpose.

I spoke about the thesis of my student James Gilliam and a paper
we wrote together:

2) James Gilliam, Lagrangian and Symplectic Techniques in Discrete
Mechanics, Ph.D. thesis, Department of Mathematics, University of
Riverside, 1996.

John Baez and James Gilliam, An algebraic approach to discrete mechanics,
Lett. Math. Phys. 31 (1994), 205-212.   Also available in LaTeX form as
<A HREF = "http://math.ucr.edu/home/baez/ca.tex">http://math.ucr.edu./home/baez/ca.tex</A>

Here the idea was to set up as much as possible of the machinery of
classical mechanics in a purely discrete context, where time proceeds
in integer steps and the space of states is also discrete.  The most
famous examples of this "discrete mechanics" are cellular automata,
which are the discrete analogs of classical field theories, but there
are also simpler systems more reminiscent of elementary classical
mechanics, like a particle moving on a line - where in this case the
"line" is the integers rather than the real numbers.  It turns out
that with a little skullduggery one can apply the techniques of
calculus to some of these situations, and do all sorts of stuff like
prove a discrete version of Noether's theorem - the famous theorem
which gives conserved quantities from symmetries.

After giving this talk, I visited my friend Robert Kotiuga in the
Functorial Electromagnetic Analysis Lab in the Photonics Building at
Boston University.  "Photonics" is the currently fashionable term for
certain aspects of optics, particularly quantum optics.  As befits its
flashy name, the Photonics Building is brand new and full of gadgets
like a device that displays Maxwell's equations in moving lights when
you speak the words "Maxwell's equations" into an inconspicuous
microphone.  (It also knows other tricks.)  Robert told me about what
he's been doing lately with topology and finite-element methods for
solving magnetostatics problems - this blend of higbrow math and
practical engineering being the reason for the somewhat
tongue-in-cheek name of his office, inscribed soberly on a plaque
outside the door.

Like the topologist Raoul Bott, Kotiuga started in electrical
engineering at McGill University, and gradually realized how much
topology there is lurking in electrical circuit theory and Maxwell's
equations.  Apparently a paper of his was the first to cite Witten's
famous work on Chern-Simons theory - though presumably this is
merely a testament to the superiority of engineers over mathematicians
and physicists when it comes to rapid publication.  In fluid dynamics,
the integral of the following quantity

v . curl(v) 

(where v is the velocity vector field) is known as the "helicity
functional".  Kotiuga been studying applications of the same mathematical
object in the context of magnetostatics, namely

A . curl(A) 

where A is the magnetic vector potential.  It shows up in impedance
tomography, for example.  But in quantum field theory, a
generalization of this quantity to other forces is known as the
"Chern-Simons functional", and Witten's work on the 3-dimensional
field theory having this as its Lagrangian turned out to revolutionize
knot theory.  Personally, I'm mainly interested in the applications to
quantum gravity - see "<A HREF = "week56.html">week56</A>" for a bit about this.  Here are some
papers Kotiuga has written on the helicity functional, or what we
mathematicians would call "U(1) Chern-Simons theory":

3) P. R. Kotiuga, Metric dependent aspects of inverse problems and
functionals based helicity, Journal of Applied Physics, 70 (1993),
5437-5439.

Analysis of finite element matrices arising from discretizations of
helicity functionals, Journal of Applied Physics, 67 (1990),
5815-5817.

Helicity functionals and metric invariance in three dimensions, IEEE
Transactions on Magnetics, MAG-25 (1989), 2813-2815.

Variational principles for three-dimensional magnetostatics based on
helicity, Journal of Applied Physics, 63 (1988), 3360-3362.

Later Jon Doyle, a computer scientist at M.I.T. who had been to my
talk, invited me to a seminar at M.I.T. where I met Gerald Sussman,
who with Jack Wisdom has run the best long-term simulations of the
solar system, trying to settle the old question of whether the darn
thing is stable!  It turns out that the system is afflicted with 
chaos and can only be predicted with any certainty for about 4 
million years... though their simulation went out to 100 million.

Here are some fun facts: 1) They need to take general relativity into
account even for the orbit of Jupiter, which precesses about one
radian per billion years.  2) They take the asteroid belt into account
only as modification of the sun's quadrupole moment (which they also
use to model its oblateness).  3) The most worrisome thing about the
whole simulation - the most complicated and unpredictable aspect of
the whole solar system in terms of its gravitational effects on
everything else - is the Earth-Moon system, with its big tidal
effects.  4) The sun loses one Earth mass per 100 million years due to
radiation, and another quarter Earth mass due to solar wind.  5) The
first planet to go is Mercury!  In their simulations, it eventually
picks up energy through a resonance and drifts away.  

For more, try:

4) Gerald Jay Sussman and Jack Wisdom, Chaotic evolution of the solar system,
Science, 257, 3 July 1992. 

Gerald Jay Sussman and Jack Wisdom, Numerical evidence that the motion
of Pluto is chaotic, Science, 241, 22 July 1988. 

James Applegate, M. Douglas, Y. Gursel, Gerald Jay Sussman, Jack Wisdom,
The outer solar system for 200 million years, Astronomical Journal, 92, 
pp 176-194, July 1986, reprinted in Lecture Notes in Physics #267 --
Use of Supercomputers in Stellar Dynamics, Springer Verlag, 1986. 

James Applegate, M. Douglas, Y. Gursel, P Hunter, C. Seitz, Gerald Jay
Sussman, A digital orrery, in IEEE Transactions on Computers, C-34,
No. 9, pp. 822-831, September 1985, reprinted in Lecture Notes in
Physics #267, Springer Verlag, 1986. 

Meanwhile, I've also been trying to keep up with recent developments
in n-category theory.  Some readers of "This Week's Finds" have expressed
frustration with how I keep tantalizing all of you with the concept of
n-category without ever quite defining it.  The reason is that it's a
lot of work to write a nice exposition of this concept!  

However, I eventually got around to taking a shot at it, so now you can
read this:

5) John Baez, Introduction to n-categories, to appear in
7th Conference on Category Theory and Computer Science, eds.
E. Moggi and G. Rosolini, Springer Lecture Notes in
Computer Science vol. 1290, Springer, Berlin.  Preprint available
as <A HREF = "http://xxx.lanl.gov/abs/q-alg/9705009">q-alg/9705009</A> or
at <A HREF = "http://math.ucr.edu/home/baez/ncat.ps">http://math.ucr.edu/home/baez/ncat.ps</A>

There are different definitions of "weak n-category" out there now and
it will take a while of sorting through them to show a bunch are
equivalent and get the whole machinery running smoothly.  In the
above paper I mainly talk about the definition that James Dolan
and I came up with.  Here are some other new papers on this sort of
thing... I'll just list them with abstracts.

6) Claudio Hermida, Michael Makkai and John Power, On weak higher 
dimensional categories, 104 pages, preprint available at 
<A HREF =
"http://hypatia.dcs.qmw.ac.uk/authors/M/MakkaiM/papers/multitopicsets/">http://hypatia.dcs.qmw.ac.uk/authors/M/MakkaiM/papers/multitopicsets/</A>

\begin{quote}
Inspired by the concept of opetopic set introduced in a recent
paper by John C. Baez and James Dolan, we give a modified notion
called multitopic set.  The name reflects the fact that, whereas
the Baez/Dolan concept is based on operads, the one in this paper
is based on multicategories.  The concept of multicategory used here
is a mild generalization of the same-named notion introduced by
Joachim Lambek in 1969.  Opetopic sets and multitopic sets are both
intended as vehicles for concepts of weak higher dimensional category.
Baez and Dolan define weak n-categories as (n+1)-dimensional opetopic
sets satisfying certain properties.  The version intended here, 
multitopic n-category, is similarly related to multitopic sets.  
Multitopic n-categories are not described in the present paper;
they are to follow in a sequel.  The present paper gives complete details
of the definitions and basic properties of the concepts involved with 
multitopic sets.  The category of multitopes, analogs of the opetopes
of Baez and Dolan, is presented in full, and it is shown that the 
category of multitopic sets is equivalent to the category of set-
valued functors on the category of multitopes.
\end{quote}
    

7) Michael Batanin, Finitary monads on globular sets and notions of
computad they generate, available as postscript files at
<A HREF = "http://www-math.mpce.mq.edu.au/~mbatanin/papers.html">http://www-math.mpce.mq.edu.au/~mbatanin/papers.html</A>

\begin{quote}
Consider a finitary monad on the category of globular sets. We prove
that the category of its algebras is isomorphic to the category of
algebras of an appropriate monad on the special category (of
computads) constructed from the data of the initial monad. In the case
of the free n-category monad this definition coincides with R.
Street's definition of n-computad. In the case of a monad generated
by a higher operad this allows us to define a pasting operation in a
weak n-category. It may be also considered as the first step toward
the proof of equivalence of the different definitions of weak
n-categories.
\end{quote}
    

7) Carlos Simpson, Limits in n-categories, approximately 90 pages,
preprint available as 
<A HREF =
"http://xxx.lanl.gov/abs/alg-geom/9708010">alg-geom/9708010</A>.
\begin{quote}
We define notions of direct and inverse limits in an n-category.  We
prove that the (n+1)-category nCAT' of fibrant n-categories admits
direct and inverse limits. At the end we speculate (without proofs) on
some applications of the notion of limit, including homotopy fiber
product and homotopy coproduct for n-categories, the notion of
n-stack, representable functors, and finally on a somewhat different
note, a notion of relative Malcev completion of the higher homotopy at
a representation of the fundamental group.
\end{quote}
    

8) Sjoerd Crans, Generalized centers of braided and sylleptic
monoidal 2-categories, Adv. Math. 136 (1998), 183-223.

\begin{quote}
Recent developments in higher-dimensional algebra due to Kapranov and
Voevodsky, Day and Street, and Baez and Neuchl include definitions of
braided, sylleptic and symmetric monoidal 2-categories, and a center
construction for monoidal 2-categories which gives a braided monoidal
2-category. I give generalized center constructions for braided and
sylleptic monoidal 2-categories which give sylleptic and symmetric
monoidal 2-categories respectively, and I correct some errors in the
original center construction for monoidal 2-categories.
\end{quote}
    











 <HR>
<em> Time definitely repeats itself: <br>
 that's its only job. <br></em>
 Edward Dorn, Sirius in January


<HR>

% </A>
% </A>
% </A>


% parser failed at source line 347
