
% </A>
% </A>
% </A>
\week{February 8, 2002}



A team of astronomers has found evidence that a dwarf galaxy near the
Milky Way is surrounded by an enormous halo of dark matter, which may be
200 times heavier than all the stars in the galaxy itself:

1) Jan T. Kleyna, Mark I. Wilkinson, N. Wyn Evans and Gerard Gilmore,
First clear signature of an extended dark matter halo in the Draco 
dwarf spheroidal, Astrophysical Journal Letters 563 (2001), L115-118.
Also available at <A HREF = "http://xxx.lanl.gov/abs/astro-ph/0111329">astro-ph/0111329</A>.  

This just emphasizes a well-known fact: "dark matter" is one
of the biggest mysteries in physics today.  Unless we're mixed up, which
is always possible, most of the energy density of the universe is made
of some invisible stuff about which we know almost nothing!  To add
insult to injury, after dark matter the second biggest constituent of
the mass/energy appears to be "dark energy".  All other forms
of matter - mainly hydrogen - come a distant third.

Perhaps I should say a word about the difference between dark matter
and dark energy, since this is awfully confusing to the uninitiated.

The main reason people believe in "dark matter" is that galaxies and
clusters of galaxies seem to have a lot more mass than can be accounted
for by all the stuff we understand: stars, gas, and so forth. It's
fairly easy to measure this mass using gravity, by seeing how fast
things orbit around each other - stars around galaxies, or galaxies
around each other.  The hard part is guessing how much stuff is in the
galaxies.  Could there be lots of faint stars we don't see?  Black
holes, maybe?  People have thought about all sorts of possibilities, but
they just don't seem to add up.  So, people postulate mysterious extra
stuff: "dark matter".

"Dark energy", on the other hand, is basically just a fashionable name
for the cosmological constant: that is, the built-in energy density of
the vacuum.  Einstein noticed that you can tinker with general
relativity by making this nonzero, but only by making the pressure
nonzero too, and of opposite sign, but with exactly the same magnitude
in units where c = G = 1.  This is very different from normal matter - 
or even dark matter, as far as we can tell - where both the energy density
and pressure are positive.

This is important because the expansion of the universe is governed both
by energy density and pressure.  More precisely, a calculation using
general relativity shows that the expansion of the universe decelerates
at a rate proportional to the energy density plus 3 times the pressure.
(In case you're wondering, the number 3 comes from the fact that space
is 3-dimensional.)

If you think about what I've told you, this means that normal matter
makes the expansion decelerate - but a positive cosmological constant
makes the expansion \emph{accelerate}, since the effects of negative pressure
dominate those of positive energy density, thanks to that factor of 3.  

Starting around 1995, convincing evidence started to build up that the
expansion of the universe is accelerating.  The simplest way to explain
this is to posit a positive cosmological constant - or in other words,
dark energy!  

In case you're dreaming up alternative theories as I speak, let me 
assure you that hundreds of papers have been written about this subject,
probing all sorts of possibilities.  Perhaps the cosmological constant
isn't really constant: maybe the negative pressure is due to a new form
of matter called "quintessence".  Perhaps general relativity is wrong:
that's what people working on "modified Newtonian dynamics" believe.  
I don't have the energy or expertise to talk about all these ideas, so 
I'm just telling you the current conventional wisdom.  

But if dark matter really exists, what could it be? There are lots of
options.  It could be an excess of familiar stuff that's somehow slipped
through our bookkeeping, or MACHOs (massive compact halo objects), or
WIMPs (weakly interacting massive particles), or... something else!

If you're trying to figure out the mystery of dark matter, you should
first study all the hoops your theory must succesfully jump through.
Besides getting galaxies and clusters to rotate faster than they 
otherwise would, dark matter should collapse under its own gravity
early in the history of the universe.  Why?  Otherwise, people seem
unable to explain why galaxies formed as soon as they did!  In the
early universe, the ordinary matter was very hot gas.  The hotter a
ball of gas is, the bigger it must be before it collapses under its 
own gravity, since this happens when the escape velocity exceeds the 
average speed of the atoms.  Without something to help it out, it 
seems that ordinary matter in the early universe could not collapse 
under its own gravity to form galaxy-sized lumps, but only much bigger 
lumps.  But it seems galaxies formed quite early!  This dilemma would 
go away if there were "cold dark matter" which clumped up under its 
own gravitation early on, seeding galaxy formation.

The new observation of this dwarf galaxy is further evidence that cold
dark matter is real and plays an important role in galaxy formation.
There are in fact 9 "dwarf spheroidal galaxies" near the Milky Way; 
the one studied is about 250,000 light years away from us in the
constellation of Draco.  Many astronomers believe that big galaxies 
like ours were formed from the accretion of such dwarfs.

Physicists are actually doing experiments to look for dark matter.
Galaxy formation and everything else would work quite nicely if dark
matter consisted of some sort of weakly interacting massive particle
with a mass of about 100 GeV.  The dark matter density near us seems to
be roughly 5 x 10^{-24} grams per cubic centimeter, which would mean
about 3 WIMPs per thousand cubic centimeters.  That's not much, but 
since these WIMPs would be moving in random orbits in the gravitational 
potential well of
the galaxy, they should be zipping past us at an average of 300 kilometers
per second.  This gives a flux of about 10^{5} WIMPs per square 
centimeter per second!

The problem is that, like neutrinos, most of these guys would pass
through matter undetected.  If you pick some specific theory concerning
these WIMPs - for example that they're some sort of "neutralino" 
in the
minimal supersymmetric extension of the Standard Model - and make some
plausible assumptions about various numbers, you'd guess that about 10
WIMPs per year would interact with a 1-kilogram lump of matter.  Of
course the actual number could easily be many orders of magnitude
different, but the point is: this is within the realm of what we might
actually detect!

One way to go about it is to use sodium iodide crystals as scintillation
detectors.  When a WIMP smacks into one of these, it should emit a
flash of light.  The problem is to eliminate other causes such as 
cosmic rays and natural radioactivity from the surroundings.  To get 
away from cosmic rays, it's good to go down into a mine.  To get away 
from radioactivity it's good to use shielding made from high-purity 
copper or aged lead.  The UK Dark Matter Collaboration has done just 
this, placing several 1-10 kilogram sodium iodide crystals 1100 meters 
below ground in the Boulby salt mine in Yorkshire.  They've been taking
data since 1997, and they've seen a number of anomalous events:

2) UK Dark Matter Collaboration (UKDMC) homepage,
<A HREF = "http://hepwww.rl.ac.uk//UKDMC/">http://hepwww.rl.ac.uk//UKDMC/</A>

The DAMA group - that's short for "dark matter" - has found even more
fascinating results.  This collaboration involves Italian and Chinese
physicists who are using nine 9.7-kilogram sodium iodide crystals in a
laboratory 1400 meters below ground, off of a tunnel on a highway near
Rome.  The idea behind this experiment is not just to \emph{detect} WIMPs -
but to look for seasonal variations in the \emph{rate} of their detection!

This may sound crazy, but it's based on sound logic.  The sun orbits the
galaxy at 232 kilometers/second, but also the earth orbits the sun at 30
kilometers/second in a plane that lies at a 60-degree angle to the
galactic plane.  As a result the earth is going through the galaxy
faster when these motions add up, in June, than when they're pointing in
opposite directions, in December.  So, if WIMPs are more or less
randomly orbiting the galaxy in all directions, we should thus see a
higher flux of WIMPs through the earth in summer than in winter!  

The DAMA group has been collecting data for four years, and 
claims to have actually seen such a "annual modulation signature".
You can see a graph of their data here:

3) DAMA collaboration, Searching for the WIMP annual signature by the 
~100 kg NaI(Tl) set-up, <A HREF = "http://www.lngs.infn.it/lngs/htexts/dama/dama39.html">http://www.lngs.infn.it/lngs/htexts/dama/dama39.html</A>

For more information, try their homepage:

4) Dark Matter (DAMA) experiment home page,
<A HREF = "http://www.lngs.infn.it/lngs/htexts/dama/welcome.html">http://www.lngs.infn.it/lngs/htexts/dama/welcome.html</A>

Unfortunately, their result is controversial, because the Cryogenic Dark
Matter Search (CDMS) was unable to replicate it.  This experiment works a
different way: it uses germanium and silicon crystals cooled to a
hundredth of a degree above absolute zero.  The idea is to detect the
phonons - that is, quantized sound waves - produced when a WIMP smacks
into an atomic nucleus.

The original CDMS experiment was done at Stanford, only 10 meters below
the ground; this meant it had to distinguish WIMPs from a background of
cosmic rays.  Now they are redoing the experiment in an abandoned mine
in Minnesota, which should give more accurate results.

For more on the CDMS experiment, try:

5) Cryogenic Dark Matter Search (CDMS) home page, 
<A HREF = "http://cdms.berkeley.edu/">http://cdms.berkeley.edu/</A>

In short, the situation is still murky.  Luckily, a bunch more dark 
matter detectors are coming online as we speak, which should help
straighten things out.  You can find websites for these dark matter
experiment and also conference here:

6) Frederic Mayet, Dark Matter Portal, <A HREF = "http://isnwww.in2p3.fr/ams/fred/dm.html">http://isnwww.in2p3.fr/ams/fred/dm.html</A>

Finally, here are some things to read if you want to learn more.  First,
some general introductions to cosmology, in roughly increasing order
of difficulty:

7) Edward R. Harrison, Cosmology, the Science of the Universe, Cambridge
University Press, Cambridge, 1981.

8) M. Berry, Cosmology and Gravitation, Adam Hilger, Bristol, 1986. 

9) John A. Peacock, Cosmological Physics, Cambridge University Press, 
Cambridge, 1999. 

Second, a nice easy review article on dark matter:

10) Shaaban Khalil and Carlos Munoz, The enigma of the dark matter,
to appear in Contemp. Phys., also available at 
<A HREF = "http://xxx.lanl.gov/abs/hep-ph/0110122">hep-ph/0110122</A>.

Third, two articles surveying candidates for what dark matter might be:
neutralinos, axions, axinos, gravitinos, MACHOs - you name it!

11) Leszek Roszkowski, Non-baryonic dark matter, available as
<A HREF = "http://xxx.lanl.gov/abs/hep-ph/0102327">hep-ph/0102327</A>.

12) B. J. Carr, Recent developments in the search for baryonic dark
matter, available as <A HREF = "http://xxx.lanl.gov/abs/astro-ph/0102389">astro-ph/0102389</A>.

Okay, now on to something more mathematical....

I've been having fun lately learning about "teleparallel" theories
gravity from Simon Clark, Chris Hillman and Stephen Speicher on
sci.phyics.research.  This is a good introduction:

13) V. C. de Andrade, L. C. T. Guillen and J. G. Pereira, Teleparallel
gravity: an overview, available at <A HREF = "http://xxx.lanl.gov/abs/gr-qc/0011087">gr-qc/0011087</A>.

In ordinary general relativity, you describe the gravitational field
using a "metric": a field that lets you measure times, distances and
angles.  In teleparallel gravity, you instead use a field that allows
you to decide whether two vectors at two points of spacetime are "the
same".   This notion of unambiguously comparing vectors at different
points of spacetime is called "distant parallelism", hence the term
"teleparallel".  

At first the idea of distant parallelism seems antithetical to general
relativity.  After all, in the usual formalism of general relativity,
you can only compare vectors at different points of a curved spaceteim
\emph{after} you pick a path from one to the other!  The wonderful thing is
that you can formulate theories of teleparallel gravity that are
equivalent to general relativity for all practical purposes.  The
philosophy is completely different: for example, in general relativity
you shouldn't think of gravity as a "force" that "accelerates"
particles, but in teleparallel gravity you can.  However, the physical
predictions are the same for a huge class of situations.

Here's a sketch of how it works.  I'm afraid I'll have to turn on the
differential geometry now.

It's easiest to start with the so-called Palatini formulation of general
relativity.  Here we take spacetime to be an orientable smooth
4-manifold M and pick a vector bundle T that is isomorphic to the
tangent bundle TM.  We equip T with a Lorentzian metric and orientation.
A good name for T would be the "fake tangent bundle", but physicists
usually call its fiber the "internal space".  The trick is then to
describe a Lorentzian metric on M by means of a vector bundle map


$$

e: TM \to  T  
$$
    
which we call a "coframe field".  We can use this to pull the metric on
T back to the tangent bundle.  If e is an isomorphism, this gives a
Lorentzian metric on M.  If it's not, we get something like a metric,
but with degenerate directions.  You can think of the Palatini
formulation as extending general relativity to allow such "degenerate
metrics", and this becomes really important in quantum gravity, but for
now let's only consider the case where e is an isomorphism.

The coframe field is one of the two basic fields in the Palatini
formulation.  The other is a metric-compatible connection on T.  
This connection is usually denoted A and called a "Lorentz connection".  
Its curvature is denoted F.

The Lagrangian for the Palatini formulation of general relativity
looks like this:


\begin{verbatim}

tr(e ^ e ^ *F)
\end{verbatim}
    
This takes a bit of explaining!  First of all, the curvature F is an
End(T)-valued 2-form, but using the metric on T we get an isomorphism
between T and its dual, so we can also think of the curvature as a
2-form taking values in T tensor T.  However, if we do this, the fact
that A is metric-compatible means that F is skew-symmetric: it takes in
the second exterior power of T, \Lambda ^{2}(T).  

Since T has a metric and orientation, we can define a Hodge star
operator on the exterior algebra \Lambda (T) just as we normally do for
differential forms on a manifold with metric and orientation.  We call
this the "internal" Hodge star operator.  Using this we can 
define *F, which is again a 2-form taking values in \Lambda ^{2}(T).  

Whew!  It takes some work making sense of that terse formula above!
We're not done yet, either.  Of course, all these verbal descriptions
can be avoided by writing down formulas packed with indices. That's what
working physicists do.  And when they've got  two different vector
bundles around, like T and the tangent bundle TM they use two different
fonts for their indices: for example, Latin letters for the "internal
indices" associated to T, and Greek letters for the "spacetime indices"
associated to TM.  Once you get used to this, it's really efficient. 
It's only mathematicians who would rather read a paragraph of
complicated verbiage than a fancy equation.  The equation helps you
compute, but the verbiage helps you \emph{understand} - at least if you
follow it!  If you don't know enough geometry, the verbiage probably
seems more confusing than helpful.

Okay.  Next, note that the coframe field e can be thought of as a
T-valued 1-form.  This allows us to define the wedge product e ^ e as  
a \Lambda ^{2}(T)-valued 2-form.  Note that this is the same sort 
of gadget as the curvature F and its internal Hodge dual *F.  This means we can
take the wedge product of the differential form parts of e ^ e and *F
while  using the metric on T to pair together their \Lambda ^{2}(T) 
parts and get a number.  The result is a plain old 4-form, which we call 
tr(e ^ e ^ *F). This is our Lagrangian!

If you work out the equations of motion coming from this Lagrangian, 
they say A that pulls back via e to a \emph{torsion-free} metric-compatible 
connection on the tangent bundle: the Levi-Civita connection!  It
follows that F pulls back to the curvature of the Levi-Civita
connection: the Riemann tensor!   Finally, it turns out that 
tr(e ^ e ^ *F) is just the Ricci scalar curvature times the volume 
form on M... so we were doing general relativity all along!  

This may seem convoluted, but one advantage of this approach is that it
describes gravity as a kind of gauge theory.  From the viewpoint of
field theory, the metric is a rather curious beast: it's a section of a
bundle, but it's required to satisfy \emph{inequalities} saying that it is
nondegenerate and has a certain signature.  Here we have tamed this
beast - or at least locked it up safely inside the formalism of
differential forms and connections.  As a spinoff, we don't get those
nasty factors of "the square root of the determinant of the metric"
which plague the old-fashioned approach to general relativity.  The
reason is that the coframe field acts like a "square root" of the
metric.

Physicists have spent a lot of time trying to recast gravity as a gauge
theory  If you read old journals, you'll see endless arguments about
what \emph{gauge group} to use.   It turns out there are a lot of right
answers.   The gauge group for the Palatini formulation of general
relativity is the Lorentz group, but we can also cook up formulations
where the gauge group is the Poincare group or the translation group
R^{4}.  I'd known about the Poincare group version - I'll explain that in
a minute - but I hadn't known you could get away with using just the
translation group!  That's where teleparallel gravity comes in.  It all
fits together in a beautiful big picture....

The Poincare group is the semidirect product of the Lorentz group
SO(3,1) and the translation 
group R^{4}.  This means that a Poincare group
connection can be written as a Lorentz group connection plus a part
related to the translation group.  We know the Palatini formalism
involves a Lorentz connection.  What about the other part?   This is
just the coframe field!   To see this, note that since each fiber of T
looks just like Minkowski spacetime, we can use T to create a principal
bundle over M whose gauge group is the Poincare group.  A connection on
this principal bundle works out to be exactly the same as a Lorentz
connection A together with a T-valued 1-form e.  

So, without lifting a finger, we can reinterpret the Palatini formalism
as a theory in which the only field is a Poincare group connection. 
Like the Poincare group itself, the curvature of this connection can be
chopped into two pieces.  The Lorentz group part is our old friend, the
\Lambda ^{2}(T)-valued 2-form 


\begin{verbatim}

F = dA + A ^ A.
\end{verbatim}
    
The translation group part is a T-valued 2-form:


\begin{verbatim}

t = de + A ^ e.
\end{verbatim}
    
Using e: TM \to  T we can pull all this stuff back to the tangent bundle,
where its meaning becomes evident.  The metric on T pulls back to a
metric on the tangent bundle, A pulls back to a metric-compatible
connection on the tangent bundle, F pulls back to the curvature of this
connection, and t pulls back to the \emph{torsion} of this connection!   As 
already hinted, one of the equations of motion says that t vanishes, so
A really pulls back to a \emph{torsion-free} metric-compatible connection:
the Levi-Civita connection.

Finally, let's see how to get rid of the Lorentz connection A and
formulate gravity using just the coframe field e, which we'll interpret
as a translation group connection.  It seems the teleparallel gravity
crowd only knows how to pull this stunt when the tangent bundle of M is
trivializable.  But this is not as bad as it sounds: every orientable
3-manifold S has a trivializable tangent bundle, so the same is true of
every orientable 4-manifold of the form R x S.  

So: suppose M is a 4-manifold with trivializable tangent bundle.  This
means we can take T to be the trivial bundle M x R^{4}.  The usual
Minkowski metric on R^{4} puts a Lorentzian metric on T, and the
trivialization gives this bundle a flat metric-compatible connection A.

We've seen a connection like this A before, but this time it won't be
one of the dynamical fields in our theory: it'll be a "fixed background
structure", cast in iron.  It's so boring it looks just like "0" when we
do calculations using our trivialization of T, but I prefer to give a
name to it nonetheless.  

The only dynamical field in teleparallel gravity is the coframe field e.
We can think of this as a T-valued 1-form, or if you prefer, a
"translation group connection": a connection on the bundle T regarded 
as a principal bundle with gauge group R^{4}.  The curvature of this
connection is a T-valued 2-form which we'll again call t.  As before
we have


\begin{verbatim}

t = de + A ^ e
\end{verbatim}
    
but using our trivialization of T this formula boils down to


\begin{verbatim}

t = de.
\end{verbatim}
    
As before, we can use e to pull stuff from T back to the tangent bundle
TM.  The metric on T pulls back to a metric on TM, the connection A
pulls back to a metric-compatible connection W on TM, and t pulls back
to a TM-valued 2-form which is just the torsion of W.  In this setup
there's no reason for t to vanish, so the connection W will have
torsion.  On the other hand, A has no curvature, so neither will W.

Folks call W the "Weitzenboeck connection".  Of course when e is an
isomorphism there's another connection on TM, too: the Levi-Civita
connection, L.  Both these are metric-compatible, but they're very
different.  The Weitzenbock connection has torsion but no curvature; 
the Levi-Civita connection has curvature but no torsion!  

Andrade and company give a nice explanation for what's going on here.

\begin{quote}
According to general relativity, curvature is used to \emph{geometrize}
spacetime, and in this way successfully describe the gravitational 
interaction.  Teleparallelism, on the other hand, attributes gravitation 
to torsion, but in this case torsion accounts for gravitation not
by geometrizing the interaction, but by acting as a \emph{force}.  This
means that, in the teleparallel equivalent of general relativity,
there are no geodesics, but force equations quite analogous to the 
Lorentz force equation of electrodynamics.  Thus, we can say that the 
gravitational interaction can be described \emph{alternatively} in terms 
of curvature, as is usually done in general relativity, or in 
terms of torsion, in which case we have the so-called teleparallel 
gravity.  Whether gravity requires a curved or torsioned spacetime, 
therefore, turns out to be a matter of convention.
\end{quote}
The difference of the Weitzenboeck and Levi-Civita connections,


\begin{verbatim}

K = W - L,
\end{verbatim}
    
goes by the charming name of the "contorsion", since it says how
much the coframe field twists around as measured by the Levi-Civita
connection.

The review article by Andrade et al gives a nice formula for the
contorsion in terms of the torsion of the Weitzenboeck connection.  
This means we can express the Levi-Civita connection completely in 
terms of the Weitzenboeck connection and its torsion.  And \emph{that}
means we can express the Ricci scalar curvature in terms of the 
Weitzenboeck connection and its torsion.  Great - so we can write 
down the Lagrangian for general relativity in this new lingo!  
Ultimately, we can express it purely in terms of the coframe field e.  

Unfortunately, I haven't smoothed down the calculations to the point
where you'd actually want to see them here.  The prettiest formula 
for the Lagrangian shows up in this paper:

14) Yakov Itin, Energy-momentum current for coframe gravity, available
as <A HREF = "http://xxx.lanl.gov/abs/gr-qc/0111036">gr-qc/0111036</A>.

Up to a constant factor, it looks like this:


$$

2(e^{i} ^ de_{i}) ^ *(e^{j} ^ de_{j}) - (e^{i} ^ de^{j}) ^ *(e_{i} ^ de_{j}) 
$$
    

where i and j are internal indices, but * is the usual "spacetime" Hodge
star operator.

By now I've probably lost everyone except people who understand this
stuff already, so I'll stop here.  If you read the references, you'll
find a nice equation for how a freely falling particle moves in 
teleparallel gravity, a nice formula for the gravitational
energy-momentum pseudotensor in teleparallel gravity, and so on.  
Itin's paper also considers versions of teleparallel gravity with
more general Lagrangians built from the coframe field, which are not
necessarily equivalent to general relativity.

Now for something completely different!  Here's the final episode of my
description of this paper by Michael Mueger:

15) From subfactors to categories and topology I: Frobenius algebras in
and Morita equivalence of tensor categories, available as
<A HREF = "http://www.arXiv.org/abs/math.CT/0111204">math.CT/0111204</A>.

In "<A HREF = "week174.html">week174</A>" I talked about Frobenius algebras and 2-categories;
in "<A HREF = "week175.html">week175</A>" I said a bit about subfactors; now it's time for me to
say something about how Mueger puts these together!   This will be 
very sketchy, I'm afraid.  

First, it's worth noting that lots of mathematical gadgets form
not just categories but also 2-categories.  For example, we all
know the category of groups, where the objects are groups and the
morphisms are homomorphisms.  But there is also a 2-category lurking
around here!  Between any morphisms 


$$

f,f': G \to  H
$$
    

we can define a 2-morphism 


$$

a: f => f'
$$
    

to be an element of H with the property that 


\begin{verbatim}

af(g) = f'(g)a    for all g in G.   
\end{verbatim}
    

This just says that f' is f conjugated by an element of H, so we
could call these 2-morphisms "conjugations".  

This definition may seem forced, but it's actually quite natural if you 
remember that a group is a special sort of category with one object and 
with all morphisms being invertible.  Functors between these special 
categories are just group homomorphisms, and natural transformations
between these functors are just conjugations!  If you don't follow this,
check out "<A HREF = "week73.html">week73</A>" - you'll see the above equation is just a special
case of the definition of "natural transformation".

For fans of group theory, one nice thing about this 2-category is that
it explains where "inner automorphisms" fit into the grand n-categorical 
scheme of things.  It also explains why conjugations become important in 
algebraic topology when you're playing around with the "fundamental group":
this is actually a 2-functor from the 2-category of


\begin{verbatim}

spaces with basepoint,
base-point-preserving maps, and
not-necessarily-basepoint-preserving homotopies
\end{verbatim}
    

to the 2-category of


\begin{verbatim}

groups,
group homomorphisms, and
conjugations.
\end{verbatim}
    

We can also cook up a 2-category of rings which works in a similar way;
the objects are rings, the morphisms are ring homomorphisms, and the
2-morphisms are conjugations, defined by the same formula as above.

Mueger's work uses this 2-category, or more precisely, a sub-2-category
where we use not \emph{all} rings, but only certain specially nice type III 
factors, and not \emph{all} homomorphisms, but only certain specially nice
*-homomorphisms.  He gives a nice simple condition for a morphism in
this 2-category to have a "two-sided adjoint" - meaning precisely 
that it's part of what I called an "ambidextrous adjunction" 
in "<A HREF = "week174.html">week174</A>".
And as we saw back then, any ambidextrous adjunction gives a Frobenius
object!  So, he gets lots of Frobenius objects from the theory of factors.  
But more importantly, he shows that a whole lot of concepts beloved by
folks who study von Neumann algebras are really concepts from 2-category
theory, applied to this situation! 

This is cool, because there are \emph{already} deep connections between 
n-categories and quantum theory - see "<A HREF = "week78.html">week78</A>" for an introduction 
to these ideas.  Since von Neumann algebras are the basic "algebras
of observables" in quantum theory, we should \emph{expect} 
them to be deeply 
n-categorical in nature.  And now, thanks to the work of Mueger, it's 
becoming a lot clearer just how.  But I don't think we're anywhere 
near the bottom of it yet - at least, not me!

By the way, it's taken me so long to explain Mueger's last paper that
he's already written another:

16) Michael Mueger, On the structure of modular categories, available
as <A HREF = "http://www.arXiv.org/abs/math.CT/0201017">math.CT/0201017</A>.




\par\noindent\rule{\textwidth}{0.4pt}
<i>"I admire the elegance of your method of computation; it must be nice to ride through these fields upon the horse of true mathematics while the like of us have to make our way laboriously on foot."</i> - Einstein to Levi-Civita 
% </A>
% </A>
% </A>
