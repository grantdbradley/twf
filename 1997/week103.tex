
% </A>
% </A>
% </A>
\week{April 26, 1997}

As I segue over from the homotopy theory conference at Northwestern
University to the conference on higher-dimensional algebra and physics
that took place right after that, it's a good time to mention Ronnie
Brown's web page:

1) Ronald Brown, Higher-dimensional group theory,
<A HREF = "http://www.bangor.ac.uk/~mas010/home.html">
http://www.bangor.ac.uk/~mas010/home.html</A>
Brown is the one who coined the phrase "higher-dimensional algebra", and
for many years he has been developing this subject, primarily as a tool
for doing homotopy theory.  I wrote a bit about his ideas two years ago,
in "<A HREF = "week53.html">week53</A>".  A lot has happened in higher-dimensional algebra since
then, and the web page above is a good place to get an overview of it.
It includes a nice bibliography on the subject.  Also, if you find the
math a bit strenuous, you can rest your brain and delight your eyes at
the related site:

2) Symbolic sculptures and mathematics,
<A HREF = "http://www.bangor.ac.uk/SculMath/">
http://www.bangor.ac.uk/SculMath/</A>

which opens with a striking image of rotating linked tori, and includes
pictures of the mathematical sculpture of John Robinson.

The Workshop on Higher Category Theory and Physics was exciting because
it pulled together a lot of people working on the interface between
these two subjects, many of whom had never before met.  It was organized
by Ezra Getzler and Mikhail Kapranov.  Getzler is probably most well-known
for his proof of the Atiyah-Singer index theorem.  This wonderful
theorem captured the imagination of mathematical physicists for many
years starting in the 1960s.  The reason is that it relates the topology
of manifolds to the the solutions of partial differential equations
on these manifolds, and thus ushered in a new age of applications of
topology to physics.  In the 1980s, working with ideas that Witten
came up with, Getzler found a nice "supersymmetric proof" of the 
Atiyah-Singer theorem.   Later Getzler turned to other things, such
as the use of "operads" (see "<A HREF = "week42.html">week42</A>") to study conformal field 
theory (which shows up naturally in string theory).  Kapranov has
also done a lot of work with operads and conformal field theory, and
many other things, but I first learned about him through his paper
with Voevodsky on "braided monoidal 2-categories" (see "<A HREF = "week4.html">week4</A>").  This 
got me very excited since it turned me on to many of the main themes of 
n-category theory.  

Alas, my description of this fascinating conference will be terse
and dry in the extreme, since I am flying to Warsaw in 3 hours for
a quantum gravity workshop.  I'll just mention a few papers that
cover some of the themes of this conference.  Ross Street gave
two talks on Batanin's definition of weak n-categories (and even
weak \omega -categories), which one can get as follows:

3) Ross Street, The role of Michael Batanin's monoidal globular
categories.  Lecture I:  Globular categories and trees.  Lecture II: 
Higher operads and weak \omega -categories. Available in gzipped 
Postscript form at 
<A HREF = "http://www.math.mq.edu.au/~street/Publications.html">
http://www.math.mq.edu.au/~street/Publications.html</A>

Subsequently Batanin has written a more thorough paper on his
definition:

4) Michael Batanin, Monoidal globular categories as a natural
environment for the theory of weak n-categories, Adv. Math 136
(1998), 39-103, preprint available at
<A HREF = "http://www-math.mpce.mq.edu.au/~mbatanin/papers.html">
http://www-math.mpce.mq.edu.au/~mbatanin/papers.html</A>

I gave a talk on Dolan's and my definition of weak n-categories,
which one can get as follows:

5) John Baez, An introduction to n-categories, to appear in
the proceedings of Category Theory and Computer Science '97,
preprint available as <A HREF = "http://xxx.lanl.gov/abs/q-alg/9705009">q-alg/9705009</A> or in 
Postscript form at <A HREF = "http://math.ucr.edu/home/baez/ncat.ps">
http://math.ucr.edu/home/baez/ncat.ps</A>

Unfortunately Tamsamani was not there to present \emph{his} definition
of weak n-categories, but at least I have learned how to get his papers
electronically:

6) Zouhair Tamsamani, Sur des notions de $\infty$-categorie et 
$\infty$-groupoide non-strictes via des ensembles multi-simpliciaux, 
preprint available as 
<A HREF = "http://xxx.lanl.gov/abs/alg-geom/9512006">alg-geom/9512006</A>.

 
Zouhair Tamsamani, Equivalence de la theorie homotopique des 
n-groupoides et celle des espaces topologiques n-tronques, 
preprint available as 
<A HREF = "http://xxx.lanl.gov/abs/alg-geom/9607010">alg-geom/9607010</A>.

Also, Carlos Simpson has written an interesting paper using 
Tamsamani's definition:

7) Carlos Simpson, A closed model structure for n-categories, internal
Hom, n-stacks and generalized Seifert-Van Kampen, preprint available as
<A HREF = "http://xxx.lanl.gov/abs/alg-geom/9704006">alg-geom/9704006</A>.

In a different but related direction, Masahico Saito discussed
a paper with Scott Carter and Joachim Rieger in which they come
up with a nice purely combinatorial description of all the ways
to embed 2-dimensional surfaces in 4-dimensional space:

8) J. Scott Carter, Joachim H. Rieger and Masahico Saito,
A combinatorial description of knotted surfaces and their isotopies,
to appear in Adv. Math., preprint available at 
<A HREF = "http://www.math.usf.edu/~saito/home.html">http://www.math.usf.edu/~saito/home.html</A>
 
My student Laurel Langford has translated their work into 
n-category theory and shown that "unframed unoriented 2-tangles 
form the free braided monoidal 2-category on one unframed
self-dual object":

9) John Baez and Laurel Langford, 2-Tangles, preprint available
as <A HREF = "http://xxx.lanl.gov/abs/q-alg/9703033">q-alg/9703033</A> and in Postscript form at 
http://math.ucr.edu/home/baez/2tang.ps  

This paper summarizes the results; the proofs will appear later.

While I was there, Carter also gave me a very nice paper
he'd done with Saito and Louis Kauffman.  This paper discusses
4-manifolds and also 2-dimensional surfaces in 3-dimensional space,
again getting a purely combinatorial description which is begging
to be translated into n-category theory:

10) J. Scott Carter, Louis H. Kauffman and Masahico Saito, 
Diagrammatics, singularities, and their algebraic interpretations,
preprint available at <A HREF = "http://www.math.usf.edu/~saito/home.html">http://www.math.usf.edu/~saito/home.html</A>
 
I am sorry not to describe these papers in more detail, but
I've been painfully busy lately.  (In fact, I am trying
to figure out how to reform my life to give myself more spare
time.  I think the key is to say "no" more often.)

Thanks to Justin Roberts for pointing out an error in "<A HREF = "week102.html">week102</A>".  
The phase ambiguity in conformal field theories is not necessarily a
24th root of unity; it's exp(2 \pi  i c / 24) where c is the central
charge of the associated Virasoro representation.   This is a big
hint as far as my puzzle goes.  

Also I thank Dan Christensen for helping me understand \pi _4(S^2)
in a simpler way, and Scott Carter for a fascinating letter on 
the themes of "<A HREF = "week102.html">week102</A>".  Alas, I have been too busy to reply
adequately to these nice emails!

Gotta run....  



\par\noindent\rule{\textwidth}{0.4pt}
\par\noindent\rule{\textwidth}{0.4pt}

% </A>
% </A>
% </A>
