
% </A>
% </A>
% </A>
\week{June 21, 1997}


There are some spooky facts in mathematics that you'd never guess in a
million years... only when someone carefully works them out do they
become clear.  One of them is called "Bott periodicity".  

A 0-dimensional manifold is pretty dull: just a bunch of points.
1-dimensional manifolds are not much more varied: the only
possibilities are the circle and the line, and things you get by
taking a union of a bunch of circles and lines.  2-dimensional
manifolds are more interesting, but still pretty tame: you've got your
n-holed tori, your projective plane, your Klein bottle, variations on
these with extra handles, and some more related things if you allow
your manifold to go on forever, like the plane, or the plane with a
bunch of handles added (possibly infinitely many!), and so on....  You
can classify all these things.  3-dimensional manifolds are a lot
more complicated: nobody knows how to classify them.  4-dimensional
manifolds are a \emph{lot} more complicated: you can \emph{prove}
that it's \emph{impossible} to classify them - that's called Markov's Theorem.  

Now, you probably wouldn't have guessed that a lot of things start
getting simpler when you get up around dimension 5.  Not everything,
just some things.  You still can't classify manifolds in these high
dimensions, but if you make a bunch of simplifying assumptions you
sort of can, in ways that don't work in lower dimensions.  Weird, huh?
But that's another story.  Bott periodicity is different.  It says
that when you get up to 8 dimensions, a bunch of things are a whole
lot like in 0 dimensions!  And when you get up to dimension 9, a bunch
of things are a lot like they were in dimension 1.  And so on - a
bunch of stuff keeps repeating with period 8 as you climb the ladder
of dimensions.

(Actually, I have this kooky theory that perhaps part of the reason
topology reaches a certain peak of complexity in dimension 4 is that
the number 4 is halfway between 0 and 8, topology being simplest in
dimension 0.  Maybe this is even why physics likes to be in 4
dimensions!  But this is a whole other crazy digression and I will
restrain myself here.)

Bott periodicity takes many guises, and I already described one in
"<A HREF = "week104.html">week104</A>".  Let's start with the real numbers, and then throw in n
square roots of -1, say e_{1},...,e_{n}.   Let's make them "anticommute",
so

e_{i} e_{j} = - e_{j} e_{i}

when i is different from j.  What we get is called the "Clifford
algebra" C_{n}.  For example, when n = 1 we get the complex numbers,
which we call C.  When n = 2 we get the quaternions, which we call H,
for Hamilton.  When n = 3 we get... the octonions??  No, not the
octonions, since we always demand that multiplication be associative!
We get the algebra consisting of \emph{pairs} of quaternions!  We call that
H + H.  When n = 4 we get the algebra consisting of 2x2 \emph{matrices} of
quaternions!  We call that H(2).  And it goes on, like this:

C_{0} = R
C_{1} =  C
C_{2} = H
C_{3} = H + H
C_{4} =  H(2)
C_{5} = C(4)
C_{6} = R(8)
C_{7} = R(8) + R(8)
C_{8} = R(16)

Note that by the time we get to n = 8 we just have 16x16 matrices of
real numbers.  And that's how it keeps going: C_{n+8} is just 16x16
matrices of guys in C_{n}!  That's Bott periodicity in its simplest form.

Actually right now I'm in Vienna, at the Schroedinger Institute, and
one of the other people visiting is Andrzej Trautman, who gave a talk
the other day on "Complex Structures in Physics", where he mentioned a
nice way to remember the above table.  Imagine the day is only 8 hours
long, and draw a clock with 8 hours.  Then label it like this:

                        

\begin{verbatim}
<b>
                                    0
                                 
                                    R

                 7                                    1
                                                   
                   R+R                             C





             6   R                                       H   2
 




                    C                             H+H
                 
                  5                                    3

                                    H

                                    4
</b>
\end{verbatim}
    


The idea here is that as the dimension of space goes up, you go around
the clock.  One nice thing about the clock is that it has a reflection
symmetry about the axis from 3 o'clock to 7 o'clock.  To use the
clock, you need to know that the dimension of the Clifford algebra
doubles each time you go up a dimension.  This lets you figure out,
for example, that the Clifford algebra in 4 dimensions is not really
H, but H(2), since the latter has dimension 16 = 2^{4}.

Now let's completely change the subject and talk about rotations
in infinite-dimensional space!  What's a rotation in infinite-dimensional 
space like?  Well, let's start from the bottom and work our way up.
You can't really rotate in 0-dimensional space.  In 1-dimensional
space you can't really rotate, you can only \emph{reflect} things... but we
will count reflections together with rotations, and say that the
operations of multiplying by 1 or -1 count as "rotations" in
1-dimensional space.  In 2-dimensional space we describe rotations by
2x2 matrices like


\begin{verbatim}

cos t     -sin t
sin t      cos t
\end{verbatim}
    
and since we're generously including reflections, also matrices like


\begin{verbatim}

cos t      sin t
sin t     -cos t
\end{verbatim}
    
These are just the matrices whose columns are orthonormal vectors.  In
3-dimensional space we describe rotations by 3x3 matrices whose
columns are orthonormal, and so on.  In n-dimensional space we call
the set of nxn matrices with orthonormal columns the "orthogonal
group" O(n).  

Note that we can think of a rotation in 2 dimensions


\begin{verbatim}

cos t     -sin t
sin t      cos t
\end{verbatim}
    
as being a rotation in 3 dimensions if we just stick one more row and one
column like this:


\begin{verbatim}

cos t     -sin t    0
sin t      cos t    0
  0          0      1
\end{verbatim}
    
This is just a rotation around the z axis.  Using the same trick
we can think of any rotation in n dimensions as a rotation in n+1
dimensions.  So we can think of O(0) as sitting inside O(1), which
sits inside O(2), which sits inside O(3), which sits inside O(4),
and so on!  Let's do that.  Then let's just take the \emph{union} of
all these guys, and we get... O(\infty )!  This is the group of
rotations, together with reflections, in infinite dimensions.

(Now if you know your math, or you read "<A HREF = "week82.html">week82</A>", you'll realize that I
didn't really change the subject, since the Clifford algebra C_{n} is
really just a handy way to study rotations in n dimensions.  But never
mind.)  

Now O(\infty ) is a very big group, but it elegantly summarizes a lot
of information about rotations in all dimensions, so it's not
surprising that topologists have studied it.  One of the thing
topologists do when studying a space is to work out its "homotopy
groups".  If you hand them a space X, and choose a point x in this
space, they will work out all the topologically distinct ways you can
stick an n-dimensional sphere in this space, where we require that the
north pole of the sphere be at x.  This is what they are paid to do.
We call the set of all such ways the homotopy group \pi _{n}(X).  For a
more precise description, try "<A HREF = "week102.html">week102</A>" - but this will do for now.

So, what are the homotopy groups of O(\infty )?  Well, they start out
looking like this:


\begin{verbatim}

n       \pi _{n}(O(\infty ))

0         Z/2
1         Z/2
2          0
3          Z
4          0
5          0
6          0
7          Z
\end{verbatim}
    
And then they repeat, modulo 8.  Bott periodicity strikes again!

But what do they mean?

Well, luckily Jim Dolan has thought about this a lot.  Discussing
it repeatedly in the little cafe we tend to hang out at, we came up
with the following story.  Most of it is known to various people
already, but it came as sort of a revelation to us.

The zeroth entry in the table is easy to understand.  \pi _{0} keeps track
of how many connected components your space has.  The rotation group
O(\infty ) has two connected components: the guys that are rotations,
and the guys that are rotations followed by a reflection.  So \pi _{0} of
O(\infty ) is Z/2, the group with two elements.  Actually this is
also true for O(n) whenever n is higher enough, namely 1 or more.
So the zeroth entry is all about "reflecting".  

The first entry is a bit subtler but very important in physics.  It
means that there is a loop in O(\infty ) that you can't pull tight,
but if you go around that loop \emph{twice}, you trace out a loop that you
\emph{can} pull tight.  In fact this is true for O(n) whenever n is 3 or
more.  This is how there can be spin-1/2 particles when space is
3-dimensional or higher.  There are lots of nice tricks for seeing
that this is true, which I hope the reader already knows and loves.
In short, the first entry is all about "rotating 360 degrees and not
getting back to where you started".

The second entry is zero.  

The third entry is even subtler but also very important in modern
physics.  It means that the ways to stick a 3-sphere into O(\infty )
are classified by the integers, Z.  Actually this is true for O(n)
whenever n is 5 or more.  It's even true for all sorts of other
groups, like all "compact simple groups".  But can I summarize this
entry in a snappy phrase like the previous nonzero entries?  Not
really.  Actually a lot of applications of topology to quantum field
theory rely on this \pi _{3} business.  For example, it's the key to stuff
like "instantons" in Yang-Mills theory, which are in turn crucial for
understanding how the \pi on gets its mass.  It's also the basis of
stuff like "Chern-Simons theory" and "BF theory".  Alas, all this
takes a while to explain, but let's just say the third entry is about
"topological field theory in 4 dimensions".

The fourth entry is zero.

The fifth entry is zero.

The sixth entry is zero.

The seventh entry is probably the most mysterious of all.  From one
point of view it is the subtlest, but from another point of view it is
perfectly trivial.   If we think of it as being about \pi _{7} it's very
subtle: it says that the ways to stick a 7-sphere into O(\infty ) are
classified by the integers.   (Actually this is true for O(n) whenever
n is 7 or more.)  But if we keep Bott periodicity in mind, there is
another way to think of it: we can think of it as being about \pi _{-1},
since 7 = -1 mod 8.  

But wait a minute!  Since when can we talk about \pi _{n} when n is
\emph{negative?!}  What's a -1-dimensional sphere, for example?

Well, the idea here is to use a trick.  There is a space very related
to O(\infty ), called kO.   As with O(\infty ), the homotopy
groups of this space repeat modulo 8.  Moreover we have:

\pi _{n}(O(\infty )) = \pi _{n+1}(kO)

Combining these facts, we see that the very subtle \pi _{7} of O(\infty )
is nothing but the very unsubtle \pi _{0} of kO, which just
keeps track of how many connected components kO has.

But what \emph{is} kO?   

Hmm.  The answer is very important and interesting, but it would
take a while to explain, and I want to postpone doing it for a while,
so I can get to the punchline.  Let me just say that when we work
it all out, we wind up seeing that the seventh entry in the table
is all about \emph{dimension}.  

To summarize: 

\pi _{0}(O(\infty )) = Z/2  is about REFLECTING
\pi _{1}(O(\infty )) = Z/2  is about ROTATING 360 DEGREES
\pi _{3}(O(\infty )) = Z    is about TOPOLOGICAL FIELD THEORY IN 4 DIMENSIONS
\pi _{7}(O(\infty )) = Z    is about DIMENSION

But wait!  What do those numbers 0, 1, 3, and 7 remind you of?  

Well, after I stared at them for a few weeks, they started to remind
me of the numbers 1, 2, 4, and 8.  And \emph{that} immediately reminded me
of the reals, the complexes, the quaternions, and the octonions!  

And indeed, there is an obvious relationship.  Let n be 1, 2, 4, or 8,
and correspondingly let A stand for either the reals R, the complex
numbers C, the quaternions H, or the octonions O.  These guys are
precisely all the "normed division algebras", meaning that the obvious
sort of absolute value satisfies

|xy| = |x||y|.

Thus if we take any guy x in A with |x| = 1, the operation
of multiplying by x is length-preserving, so it's a reflection or 
rotation in A.  This gives us a function from the unit sphere in
A to O(n), or in other words from the (n-1)-sphere to O(n).  We thus
get nice elements of

\pi _{0}(O(1))
\pi _{1}(O(2))
\pi _{3}(O(4))
\pi _{7}(O(8))

which turn out to be precisely why these particular homotopy groups
of O(\infty ) are nontrivial.

So now we have the following fancier chart:

\pi _{0}(O(\infty )) is about REFLECTING and the REAL NUMBERS
\pi _{1}(O(\infty )) is about ROTATING 360 DEGREES and the COMPLEX NUMBERS
\pi _{3}(O(\infty )) is about TOPOLOGICAL FIELD THEORY IN 4 DIMENSIONS and the
                           QUATERNIONS
\pi _{7}(O(\infty )) is about DIMENSION and the OCTONIONS

Now this is pretty weird.  It's not so surprising that reflections and the
real numbers are related: after all, the only "rotations" in the real
line are the reflections.  That's sort of what 1 and -1 are all about.
It's also not so surprising that rotations by 360 degrees are related to
the complex numbers.  That's sort of what the unit circle is all
about.  While far more subtle, it's also not so surprising that
topological field theory in 4 dimensions is related to the quaternions.  
The shocking part is that something so basic-sounding as "dimension"
should be related to something so erudite-sounding as the "octonions"!

But this is what Bott periodicity does, somehow: it wraps things around
so the most complicated thing is also the least complicated.

That's more or less the end of what I have to say, except for some
references and some remarks of a more technical nature.

Bott periodicity for O(\infty ) was first proved by Raoul Bott in
1959.  Bott is a wonderful explainer of mathematics and one of the
main driving forces behind applications of topology to physics, and
a lot of his papers have now been collected in book form:

1) The Collected Papers of Raoul Bott, ed. R. D. MacPherson.  Vol. 1:
Topology and Lie Groups (the 1950s).  Vol. 2: Differential Operators
(the 1960s).  Vol. 3: Foliations (the 1970s).  Vol. 4: Mathematics
Related to Physics (the 1980s).  Birkhauser, Boston, 1994, 2355 pages
total.  

A good paper on the relation between O(\infty ) and Clifford algebras
is:

2) M. F. Atiyah, R. Bott, and A. Sha\pi ro, Clifford modules, Topology 
(3) 1964, 3-38.  

For more stuff on division algebras and Bott periodicity try Dave Rusin's
web page, especially his answer to "Q5.  What's the question with the
answer n = 1, 2, 4, or 8?"

3) Dave Rusin, Binary products, algebras, and division rings,
<A HREF = "http://www.math.niu.edu/~rusin/known-math/95/division.alg">http://www
.math.niu.edu/~rusin/known-math/95/division.alg</A>

Let me briefly explain this kO business.  The space kO is 
related to a simpler space called BO(\infty ) by means of the
equation
kO = BO(\infty ) x Z,
so let me first describe BO(\infty ).  For any
topological group G you can cook up a space BG whose loop space is
homotopy equivalent to G.  In other words, the space of (base-point-preserving)
maps from S^{1} to BG is homotopy equivalent to G.  It follows that

\pi _{n}(G) = \pi _{n+1}(BG).

This space BG is called the classifying space of G because it has a
principal G-bundle over it, and given \emph{any} decent topological space X
(say a CW complex) you can get all principal G-bundles over X (up to
isomorphism) by taking a map f: X \to  BG and pulling back this
principal G-bundle over BG.  Moreover, homotopic maps to BG give
isomorphic G-bundles over X this way.  
Now a principal O(n)-bundle is
basically the same thing as an n-dimensional real vector bundle -
there are obvious ways to go back and forth between these concepts.  A
principal O(\infty )-bundle is thus very much like a real vector
bundle of \emph{arbitrary} dimension, but where we don't care about adding
on arbitrarily many 1-dimensional trivial bundles.  If we take the
collection of isomorphism classes of real vector bundles over X and
decree two to be equivalent if they become isomorphic after adding on
trivial bundles, we get something called KX, the "real K-theory of X".
It's not hard to see that this is a group.  Taking what I've said and
working a bit, it follows that

KX = [X, BO(\infty )]

where the right-hand side means "homotopy classes of maps from X to
BO(\infty )".  If we take X to be S^{n+1}, we see

KS^{n+1} = \pi _{n+1}(BO(\infty )) = \pi _{n}(O(\infty ))

It follows that we can get all elements of \pi _{n} of O(\infty )
from real vector bundles over S^{n+1}.

Of course, the above equations are true only for nonnegative n, since
it doesn't make sense to talk about \pi _{-1} of a space.  However, 
to make Bott periodicity work out smoothly, it would be nice if we could 
pretend that 
KS^{-1} = \pi _{0}(BO(\infty )) = \pi _{-1}(O(\infty )) 
= \pi _{7}(O(\infty )) = Z
Alas, the equations don't make sense, and BO(\infty ) is connected,
so we don't have  \pi _{0}(BO(\infty )) = Z.  However, 
we can cook up a slightly improved space kO, which has
\pi _{n}(kO) = \pi _{n}(BO(\infty )) 
when n > 0, but also has
\pi _{0}(kO) = Z
as desired.  It's easy - we just let
kO = BO(\infty ) x Z.
So, let's use this instead of BO(\infty ) from now on.
Taking n = 0, we can think of S^{1} as RP^{1}, the real projective line,
i.e. the space of 1-dimensional real subspaces of R^{2}.  This has a
"canonical line bundle" over it, that is, a 1-dimensional real vector
bundle which to each point of RP^{1} assigns the 1-dimensional subspace of
R^{2} that \emph{is} that point.   
This vector bundle over S^{1} gives the generator
of KS^{1}, or in other words, \pi _{0}(O(\infty )). 

Taking n = 1, we can think of S^{2} as the "Riemann sphere", or in other
words CP^{1}, the space of 1-dimensional complex subspaces of C^{2}.  This
too has a "canonical line bundle" over it, which is a 1-dimensional
complex vector bundle, or 2-dimensional real vector bundle.  This
bundle over S^{2} gives the generator of KS^{2}, or in other words,
\pi _{1}(O(\infty )).  

Taking n = 3, we can think of S^{4} as HP^{1}, the space of 1-dimensional 
quaternionic subspaces of H^{2}.  The "canonical line bundle" over this
gives the generator of KS^{4}, or in other words, \pi _{3}(O(\infty )).  

Taking n = 7, we can think of S^{8} as OP^{1}, the space of 
1-dimensional octonionic subspaces of O^{2}.  The "canonical 
line bundle" over this gives the generator of KS^{8}, or in 
other words, \pi _{7}(O(\infty )).

By Bott periodicity, 

\pi _{7}(O(\infty )) = \pi _{8}(kO) = \pi _{0}(kO)

so the canonical line bundle over OP^{1} also defines an element of
\pi _{0}(kO).  But 

\pi _{0}(kO) = [S^{0},kO] = KS^{0}

and KS^{0} simply records the \emph{difference in dimension} between
the two fibers of a vector bundle over S^{0}, which can be
any integer.  This is why the octonions are related to dimension.

If for any pointed space we define

K^{n}(X) = K(S^{n} smash X)

we get a cohomology theory called K-theory, and it turns out that 

K^{n+8}(X) = K(X)

which is another say of stating Bott periodicity.  Now if * denotes
a single point, K(*) is a ring (this is quite common for cohomology theories),
and it is generated by elements of degrees 1, 2, 4, and 8.  The
generator of degree 8 is just the canonical line bundle over OP^{1}
and multiplication by this generator gives a map

K^{n}(*) \to  K^{n+8}(*)

which is an isomorphism of groups - namely, Bott periodicity!
In this sense the octonions are responsible for Bott periodicity.
\par\noindent\rule{\textwidth}{0.4pt}
\textbf{Addendum}: The Clifford algebra clock is even better than I described
above, because it lets you work out the fancier Clifford algebras
C_{p,q}, which are generated by p square roots of -1 and q square
roots of 1, which all anticommute with each other.  These Clifford
algebras are good when you have p dimensions of "space" and q
dimensions of "time", and I described the physically important case
where q = 1 in "<A HREF = "week93.html">week93</A>".  To figure them 
out, you just work out p - q
mod 8, look at what the clock says for that hour, and then take NxN
matrices of what you see, with N chosen so that C_{p,q} gets the right
dimension, namely 2^{p+q}.  So say you're a string theorist and you
think there are 9 space dimensions and 1 time dimension.  You say:
"Okay, 9 - 1 = 8, so I look and see what's at 8 o'clock.  Okay, that's
R, the real numbers.  But my Clifford algebra C_{9,1} is supposed to 
have dimension 2^{9 + 1} = 1024 = 32^{2}, so my Clifford 
algebra must consist of 32x32 \emph{matrices} with real entries."

By the way, it's not so easy to see that the canonical line bundle
over OP^{1} is the generator of KS^{8} - or equivalently, 
that left multiplication
by unit octonions defines a map from S^{7} into SO(8) corresponding 
to the generator of \pi _{7}(O(\infty )).  I claimed it's 
true above, but when someone 
asked me why this was true, I realized I couldn't prove it!  That made me 
nervous.  So I asked on sci.math.research if it was really true, and I got 
this reply:


\begin{verbatim}

From: Linus Kramer 
Newsgroups: sci.math.research
Subject: \pi _7(O) and octonions
Date: Tue, 09 Nov 1999 12:44:33 +0100

John Baez asked if \pi _7(O) is generated by
the (multiplication by) unit octonions.

View this as a question in KO-theory: the claim is
that H^8 generates the reduced real K-theory
\tilde KO(S^8) of the 8-sphere; the bundle
H^8 over S^8 is obtained by the standard glueing
process along the equator S^7, using the octonion
multiplication. So H^8 is the octonion Hopf bundle.
Its Thom space is the projective Cayley plane
OP^2. Using this and Hirzebruch's signature theorem,
one sees that the Pontrjagin class of H^8 is
p_8(H^8)=6x, for a generator x of the 8-dimensional
integral cohomology of S^8 [a reference for this
calulation is my paper 'The topology of smooth
projective planes', Arch. Math 63 (1994)].
We have a diagram

        cplx         ch
KO(S^8) ---> K(S^8) ---> H(S^8)

the left arrow is complexification, the second arrow
is the Chern character. In dimension 8, these maps form
an isomorphism. Now ch(cplx(H^8))=8+x (see the formula
in the last paragraph in Husemoller's "Fibre bundles",
the chapter on "Bott periodicity and integrality
theorems". The constant factor is unimportant, so the
answer is yes, \pi _7(O) is generated by the map
S^7---> O which sends a unit octonion A to the
map l_A:x \to  Ax in SO(8).

Linus Kramer

\end{verbatim}
    

More recently I got an 
email from Todd Trimble which cites another reference
to this fact:


\begin{verbatim}

From: Todd Trimble
Subject: Hopf bundles
To: John Baez 
Date: Fri, 25 Mar 2005 16:37:11 -0500

John, 

In the book Numbers (GTM 123), there is an article by 
Hirzebruch where the Bott periodicity result is formulated 
as saying that the generators of \tilde KO(S^n)  in the cases 
n = 1, 2, 4, 8  are given by [\eta ] - 1  where \eta  is the Hopf 
bundle corresponding to R, C, H, O  and 1 is the trivial 
line bundle over these scalar "fields" (of real dimension 
1, 2, 4, 8), and is 0 for n = 3, 5, 6, 7 [p. 294].  Also that 
the Bott periodicity isomorphism  

        \tilde KO(S^n) ---> \tilde KO(S^{n+8}) 

is induced by  [\eta (O)] - 1  [p. 295].  I know you are aware 
of this already (courtesy of the response of Linus Kramers 
to your sci.math.research query), but I thought you might 
find a published reference, on the authority of no less than 
Hirzebruch, handier (should you need it) than referring to a 
sci.math.research exchange. 

Unfortunately no proof is given.  Hirzebruch says (p. 295), 

 Remark.  Our formulation of the Bott periodicity theorem 
 will be found, in essentials, in [reference to Bott's Lectures 
 on K(X), without proofs].  A detailed proof within the 
 framework of K-theory is given in the textbook [reference 
 to Karoubi's K-theory].  The reader will have a certain amount 
 of difficulty, however, in extracting the results used here from 
 Karoubi's formulation.

Todd
\end{verbatim}
    



\par\noindent\rule{\textwidth}{0.4pt}
<em>... for geometry, you
know, is the gate of science,
and the gate is so low and small that one can only enter it as a little child.
</em> - William Clifford  
\par\noindent\rule{\textwidth}{0.4pt}

% </A>
% </A>
% </A>
