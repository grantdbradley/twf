
% </A>
% </A>
% </A>
\week{February 1, 1996}



It's been a while since I've written an issue of This Week's Finds...
due to holiday distractions and a bunch of papers that need writing up.
But tonight I just can't seem to get any work done, so let me do a bit
of catching up.

I'm no string theorist, but I still can't help hearing all the rumbling
noises over in that direction: first about all the dualities relating
seemingly different string theories, and then about the mysterious
"M-theory" in 11 dimensions which seems to underlie all these
developments.  Let me try to explain a bit of this stuff... in the hopes
that I prompt some string theorists to correct me and explain it better!
I will simplify everything a lot to keep people from getting scared of
the math involved.  But I may also make some mistakes, so the experts
should be kind to me and try to distinguish between the simplifications
and the mistakes.

Recall that it's hard to get a consistent string theory - one that's
not plagued by infinite answers to interesting questions.  But this
difficulty is generally regarded as a good thing, because it drastically
limits the number of different versions of string theory one needs to
think about.  It's often said that there are only 5 consistent string
theories: the type I theory, the type IIA and IIB theory, and the two kinds
of heterotic string theory.  I'm not sure exactly what this statement
means, but certainly it's only meant to cover supersymmetric string
theories, which can handle fermions (like the electron and neutrino)
in addition to bosons (like the photon).  

Type I strings are "open strings" - not closed loops - and they live
in 10 dimensional spacetime, meaning that you need the dimension to be
10 to make certain nasty infinities cancel out.  Type II strings also
live in 10 dimensions, but they are "closed strings".  That means that
they look like a circle, so there are vibrational modes that march
around clockwise and other modes that march around counterclockwise, and
these are supposed to correspond to different particles that we see.  We
can think of these vibrational modes as moving around the circle at the
speed of light; they are called "left-movers" and "right-movers".  Now
fermions which move at the speed of light are able to be rather
asymmetric and only spin one way (when viewed head-on).  We say they
have a "chirality" or handedness.  Ordinary neutrinos, for example, are
left-handed.  This asymmetry of nature shocked everyone when first
discovered, but it appears to be a fact of life, and it's certainly a
fact of mathematics.  In the type IIA string theory, the left-moving
and right-moving fermionic vibrational modes have opposite chiralities,
while in the IIB theory, they have the same chirality.  When I last
checked, the type IIA theory seemed to fit our universe a bit better than
the IIB theory.

But lots of people say the heterotic theory matches our universe even
better.  The name "heterotic" refers to the fact that this theory is
supposed to have "hybrid vigor".  It's quite bizarre: the left-movers
are purely bosonic - no fermions - and live in 26-dimensional
spacetime, the way non-supersymmetric string theories do.  The
right-movers, on the other hand, are supersymmetric and live in 10-
dimensional spacetime.  It sounds not merely heterotic, but downright
schizophrenic!  But in fact, the 26-dimensional spacetime can also thought of
as being 10-dimensional, with 16 extra "curled-up dimensions" in the
shape of a torus.  This torus has two possible shapes:
R^16 modulo the E8 x E8 lattice or the D16* lattice.  
(For some of the wonders of
E8 and other lattices, check out 
"<A HREF = "week64.html">week64</A>" and "<A HREF = "week65.html">week65</A>".  The D16* lattice is related to the D16 lattice
described in those Weeks, but not quite the same.)

Now there is still lots of room for toying with these theories
depending on how you "compactify": how you think of 10-dimensional
spacetime as 4-dimensional spacetime plus 6 curled-up dimensions.
That's because there are lots of 6-dimensional manifolds that will do
the job (the so-called "Calabi-Yau" manifolds).  Different choices give
different physics, and there is a lot of work to be done to pick the
right one.  

However, recently it's beginning to seem that all five of the basic
sorts of string theory are beginning to look like different
manifestations of the same theory in 11 dimensions... some monstrous
thing called M-theory!  Let me quote the following paper:

1) Kelly Jay Davis, M-Theory and String-String Duality, 28 pages,
available as <A HREF = "http://xxx.lanl.gov/abs/hep-th/9601102">hep-th/9601102</A>, uses harvmac.tex.

The idea seems to be roughly that depending on how one compactifies the
11th dimension, one gets different 10-dimensional theories from
M-theory:

\begin{quote}
"In the past year much has happened in the field of string theory.
Old results relating the two Type II string theories and the
two Heterotic string theories have been combined with newer
results relating the Type II theory and the Heterotic theory, 
as well as the Type I theory and the Heterotic theory, 
to obtain a single "String Theory."  In addition, there has been much
recent progress in interpreting some, if not all, properties of String
Theory in terms of an eleven-dimensional M-Theory. 
In this paper we will perform a 
self-consistency check on the various relations between M-Theory and
String Theory. In particular, we will examine the relation between String
Theory and M-Theory by examining its consistency with the string-string
duality conjecture of six-dimensional String Theory. So, let us now take a
quick look at the relations between M-Theory and String Theory some of
which we will be employing in this article.

In Witten's paper he established that the strong coupling
limit of Type IIA string theory in ten dimensions is equivalent to
eleven-dimensional supergravity on a "large" S^1.
[Note: S^1 just means the circle - jb.]  As the low energy limit of
M-theory is eleven-dimensional supergravity, this relation
states that the strong coupling limit of Type IIA string theory in
ten-dimensions is equivalent to the low-energy limit of M-Theory on a
"large" S^1. In the paper of Witten and Horava, they
establish that the strong coupling limit of the ten-dimensional E8 x E8
Heterotic string theory is equivalent to M-Theory on a 
"large" S^1/Z_2.

Recently, Witten, motivated by Dasgupta and Mukhi,
examined M-Theory on a Z_2 orbifold of the five-torus and
established a relation between M-Theory on this orbifold and Type IIB
string theory on K3.  [Note: most of these undefined terms refer to
various spaces; for example, the five-torus is the 5-dimensional version
of a doughnut, while K3 is a certain 4-dimensional manifold - jb.]  Also,
Schwarz very recently looked at M-Theory and its relation to
T-Duality.

As stated above, M-Theory on a "large" S^1 is equivalent to a 
strongly
coupled Type IIA string theory in ten-dimensions. Also, M-theory on a
"large" S^1/Z_2 is equivalent to a strongly coupled E8 x E8 Heterotic
string theory in ten dimensions.  However, the string-string duality
conjecture in six dimensions states that the strongly coupled limit of a
Heterotic string theory in six dimensions on a four-torus is equivalent
to a weakly coupled Type II string theory in six-dimensions on K3.
Similarly, it states that the strongly coupled limit of a Type II theory
in six dimensions on K3 is equivalent to a weakly coupled Heterotic
string theory in six-dimensions on a four-torus.  Now, as a strongly
coupled Type IIA string theory in ten-dimensions is equivalent to the
low energy limit of M-Theory on a "large" S^1, the low energy limit of
M-Theory on S^1 x K3 should be equivalent to a weakly coupled Heterotic
string theory on a four-torus by way of six-dimensional string-string
duality.  Similarly, as a strongly coupled E8 x E8 Heterotic string
theory in ten-dimensions is equivalent to the low energy limit of
M-Theory on a "large" S^1/Z_2, the low energy limit of M-Theory on
S^1/Z_2 x T^4 should be equivalent to a weakly coupled Type II string
theory on K3.  The first of the above two consistency checks on the
relation between M-Theory and String Theory will be the subject of this
article.  However, we will comment on the second consistency check in our
conclusion."
\end{quote}

So, as you can see, there is a veritable jungle of relationships out
there.  But you must be wondering by now: \emph{what's M-theory?} 
According to

2) Edward Witten, Five-branes and M-Theory on an orbifold, 
available as <A HREF = "http://xxx.lanl.gov/abs/hep-th/9512219">hep-th/9512219</A>.  

the M stands for "magic", "mystery", or "membrane", according to taste.
From a mathematical viewpoint a better term might be "murky", since
apparently everything known about M-theory is indirect and
circumstantial, except for the classical limit, in which it seems to act
as a theory of 2-branes and 5-branes, where an "n-brane" is an
n-dimensional analog of a membrane or surface.  

Well, here I must leave off, for reasons of ignorance.  I don't really
understand the evidence for the existence of the M-theory... I can only
await the day when the murk clears and it becomes possible to learn
about this stuff a bit more easily.  It has been suggested that string
theory is a bit of 21st-century mathematics that accidentally fell into
the 20th century.  I think this is right, and that eventually much of
this stuff will be seen as much simpler than it seems now.


Now let me briefly describe some papers I actually sort of understand.

3) Abhay Ashtekar, Polymer geometry at Planck scale and quantum Einstein
equations, available as <A HREF = "http://xxx.lanl.gov/abs/hep-th/9601054">hep-th/9601054</A>.

Roumen Borissov, Seth Major and Lee Smolin, The geometry of quantum spin
networks, available as <A HREF = "http://xxx.lanl.gov/abs/gr-qc/9512043">gr-qc/9512043</A>, 35 Postscript figures,
uses epsfig.sty.  

Bernd Bruegmann, On the constraint algebra of quantum gravity in the
loop representation, available as <A HREF = "http://xxx.lanl.gov/abs/gr-qc/9512036">gr-qc/9512036</A>.  

Kiyoshi Ezawa, Nonperturbative solutions for canonical quantum gravity:
an overview, available as <A HREF = "http://xxx.lanl.gov/abs/gr-qc/9601050">gr-qc/9601050</A>

Kiyoshi Ezawa, A semiclassical interpretation of the topological solutions for
canonical quantum gravity, available as <A HREF = "http://xxx.lanl.gov/abs/gr-qc/9512017">gr-qc/9512017</A>.  

Jorge Griego, Extended knots and the space of states of quantum gravity,
available as <A HREF = "http://xxx.lanl.gov/abs/gr-qc/9601007">gr-qc/9601007</A>.  

Seth Major and Lee Smolin, Quantum deformation of quantum gravity,
available as <A HREF = "http://xxx.lanl.gov/abs/gr-qc/9512020">gr-qc/9512020</A>.


Work on the loop representation of quantum gravity proceeds apace.  
The paper by Ashtekar and the first one by Ezawa review various recent
developments and might be good to look at if one is just getting
interested in this subject.  Smolin has been pushing the idea of
combining ideas about the quantum group SU_q(2) with the loop
representation, and his papers with Borissov and Major are about that.
This seems rather interesting but still a bit mysterious to me.  I
suspect that what it amounts to is thinking of loops as excitations not
of the Ashtekar-Lewandowksi vacuum state but the Chern-Simons state.
I'd love to see this clarified, since these two states are two very
important exact solutions of quantum gravity, and the latter has the
former as a limit as the cosmological constant goes to zero.  In the
second paper listed, Ezawa gives semiclassical interpretations of these
and other exact solutions of quantum gravity. 

4) Thomas Kerler, Genealogy of nonperturbative quantum-invariants of
3-Manifolds: the surgical family, available as <A HREF = "http://xxx.lanl.gov/abs/q-alg/9601021">q-alg/9601021</A>.

Kerler brings a bit more order to the study of quantum
invariants of 3-manifolds, in particular, the Reshetikhin-Turaev,
Hennings-Kauffman-Radford, and Lyubashenko invariants.  All of these are
constructed using certain braided monoidal categories, like the category
of (nice) representations of a quantum group.  He describes how
Lyubashenko's invariant specializes to the Reshetikhin-Turaev invariant
for semisimple categories and to the Hennings-Kauffman-Radford invariant
for Tannakian categories.  People interested in extended TQFTs and
2-categories will find his work especially interesting, because he
works with these invariants using these techniques.  James Dolan and I
have argued that it's only this way that one will really understand
these TQFTs (see "<A HREF = "week49.html">week49</A>").  

In future editions of This Week's Finds I will say more about
n-categories and topological quantum field theory.  I have a feeling
that while I've discussed these a lot, I have never really explained the
basic ideas very well.  As I gradually understand the basic ideas
better, they seem simpler and simpler to me, so I think I should try to
explain them.




\par\noindent\rule{\textwidth}{0.4pt}

% </A>
% </A>
% </A>
